\documentclass[11pt]{report}

\usepackage{epsf,amsmath,amsfonts}
\usepackage{graphicx}
\usepackage{listings}
\usepackage{color}

\usepackage{color}
\definecolor{gray}{rgb}{0.4,0.4,0.4}
\definecolor{darkblue}{rgb}{0.0,0.0,0.6}
\definecolor{cyan}{rgb}{0.0,0.6,0.6}

\lstset{
  basicstyle=\ttfamily,
  columns=fullflexible,
  showstringspaces=false,
  commentstyle=\color{gray}\upshape
}

\lstdefinelanguage{XML}
{
  morestring=[b]",
  morestring=[s]{>}{<},
  morecomment=[s]{<?}{?>},
  stringstyle=\color{black},
  identifierstyle=\color{darkblue},
  keywordstyle=\color{cyan},
  morekeywords={xmlns,version,type, name, description, units, dimensions, 
  				missing\_value, possible\_values, long\_name, short\_name, 
				default\_value, nml\_group, interval, prefix, name\_template,
				frames}% list your attributes here
}

\begin{document}

\title{Field Statistics Module: \\
Requirements and Design}
\author{MPAS Development Team}

\maketitle
\tableofcontents

%-----------------------------------------------------------------------

\chapter{Summary}

This document describes a module which can compute generic statistics given a
field. The statistics described in this document are specifically time
averages, and moments, however this could be extended at a later time.

%-----------------------------------------------------------------------

\chapter{Requirements}

\section{Requirement: Field Time Average Module}
Date last modified: 02/26/13 \\
Contributors: (Doug Jacobsen) \\

Some cores require the ability to time average fields. It would be useful to
have this capability built into the shared framework, so fields can arbitrarily
be time averaged and written to an output stream.

\section{Requirement: Moment Computation Module}
Date last modified: 02/26/13 \\
Contributors: (Doug Jacobsen) \\

In addition to time averages, cores might benefit from computing the moments of
fields. 

\section{Requirement: Run-time configuration}
Date last modified: 02/26/13 \\
Contributors: (Doug Jacobsen) \\

The decision to compute time averages or moments of fields should be handled at
run-time rather than build time.

%-----------------------------------------------------------------------

\chapter{Design and Implementation}

\section{Implementation: Time Average Module}
Date last modified: 02/26/13 \\
Contributors: (Doug Jacobsen) \\

One issue with time averaging fields is they generally require the addition of
another field, which represents the time average of the full field. Currently
the structure for adding a time average field would include creating a new
field, and within a core averaging the full field into that new field every
time step.

An alternative, is to add another array to each field type as follows:
{\tiny
\begin{lstlisting}[language=fortran,escapechar=@,frame=single]
! Derived type for storing fields
type field5DReal

   ! Back-pointer to the containing block
   type (block_type), pointer :: block

   ! Raw array holding field data on this block
   real (kind=RKIND), dimension(:,:,:,:,:), pointer :: array
   @\colorbox{yellow}{real (kind=RKIND), dimension(:,:,:,:,:), pointer :: tavgArray}@

   ! Information used by the I/O layer
   type (io_info), pointer :: ioinfo       ! to be removed later
   character (len=StrKIND) :: fieldName
   character (len=StrKIND), dimension(:), pointer :: constituentNames => null()
   character (len=StrKIND), dimension(5) :: dimNames
   integer, dimension(5) :: dimSizes
   logical :: hasTimeDimension
   logical :: isSuperArray
   type (att_list_type), pointer :: attList => null()     

   ! Pointers to the prev and next blocks for this field on this task
   type (field5DReal), pointer :: prev, next

   ! Halo communication lists
   type (mpas_multihalo_exchange_list), pointer :: sendList
   type (mpas_multihalo_exchange_list), pointer :: recvList
   type (mpas_multihalo_exchange_list), pointer :: copyList
end type field5DReal
\end{lstlisting}
}

At runtime, the namelist for a particular field will determine if this field is
supposed to be time averaged or not. If the field is determined to be time
averaged, this new tavgArray array will be allocated.

A new time average module will be created, which will include checks on all
fields that have 2 or more time levels defined and will compute the time averages
of all fields that have an allocated tavgArray array.

This new module will largely be created by Registry, and will allow cores to
call a single set of subroutines to zero, increment, and normalize time
averaged fields. 

One requirement for this formulation, is that a field and it's time averages
share the same output streams. This means a stream can't contain only the time
average of a field, or just the field. If one is to be written out, and both
are computed, both are written out.

\section{Implementation: Moment computation module}
Date last modified: 02/26/13 \\
Contributors: (Doug Jacobsen) \\

The implementation of a moment computation module largely follows the time
average module formulation. A moment array is added to each field, along with a
moment ID array, as follows.

{\tiny
\begin{lstlisting}[language=fortran,escapechar=@,frame=single]
! Derived type for storing fields
type field5DReal

   ! Back-pointer to the containing block
   type (block_type), pointer :: block

   ! Raw array holding field data on this block
   real (kind=RKIND), dimension(:,:,:,:,:), pointer :: array
   @\colorbox{yellow}{real (kind=RKIND), dimension(:,:,:,:,:,:), pointer :: tavgMoments}@
   @\colorbox{yellow}{real (kind=RKIND), dimension(:,:,:,:,:,:), pointer :: momentsArray}@
   @\colorbox{yellow}{real (kind=RKIND), dimension(:), pointer :: momentIDs}@

   ! Information used by the I/O layer
   type (io_info), pointer :: ioinfo       ! to be removed later
   character (len=StrKIND) :: fieldName
   character (len=StrKIND), dimension(:), pointer :: constituentNames => null()
   character (len=StrKIND), dimension(5) :: dimNames
   integer, dimension(5) :: dimSizes
   logical :: hasTimeDimension
   logical :: isSuperArray
   type (att_list_type), pointer :: attList => null()     

   ! Pointers to the prev and next blocks for this field on this task
   type (field5DReal), pointer :: prev, next

   ! Halo communication lists
   type (mpas_multihalo_exchange_list), pointer :: sendList
   type (mpas_multihalo_exchange_list), pointer :: recvList
   type (mpas_multihalo_exchange_list), pointer :: copyList
end type field5DReal
\end{lstlisting}
}

The momentsArray array is one higher dimension than the field array, to allow
multiple moments to be stored within the same array.  Moments to compute will
be specified in the I/O namelist, the same way time averaging is specified.
Potentially a space delimited list of moment numbers to compute will be
specified as follows:

\begin{lstlisting}
&fieldName
	config_fieldName_moments = "2 4"
/
\end{lstlisting}

This input string will be broken apart and stored in the momentIDs array. A
call to a shared subroutine for moment computation will compute all requested
moments, and store them in the momentsArray array. As with the time averages,
these arrays are not allocated if moments are not requested for the specific
field. When the moment computation module is called, all fields are checked to
see if any moments should be computed, based on if the momentIDs array is
allocated with values or not.

Additionally, a tavgMoments array is provided to compute time averages of the
moments.

As with the time averages, all moments and time averages of moments are share
the same output streams as the parent field.

\section{Implementation: Run-time configuration}
Date last modified: 02/26/13 \\
Contributors: (Doug Jacobsen) \\

The implementation for run-time configuration of field statistics requires the
implementation of a run-time I/O layer. After the run-time I/O layer is
complete, the namelist that controls I/O will contain flags for each field to
determine which statistics (if any) should be computed.

At run-time, these flags will be checked and MPAS will either compute the
statistics for a particular field or will skip the field all together.

%-----------------------------------------------------------------------

\chapter{Testing}

\section{Testing and Validation: Run-time I/O Layer}
Date last modified: 02/26/13 \\
Contributors: (Doug Jacobsen) \\



%-----------------------------------------------------------------------

\end{document}
