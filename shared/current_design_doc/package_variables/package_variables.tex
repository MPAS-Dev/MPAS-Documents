\documentclass[11pt]{report}

\usepackage{epsf,amsmath,amsfonts}
\usepackage{graphicx}
\usepackage{listings}
\usepackage{color}

\setlength{\topmargin}{0in}
\setlength{\headheight}{0in}
\setlength{\headsep}{0in}
\setlength{\textheight}{9.0in}
\setlength{\textwidth}{6.0in}
\setlength{\evensidemargin}{0.25in}
\setlength{\oddsidemargin}{0.25in}

\definecolor{gray}{rgb}{0.4,0.4,0.4}
\definecolor{darkblue}{rgb}{0.0,0.0,0.6}
\definecolor{cyan}{rgb}{0.0,0.6,0.6}

\lstset{
	basicstyle=\ttfamily,
		columns=fullflexible,
		showstringspaces=false,
		commentstyle=\color{gray}\upshape
}

\lstdefinelanguage{XML}
{
	morestring=[b]",
	morestring=[s]{>}{<},
	morecomment=[s]{<?}{?>},
	stringstyle=\color{black},
	identifierstyle=\color{darkblue},
	keywordstyle=\color{cyan},
	morekeywords={xmlns,version,type,name,type,dimensions,name\_in\_code,units,description,template,streams,persistence}% list your attributes here
}

\begin{document}

\title{Package Variables: \\
Requirements and Design}
\author{MPAS Development Team}

\maketitle
\tableofcontents

%-----------------------------------------------------------------------

\chapter{Summary}

This document introduces package variables, and describes requirements and
design specifications for the implementation and use of package variables.

Package variables can most easily be explained through the concept of optional
physics packages. For example, one simulation might have physics package A on
while the next might have physics package A off. During a simulation we might
not want to have all variables allocated for this physics package when it's not
being used.

As such, package variables are introduced. These are groupings of variables
whos allocation depends on the choices of namelist options. 

%-----------------------------------------------------------------------

\chapter{Requirements}

To support the increasing complexity and breadth of options in MPAS cores,
while keeping memory usage to a minimum, the package variable capability in MPAS
must meet the following requirements.

\begin{enumerate}

\item MPAS must be capable of enabling or disabling individual variables, 
   constituents of variable arrays (super-arrays), and variable groups at run-time.
   ``Enabling'' a variable means that the variable should be allocated and fully usable within
   an MPAS core; ``disabling'' means that a variable should use a little memory as possible 
   while still allowing an MPAS core to compile and run using a set of options that do not 
   require the variable. Packages are the means by which MPAS variables will be
   enabled and disabled.
   
\item It must be possible to include arbitrary sets of variables in a package. A package may, therefore,
   include a mix of regular variables, constituent variables, and variable groups. 

\item Variables are not required to belong to any package, and the behavior of variables that do
   not belong to any package should not change from current behavior in MPAS.
                                                                                                    
\item MPAS must support the ability to enable packages using arbitrary logic.

\item MPAS core and infrastructure code must be able to determine whether a variable is
   enabled or disabled.
   
\item MPAS I/O should only read/write variables and constituents that are enabled. If a variable
   is disabled, and therefore has no associated storage, reading the variable makes no sense; and
   it wouldn't appear to be useful to write garbage or default values for a variable that is never used
   during a particular execution of an MPAS core.

\item MPAS must define packages in the XML Registry files.

\item Packages must allow documentation describing the pacakge and it's intended use.


\item Variables may belong to multiple packages, in which case if any of the packages are enabled the variable will be active.

\item The enabling and disabling of a particular package may depend on other packages being enabled or disabled.

\end{enumerate}                                                             


%-----------------------------------------------------------------------

\chapter{Design and Implementation}
\section{Implementation: Package Variables}
Date last modified: 10/24/2013 \\
Contributors: ( Doug Jacobsen ) \\

Package variables are defined by modifying the persistence of a particular
variable. Previously variables could have their persistence defined as
persistent, or scratch, and now can additionally be defined as package.

The persistence option is added to the var\_struct constructs within the Registry.xml file. When persistence is
speified on a var\_struct it does not cause the entire var\_struct to be
persistent, scratch, or package, it simply defines the default persistence for
all variables defined within that particular var\_struct. This default can be
overridden by specifying the persistence on a per var or var\_array basis.

Additionally, when persistence is set to package, another attribute
``packages'' is required. This defines the name of the package the var,
var\_struct, or var\_array belongs to. The ``packages'' attribute is an
optional attribute on constituent vars within a var\_array. This allows
constituent vars to be controlled through the package variable process.
Persistence is not added to constituent variables because a var\_array is not
allow to have constituents which are persistence, and scratch at the same time.

Packages also need to be defined. Witihin registry, at the same level as
var\_struct a new construct is created called packages. All packages must be
defined at this level before being used throughout the var\_struct groups.
Below is an example of a package definition:

{\scriptsize
\begin{lstlisting}[language=XML]
   ...
</nml_record>
<packages>
	<package name=``package_a'' description=``Description of package a''/>
	<package name=``package_b'' description=``Description of package b''/>
</packages>
<var_struct name=``mesh'' time_levs=``0''>
   ...
\end{lstlisting}
}

After the package is defined, any var\_struct, var, var\_array, or constituent
var constructs can be attached to it as follows:

{\scriptsize
\begin{lstlisting}[language=XML]
<var_struct name=``physicsA'' time_levs=``0'' persistence=``package'' packages=``package_a''>
	<var name=``physVar1'' dimensions=``nCells Time'' .../>
</var_struct>
<var_struct name=``physicsB'' time_levs=``0''>
	<var_array name=``physVarArray'' dimensions=``nVertLevels nCells Time'' persistence=``package'' packages=``package_b''>
		<var ... />
		...
	</var_array>
	<var array name=``physVarArray2'' dimensions=``nVertLevels nCells Time''>
		<var name=``consVar1'' packages=``package_b'' ... />
		<var name=``consVar2'' ... />
		...
	</var_array>
</var_struct>
\end{lstlisting}
}

Within the shared framework, a module named mpas\_packages is created that
contains logicals of the format ``package\_name\_on''. These logicals are set
to ``.false.'' by default, but when set to true at the beginning of a
simulation, they enable the allocation and I/O of the package variables. 

Cores are responsible for the proper initialization of package logicals. This
is done through a routine called mpas\_core\_setup\_packages, which should only have
an error argument. This subroutine is written on a per-core basis, and can
contain arbitrary logic to enable packages. This subroutine should be contained
within the mpas\_core module.

For example:
{\scriptsize
\begin{lstlisting}[language=Fortran]
subroutine mpas_core_setup_package(ierr)
  use mpas_configure

  implicit none

  integer, intent(out) :: ierr

  if(config_physics_option == trim('A')) then
    physics_a_on = .true.
  else if(config_physics_option == trim('B') .and. config_num_halos .ge. 3) then
    physics_b_on = .true.
  end if
end subroutine mpas_core_setup_packages
\end{lstlisting}
}

The core should not try to allocate or deallocate the variables contained in
the package. The core is responsible for ensuring that variables in a package
that is not active are not used.

There are two methods of determining if a variable is part of an active package
or not. The first is to check the ``packageOn'' logical in the
mpas\_packages module. The second depends on what type of variable is being
queried. For all variables that are not constituents of a var\_array, the
isActive attribute can be queried. If field \% isActive is equal to .true. then
the package controling that variables allocation is active. For constituent
variables that don't conatin an isActive attribute the index\_varname index is
set to a value of -1. 

%-----------------------------------------------------------------------

\chapter{Testing}

\section{Testing and Validation: Package Variables}
Date last modified: 10/03/2013 \\
Contributors: ( Doug Jacobsen ) \\

Package variables will be added to a component.

A run with the package on and off will be performed, and then should both run
to completion and produce bit-idential results to runs where the package
variables are defined as persistent.

%-----------------------------------------------------------------------

\end{document}
