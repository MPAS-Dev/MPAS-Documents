\documentclass[11pt]{report}

\usepackage{epsf,amsmath,amsfonts}
\usepackage{graphicx}

\setlength{\textwidth}{6.5in}
\setlength{\oddsidemargin}{0in}
\setlength{\evensidemargin}{0in}
\setlength{\textheight}{8.5in}
\setlength{\topmargin}{0in}

\begin{document}

\title{
Requirements and Design\\
Tensor Operators}
\author{MPAS Development Team}

\maketitle
\tableofcontents

%-----------------------------------------------------------------------

\chapter{Summary}

The computation of the stress tensor and the divergence of the stress tensor are required for turbulence closures and the sea-ice momentum equation.  These operations on an unstructured grid are significantly more complicated than on a quadrilateral grid.  Here we provide algorithms to compute the strain rate tensor and the divergence of any symmetric rank-2 tensor . The stress tensor is related to the strain rate tensor through a constitutive relation that is not specified herein.  Analytic test cases are provided to validate the discrete operators.


%figure template
%\begin{figure}
%  \center{\includegraphics[width=14cm]{./Figure1.pdf}}
%  \caption{A graphical representation of the discrete boundary.}
%  \label{Figure1}
%\end{figure} 

%-----------------------------------------------------------------------

\chapter{Requirements}

\section{Requirement: Strain rate tensor}
Date last modified: 4/23/2013 \\
Contributors: Mark Petersen \\

 Given a discrete horizontal velocity field on the edges of an unstructured C-grid, compute the strain rate tensor.

\section{Requirement: Divergence of a tensor}
Date last modified: 4/23/2013 \\
Contributors: Mark Petersen \\

Given a rank-2, symmetric tensor, $\sigma_{ij}$, at the edges of an unstructured C-grid, compute $\nabla\cdot\sigma$, the divergence of the tensor.  The output is a rank-1 tensor measured in $\mathbb{R}^3$.

\section{Requirement: Quantities must be available at all locations}
Date last modified: 4/23/2013 \\
Contributors: Mark Petersen \\

The strain rate tensor and tensor divergence must be available at cell centers, vertices, and edges.  One location may be the primary location, and others may be computed through interpolation.

\section{Requirement: Quantities must be available in $\mathbb{R}^3$ and $\mathbb{R}^2$ forms.}
Date last modified: 4/23/2013 \\
Contributors: Mark Petersen \\

The $\mathbb{R}^3$ form is measured in Cartesian $(x,y,z)$ coordinates.  In the case of a sphere, the center of the sphere is located at the origin.  The $\mathbb{R}^2$ coordinates are a local Cartesian coordinate, such as latitude-longitude coordinates on the sphere.  Operators must be provided to convert quantities between the two coordinate systems.

\section{Requirement: Validation with analytic test functions}
Date last modified: 4/23/2013 \\
Contributors: Mark Petersen \\

Analytic test functions and solutions should provided in a testing subroutine for both Cartesian and spherical geometries.  These will be sufficiently general to fully exercise the numerical discretization.  This should include an arbitrary rotation with respect to the primary axes, and complex enough so that the divergence is not a constant.  For a fixed domain size, the root-mean-squared error should converge towards zero with a fixed order as grid-cell size decreases.  The required order of convergence is unknown, but should be nearly order one or greater.


%-----------------------------------------------------------------------

\chapter{Algorithmic Formulations}

\section{Design Solution: Strain rate tensor}
Date last modified: 4/23/2013 \\
Contributors: Mark Petersen \\

The strain rate tensor is an important operator for many applications.  In CICE, the stress tensor is computed from the strain rate tensor in each grid cell with a system of ODEs \cite[section 3.4]{CICE_manual_4.1}.  The computation of the stress tensor depends on the details of the model.  For example, they differ between viscous-plastic and elastic-viscous-plastic in CICE.  Computation of the strain rate tensor is a general operation, and can be placed in the operators subdirectory so it may be used by all MPAS cores.

The continuous strain rate tensor in Cartesian coordinates is
\begin{equation}
\dot{\epsilon}_{ij} = \frac{1}{2}\left(\frac{\partial u_i}{\partial x_j} 
                                     + \frac{\partial u_j}{\partial x_i}\right),
\end{equation}
where ${\bf u} = (u_1,u_2,u_3)$ is the velocity measured in Cartesian coordinates $(x_1,x_2,x_3)$.  In an arbitrary reference frame the strain rate tensor may be written using the Jacobian operator $J=\nabla {\bf u}$ as
\begin{equation}
{\bf\dot{\epsilon}} = \nabla_s {\bf u} \equiv \frac{1}{2}\left(J + J^T\right),
\end{equation}
where $\nabla_s$ denotes the symmetric gradient operator.  The weak form of the gradient, in continuous form, over a control surface $A$  is
\begin{equation}
\label{cont weak strain rate tensor}
\nabla {\bf u}
= \lim_{A\rightarrow(x,y)} \frac{1}{\tilde{A}} 
\int_{\partial A}
  {\bf n}\otimes{\bf u} \; dl
\end{equation}
\cite[equation 22]{Ringler06unpub} where $\tilde{A} = \int_A dA$ is the area, $dl$ is an increment along the boundary $\partial A$, and ${\bf n}$ is a unit vector normal to the boundary (Figure \ref{fig:2D domain}).  {\it Mark: I need to find a published reference for (\ref{cont weak strain rate tensor})}  The symbol $\otimes$ is the tensor product, which is the same as an outer product for two vectors.  In two-dimensional Cartesian space, ${\bf n}$ and ${\bf u}$ are both 2x1 vectors, and 
\begin{equation}
{\bf n}\otimes{\bf u} 
= \left( \begin{array}{c} n_1 \\ n_2 \end{array}   \right)
  \left( \begin{array}{cc} u_1 &  u_2 \end{array} \right)
= \left( \begin{array}{cc} n_1 u_1 & n_1 u_2 \\  n_2 u_1 & n_2 u_2 \end{array}   \right).
\end{equation}

\begin{figure}[htbp]
 \center
 \includegraphics[trim=3.2in 3.2in 3.0in 3.5in, clip=true, scale=0.8]{f/A_surface.pdf}
 \caption{Two-dimensional surface $A$.}
 \label{fig:2D domain}
\end{figure}

The discrete form of (\ref{cont weak strain rate tensor}), where the control surface is a single cell, is
\begin{equation}
\label{cell strain rate tensor}
\left[\nabla {\bf u} \right]_i
= 
\frac{1}{A_i} 
\sum_{e\in EC(i)}{
  {\bf n}_e\otimes{\bf u}_e \; l_e },
\end{equation}
where $A_i$ is the area of cell $i$; $n_e$ is the unit vector normal to the edge $e$ measured in $\mathbb{R}^3$. Note that $n_e$ lies in the  tangent plane of the domain being used; $\bf{u_e}$ is the velocity vector at edge $e$ measured in $\mathbb{R}^3$; $l_e$ is the length of the edge, and $EC(i)$ is the set of edges surrounding cell $i$.  In the MPAS-Ocean code the velocity at an edge is written as 
\begin{eqnarray}
{\bf u}_e = u_e {\bf n}_e + v_e \tilde{\bf n}_e,
\end{eqnarray}
where $u_e$ is the \verb|normalVelocity| variable, $v_e$ is the \verb|tangentialVelocity| variable.  Here $\tilde{\bf n}_e$ is the unit tangent vector on an edge,
\begin{eqnarray}
{\bf n}_e = \frac{{\bf x}_{i2}-{\bf x}_{i1}}{\left||{\bf x}_{i2}-{\bf x}_{i1}\right||}
  = \frac{1}{d_e} \sum_{i\in CE(e)}n_{e,i}{\bf x}_{i} 
\\
 \tilde{\bf n}_e = \frac{{\bf x}_{v2}-{\bf x}_{v1}}{\left||{\bf x}_{v2}-{\bf x}_{v1}\right||}
  = \frac{1}{l_e} \sum_{v\in VE(e)}n_{e,v}{\bf x}_{v},
\end{eqnarray}
where ${\bf x}$ provide locations of cell centers and vertices, $d_e$ is the distance between cell centers, and $n_{e,i}$ and $n_{e,v}$ provide an sign convention based on the ordering of cells and vertices on an edge.  Continuing from (\ref{cell strain rate tensor}), the velocity gradient is computed as
\begin{equation}
\label{cell strain rate tensor uv}
\left[\nabla {\bf u} \right]_i
= 
\frac{1}{A_i} 
\sum_{e\in EC(i)}{\left(
  u_e{\bf n}_e\otimes{\bf n}_e 
 +v_e{\bf n}_e\otimes\tilde{\bf n}_e 
\right)l_e },
\end{equation}
and the symmetric strain rate tensor as
\begin{equation}
\label{cell strain rate tensor sym}
\left[\dot\epsilon  \right]_i
= 
\left[\nabla_s {\bf u} \right]_i
= 
\left[\nabla {\bf u} \right]_i
+
\left[\nabla {\bf u} \right]^T_i.
\end{equation}

This algorithm computes a cell-centered strain rate measured in measured in $\mathbb{R}^3$ given a velocity field at edges.  This output may then be interpolated to edges or vertices, and rotated to a 2D symmetric tensor aligned with edges or an arbitrary reference frame.  Boundary conditions are simple; velocities at a land boundary are zero upon input.  If prognostic velocities are native to vertices or cell centers, as may be the case in MPAS-CICE, they must be interpolated to edges, and zeroed at land-boundary edges, before computing the strain rate.

\newpage
\section{Design Solution: Divergence of a tensor}
Date last modified: 4/23/2013 \\
Contributors: Mark Petersen \\

The weak form of the divergence of any symmetric tensor ${\mathbf F}$ is 
\begin{equation}
\label{cont div}
\nabla \cdot \mathbf{F} 
= 
\lim_{A\rightarrow(x,y)} \frac{1}{\tilde{A}} 
\int_{\partial A}
  {\bf n}\cdot{\mathbf F} \; dl
\end{equation}
in continuous form.  The divergence operator reduces the rank of $\mathbf{F}$ by 1.  In discrete form, the divergence of a rank-1 tensor (i.e. a vector) is
\begin{equation}
\label{discrete div}
[\nabla \cdot \mathbf{F}_:]_{i} = \frac{1} {A_i} \sum\limits_{e \in EC(i)} {n_{e,i} \, F_e \, l_e},
\end{equation}
where the operator computes a cell-centered divergence from a tensor field defined at edges. Here $F_e$ is the vector component normal to the edge, and $n_{e,i}$ indicates the sign of the vector at edge $e$ with respect to cell $i$.  The general discrete formulation of the divegence of a tensor is
\begin{equation}
\label{discrete div tensor}
[\nabla \cdot \mathbf{F}_:]_{i} = \frac{1} {A_i} \sum\limits_{e \in EC(i)} {\bf n}_e\cdot\mathbf{F}_e \; l_e.
\end{equation}

For the applications in this design document, tensor divergence reduces a rank-2 symmetric $\mathbb{R}^3$ tensor (written as a $3\times 3$ matrix) to a rank-1 $\mathbb{R}^3$ tensor (i.e. a vector in $\mathbb{R}^3$).  Common examples of rank-2 symmetric $\mathbb{R}^3$ tensors in fluid mechanics are the strain rate tensor and the stress tensor.  The discrete divergence begins with edge quantities and produces a cell-centered quantity.  Additional subroutines will interpolate the vector to edges or vertices, and rotate the vector into any 2D reference frame needed.  For example, the component of the stress tensor normal to an edge is simply
\begin{equation}
{\bf n}_e\cdot[\nabla \cdot \sigma]_{e}.
\end{equation}

Boundary conditions for $\sigma$ at land edges are not necessarily zero, as there can be shear stress at the wall.  If the stress tensor is native to vertices, interpolation to edge values may be done by simple averaging.  If the stress tensor is native to cell centers, the stress tensor at the land cell is zero, so a simple average to the edge will be half the value of the ocean cell.


\section{Requirement: Quantities must be available in $\mathbb{R}^3$ and $\mathbb{R}^2$ forms.}
A rank-2 tensor $\epsilon_{sph}$ may be converted from spherical coordinates to $\mathbb{R}^3$ using tensor rotation as follows
\begin{eqnarray}
\epsilon_{R3} = R \epsilon_{sph} R^T
\end{eqnarray}
where $T$ is the transpose and $R$ is the rotation matrix composed of unit vectors in the zonal, meridional, and vertical directions,
\begin{eqnarray}
R = \left[ \begin{array}{ccc}
{\bf n}_\lambda & {\bf n}_\phi & {\bf n}_r
\end{array}\right]
= \left[ \begin{array}{ccc}
-\sin\lambda & -\sin\phi\cos\lambda & \cos\phi\cos\lambda \\
 \cos\lambda & -\sin\phi\sin\lambda & \cos\phi\sin\lambda \\
    0     &  \cos\phi         & \sin\phi
\end{array}\right].
\end{eqnarray}
Note that a matrix consisting of three orthogonal unit vectors is an orthonormal matrix, has determinant one, and has the property $R^{-1}=R^T$.  Thus the inverse operation is simply
\begin{eqnarray}
\epsilon_{sph} = R^T \epsilon_{R3} R.
\end{eqnarray}

%-----------------------------------------------------------------------

\chapter{Design and Implementation}

\section{Implementation: Workflow for MPAS-O turbulence closure}
Date last modified: 4/23/2013 \\
Contributors: Mark Petersen \\

An example of a workflow for an MPAS-Ocean turbulence closure using the strain rate and stress tensor is as follows.  The turbulence closure routine may be executed on cell centers, vertices, or edges.  A graphical version is shown in figure \ref{fig:mpaso_tensor_workflow}
\\
\begin{tabular}{ |l| l| l| }
\hline
  {\bf subroutine} &  {\bf input} &  {\bf output} \\
\hline
\verb|mpas_tangential_velocity|  & $u_e$ \verb|normalVelocity| & $v_e$ \verb|tangentialVelocity|  \\
\verb|mpas_strain_rate_R3Cell|   & $u_e$ \verb|normalVelocity| & $\left[\dot\epsilon  \right]_i$ \verb|strainRateR3Cell|\\
&  $v_e$ \verb|tangentialVelocity| &\\
interpolate to edge or vertex and 2D if desired &  $\left[\dot\epsilon  \right]_i$ &  $\left[\dot\epsilon  \right]_e$ or $\left[\dot\epsilon  \right]_v$\\
turbulence closure: compute stress tensor &  $\left[\dot\epsilon  \right]_:$ &  $\left[\sigma  \right]_:$\\
interpolate to edge, if needed   & $\left[\sigma  \right]_:$&  $\left[\sigma  \right]_e$\\
\verb|mpas_divergence_of_tensor_R3Cell|   &  $\left[\sigma  \right]_e$ \verb|strainRateR3Edge|& $[\nabla \cdot \sigma]_{i}$ \verb|divTensorR3Cell|\\
\verb|mpas_vector_R3Cell_to_2DEdge|  &  $[\nabla \cdot \sigma]_{i}$ \verb|divTensorR3Cell| & ${\bf n}_e\cdot[\nabla \cdot \sigma]_{e}$, ${\bf \tilde n}_e\cdot[\nabla \cdot \sigma]_{e}$ \\
add $[\nabla \cdot \sigma]$ terms to momentum equation & & \\
\hline
\end{tabular}

\begin{figure}[htbp]
 \center
 \includegraphics[scale=0.9]{f/mpaso_tensor_workflow.pdf}
 \caption{MPAS-Ocean workflow}
 \label{fig:mpaso_tensor_workflow}
\end{figure}



\newpage

\section{Implementation: Workflow for MPAS-CICE}
Date last modified: 4/23/2013 \\
Contributors: Mark Petersen \\

The formulation for strain rate and tensor divergence provided in this library are in $\mathbb{R}^3$.  Operators are also included to rotate vectors and tensors to a local two-dimensional coordinate system, as well as interpolation routines to move variables amongst cell, vertex, and edge locations.  Here we provide an example of a potential MPAS-CICE workflow for solving the momentum equation at vertices in a local 2D latitude-longitude coordinate system.  This is the same as the current CICE code when on quadrilateral cells.  In addition, the constitutive relation between strain rate and stress is solved in the 2D form, as in CICE.  An example of the MPAS-CICE workflow for three-dimensional tensor operations is as follows, and is shown graphically in figure \ref{fig:mpasCICE_tensor_workflow1}.  
%Alternately, all tensor operations could be conducted in 2D, as shown in figure \ref{fig:mpasCICE_tensor_workflow2}.\\
\begin{tabular}{ |l| l| l| }
\hline
  {\bf subroutine} &  {\bf input} &  {\bf output} \\
\hline
{\bf solve momentum equation} at vertices &$[\nabla \cdot \sigma]_v$ (2D) & $u_v$, $v_v$\\
interpolate to edge, expand to R3 & $u_v$, $v_v$ & ${\bf u}_e$ \\
\verb|mpas_strain_rate_R3Cell|   & ${\bf u}_e$ & $\left[\dot\epsilon  \right]_i$ \\
rotate to 2D &  $\left[\dot\epsilon  \right]_i$ (R3) &  $\left[\dot\epsilon  \right]_i$ (2D)\\
{\bf compute stress tensor in 2D} &  $\left[\dot\epsilon  \right]_i$ (2D) &  $\left[\sigma  \right]_i$ (2D)\\
rotate to R3, interpolate to edge  & $\left[\sigma  \right]_i$ (2D) &  $\left[\sigma  \right]_e$ (R3) \\
\verb|mpas_divergence_of_tensor_R3Vertex|   &  $\left[\sigma  \right]_e$ \verb|strainRateR3Edge|& $[\nabla \cdot \sigma]_v$ \verb|divTensorR3Vertex|\\
rotate to 2D for momentum forcing &  $[\nabla \cdot \sigma]_v$ (R3) & $[\nabla \cdot \sigma]_v$ (2D) \\
\hline
\end{tabular}
\vspace{8pt}

The MPAS-CICE routines required for this workflow are to solve the momentum equation and compute the stress tensor.  These operations are point-wise (i.e., they do not involve neighbors given these inputs), and so the subroutines can operate on cells centers, vertices, or edges.  

If the momentum equation is solved on edges or vertices, boundary conditions for velocity is simply zero on land boundaries.  If the momentum equation is solved at cell centers, the velocity interpolated to the edge could be set to zero before the advection and strain rate are computed.  For the strain rate and stress tensor, use a land-cell value of zero when averaging from cell centers to edges, and use a straight average from vertices to edges.

\begin{figure}[htbp]
 \center
 \includegraphics[scale=0.9]{f/mpasCICE_tensor_workflow1.pdf}
 \caption{MPAS-CICE workflow, tensor operations in R3.  In step 7, the line integral is about the dual cell, i.e. the triangle surrounding the vertex.}
 \label{fig:mpasCICE_tensor_workflow1}
\end{figure}


%\begin{figure}[htbp]
% \center
% \includegraphics[scale=0.9]{f/mpasCICE_tensor_workflow2.pdf}
% \caption{MPAS-CICE workflow, tensor operations in 2D}
% \label{fig:mpasCICE_tensor_workflow2}
%\end{figure}


%-----------------------------------------------------------------------

\chapter{Testing and Validation}

\section{Testing functions for strain rate and its divergence on a plane}
Date last modified: 4/23/2013 \\
Contributors: Mark Petersen \\

The following analytic test cases may be used to test the tensor operators in $\mathbb{R}^2$, including quads, hexes, and variable resolution meshs.  All cases are implemented in the subroutine \verb|mpas_test_tensor|.  It is clear that some later test cases are a superset of earlier ones.  For example, the linear case is the power function with $n=1$.  But the earlier simple cases are instructional and useful for debugging, and so are included here and in the code.

The constant coefficients $c_n$ and $c_s$ are for the normal and shear components of the strain, respectively.  When velocities are aligned with an axis, like in cases \ref{linear_x}, \ref{linear_y}, \ref{power_x}, and \ref{power_y}, the normal strain appears on the diagonals of the strain rate tensor, and the shear appears on the off-diagonals.  However, for arbitrary rotations, both components appear in all the elements of the strain rate tensor.

\subsection{Linear in x\label{linear_x}}
\begin{eqnarray}
{\bf u}(x,y) &=& \left( \begin{array}{c} c_n \\ c_s \end{array}   \right)x \\
\nabla{\bf u}^* &=& \left( \begin{array}{c} \partial_x \\ \partial_y \end{array}   \right) \left( \begin{array}{cc} c_nx & c_sx \end{array}   \right)
 =  \left( \begin{array}{cc} c_n & c_s \\ 0 & 0 \end{array}   \right) \\
\nabla_s{\bf u}^* &=&  \left( \begin{array}{cc} c_n & \frac{1}{2}c_s \\ \frac{1}{2}c_s & 0 \end{array}   \right) \\
\nabla\cdot\left(\nabla_s{\bf u}\right) &=&  \left( \begin{array}{cc} 0 & 0 \end{array}   \right)
\end{eqnarray}

\subsection{Linear in y\label{linear_y}}
\begin{eqnarray}
{\bf u}(x,y) &=& \left( \begin{array}{c} -c_s \\ c_n \end{array}   \right)y \\
\nabla{\bf u}^* &=& \left( \begin{array}{c} \partial_x \\ \partial_y \end{array}   \right) \left( \begin{array}{cc} -c_sy & c_ny \end{array}   \right)
 =  \left( \begin{array}{cc}  0 & 0 \\ -c_s & c_n  \end{array}   \right) \\
\nabla_s{\bf u}^* &=&  \left( \begin{array}{cc} 0 & -\frac{1}{2}c_s \\ -\frac{1}{2}c_s & c_n \end{array}   \right) \\
\nabla\cdot\left(\nabla_s{\bf u}\right) &=&  \left( \begin{array}{cc} 0 & 0 \end{array}   \right)
\end{eqnarray}

\subsection{Linear, rotated about an arbitrary angle $\theta_r$\label{linear_rotation}}
\begin{eqnarray}
{\bf u}(r,\theta;\theta_r) &=& r\cos{(\theta-\theta_r)} R_{\theta_r} \left( \begin{array}{c} c_n \\ c_s \end{array}   \right) = f{\bf g}
\end{eqnarray}
where 
\begin{eqnarray}
\label{def f}
f(r,\theta;\theta_r)&\equiv& r\cos{(\theta-\theta_r)},\\
\label{def g}
{\bf g}(\theta_r) &\equiv& R_{\theta_r} \left( \begin{array}{c} c_n \\ c_s \end{array}   \right) \\
&=& \left(\begin{array}{cc} \cos{\theta_r}& -\sin{\theta_r} \\
                          \sin{\theta_r}&  \cos{\theta_r}   \end{array}\right)
 \left( \begin{array}{c} c_n \\ c_s \end{array}   \right)\\
&=& \left(\begin{array}{c} c_n\cos{\theta_r} -c_s\sin{\theta_r} \\
                          c_n\sin{\theta_r} +  c_s \cos{\theta_r}   \end{array}\right)
\end{eqnarray}
Note that when $\theta_r=0$ this simplifies to case \ref{linear_x} and when $\theta_r=\pi/2$ this simplifies to case \ref{linear_y}.  Recall the trig identity
\begin{equation}
\cos{(a-b)} = \cos{a}\cos{b} + \sin{a}\sin{b}
\end{equation}
so that 
\begin{eqnarray}
f&\equiv& r\cos{(\theta-\theta_r)}\\
 &=& r\cos{\theta}\cos{\theta_r} + r\sin{\theta}\sin{\theta_r}\\
 &=& x\cos{\theta_r} + y\sin{\theta_r}.
\end{eqnarray}
Finally, the operators of interest for this test case are
\begin{eqnarray}
\nabla{\bf u}^* &=& \left( \nabla f \right) {\bf g}^* 
 = \left( \begin{array}{c} \cos{\theta_r} \\ \sin{\theta_r} \end{array}    \right)  
\left(\begin{array}{cc} g_1 & g_2  \end{array} \right)
= \left(\begin{array}{cc} g_1\cos{\theta_r} & g_2\cos{\theta_r} \\ g_1\sin{\theta_r} & g_2\sin{\theta_r}  \end{array} \right)\\
\nabla_s{\bf u}^* &=& \left(\begin{array}{cc} g_1\cos{\theta_r} & \frac{1}{2}\left( g_1\sin{\theta_r} + g_2\cos{\theta_r} \right)\\
                      (sym) & g_2\sin{\theta_r}  \end{array} \right)\\
\nabla\cdot\left(\nabla_s{\bf u}\right) &=&  \left( \begin{array}{cc} 0 & 0 \end{array}   \right).
\end{eqnarray}

\subsection{Power Function in x\label{power_x}}
\begin{eqnarray}
{\bf u}(x,y) &=& \left( \begin{array}{c} c_n \\ c_s \end{array}   \right)x^{n} \\
\nabla{\bf u}^* &=& \left( \begin{array}{c} \partial_x \\ \partial_y \end{array}   \right) \left( \begin{array}{cc} c_nx^{n} & c_sx^{n} \end{array}   \right)
 =  \left( \begin{array}{cc} c_n & c_s \\ 0 & 0 \end{array}   \right) nx^{n-1}\\
\nabla_s{\bf u}^* &=&  \left( \begin{array}{cc} c_n & \frac{1}{2}c_s \\ \frac{1}{2}c_s & 0 \end{array}   \right) nx^{n-1}\\
\nabla\cdot\left(\nabla_s{\bf u}\right) &=&  \left( \begin{array}{cc} \partial_x & \partial_y \end{array}   \right)
\left( \begin{array}{cc} c_n & \frac{1}{2}c_s \\ \frac{1}{2}c_s & 0 \end{array}   \right) nx^{n-1}
 = \left( \begin{array}{cc} c_n & \frac{1}{2}c_s \end{array}   \right) n(n-1)x^{n-2}
\end{eqnarray}

\subsection{Power Function in y\label{power_y}}
\begin{eqnarray}
{\bf u}(x,y) &=& \left( \begin{array}{c} -c_s \\ c_n \end{array}   \right)y^{n} \\
\nabla{\bf u}^* &=& \left( \begin{array}{c} \partial_x \\ \partial_y \end{array}   \right) \left( \begin{array}{cc} -c_sy & c_ny \end{array}   \right)
 =  \left( \begin{array}{cc}  0 & 0 \\ -c_s & c_n  \end{array}   \right) ny^{n-1}\\
\nabla_s{\bf u}^* &=&  \left( \begin{array}{cc} 0 & -\frac{1}{2}c_s \\ -\frac{1}{2}c_s & c_n \end{array}   \right) ny^{n-1}\\
\nabla\cdot\left(\nabla_s{\bf u}\right) &=&  \left( \begin{array}{cc} \partial_x & \partial_y \end{array}   \right) 
\left( \begin{array}{cc} 0 & -\frac{1}{2}c_s \\ -\frac{1}{2}c_s & c_n \end{array}   \right) ny^{n-1}
= \left( \begin{array}{cc} -\frac{1}{2}c_s & c_n \end{array}   \right) n(n-1)y^{n-2}
\end{eqnarray}

\subsection{Power Function, rotated about an arbitrary angle $\theta_r$\label{power_rotation}}
Using the definitions of f and g in (\ref{def f}-\ref{def g}),
\begin{eqnarray}
{\bf u}(r,\theta;\theta_r) &=& f^n{\bf g}\\
\nabla{\bf u}^* &=& \left( \nabla f^n \right) {\bf g}^* 
=  n f^{n-1} \nabla f {\bf g}^* 
 = n f^{n-1} \left( \begin{array}{c} \cos{\theta_r} \\ \sin{\theta_r} \end{array}    \right)  {\bf g}^* \\
\nabla_s{\bf u}^* &=&  n f^{n-1}
        \left(\begin{array}{cc} g_1\cos{\theta_r} & \frac{1}{2}\left( g_1\sin{\theta_r} + g_2\cos{\theta_r} \right)\\
                        (sym) & g_2\sin{\theta_r}  \end{array} \right)\\
\nabla\cdot\left(\nabla_s{\bf u}\right) &=&  \left( \begin{array}{cc} \partial_x & \partial_y \end{array}   \right)
n f^{n-1} \left(\begin{array}{cc} g_1\cos{\theta_r} & \frac{1}{2}\left( g_1\sin{\theta_r} + g_2\cos{\theta_r} \right)\\
                        (sym) & g_2\sin{\theta_r}  \end{array} \right)\\
 &=& n(n-1) f^{n-2} \left( \begin{array}{c} \cos{\theta_r} \\ \sin{\theta_r} \end{array}    \right)^*  
\left(\begin{array}{cc} g_1\cos{\theta_r} & \frac{1}{2}\left( g_1\sin{\theta_r} + g_2\cos{\theta_r} \right)\\
                        (sym) & g_2\sin{\theta_r}  \end{array} \right)\\
 &=& n(n-1) f^{n-2} \left(\begin{array}{c} 
   g_1\left(\cos^2{\theta_r} + \frac{1}{2}\sin^2{\theta_r}\right) + g_2\frac{1}{2}\cos{\theta_r}\sin{\theta_r} \\
   g_2\left(\sin^2{\theta_r} + \frac{1}{2}\cos^2{\theta_r}\right) + g_1\frac{1}{2}\cos{\theta_r}\sin{\theta_r} \\
\end{array} \right)^*
\end{eqnarray}

\subsection{Sine function, rotated about an arbitrary angle $\theta_r$\label{sine_rotation}}
Using the definitions of f and g in (\ref{def f}-\ref{def g}),
\begin{eqnarray}
{\bf u}(r,\theta;\theta_r) &=& \sin{\left(\frac{2\pi}{L}r\cos{(\theta-\theta_r)}\right)}
   R_{\theta_r} \left( \begin{array}{c} c_n \\ c_s \end{array}   \right) \\
&=& \sin{\left(\frac{2\pi}{L}f\right)}{\bf g} \\
\nabla{\bf u}^* &=& \frac{2\pi}{L}\cos{\left(\frac{2\pi}{L}f\right)}\left( \nabla f \right) {\bf g}^* \\
 &=& \frac{2\pi}{L}\cos{\left(\frac{2\pi}{L}f\right)}
\left( \begin{array}{c} \cos{\theta_r} \\ \sin{\theta_r} \end{array}    \right)  
\left(\begin{array}{cc} g_1 & g_2  \end{array} \right) \\
&=& \frac{2\pi}{L}\cos{\left(\frac{2\pi}{L}f\right)}
\left(\begin{array}{cc} g_1\cos{\theta_r} & g_2\cos{\theta_r} \\ g_1\sin{\theta_r} & g_2\sin{\theta_r}  \end{array} \right)\\
\nabla_s{\bf u}^* &=& \frac{2\pi}{L}\cos{\left(\frac{2\pi}{L}f\right)}
\left(\begin{array}{cc} g_1\cos{\theta_r} & \frac{1}{2}\left( g_1\sin{\theta_r} + g_2\cos{\theta_r} \right)\\
                      (sym) & g_2\sin{\theta_r}  \end{array} \right)\\
\nabla\cdot\left(\nabla_s{\bf u}\right) &=&  \left( \begin{array}{cc} \partial_x & \partial_y \end{array}   \right)
\frac{2\pi}{L}\cos{\left(\frac{2\pi}{L}f\right)}
\left(\begin{array}{cc} g_1\cos{\theta_r} & \frac{1}{2}\left( g_1\sin{\theta_r} + g_2\cos{\theta_r} \right)\\
                      (sym) & g_2\sin{\theta_r}  \end{array} \right)\\
&=& -\left(\frac{2\pi}{L}\right)^2
\sin{\left(\frac{2\pi}{L}f\right)}\left( \nabla f \right)^* 
\left(\begin{array}{cc} g_1\cos{\theta_r} & \frac{1}{2}\left( g_1\sin{\theta_r} + g_2\cos{\theta_r} \right)\\
                      (sym) & g_2\sin{\theta_r}  \end{array} \right)\\
&=& -\left(\frac{2\pi}{L}\right)^2
\sin{\left(\frac{2\pi}{L}f\right)}
\left( \begin{array}{c} \cos{\theta_r} \\ \sin{\theta_r} \end{array}    \right)  ^* 
\left(\begin{array}{cc} g_1\cos{\theta_r} & \frac{1}{2}\left( g_1\sin{\theta_r} + g_2\cos{\theta_r} \right)\\
                      (sym) & g_2\sin{\theta_r}  \end{array} \right)\\
&=& -\left(\frac{2\pi}{L}\right)^2
\sin{\left(\frac{2\pi}{L}f\right)}
\left(\begin{array}{c} g_1\cos^2{\theta_r} + \frac{1}{2}\left( g_1\sin^2{\theta_r} + g_2\cos{\theta_r}\sin{\theta_r} \right)\\
                       g_2\sin^2{\theta_r} + \frac{1}{2}\left( g_2\cos^2{\theta_r} + g_1\cos{\theta_r}\sin{\theta_r} \right) 
 \end{array} \right)^* 
\end{eqnarray}

\section{Test functions on a sphere}
Date last modified: 5/6/2013 \\
Contributors: Mark Petersen \\

\subsection{Strain Rate}

The strain rate tensor in standard spherical coordinates $(\varphi,\theta,r)$, where $\theta=0$ points towards the north pole, ${\bf x}=(0,0,1)$, is \cite[p. 868]{Kundu_ea12bk}
\begin{eqnarray}
\epsilon = \left[\begin{array}{ccc}
\displaystyle
\frac{1}{r\sin\theta} \left( u_{\varphi,\varphi} + u_r\sin\theta +  u_\theta\cos\theta \right) &  & (sym) \\
\displaystyle
\frac{1}{2r}\left(u_{\varphi,\theta} - u_\varphi \cot\theta  +  \frac{1}{\sin\theta}u_{\theta,\varphi}  \right)
\displaystyle
   & \frac{1}{r} u_{\theta,\theta} + \frac{1}{r} u_{r} & \\
\displaystyle
\frac{1}{2r\sin\theta}u_{r,\varphi} + \frac{1}{2}u_{\varphi,r} - \frac{1}{2r}u_\varphi &  
\displaystyle
\frac{1}{2r}u_{r,\theta} + \frac{1}{2}u_{\theta,r} - \frac{1}{2r}u_\theta  & u_{r,r}
\end{array}\right],
\end{eqnarray}
where variables in subscripts after commas are derivatives.  For fluids on a sphere, radial velocities and derivatives are assumed to be zero, so the strain rate tensor reduces to
\begin{eqnarray}
\epsilon = \left[\begin{array}{ccc}
\displaystyle
\frac{1}{r\sin\theta} \left( u_{\varphi,\varphi} +  u_\theta\cos\theta \right) &  & (sym) \\
\displaystyle
\frac{1}{2r}\left(u_{\varphi,\theta} - u_\varphi \cot\theta  +  \frac{1}{\sin\theta}u_{\theta,\varphi}  \right)
\displaystyle
   & \frac{1}{r} u_{\theta,\theta}  & \\
\displaystyle
 - \frac{1}{2r}u_\varphi &  
\displaystyle
 - \frac{1}{2r}u_\theta  & 0
\end{array}\right]
\end{eqnarray}
We now change to longidude-latitude-radial coordinates using the variables $(\lambda, \phi, r)$.  These differ from those in \cite{Kundu_ea12bk} in that $\phi=0$ at the equator and $\phi=\pi/2$ radians at the north pole.  Then $\phi=\pi/2-\theta$, $\partial/\partial\phi=-\partial/\partial\theta$, $\sin\theta=\cos\phi$, and $\cos\theta=\sin\phi$.  The strain rate tensor in longitude-latitude-radial coordinates is now
\begin{eqnarray}
\epsilon = \left[\begin{array}{ccc}
\displaystyle
\frac{1}{r\cos\phi} \left( u_{\lambda,\lambda} +  u_\phi\sin\phi \right) &  & (sym) \\
\displaystyle
\frac{1}{2r}\left(-u_{\lambda,\phi} - u_\lambda \tan\phi  +  \frac{1}{\cos\phi}u_{\phi,\lambda}  \right)
\displaystyle
   & -\frac{1}{r} u_{\phi,\phi}  & \\
\displaystyle
 - \frac{1}{2r}u_\lambda &  
\displaystyle
 - \frac{1}{2r}u_\phi  & 0
\end{array}\right].
\end{eqnarray}
Although this seems like a straight forward substitution, some of the signs are not correct.  Going back the the derivation of the stress tensor in curvilinear coordinates \cite[Appendix 2, page 600]{Batchelor67book}, as summarized in \cite[Section 2b]{Hunke_Dukowicz02mwr}, 
\begin{eqnarray}
\epsilon_{11}&=&\frac{1}{h_1}\frac{\partial u_1 }{\partial \xi_1} + \frac{u_2}{h_1 h_2}\frac{\partial h_1 }{\partial \xi_2} + \frac{u_3}{h_3 h_1}\frac{\partial h_1 }{\partial \xi_3} \\
\epsilon_{22}&=&\frac{1}{h_2}\frac{\partial u_2 }{\partial \xi_2} + \frac{u_3}{h_2 h_3}\frac{\partial h_2 }{\partial \xi_3} + \frac{u_1}{h_1 h_2}\frac{\partial h_2 }{\partial \xi_1} \\
\epsilon_{33}&=&\frac{1}{h_3}\frac{\partial u_3 }{\partial \xi_3} + \frac{u_1}{h_3 h_1}\frac{\partial h_3 }{\partial \xi_1} + \frac{u_2}{h_2 h_3}\frac{\partial h_3 }{\partial \xi_2} \\
\epsilon_{12}&=&\frac{h_2}{2h_1} \frac{\partial }{\partial \xi_1}\left(\frac{u_2}{h_2}\right) + \frac{h_1}{2h_2} \frac{\partial }{\partial \xi_2}\left(\frac{u_1}{h_1}\right) \\
\epsilon_{23}&=&\frac{h_3}{2h_2} \frac{\partial }{\partial \xi_2}\left(\frac{u_3}{h_3}\right) + \frac{h_2}{2h_3} \frac{\partial }{\partial \xi_3}\left(\frac{u_2}{h_2}\right) \\
\epsilon_{31}&=&\frac{h_1}{2h_3} \frac{\partial }{\partial \xi_3}\left(\frac{u_1}{h_1}\right) + \frac{h_3}{2h_1} \frac{\partial }{\partial \xi_1}\left(\frac{u_3}{h_3}\right) 
\end{eqnarray}
where subscripts after commas are derivatives.  For the longidude-latitude-radial coordinates, set $u_3=0$, $\xi_1=\lambda$ (longitude), $\xi_2=\phi$ (latitude), $\xi_3=r$.  Then the scale factors are $h_1=r \cos\phi$, $h_2=r$, and $h_3=1$, and the strain rate is 
\begin{eqnarray}
\epsilon_{11}&=&\frac{1}{r \cos\phi} \frac{\partial u_\lambda }{\partial \lambda} + \frac{u_\phi}{r \cos\phi r}\frac{\partial r \cos\phi }{\partial \phi} \\
\epsilon_{22}&=&\frac{1}{r} \frac{\partial u_\phi }{\partial \phi} + \frac{u_\lambda}{r \cos\phi r}\frac{\partial r }{\partial \lambda} \\
\epsilon_{33}&=& \frac{u_1}{ r \cos\phi}\frac{\partial 1 }{\partial \lambda} + \frac{u_2}{r }\frac{\partial 1 }{\partial \phi} \\
\epsilon_{12}&=&\frac{r}{2r \cos\phi} \frac{\partial }{\partial \lambda}\left(\frac{u_\phi}{r}\right) + \frac{r \cos\phi}{2r} \frac{\partial }{\partial \phi}\left(\frac{u_\lambda}{r \cos\phi}\right) \\
\epsilon_{23}&=&\frac{r}{2} \frac{\partial }{\partial r}\left(\frac{u_\phi}{r}\right) \\
\epsilon_{31}&=&\frac{r \cos\phi}{2} \frac{\partial }{\partial r}\left(\frac{u_\lambda}{r \cos\phi}\right) 
\end{eqnarray}

The strain rate tensor in spherical coordinates is then
\begin{eqnarray}
\epsilon = \left[\begin{array}{ccc}
\displaystyle
\frac{1}{r\cos\phi} \left( u_{\lambda,\lambda} -  u_\phi\sin\phi \right) &  & (sym) \\
\displaystyle
\frac{1}{2r}\left(u_{\lambda,\phi} + u_\lambda \tan\phi  +  \frac{1}{\cos\phi}u_{\phi,\lambda}  \right)
\displaystyle
   & \frac{1}{r} u_{\phi,\phi}  & \\
\displaystyle
 - \frac{1}{2r}u_\lambda &  
\displaystyle
 - \frac{1}{2r}u_\phi  & 0
\end{array}\right].
\end{eqnarray}

In my numerical simulations, I found that $\epsilon_{23}$ and $\epsilon_{31}$ were off by a factor of three from this derivation.  That is, my numerical solutions converged to
\begin{eqnarray}
\epsilon = \left[\begin{array}{ccc}
\displaystyle
\frac{1}{r\cos\phi} \left( u_{\lambda,\lambda} -  u_\phi\sin\phi \right) &  & (sym) \\
\displaystyle
\frac{1}{2r}\left(u_{\lambda,\phi} + u_\lambda \tan\phi  +  \frac{1}{\cos\phi}u_{\phi,\lambda}  \right)
\displaystyle
   & \frac{1}{r} u_{\phi,\phi}  & \\
\displaystyle
 - \frac{3}{2r}u_\lambda &  
\displaystyle
 - \frac{3}{2r}u_\phi  & 0
\end{array}\right].
\end{eqnarray}

The simplest cases are when the zonal flow $u_\lambda=1$ and meridional flow $u_\phi=0$, 
\begin{eqnarray}
\epsilon = \left[\begin{array}{ccc}
0 &  & (sym) \\
\displaystyle
\frac{\tan\phi}{2r} & 0 &\\
\displaystyle
 - \frac{3}{2r} & 0 & 0
\end{array}\right], 
\end{eqnarray}
when the zonal flow $u_\lambda=0$ and meridional flow $u_\phi=1$,
\begin{eqnarray}
\epsilon = \left[\begin{array}{ccc}
\displaystyle
-\frac{\tan\phi}{r} &  & (sym) \\
0 & 0 &\\
 0 & \displaystyle - \frac{3}{2r} & 0 
\end{array}\right], 
\end{eqnarray}
and solid-body rotation, $u_\lambda=\cos\phi$, $u_\phi=0$, so that the two terms in $\epsilon_{12}$ subtract out and 
\begin{eqnarray}
\epsilon = \left[\begin{array}{ccc}
\displaystyle
0 &  & (sym) \\
0 & 0 &\\
 \displaystyle - \frac{3}{2r}\cos\phi & 0 & 0
\end{array}\right].
\end{eqnarray}
For more general testing, we chose the functions $u_\lambda=\cos\lambda (1+\cos(2\phi))$, $u_\phi=0$, which has the solution shown in Figure \ref{fig:mathematica1}, and $u_\lambda=0$, $u_\phi=\cos\lambda (1+\cos(2\phi))$, which has the solution shown in Figure \ref{fig:mathematica2}.  These forms were chosen so that the test functions have a zero value and zero derivative at the poles.  Convergence plots are shown in figures \ref{fig:convergence1} and \ref{fig:convergence2}, respectively.

\begin{figure}[htbp]
 \center
 \includegraphics[scale=0.8, trim = 0 4in 0 0, clip]{f/130520_u_coscos.pdf}
 \caption{Mathematica worksheet to compute strain rate.  Test function is $u_{\lambda}=\cos \lambda (1+\cos(2\phi))$, $u_{\phi}=0$}
 \label{fig:mathematica1}
\end{figure}

\begin{figure}[htbp]
 \center
 \includegraphics[scale=0.8, trim = 0 0 0 4in, clip]{f/a19_sph_conv_test_1.pdf}
 \caption{Convergence of strain rate computation on a quasi-uniform spherical mesh.  Test function is $u_{\lambda}=\cos \lambda (1+\cos(2\phi))$, $u_{\phi}=0$}
 \label{fig:convergence1}
\end{figure}

\begin{figure}[htbp]
 \center
 \includegraphics[scale=0.8, trim = 0 0 0 4in, clip]{f/a19_sph_conv_test_6.pdf}
 \caption{Convergence of strain rate computation on a variable resolution spherical mesh.  Test function is $u_{\lambda}=\cos \lambda (1+\cos(2\phi))$, $u_{\phi}=0$.  The 120km mesh has two obtuse triangles, showing that the CVT grid is not fully converged.  This produces higher errors for that resolution.}
 \label{fig:convergence1b}
\end{figure}

\begin{figure}[htbp]
 \center
 \includegraphics[scale=0.8, trim = 0 4in 0 0, clip]{f/130520_v_coscos.pdf}
 \caption{Mathematica worksheet to compute strain rate.  Test function is $u_{\lambda}=0$, $u_{\phi}=\cos \lambda (1+\cos(2\phi))$}
 \label{fig:mathematica2}
\end{figure}

\begin{figure}[htbp]
 \center
 \includegraphics[scale=0.8, trim = 0 0 0 4in, clip]{f/a19_sph_conv_test_2.pdf}
 \caption{Convergence of strain rate computation on a quasi-uniform spherical mesh.  Test function is $u_{\lambda}=0$, $u_{\phi}=\cos \lambda (1+\cos(2\phi))$}
 \label{fig:convergence2}
\end{figure}


\subsection{Divergence of a tensor}

The divergence of a tensor $\sigma$ in longidude-latitude-radial coordinates using the variables $(\lambda, \phi, r)$, is 
\begin{eqnarray}
\nabla\cdot\sigma
&=& \frac{1}{r\cos\phi} \left( \begin{array}{c} \sigma_{\lambda\lambda,\lambda} \\ \sigma_{\phi\lambda,\lambda} \\ \sigma_{r\lambda,\lambda} \end{array} \right)
+ \frac{1}{r}       \left( \begin{array}{c} \sigma_{\lambda\phi,\phi} \\ \sigma_{\phi\phi,\phi} \\ \sigma_{r\phi,\phi} \end{array} \right)
+                   \left( \begin{array}{c} \sigma_{\lambda r,r} \\ \sigma_{\phi r,r} \\ \sigma_{rr,r} \end{array} \right) \nonumber
\\ && 
+ \frac{1}{r}       \left( \begin{array}{c} 0 \\ \tan\phi \\ -1 \end{array} \right) \sigma_{\lambda\lambda} 
+ \frac{1}{r}       \left( \begin{array}{c} 0 \\ -\tan\phi \\ -1 \end{array} \right) \sigma_{\phi\phi} 
+ \frac{1}{r}       \left( \begin{array}{c} 0 \\ 0 \\ 0 \end{array} \right) \sigma_{r r}   \nonumber
\\ &&
+ \frac{1}{r}       \left( \begin{array}{c} -2\tan\phi \\ 0 \\ 0 \end{array} \right) \sigma_{\lambda\phi} 
+ \frac{1}{r}       \left( \begin{array}{c} 0 \\ 1 \\ -\tan\phi \end{array} \right) \sigma_{r\phi} 
+ \frac{1}{r}       \left( \begin{array}{c} 1 \\ 0 \\ 0 \end{array} \right) \sigma_{r\lambda}.
\end{eqnarray}

This formula was found on a few on-line sources, such as Continuum Mechanics notes by Prof. R Teyssier.  I was not able to find it in a textbook yet.

For the test function $\sigma_{\lambda\lambda}=\cos\lambda (1+\cos(2\phi))$ and $\sigma=0$ for other tensor elements, the analytical solution is shown in Figure \ref{fig:mathematica3}, and a convergence plot in Figure \ref{fig:convergence3}.  For the test function $\sigma_{\lambda\phi}=\cos\lambda (1+\cos(2\phi))$ and $\sigma=0$ for other tensor elements, the analytical solution is shown in Figure \ref{fig:mathematica4}, and a convergence plot in Figure \ref{fig:convergence4}. 


\begin{figure}[htbp]
 \center
 \includegraphics[scale=0.8, trim = 0 4in 0 0, clip]{f/130520_ELonLon_coscos.pdf}
 \caption{Mathematica worksheet to compute strain rate.  Test function is $\sigma_{\lambda\lambda}=\cos \lambda (1+\cos(2\phi))$, $\sigma_{ii}=0$ otherwise.}
 \label{fig:mathematica3}
\end{figure}

\begin{figure}[htbp]
 \center
 \includegraphics[scale=0.8, trim = 0 0 0 4in, clip]{f/a19_sph_conv_test_3.pdf}
 \caption{Convergence of divergence of a tensor computation on a quasi-uniform spherical mesh.  Test function is $\sigma_{\lambda\lambda}=\cos \lambda (1+\cos(2\phi))$, $\sigma_{ii}=0$ otherwise.}
 \label{fig:convergence3}
\end{figure}

\begin{figure}[htbp]
 \center
 \includegraphics[scale=0.8, trim = 0 4in 0 0, clip]{f/130520_ELonLat_coscos.pdf}
 \caption{Mathematica worksheet to compute strain rate.  Test function is $\sigma_{\phi\phi}=\cos \lambda (1+\cos(2\phi))$, $\sigma_{ii}=0$ otherwise.}
 \label{fig:mathematica4}
\end{figure}

\begin{figure}[htbp]
 \center
 \includegraphics[scale=0.8, trim = 0 0 0 4in, clip]{f/a19_sph_conv_test_4.pdf}
 \caption{Convergence of divergence of a tensor computation on a quasi-uniform spherical mesh.  Test function is $\sigma_{\phi\phi}=\cos \lambda (1+\cos(2\phi))$, $\sigma_{ii}=0$ otherwise.}
 \label{fig:convergence4}
\end{figure}

\bibliographystyle{alpha}
\bibliography{tensor_operations}



%-----------------------------------------------------------------------
\end{document}
