\documentclass[11pt]{report}

\usepackage{epsf,amsmath,amsfonts}
\usepackage{graphicx}
\usepackage{listings}

\begin{document}

\title{Scratch Variables: \\
Requirements and Design}
\author{MPAS Development Team}

\maketitle
\tableofcontents

%-----------------------------------------------------------------------

\chapter{Summary}


This document describes the requirements and implementation of scratch variables, along with the general use guidelines. 

%-----------------------------------------------------------------------

\chapter{Requirements}

\section{Requirement: Non-Persistent Global Fields}
Date last modified: 03/07/13 \\
Contributors: (Doug Jacobsen) \\

Scratch variables are defined as non-persistent fields. Currently these types
are defined, allocated, and deallocated at the module level. As OpenMP
implementations have yet been determined for core/shared code, scratch
varaibles are created to allow heap defined arrays that can be public to
multiple threads.

%-----------------------------------------------------------------------

\chapter{Design and Implementation}

\section{Implementation: Non-Persistent Global Fields}
Date last modified: 03/07/13 \\
Contributors: (Doug Jacobsen) \\

Currently in Registry, one defines a persistent variable in a line similar to the following:
\begin{lstlisting}
var persistent integer cellsOnEdge ( TWO nEdges ) 0 iro cellsOnEdge mesh - -
\end{lstlisting}

where the keyword is persistent. This means that MPAS is responsible for
allocating and deallocating this field at the beginning and end of a
simulation. The alternative would be changing the word persisntent to scratch.

This modification would mean that the developer is responsible for allocating
and deallocating the field at the beginning and end of use (or a subroutine). 

In order to accomplish this, MPAS needs to provide a developer with subroutines
that can allocate and deallocate these scratch fields. MPAS will provide
generic interfaces for allocation and deallocation as follows:

\begin{lstlisting}
subroutine mpas_allocate_scratch_field(f, single_block_in)
subroutine mpas_deallocate_scratch_field(f, single_block_in)
\end{lstlisting}

These generic interfaces lie on top of more specific interfaces such as:

\begin{lstlisting}
subroutine mpas_allocate_scratch_field4d_real(f, single_block_in)
\end{lstlisting}

Each of these subroutines takes one required argument and one optional
argument. The optional argument is single\_block\_in, which tells the routine
if you want to allocate/deallocate a single block, or all blocks in a
blocklist.

The subroutines will then allocate or deallocate all field pointers requested
(either a single block or all blocks), and return.

Once a scratch field is allocated, it can be used exactly as a persisntent
field. However, one caveat is that a scratch variable can't be included in any
I/O streams, as the I/O layer has to assume since it's a scratch variable that
it will not be allocated when it has access to it.

%-----------------------------------------------------------------------

%\chapter{Testing}
%
%\section{Testing and Validation: XXX}
%Date last modified: 2011/01/05 \\
%Contributors: (add your name to this list if it does not appear) \\
%
%How will XXX be tested? i.e. how will be we know when we have met requirement XXX. Will these unit tests be included in the ongoing going forward?
%

%-----------------------------------------------------------------------

\chapter{Use Guidelines}

Scratch variables are intended to be used as temporary storage space. The
alternative is persisntent variables which are intended to be used as permanent
storage space. Because scratch is inheriently temporary, a developer should
assume they are responsible for all memory management with scratch varaibles. 

This means a general requirement is that the developer should ensure they call
the allocate and deallocate routines at the correct places within the code. The
definition of correct largely depends on the application, but a good rule of
thumb is that correct is immediately before and immediately after the variable
is required, and incorrect would be anywhere else in the core. Typically
scratch variables should be allocated and deallocated within the subroutine
that makes use of them. Allocation would happen at the beginning of the
subroutine, and deallocation would happen at the end of the subroutine. The
easiest use case would be replacing module level variables that have a
dimension of nElements with a scratch variable, everything is almost identical
between the two.

Scratch variables should have useful names, similarly to persisntent variables,
that represent the quantity (although it is temporary) that should be stored in
them.

If a scratch variable is going to be used at a level below where it's allocated
(within another subroutine) that subroutine should expect it as an argument,
and assume that it's already allocated upon calling. In general, a scratch data
structure can be passed around (similarly to grid or mesh) that contains all
scratch variables. The subroutine can then use only the relevent variables from
that data structure.

Scratch variables can't be used for simple 1D arrays that *should* be thread
private. Examples of this are 1D arrays that are dimensioned by nVertLevels.
The scratch counterpart of this 1D array would be a 2D array that is
dimensioned nVertLevels by nElements. However, 1D arrays that are dimensioned
by nElements are fine to use in scratch variables.

In general, there's no reason one can't manually allocate the array pointer for
a scratch variable and make use of it exactly as a module level variable. This
should be discouraged as it doesn't follow the general use guidelines of
traditional variables, and could cause other developers problems when using
that scratch variable.
\end{document}
