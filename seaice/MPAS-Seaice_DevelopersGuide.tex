\documentclass{article}

\begin{document}

\title{MPAS-Seaice developer guide}
\author{Adrian K. Turner} 
\date{\today}
\maketitle

%-------------------------------------------------------------------------------------------------------------------

\section{Preliminaries}
\label{sec:preliminaries}

Here we describe preliminary tasks required before MPAS can be installed.

\begin{enumerate}

\item Gain access to the MPAS git repository

\item Create a git fork of the MPAS/MPAS-Dev repository on Github

\item Install the libraries required for MPAS: netcdf, pnetcdf and PIO

\begin{itemize}

\item On LANL institutional computing use the pre-compiled library modules:

\begin{verbatim}
module use /usr/projects/climate/SHARED_CLIMATE/modulefiles/all/
module load \
  gcc/4.8.2 \
  openmpi/1.6.5 \
  netcdf/4.4.0 \
  parallel-netcdf/1.5.0 \
  pio/1.7.2
\end{verbatim}

\item Otherwise, manually install and set the \verb+$NETCDF+, \verb+$PNETCDF+ and \verb+$PIO+ environment variables. Netcdf version 3.1.4, Pnetcdf version 1.7.0 and PIO version 1.7.2 are known to work.

\end{itemize}

\item Install fortran compilers including MPI versions.

\end{enumerate}

%-------------------------------------------------------------------------------------------------------------------

\section{Quick start guide}
\label{sec:quick_start_guide}

Here we describe how to download, build and run MPAS-Seaice using the gfortran compiler.

\begin{enumerate}

\item Check out the code from the git repository

\verb+git clone git@github.com:<git_username>/MPAS.git+

Here \verb+<git_username>+ is your MPAS repository username

\item Move into the MPAS repository

\verb+cd MPAS+

\item Checkout the \verb+cice/develop+ branch

\verb+git checkout cice/develop+

\item Compile the code

\verb+make CORE=cice gfortran+

\item Create a run directory

\verb+cd ../ ; mkdir rundir ; cd rundir+

\item Set up the run directory (see section \ref{sec:standard_run}; in this case \verb+<MPAS_directory>+ is \verb+../MPAS/+) 

\item Run the model

\verb+./cice_model+

\end{enumerate}

%-------------------------------------------------------------------------------------------------------------------

\section{Setting up a standard run in a run directory}
\label{sec:standard_run}

Here we describe how to set up a standard QU120km run of MPAS-Seaice in a run directory

\begin{enumerate}

\item Move to your run directory

\verb+cd <run_directory>+

\item Create symbolic links to the grid file, standard forcing files and graph files for a QU120km domain

\verb+<MPAS-Seaice_data_directory>/domains/domain_QU120km/get_domain.py+

\item Add the namelist and streams file for a standard test run

\verb+cp <MPAS_Directory>/testing_and_setup/seaice/configurations/standard_physics/* .+

\item Get a symbolic link to the MPAS-Seaice executable

\verb+ln -s <MPAS_directory>/cice_model .+

\end{enumerate}

Note: On LANL IC machines at LANL \verb+<MPAS-Seaice_data_directory>+ is \verb+/turquoise/usr/projects/climate/akt/MPAS-seaice/+

%-------------------------------------------------------------------------------------------------------------------

\section{Development workflow}
\label{sec:development}

Here we describe the work flow needed to add new features to the sea ice model.

\begin{enumerate}

\item Create a feature branch directory:

\verb+mkdir <feature_directory>+

\verb+cd <feature_directory>+

\item Create a development and base directory within this feature directory. Code changes will be made in the development directory and compared during testing against an unchanged version in base.

\verb+mkdir development+

\verb+mkdir base+

\item Create the run directory that will be used for bit reproducible testing

\verb+cd base/+

\item Clone and compile \verb+cice/develop+ in the base directory.
\begin{verbatim}
git clone git@github.com:MPAS-Dev/MPAS.git
cd MPAS/
git checkout cice/develop
make CORE=cice gfortran
\end{verbatim}

\item Enter the development directory and create a testing run directory

\verb+cd ../../development+



\item Create a new feature branch from cice/develop to develop the new feature in and compile it.
\begin{verbatim}
git clone git@github.com:<git_username>/MPAS.git
cd MPAS/
git remote add MPAS-Dev git@github.com:MPAS-Dev/MPAS.git
git fetch --all
git checkout -b cice/<feature_branch_name> MPAS-Dev/cice/develop
make CORE=cice gfortran
\end{verbatim}
 
\item Make the changes to the code needed to implement the desired feature

\item Test that your changes haven't broken the code (see section \ref{sec:testing}).

\item Commit your changes to the repository (see section \ref{sec:committing}).

\item Push your changes to your fork on Github and create a pull request for these features to be merged into \verb+cice/develop+ (see section \ref{sec:pull_request}).
  
\end{enumerate}

%-------------------------------------------------------------------------------------------------------------------

\section{Testing MPAS-Seaice}
\label{sec:testing}

Here we describe how to run the standard MPAS-Seaice tests:

\begin{enumerate}

\item Create a testing directory and cd into it

\verb+mkdir testdir ; cd testdir+

\item Run the MPAS testing system

\begin{verbatim}
<MPAS_Directory>/testing_and_setup/seaice/testing/test_mpas-seaice.py \
   -d <Development_MPAS_Dir> \
   -b <Base_MPAS_Dir>
\end{verbatim}

\verb+<Development_MPAS_Dir>+ is the development MPAS directory, and \verb+<Base_MPAS_Dir>+ is the base MPAS directory that will be compared against the development MPAS directory.

\end{enumerate}

Note: the testing system assumes that python 2.7 is used and a number of python packages are installed.

%-------------------------------------------------------------------------------------------------------------------

\section{Committing changes to the repository}
\label{sec:committing}

Once you are happy with the changes made to your code the changes must be committed to the local repository

\begin{enumerate}

\item Move to the MPAS main directory

\verb+cd <MPAS_directory>+

\item See what files have changed since the last commit

\verb+git status+

\item See what has changed within those files

\verb+git diff+

\item Add newly created files to the commit

\verb+git add <new_file>+

\item Add changed files to the commit

\verb+git add <changed_file>+

\item Commit all the changes that have been queued up for commit by the last two sets of command

\verb+git commit+

This should open up a text editor that allows a commit message to be written. Saving and closing this file triggers the commit.

\item Alternatively, all the file changes can be added to the commit in a single command. Note: this doesn't add new files.

\verb+git commit -a+


\end{enumerate}

%-------------------------------------------------------------------------------------------------------------------

\section{Making pull requests into cice/develop}
\label{sec:pull_request}

Once all the changes and new files needed to implement a new feature are implemented then a request must be made to have this feature merged into \verb+cice/develop+. We assume all desired changes have been committed into the local repository as described in section \ref{sec:committing}.

\begin{enumerate}

\item Firstly, its useful to see what branch we are currently on.

\verb+git branch+

The current branch is highlighted with an asterix.

\item Move to the branch we want to submit the pull request for (you're most likely already on it if you're following these instructions).

\verb+git checkout <feature_branch>+

\item Push the branch to your fork on Github

\verb+git push origin <feature_branch>+

\item Go to github.com and find the branch you just pushed there. Click on the submit pull request button and make sure the pull request is into \verb+cice/develop+ and not \verb+develop+ (which is the default). Add the current MPAS-Seaice integrator (currently \verb+akturner+) as the assignee and add a \verb+Sea ice+ label.

\item Go tell the MPAS-Seaice integrator that you have submitted a pull request (currently Adrian Turner). Do not rely on them seeing the automated Github notification email.

\end{enumerate}



%-------------------------------------------------------------------------------------------------------------------

\end{document}