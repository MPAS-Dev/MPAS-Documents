\documentclass[11pt]{report}

\usepackage{epsf,amsmath,amsfonts}
\usepackage{listings}
\usepackage{graphicx}

\begin{document}

\title{Surface Forcing Requirements and Design}
\author{MPAS Development Team}

\maketitle
\tableofcontents

%-----------------------------------------------------------------------

\chapter{Summary}

This document will describe the setup for surface forcing. Surface forcing
includes wind stress, temperature, and salinity.

%-----------------------------------------------------------------------

\chapter{Requirements}

Requirements for surface forcing can be seen below:
\begin{itemize}
	\item The forcing framework should provide capabilities to force
		temperature, salinity, and velocity at the surface.
	\item Forcing should be robust in terms of temporal frequency.
	\item The choice of temporal frequency should not change the variables that
		are used.
	\item Constant and monthly forcing should be provided with the ocean core.
\end{itemize}

These requirements will be used to create an infrastructure that will support
robust forcing of the ocean surface. As mentioned, constant and monthly forcing
will be provided with the ocean core but the framework created should allow
this to be easily extended into other temporal frequencies.
%-----------------------------------------------------------------------

\chapter{Design and Implementation}

In order to support various types of forcing, a new grid dimension will be
created and three variables will be modified.

\begin{lstlisting}[language=fortran,escapechar=@,frame=single]
dim nForcingTimeSlices
\end{lstlisting}
will be created in order to determine how many time slices there are within the
forcing fields.


The variables below
\begin{lstlisting}[language=fortran,escapechar=@,frame=single]
u_src ( nVertLevels nEdges )
temperatureRestore ( nCells )
salinityRestore ( nCells )
windStressMonthly ( nMonths nEdges )
temperatureRestoreMonthly ( nMonths nCells )
salinityRestoreMonthly ( nMonths nCells )
\end{lstlisting}

will be changed into
\begin{lstlisting}[language=fortran,escapechar=@,frame=single]
windStress ( nEdges )
sstRestore ( nCells )
sssRestore ( nCells )
windStressField ( nForcingTimeSlices nEdges )
sstRestoringField ( nForcingTimeSlices nCells )
sssRestoringField ( nForcingTimeSlices nCells )
\end{lstlisting}

where nForcingTimeSlices can be changes based on the frequency of forcing data. As an
example, 1 would represent constant forcing while 12 would represent monthly
forcing.

The namelist variable config\_use\_monthly\_forcing will be changed to
config\_forcing\_interval. Rather than being a logical variable, it will be a
character string. By default the ocean core will support options for 'monthly'
and 'constant'. These can be extended later if needed.

Within this framework, the surfaceWindStresses, sstRestorings, and
sssRestorings variables will be copied or interpolated (based on the choice of
forcing frequency) into the surfaceWindStress, sstRestore, and sssRestore
variables (respectively). These arrays are then used to perform the actual
forcing within the ocean core.

%-----------------------------------------------------------------------

\chapter{Testing}

In order to test, we should ensure bit-for-bit reproduction of data compared
with the current version of the trunk. Using both monthly and constant forcing
methods.

%-----------------------------------------------------------------------

\end{document}
