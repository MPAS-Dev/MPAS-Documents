%
\documentclass[11pt]{report}
\usepackage{amsmath}
\usepackage{amsfonts}
\usepackage{subfigure}
\usepackage{placeins}
\usepackage{graphicx}
\usepackage{listings}
\usepackage{float}
\usepackage{algorithm}
\usepackage{algpseudocode}

\newcommand{\bdot}[0]{\boldsymbol{\cdot}}
\newcommand{\bunderline}[1]{\underline{\bf #1}}
\newcommand{\bvec}[1]{\vec{\bf #1}}
\newcommand{\sgn}[0]{\mbox{sgn}}
\newcommand{\pde}[2]{\frac{\partial {#1}}{\partial {#2}}}
\newcommand{\mDeriv}[1]{\frac{D {#1}}{D t}}
\newcommand{\deriv}[2]{\frac{d {#1}}{d {#2}}}
\newcommand{\subtext}[1]{{\mbox{\tiny #1}}}
\newcommand{\order}[1]{{O(#1)}}
\newcommand{\outprod}[0]{\otimes}
\newcommand{\half}[0]{\frac{1}{2}}
\newcommand{\third}[0]{\frac{1}{3}}
\newcommand{\fourth}[0]{\frac{1}{4}}
\newcommand{\D}[0]{\displaystyle}
\newcommand{\Dfrac}[2]{{\displaystyle \frac{#1}{#2}}}
\newcommand{\svec}[1]{{\Vec{#1}}}
\newcommand{\grad}[0]{\nabla}

\title{Tracer Advection using Characteristic Discontinuous Galerkin}
\author{Robert B. Lowrie}
\setcounter{secnumdepth}{5}
\setcounter{tocdepth}{5}
\begin{document}
\maketitle

\chapter{Summary}
This document outlines the implementation of the Characteristic Discontinuous
Galerkin (CDG) tracer advection scheme within the MPAS framework.
In this document, we refer to the existing advection scheme as FVM, for
finite-volume method.

\section{TODO}

\begin{itemize}
\item Add limiter algorithm.  This will probably change the overall algorithm
  quite a bit, but I still want the base algorithm documented as such, since
  there may be many other ways to limit than the current approach.
\end{itemize}

%%%%%%%%%%%%%%%%%%%%%%%%%%%%%%%%%%%%%%%%%%%%%%%%%%%%%%%%%%%%%%%%%%%%%%%%%%%%%%%%%%%%%%%%%%%%%%%%%%%%%%%%%%%%%%%%
\chapter{Requirements}
%%%%%%%%%%%%%%%%%%%%%%%%%%%%%%%%%%%%%%%%%%%%%%%%%%%%%%%%%%%%%%%%%%%%%%%%%%%%%%%%%%%%%%%%%%%%%%%%%%%%%%%%%%%%%%%%

\begin{enumerate}
	\item Horizontal and vertical advection of tracers.  The horizontal
          and vertical updates may be spatially-split, as they are in the
          existing advection module.  Comments:
          \begin{enumerate}
          \item We might want to use CDG in the horizontal and FVM in the
            vertical, and vice versa. However, how does FVM update the CDG
            polynomial coefficients, aside from the cell-average?
            Consequently, for now, the same method must be used for both
            horizontal and vertical.
          \end{enumerate}
	\item Preserves physical bounds on tracer quantities.  This may also
          be extended to monotone advection, a more stringent requirement.
        \item User-selectable polynomial degree for the representation of the
          tracers.  For example, bi-quadratic would be analogous to Prather's
          method.
            Comments:
          \begin{enumerate}
          \item  At this point, the polynomial degree will be the same
          across all mesh cells and for all tracers.
          \item We likely need ``diamond truncation'' (linear example:
            $\{1,x,y,xy\}$) versus ``triangle truncation'' ($\{1,x,y\}$), but
            we might want to make this an option.
          \end{enumerate}
        \item Operates within current FVM implementation.
        \item The height-function update is performed elsewhere by MPAS.  A
          constant tracer field must maintain this update, to within
          round-off. \label{req:constantq}
        \item Threadable and bit reproducible.
          \label{req:thread}
        \item Allocate additional tracer polynomial coefficients only if CDG is
          selected as the advection scheme.  The tracer cell-average will be CDG's
          leading coefficient. \label{req:cell_avg}
          Comments:
          \begin{enumerate}
          \item Might just add another index to {\tt tracers} array, with the
            first index the cell average.  Another possibility is to store
            separately the CDG tracers and FVM tracers.
          \end{enumerate}
	\item In order to share this module with other dynamical cores,
          isolate code dependencies on MPAS. \label{req:isolate}
        \item Isolate polygon intersections for remap step.  Start with CORE,
          and if available and useful via SciDAC3, consider using MOAB.
          \label{req:remap}
        \item The time step is limited by the number of neighboring cells one
          intersects with each face's Lagrangian pre-image.  The algorithm
          should allow this ``halo''
          (or ``depth'') to be variable and likely will use whatever halo is
          used by the current code, with its implied time step limitation.
          \label{req:halo}
\end{enumerate}

%%%%%%%%%%%%%%%%%%%%%%%%%%%%%%%%%%%%%%%%%%%%%%%%%%%%%%%%%%%%%%%%%%%%%%%%%%%%%%%%%%%%%%%%%%%%%%%%%%%%%%%%%%%%%%%%
\chapter{Formulation for Horizontal Advection}
%%%%%%%%%%%%%%%%%%%%%%%%%%%%%%%%%%%%%%%%%%%%%%%%%%%%%%%%%%%%%%%%%%%%%%%%%%%%%%%%%%%%%%%%%%%%%%%%%%%%%%%%%%%%%%%%

\section{Equations Solved}

The tracer advection system can be written as follows:
\begin{subequations}
\label{eq:tracer_system}
\begin{gather}
  \label{eq:rho}
  \partial_t h + \grad \cdot (h \svec{u}) = 0\,, \\
   \label{eq:tracer}
  \partial_t (h q) + \grad \cdot (h q \svec{u}) = 0\,,
\end{gather}
\end{subequations}
where $h(\svec{x},t)$ is the height field and $q(\svec{x},t)$ the tracer.

At each time-level $t^n$, in each cell $\Omega_z$, we represent the tracer as
\begin{equation}
  \label{eq:expanT}
   q(\svec{x},t^n) = \sum_{j=1}^{N} c^{n}_{z,j} \beta_{z,j}(\svec{x})
       \,, \quad \svec{x} \in \Omega_z\,,
\end{equation}
where the coefficients $\{c^{n}_{z,j}\}_{j=1}^N$ are to be updated by the CDG
method.  The choice of basis functions $\{\beta_{z,j}(\svec{x})\}_{j=1}^N$ will
  be discussed in \S\ref{sec:basis}.

To update the polynomial coefficients, we solve the system
\begin{equation}
  \label{eq:scdg}
  h_z^{n+1} \int\limits_{\Omega_z} \beta_{z,i} q^{n+1} \, d\Omega - 
  h_z^{n}   \int\limits_{\Omega_z} (\phi_{z,i} q)^{n}   \, d\Omega +
  \sum_{f\in\mathcal{F}(z)}
     \frac{m_f^n}{V_f} \int\limits_{\Delta\Omega_{f}}
  (\phi_{z,i} q)^n \, d\Omega\ = 0\,,
\end{equation}
for $i = 1, \ldots, N$, where $m_f^n$ is the height flux through face-$f$ (with the proper sign with
respect to cell-$z$), $\Delta\Omega_{f}$ is the Lagrangian pre-image of
face-$f$, $V_f = |\Delta\Omega_{f}|$, and $\mathcal{F}(z)$ is the set of
faces that make up cell-$z$.  The function $\phi_{z,i}(\svec{x},t^n)$ will be
discussed in \S\ref{sec:phi}.

Substitute (\ref{eq:expanT}) into the first integral in (\ref{eq:scdg}) to
obtain
\begin{equation}
  \label{eq:scdg_massMatrix}
  h_z^{n+1} M_{z,i,j} c^{n+1}_{z,j} - 
  h_z^{n}   \int\limits_{\Omega_z} (\phi_{z,i} q)^{n}   \, d\Omega +
  \sum_{f\in\mathcal{F}(z)}
     \frac{m_f^n}{V_f} \int\limits_{\Delta\Omega_{f}}
  (\phi_{z,i} q)^n \, d\Omega\ = 0\,,
\end{equation}
where for each cell-$z$, $M_{z,i,j}$ is an $N \times N$ matrix, given by
\begin{equation}
  \label{eq:massMatrix}
  M_{z,i,j} = \int\limits_{\Omega_z} \beta_{z,i} \beta_{z,j} \, d\Omega\,.
\end{equation}
The remaining integrals are a function of time-level-$n$ data, so that
(\ref{eq:scdg_massMatrix}) shows the manner in which the coefficients
$c^{n+1}_{z,j}$ may be computed.  For each time step, in order to expedite the
computation of $c^{n+1}_{z,j}$, the matrix inverse $M^{-1}_{z,i,j}$ may be
computed once and stored.  More discussion of this issue will be given in
\S\ref{sec:basis}.  Indeed, the remainder of this chapter discusses how the
terms in (\ref{eq:scdg_massMatrix}) are computed.

%%%%%%%%%%%%%%%%%%%%%%%%%%%%%%%%%%%%%%%%%%%%%%%%%%%%%%%%%%%%%%%%%%%%%%%%%%%%%%%%%%%%%%%%
\section{Determination of $\phi_{z,i}(\svec{x},t^n)$}
\label{sec:phi}
%%%%%%%%%%%%%%%%%%%%%%%%%%%%%%%%%%%%%%%%%%%%%%%%%%%%%%%%%%%%%%%%%%%%%%%%%%%%%%%%%%%%%%%%

Within each time slab $t^n \le t \le t^{n+1}$, the function
$\phi_{z,i}(\svec{x},t)$ is determined by integrating
\begin{equation}
  \label{eq:advectPhi}
  \mDeriv{\phi_{z,i}} = 0\,,
\end{equation}
with $\phi_{z,i}(\svec{x},t^{n+1}) = \beta_{z,i}(\svec{x})$.  An alternative
form is
\begin{equation}
  \label{eq:phiChar2}
   \phi_{z,i}(\svec{x},t) = \beta_{z,i}(\svec{\Gamma}(\svec{x},t))\,,
\end{equation}
where
\begin{equation}
  \label{eq:xc}
  \svec{\Gamma}(\svec{x}, t) = \svec{x} + \int\limits_{t}^{t^{n+1}} 
            \svec{u}(\svec{\Gamma}(\svec{x},\xi), \xi) \, d\xi\,.
\end{equation}
For a constant velocity field, we have that
\begin{equation}
  \phi_{z,i}(\svec{x},t^n) = \beta_{z,i}(\svec{x} + \svec{u} \Delta t)\,.
\end{equation}
In general, (\ref{eq:xc}) may be integrated using Runge-Kutta, or some
other time-integration scheme, to determine $\phi_{z,i}(\svec{x},t^{n})$.
See \S\ref{sec:characteristic} for more discussion.

%%%%%%%%%%%%%%%%%%%%%%%%%%%%%%%%%%%%%%%%%%%%%%%%%%%%%%%%%%%%%%%%%%%%%%%%%%%%%%%%%%%%%%%%
\section{Remap}
\label{sec:remap}
%%%%%%%%%%%%%%%%%%%%%%%%%%%%%%%%%%%%%%%%%%%%%%%%%%%%%%%%%%%%%%%%%%%%%%%%%%%%%%%%%%%%%%%%

This section describes the formation of the Lagrangian pre-image,
$\Delta\Omega_{f}$, which can be viewed as a ``remap'' step.  For each
face-$f$, we approximate the pre-image and compute the integral as follows:
\begin{enumerate}
\item Find the departure points $\svec{d}_1$ and $\svec{d}_2$ for face
  endpoints $\svec{x}_1$ and $\svec{x}_2$.  In general, a
  departure point is found via a relation very similar to (\ref{eq:xc}).  See
  \S\ref{sec:characteristic} for more information.
\item Determine whether the line $(\svec{d}_1, \svec{d}_2)$ intersects
  $(\svec{x}_1, \svec{x}_2)$.
  \begin{itemize}
  \item If the lines do not intersect, then $\Delta\Omega_{f}$ is the
    quadrilateral with points $\{\svec{x}_1, \svec{x}_2, \svec{d}_2,
    \svec{d}_1\}$.
  \item If the lines intersect, let $\svec{x}_I$ be the
    intersection point.  Then $\Delta\Omega_{f}$ is made up of
    two triangles, $\{\svec{x}_1, \svec{x}_I, \svec{d}_1\}$ and $\{\svec{x}_2,
    \svec{x}_I, \svec{d}_2\}$. 
  \end{itemize}
\item Intersect $\Delta\Omega_{f}$ with any number of cells that neighbor
  the face.  The larger the number of cells considered, the larger the maximum
  allowable $\Delta t$.  See Requirement (\ref{req:halo}).
\item For each region found in the previous step, subdivide into triangles and
  integrate numerically.  The details of this integration will be given in
  \S\ref{sec:quadrature}.
\end{enumerate}
We note that the majority of this algorithm can be performed outside of
the inner-loop over tracers.

%%%%%%%%%%%%%%%%%%%%%%%%%%%%%%%%%%%%%%%%%%%%%%%%%%%%%%%%%%%%%%%%%%%%%%%%%%%%%%%%%%%%%%%%
\section{Characteristic Tracing}
\label{sec:characteristic}
%%%%%%%%%%%%%%%%%%%%%%%%%%%%%%%%%%%%%%%%%%%%%%%%%%%%%%%%%%%%%%%%%%%%%%%%%%%%%%%%%%%%%%%%

The velocity characteristics must be traced in two places for the update:
\begin{enumerate}
\item To determine $\phi(\svec{x},t^n)$ in the second and third integrals of
  eq. (\ref{eq:scdg}).  Specifically, given a Gauss point $\svec{x}_g$ at
  $t=t^n$, we must integrate (\ref{eq:xc}) to find where it lands at time
  $t=t^{n+1}$.
\item To determine the departure points $(\svec{d}_1, \svec{d}_2)$, for each
  of the face's endpoints $(\svec{x}_1, \svec{x}_2)$, as described in
  \S\ref{sec:remap}.  Specifically, given $(\svec{x}_1, \svec{x}_2)$ at
  time $t=t^{n+1}$, their departure points are found at $t=t^n$.
\end{enumerate}

The difficulty is that the velocity field is discrete.  In this section, we
describe a second-order accurate, predictor-corrector algorithm.
For Case 1, we have the steps:
\begin{enumerate}
\item Interpolate edge velocities $\svec{u}^n_e$ to at $\svec{x}_g^n$ to find
  $\svec{u}_I^n$.
\item Predict: $\svec{x}_g^{n+1} = \svec{x}_g^{n} + \Delta
  t\svec{u}_I^n$
\item Interpolate edge velocities $\svec{u}^{n+1}_e$ at $\svec{x}_g^{n+1}$
   to find $\svec{u}_I^{n+1}$
\item Let $\svec{u}_I^{n+1/2} = (\svec{u}_I^{n} + \svec{u}_I^{n+1})/2$
\item Correct: $\svec{x}_g^{n+1} = \svec{x}_g^{n} + \Delta
  t\svec{u}_I^{n+1/2}$
\end{enumerate}
Assuming the velocity interpolation is at least second-order accurate in
space, this algorithm is second-order accurate in space and time.

Case 2 is very similar.  For a given vertex $\svec{x}_i$:
\begin{enumerate}
\item Interpolate edge velocities $\svec{u}^{n+1}_e$ to
  $\svec{x}_i$ to find $\svec{u}_I^{n+1}$.
\item Predict: $\svec{d}_i = \svec{x}_i - \Delta
  t\svec{u}_I^{n+1}$
\item Interpolate edge velocities $\svec{u}^{n}_e$ at $\svec{d}_i$
   to find $\svec{u}_I^{n}$
\item Let $\svec{u}_I^{n+1/2} = (\svec{u}_I^{n} + \svec{u}_I^{n+1})/2$
\item Correct: $\svec{d}_i = \svec{x}_i - \Delta
  t\svec{u}_I^{n+1/2}$.
\end{enumerate}

%%%%%%%%%%%%%%%%%%%%%%%%%%%%%%%%%%%%%%%%%%%%%%%%%%%%%%%%%%%%%%%%%%%%%%%%%%%%%%%%%%%%%%%%
\section{Mass Conservation and the Preservation of a Constant Tracer}
%%%%%%%%%%%%%%%%%%%%%%%%%%%%%%%%%%%%%%%%%%%%%%%%%%%%%%%%%%%%%%%%%%%%%%%%%%%%%%%%%%%%%%%%

Any basis will be a ``partition of unity.''  Specifically, for each
cell-$z$, there exists constants $\{\alpha_{j}\}_{j=1}^N$ such that
\begin{equation}
  \sum_{i=1}^N \alpha_{i} \beta_{z,i}(\svec{x}) = 1\,,
\end{equation}
for all $\svec{x}$.
As a result, if we take this same linear combination of (\ref{eq:scdg}), and
assume that $q$ is a constant, then (\ref{eq:scdg}) reduces to
\begin{equation}
  \label{eq:massBal_tmp}
  V_z (h_z^{n+1} - h_z^{n}) +
  \sum_{f\in\mathcal{F}(z)} m_f^n = 0\,.
\end{equation}
This is simply the mass balance over the cell, corresponding to
(\ref{eq:rho}).  We assume that the flow solver satisfies
(\ref{eq:massBal_tmp}), and this is the sense that Requirement
(\ref{req:constantq}) is satisfied.
Note that for MPAS,
\begin{equation}
  m_f^n = \half \Delta t (h_z^n + h_{z'}^n) \svec{u}_f^n \cdot \svec{n}_f\,,
\end{equation}
where cell-$z'$ is the cell neighbor across face-$f$, and the area-weighted
normal $\svec{n}_f$ points in this direction.

%%%%%%%%%%%%%%%%%%%%%%%%%%%%%%%%%%%%%%%%%%%%%%%%%%%%%%%%%%%%%%%%%%%%%%%%%%%%%%%%%%%%%%%%
\section{Choice of Basis Functions}
\label{sec:basis}
%%%%%%%%%%%%%%%%%%%%%%%%%%%%%%%%%%%%%%%%%%%%%%%%%%%%%%%%%%%%%%%%%%%%%%%%%%%%%%%%%%%%%%%%

This section discusses the choice of basis functions
$\{\beta_{z,j}(\svec{x})\}_{j=1}^N$ in eq. (\ref{eq:expanT}).  Many DG
implementations use an orthonormal basis on each cell, but such a basis does
not exist for hexagonal cells.  Two options are discussed in detail in this
section.

To begin with,
the basis functions may be selected independent of the cell-index $k$ as the
set of polynomials
\begin{equation}
  \label{eq:basis_quad_raw}
  \beta_{z,j} = \mathcal{P}_j = x^{(j-1)\bmod (p+1)} 
      y^{\lfloor (j-1)/(p+1) \rfloor}\,.
\end{equation}
where $p$ is the polynomial order and $1 \le j \le (p+1)^2$.
For $p=2$ (bi-quadratic), $N = 9$ in (\ref{eq:expanT}), and we have
\begin{equation}
  \label{eq:biquadradic}
  \{\mathcal{P}_j\}_{j=1}^N = \{1,x,x^2,y,xy,x^2y,y^2,xy^2,x^2y^2\}\,.
\end{equation}
This choice of basis is not ideal, for a number of reasons, all of which we
won't go into here.

One issue is that a straightforward way to meet Requirement
(\ref{req:cell_avg}) is for
\begin{equation}
  c^{n}_{z,1} = \bar{q}^n_z \equiv \frac{1}{V_z} \int_{\Omega_z} q(\svec{x},t^n) d\Omega\,,
\end{equation}
where $\bar{q}^n_z$ is the cell average.  The basis (\ref{eq:basis_quad_raw})
can be adjusted to satisfy this property with
\begin{equation}
  \label{eq:basis_quad}
  \beta_{z,j} = \mathcal{P}_j - b_{z,j}\,,
\end{equation}
where for each cell, the constants $\{b_{z,j}\}_{j=1}^N$ are given by
\begin{equation}
  \label{eq:barray}
  b_{z,j} = 
  \begin{cases}
    0 & \text{for $j=1$},\\
    \int_{\Omega_z} \mathcal{P}_j d\Omega & \text{otherwise.}
  \end{cases}
\end{equation}
This can viewed as the first step of a Gram-Schmidt orthogonalization.

Indeed, a Gram-Schmidt orthogonalization of $\mathcal{P}_j$ is given by the
recursion relation
\begin{equation}
  \label{eq:gs}
  \beta_{z,j} = \mathcal{P}_j - \sum_{k=1}^{j-1} g_{z,j,k} \beta_{z,k} 
\end{equation}
where
\begin{equation}
  g_{z,j,k} = \int_{\Omega_z} \mathcal{P}_j \left. \beta_{z,k} d\Omega \middle/ 
                 \int_{\Omega_z} \beta_{z,k}^2 d\Omega \right. \,.
\end{equation}
We claim that we can then write (\ref{eq:gs}) as
\begin{equation}
  \label{eq:gs_mat}
  \beta_{z,j} = \sum_{k=1}^{j} L_{z,j,k} \mathcal{P}_k
\end{equation}
where for each cell-$z$, $L_{z,j,k}$ is an $N\times N$, lower-triangular
matrix.  The precise form of $L_{z,j,k}$ won't be given here, but we claim it
may be computed so that
\begin{equation}
  M_{z,i,j} = V_z \delta_{i,j}\,,
\end{equation}
which might appear to greatly decrease the computation cost for
(\ref{eq:scdg_massMatrix}).

However, consider the costs of using the non-orthogonal basis
(\ref{eq:basis_quad}) versus the orthonormal basis (\ref{eq:gs_mat}).  Assume
the update (\ref{eq:scdg}) is performed inside a cell loop, so that
Requirement (\ref{req:thread}) is satisfied.  Then the costs may be summarized
as follows:
\begin{itemize}
\item {\bf Use of non-orthogonal basis (\ref{eq:basis_quad})}: For each
  cell-$z$, the following must be computed:
  \begin{itemize}
  \item The matrix $M_{z,j,k}$, which has $N^2$ elements.
  \item The inverse $M^{-1}_{z,j,k}$.  This computation costs approximately
    $O(N^3)$ FLOPS.  Note that $M^{-1}_{z,j,k}$ may be pre-computed and
    stored, outside of the cell-update loop, and then $M_{z,j,k}$ discarded.
  \item The array $b_{z,j}$, which contains $N$ elements.
  \end{itemize}
\item {\bf Use of orthonormal basis (\ref{eq:gs_mat})}:
  \begin{itemize}
  \item For each cell-$z$, the matrix $L_{z,j,k}$ must be
  computed, which has $N^2 / 2$ nonzero elements.
  \item For each quadrature point used for the integrals in
    (\ref{eq:scdg_massMatrix}), the matrix-vector product (\ref{eq:gs_mat})
    requires $O(N^2)$ extra FLOPS, compared with using the non-orthogonal
    basis.  There is an absolute minimum of $N^2$ quadrature points (and
    typically many more), so that at least $O(N^4)$ extra FLOPS are required.
  \item If $L_{z,j,k}$ is computed within the cell-update loop, rather than
    pre-computed, then it must also be computed for all of cell-$z$'s
    neighbors that are covered by a Lagrangian pre-image.  In contrast,
    $M^{-1}_{z,j,k}$ needs to be computed only for cell being updated.  Note
    that such ``on-the-fly'' computation may be used for certain computer
    architectures that have low-memory bandwidth.
  \end{itemize}
\end{itemize}
Consequently, at this time, we will use the non-orthogonal basis
(\ref{eq:basis_quad}).

%%%%%%%%%%%%%%%%%%%%%%%%%%%%%%%%%%%%%%%%%%%%%%%%%%%%%%%%%%%%%%%%%%%%%%%%%%%%%%%%%%%%%%%%
\section{Quadrature}
\label{sec:quadrature}
%%%%%%%%%%%%%%%%%%%%%%%%%%%%%%%%%%%%%%%%%%%%%%%%%%%%%%%%%%%%%%%%%%%%%%%%%%%%%%%%%%%%%%%%

For the mesh cells used in MPAS, all of the integrals in (\ref{eq:scdg}) are
computed with quadrature.  Generally, the integration regions are
triangulated, and each triangle is integrated with known integration weights
and points.  For the region $\Omega_z$, the triangulation is formed by
computing the centroid and then connecting the centroid with each of the
cell's vertices.  For $N^\mathrm{vertices}$ vertices, 
this forms $N^{\mathrm{tri}}(\Omega_z) = N^\mathrm{vertices}-1$ triangles.

On each triangle, we use the same quadrature rule with $N^{\mathrm{gauss}}$
points, so that for example the second integral in (\ref{eq:scdg}), we have
\begin{equation}
  \label{eq:gauss_tri}
  \int\limits_{\Omega_z} (\phi_{z,i} q)^{n} \, d\Omega \approx 
     \sum_{it=1}^{N^{\mathrm{tri}}(\Omega_z)} A_{it}
     \sum_{g=1}^{N^{\mathrm{gauss}}}
       w_{g} (\phi^n_{z,i} q^n)|_{\svec{x}=\svec{x}_{g}}\,,
\end{equation}
where $A_{it}$ is the triangle's area and $\svec{x}_{g}$ is the Gauss-point
location, with corresponding integration weight $w_{g}$.  We assume here
that the weights are normalized such that
\begin{equation}
  \sum_{g=1}^{N^\mathrm{gauss}} w_{g} = 1\,,
\end{equation}
and that the $A_{it}$ satisfy
\begin{equation}
  V_z = \sum_{it=1}^{N^{\mathrm{tri}}(\Omega_z)} A_{it}\,.
\end{equation}
As part of a pre-processing step, we can accumulate all the Gauss points and
weights into a single array for cell-$z$, so that (\ref{eq:gauss_tri}) takes
the form
\begin{equation}
  \label{eq:gauss_vor}
  \int\limits_{\Omega_z} (\phi_{z,i} q)^{n} \, d\Omega \approx 
     V_z \sum_{g=1}^{N^{\mathrm{gauss}}_z}
       w_{z,g} (\phi^n_{z,i} q^n)|_{\svec{x}=\svec{x}_{z,g}}\,,
\end{equation}
where
\begin{align}
  N^{\mathrm{gauss}}_z &= N^{\mathrm{gauss}} N^{\mathrm{tri}}(\Omega_z)\,,\\
  w_{z,g'} &= w_g A_{it} / V_z\,,
\end{align}
with $g'=g+(z-1)N^{\mathrm{gauss}}$.

We strive to perform as much computation as possible, outside of the loop
over tracers.  To this end, we use (\ref{eq:expanT}) and write
(\ref{eq:gauss_vor}) as
\begin{equation}
  \label{eq:gauss_vor2}
  \int\limits_{\Omega_z} (\phi_{z,i} q)^{n} \, d\Omega \approx 
     V_z \sum_{j=1}^N c^n_{z,j} \sum_{g=1}^{N^{\mathrm{gauss}}_z}
       w_{z,g} \phi^n_{z,i}(\svec{x}_{z,g})
       \beta_{z,j}(\svec{x}_{z,g})\,.
\end{equation}
As an initialization step, before the time-step loop, we can compute
\begin{equation}
  \label{eq:bfactor}
  B_{z,j,g} = V_z w_{z,g} \beta_{z,j}(\svec{x}_{z,g})\,,
\end{equation}
which requires storing a $N \times N^{\mathrm{gauss}}_z$ matrix for each
cell-$z$.  Within the time-step loop, we can evaluate $\phi$ for all tracers,
and compute
\begin{equation}
  \label{eq:Kmat}
  K^n_{z,i,j} = \sum_{g=1}^{N^{\mathrm{gauss}}_z} B_{z,j,g} \phi^n_{z,i}(\svec{x}_{z,g})\,,
\end{equation}
so that (\ref{eq:gauss_vor2}) reduces to
\begin{equation}
  \label{eq:gauss_vor3}
  \int\limits_{\Omega_z} (\phi_{z,i} q)^{n} \, d\Omega \approx 
     \sum_{j=1}^N K^n_{z,i,j} c^n_{z,j} \,;
\end{equation}
that is, a matrix-vector product.  Note that if the overall update is within a cell loop, then $K^n_{z,i,j}$ can
be computed within this loop, and there is no need to permanently store it for
each cell-$z$.

The integration over each Lagrangian pre-image may be computed in a similar
fashion.  Because the velocity is time dependent, the pre-image region is also
time dependent, so all of the setup must be within the time-step loop.
However, much of the work may be completed before the inner tracer loop.  To
begin with, the integration over the pre-image may be written as:
\begin{equation}
  \int\limits_{\Delta\Omega_{f}}
  (\phi_{z,i} q)^n \, d\Omega\ = \sum_{z'}
   \int\limits_{\omega_{z'}} (\phi_{z,i} q)^n \, d\Omega\
\end{equation}
where $z'$ sums over the halo of neighbors of face-$f$ and $\omega_{z'} =
\Omega_{z'} \cap \Delta\Omega_{f}$.  Note here that $\phi$ is with respect to the
basis in cell-$z$, rather than cell-$z'$.  As discussed in \S\ref{sec:remap},
$\Delta\Omega_{f}$ may itself consist of two separate triangles (``Case 2''),
but this case is an obvious extension of the case when $\Delta\Omega_{f}$ is a
single region (``Case 1'').  Next, we write the integral above in a similar
form as (\ref{eq:gauss_vor3}):
\begin{equation}
   \label{eq:subflux}
   \int\limits_{\omega_{z'}} (\phi_{z,i} q)^n \, d\Omega\ \approx
    \sum_{j=1}^N F^n_{z',i,j} c^n_{z',j}
\end{equation}
For each cell-neighbor $z'$, the $N\times N$ matrix $F^n_{z',i,j}$ is computed
by subdividing $\omega_{z'}$ into triangles and using quadrature on each of
these triangles:
\begin{equation}
  \label{eq:Fmat_tmp}
  F^n_{z',i,j} = \sum_{it=1}^{N^{\mathrm{tri}}(\omega_{z'})} A_{it} \sum_{g=1}^{N^\mathrm{gauss}} w_g
  \phi_{z,i}^n(\svec{x}_g) \beta_{z',j}(\svec{x}_g)\,.
\end{equation}

However, eqs. (\ref{eq:subflux}-\ref{eq:Fmat_tmp}) get only the contribution of $z'$ for a single face-$f$.
Referring to (\ref{eq:scdg}), we can sum the contributions
of all faces as
\begin{equation}
\sum_{f\in\mathcal{F}(z)}
     \frac{m_f^n}{V_f} \int\limits_{\Delta\Omega_{f}}
  (\phi_{z,i} q)^n \, d\Omega \approx
     \sum_{z'} \sum_{j=1}^N F^n_{z',i,j} c^n_{z',j}
\end{equation}
where now $z'$ sums over union of all the face's neighbors and
\begin{equation}
  \label{eq:Fmat}
  F^n_{z',i,j} = \sum_{f\in\mathcal{F}(z)}
     \frac{m_f^n}{V_f} \sum_{it=1}^{N^{\mathrm{tri}}(\omega_{z'})} A_{it} \sum_{g=1}^{N^\mathrm{gauss}} w_g
  \phi_{z,i}^n(\svec{x}_g) \beta_{z',j}(\svec{x}_g)\,.
\end{equation}

%%%%%%%%%%%%%%%%%%%%%%%%%%%%%%%%%%%%%%%%%%%%%%%%%%%%%%%%%%%%%%%%%%%%%%%%%%%%%%%%%%%%%%%%
\section{Initial Condition}
\label{sec:IC}
%%%%%%%%%%%%%%%%%%%%%%%%%%%%%%%%%%%%%%%%%%%%%%%%%%%%%%%%%%%%%%%%%%%%%%%%%%%%%%%%%%%%%%%%

Given a tracer initial condition $q^0(\svec{x})$, its initial expansion
coefficients are computed with the matrix-vector product
\begin{equation}
  \label{eq:cic}
  c^0_{z,i} = M^{-1}_{i,j} v^0_{z,j}\,,
\end{equation}
where
\begin{equation}
  \label{eq:icproject}
  v^0_{z,j} = \int\limits_{\Omega_z} q^0\beta_{z,j} \, d\Omega\,.
\end{equation}
This integral is computed with the same quadrature rule as in
(\ref{eq:gauss_vor}). 

%%%%%%%%%%%%%%%%%%%%%%%%%%%%%%%%%%%%%%%%%%%%%%%%%%%%%%%%%%%%%%%%%%%%%%%%%%%%%%%%%%%%%%%%%%%%%%%%%%%%%%%%%%%%%%%%
\section{Pseudocode Formulation}
\label{chap:pseudocode}
%%%%%%%%%%%%%%%%%%%%%%%%%%%%%%%%%%%%%%%%%%%%%%%%%%%%%%%%%%%%%%%%%%%%%%%%%%%%%%%%%%%%%%%%%%%%%%%%%%%%%%%%%%%%%%%%

The overall update is broken into two main loops over the cells:
\begin{enumerate}
\item \emph{Initial setup}: Initial computations that are performed before
  entering the time-advance loop.  These computations are given in Algorithm
  \ref{alg:init_setup}.
\item \emph{Update loop}: This loop is summarized in Algorithm
  \ref{alg:update}, with the details given in Algorithms
  \ref{alg:update_shared} and \ref{alg:update_tracers}.  This algorithm is
  within the time-step loop.  Note that there are many computations that are
  common to all tracers (Algorithm \ref{alg:update_shared}), performed before
  a loop over all tracers (Algorithm \ref{alg:update_tracers}).
\end{enumerate}
Note that if one does not want to store quantities performed in the initial
setup, such as the matrix $M^{-1}_{z,i,j}$ for all cells, then the two cell
loops of Algorithms \ref{alg:init_setup}-\ref{alg:update} may be combined.

\begin{algorithm}[ht]
  \caption{Initial setup. We may place this loop outside of the time-step loop.}
  \label{alg:init_setup}
  \begin{algorithmic}[1]
    \State Load Gauss points and weights for triangle
    \For{$z=1,N^\mathrm{cells}$} \Comment{loop over all owned cells}
      \State Compute centroid of cell-$z$
      \State Compute triangulation of each cell-$z$
      \State $N^\mathrm{tri}(\Omega_z) = N_z^\mathrm{vertices} - 1$
      \For {$it=1,N^\mathrm{tri}(\Omega_z)$} \Comment{loop over triangles for this cell} 
        \State Compute area of triangle, $A_{it}$
        \State Normalize triangle Gauss weights by $A_{it} / V_z$
        \State Accumulate new weights and points into $w_{z,g}$ and $\svec{x}_{z,g}$ 
      \EndFor
      \State Compute $M^{-1}_{z,i,j}$ \Comment{using quadrature set above}
      \State Compute $b_{z,j}$ \Comment{using quadrature set above}
      \State Compute $B_{z,j,g}$ \Comment{see eq. (\ref{eq:bfactor})}
      \State Compute $v^0_{z,j}$ \Comment{using quadrature set above}
      \State Initialize $c^0_{z,i}$ \Comment{see eq. (\ref{eq:cic})}
    \EndFor
    \State \Return $M^{-1}_{z,i,j}$, $b_{z,j}$, $B_{z,j,g}$, $c^0_{z,i}$
  \end{algorithmic}
\end{algorithm}

\begin{algorithm}[ht]
  \caption{Main update loop.}
  \label{alg:update}
  \begin{algorithmic}[1]
    \State Fill off-processor cell neighbors with data
    \For {$z=1,N^\mathrm{cells}$} \Comment{loop over all owned cells}
      \State Computations shared by all tracers, Algorithm \ref{alg:update_shared}
      \State Update tracers, Algorithm \ref{alg:update_tracers}
    \EndFor \Comment{cell loop}
  \end{algorithmic}
\end{algorithm}

\begin{algorithm}[ht]
  \caption{Computations shared by all tracers, within main update loop (see
    Algorithm \ref{alg:update}). Even though $K_{z,i,j}$ has a cell-index $z$,
    it does not need to be stored for all cells.  For $F_{z',i,j}$, $z'$ spans
    the cells that intersect the Lagrangian pre-image of any face-$f$.}
  \label{alg:update_shared}
  \begin{algorithmic}[1]
      \State $K^n_{z,i,j} = 0$ \Comment{initialize array}
      \For {$i=1,N$} \Comment{loop over number of basis}
        \For {$g=1,N^{\mathrm{gauss}}_z$} \Comment{loop over gauss points for this cell}
          \State Compute $\phi^n_{z,i}(\svec{x}_{z,g})$
          \For {$j=1,N$}
             \State $K^n_{z,i,j} = K^n_{z,i,j} + B_{z,j,q} \phi^n_{z,i}(\svec{x}_{z,g})$ \Comment{see eq. (\ref{eq:Kmat})}
          \EndFor
        \EndFor
      \EndFor
      \State $F^n_{z',i,j} = 0$ \Comment{initialize array}
      \For {$f=1,N^{\mathrm{faces}}_z$} \Comment{loop over faces for this cell} 
        \State Compute Lagrangian pre-image of face-$f$... 
        \State ... to form subregions $\omega_p$, $p=1,N^\mathrm{preimages}$ \Comment{see \S\ref{sec:remap}}
        \For {$zf=1,N^{\mathrm{neighbors}}_f$} \Comment{loop over face's cell neighbors}
          \For {$p=1,N^\mathrm{preimages}$} \Comment{loop over preimage regions}
            \State Intersect $\omega_p$ with $\Omega_{zf}$ to form $R$
            \If {$R$ is not an empty region}
              \State Put $zf$ in list of cell-$z$'s neighbors, with linear index $z'$
              \State Split $R$ into triangles
              \State Integrate over triangles and sum contributions...
              \State ... to $F^n_{z',i,j}$ and $V_f$ \Comment{see eq. (\ref{eq:Fmat})}
            \EndIf
          \EndFor \Comment{pre-image loop}
        \EndFor \Comment{cell-neighbor loop}
      \EndFor \Comment{face loop}
  \State \Return $K^n_{z,i,j}$, $F^n_{z',i,j}$, list of cells that $z'$ spans
  \end{algorithmic}
\end{algorithm}

\begin{algorithm}[ht]
  \caption{Update tracers, within main update loop (see
    Algorithm \ref{alg:update})}
  \label{alg:update_tracers}
  \begin{algorithmic}[1]
      \For {$iq=1,N^{\mathrm{tracers}}$} \Comment{loop over number of tracers}
         \State $sumFlux_i = 0$ \Comment{$N$-array to accumulate fluxes}
         \State $sumFlux_i = sumFlux_i + K_{z,i,j} c^n_{z,j}$ \Comment{matrix-vector multiply} 
         \For {$z'=1,N^{\mathrm{neighbors}}_z$} \Comment{loop over cell's neighbors}
           \State $sumFlux_i = sumFlux_i - F_{z',i,j} c^n_{z',j}$ \Comment{matrix-vector multiply}
         \EndFor
         \State $c^{n+1}_{z,iq,i} = M^{-1}_{z,i,j} (sumFlux_j)$ \Comment{matrix-vector multiply} 
      \EndFor \Comment{tracer loop}
  \end{algorithmic}
\end{algorithm}

%%%%%%%%%%%%%%%%%%%%%%%%%%%%%%%%%%%%%%%%%%%%%%%%%%%%%%%%%%%%%%%%%%%%%%%%%%%%%%%%%%%%%%%%%%%%%%%%%%%%%%%%%%%%%%%%
\chapter{Formulation for Vertical Advection}
%%%%%%%%%%%%%%%%%%%%%%%%%%%%%%%%%%%%%%%%%%%%%%%%%%%%%%%%%%%%%%%%%%%%%%%%%%%%%%%%%%%%%%%%%%%%%%%%%%%%%%%%%%%%%%%%

To be written.

%%%%%%%%%%%%%%%%%%%%%%%%%%%%%%%%%%%%%%%%%%%%%%%%%%%%%%%%%%%%%%%%%%%%%%%%%%%%%%%%%%%%%%%%%%%%%%%%%%%%%%%%%%%%%%%%
\chapter{Design and Implementation}
%%%%%%%%%%%%%%%%%%%%%%%%%%%%%%%%%%%%%%%%%%%%%%%%%%%%%%%%%%%%%%%%%%%%%%%%%%%%%%%%%%%%%%%%%%%%%%%%%%%%%%%%%%%%%%%%

%%%%%%%%%%%%%%%%%%%%%%%%%%
\section{Namelist Options}
%%%%%%%%%%%%%%%%%%%%%%%%%%

The namelist options are as follows:
\begin{itemize}
\item {\tt config\_tracer\_adv\_meth}\\ This is a flag corresponding to either
  the CDG or FVM methods.  The options for FVM are as before.   The options
  for CDG are given below.\\
  Available options: CDG or FVM
\item {\tt config\_cdg\_ord}\\
  Polynomial order of basis, for all cells.\\
  Available options: Any non-negative integer, although effectively limited by
  {\tt config\_cdg\_gauss} 
\item {\tt config\_cdg\_gauss}\\
  Max polynomial order integrated exactly by
  quadrature on triangles.  If -1, estimate from {\tt config\_cdg\_ord}. Note
  that general quadrature on triangles, for any order, does not exist.\\
  Available options: 0 to 25
\item {\tt config\_cdg\_limit}\\
  Selects how tracer bounds are computed by the limiter.\\
  {\tt none} means don't limit at all.\\
  {\tt local} means use local cell averages to compute the bounds.\\
  {\tt global} means use pre-computed, global minimum and maximums.\\
  Available options: none, bounds, global
\end{itemize}

%%%%%%%%%%%%%%%%%%%%%%%%%%%%%%%%%%%%%%%%%%%%%%%%%%%%%%%%%%%%%%%%%%%%%%%%%%%%%%%%%%%%%%%%%%%%%%%%%%%%%%%%%%%%%%%%
\section{Implementation Issues}
%%%%%%%%%%%%%%%%%%%%%%%%%%%%%%%%%%%%%%%%%%%%%%%%%%%%%%%%%%%%%%%%%%%%%%%%%%%%%%%%%%%%%%%%%%%%%%%%%%%%%%%%%%%%%%%%

\begin{itemize}
\item Use {\tt advCellsForEdge} index array as the cell halo to search for
  each edge's Lagrangian pre-image.
\item \S\ref{sec:characteristic} requires the edge velocities to be
  interpolated to any spatial coordinate, at either $t=t^n$ or $t=t^{n+1}$.
  This interpolation call should be its own routine, so various methods may be
  explored.
\end{itemize}

\end{document}

