\chapter*{Foreword}
\label{chap:foreword}
The Model for Prediction Across Scales-Ocean (MPAS-Ocean) is an unstructured-mesh ocean model capable of using enhanced horizontal resolution in selected regions of the ocean domain.  Model domains may be spherical with bottom topography to simulate the earth's oceans, or on Cartesian domains for idealized experiments.  The global meshes, created using Spherical Centroidal Voronoi Tesselations \citep{Ringler_ea08od,Ringler_ea11mwr} consist of gridcells that vary smoothly from low to high resolution regions. Numerical algorithms specifically designed for these grids guarantee that mass, tracers, potential vorticity (in isopycnal mode) and energy are conserved \citep{Thurburn_ea09jcp,Ringler_ea10jcp}.  MPAS-Ocean high-resolution and variable-resolution global simulations, as well as descriptions of mesh generation, model capabilities, and algorithms, are presented in \citet{Ringler_ea13om}.

MPAS-Ocean is one component within the MPAS framework of climate models that is developed in cooperation between Los Alamos National Laboratory (LANL) and the National Center for Atmospheric Research (NCAR).  Functionality that is required by all cores, such as i/o, time management, block decomposition, etc, is developed collaboratively, and this code is shared across cores within the same repository.  Each core then solves its own differential equations and physical parameterizations within this framework.  This user's guide reflects the spirit of this collaborative process, where Part I, ``The MPAS Framework'', applies to all cores, and the remaining parts apply to MPAS-Ocean.

Here we would normally describe the new features of this version.  Instead, for this initial release we describe those features that users will want but are not included in this public release. Most importantly, {\bf this release does not include the ability to create or modify ocean grids or ocean initial conditions}. We appreciate that such functionality is important and we intend to release this capability when it meets our software standards. In addition, software interfaces to coupled MPAS-O into a coupled system model, such at the Community Earth System Model, is not contained in this release. And lastly, the suite of analysis tools available to process, interpret and visualize model output is limited. In summary, this release is intended help introduce MPAS-O to the scientific community with the expectation of expanding model's capability in future releases.

Information about MPAS-Ocean, including the most recent code, user's guide, and test cases, may be found at \url{http://mpas-dev.github.com}.  This user's guide refers to version \version, with corresponding downloads at \href{http://mpas-dev.github.com/ocean/release_\version/release_\version.html}{release \version}. \\

\vspace{8pt}
\noindent
{\bf Contributors to this guide:}\\
Doug Jacobsen, Mark Petersen, Todd Ringler\\
{\bf Additional contributors to MPAS Framework sections:}\\
Michael Duda

\vspace{8pt}
\noindent
{\it Funding for the development of MPAS-Ocean was provided by the United States Department of Energy, Office of Science.}




