\chapter*{Foreword}
\label{chap:foreword}

The Model for Prediction Across Scales-Ocean (MPAS-Ocean) is an unstructured-mesh ocean model capable of using enhanced horizontal resolution in selected regions of the ocean domain.  This allows researchers to perform high-resolution regional simulations at a lower computational cost, while providing realistic currents and water-mass properties from the low-resolution regions. Model domains may be spherical with bottom topography to simulate the earth's oceans, or on Cartesian domains for idealized experiments.  The global meshes, created using Spherical Centroidal Voronoi Tesselations \citep{Ringler_ea08od,Ringler_ea11mwr} consist of gridcells that vary smoothly from low to high resolution regions. Numerical algorithms specifically designed for these grids guarantee that mass, tracers, potential vorticity (in isopycnal mode) and energy are conserved \citep{Thurburn_ea09jcp,Ringler_ea10jcp}.  MPAS-Ocean high-resolution and variable-resolution global simulations, as well as descriptions of mesh generation, model capabilities, and algorithms, are presented in \citet{Ringler_ea13om}.

MPAS-Ocean is one component within the MPAS framework of climate models that is developed in cooperation between Los Alamos National Laboratory (LANL) and the National Center for Atmospheric Research (NCAR).  Functionality that is required by all cores, such as i/o, time management, block decomposition, etc, is developed collaboratively, and this code is shared across cores within the same repository.  Each core then solves its own differential equations and physical parameterizations within this framework.  This user's guide reflects the spirit of this collaborative process, where Part I, ``The MPAS Framework'', applies to all cores, and the remaining parts apply to MPAS-Ocean.

Here we would normally describe the new features of this version.  For the initial release, we will simply review the major features of MPAS-Ocean.  We employ a finite-volume discretization of the Boussinesq equations using a C-grid staggering in the horizontal.  
The vertical coordinate is Arbitrary Lagrangian-Eulerian (ALE), with a choice of z-level, z-star, and idealized isopycnal.  The time-stepping method may be split-explicit or fourth-order Runge-Kutta.  Many choices are available for advection and turbulence closures.  Our current high-resolution global simulations use a quasi 3$^{rd}$-order monotone advection scheme for tracers \citep{Skamarock:2011tc} and the Leith, enstrophy-cascade turbulence closure \citep{Leith:1996wu}.  


% Include this in future versions of the forward:
%A history of past releases is as follows:
%\begin{tabular}{lll} 
%\hline\hline version & date & description  \\
%\hline 
%1.0.0 & June 3, 2013 & Initial public release \\
%\hline 
%\end{tabular} 

Information about MPAS-Ocean, including the most recent code, user's guide, and test cases, may be found at \url{http://mpas-dev.github.com}.  This user's guide refers to version \version, with corresponding downloads at \href{http://mpas-dev.github.com/ocean/release_\version/release_\version.html}{release \version}. \\

\vspace{8pt}
\noindent
{\bf Contributors to this guide:}\\
Doug Jacobsen, Mark Petersen, Todd Ringler\\
{\bf Additional contributors to MPAS Framework sections:}\\
Michael Duda

\vspace{8pt}
\noindent
{\it Funding for the development of MPAS-Ocean was provided by the United States Department of Energy, Office of Science.}




