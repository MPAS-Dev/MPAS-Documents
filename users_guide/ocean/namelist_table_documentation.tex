\chapter[Namelist options]{\hyperref[chap:namelist_sections]{Namelist options}}
\label{chap:namelist_tables}
Embedded links point to more detailed namelist information in the appendix.
\section[run\_modes]{\hyperref[sec:nm_sec_run_modes]{run\_modes}}
\label{sec:nm_tab_run_modes}
MPAS-Ocean may be run in forward or analysis mode.  Forward mode is the default, and invokes the time stepping routine to run through the specified time duration.  Analysis mode simply reads in files upon initialization, runs all enabled analysis members, writes output, and shuts down.

\vspace{0.5in}
{\small
\begin{center}
\begin{longtable}{| p{2.0in} || p{4.0in} |}
    \hline
    {\bf Name} & {\bf Description} \endfirsthead
    \hline 
    {\bf Name} & {\bf Description} (Continued) \endhead
    \hline
    \hline
    \hyperref[subsec:nm_sec_config_ocean_run_mode]{config\_ocean\_run\_mode} & Determines which run mode will be used for the ocean model. \\
    \hline
\end{longtable}
\end{center}
}
\section[time\_management]{\hyperref[sec:nm_sec_time_management]{time\_management}}
\label{sec:nm_tab_time_management}
General time management is handled by the time\_management namelist record.
Included options handle time-related parts of MPAS, such as the calendar and if the simulation is a restart or not.

Users should use this record to specify the beginning time of the simulation,
and either the duration or the end of the simulation. Only the end or the
duration need to be specified as the other is derived within MPAS from the
beginning time and other specified one.

{\bf TBA: If both duration and stop are specified, then what happens?)}

\vspace{0.5in}
{\small
\begin{center}
\begin{longtable}{| p{2.0in} || p{4.0in} |}
    \hline
    {\bf Name} & {\bf Description} \endfirsthead
    \hline 
    {\bf Name} & {\bf Description} (Continued) \endhead
    \hline
    \hline
    \hyperref[subsec:nm_sec_config_do_restart]{config\_do\_restart} & Determines if the initial conditions should be read from a restart file, or an input file. \\
    \hline
    \hyperref[subsec:nm_sec_config_restart_timestamp_name]{config\_restart\_timestamp\_name} & Path to the filename for restart timestamps to be read and written from. \\
    \hline
    \hyperref[subsec:nm_sec_config_start_time]{config\_start\_time} & Timestamp describing the initial time of the simulation. If it is set to 'file', the initial time is read from restart\_timestamp. \\
    \hline
    \hyperref[subsec:nm_sec_config_stop_time]{config\_stop\_time} & Timestamp descriping the final time of the simulation. If it is set to 'none' the final time is determined from config\_start\_time and config\_run\_duration. \\
    \hline
    \hyperref[subsec:nm_sec_config_run_duration]{config\_run\_duration} & Timestamp describing the length of the simulation. If it is set to 'none' the duration is determined from config\_start\_time and config\_stop\_time. config\_run\_duration overrides inconsistent values of config\_stop\_time. \\
    \hline
    \hyperref[subsec:nm_sec_config_calendar_type]{config\_calendar\_type} & Selection of the type of calendar that should be used in the simulation. \\
    \hline
\end{longtable}
\end{center}
}
\section[io]{\hyperref[sec:nm_sec_io]{io}}
\label{sec:nm_tab_io}
The io namelist record provides options for modifications to the I/O system of
MPAS. These include frequency, file name, and parallelization options.

\vspace{0.5in}
{\small
\begin{center}
\begin{longtable}{| p{2.0in} || p{4.0in} |}
    \hline
    {\bf Name} & {\bf Description} \endfirsthead
    \hline 
    {\bf Name} & {\bf Description} (Continued) \endhead
    \hline
    \hline
    \hyperref[subsec:nm_sec_config_write_output_on_startup]{config\_write\_output\_on\_\-startup} & Logical flag determining if an output file should be written prior to the first time step. \\
    \hline
    \hyperref[subsec:nm_sec_config_pio_num_iotasks]{config\_pio\_num\_iotasks} & Integer specifying how many IO tasks should be used within the PIO library. A value of 0 causes all MPI tasks to also be IO tasks. IO tasks are requried to write contiguous blocks of data to a file. \\
    \hline
    \hyperref[subsec:nm_sec_config_pio_stride]{config\_pio\_stride} & Integer specifying the stride of each IO task. \\
    \hline
\end{longtable}
\end{center}
}
\section[decomposition]{\hyperref[sec:nm_sec_decomposition]{decomposition}}
\label{sec:nm_tab_decomposition}
Namelist parameters for the \verb+decomposition+ namelist group.

\vspace{0.5in}
{\small
\begin{center}
\begin{longtable}{| p{2.0in} || p{4.0in} |}
    \hline
    {\bf Name} & {\bf Description} \endfirsthead
    \hline 
    {\bf Name} & {\bf Description} (Continued) \endhead
    \hline
    \hline
    \hyperref[subsec:nm_sec_config_num_halos]{config\_num\_halos} & Determines the number of halo cells extending from a blocks owned cells (Called the 0-Halo). The default of 3 is the minimum that can be used with monotonic advection. \\
    \hline
    \hyperref[subsec:nm_sec_config_block_decomp_file_prefix]{config\_block\_decomp\_file\_\-prefix} & Defines the prefix for the block decomposition file. Can include a path. The number of blocks is appended to the end of the prefix at run-time. \\
    \hline
    \hyperref[subsec:nm_sec_config_number_of_blocks]{config\_number\_of\_blocks} & Determines the number of blocks a simulation should be run with. If it is set to 0, the number of blocks is the same as the number of MPI tasks at run-time. \\
    \hline
    \hyperref[subsec:nm_sec_config_explicit_proc_decomp]{config\_explicit\_proc\_decomp} & Determines if an explicit processor decomposition should be used. This is only useful if multiple blocks per processor are used. \\
    \hline
    \hyperref[subsec:nm_sec_config_proc_decomp_file_prefix]{config\_proc\_decomp\_file\_prefix} & Defines the prefix for the processor decomposition file. This file is only read if config\_explicit\_proc\_decomp is .true. The number of processors is appended to the end of the prefix at run-time. \\
    \hline
\end{longtable}
\end{center}
}
\section[init\_setup]{\hyperref[sec:nm_sec_init_setup]{init\_setup}}
\label{sec:nm_tab_init_setup}
\vspace{0.5in}
{\small
\begin{center}
\begin{longtable}{| p{2.0in} || p{4.0in} |}
    \hline
    {\bf Name} & {\bf Description} \endfirsthead
    \hline 
    {\bf Name} & {\bf Description} (Continued) \endhead
    \hline
    \hline
    \hyperref[subsec:nm_sec_config_vert_levels]{config\_vert\_levels} & Number of vertical levels to create within the configuration. \\
    \hline
    \hyperref[subsec:nm_sec_config_init_configuration]{config\_init\_configuration} & Name of configuration to create. \\
    \hline
    \hyperref[subsec:nm_sec_config_expand_sphere]{config\_expand\_sphere} & Logical flag that controls if a spherical mesh is expanded to an earth sized sphere or not. \\
    \hline
    \hyperref[subsec:nm_sec_config_realistic_coriolis_parameter]{config\_realistic\_coriolis\_\-parameter} & Logical flag that controls if a spherical mesh will get realistic coriolis parameters or not. \\
    \hline
    \hyperref[subsec:nm_sec_config_write_cull_cell_mask]{config\_write\_cull\_cell\_mask} & Logicial flag that controls if the cullCell field is written to output. \\
    \hline
    \hyperref[subsec:nm_sec_config_vertical_grid]{config\_vertical\_grid} & Name of vertical grid to use in configuration generation \\
    \hline
\end{longtable}
\end{center}
}
\section[CVTgenerator]{\hyperref[sec:nm_sec_CVTgenerator]{CVTgenerator}}
\label{sec:nm_tab_CVTgenerator}
\vspace{0.5in}
{\small
\begin{center}
\begin{longtable}{| p{2.0in} || p{4.0in} |}
    \hline
    {\bf Name} & {\bf Description} \endfirsthead
    \hline 
    {\bf Name} & {\bf Description} (Continued) \endhead
    \hline
    \hline
    \hyperref[subsec:nm_sec_config_1dCVTgenerator_stretch1]{config\_1dCVTgenerator\_\-stretch1} & Parameter for the 1D CVT vertical grid generator. \\
    \hline
    \hyperref[subsec:nm_sec_config_1dCVTgenerator_stretch2]{config\_1dCVTgenerator\_\-stretch2} & Parameter for the 1D CVT vertical grid generator. \\
    \hline
    \hyperref[subsec:nm_sec_config_1dCVTgenerator_dzSeed]{config\_1dCVTgenerator\_dzSeed} & Seed thickness of the first layer for the 1D CVT vertical grid generator. \\
    \hline
\end{longtable}
\end{center}
}
\section[init\_ssh\_and\_landIcePressure]{\hyperref[sec:nm_sec_init_ssh_and_landIcePressure]{init\_ssh\_and\_landIcePressure}}
\label{sec:nm_tab_init_ssh_and_landIcePressure}
\vspace{0.5in}
{\small
\begin{center}
\begin{longtable}{| p{2.0in} || p{4.0in} |}
    \hline
    {\bf Name} & {\bf Description} \endfirsthead
    \hline 
    {\bf Name} & {\bf Description} (Continued) \endhead
    \hline
    \hline
    \hyperref[subsec:nm_sec_config_iterative_init_variable]{config\_iterative\_init\_variable} & Which variable, ssh or landIcePressure, is computed from the other.  If landIcePressure is to be computed, 'landIcePressure\_from\_top\_density' indicates that the top density (rather than the mean density displaced by land ice) should be used to compute the landIcePressure. \\
    \hline
\end{longtable}
\end{center}
}
\section[time\_integration]{\hyperref[sec:nm_sec_time_integration]{time\_integration}}
\label{sec:nm_tab_time_integration}
The time integration namelist controls parameters that pertain to all time-stepping methods.  Options that are specific to a particular time-stepping method are contained in a separate namelist for that method, below.

\vspace{0.5in}
{\small
\begin{center}
\begin{longtable}{| p{2.0in} || p{4.0in} |}
    \hline
    {\bf Name} & {\bf Description} \endfirsthead
    \hline 
    {\bf Name} & {\bf Description} (Continued) \endhead
    \hline
    \hline
    \hyperref[subsec:nm_sec_config_dt]{config\_dt} & Length of model time-step. \\
    \hline
    \hyperref[subsec:nm_sec_config_time_integrator]{config\_time\_integrator} & Time integration method. \\
    \hline
\end{longtable}
\end{center}
}
\section[ALE\_vertical\_grid]{\hyperref[sec:nm_sec_ALE_vertical_grid]{ALE\_vertical\_grid}}
\label{sec:nm_tab_ALE_vertical_grid}
The MPAS-Ocean vertical grid is structured, and uses the arbitrary Lagrangian-Eulerian (ALE) method \citet{Petersen_ea14om}.   ALE provides a great deal of freedom in the choice of vertical coordinate: when fully Eulerian, MPAS-Ocean is a z-level model with fixed thicknesses; when fully Lagrangian, there is no vertical transport between layers, and layers expand and contract like an isopycnal ocean model.  In between are many additional options, such as z-star where layers expand in proportion to the sea surface height, and sigma, where coordinates are terrain-following.

MPAS-Ocean employs the continuity equation,
\begin{eqnarray}
\label{ocean:\mode_continuity thickness}
\frac{\partial h_{k}}{\partial t} + D_k + w^t_k - w^t_{k+1} =0
\end{eqnarray}
for the ALE formulation, where variables are defined in Table \ref{oceanTable:\mode_ALE_variables}.  The ALE algorithm is as follows:
\begin{enumerate}
\item ALE step: Compute desired thickness for the new time,
\begin{eqnarray}
\label{ocean:\mode_desired thickness}
h_k^{ALE} = h_k^{rest} + h_k^{SSH} + h_k^{hf} + h_k^{min}
\end{eqnarray}
\item ALE step: Solve for vertical transport $w^t$ from (\ref{ocean:\mode_continuity thickness}),
\begin{eqnarray}
\label{ocean:\mode vert tranport}
w^t_k = w^t_{k+1} - D_k - \frac{h^{ALE}_k - h^n_k}{\Delta t}
\end{eqnarray}
\item Solve for new thickness, $h_{k}^{n+1}$, using the continuity equation (\ref{ocean:\mode_continuity thickness}) within the time integration routine.
\end{enumerate}
The redundancy in steps 2 and 3 are intentional, so that step 2 is isolated within the ALE subroutine, and step 3 is solved in the timestepping subroutine in the identical manner as the tracer equation (\ref{ocean:tracer}).

The desired ALE thickness includes contributions from four terms (\ref{ocean:\mode_desired thickness}):
\begin{enumerate}
\item {\bf Resting thickness}, $h^{rest}$, is the layer thickness when the ocean is at rest, i.e. without SSH or internal perturbations.  For z-type coordinates, the resting thickness is constant in each horizontal layer.
\item {\bf SSH alteration}, $h^{SSH}$, alters layer thicknesses so that they change in proportion to the sea surface height (SSH),
\begin{eqnarray}
\label{ocean:\mode_h ssh}
   h_k^{SSH} =  \zeta \frac{W_k h^{rest}_k}{\sum_{k'=1}^{kmax}W_{k'}h^{rest}_{k'}}
\end{eqnarray}
The weights $W_k$ determine how SSH oscillations are distributed amongst the layers, as described in Table \ref{oceanTable:\mode_vertical coordinates}.
\item {\bf High-frequency thickness}, $h^{hf}$, allows high-frequency thickness oscillations, such as internal gravity waves, to be treated in a Lagrangian manner.  This is the ``z-tilde'' scheme of \citet{Leclair_Madec11om} described in the next section.
\item {\bf Minimum thickness alteration}, $h^{min}$, is the change in thickness required to enforce the minimum thickness in each cell.  When a cell is too thin, $h^{min}$ is positive, while nearby cells in the vertical incur a corresponding negative $h^{min}$ to conserve volume in the column.
\end{enumerate}
Of the four terms, resting thickness is always positive, while the others are small alterations about zero.  Summing a column,
\begin{eqnarray}
\sum_{k=1}^{kmax} h_k^{ALE} &=& \sum_{k=1}^{kmax} h_k^{rest} + \sum_{k=1}^{kmax}h_k^{SSH} + \sum_{k=1}^{kmax}h_k^{hf} + \sum_{k=1}^{kmax}h_k^{min} 
\nonumber \\
&=& H + \zeta + 0 + 0.
\nonumber
\end{eqnarray}
Thus the first two terms are always included so that the column thickness sums to $H+\zeta$, while the second two terms are optional and may be turned on with flags.

In order to assist users in choosing the correct settings, we provide a description of traditional vertical coordinate names in Table \ref{oceanTable:\mode_vertical coordinates}.
The vertical coordinate type also depends upon the set-up of the layerThickness variable in the initial condition file.  For all Z-type vertical coordinates, initial layer thicknesses are horizontally constant.  For sigma coordinates, layers are terrain-following and all layers are employed.  In this case, the user may still choose amongst the flags in SSH may be distributed through the column just like with z-type models in Table \ref{oceanTable:\mode_vertical coordinates}.

In order to run an isopycnal configuration, use config\_vert\_coord\_movement='impermeable\_interfaces' and set initial temperature and salinity to be constant within each layer.  All vertical tracer diffusion must be off so that the density in each layer remains constant.  For an isopycnal set-up, the equation of state is still called at each timestep, so a linear equation of state is recommended to avoid depth-dependancy of the density.  The user may choose a Montgomery Potential (\ref{ocean:\mode_Montgomery Potential}) or standard pressure gradient (\ref{ocean:\mode_grad p}); Montgomery Potential is a more common choice for isopycnal configurations, but both are tested and functional.  MPAS-Ocean does not support massless layers at this time, so isopycnal vertical coordinates may only be used for idealized domains.

\begin{table}[ht] 
\caption{Vertical coordinate settings for traditional names.}
\vspace{0.5cm} \centering 
\begin{tabular}{c c c c c c} 
\hline\hline flag name &  {\bf Z-Level} & {\bf Z-star} & {\bf weighted Z-star} &  {\bf isopycnal}  \\
\hline 
config\_vert\_ & 'fixed' & 'uniform\_stretching' & 'user\_specified' & 'impermeable\_ \\
coord\_movement & & & & interfaces'
\\
weights & $W_k =\left\{ \begin{array}{c} 1\; k=1\\ 0\; k>1 \end{array}\right.$ & $W_k=1\;\;\forall\;\;k$ & input file & not applicable \\
\hline 
\end{tabular} \label{oceanTable:\mode_vertical coordinates} 
\end{table}

\begin{table}[h!t] 
\caption{Variables used in ALE equation sets.  Column 3 shows the native horizontal grid location.  A subscript $k$ indicates indicates the layer index.  The $\nabla$ indicates a horizontal gradient within a single layer.} 
\vspace{0.5cm} \centering 
\begin{tabular}{c c c c } 
\hline\hline symbol &  name & grid &  notes  \\
\hline 
$D$  & thickness-weighted divergence & cell & $D_k = \nabla \cdot  \left( h_k {\bf u}_k \right)$  \\ 
${\bar D}$ & barotropic divergence & cell & ${\overline D} = \sum\limits_{k=1}^{kmax} D_k$  \\ 
$D'$  & baroclinic divergence & cell & $D'_k = D_k-\frac{h_k}{H}{\bar D}$  \\ 
$D^{lf}$  & low-frequency divergence & cell & see (\ref{ocean:\mode_Dlf})  \\ 
$D^{hf}$  & high-frequency divergence & cell & $D^{hf}_k = D'_k - D^{lf}_k$  \\ 
$h$  & layer thickness & cell &   \\ 
$h^{ALE}$  & desired ALE thickness & cell & see (\ref{ocean:\mode_desired thickness})  \\ 
$h^{rest}$  & resting thickness & cell &   \\ 
$h^{SSH}$  & SSH thickness alteration & cell & see (\ref{ocean:\mode_h ssh})  \\ 
$h^{hf}$  & high-freq. thickness alteration & cell &   see (\ref{ocean:\mode_hhf})  \\ 
$h^{min}$  & minimum thickness alteration & cell &   \\ 
$H$  & total resting thickness & cell & $H= \sum\limits_{k=1}^{kmax} h_k^{rest}$  \\ 
${\bf u}$  & velocity & edge &   \\ 
$w^t$ & vertical transport & cell  & top of layer in vertical \\
$W$  & SSH thickness weights & cell &   \\ 
$\tau_{Dlf}$  & frequency filter time scale & constant &   \\ 
$\tau_{hhf}$  & restoring time scale for $h^{hf}$ & constant &   \\ 
$\kappa_{hhf}$  & $h^{hf}$ diffusion & constant &   \\ 
$\zeta$  & sea surface height & cell &  $\zeta= \sum\limits_{k=1}^{kmax} h_k^{rest} - H$  \\ 
\hline 
\end{tabular} \label{oceanTable:\mode_ALE_variables} 
\end{table}


\vspace{0.5in}
{\small
\begin{center}
\begin{longtable}{| p{2.0in} || p{4.0in} |}
    \hline
    {\bf Name} & {\bf Description} \endfirsthead
    \hline 
    {\bf Name} & {\bf Description} (Continued) \endhead
    \hline
    \hline
    \hyperref[subsec:nm_sec_config_vert_coord_movement]{config\_vert\_coord\_movement} & Determines the vertical coordinate movement type. 'uniform\_stretching' distributes SSH perturbations through all vertical levels (z-star vertical coordinate); 'fixed' places them all in the top level (z-level vertical coordinate); 'user\_specified' allows the input file to determine the distribution using the variable vertCoordMovementWeights (weighted z-star vertical coordinate); and 'impermeable\_interfaces' makes the vertical transport between layers zero, i.e. $w^t=0$ (idealized isopycnal). \\
    \hline
    \hyperref[subsec:nm_sec_config_use_min_max_thickness]{config\_use\_min\_max\_thickness} & If true, a minimum and maximum thicknesses are enforced to prevent massless and very thick layers. \\
    \hline
    \hyperref[subsec:nm_sec_config_min_thickness]{config\_min\_thickness} & Minimum thickness allowed. \\
    \hline
    \hyperref[subsec:nm_sec_config_max_thickness_factor]{config\_max\_thickness\_factor} & Maximum thickness allowed. This is a factor times the resting thickness, i.e., maximum thickness = config\_max\_thickness\_factor*$h^{rest}$. \\
    \hline
    \hyperref[subsec:nm_sec_config_dzdk_positive]{config\_dzdk\_positive} & Determines if the positive Z axis is aligned with the positive K index direction. \\
    \hline
\end{longtable}
\end{center}
}
\section[ALE\_frequency\_filtered\_thickness]{\hyperref[sec:nm_sec_ALE_frequency_filtered_thickness]{ALE\_frequency\_filtered\_thickness}}
\label{sec:nm_tab_ALE_frequency_filtered_thickness}
The high-frequency thickness alteration, $h^{hf}$, in (\ref{ocean:\mode_desired thickness}) allows the thicknesses to oscillate so that high-frequency motions, such as internal gravity waves, are treated in a Lagrangian manner.  Low-freqency motions, such as seasonal changes or slow motions of water masses, are treated in an Eulerian manner.  This is the ``z-tilde'' scheme of \citet{Leclair_Madec11om}, which generally reduces spurious vertical mixing and preserves water mass properties.  Two additional prognostic equations are solved when config\_use\_freq\_filtered\_thickness is true,
\begin{eqnarray}
\label{ocean:\mode_Dlf}
 & \displaystyle
  \frac{\partial D^{lf}_k}{\partial t} = - \frac{2\pi}{\tau_{Dlf}} \left( D^{lf}_k - D'_k \right), 
\\ & \displaystyle
\label{ocean:\mode_hhf}
\frac{\partial h^{hf}_k}{\partial t} =  - D^{hf}_k - \frac{2\pi}{\tau_{hhf}} h^{hf}_k + \nabla_h\cdot \left( \kappa_{hhf} \nabla_h h^{hf}_k \right) 
\end{eqnarray}
where $\tau_{Dlf}$ is the filter timescale and other variables are defined in Table \ref{oceanTable:\mode_ALE_variables}.  This may be used in addition to any of the z-type or sigma-type vertical coordinates in Table \ref{oceanTable:\mode_vertical coordinates}.  Some combination of thickness restoring and diffusion are recommended to avoid long-term drift of $h^{hf}$ away from zero.




\vspace{0.5in}
{\small
\begin{center}
\begin{longtable}{| p{2.0in} || p{4.0in} |}
    \hline
    {\bf Name} & {\bf Description} \endfirsthead
    \hline 
    {\bf Name} & {\bf Description} (Continued) \endhead
    \hline
    \hline
    \hyperref[subsec:nm_sec_config_use_freq_filtered_thickness]{config\_use\_freq\_filtered\_\-thickness} & If true, $h^{hf}$ is included in the desired ALE thickness, and the prognostic equations for $D^{lf}$ and $h^{hf}$ are integrated in the code. \\
    \hline
    \hyperref[subsec:nm_sec_config_thickness_filter_timescale]{config\_thickness\_filter\_\-timescale} & Filter time scale for the low-frequency baroclinic divergence, $\tau_{Dlf}$. \\
    \hline
    \hyperref[subsec:nm_sec_config_use_highFreqThick_restore]{config\_use\_highFreqThick\_\-restore} & If true, the high frequency thickness prognostic equation ($h^{hf}$) includes term 2 on the RHS, the restoring term.  The high frequency thickness is restored to zero with time scale $\tau_{hhf}$. \\
    \hline
    \hyperref[subsec:nm_sec_config_highFreqThick_restore_time]{config\_highFreqThick\_restore\_\-time} & Restoring time scale for high-frequency thickness, $\tau_{hhf}$. \\
    \hline
    \hyperref[subsec:nm_sec_config_use_highFreqThick_del2]{config\_use\_highFreqThick\_del2} & If true, high frequency thickness prognostic equation ($h^{hf}$) includes term 3 on the RHS, the diffusion term. \\
    \hline
    \hyperref[subsec:nm_sec_config_highFreqThick_del2]{config\_highFreqThick\_del2} & Horizonal high frequency thickness diffusion, $\kappa_{hhf}$. \\
    \hline
\end{longtable}
\end{center}
}
\section[partial\_bottom\_cells]{\hyperref[sec:nm_sec_partial_bottom_cells]{partial\_bottom\_cells}}
\label{sec:nm_tab_partial_bottom_cells}
MPAS-Ocean uses partial bottom cells (PBCs) to better represent topography \citep{Pacanowski_Gnanadesikan98mwr}.  A simulation that uses PBCs will alter the layer thickness of the bottom cell based of the bottom depth of the column.  The bottom of a PBC cell remains flat like stair-steps (Figure \ref{oceanFigure:pbcs}), rather than a piecewise linear fit of the bottom topography to the base of the cell \citep{Adcroft_ea97mwr}. 

There are two variables for bottom depth: refBottomDepth(k) is the reference, or typical, bottom depth for all cells in layer $k$; bottomDepth(iCell), $H_i$, is the bottom depth for cell iCell.  In order to use PBCs, alter the thickness of the bottom cell, $h_{kmax,i}$, such that
\begin{eqnarray}
\label{ocean:pbc thickness}
H_i =  \sum_{k=1}^{kmax}h_{k,i}.
\end{eqnarray}
This can be done when creating the initial conditions, or upon start-up of the simulation the first time using the flags below.  In order to avoid extremely small cells, a minimum cell fraction setting is provided.  Upon start-up, one may measure the sea surface height,
\begin{eqnarray}
\label{ocean:pbc SSH}
\zeta_i = \sum_{k=1}^{kmax}h_{k,i} - H_i,
\end{eqnarray}
and check that it is less than 2m.  If not, there is likely an error in the set-up of the layer thickness, and MPAS-Ocean ends with an error message.

When PBCs are in use, PBC thicknessess are used for $h$ in the governing equations, (\ref{ocean:momentum}--\ref{ocean:tracer}).  Note there is no flag that turns PBCs on in the main part of the code.  The flags below simply alter the initial conditions upon the first start-up.

\begin{figure}[htb]
\centering
\includegraphics[scale=0.4]{ocean/figures/partial_bottom_cells.png}
\caption{Illustration of partial bottom cells.}
\label{oceanFigure:pbcs}
\end{figure}

For typical global domains, the initial condition file includes the following variables:
\begin{itemize}
\item layerThickness: all full cells
\item tracers: values at middle (in vertical) of full cells
\item bottomDepth: depth of column from bathymetry data
\end{itemize}
Note that there is a mismatch in these data fields, because layerThickness and tracers are initialized for full cells, but bottomDepth is from bathymetry and so may fall anywhere within the bottom cell.  This configuration of initial conditions was chosen so that a single file may be used for any PBC setting upon start-up.  These variables are changed for the PBC configuration when starting from initial conditions, and then the PBC settings remain the same for all subsequent restart simulations.  See flags below for a description of how the variables are altered for PBCs.

\vspace{0.5in}
{\small
\begin{center}
\begin{longtable}{| p{2.0in} || p{4.0in} |}
    \hline
    {\bf Name} & {\bf Description} \endfirsthead
    \hline 
    {\bf Name} & {\bf Description} (Continued) \endhead
    \hline
    \hline
    \hyperref[subsec:nm_sec_config_alter_ICs_for_pbcs]{config\_alter\_ICs\_for\_pbcs} & If true, initial conditions are altered according to the config\_pbc\_alteration\_type flag. The alteration of layer thickness only occurs when config\_do\_restart is false, i.e. on an initial run. \\
    \hline
    \hyperref[subsec:nm_sec_config_pbc_alteration_type]{config\_pbc\_alteration\_type} & Determines the method of initial condition alteration for partial bottom cells. 'partial\_cell' alters layerThickness, interpolates all tracers in the bottom cell based on the bottomDepth variable, and alters bottomDepth to enforce the minimum pbc fraction. 'full\_cell' alters bottomDepth to have full cells everywhere, based on the refBottomDepth variable. \\
    \hline
    \hyperref[subsec:nm_sec_config_min_pbc_fraction]{config\_min\_pbc\_fraction} & Determines the minimum fraction of a cell altering the initial conditions can create. The alteration of layer thickness only occurs when config\_do\_restart is false, i.e. on an initial run. \\
    \hline
\end{longtable}
\end{center}
}
\section[hmix]{\hyperref[sec:nm_sec_hmix]{hmix}}
\label{sec:nm_tab_hmix}
There are several choices of horizontal mixing schemes available for the 
momentum and tracer equations.  Each of these is a turbulence closure, 
and attempts to account for subgrid-scale mixing and diffusion.  These 
schemes have the practical effect of reducing grid-scale noise in the 
velocity and tracer fields, and improving numerical stability.

Each horizontal mixing scheme has its own namelist, and may be turned
on with the \verb|_use_| logical configuration flags.  Multiple
schemes may be run simultaneously.  The horizontal mixing terms in the
governing equations (\ref{ocean:momentum},
\ref{ocean:tracer}) are ${\bf D}^u_h$ for momentum and
$D^\varphi_h$ for tracers.  No horizontal mixing is applied to the
thickness equation.

All horizontal mixing coefficients can be set to scale with the mesh as $\rho_m^{-3/4}$ in equations (\ref{ocean:h_mom_del2}, \ref{ocean:h_tr_del2}, \ref{ocean:h_mom_del4}, \ref{ocean:h_tr_del4}).  The mesh density, $\rho_m$, is a variable in the input and restart file.  It can vary between zero and one, and is one in the highest resolution region.  Scaling with the mesh can be turned off, as described in the options below.

The anticipated potential Vorticity (APV) method is a parameterization of the effects of subgrid or unresolved scales on those explicitly resolved \citep{Vallis_Hua88jas}.  It contributes an upstream bias to the vorticity in the del2 and del4 momentum terms as follows,
\begin{equation}
\eta_{apv} = \eta - c_{apv} dt \left({\bf u}\cdot \nabla \eta\right),
\end{equation}
where the altered vorticity $\eta_{apv}$ is used in equations (\ref{ocean:h_mom_del2}, \ref{ocean:h_mom_del4a}, \ref{ocean:h_mom_del4b}).

\vspace{0.5in}
{\small
\begin{center}
\begin{longtable}{| p{2.0in} || p{4.0in} |}
    \hline
    {\bf Name} & {\bf Description} \endfirsthead
    \hline 
    {\bf Name} & {\bf Description} (Continued) \endhead
    \hline
    \hline
    \hyperref[subsec:nm_sec_config_hmix_scaleWithMesh]{config\_hmix\_scaleWithMesh} & If false, del2 and del4 coefficients are constant throughout the mesh (equivalent to setting $\rho_m=1$ throughout the mesh).  If true, these coefficients scale as mesh density to the -3/4 power. \\
    \hline
    \hyperref[subsec:nm_sec_config_maxMeshDensity]{config\_maxMeshDensity} & Global maximum of the mesh density \\
    \hline
    \hyperref[subsec:nm_sec_config_apvm_scale_factor]{config\_apvm\_scale\_factor} & Anticipated potential vorticity (APV) method scale factor, $c_{apv}$. When zero, APV is off. \\
    \hline
\end{longtable}
\end{center}
}
\section[hmix\_del2]{\hyperref[sec:nm_sec_hmix_del2]{hmix\_del2}}
\label{sec:nm_tab_hmix_del2}
The ``del2'', or Laplacian, turbulence closures are
\begin{eqnarray}
\label{ocean:h_mom_del2}
& {\bf D}^u_h=\displaystyle\frac{\nu_h}{\rho_m^{3/4}} \nabla^2 {\bf u} 
= \displaystyle\frac{\nu_h}{\rho_m^{3/4}}(\nabla \delta + {\bf 
k}\times \nabla \eta),\\
\label{ocean:h_tr_del2}
& D^\varphi_h = \nabla\cdot\left(h 
   \displaystyle\frac{\kappa_h}{\rho_m^{3/4}} \nabla\varphi \right)
\end{eqnarray}
for momentum and tracers, respectively.  Variable definitions appear in Tables \ref{oceanTable:variables} and \ref{oceanTable:variables_Greek}.  The momentum diffusion is in divergence-vorticity form because it is a natural discretization of the vector Laplacian operator with a C-grid staggering.  

The Laplacian operator smooths the momentum and 
tracer fields, and smooths more strongly at small scales than at large 
scales.  This operator is the two dimensional form of the heat equation, 
$u_t=\nu u_{xx}$, described in introductory books on partial 
differential equations.  The strength of mixing is controlled by the 
viscosity, $\nu_h$, for the momentum equation, and the diffusion, 
$\kappa_h$, for the tracer equation.

\vspace{0.5in}
{\small
\begin{center}
\begin{longtable}{| p{2.0in} || p{4.0in} |}
    \hline
    {\bf Name} & {\bf Description} \endfirsthead
    \hline 
    {\bf Name} & {\bf Description} (Continued) \endhead
    \hline
    \hline
    \hyperref[subsec:nm_sec_config_use_mom_del2]{config\_use\_mom\_del2} & If true, Laplacian horizontal mixing is used on the momentum equation. \\
    \hline
    \hyperref[subsec:nm_sec_config_use_tracer_del2]{config\_use\_tracer\_del2} & If true, Laplacian horizontal mixing is used on the tracer equation. \\
    \hline
    \hyperref[subsec:nm_sec_config_mom_del2]{config\_mom\_del2} & Horizonal viscosity, $\nu_h$. \\
    \hline
    \hyperref[subsec:nm_sec_config_tracer_del2]{config\_tracer\_del2} & Horizonal diffusion, $\kappa_h$. \\
    \hline
\end{longtable}
\end{center}
}
\section[hmix\_del4]{\hyperref[sec:nm_sec_hmix_del4]{hmix\_del4}}
\label{sec:nm_tab_hmix_del4}
The ``del4'', or biharmonic, turbulence closures are
\begin{eqnarray}
\label{ocean:h_mom_del4}
& {\bf D}^u_h=-\displaystyle\frac{\nu_h}{\rho_m^{3/4}} \nabla^4 {\bf u} \\
\label{ocean:h_tr_del4}
& D^\varphi_h = -\nabla\cdot\left(h \displaystyle\frac{\kappa_h}{\rho_m^{3/4}} \nabla \left[\nabla\cdot\left(h \nabla\varphi \right) \right] \right)
\end{eqnarray}
for momentum and tracers  These are both computed by applying the Laplacian operator twice.  For momentum, this can be written in terms of divergence and vorticity as
\begin{eqnarray}
&\delta=\nabla\cdot{\bf u}\\
&\eta={\bf k} \cdot \left( \nabla \times {\bf u} \right)+f\\
\label{ocean:h_mom_del4a}
&\nabla^2{\bf u}=(\nabla \delta + {\bf k}\times \nabla \eta) \\
&\delta_2=\nabla\cdot(\nabla^2{\bf u})\\
&\eta_2={\bf k} \cdot \left( \nabla \times (\nabla^2{\bf u}) \right)+f\\
\label{ocean:h_mom_del4b}
& {\bf D}^u_h= \displaystyle\frac{\nu_h}{\rho_m^{3/4}} (\nabla \delta_2 + {\bf k}\times \nabla \eta_2).
\end{eqnarray}
The biharmonic operator is similar to the Laplacian operator, but smooths more strongly at high wavenumbers.  

\vspace{0.5in}
{\small
\begin{center}
\begin{longtable}{| p{2.0in} || p{4.0in} |}
    \hline
    {\bf Name} & {\bf Description} \endfirsthead
    \hline 
    {\bf Name} & {\bf Description} (Continued) \endhead
    \hline
    \hline
    \hyperref[subsec:nm_sec_config_use_mom_del4]{config\_use\_mom\_del4} & If true, biharmonic horizontal mixing is used on the momentum equation. \\
    \hline
    \hyperref[subsec:nm_sec_config_use_tracer_del4]{config\_use\_tracer\_del4} & If true, biharmonic horizontal mixing is used on the tracer equation. \\
    \hline
    \hyperref[subsec:nm_sec_config_mom_del4]{config\_mom\_del4} & Coefficient for horizontal biharmonic operator on momentum. \\
    \hline
    \hyperref[subsec:nm_sec_config_mom_del4_div_factor]{config\_mom\_del4\_div\_factor} & The divergence portion of the del4 operator is scaled by this factor. \\
    \hline
    \hyperref[subsec:nm_sec_config_tracer_del4]{config\_tracer\_del4} & Coefficient for horizontal biharmonic operator on tracers. \\
    \hline
\end{longtable}
\end{center}
}
\section[hmix\_Leith]{\hyperref[sec:nm_sec_hmix_Leith]{hmix\_Leith}}
\label{sec:nm_tab_hmix_Leith}
The \cite{Leith:1996wu} closure is the enstrophy-cascade analogy to the \cite{Smagorinsky:1963wc} energy-cascade closure, i.e.  the Leith closure assumes an inertial range of enstrophy flux moving toward the grid scale. The assumption of an enstrophy cascade and dimensional analysis produces right-hand-side dissipation, $\bf{D}^u_h$, of velocity of the form

\begin{equation}
\label{eq:\mode_Leith}
{\bf D}^u_h =\nabla \cdot \left( \nu_h \nabla {\bf u} \right) = \nabla \cdot \left( \Gamma \left| \nabla \omega  \right| \left( \Delta x \right)^3 \nabla \bf{u} \right)
\end{equation}
where $\omega$ is the relative vorticity, ${\bf u}$ is the horizontal velocity, $\Delta x$ is the local grid spacing and $\Gamma$ is a non-dimensional, $O(1)$ parameter. This beta release approximates the RHS of the \ref{eq:\mode_Leith} as

\begin{equation}
\bf{D}^u_h=\nu_\ast \nabla_h^2 {\bf u}
\end{equation}
where the $\nabla^2 {\bf u}$ is computed using the form shown in \ref{ocean:\mode_h_mom_del4a}. Future releases will remove this approximation by computing the rate-of-strain, i.e. $\nabla {\bf u}$, directly.

\vspace{0.5in}
{\small
\begin{center}
\begin{longtable}{| p{2.0in} || p{4.0in} |}
    \hline
    {\bf Name} & {\bf Description} \endfirsthead
    \hline 
    {\bf Name} & {\bf Description} (Continued) \endhead
    \hline
    \hline
    \hyperref[subsec:nm_sec_config_use_Leith_del2]{config\_use\_Leith\_del2} & If true, the Leith enstrophy-cascade closure is turned on \\
    \hline
    \hyperref[subsec:nm_sec_config_Leith_parameter]{config\_Leith\_parameter} & Non-dimensional Leith closure parameter \\
    \hline
    \hyperref[subsec:nm_sec_config_Leith_dx]{config\_Leith\_dx} & Characteristic length scale, usually the smallest dx in the mesh \\
    \hline
    \hyperref[subsec:nm_sec_config_Leith_visc2_max]{config\_Leith\_visc2\_max} & Upper bound on the allowable value of Leith-computed viscosity \\
    \hline
\end{longtable}
\end{center}
}
\section[mesoscale\_eddy\_parameterization]{\hyperref[sec:nm_sec_mesoscale_eddy_parameterization]{mesoscale\_eddy\_parameterization}}
\label{sec:nm_tab_mesoscale_eddy_parameterization}


Mesoscale eddy parameterization (MEP) is an important process in determining the distribution of temperature, salinity and, as a result, density throughout the world ocean.   The MPAS-Ocean MEP includes the Gent-McWilliams (GM) parameterization (\cite{Gent_McWilliams90jpo,Gent_ea95jpo,Gent2011om}), which is composed of an enhanced tracer advection (Bolus component), and a tracer diffusion that is aligned with isopycnals (Redi component).

For the purpose of this section, consider the thickness and the passive tracer equations of MPAS-Ocean with advective terms only,
\begin{align}   
\dfrac{\partial h_k}{\partial t} 
 + \nabla \cdot \left( h_k {\bf u}_k \right) 
% + w^{top}_{k+1} - w^{top}_{k} = 0
 + w_k^\textrm{t} - w_k^\textrm{b} = 0\label{ocean:GM:1},
\\
\dfrac{\partial h_k\varphi_k}{\partial t} 
 + \nabla \cdot \left( h_k\varphi_k {\bf u}_k \right) 
%+  w^{top}_{k+1}\varphi^{top}_{k+1} - w^{top}_{k}\varphi^{top}_{k}
  + \varphi_k^\textrm{t} w_k^\textrm{t} - \varphi_k^\textrm{b} w_k^\textrm{b}
= 0.\label{ocean:GM:4}
\end{align}
Here the superscripts 't' and 'b' denote the top and bottom surfaces,
respectively, of a fluid layer with index $k$.
The Gent-McWilliams closure replaces the mean transport velocity ${\bf u}$
and $w$ by the effective transport velocity ${\bf U}$ and $W$,
\begin{align}   
\dfrac{\partial h_k}{\partial t} 
 + \nabla \cdot \left( h_k {\bf U}_k \right) 
% + w^{top}_{k+1} - w^{top}_{k} = 0
 + W_k^\textrm{t} - W_k^\textrm{b} = 0,\label{ocean:GM:5}
\\
\dfrac{\partial h_k\varphi_k}{\partial t} 
 + \nabla \cdot \left( h_k\varphi_k{\bf U}_k \right) 
%+  w^{top}_{k+1}\varphi^{top}_{k+1} - w^{top}_{k}\varphi^{top}_{k}
  + \varphi_k^\textrm{t} W_k^\textrm{t} - \varphi_k^\textrm{b} W_k^\textrm{b}  
= \nabla\cdot\left(h_k {\bf K} \cdot \nabla\varphi_k \right).\label{ocean:GM:6}
\end{align}
In the above, the effective transport velocities can be decomposed
into the mean transport velocity and the GM bolus velocities, 
\begin{align}
&{\bf U}_k = u_k + u^\ast_k,\label{ocean:GM:7}\\ 
& W_k = w_k + w^\ast_k,\label{ocean:GM:8},
\end{align}
and ${\bf K}$ is a $3\times 3$ tensor that specifies the Redi diffusion.

The Bolus velocities are computed using a stream function formulation.  The stream function is defined as
\begin{equation}
  \label{ocean:GM:10}
  \Psi = \boldsymbol{\gamma}\times{\bf k},
\end{equation}
where $\boldsymbol{\gamma}$ is a horizontal 2D vector and ${\bf k}$ is the
vertical unit vector. Let 
\begin{equation}
  \label{ocean:GM:2}
  {\bf v} \equiv ({\bf u}^\ast,\,w^\ast) = \nabla_3 \times\Psi.
\end{equation}
Then we find that 
\begin{align}
  &{\bf u}^\ast = \partial_z\boldsymbol{\gamma},\label{ocean:GM:13}\\
 &w^\ast = -\nabla_z\cdot\boldsymbol{\gamma}.\label{ocean:GM:14}
\end{align}
Following \cite{Ferrari_ea10om}, one solves a boundary value
problem for the horizontal 2D vector
$\boldsymbol{\gamma}$ in \eqref{ocean:GM:13} and \eqref{ocean:GM:14},
\begin{align}
  &\left( c^2\dfrac{\,\mathrm{d}^2}{\,\mathrm{d} z^2} - N^2\right)\boldsymbol{\gamma} =
  (g/\rho_0)\kappa \nabla_z\rho,\label{ocean:GM:15}\\
  &\boldsymbol{\gamma}(\eta) = \boldsymbol{\gamma}(-H) = 0.\label{ocean:GM:3}
\end{align}
Here $\kappa$ is the GM eddy transport coefficient, and is the main parameter to adjust the strength of the Bolus component; $c$ is the gravity wave speed for the boundary value problem; $\eta$ is the sea surface; $-H$ is the bottom; $N^2$ is the Brunt-Vasalai frequency, and is given by $N^2 = -(g/\rho_0)\partial_z\rho$. 

\vspace{0.5in}
{\small
\begin{center}
\begin{longtable}{| p{2.0in} || p{4.0in} |}
    \hline
    {\bf Name} & {\bf Description} \endfirsthead
    \hline 
    {\bf Name} & {\bf Description} (Continued) \endhead
    \hline
    \hline
    \hyperref[subsec:nm_sec_config_use_standardGM]{config\_use\_standardGM} & If true, the standard GM for the tracer advection and mixing is turned on. This includes the redi portion which is captured in hmix. \\
    \hline
    \hyperref[subsec:nm_sec_config_use_Redi_surface_layer_tapering]{config\_use\_Redi\_surface\_\-layer\_tapering} & If true, the Redi K33 vertical mixing is limited in and just below the ocean boundary layer. \\
    \hline
    \hyperref[subsec:nm_sec_config_Redi_surface_layer_tapering_extent]{config\_Redi\_surface\_layer\_\-tapering\_extent} & Vertical component of Redi mixing limited in top config\_Redi\_surface\_layer\_tapering\_extent*boundaryLayerDepth \\
    \hline
    \hyperref[subsec:nm_sec_config_use_Redi_bottom_layer_tapering]{config\_use\_Redi\_bottom\_\-layer\_tapering} & If true, the Redi K33 vertical mixing is limited in bottom boundary layer. \\
    \hline
    \hyperref[subsec:nm_sec_config_Redi_bottom_layer_tapering_depth]{config\_Redi\_bottom\_layer\_\-tapering\_depth} & Vertical component of Redi mixing limited in bottom config\_Redi\_surface\_layer\_tapering\_depth \\
    \hline
    \hyperref[subsec:nm_sec_config_standardGM_tracer_kappa]{config\_standardGM\_tracer\_\-kappa} & Coefficient of standard GM parametrization of eddy transport (Bolus component), $\kappa$ \\
    \hline
    \hyperref[subsec:nm_sec_config_Redi_kappa]{config\_Redi\_kappa} & The Redi diffusion coefficient \\
    \hline
    \hyperref[subsec:nm_sec_config_gravWaveSpeed_trunc]{config\_gravWaveSpeed\_trunc} & Gravity wave speed truncation threshold for the vertical stream function boundary value problem, $c$ \\
    \hline
    \hyperref[subsec:nm_sec_config_max_relative_slope]{config\_max\_relative\_slope} & Tapering parameter for Redi diffusion coeffient.  Tapering occurs when relativeSlopeTopOfEdge is greater than config\_max\_relative\_slope \\
    \hline
\end{longtable}
\end{center}
}
\section[hmix\_del2\_tensor]{\hyperref[sec:nm_sec_hmix_del2_tensor]{hmix\_del2\_tensor}}
\label{sec:nm_tab_hmix_del2_tensor}
A tensor version of the ``del2'' operator is provided for the momentum closure,
\begin{eqnarray}
\label{ocean:h_mom_del2_tensor}
& {\bf D}^u_h = \nabla\cdot\left( 
   \displaystyle\frac{\nu_h}{\rho_m^{3/4}} \nabla {\bf u}  \right).
\end{eqnarray}
The standard form for the del2 momentum closure is the divergence-vorticity form, described in the hmix\_del2 namelist above.  The tensor-based mixing operations are provided to show the functionality of the tensor subroutines.  Tensor-based del2 mixing is not fully vetted and not recommended for production simulations.

\vspace{0.5in}
{\small
\begin{center}
\begin{longtable}{| p{2.0in} || p{4.0in} |}
    \hline
    {\bf Name} & {\bf Description} \endfirsthead
    \hline 
    {\bf Name} & {\bf Description} (Continued) \endhead
    \hline
    \hline
    \hyperref[subsec:nm_sec_config_use_mom_del2_tensor]{config\_use\_mom\_del2\_tensor} & If true, tensor-based Laplacian horizontal mixing is used on the momentum equation. The tensor-based mixing operations are provided to show the functionality of the tensor subroutines. Tensor-based del2 mixing is not fully vetted and not recommended for production simulations. \\
    \hline
    \hyperref[subsec:nm_sec_config_mom_del2_tensor]{config\_mom\_del2\_tensor} & Horizonal viscosity, $\nu_h$. \\
    \hline
\end{longtable}
\end{center}
}
\section[hmix\_del4\_tensor]{\hyperref[sec:nm_sec_hmix_del4_tensor]{hmix\_del4\_tensor}}
\label{sec:nm_tab_hmix_del4_tensor}
A tensor version of the ``del4'' operator is provided for the momentum closure,
\begin{eqnarray}
\label{ocean:h_mom_del4_tensor}
& {\bf D}^u_h = \nabla\cdot\left( 
   \displaystyle\frac{\nu_h}{\rho_m^{3/4}} \nabla 
\left[
\nabla\cdot \left(
   \displaystyle \nabla {\bf u} \right)  \right]
  \right).
\end{eqnarray}
The standard form for the del4 momentum closure is the divergence-vorticity form, described in the hmix\_del4 namelist above.  The tensor-based mixing operations are provided to show the functionality of the tensor subroutines.  Tensor-based del4 mixing is not fully vetted and not recommended for production simulations.

\vspace{0.5in}
{\small
\begin{center}
\begin{longtable}{| p{2.0in} || p{4.0in} |}
    \hline
    {\bf Name} & {\bf Description} \endfirsthead
    \hline 
    {\bf Name} & {\bf Description} (Continued) \endhead
    \hline
    \hline
    \hyperref[subsec:nm_sec_config_use_mom_del4_tensor]{config\_use\_mom\_del4\_tensor} & If true, tensor-based biharmonic horizontal mixing is used on the momentum equation. The tensor-based mixing operations are provided to show the functionality of the tensor subroutines. Tensor-based del4 mixing is not fully vetted and not recommended for production simulations. \\
    \hline
    \hyperref[subsec:nm_sec_config_mom_del4_tensor]{config\_mom\_del4\_tensor} & Coefficient for horizontal biharmonic operator on momentum. \\
    \hline
\end{longtable}
\end{center}
}
\section[Rayleigh\_damping]{\hyperref[sec:nm_sec_Rayleigh_damping]{Rayleigh\_damping}}
\label{sec:nm_tab_Rayleigh_damping}
A linear damping toward a state of rest is available with this namelist option.  It is implemented with a term on the RHS of the momentum equation (\ref{ocean:momentum}) of the form 
\begin{equation}
{\cal F}^u = -c_R {\bf u}.
\end{equation}


\vspace{0.5in}
{\small
\begin{center}
\begin{longtable}{| p{2.0in} || p{4.0in} |}
    \hline
    {\bf Name} & {\bf Description} \endfirsthead
    \hline 
    {\bf Name} & {\bf Description} (Continued) \endhead
    \hline
    \hline
    \hyperref[subsec:nm_sec_config_Rayleigh_friction]{config\_Rayleigh\_friction} & If true, Rayleigh friction is included in the momentum equation at every depth level. \\
    \hline
    \hyperref[subsec:nm_sec_config_Rayleigh_damping_coeff]{config\_Rayleigh\_damping\_coeff} & Inverse-time coefficient for the Rayleigh damping term, $c_R$, applied at every depth level. \\
    \hline
    \hyperref[subsec:nm_sec_config_Rayleigh_bottom_friction]{config\_Rayleigh\_bottom\_\-friction} & If true, Rayleigh friction is only applied to the bottom \\
    \hline
    \hyperref[subsec:nm_sec_config_Rayleigh_bottom_damping_coeff]{config\_Rayleigh\_bottom\_\-damping\_coeff} & Inverse-time coefficient for the Rayleigh damping term, $c_R$, only applied at the bottom. \\
    \hline
\end{longtable}
\end{center}
}
\section[cvmix]{\hyperref[sec:nm_sec_cvmix]{cvmix}}
\label{sec:nm_tab_cvmix}
The CVMix namelist record is intended to control the Community Vertical Mixing package \url{https://code.google.com/p/cvmix/}. It is not fully functional in the ocean core yet.

\vspace{0.5in}
{\small
\begin{center}
\begin{longtable}{| p{2.0in} || p{4.0in} |}
    \hline
    {\bf Name} & {\bf Description} \endfirsthead
    \hline 
    {\bf Name} & {\bf Description} (Continued) \endhead
    \hline
    \hline
    \hyperref[subsec:nm_sec_config_use_cvmix]{config\_use\_cvmix} & If true, use the Community Vertical MIXing routines to compute vertical diffusivity and viscosity \\
    \hline
    \hyperref[subsec:nm_sec_config_cvmix_prandtl_number]{config\_cvmix\_prandtl\_number} & Prandtl number to be used within the CVMix parameterization suite \\
    \hline
    \hyperref[subsec:nm_sec_config_use_cvmix_background]{config\_use\_cvmix\_background} & If true, background diffusivity and viscosity is computed using CVMix \\
    \hline
    \hyperref[subsec:nm_sec_config_cvmix_background_diffusion]{config\_cvmix\_background\_\-diffusion} & Background vertical diffusion applied to tracer quantities \\
    \hline
    \hyperref[subsec:nm_sec_config_cvmix_background_viscosity]{config\_cvmix\_background\_\-viscosity} & Background vertical viscosity applied to horizontal velocity \\
    \hline
    \hyperref[subsec:nm_sec_config_use_cvmix_convection]{config\_use\_cvmix\_convection} & If true, convective diffusivity and viscosity is computed using CVMix \\
    \hline
    \hyperref[subsec:nm_sec_config_cvmix_convective_diffusion]{config\_cvmix\_convective\_\-diffusion} & Convective vertical diffusion applied to tracer quantities \\
    \hline
    \hyperref[subsec:nm_sec_config_cvmix_convective_viscosity]{config\_cvmix\_convective\_\-viscosity} & Convective vertical viscosity applied to horizontal velocity components \\
    \hline
    \hyperref[subsec:nm_sec_config_cvmix_convective_basedOnBVF]{config\_cvmix\_convective\_based\-OnBVF} & If true, convection is triggered based on value of config\_cvmix\_convective\_triggerBVF \\
    \hline
    \hyperref[subsec:nm_sec_config_cvmix_convective_triggerBVF]{config\_cvmix\_convective\_trigger\-BVF} & Value of Brunt Viasala frequency squared below which convective mixing is triggered \\
    \hline
    \hyperref[subsec:nm_sec_config_use_cvmix_shear]{config\_use\_cvmix\_shear} & If true, shear-based mixing is computed using CVMix \\
    \hline
    \hyperref[subsec:nm_sec_config_cvmix_num_ri_smooth_loops]{config\_cvmix\_num\_ri\_smooth\_\-loops} & Number of smoothing passes over RiTopOfCell for LMD94 shear instability mixing \\
    \hline
    \hyperref[subsec:nm_sec_config_cvmix_use_BLD_smoothing]{config\_cvmix\_use\_BLD\_\-smoothing} & If true KPP bld is smoothed with a laplacian filter \\
    \hline
    \hyperref[subsec:nm_sec_config_cvmix_shear_mixing_scheme]{config\_cvmix\_shear\_mixing\_\-scheme} & Choose between Pacanowski/Philander or Large et al. shear mixing \\
    \hline
    \hyperref[subsec:nm_sec_config_cvmix_shear_PP_nu_zero]{config\_cvmix\_shear\_PP\_nu\_\-zero} & Numerator of Pacanowski and Philander (1981) Eq (1) \\
    \hline
    \hyperref[subsec:nm_sec_config_cvmix_shear_PP_alpha]{config\_cvmix\_shear\_PP\_alpha} & Alpha values used in Pacanowski and Philander (1981) Eqs (1) and (2) \\
    \hline
    \hyperref[subsec:nm_sec_config_cvmix_shear_PP_exp]{config\_cvmix\_shear\_PP\_exp} & Exponent used in denominator of Pacanowski and Philander (1981) Eqs (1) \\
    \hline
    \hyperref[subsec:nm_sec_config_cvmix_shear_KPP_nu_zero]{config\_cvmix\_shear\_KPP\_nu\_\-zero} & Maximum diffusivity produced by shear-generated mixing \\
    \hline
    \hyperref[subsec:nm_sec_config_cvmix_shear_KPP_Ri_zero]{config\_cvmix\_shear\_KPP\_Ri\_\-zero} & Theshold gradient Richardson number to produced enhanced diffusivities, See Large et al. (1994) Eq (28a,b,c) \\
    \hline
    \hyperref[subsec:nm_sec_config_cvmix_shear_KPP_exp]{config\_cvmix\_shear\_KPP\_exp} & Exponent relating diffusivities to $Ri_g$. Referred to as $p_1$ in Large et al. (1994) Eq (28b) \\
    \hline
    \hyperref[subsec:nm_sec_config_use_cvmix_tidal_mixing]{config\_use\_cvmix\_tidal\_mixing} & If true, diffusivity and viscosity is computed using CVMix tidal mixing \\
    \hline
    \hyperref[subsec:nm_sec_config_use_cvmix_double_diffusion]{config\_use\_cvmix\_double\_\-diffusion} & If true, diffusivity and viscosity is computed using CVMix double diffusion \\
    \hline
    \hyperref[subsec:nm_sec_config_use_cvmix_kpp]{config\_use\_cvmix\_kpp} & If true, diffusivity and viscosity is computed using CVMix KPP \\
    \hline
    \hyperref[subsec:nm_sec_config_use_cvmix_fixed_boundary_layer]{config\_use\_cvmix\_fixed\_\-boundary\_layer} & If true, boundary layer depth is specified as config\_cvmix\_kpp\_boundary\_layer\_depth \\
    \hline
    \hyperref[subsec:nm_sec_config_cvmix_kpp_boundary_layer_depth]{config\_cvmix\_kpp\_boundary\_\-layer\_depth} & If config\_use\_cvmix\_fixed\_boundary\_layer, then KPP OBL calculation is overwritten with this value \\
    \hline
    \hyperref[subsec:nm_sec_config_cvmix_kpp_criticalBulkRichardsonNumber]{config\_cvmix\_kpp\_criticalBulk\-RichardsonNumber} & Critical bulk Richardson number used to determine bottom of ocean mixed layer \\
    \hline
    \hyperref[subsec:nm_sec_config_cvmix_kpp_matching]{config\_cvmix\_kpp\_matching} & Determines how the KPP diffusivities are matched to values at base of boundary layer \\
    \hline
    \hyperref[subsec:nm_sec_config_cvmix_kpp_EkmanOBL]{config\_cvmix\_kpp\_EkmanOBL} & If true, boundary layer depth is limited by Ekman layer depth \\
    \hline
    \hyperref[subsec:nm_sec_config_cvmix_kpp_MonObOBL]{config\_cvmix\_kpp\_MonObOBL} & If true, boundary layer depth is limited by Monin-Obukhov layer depth \\
    \hline
    \hyperref[subsec:nm_sec_config_cvmix_kpp_interpolationOMLType]{config\_cvmix\_kpp\_interpolation\-OMLType} & Determine bottom of ocean mixed layer using linear, quadratic or cubic interpolation \\
    \hline
    \hyperref[subsec:nm_sec_config_cvmix_kpp_surface_layer_extent]{config\_cvmix\_kpp\_surface\_\-layer\_extent} & The non-dimensional extent of the surface layer, measured as fraction of boundary layer depth \\
    \hline
    \hyperref[subsec:nm_sec_config_cvmix_kpp_surface_layer_averaging]{config\_cvmix\_kpp\_surface\_\-layer\_averaging} & The thickness over which to average when computing surface-averaged velocity and buoyancy \\
    \hline
    \hyperref[subsec:nm_sec_configure_cvmix_kpp_minimum_OBL_under_sea_ice]{configure\_cvmix\_kpp\_\-minimum\_OBL\_under\_sea\_ice} & The minimum allowable boundary layer depth with sea-ice is present \\
    \hline
    \hyperref[subsec:nm_sec_config_cvmix_kpp_stop_OBL_search]{config\_cvmix\_kpp\_stop\_OBL\_\-search} & The search for boundary layer depth is terminated when bulk Richardson number is greater than config\_cvmix\_kpp\_stop\_OBL\_search*config\_cvmix\_kpp\_criticalBulkRichardsonNumber \\
    \hline
    \hyperref[subsec:nm_sec_config_cvmix_kpp_use_enhanced_diff]{config\_cvmix\_kpp\_use\_\-enhanced\_diff} & Flag for use of enhanced diffusion at boundary layer base as in Large et al (1994) \\
    \hline
    \hyperref[subsec:nm_sec_config_cvmix_kpp_nonlocal_with_implicit_mix]{config\_cvmix\_kpp\_nonlocal\_\-with\_implicit\_mix} & flag that moves the non-local computation and application of tendency to after main timestep loop \\
    \hline
\end{longtable}
\end{center}
}
\section[vmix\_const]{\hyperref[sec:nm_sec_vmix_const]{vmix\_const}}
\label{sec:nm_tab_vmix_const}
Here the vertical viscosity $\nu_v$ and diffusion $\kappa_v$ are simply constant throughout the domain.  This option is useful for testing and idealized domains but is not appropriate for real-world simulations, as the deep ocean should have much less mixing that the mixed layer.

\vspace{0.5in}
{\small
\begin{center}
\begin{longtable}{| p{2.0in} || p{4.0in} |}
    \hline
    {\bf Name} & {\bf Description} \endfirsthead
    \hline 
    {\bf Name} & {\bf Description} (Continued) \endhead
    \hline
    \hline
    \hyperref[subsec:nm_sec_config_use_const_visc]{config\_use\_const\_visc} & If true, constant vertical viscosity is included in the momentum equation \\
    \hline
    \hyperref[subsec:nm_sec_config_use_const_diff]{config\_use\_const\_diff} & If true, constant vertical diffusion is included in the tracer equation \\
    \hline
    \hyperref[subsec:nm_sec_config_vert_visc]{config\_vert\_visc} & Vertical viscosity, applied uniformly throughout domain \\
    \hline
    \hyperref[subsec:nm_sec_config_vert_diff]{config\_vert\_diff} & Vertical diffusion, applied uniformly throughout domain \\
    \hline
\end{longtable}
\end{center}
}
\section[vmix\_rich]{\hyperref[sec:nm_sec_vmix_rich]{vmix\_rich}}
\label{sec:nm_tab_vmix_rich}
In the Richardson-number based vertical mixing parameterization \citep{Pacanowski_Philander81jpo}, the vertical diffusivity and viscosity are functions of the Richardson Number,
\begin{eqnarray}   
\label{ocean:\mode_Ri1}
Ri &=& N^2
\left(\frac{\partial V}{\partial z} \right)^{-2}
 = -\frac{g}{\rho_0}\frac{\partial \rho}{\partial z}
\left(\frac{\partial V}{\partial z} \right)^{-2},
\end{eqnarray}
where $V=\sqrt{u^2+v^2}=\sqrt{2ke}$ is the velocity magnitude.  The discrete version is
\begin{eqnarray}   
Ri^{top}_k &=& -\frac{g}{\rho_0}\frac{\rho^*_{k-1}-\rho^*_k}{\frac{1}{2}(h_{k-1}+h_k)}
\left(\frac{u_{k-1}-u_k}{\frac{1}{2}(h_{k-1}+h_k)}\right)^{-2}\\
 &=& -\frac{g}{\rho_0}\frac{(\rho^*_{k-1}-\rho^*_k)\frac{1}{2}(h_{k-1}+h_k)}
{(u_{k-1}-u_k)^2+\epsilon}
\end{eqnarray}
where $top$ indicates a layer interface, $ke$ is the cell-centered kinetic energy, $\rho^*_k$ is the density in layer $k$ adiabatically displaced to the surface, and $\epsilon$ is a small number to avoid dividing by zero.  

The variable $Ri^{top}$ must be available at cell edges for the viscoscity $\nu_v$ and at cell centers for the tracer diffusion $\kappa_v$.  In addition, the computation of shear is native to the edges, while density is native to the cell centers.  

The functional forms for vertical viscosity and diffusivity at each layer interface are as follows,
\begin{eqnarray} \label{ocean:\mode_visc1}  
\nu_v &=& \nu_{bkrd} + c_{Ri}/(1+5Ri)^2\\
\kappa_v &=& \kappa_{bkrd} + (\nu_{bkrd} + c_{Ri}/(1+5Ri)^2)/(1+5Ri)
\end{eqnarray}
for $Ri>=0$.  For unstable stratification, $Ri<0$ and the viscosity and diffusion are set to the convective values, which are typically very high.

Note: The functionality of this namelist has been superceded by the CVMix package, which has it's own namelist.  This namelist has been kept for backwards compatibility with and verification against version 2.0.

\vspace{0.5in}
{\small
\begin{center}
\begin{longtable}{| p{2.0in} || p{4.0in} |}
    \hline
    {\bf Name} & {\bf Description} \endfirsthead
    \hline 
    {\bf Name} & {\bf Description} (Continued) \endhead
    \hline
    \hline
    \hyperref[subsec:nm_sec_config_use_rich_visc]{config\_use\_rich\_visc} & If true, Richardson-number based vertical viscosity is included in the momentum equation \\
    \hline
    \hyperref[subsec:nm_sec_config_use_rich_diff]{config\_use\_rich\_diff} & If true, Richardson-number based vertical diffusion is included in the tracer equation \\
    \hline
    \hyperref[subsec:nm_sec_config_bkrd_vert_visc]{config\_bkrd\_vert\_visc} & Background vertical viscosity for Richardson-number based vertical mixing, $\nu_{bkrd}$ \\
    \hline
    \hyperref[subsec:nm_sec_config_bkrd_vert_diff]{config\_bkrd\_vert\_diff} & Background vertical diffusion for Richardson-number based vertical mixing, $\kappa_{bkrd}$ \\
    \hline
    \hyperref[subsec:nm_sec_config_rich_mix]{config\_rich\_mix} & Coefficient for Richardon-number vertical mixing function, $c_{Ri}$ \\
    \hline
\end{longtable}
\end{center}
}
\section[vmix]{\hyperref[sec:nm_sec_vmix]{vmix}}
\label{sec:nm_tab_vmix}
There are several choices of vertical mixing schemes available for the 
momentum and tracer equations.  These are all applied with an implicit solve at the end of each time step using an operator splitting scheme.  

Implicit vertical mixing is required in ocean models because vertical mixing between unstably stratified layers occurs at timescales faster that other model processes.  The timestep requirement for explicit timestepping is usually set by the horizontal advective CFL condition.  In order to include realistic vertical mixing without very small time steps, we use operator splitting so that the vertical momentum and tracer diffusion terms are treated with an implicit timestep, while the remaining terms of the momentum and tracer equations use an explicit method.

Each vertical mixing scheme has its own namelist, and may be turned
on with the \verb|_use_| logical configuration flags.  Multiple
schemes may be run simultaneously.  The vertical mixing terms in the
governing equations (\ref{ocean:momentum}, \ref{ocean:tracer}) are 
\begin{eqnarray} 
\label{ocean:\mode_v_mom_diff}
& \displaystyle {\bf D}^u_v=\frac{\partial }{\partial z} 
\left( \nu_v \frac{\partial {\bf u}}{\partial z} \right), \\
\label{\mode_v_tr_diff}
& \displaystyle D^\varphi_v = \rho \frac{\partial }{\partial z} 
  \left( \kappa_v \frac{\partial \varphi}{\partial z} \right),
\end{eqnarray}
for momentum and tracers, respectively.  No vertical mixing is applied to the
thickness equation.

\vspace{0.5in}
{\small
\begin{center}
\begin{longtable}{| p{2.0in} || p{4.0in} |}
    \hline
    {\bf Name} & {\bf Description} \endfirsthead
    \hline 
    {\bf Name} & {\bf Description} (Continued) \endhead
    \hline
    \hline
    \hyperref[subsec:nm_sec_config_convective_visc]{config\_convective\_visc} & Value of vertical viscosity to be used in unstable conditions (i.e. Richardson number less than zero) \\
    \hline
    \hyperref[subsec:nm_sec_config_convective_diff]{config\_convective\_diff} & Value of vertical diffusion to be used in unstable conditions (i.e. Richardson number less than zero) \\
    \hline
\end{longtable}
\end{center}
}
\section[vmix\_tanh]{\hyperref[sec:nm_sec_vmix_tanh]{vmix\_tanh}}
\label{sec:nm_tab_vmix_tanh}
\textbf{NOTE}: This option is defunct and may be depracated in a future release.  Use of this parameterization is not recommended.

Here the vertical viscosity $\nu_v$ and diffusion $\kappa_v$ are produced with a hyperbolic tangent function, which produces more mixing in shallow waters and less in the deep ocean.  It is uniform horizontally and in time.  This option is not reproducible in CVMix.


\vspace{0.5in}
{\small
\begin{center}
\begin{longtable}{| p{2.0in} || p{4.0in} |}
    \hline
    {\bf Name} & {\bf Description} \endfirsthead
    \hline 
    {\bf Name} & {\bf Description} (Continued) \endhead
    \hline
    \hline
    \hyperref[subsec:nm_sec_config_use_tanh_visc]{config\_use\_tanh\_visc} & If true, tanh-based vertical viscosity is included in the momentum equation \\
    \hline
    \hyperref[subsec:nm_sec_config_use_tanh_diff]{config\_use\_tanh\_diff} & If true, tanh-based vertical diffusion is included in the tracer equation \\
    \hline
    \hyperref[subsec:nm_sec_config_max_visc_tanh]{config\_max\_visc\_tanh} & maximum viscosity value for tanh-fit function \\
    \hline
    \hyperref[subsec:nm_sec_config_min_visc_tanh]{config\_min\_visc\_tanh} & minimum viscosity value for tanh-fit function \\
    \hline
    \hyperref[subsec:nm_sec_config_max_diff_tanh]{config\_max\_diff\_tanh} & maximum diffusion value for tanh-fit function \\
    \hline
    \hyperref[subsec:nm_sec_config_min_diff_tanh]{config\_min\_diff\_tanh} & minimum diffusion value for tanh-fit function \\
    \hline
    \hyperref[subsec:nm_sec_config_zMid_tanh]{config\_zMid\_tanh} & z-coodinate location of center of tanh function \\
    \hline
    \hyperref[subsec:nm_sec_config_zWidth_tanh]{config\_zWidth\_tanh} & vertical width parameter for tanh function \\
    \hline
\end{longtable}
\end{center}
}
\section[forcing]{\hyperref[sec:nm_sec_forcing]{forcing}}
\label{sec:nm_tab_forcing}
Forcing may be applied to the RHS of the momentum equation (\ref{ocean:momentum}) through the term 
\begin{equation}
{\cal F}^u = \frac{1}{\rho_0 h}\tau \label{ocean:\mode_mom_surf_forcing}
\end{equation}
where $\tau$ is typically the wind stress in $N/m^2$ applied to the top layer.  More generally, momentum forcing may be applied to any layer in the ocean.  The momentum forcing may be constant or monthly, as described in the configuration settings below.  When running within the E3SM, the wind stress is provided by the coupler (see Chapter \ref{chap:E3SM_ocean_coupling}).

Temperature and salinity restoring are applied to the tracer equation (\ref{ocean:tracer}) through the term
\begin{equation}
{\cal F}^\varphi = -h\frac{\varphi-\varphi_{r}}{\tau_{r}}
\end{equation}
where $\varphi_{r}$ is the tracer restoring value and $\tau_{r}$ is the restoring timescale.  This term is only applied at the top layer, and may be constant or monthly, as described in the configuration settings below.  When running within the E3SM, the coupler provides surface heat, salinity, and freshwater fluxes rather than a restoring in this form (see Chapter \ref{chap:E3SM_ocean_coupling}).

\vspace{0.5in}
{\small
\begin{center}
\begin{longtable}{| p{2.0in} || p{4.0in} |}
    \hline
    {\bf Name} & {\bf Description} \endfirsthead
    \hline 
    {\bf Name} & {\bf Description} (Continued) \endhead
    \hline
    \hline
    \hyperref[subsec:nm_sec_config_use_bulk_wind_stress]{config\_use\_bulk\_wind\_stress} & Controls if zonal and meridional components of windstress are used to build surface wind stress. \\
    \hline
    \hyperref[subsec:nm_sec_config_use_bulk_thickness_flux]{config\_use\_bulk\_thickness\_flux} & Controls if a bulk thickness flux will be computed for surface forcing. \\
    \hline
    \hyperref[subsec:nm_sec_config_flux_attenuation_coefficient]{config\_flux\_attenuation\_\-coefficient} & The length scale of exponential decay of surface fluxes. Fluxes are multiplied by $e^{z/\gamma}$, where this coefficient is $\gamma$. \\
    \hline
    \hyperref[subsec:nm_sec_config_flux_attenuation_coefficient_runoff]{config\_flux\_attenuation\_\-coefficient\_runoff} & The length scale of exponential decay of river runoff. Fluxes are multiplied by $e^{z/\gamma}$, where this coefficient is $\gamma$. \\
    \hline
\end{longtable}
\end{center}
}
\section[coupling]{\hyperref[sec:nm_sec_coupling]{coupling}}
\label{sec:nm_tab_coupling}
\vspace{0.5in}
{\small
\begin{center}
\begin{longtable}{| p{2.0in} || p{4.0in} |}
    \hline
    {\bf Name} & {\bf Description} \endfirsthead
    \hline 
    {\bf Name} & {\bf Description} (Continued) \endhead
    \hline
    \hline
    \hyperref[subsec:nm_sec_config_ssh_grad_relax_timescale]{config\_ssh\_grad\_relax\_\-timescale} & Timescale for relaxation of the ssh gradient for coupling. A value of 0.0 (default) removes any relaxation and gives instantaneous response. \\
    \hline
\end{longtable}
\end{center}
}
\section[shortwaveRadiation]{\hyperref[sec:nm_sec_shortwaveRadiation]{shortwaveRadiation}}
\label{sec:nm_tab_shortwaveRadiation}
\vspace{0.5in}
{\small
\begin{center}
\begin{longtable}{| p{2.0in} || p{4.0in} |}
    \hline
    {\bf Name} & {\bf Description} \endfirsthead
    \hline 
    {\bf Name} & {\bf Description} (Continued) \endhead
    \hline
    \hline
    \hyperref[subsec:nm_sec_config_sw_absorption_type]{config\_sw\_absorption\_type} & Name of shortwave absorption type used in simulation.  \\
    \hline
    \hyperref[subsec:nm_sec_config_jerlov_water_type]{config\_jerlov\_water\_type} & Integer value defining the water type used in Jerlov short wave absorption. \\
    \hline
    \hyperref[subsec:nm_sec_config_surface_buoyancy_depth]{config\_surface\_buoyancy\_depth} & Depth over which to apply penetrating SW to sfcBuoyancyFlux \\
    \hline
\end{longtable}
\end{center}
}
\section[frazil\_ice]{\hyperref[sec:nm_sec_frazil_ice]{frazil\_ice}}
\label{sec:nm_tab_frazil_ice}
\vspace{0.5in}
{\small
\begin{center}
\begin{longtable}{| p{2.0in} || p{4.0in} |}
    \hline
    {\bf Name} & {\bf Description} \endfirsthead
    \hline 
    {\bf Name} & {\bf Description} (Continued) \endhead
    \hline
    \hline
    \hyperref[subsec:nm_sec_config_use_frazil_ice_formation]{config\_use\_frazil\_ice\_formation} & Controls if fluxes related to frazil ice process are computed. \\
    \hline
    \hyperref[subsec:nm_sec_config_frazil_in_open_ocean]{config\_frazil\_in\_open\_ocean} & If frazil formation is used, controls if frazil fluxes are computed in the open ocean (as opposed to under land ice). \\
    \hline
    \hyperref[subsec:nm_sec_config_frazil_under_land_ice]{config\_frazil\_under\_land\_ice} & If frazil formation is used, controls if frazil fluxes are computed under land ice. \\
    \hline
    \hyperref[subsec:nm_sec_config_frazil_heat_of_fusion]{config\_frazil\_heat\_of\_fusion} & Energy per kilogram released when sea water freezes. NOTE: test and make consistent with ACME. \\
    \hline
    \hyperref[subsec:nm_sec_config_frazil_ice_density]{config\_frazil\_ice\_density} & Assumed density of frazil. NOTE: test and make consistent with ACME. \\
    \hline
    \hyperref[subsec:nm_sec_config_frazil_fractional_thickness_limit]{config\_frazil\_fractional\_\-thickness\_limit} & maximum fraction of layer thickness than can be used or created at an instant by frazil. \\
    \hline
    \hyperref[subsec:nm_sec_config_specific_heat_sea_water]{config\_specific\_heat\_sea\_water} & Energy per kilogram per C needed to raise ocean temperature 1 C. NOTE: test and make consistent with ACME. \\
    \hline
    \hyperref[subsec:nm_sec_config_frazil_maximum_depth]{config\_frazil\_maximum\_depth} & maximum depth for the formation of frazil \\
    \hline
    \hyperref[subsec:nm_sec_config_frazil_sea_ice_reference_salinity]{config\_frazil\_sea\_ice\_reference\_\-salinity} & assumed salinity of frazil ice in the open ocean. \\
    \hline
    \hyperref[subsec:nm_sec_config_frazil_land_ice_reference_salinity]{config\_frazil\_land\_ice\_\-reference\_salinity} & assumed salinity of frazil ice under land ice. \\
    \hline
    \hyperref[subsec:nm_sec_config_frazil_maximum_freezing_temperature]{config\_frazil\_maximum\_\-freezing\_temperature} & Maximum freezing temperature for the creation of frazil \\
    \hline
    \hyperref[subsec:nm_sec_config_frazil_use_surface_pressure]{config\_frazil\_use\_surface\_\-pressure} & Flag that controls if frazil formation will exert a surface pressure as it is formed. \\
    \hline
\end{longtable}
\end{center}
}
\section[land\_ice\_fluxes]{\hyperref[sec:nm_sec_land_ice_fluxes]{land\_ice\_fluxes}}
\label{sec:nm_tab_land_ice_fluxes}
MPAS-Ocean supports the option to compute fluxes across 
the land ice-ocean interface in either ``standalone'' or ``coupled''
mode.  

In either mode, quadratic top drag is applied as a surface stress 
in (\ref{ocean:\mode_mom_surf_forcing}), where
\begin{align}
\tau = & \rho_0 C_\textrm{D,top} \overline{\left(|u| u\right)}^\textrm{BL},
\end{align}
and where $C_\textrm{D,top}$ is the dimensionless top drag coefficient
and the operator $\overline{\left(\cdot\right)}^\textrm{BL}$ indicates 
the vertical average over the boundary layer below land ice, which
has a depth $H_\textrm{BL}$.  Several additional diagnostics are computed
in either mode:
\begin{align}
u_* = & \sqrt{C_\textrm{D,top} \overline{\left(|u|^2+u_\textrm{tidal}^2\right)}^\textrm{BL}}, \\
\Theta_\textrm{BL} = & \overline{\left(\Theta\right)}^\textrm{BL}, \\
S_\textrm{BL} = & \overline{\left(S\right)}^\textrm{BL}.
\end{align}

The boundary conditions at the ice-ocean interface arise
from conservation of enthalpy and salt across the interface and the requirement that the interface
be at the pressure- and salinity-dependent potential freezing point:
\begin{align}

  \rho_\textrm{fw} L m_w & = \mathcal{F}_\textrm{H,ice} - \rho_\textrm{sw} c_p \gamma_\Theta \left(\Theta_b - \Theta_\textrm{BL} \right), \label{ocean:\mode_land_ice_enthalpy_balance}\\

  \rho_\textrm{fw} m_w S_b & = - \rho_\textrm{sw} \gamma_S \left(S_b - S_\textrm{BL}\right),  \\

  \Theta_b & = \Theta_f(S,p), \\
\end{align}
where $\rho_\textrm{fw}$ is the density of freshwater, $L$ is the latent heat of fusion of
water, $\mathcal{F}_\textrm{H,ice}$ is the sensible heat flux into the ice (always negative),
$\rho_\textrm{sw}$ is the reference denstiy of seawater $c_p$ is the heat capacity of seawater
$\gamma_\Theta$ and $\gamma_S$ are thermal- and salt-transfer velocities,
$\Theta_f$ is the freezing potential temperature and $\Theta_b$, $S_b$ and $m_w$ are the
unknown potential temperature and salinity at the boundary and the melt rate (expressed in 
water-equivalent).  The associated surface freshwater, heat and salt fluxes are
\begin{align}

  \mathcal{F}_\textrm{fw} & = \rho_\textrm{fw} m_w, \\

  \mathcal{F}_\textrm{H} & = c_p \left[F_\textrm{fw} \Theta_b + \rho_\textrm{sw} \gamma_\Theta \left(\Theta_b - \Theta_\textrm{BL}\right)\right], \\

  \mathcal{F}_\textrm{S} & = 0.
\end{align}

In ``coupled'' mode, the time-averaged values of $\Theta_\textrm{BL}$, $S_\textrm{BL}$,
$\gamma_\Theta$ and $\gamma_S$ are computed in MPAS-Ocean, and the fluxes are computed
in the coupler.

In ``standalone'' mode, the boundary conditions are solved and the fluxes computed in 
MPAS-Ocean.  Three flux formulations are supported, ``isomip'', ``jenkins'' and 
``hollandJenkins'', each with its own definition of the freezing potential temperature
and the transfer velocities.

\hfill \break \noindent \textit{isomip:}

\noindent The ISOMIP formulation \citep{Hunter2006} is the simplest, using the 
boundary-layer salinity to compute the freezing potential temperature and a 
velocity independent heat-transfer velocity:
\begin{align}
  \Theta_f & = \lambda_1 S_\textrm{BL} + \lambda_2 + \lambda_3 p_b, \\
  \gamma_\Theta & = \gamma_\textrm{ISO}.
\end{align}
The salt-transfer velocity $\gamma_S$ and the salinity at the boundary $S_b$
are not needed for the flux computation and are not computed for this
formulation.

\hfill \break \noindent \textit{jenkins:}

\noindent The \citet{Jenkins2010} boundary conditions are more sophisticated, 
using the interface salinity in the freezing potential temperature 
and including the spatially variable friction velocity in the computation of 
heat and salt transfer across the boundary layer:
\begin{align}
  \Theta_f & = \lambda_1 S_b + \lambda_2 + \lambda_3 p_b, \\
  \gamma_\Theta & = u_* \Gamma_\Theta, \\
  \gamma_S & = u_* \Gamma_S,
\end{align}
where $\Gamma_\Theta$ and $\Gamma_S$ are constants calibrated from observations 
\citep[e.g.][]{Jenkins2010}.

\hfill \break \noindent \textit{holland-jenkins:}

\noindent The \citet{Holland1999} formulation includes slightly nonlinear velocity 
dependence in heat and salt transfer velocities, and is commonly used in many studies.  
The boundary conditions are as in the ``jenkins'' case except that $\Gamma_\Theta$ and 
$\Gamma_S$ are not constants, but are given by
\begin{align}
  \Gamma_{\Theta,S} & = \frac{1}{\Lambda_\textrm{turb} + \Lambda_\textrm{mol}^{\Theta,S}}, \\
  \Lambda_\textrm{turb} & = \frac{1}{k} \textrm{ln}\left(\frac{u_* \xi_N}{f h_\nu}\right) + \frac{1}{2 \xi_N} - \frac{1}{k}, \\
  \Lambda_\textrm{mol}^{\Theta,S} & = 12.5~(\textrm{Pr},\textrm{Sc})^{2/3} - 6, \\
  h_\nu & = 5 \frac{\nu}{u_*},
\end{align}
where the various parameters $k$, $\xi_n$, $h_\nu$, $\textrm{Pr}$ and $\textrm{Sc}$ have the values defined 
in \citet{Holland1999}.

The sensible heat flux $\mathcal{F}_\textrm{H,ice}$ into the ice in 
(\ref{ocean:\mode_land_ice_enthalpy_balance}) is not known in ``standalone'' mode.  This flux
is generally small, with the dominant balance between latent heat released through melting
and ocean sensible heat fluxes.  The default assumption is that 
$\mathcal{F}_\textrm{H,ice} = 0$, meaning that the ice is perfectly insulating.  An alternative
is to use the advection-diffusion method of \citet{Holland1999}, where the ice is assumed
to advect vertically at the melt rate and its temperature is assumed
to be at a reference surface temperature under melting conditions and at 
the freezing point for freezing conditions.


\vspace{0.5in}
{\small
\begin{center}
\begin{longtable}{| p{2.0in} || p{4.0in} |}
    \hline
    {\bf Name} & {\bf Description} \endfirsthead
    \hline 
    {\bf Name} & {\bf Description} (Continued) \endhead
    \hline
    \hline
    \hyperref[subsec:nm_sec_config_land_ice_flux_mode]{config\_land\_ice\_flux\_mode} & Selects the mode in which land-ice fluxes are computed. \\
    \hline
    \hyperref[subsec:nm_sec_config_land_ice_flux_formulation]{config\_land\_ice\_flux\_\-formulation} & Name of land-ice flux formulation. \\
    \hline
    \hyperref[subsec:nm_sec_config_land_ice_flux_useHollandJenkinsAdvDiff]{config\_land\_ice\_flux\_use\-HollandJenkinsAdvDiff} & If .true. then uses the advection/diffusion scheme of Holland and Jenkins (1999) for ice-shelf heat fluxes \\
    \hline
    \hyperref[subsec:nm_sec_config_land_ice_flux_attenuation_coefficient]{config\_land\_ice\_flux\_\-attenuation\_coefficient} & The vertical length scale of exponential decay for surface fluxes under land ice. \\
    \hline
    \hyperref[subsec:nm_sec_config_land_ice_flux_boundaryLayerThickness]{config\_land\_ice\_flux\_boundary\-LayerThickness} & The thickness of the sub-ice-shelf boundary layer, over which T and S will be averaged. \\
    \hline
    \hyperref[subsec:nm_sec_config_land_ice_flux_boundaryLayerNeighborWeight]{config\_land\_ice\_flux\_boundary\-LayerNeighborWeight} & The for horizontal neighbors used to horizontally smooth boundary layer T and S. \\
    \hline
    \hyperref[subsec:nm_sec_config_land_ice_flux_cp_ice]{config\_land\_ice\_flux\_cp\_ice} & The specific heat capacity for ice. \\
    \hline
    \hyperref[subsec:nm_sec_config_land_ice_flux_rho_ice]{config\_land\_ice\_flux\_rho\_ice} & The density of land ice. \\
    \hline
    \hyperref[subsec:nm_sec_config_land_ice_flux_topDragCoeff]{config\_land\_ice\_flux\_topDrag\-Coeff} & The top drag coefficient. \\
    \hline
    \hyperref[subsec:nm_sec_config_land_ice_flux_ISOMIP_gammaT]{config\_land\_ice\_flux\_\-ISOMIP\_gammaT} & The constant heat transport velocity through the boundary layer under land ice used in the ISOMIP test cases. \\
    \hline
    \hyperref[subsec:nm_sec_config_land_ice_flux_rms_tidal_velocity]{config\_land\_ice\_flux\_rms\_\-tidal\_velocity} & Parameterization of tidal velocity used in computing the sub-ice-shelf friction velocity \\
    \hline
    \hyperref[subsec:nm_sec_config_land_ice_flux_jenkins_heat_transfer_coefficient]{config\_land\_ice\_flux\_jenkins\_\-heat\_transfer\_coefficient} & constant nondimensional heat transfer coefficient across the ice-ocean boundary layer \\
    \hline
    \hyperref[subsec:nm_sec_config_land_ice_flux_jenkins_salt_transfer_coefficient]{config\_land\_ice\_flux\_jenkins\_\-salt\_transfer\_coefficient} & constant nondimensional salt transfer coefficient across the ice-ocean boundary layer \\
    \hline
\end{longtable}
\end{center}
}
\section[advection]{\hyperref[sec:nm_sec_advection]{advection}}
\label{sec:nm_tab_advection}
Three-dimensional tracer advection can be computed using 2$^{nd}$, 3$^{rd}$ or 4$^{th}$ flux reconstructions in the horizontal and vertical. In the horizontal, the high-order (i.e. 3$^{rd}$ or 4$^{th}$) flux reconstruction is done following \cite{Skamarock:2011tc}. Typically, the scheme is implemented with an upwind-bias ($\beta$=0.25 in (11) from \cite{Skamarock:2011tc}) to produce a 3$^{rd}$-order accurate reconstruction of tracer flux divergence on uniform hexagonal meshes. In the vertical, high-order estimates of tracer values at layer edges are reconstructed using a cubic spline. Monotone transport is guaranteed by blending these high-order flux approximations with the 1$^{st}$-order, upstream flux using the \cite{Zalesak:1979wm} flux-corrected transport scheme.
\vspace{0.5in}
{\small
\begin{center}
\begin{longtable}{| p{2.0in} || p{4.0in} |}
    \hline
    {\bf Name} & {\bf Description} \endfirsthead
    \hline 
    {\bf Name} & {\bf Description} (Continued) \endhead
    \hline
    \hline
    \hyperref[subsec:nm_sec_config_vert_tracer_adv]{config\_vert\_tracer\_adv} & Method for interpolating tracer values from layer centers to layer edges \\
    \hline
    \hyperref[subsec:nm_sec_config_vert_tracer_adv_order]{config\_vert\_tracer\_adv\_order} & Order of polynomial used for tracer reconstruction at layer edges \\
    \hline
    \hyperref[subsec:nm_sec_config_horiz_tracer_adv_order]{config\_horiz\_tracer\_adv\_order} & Order of polynomial used for tracer reconstruction at cell edges \\
    \hline
    \hyperref[subsec:nm_sec_config_coef_3rd_order]{config\_coef\_3rd\_order} & Reconstruction of 3rd-order reconstruction to blend with 4th-order reconstuction \\
    \hline
    \hyperref[subsec:nm_sec_config_monotonic]{config\_monotonic} & If .true. then fluxes are limited to produce a monotonic advection scheme \\
    \hline
\end{longtable}
\end{center}
}
\section[bottom\_drag]{\hyperref[sec:nm_sec_bottom_drag]{bottom\_drag}}
\label{sec:nm_tab_bottom_drag}
The bottom drag is applied as a bottom boundary condition within the implicit solve of vertical mixing in the momentum equation (\ref{ocean:momentum}), as
\begin{equation}
\lim_{z\rightarrow z_{bot}} \nu_v \frac{\partial u}{\partial z} = c_{drag} \left|u\right| u,
\end{equation}
where $c_{drag}$ is the dimensionless bottom drag coefficient, and $z_{bot}$ is the $z$-location of the ocean bottom.

\vspace{0.5in}
{\small
\begin{center}
\begin{longtable}{| p{2.0in} || p{4.0in} |}
    \hline
    {\bf Name} & {\bf Description} \endfirsthead
    \hline 
    {\bf Name} & {\bf Description} (Continued) \endhead
    \hline
    \hline
    \hyperref[subsec:nm_sec_config_use_implicit_bottom_drag]{config\_use\_implicit\_bottom\_\-drag} & If true, implicit bottom drag is used on the momentum equation. \\
    \hline
    \hyperref[subsec:nm_sec_config_implicit_bottom_drag_coeff]{config\_implicit\_bottom\_drag\_\-coeff} & Dimensionless bottom drag coefficient, $c_{drag}$. \\
    \hline
    \hyperref[subsec:nm_sec_config_use_explicit_bottom_drag]{config\_use\_explicit\_bottom\_\-drag} & If true, explicit bottom drag is used on the momentum equation. \\
    \hline
    \hyperref[subsec:nm_sec_config_explicit_bottom_drag_coeff]{config\_explicit\_bottom\_drag\_\-coeff} & Dimensionless bottom drag coefficient, $c_{drag}$. \\
    \hline
\end{longtable}
\end{center}
}
\section[ocean\_constants]{\hyperref[sec:nm_sec_ocean_constants]{ocean\_constants}}
\label{sec:nm_tab_ocean_constants}
\vspace{0.5in}
{\small
\begin{center}
\begin{longtable}{| p{2.0in} || p{4.0in} |}
    \hline
    {\bf Name} & {\bf Description} \endfirsthead
    \hline 
    {\bf Name} & {\bf Description} (Continued) \endhead
    \hline
    \hline
    \hyperref[subsec:nm_sec_config_density0]{config\_density0} & Density used as a coefficient of the pressure gradient terms, $\rho_0$. This is a constant due to the Boussinesq approximation. \\
    \hline
\end{longtable}
\end{center}
}
\section[pressure\_gradient]{\hyperref[sec:nm_sec_pressure_gradient]{pressure\_gradient}}
\label{sec:nm_tab_pressure_gradient}
For most choices of the vertical coordinate, the pressure gradient terms in the momentum equation will have the form
\begin{equation}
\label{ocean:grad p}
- \frac{1}{\rho_0}\nabla p - \frac{\rho g}{\rho_0}\nabla z^{mid}.
\end{equation}
For isopycnal vertical coordinates, the user may choose to use the Montgomery potential,
\begin{equation}
\label{ocean:Montgomery Potential}
M = \frac{1}{\rho}p+gz
\end{equation}
and replace the pressure terms above with
\begin{equation}
- \nabla M.
\end{equation}
See \citet[section 2.1]{Higdon05jcp} for details on the derivation and computation of the Montgomery potential.

\vspace{0.5in}
{\small
\begin{center}
\begin{longtable}{| p{2.0in} || p{4.0in} |}
    \hline
    {\bf Name} & {\bf Description} \endfirsthead
    \hline 
    {\bf Name} & {\bf Description} (Continued) \endhead
    \hline
    \hline
    \hyperref[subsec:nm_sec_config_pressure_gradient_type]{config\_pressure\_gradient\_type} & Form of pressure gradient terms in momentum equation. For most applications, the gradient of pressure and layer mid-depth are appropriate.  For isopycnal coordinates, one may use the gradient of the Montgomery potential. \\
    \hline
    \hyperref[subsec:nm_sec_config_common_level_weight]{config\_common\_level\_weight} & The weight between standard Jacobian and weighted Jacobian, $\gamma$. \\
    \hline
\end{longtable}
\end{center}
}
\section[eos]{\hyperref[sec:nm_sec_eos]{eos}}
\label{sec:nm_tab_eos}
Two forms of EOS are supported. The full EOS from \cite{Jackett_McDougall95jaot} and a linear EOS. When using the full EOS, options are available to compute density after an adiabatic displacement of the particle to a different vertical level.

\vspace{0.5in}
{\small
\begin{center}
\begin{longtable}{| p{2.0in} || p{4.0in} |}
    \hline
    {\bf Name} & {\bf Description} \endfirsthead
    \hline 
    {\bf Name} & {\bf Description} (Continued) \endhead
    \hline
    \hline
    \hyperref[subsec:nm_sec_config_eos_type]{config\_eos\_type} & Character string to choose EOS formulation \\
    \hline
    \hyperref[subsec:nm_sec_config_open_ocean_freezing_temperature_coeff_0]{config\_open\_ocean\_freezing\_\-temperature\_coeff\_0} & The freezing temperature at zero pressure in open ocean. \\
    \hline
    \hyperref[subsec:nm_sec_config_open_ocean_freezing_temperature_coeff_S]{config\_open\_ocean\_freezing\_\-temperature\_coeff\_S} & The coefficient for the term proportional to salinity in the freezing temperature in the open ocean. \\
    \hline
    \hyperref[subsec:nm_sec_config_open_ocean_freezing_temperature_coeff_p]{config\_open\_ocean\_freezing\_\-temperature\_coeff\_p} & The coefficient for the term proportional to the (limited) pressure in the freezing temperature in the open ocean. \\
    \hline
    \hyperref[subsec:nm_sec_config_open_ocean_freezing_temperature_coeff_pS]{config\_open\_ocean\_freezing\_\-temperature\_coeff\_pS} & The coefficient for the term proportional to salinity times pressure in the freezing temperature in the open ocean. \\
    \hline
    \hyperref[subsec:nm_sec_config_open_ocean_freezing_temperature_reference_pressure]{config\_open\_ocean\_freezing\_\-temperature\_reference\_pressure} & The reference pressure above which the freezing temperature is equal to config\_freezing\_temperature\_coeff\_0 in the open ocean. \\
    \hline
    \hyperref[subsec:nm_sec_config_land_ice_cavity_freezing_temperature_coeff_0]{config\_land\_ice\_cavity\_\-freezing\_temperature\_coeff\_0} & The freezing temperature at zero pressure in land-ice cavities. \\
    \hline
    \hyperref[subsec:nm_sec_config_land_ice_cavity_freezing_temperature_coeff_S]{config\_land\_ice\_cavity\_\-freezing\_temperature\_coeff\_S} & The coefficient for the term proportional to salinity in the freezing temperature in land-ice cavities. \\
    \hline
    \hyperref[subsec:nm_sec_config_land_ice_cavity_freezing_temperature_coeff_p]{config\_land\_ice\_cavity\_\-freezing\_temperature\_coeff\_p} & The coefficient for the term proportional to the (limited) pressure in the freezing temperature in land-ice cavities. \\
    \hline
    \hyperref[subsec:nm_sec_config_land_ice_cavity_freezing_temperature_coeff_pS]{config\_land\_ice\_cavity\_\-freezing\_temperature\_coeff\_pS} & The coefficient for the term proportional to salinity times pressure in the freezing temperature in land-ice cavities. \\
    \hline
    \hyperref[subsec:nm_sec_config_land_ice_cavity_freezing_temperature_reference_pressure]{config\_land\_ice\_cavity\_\-freezing\_temperature\_reference\_\-pressure} & The reference pressure above which the freezing temperature is equal to config\_freezing\_temperature\_coeff\_0 in land-ice cavities. \\
    \hline
\end{longtable}
\end{center}
}
\section[eos\_linear]{\hyperref[sec:nm_sec_eos_linear]{eos\_linear}}
\label{sec:nm_tab_eos_linear}
The linear equation of state (leos) is specified as follows:
\begin{equation}
\rho = \rho_{ref} - \alpha_{leos}(T-T_{ref})+\beta_{leos}(S-S_{ref})
\end{equation}
\vspace{0.5in}
{\small
\begin{center}
\begin{longtable}{| p{2.0in} || p{4.0in} |}
    \hline
    {\bf Name} & {\bf Description} \endfirsthead
    \hline 
    {\bf Name} & {\bf Description} (Continued) \endhead
    \hline
    \hline
    \hyperref[subsec:nm_sec_config_eos_linear_alpha]{config\_eos\_linear\_alpha} & Linear thermal expansion coefficient \\
    \hline
    \hyperref[subsec:nm_sec_config_eos_linear_beta]{config\_eos\_linear\_beta} & Linear haline contraction coefficient \\
    \hline
    \hyperref[subsec:nm_sec_config_eos_linear_Tref]{config\_eos\_linear\_Tref} & Reference temperature \\
    \hline
    \hyperref[subsec:nm_sec_config_eos_linear_Sref]{config\_eos\_linear\_Sref} & Reference salinity \\
    \hline
    \hyperref[subsec:nm_sec_config_eos_linear_densityref]{config\_eos\_linear\_densityref} & Reference density, i.e. density when T=Tref and S=Sref \\
    \hline
\end{longtable}
\end{center}
}
\section[split\_explicit\_ts]{\hyperref[sec:nm_sec_split_explicit_ts]{split\_explicit\_ts}}
\label{sec:nm_tab_split_explicit_ts}
The split explicit time-stepping method solves the barotropic (vertically-integrated) velocities separately from the remaining baroclinic velocities.  The time step for the barotropic solve is limited by fast surface gravity waves, and so is subcycled within a large timestep of the baroclinic velocity solve.  This provides a 10 to 12-times speed-up over fourth-order Runge-Kutta time stepping.

A single large timestep in the split explicit algorithm may be summarized as
\begin{itemize}
\item Stage 1: solve for baroclinic velocity (3D)
\item Stage 2: solve for barotropic velocity (2D) with explicit sub-cycling
\item Stage 3: update thickness, tracers, density and pressure
\end{itemize}
The algorithm includes iterations within stage 1, within each subcycle of stage 2, and over the full three-stage process.  Further details are provided in \citet[Appendix A.5]{Ringler_ea13om}

\vspace{0.5in}
{\small
\begin{center}
\begin{longtable}{| p{2.0in} || p{4.0in} |}
    \hline
    {\bf Name} & {\bf Description} \endfirsthead
    \hline 
    {\bf Name} & {\bf Description} (Continued) \endhead
    \hline
    \hline
    \hyperref[subsec:nm_sec_config_n_ts_iter]{config\_n\_ts\_iter} & number of large iterations over stages 1-3 \\
    \hline
    \hyperref[subsec:nm_sec_config_n_bcl_iter_beg]{config\_n\_bcl\_iter\_beg} & number of iterations of stage 1 (baroclinic solve) on the first split-explicit iteration \\
    \hline
    \hyperref[subsec:nm_sec_config_n_bcl_iter_mid]{config\_n\_bcl\_iter\_mid} & number of iterations of stage 1 (baroclinic solve) on any split-explicit iterations between first and last \\
    \hline
    \hyperref[subsec:nm_sec_config_n_bcl_iter_end]{config\_n\_bcl\_iter\_end} & number of iterations of stage 1 (baroclinic solve) on the last split-explicit iteration \\
    \hline
    \hyperref[subsec:nm_sec_config_btr_dt]{config\_btr\_dt} & Timestep to use for the barotropic mode in the split explicit time integrator \\
    \hline
    \hyperref[subsec:nm_sec_config_n_btr_cor_iter]{config\_n\_btr\_cor\_iter} & number of iterations of the velocity corrector step in stage 2 \\
    \hline
    \hyperref[subsec:nm_sec_config_vel_correction]{config\_vel\_correction} & If true, the velocity correction term is included in the horizontal advection of thickness and tracers \\
    \hline
    \hyperref[subsec:nm_sec_config_btr_subcycle_loop_factor]{config\_btr\_subcycle\_loop\_\-factor} & Barotropic subcycles proceed from $t$ to $t+n\Delta t$, where $n$ is this configuration option. \\
    \hline
    \hyperref[subsec:nm_sec_config_btr_gam1_velWt1]{config\_btr\_gam1\_velWt1} & Weighting of velocity in the SSH predictor step in stage 2. When zero, previous subcycle time is used; when one, new subcycle time is used. \\
    \hline
    \hyperref[subsec:nm_sec_config_btr_gam2_SSHWt1]{config\_btr\_gam2\_SSHWt1} & Weighting of SSH in the velocity corrector step in stage 2. When zero, previous subcycle time is used; when one, new subcycle time is used. \\
    \hline
    \hyperref[subsec:nm_sec_config_btr_gam3_velWt2]{config\_btr\_gam3\_velWt2} & Weighting of velocity in the SSH corrector step in stage 2. When zero, previous subcycle time is used; when one, new subcycle time is used. \\
    \hline
    \hyperref[subsec:nm_sec_config_btr_solve_SSH2]{config\_btr\_solve\_SSH2} & If true, execute the SSH corrector step in stage 2 \\
    \hline
\end{longtable}
\end{center}
}
\section[testing]{\hyperref[sec:nm_sec_testing]{testing}}
\label{sec:nm_tab_testing}
Upon start-up, a series of tests may be run to confirm normal operations.  Some tests may require specific domains in order to produce reasonable error statistics.  Available tests include the following.

\begin{enumerate}
\item {\bf Tensor test} verifies subroutines that compute the gradient of a vector and the divergence of a tensor.  Differences between computed and analytic solutions are printed to the log file.  All test functions beginning with 'sph' should be run on a spherical domain without land, and all others should be run on a plane periodic Cartesian domain.  The test has the following workflow:
\begin{itemize}
\item compute tangential velocity at each edge, $v_e$
\item compute strain rate tensor at a cell center, $\sigma_i=\left[\dot\epsilon  \right]_i=\nabla {\bf u}$
\item interpolate strain rate tensor to edge, $\left[\dot\epsilon  \right]_e$
\item compute divergence of tensor at cell center, $\left[\nabla \cdot \sigma\right]_i$
\item interpolate to edge, $\left[\nabla \cdot \sigma\right]_i$
\item compute normal and tangential components, ${\bf n}_e\cdot[\nabla \cdot \sigma]_{e}$
\item compute rms of difference between computed and analytic solution, when available for that test case.
\end{itemize}
\end{enumerate}

\vspace{0.5in}
{\small
\begin{center}
\begin{longtable}{| p{2.0in} || p{4.0in} |}
    \hline
    {\bf Name} & {\bf Description} \endfirsthead
    \hline 
    {\bf Name} & {\bf Description} (Continued) \endhead
    \hline
    \hline
    \hyperref[subsec:nm_sec_config_conduct_tests]{config\_conduct\_tests} & If true, run testing suite. This is the overarching control on the test suite. Individual flags must be set to true below to conduct each test. \\
    \hline
    \hyperref[subsec:nm_sec_config_test_tensors]{config\_test\_tensors} & If true, tensor operations are tested upon start-up. \\
    \hline
    \hyperref[subsec:nm_sec_config_tensor_test_function]{config\_tensor\_test\_function} & Character string to choose tensor test fuction \\
    \hline
\end{longtable}
\end{center}
}
\section[debug]{\hyperref[sec:nm_sec_debug]{debug}}
\label{sec:nm_tab_debug}
At run-time a user can enable debugging features within MPAS-Land Ice. 
Currently the only debug option is to print more detailed information about
thickness advection.
Potential future debug options would be to include disabling of any 
tendencies to help determine why an issue might
be happening; various checks on certain fields;
and the ability to prescribe both a thickness and velocity field at run-time
which are constant throughout a simulation. All options that control these
debugging features are specified within the debug namelist record.

\vspace{0.5in}
{\small
\begin{center}
\begin{longtable}{| p{2.0in} || p{4.0in} |}
    \hline
    {\bf Name} & {\bf Description} \endfirsthead
    \hline 
    {\bf Name} & {\bf Description} (Continued) \endhead
    \hline
    \hline
    \hyperref[subsec:nm_sec_config_disable_redi_k33]{config\_disable\_redi\_k33} & If true, disables k33 portion of Redi neutral surface mixing. \\
    \hline
    \hyperref[subsec:nm_sec_config_disable_redi_horizontal_term1]{config\_disable\_redi\_\-horizontal\_term1} & If true, disables first term in horizonal mixing of Redi neutral surface mixing. \\
    \hline
    \hyperref[subsec:nm_sec_config_disable_redi_horizontal_term2]{config\_disable\_redi\_\-horizontal\_term2} & If true, disables first term in horizonal mixing of Redi neutral surface mixing. \\
    \hline
    \hyperref[subsec:nm_sec_config_disable_redi_horizontal_term3]{config\_disable\_redi\_\-horizontal\_term3} & If true, disables first term in horizonal mixing of Redi neutral surface mixing. \\
    \hline
    \hyperref[subsec:nm_sec_config_check_zlevel_consistency]{config\_check\_zlevel\_consistency} & Enables a run-time check for consistency for a zlevel grid. Ensures relevant variables correctly define the bottom of the ocean. \\
    \hline
    \hyperref[subsec:nm_sec_config_check_ssh_consistency]{config\_check\_ssh\_consistency} & Enables a run-time check to determine if the SSH is within 2m of the surface.  See equation for $\zeta_i$. \\
    \hline
    \hyperref[subsec:nm_sec_config_filter_btr_mode]{config\_filter\_btr\_mode} & Enables filtering of the barotropic mode. \\
    \hline
    \hyperref[subsec:nm_sec_config_prescribe_velocity]{config\_prescribe\_velocity} & Enables a prescribed velocity field. This velocity field is read on input, and remains constant through a simulation. \\
    \hline
    \hyperref[subsec:nm_sec_config_prescribe_thickness]{config\_prescribe\_thickness} & Enables a prescribed thickness field. This thickness field is read on input, and remains constant through a simulation. \\
    \hline
    \hyperref[subsec:nm_sec_config_include_KE_vertex]{config\_include\_KE\_vertex} & If true, the kinetic energy in each cell is computed by blending cell-based and vertex-based values of kinetic energy. \\
    \hline
    \hyperref[subsec:nm_sec_config_check_tracer_monotonicity]{config\_check\_tracer\_\-monotonicity} & Enables a change on tracer monotonicity at the end of the monotonic advection routine. Only used if config\_monotonic is set to .true. \\
    \hline
    \hyperref[subsec:nm_sec_config_compute_active_tracer_budgets]{config\_compute\_active\_tracer\_\-budgets} & Enables the computation of tracer budget terms \\
    \hline
    \hyperref[subsec:nm_sec_config_disable_thick_all_tend]{config\_disable\_thick\_all\_tend} & Disables all tendencies on the thickness field. \\
    \hline
    \hyperref[subsec:nm_sec_config_disable_thick_hadv]{config\_disable\_thick\_hadv} & Disable tendencies on the thickness field from horizontal advection. \\
    \hline
    \hyperref[subsec:nm_sec_config_disable_thick_vadv]{config\_disable\_thick\_vadv} & Disables tendencies on the thickness field from vertical advection. \\
    \hline
    \hyperref[subsec:nm_sec_config_disable_thick_sflux]{config\_disable\_thick\_sflux} & Disables tendencies on the thickness field from surface fluxes. \\
    \hline
    \hyperref[subsec:nm_sec_config_disable_vel_all_tend]{config\_disable\_vel\_all\_tend} & Disables all tendencies on the velocity field. \\
    \hline
    \hyperref[subsec:nm_sec_config_disable_vel_coriolis]{config\_disable\_vel\_coriolis} & Diables tendencies on the velocity field from the Coriolis force. \\
    \hline
    \hyperref[subsec:nm_sec_config_disable_vel_pgrad]{config\_disable\_vel\_pgrad} & Disables tendencies on the velocity field from the horizontal pressure gradient. \\
    \hline
    \hyperref[subsec:nm_sec_config_disable_vel_hmix]{config\_disable\_vel\_hmix} & Disables tendencies on the velocity field from horizontal mixing. \\
    \hline
    \hyperref[subsec:nm_sec_config_disable_vel_surface_stress]{config\_disable\_vel\_surface\_\-stress} & Disables tendencies on the velocity field from horizontal surface stresses (e.g. wind stress and top drag). \\
    \hline
    \hyperref[subsec:nm_sec_config_disable_vel_explicit_bottom_drag]{config\_disable\_vel\_explicit\_\-bottom\_drag} & Disables tendencies on the velocity field from explicit bottom drag \\
    \hline
    \hyperref[subsec:nm_sec_config_disable_vel_vmix]{config\_disable\_vel\_vmix} & Disables tendencies on the velocity field from vertical mixing. \\
    \hline
    \hyperref[subsec:nm_sec_config_disable_vel_vadv]{config\_disable\_vel\_vadv} & Disables tendencies on the velocity field from vertical advection. \\
    \hline
    \hyperref[subsec:nm_sec_config_disable_tr_all_tend]{config\_disable\_tr\_all\_tend} & Disables all tendencies on tracer fields. \\
    \hline
    \hyperref[subsec:nm_sec_config_disable_tr_adv]{config\_disable\_tr\_adv} & Disables tendencies on tracer fields from advection, both horizontal and vertical. \\
    \hline
    \hyperref[subsec:nm_sec_config_disable_tr_hmix]{config\_disable\_tr\_hmix} & Disables tendencies on tracer fields from horizontal mixing. \\
    \hline
    \hyperref[subsec:nm_sec_config_disable_tr_vmix]{config\_disable\_tr\_vmix} & Disables tendencies on tracer fields from vertical mixing. \\
    \hline
    \hyperref[subsec:nm_sec_config_disable_tr_sflux]{config\_disable\_tr\_sflux} & Disables tendencies on tracer fields from surface fluxes. \\
    \hline
    \hyperref[subsec:nm_sec_config_disable_tr_nonlocalflux]{config\_disable\_tr\_nonlocalflux} & Disables tendencies on the tracer fields from CVMix/KPP nonlocal fluxes. \\
    \hline
    \hyperref[subsec:nm_sec_config_read_nearest_restart]{config\_read\_nearest\_restart} & This flag is intended for the expert user.  If false, forward model will error out if time given by config\_start\_time (or Restart\_timestamp file if config\_start\_time='file') does not match any xtime strings in the restart file.  If true, forward model will read in record with xtime nearest to config\_start\_time.  Note that the restart file name is still given by config\_start\_time (or Restart\_timestamp file), regardless of the state of this flag. \\
    \hline
\end{longtable}
\end{center}
}
\section[constrain\_Haney\_number]{\hyperref[sec:nm_sec_constrain_Haney_number]{constrain\_Haney\_number}}
\label{sec:nm_tab_constrain_Haney_number}
\vspace{0.5in}
{\small
\begin{center}
\begin{longtable}{| p{2.0in} || p{4.0in} |}
    \hline
    {\bf Name} & {\bf Description} \endfirsthead
    \hline 
    {\bf Name} & {\bf Description} (Continued) \endhead
    \hline
    \hline
    \hyperref[subsec:nm_sec_config_use_rx1_constraint]{config\_use\_rx1\_constraint} & Initialize using Haney number constraint under ice shelves \\
    \hline
    \hyperref[subsec:nm_sec_config_rx1_outer_iter_count]{config\_rx1\_outer\_iter\_count} & The number of outer iterations (first smoothing then rx1 constraint) during initialization of the vertical grid. \\
    \hline
    \hyperref[subsec:nm_sec_config_rx1_inner_iter_count]{config\_rx1\_inner\_iter\_count} & The number of iterations used to constrain rx1 in each layer. \\
    \hline
    \hyperref[subsec:nm_sec_config_rx1_init_inner_weight]{config\_rx1\_init\_inner\_weight} & The weight by which layer thicknesses are altered at the beginning of inner iteration. This weight linearly increases to 1.0 by the final iteration. \\
    \hline
    \hyperref[subsec:nm_sec_config_rx1_max]{config\_rx1\_max} & The maximum value rx1Max of the Haney number (rx1) after modification of the vertical grid \\
    \hline
    \hyperref[subsec:nm_sec_config_rx1_horiz_smooth_weight]{config\_rx1\_horiz\_smooth\_\-weight} & Relative weight of horizontal neighbors compared to this cell when smoothing vertical stretching \\
    \hline
    \hyperref[subsec:nm_sec_config_rx1_vert_smooth_weight]{config\_rx1\_vert\_smooth\_weight} & Relative weight of vertical neighbors compared to this cell when smoothing vertical stretching \\
    \hline
    \hyperref[subsec:nm_sec_config_rx1_slope_weight]{config\_rx1\_slope\_weight} & Weight used to nudge level interfaces toward being flat (thus decreasing the Haney number) \\
    \hline
    \hyperref[subsec:nm_sec_config_rx1_zstar_weight]{config\_rx1\_zstar\_weight} & Weight used to nudge vertical stretching toward z-star during each outer iteration \\
    \hline
    \hyperref[subsec:nm_sec_config_rx1_horiz_smooth_open_ocean_cells]{config\_rx1\_horiz\_smooth\_\-open\_ocean\_cells} & The size (in cells) of a buffer region around land ice for smoothing.  Smoothing is performed under land ice and in the buffer region of open ocean. \\
    \hline
    \hyperref[subsec:nm_sec_config_rx1_min_levels]{config\_rx1\_min\_levels} & The minimum number of layers in the ocean column in the smoothed region. \\
    \hline
    \hyperref[subsec:nm_sec_config_rx1_min_layer_thickness]{config\_rx1\_min\_layer\_thickness} & The minimum layer thickness in the smoothed region. \\
    \hline
\end{longtable}
\end{center}
}
\section[baroclinic\_channel]{\hyperref[sec:nm_sec_baroclinic_channel]{baroclinic\_channel}}
\label{sec:nm_tab_baroclinic_channel}
\vspace{0.5in}
{\small
\begin{center}
\begin{longtable}{| p{2.0in} || p{4.0in} |}
    \hline
    {\bf Name} & {\bf Description} \endfirsthead
    \hline 
    {\bf Name} & {\bf Description} (Continued) \endhead
    \hline
    \hline
    \hyperref[subsec:nm_sec_config_baroclinic_channel_vert_levels]{config\_baroclinic\_channel\_\-vert\_levels} & Number of vertical levels in baroclinic channel test case. Typical value is 20. \\
    \hline
    \hyperref[subsec:nm_sec_config_baroclinic_channel_use_distances]{config\_baroclinic\_channel\_use\_\-distances} & Logical flag that determines if locations of features are defined by distances of fractions. False means fractions. \\
    \hline
    \hyperref[subsec:nm_sec_config_baroclinic_channel_surface_temperature]{config\_baroclinic\_channel\_\-surface\_temperature} & Temperature of the surface in the northern half of the domain. \\
    \hline
    \hyperref[subsec:nm_sec_config_baroclinic_channel_bottom_temperature]{config\_baroclinic\_channel\_\-bottom\_temperature} & Temperature of the bottom in the northern half of the domain. \\
    \hline
    \hyperref[subsec:nm_sec_config_baroclinic_channel_temperature_difference]{config\_baroclinic\_channel\_\-temperature\_difference} & Difference in the temperature field between the northern and southern halves of the domain. \\
    \hline
    \hyperref[subsec:nm_sec_config_baroclinic_channel_gradient_width_frac]{config\_baroclinic\_channel\_\-gradient\_width\_frac} & Fraction of domain in Y direction the temperature gradient should be linear over. \\
    \hline
    \hyperref[subsec:nm_sec_config_baroclinic_channel_gradient_width_dist]{config\_baroclinic\_channel\_\-gradient\_width\_dist} & Width of the temperature gradient around the center sin wave. Default value is relative to a 500km domain in Y. \\
    \hline
    \hyperref[subsec:nm_sec_config_baroclinic_channel_bottom_depth]{config\_baroclinic\_channel\_\-bottom\_depth} & Depth of the bottom of the ocean for the baroclinic channel test case. \\
    \hline
    \hyperref[subsec:nm_sec_config_baroclinic_channel_salinity]{config\_baroclinic\_channel\_\-salinity} & Salinity of the water in the entire domain. \\
    \hline
    \hyperref[subsec:nm_sec_config_baroclinic_channel_coriolis_parameter]{config\_baroclinic\_channel\_\-coriolis\_parameter} & Coriolis parameter for entrie domain. \\
    \hline
\end{longtable}
\end{center}
}
\section[lock\_exchange]{\hyperref[sec:nm_sec_lock_exchange]{lock\_exchange}}
\label{sec:nm_tab_lock_exchange}
\vspace{0.5in}
{\small
\begin{center}
\begin{longtable}{| p{2.0in} || p{4.0in} |}
    \hline
    {\bf Name} & {\bf Description} \endfirsthead
    \hline 
    {\bf Name} & {\bf Description} (Continued) \endhead
    \hline
    \hline
    \hyperref[subsec:nm_sec_config_lock_exchange_vert_levels]{config\_lock\_exchange\_vert\_\-levels} & Number of vertical levels in lock exchange test case. Typical value is 20. \\
    \hline
    \hyperref[subsec:nm_sec_config_lock_exchange_bottom_depth]{config\_lock\_exchange\_bottom\_\-depth} & Depth of the bottom of the ocean in the lock exchange test case. \\
    \hline
    \hyperref[subsec:nm_sec_config_lock_exchange_cold_temperature]{config\_lock\_exchange\_cold\_\-temperature} & Temperature of water in the cold half of the domain. \\
    \hline
    \hyperref[subsec:nm_sec_config_lock_exchange_warm_temperature]{config\_lock\_exchange\_warm\_\-temperature} & Temperature of water in the warm half of the domain. \\
    \hline
    \hyperref[subsec:nm_sec_config_lock_exchange_direction]{config\_lock\_exchange\_direction} & If y, warm/cold changes in the y-direction.  If z, warm/cold changes in z-direction. \\
    \hline
    \hyperref[subsec:nm_sec_config_lock_exchange_salinity]{config\_lock\_exchange\_salinity} & Salinity of the water in the entire domain. \\
    \hline
    \hyperref[subsec:nm_sec_config_lock_exchange_layer_type]{config\_lock\_exchange\_layer\_\-type} & Vertical grid type \\
    \hline
    \hyperref[subsec:nm_sec_config_lock_exchange_isopycnal_min_thickness]{config\_lock\_exchange\_\-isopycnal\_min\_thickness} & minimum layer thickness for isopycnal case \\
    \hline
\end{longtable}
\end{center}
}
\section[internal\_waves]{\hyperref[sec:nm_sec_internal_waves]{internal\_waves}}
\label{sec:nm_tab_internal_waves}
\vspace{0.5in}
{\small
\begin{center}
\begin{longtable}{| p{2.0in} || p{4.0in} |}
    \hline
    {\bf Name} & {\bf Description} \endfirsthead
    \hline 
    {\bf Name} & {\bf Description} (Continued) \endhead
    \hline
    \hline
    \hyperref[subsec:nm_sec_config_internal_waves_vert_levels]{config\_internal\_waves\_vert\_\-levels} & Number of vertical levels in internal waves test case. Typical value is 20. \\
    \hline
    \hyperref[subsec:nm_sec_config_internal_waves_use_distances]{config\_internal\_waves\_use\_\-distances} & Logical flag that determines if locations of features are defined by distances of fractions. False means fractions. \\
    \hline
    \hyperref[subsec:nm_sec_config_internal_waves_surface_temperature]{config\_internal\_waves\_surface\_\-temperature} & Temperature of the surface in the northern half of the domain. \\
    \hline
    \hyperref[subsec:nm_sec_config_internal_waves_bottom_temperature]{config\_internal\_waves\_bottom\_\-temperature} & Temperature of the bottom in the northern half of the domain. \\
    \hline
    \hyperref[subsec:nm_sec_config_internal_waves_temperature_difference]{config\_internal\_waves\_\-temperature\_difference} & Maximum temperature difference in the amplitude. \\
    \hline
    \hyperref[subsec:nm_sec_config_internal_waves_amplitude_width_frac]{config\_internal\_waves\_\-amplitude\_width\_frac} & Percent of domain in Y direction the initial amplitude should exist over. \\
    \hline
    \hyperref[subsec:nm_sec_config_internal_waves_amplitude_width_dist]{config\_internal\_waves\_\-amplitude\_width\_dist} & Width in Y direction the initial amplitude should exist over. Default is relative to a 250km domain. \\
    \hline
    \hyperref[subsec:nm_sec_config_internal_waves_bottom_depth]{config\_internal\_waves\_bottom\_\-depth} & Depth of the bottom of the ocean for the internal waves test case. \\
    \hline
    \hyperref[subsec:nm_sec_config_internal_waves_salinity]{config\_internal\_waves\_salinity} & Salinity of the water in the entire domain. \\
    \hline
    \hyperref[subsec:nm_sec_config_internal_waves_layer_type]{config\_internal\_waves\_layer\_\-type} & Logical flag that controls how the initial conditions should be generated. \\
    \hline
    \hyperref[subsec:nm_sec_config_internal_waves_isopycnal_displacement]{config\_internal\_waves\_\-isopycnal\_displacement} & Max distance isopycnal layers are displaced upwards. \\
    \hline
\end{longtable}
\end{center}
}
\section[overflow]{\hyperref[sec:nm_sec_overflow]{overflow}}
\label{sec:nm_tab_overflow}
\vspace{0.5in}
{\small
\begin{center}
\begin{longtable}{| p{2.0in} || p{4.0in} |}
    \hline
    {\bf Name} & {\bf Description} \endfirsthead
    \hline 
    {\bf Name} & {\bf Description} (Continued) \endhead
    \hline
    \hline
    \hyperref[subsec:nm_sec_config_overflow_vert_levels]{config\_overflow\_vert\_levels} & Number of vertical levels in overflow test case. Typical values are 40 and 100. \\
    \hline
    \hyperref[subsec:nm_sec_config_overflow_use_distances]{config\_overflow\_use\_distances} & Logical flag that determines if locations of features are defined by distances of fractions. False means fractions. \\
    \hline
    \hyperref[subsec:nm_sec_config_overflow_bottom_depth]{config\_overflow\_bottom\_depth} & Depth of the bottom of the ocean in the overflow test case. \\
    \hline
    \hyperref[subsec:nm_sec_config_overflow_ridge_depth]{config\_overflow\_ridge\_depth} & Depth of the bottom of the ocean on the ridge in the over flow test case. \\
    \hline
    \hyperref[subsec:nm_sec_config_overflow_plug_temperature]{config\_overflow\_plug\_\-temperature} & Temperature of water in plug at the southern end of the domain. \\
    \hline
    \hyperref[subsec:nm_sec_config_overflow_domain_temperature]{config\_overflow\_domain\_\-temperature} & Temperature of water outside of the plug. \\
    \hline
    \hyperref[subsec:nm_sec_config_overflow_salinity]{config\_overflow\_salinity} & Salinity of the water in the entire domain. \\
    \hline
    \hyperref[subsec:nm_sec_config_overflow_plug_width_frac]{config\_overflow\_plug\_width\_\-frac} & Fraction of the domain the plug should take up initially. Only in the y direction. \\
    \hline
    \hyperref[subsec:nm_sec_config_overflow_slope_center_frac]{config\_overflow\_slope\_center\_\-frac} & Location of the center of the slope. Given as a fraction of the total y domain range. Position is relative to the minimum y value. \\
    \hline
    \hyperref[subsec:nm_sec_config_overflow_slope_width_frac]{config\_overflow\_slope\_width\_\-frac} & Half width of the slope. Given as a fraction of the total y domain range. \\
    \hline
    \hyperref[subsec:nm_sec_config_overflow_plug_width_dist]{config\_overflow\_plug\_width\_\-dist} & Distance from the minimum Y value of the domain the plug should take up initially. Default is relative to a 200km domain. \\
    \hline
    \hyperref[subsec:nm_sec_config_overflow_slope_center_dist]{config\_overflow\_slope\_center\_\-dist} & Location of the center of the slope. Given as a distance from the minimum y value. Default is relative to a 200km domain. \\
    \hline
    \hyperref[subsec:nm_sec_config_overflow_slope_width_dist]{config\_overflow\_slope\_width\_\-dist} & Half width of the slope. Default is relative to a 200km domain. \\
    \hline
    \hyperref[subsec:nm_sec_config_overflow_layer_type]{config\_overflow\_layer\_type} & Logical flag that controls how the initial conditions should be generated. \\
    \hline
    \hyperref[subsec:nm_sec_config_overflow_isopycnal_min_thickness]{config\_overflow\_isopycnal\_\-min\_thickness} & minimum layer thickness \\
    \hline
\end{longtable}
\end{center}
}
\section[global\_ocean]{\hyperref[sec:nm_sec_global_ocean]{global\_ocean}}
\label{sec:nm_tab_global_ocean}
\vspace{0.5in}
{\small
\begin{center}
\begin{longtable}{| p{2.0in} || p{4.0in} |}
    \hline
    {\bf Name} & {\bf Description} \endfirsthead
    \hline 
    {\bf Name} & {\bf Description} (Continued) \endhead
    \hline
    \hline
    \hyperref[subsec:nm_sec_config_global_ocean_minimum_depth]{config\_global\_ocean\_\-minimum\_depth} & Minimum depth where column contains all water-filled cells.  The first layer with refBottomDepth greater than this value is included in whole, i.e. no partial bottom cells are used in this layer. \\
    \hline
    \hyperref[subsec:nm_sec_config_global_ocean_depth_file]{config\_global\_ocean\_depth\_file} & Path to the depth initial condition file. \\
    \hline
    \hyperref[subsec:nm_sec_config_global_ocean_depth_dimname]{config\_global\_ocean\_depth\_\-dimname} & Name of the dimension defining the number of vertical levels in input files. \\
    \hline
    \hyperref[subsec:nm_sec_config_global_ocean_depth_varname]{config\_global\_ocean\_depth\_\-varname} & Name of the variable containing mid-depth of levels in temperature and salinity initial condition files. \\
    \hline
    \hyperref[subsec:nm_sec_config_global_ocean_depth_conversion_factor]{config\_global\_ocean\_depth\_\-conversion\_factor} & Conversion factor for depth levels. Should convert units on input depth levels to meters. \\
    \hline
    \hyperref[subsec:nm_sec_config_global_ocean_temperature_file]{config\_global\_ocean\_\-temperature\_file} & Path to the temperature initial condition file. Must be interpolated to vertical layers defined in depth file. \\
    \hline
    \hyperref[subsec:nm_sec_config_global_ocean_salinity_file]{config\_global\_ocean\_salinity\_\-file} & Path to the salinity initial condition file. Must be interpolated to vertical layers defined in depth file. \\
    \hline
    \hyperref[subsec:nm_sec_config_global_ocean_tracer_nlat_dimname]{config\_global\_ocean\_tracer\_\-nlat\_dimname} & Name of the dimension that determines number of latitude lines in tracer initial condition files. \\
    \hline
    \hyperref[subsec:nm_sec_config_global_ocean_tracer_nlon_dimname]{config\_global\_ocean\_tracer\_\-nlon\_dimname} & Name of the dimension that determines number of longitude lines in tracer initial condition files. \\
    \hline
    \hyperref[subsec:nm_sec_config_global_ocean_tracer_ndepth_dimname]{config\_global\_ocean\_tracer\_\-ndepth\_dimname} & Name of the dimension that determines number of vertical levels in tracer initial condition files. \\
    \hline
    \hyperref[subsec:nm_sec_config_global_ocean_tracer_depth_conversion_factor]{config\_global\_ocean\_tracer\_\-depth\_conversion\_factor} & Conversion factor for tracer initial condition depth levels. Should convert units on input depth levels to meters. \\
    \hline
    \hyperref[subsec:nm_sec_config_global_ocean_tracer_vert_levels]{config\_global\_ocean\_tracer\_\-vert\_levels} & Number of vertical levels in tracer initial condition file.  Set to -1 to read from file with config\_global\_ocean\_tracer\_ndepth\_dimname \\
    \hline
    \hyperref[subsec:nm_sec_config_global_ocean_temperature_varname]{config\_global\_ocean\_\-temperature\_varname} & Name of the variable containing temperature in temperature initial condition file. \\
    \hline
    \hyperref[subsec:nm_sec_config_global_ocean_salinity_varname]{config\_global\_ocean\_salinity\_\-varname} & Name of the variable containing salinity in salinity initial condition file. \\
    \hline
    \hyperref[subsec:nm_sec_config_global_ocean_tracer_latlon_degrees]{config\_global\_ocean\_tracer\_\-latlon\_degrees} & Logical flag that controls if the Lat/Lon fields for tracers should be converted to radians. True means input is degrees, false means input is radians. \\
    \hline
    \hyperref[subsec:nm_sec_config_global_ocean_tracer_lat_varname]{config\_global\_ocean\_tracer\_\-lat\_varname} & Name of the variable containing latitude coordinates for tracer values in temperature initial condition file. \\
    \hline
    \hyperref[subsec:nm_sec_config_global_ocean_tracer_lon_varname]{config\_global\_ocean\_tracer\_\-lon\_varname} & Name of the variable containing longitude coordinates for tracer values in temperature initial condition file. \\
    \hline
    \hyperref[subsec:nm_sec_config_global_ocean_tracer_depth_varname]{config\_global\_ocean\_tracer\_\-depth\_varname} & Name of the variable containing depth coordinates for tracer values in temperature initial condition file. \\
    \hline
    \hyperref[subsec:nm_sec_config_global_ocean_tracer_method]{config\_global\_ocean\_tracer\_\-method} & Method to interpolate tracer data to MPAS mesh. \\
    \hline
    \hyperref[subsec:nm_sec_config_global_ocean_smooth_TS_iterations]{config\_global\_ocean\_smooth\_\-TS\_iterations} & Number of smoothing iterations on temperature and salinity. \\
    \hline
    \hyperref[subsec:nm_sec_config_global_ocean_swData_file]{config\_global\_ocean\_swData\_\-file} & Name of the file containing shortwaveData (chlA, zenith Angle, clear sky radiation) \\
    \hline
    \hyperref[subsec:nm_sec_config_global_ocean_swData_nlat_dimname]{config\_global\_ocean\_swData\_\-nlat\_dimname} & Name of the dimension that determines number of latitude lines in swData initial condition files. \\
    \hline
    \hyperref[subsec:nm_sec_config_global_ocean_swData_nlon_dimname]{config\_global\_ocean\_swData\_\-nlon\_dimname} & Name of the dimension that determines number of longitude lines in swData initial condition files. \\
    \hline
    \hyperref[subsec:nm_sec_config_global_ocean_swData_lat_varname]{config\_global\_ocean\_swData\_\-lat\_varname} & Name of the variable containing latitude coordinates for swData values in swData initial condition file. \\
    \hline
    \hyperref[subsec:nm_sec_config_global_ocean_swData_lon_varname]{config\_global\_ocean\_swData\_\-lon\_varname} & Name of the variable containing longitude coordinates for swData values in swData initial condition file. \\
    \hline
    \hyperref[subsec:nm_sec_config_global_ocean_swData_latlon_degrees]{config\_global\_ocean\_swData\_\-latlon\_degrees} & Logical flag that controls if the Lat/Lon fields for swData should be converted to radians. True means input is degrees, false means input is radians. \\
    \hline
    \hyperref[subsec:nm_sec_config_global_ocean_swData_method]{config\_global\_ocean\_swData\_\-method} & Method to interpolate shortwave data to MPAS mesh. \\
    \hline
    \hyperref[subsec:nm_sec_config_global_ocean_chlorophyll_varname]{config\_global\_ocean\_\-chlorophyll\_varname} & Name of the variable containing chlorophyll in sw Data initial condition file. \\
    \hline
    \hyperref[subsec:nm_sec_config_global_ocean_zenithAngle_varname]{config\_global\_ocean\_zenith\-Angle\_varname} & Name of the variable containing zenith angle in swData initial condition file. \\
    \hline
    \hyperref[subsec:nm_sec_config_global_ocean_clearSky_varname]{config\_global\_ocean\_clearSky\_\-varname} & Name of the variable containing clear sky radiation in clear sky radiation initial condition file. \\
    \hline
    \hyperref[subsec:nm_sec_config_global_ocean_piston_velocity]{config\_global\_ocean\_piston\_\-velocity} & Parameter controlling rate to which SST and SST are restored. \\
    \hline
    \hyperref[subsec:nm_sec_config_global_ocean_interior_restore_rate]{config\_global\_ocean\_interior\_\-restore\_rate} & Parameter controlling rate to which interior temperature and salinity are restored. \\
    \hline
    \hyperref[subsec:nm_sec_config_global_ocean_topography_file]{config\_global\_ocean\_\-topography\_file} & Path to the topography initial condition file. \\
    \hline
    \hyperref[subsec:nm_sec_config_global_ocean_topography_nlat_dimname]{config\_global\_ocean\_\-topography\_nlat\_dimname} & Dimension name for the latitude in the topography file. \\
    \hline
    \hyperref[subsec:nm_sec_config_global_ocean_topography_nlon_dimname]{config\_global\_ocean\_\-topography\_nlon\_dimname} & Dimension name for the longitude in the topography file. \\
    \hline
    \hyperref[subsec:nm_sec_config_global_ocean_topography_latlon_degrees]{config\_global\_ocean\_\-topography\_latlon\_degrees} & Logical flag that controls if the Lat/Lon fields for topography should be converted to radians. True means input is degrees, false means input is radians. \\
    \hline
    \hyperref[subsec:nm_sec_config_global_ocean_topography_lat_varname]{config\_global\_ocean\_\-topography\_lat\_varname} & Variable name for the latitude in the topography file. \\
    \hline
    \hyperref[subsec:nm_sec_config_global_ocean_topography_lon_varname]{config\_global\_ocean\_\-topography\_lon\_varname} & Variable name for the longitude in the topography file. \\
    \hline
    \hyperref[subsec:nm_sec_config_global_ocean_topography_varname]{config\_global\_ocean\_\-topography\_varname} & Variable name for the topography in the topography file. \\
    \hline
    \hyperref[subsec:nm_sec_config_global_ocean_topography_has_ocean_frac]{config\_global\_ocean\_\-topography\_has\_ocean\_frac} & Logical flag that controls if topograhy file contains a field for the fraction of each cell that contains ocean (vs. land or grounded ice). \\
    \hline
    \hyperref[subsec:nm_sec_config_global_ocean_topography_ocean_frac_varname]{config\_global\_ocean\_\-topography\_ocean\_frac\_\-varname} & Variable name for the ocean mask in the topography file. \\
    \hline
    \hyperref[subsec:nm_sec_config_global_ocean_topography_method]{config\_global\_ocean\_\-topography\_method} & Method to interpolate topography data to MPAS mesh. \\
    \hline
    \hyperref[subsec:nm_sec_config_global_ocean_smooth_topography]{config\_global\_ocean\_smooth\_\-topography} & Logical flag that controls if topograhy should be smoothed after being interpolated. \\
    \hline
    \hyperref[subsec:nm_sec_config_global_ocean_deepen_critical_passages]{config\_global\_ocean\_deepen\_\-critical\_passages} & Logical flag that controls if critical passages are deepened to a minimum depth. \\
    \hline
    \hyperref[subsec:nm_sec_config_global_ocean_depress_by_land_ice]{config\_global\_ocean\_depress\_\-by\_land\_ice} & Logical flag that controls if sea surface pressure and layer thicknesses should be altered by an overlying ice sheet/shelf. \\
    \hline
    \hyperref[subsec:nm_sec_config_global_ocean_land_ice_topo_file]{config\_global\_ocean\_land\_ice\_\-topo\_file} & Path to the land ice topography initial condition file. \\
    \hline
    \hyperref[subsec:nm_sec_config_global_ocean_land_ice_topo_nlat_dimname]{config\_global\_ocean\_land\_ice\_\-topo\_nlat\_dimname} & Dimension name for the latitude in the land ice topography file. \\
    \hline
    \hyperref[subsec:nm_sec_config_global_ocean_land_ice_topo_nlon_dimname]{config\_global\_ocean\_land\_ice\_\-topo\_nlon\_dimname} & Dimension name for the longitude in the land ice topography file. \\
    \hline
    \hyperref[subsec:nm_sec_config_global_ocean_land_ice_topo_latlon_degrees]{config\_global\_ocean\_land\_ice\_\-topo\_latlon\_degrees} & Logical flag that controls if the Lat/Lon fields for land ice topography should be converted to radians. True means input is degrees, false means input is radians. \\
    \hline
    \hyperref[subsec:nm_sec_config_global_ocean_land_ice_topo_lat_varname]{config\_global\_ocean\_land\_ice\_\-topo\_lat\_varname} & Variable name for the latitude in the land ice topography file. \\
    \hline
    \hyperref[subsec:nm_sec_config_global_ocean_land_ice_topo_lon_varname]{config\_global\_ocean\_land\_ice\_\-topo\_lon\_varname} & Variable name for the longitude in the land ice topography file. \\
    \hline
    \hyperref[subsec:nm_sec_config_global_ocean_land_ice_topo_thickness_varname]{config\_global\_ocean\_land\_ice\_\-topo\_thickness\_varname} & Variable name for the land ice thickness in the land ice topography file. \\
    \hline
    \hyperref[subsec:nm_sec_config_global_ocean_land_ice_topo_draft_varname]{config\_global\_ocean\_land\_ice\_\-topo\_draft\_varname} & Variable name for the land ice draft in the land ice topography file. \\
    \hline
    \hyperref[subsec:nm_sec_config_global_ocean_land_ice_topo_ice_frac_varname]{config\_global\_ocean\_land\_ice\_\-topo\_ice\_frac\_varname} & Variable name for the land ice fraction in the land ice topography file. \\
    \hline
    \hyperref[subsec:nm_sec_config_global_ocean_land_ice_topo_grounded_frac_varname]{config\_global\_ocean\_land\_ice\_\-topo\_grounded\_frac\_varname} & Variable name for the grounded land ice fraction in the land ice topography file. \\
    \hline
    \hyperref[subsec:nm_sec_config_global_ocean_use_constant_land_ice_cavity_temperature]{config\_global\_ocean\_use\_\-constant\_land\_ice\_cavity\_\-temperature} & Logical flag that controls if ocean temperature in land-ice cavities is set to a constant temperature. \\
    \hline
    \hyperref[subsec:nm_sec_config_global_ocean_constant_land_ice_cavity_temperature]{config\_global\_ocean\_constant\_\-land\_ice\_cavity\_temperature} & The constant temperature value to be used under land ice, typically something close to the freezing point. \\
    \hline
    \hyperref[subsec:nm_sec_config_global_ocean_cull_inland_seas]{config\_global\_ocean\_cull\_\-inland\_seas} & Logical flag that controls if inland seas should be removed. \\
    \hline
    \hyperref[subsec:nm_sec_config_global_ocean_windstress_file]{config\_global\_ocean\_\-windstress\_file} & Path to the windstress initial condition file. \\
    \hline
    \hyperref[subsec:nm_sec_config_global_ocean_windstress_nlat_dimname]{config\_global\_ocean\_\-windstress\_nlat\_dimname} & Dimension name for the latitude in the windstress file. \\
    \hline
    \hyperref[subsec:nm_sec_config_global_ocean_windstress_nlon_dimname]{config\_global\_ocean\_\-windstress\_nlon\_dimname} & Dimension name for the longitude in the windstress file. \\
    \hline
    \hyperref[subsec:nm_sec_config_global_ocean_windstress_latlon_degrees]{config\_global\_ocean\_\-windstress\_latlon\_degrees} & Logical flag that controls if the Lat/Lon fields for windstress should be converted to radians. True means input is degrees, false means input is radians. \\
    \hline
    \hyperref[subsec:nm_sec_config_global_ocean_windstress_lat_varname]{config\_global\_ocean\_\-windstress\_lat\_varname} & Variable name for the latitude in the windstress file. \\
    \hline
    \hyperref[subsec:nm_sec_config_global_ocean_windstress_lon_varname]{config\_global\_ocean\_\-windstress\_lon\_varname} & Variable name for the longitude in the windstress file. \\
    \hline
    \hyperref[subsec:nm_sec_config_global_ocean_windstress_zonal_varname]{config\_global\_ocean\_\-windstress\_zonal\_varname} & Variable name for the zonal component of windstress in the windstress file. \\
    \hline
    \hyperref[subsec:nm_sec_config_global_ocean_windstress_meridional_varname]{config\_global\_ocean\_\-windstress\_meridional\_varname} & Variable name for the meridional component of windstress in the windstress file. \\
    \hline
    \hyperref[subsec:nm_sec_config_global_ocean_windstress_method]{config\_global\_ocean\_\-windstress\_method} & Method to interpolate windstress data to MPAS mesh. \\
    \hline
    \hyperref[subsec:nm_sec_config_global_ocean_windstress_conversion_factor]{config\_global\_ocean\_\-windstress\_conversion\_factor} & Factor to convert input windstress to $N$ $m^{-1}$ \\
    \hline
    \hyperref[subsec:nm_sec_config_global_ocean_ecosys_file]{config\_global\_ocean\_ecosys\_file} & Name of file containing global values of ecosys variables \\
    \hline
    \hyperref[subsec:nm_sec_config_global_ocean_ecosys_forcing_file]{config\_global\_ocean\_ecosys\_\-forcing\_file} & Name of file containing global values of ecosys forcing fields \\
    \hline
    \hyperref[subsec:nm_sec_config_global_ocean_ecosys_nlat_dimname]{config\_global\_ocean\_ecosys\_\-nlat\_dimname} & Name of the dimension that determines number of latitude lines in ecosys initial condition files. \\
    \hline
    \hyperref[subsec:nm_sec_config_global_ocean_ecosys_nlon_dimname]{config\_global\_ocean\_ecosys\_\-nlon\_dimname} & Name of the dimension that determines number of longitude lines in ecosys initial condition files. \\
    \hline
    \hyperref[subsec:nm_sec_config_global_ocean_ecosys_ndepth_dimname]{config\_global\_ocean\_ecosys\_\-ndepth\_dimname} & Name of the dimension that determines number of vertical levels in ecosys initial condition files. \\
    \hline
    \hyperref[subsec:nm_sec_config_global_ocean_ecosys_depth_conversion_factor]{config\_global\_ocean\_ecosys\_\-depth\_conversion\_factor} & Conversion factor for ecosys initial condition depth levels. Should convert units on input depth levels to meters. \\
    \hline
    \hyperref[subsec:nm_sec_config_global_ocean_ecosys_vert_levels]{config\_global\_ocean\_ecosys\_\-vert\_levels} & Number of vertical levels in ecosys initial condition file.  Set to -1 to read from file with config\_global\_ocean\_ecosys\_ndepth\_dimname \\
    \hline
    \hyperref[subsec:nm_sec_config_global_ocean_ecosys_lat_varname]{config\_global\_ocean\_ecosys\_\-lat\_varname} & Name of the variable containing latitude coordinates for ecosys values in ecosys initial condition file. \\
    \hline
    \hyperref[subsec:nm_sec_config_global_ocean_ecosys_lon_varname]{config\_global\_ocean\_ecosys\_\-lon\_varname} & Name of the variable containing longitude coordinates for ecosys values in ecosys initial condition file. \\
    \hline
    \hyperref[subsec:nm_sec_config_global_ocean_ecosys_depth_varname]{config\_global\_ocean\_ecosys\_\-depth\_varname} & Name of the variable containing depth coordinates for ecosys values in ecosys initial condition file. \\
    \hline
    \hyperref[subsec:nm_sec_config_global_ocean_ecosys_latlon_degrees]{config\_global\_ocean\_ecosys\_\-latlon\_degrees} & Logical flag that controls if the Lat/Lon fields for ecosys should be converted to radians. True means input is degrees, false means input is radians. \\
    \hline
    \hyperref[subsec:nm_sec_config_global_ocean_ecosys_method]{config\_global\_ocean\_ecosys\_\-method} & Method to interpolate shortwave data to MPAS mesh. \\
    \hline
    \hyperref[subsec:nm_sec_config_global_ocean_ecosys_forcing_time_dimname]{config\_global\_ocean\_ecosys\_\-forcing\_time\_dimname} & Name of the dimension that determines the times in ecosys forcing files. \\
    \hline
    \hyperref[subsec:nm_sec_config_global_ocean_smooth_ecosys_iterations]{config\_global\_ocean\_smooth\_\-ecosys\_iterations} & Number of smoothing iterations on ecosystem variables. \\
    \hline
\end{longtable}
\end{center}
}
\section[cvmix\_WSwSBF]{\hyperref[sec:nm_sec_cvmix_WSwSBF]{cvmix\_WSwSBF}}
\label{sec:nm_tab_cvmix_WSwSBF}
\vspace{0.5in}
{\small
\begin{center}
\begin{longtable}{| p{2.0in} || p{4.0in} |}
    \hline
    {\bf Name} & {\bf Description} \endfirsthead
    \hline 
    {\bf Name} & {\bf Description} (Continued) \endhead
    \hline
    \hline
    \hyperref[subsec:nm_sec_config_cvmix_WSwSBF_vert_levels]{config\_cvmix\_WSwSBF\_vert\_\-levels} & Number of vertical levels in cvmix WSwSBF unit test case. \\
    \hline
    \hyperref[subsec:nm_sec_config_cvmix_WSwSBF_surface_temperature]{config\_cvmix\_WSwSBF\_\-surface\_temperature} & Temperature of the surface of the ocean. \\
    \hline
    \hyperref[subsec:nm_sec_config_cvmix_WSwSBF_surface_salinity]{config\_cvmix\_WSwSBF\_\-surface\_salinity} & Salinity of the surface of the ocean. \\
    \hline
    \hyperref[subsec:nm_sec_config_cvmix_WSwSBF_surface_restoring_temperature]{config\_cvmix\_WSwSBF\_\-surface\_restoring\_temperature} & Temperature to restore towards when surface restoring is turned on. \\
    \hline
    \hyperref[subsec:nm_sec_config_cvmix_WSwSBF_surface_restoring_salinity]{config\_cvmix\_WSwSBF\_\-surface\_restoring\_salinity} & Salinity to restore towards when surface restoring is turned on. \\
    \hline
    \hyperref[subsec:nm_sec_config_cvmix_WSwSBF_temperature_piston_velocity]{config\_cvmix\_WSwSBF\_\-temperature\_piston\_velocity} & Piston velocity to control rate of restoring toward config\_cvmix\_WSwSBF\_surface\_restoring\_temperature. \\
    \hline
    \hyperref[subsec:nm_sec_config_cvmix_WSwSBF_salinity_piston_velocity]{config\_cvmix\_WSwSBF\_\-salinity\_piston\_velocity} & Piston velocity to control rate of restoring toward config\_cvmix\_WSwSBF\_surface\_restoring\_salinity. \\
    \hline
    \hyperref[subsec:nm_sec_config_cvmix_WSwSBF_sensible_heat_flux]{config\_cvmix\_WSwSBF\_\-sensible\_heat\_flux} & Net sensible heat flux applied when bulk forcing is used. Positive values indicate a net input of heat to ocean. \\
    \hline
    \hyperref[subsec:nm_sec_config_cvmix_WSwSBF_latent_heat_flux]{config\_cvmix\_WSwSBF\_\-latent\_heat\_flux} & Net latent heat flux applied when bulk forcing is used. Positive values indicate a net input of heat to ocean. \\
    \hline
    \hyperref[subsec:nm_sec_config_cvmix_WSwSBF_shortwave_heat_flux]{config\_cvmix\_WSwSBF\_\-shortwave\_heat\_flux} & Net solar shortwave heat flux applied when bulk forcing is used. Positive values indicate a net input of heat to ocean. \\
    \hline
    \hyperref[subsec:nm_sec_config_cvmix_WSwSBF_rain_flux]{config\_cvmix\_WSwSBF\_rain\_\-flux} & Net surface rain flux when bulk forcing is used. Positive values indicate a net input of water to ocean. \\
    \hline
    \hyperref[subsec:nm_sec_config_cvmix_WSwSBF_evaporation_flux]{config\_cvmix\_WSwSBF\_\-evaporation\_flux} & Net surface evaporation when bulk forcing is used. Positive values indicate a net input of water to ocean. \\
    \hline
    \hyperref[subsec:nm_sec_config_cvmix_WSwSBF_interior_temperature_restoring_rate]{config\_cvmix\_WSwSBF\_\-interior\_temperature\_restoring\_\-rate} & Rate at which temperature is restored toward the initial condition. \\
    \hline
    \hyperref[subsec:nm_sec_config_cvmix_WSwSBF_interior_salinity_restoring_rate]{config\_cvmix\_WSwSBF\_\-interior\_salinity\_restoring\_rate} & Rate at which salinity is restored toward the initial condition. \\
    \hline
    \hyperref[subsec:nm_sec_config_cvmix_WSwSBF_temperature_gradient]{config\_cvmix\_WSwSBF\_\-temperature\_gradient} & d/dz of temperature. \\
    \hline
    \hyperref[subsec:nm_sec_config_cvmix_WSwSBF_salinity_gradient]{config\_cvmix\_WSwSBF\_\-salinity\_gradient} & d/dz of salinity. \\
    \hline
    \hyperref[subsec:nm_sec_config_cvmix_WSwSBF_temperature_gradient_mixed_layer]{config\_cvmix\_WSwSBF\_\-temperature\_gradient\_mixed\_\-layer} & d/dz of temperature in mixed temperature layer \\
    \hline
    \hyperref[subsec:nm_sec_config_cvmix_WSwSBF_salinity_gradient_mixed_layer]{config\_cvmix\_WSwSBF\_\-salinity\_gradient\_mixed\_layer} & d/dz of salinity in mixed salinity layer \\
    \hline
    \hyperref[subsec:nm_sec_config_cvmix_WSwSBF_mixed_layer_depth_temperature]{config\_cvmix\_WSwSBF\_\-mixed\_layer\_depth\_\-temperature} & depth mixed temperature layer \\
    \hline
    \hyperref[subsec:nm_sec_config_cvmix_WSwSBF_mixed_layer_depth_salinity]{config\_cvmix\_WSwSBF\_\-mixed\_layer\_depth\_salinity} & depth mixed salinity layer \\
    \hline
    \hyperref[subsec:nm_sec_config_cvmix_WSwSBF_mixed_layer_temperature_change]{config\_cvmix\_WSwSBF\_\-mixed\_layer\_temperature\_\-change} & temperature jump when moving downward across the mixed layer interface \\
    \hline
    \hyperref[subsec:nm_sec_config_cvmix_WSwSBF_mixed_layer_salinity_change]{config\_cvmix\_WSwSBF\_\-mixed\_layer\_salinity\_change} & salinity jump when moving downward across the mixed layer interface \\
    \hline
    \hyperref[subsec:nm_sec_config_cvmix_WSwSBF_vertical_grid]{config\_cvmix\_WSwSBF\_\-vertical\_grid} & prescription of setting the vertical resolution of the test case \\
    \hline
    \hyperref[subsec:nm_sec_config_cvmix_WSwSBF_bottom_depth]{config\_cvmix\_WSwSBF\_\-bottom\_depth} & Depth of the bottom of the ocean for the CVMix WSwSBF unit test case. \\
    \hline
    \hyperref[subsec:nm_sec_config_cvmix_WSwSBF_max_windstress]{config\_cvmix\_WSwSBF\_max\_\-windstress} & Maximum surface windstress over the domain. \\
    \hline
    \hyperref[subsec:nm_sec_config_cvmix_WSwSBF_coriolis_parameter]{config\_cvmix\_WSwSBF\_\-coriolis\_parameter} & Coriolis parameter for WSwSBF test case \\
    \hline
\end{longtable}
\end{center}
}
\section[iso]{\hyperref[sec:nm_sec_iso]{iso}}
\label{sec:nm_tab_iso}
\vspace{0.5in}
{\small
\begin{center}
\begin{longtable}{| p{2.0in} || p{4.0in} |}
    \hline
    {\bf Name} & {\bf Description} \endfirsthead
    \hline 
    {\bf Name} & {\bf Description} (Continued) \endhead
    \hline
    \hline
    \hyperref[subsec:nm_sec_config_iso_vert_levels]{config\_iso\_vert\_levels} & Number of vertical levels in ISO. \\
    \hline
    \hyperref[subsec:nm_sec_config_iso_main_channel_depth]{config\_iso\_main\_channel\_\-depth} & Depth of the main channel in the ISO. \\
    \hline
    \hyperref[subsec:nm_sec_config_iso_north_wall_lat]{config\_iso\_north\_wall\_lat} & Latitude of the vertical north wall in the ISO domain. \\
    \hline
    \hyperref[subsec:nm_sec_config_iso_south_wall_lat]{config\_iso\_south\_wall\_lat} & Latitude of the top of the main channel south wall wall in the ISO domain. \\
    \hline
    \hyperref[subsec:nm_sec_config_iso_ridge_flag]{config\_iso\_ridge\_flag} & Logical flag that controls if a ridge is used or not. \\
    \hline
    \hyperref[subsec:nm_sec_config_iso_ridge_center_lon]{config\_iso\_ridge\_center\_lon} & Longitude of the center of the ridge in the ISO. \\
    \hline
    \hyperref[subsec:nm_sec_config_iso_ridge_height]{config\_iso\_ridge\_height} & Maximum height of the ridge at the zonal middle of the ISO domain. \\
    \hline
    \hyperref[subsec:nm_sec_config_iso_ridge_width]{config\_iso\_ridge\_width} & Width of the ridge at the zonal middle of the ISO domain. \\
    \hline
    \hyperref[subsec:nm_sec_config_iso_plateau_flag]{config\_iso\_plateau\_flag} & Logical flag that controls if a plateau is used or not. \\
    \hline
    \hyperref[subsec:nm_sec_config_iso_plateau_center_lon]{config\_iso\_plateau\_center\_lon} & Longitude of the center of the plateau in the ISO. \\
    \hline
    \hyperref[subsec:nm_sec_config_iso_plateau_center_lat]{config\_iso\_plateau\_center\_lat} & Latitude of the center of the plateau in the ISO. \\
    \hline
    \hyperref[subsec:nm_sec_config_iso_plateau_height]{config\_iso\_plateau\_height} & Height of the top of the plateau in the ISO domain. \\
    \hline
    \hyperref[subsec:nm_sec_config_iso_plateau_radius]{config\_iso\_plateau\_radius} & Radius at the top of the plateau in the ISO domain. \\
    \hline
    \hyperref[subsec:nm_sec_config_iso_plateau_slope_width]{config\_iso\_plateau\_slope\_\-width} & Width of the sloping region of the plateau in the ISO domain. \\
    \hline
    \hyperref[subsec:nm_sec_config_iso_shelf_flag]{config\_iso\_shelf\_flag} & Logical flag that controls if a shelf is used or not. \\
    \hline
    \hyperref[subsec:nm_sec_config_iso_shelf_depth]{config\_iso\_shelf\_depth} & Depth of the shelf in the ISO. \\
    \hline
    \hyperref[subsec:nm_sec_config_iso_shelf_width]{config\_iso\_shelf\_width} & Width of the shelf in the ISO. \\
    \hline
    \hyperref[subsec:nm_sec_config_iso_cont_slope_flag]{config\_iso\_cont\_slope\_flag} & Logical flag that controls if a continental slope is used or not. \\
    \hline
    \hyperref[subsec:nm_sec_config_iso_max_cont_slope]{config\_iso\_max\_cont\_slope} & Maximum slope of the continental slope in the ISO. \\
    \hline
    \hyperref[subsec:nm_sec_config_iso_embayment_flag]{config\_iso\_embayment\_flag} & Logical flag that controls if an embayment is used or not. \\
    \hline
    \hyperref[subsec:nm_sec_config_iso_embayment_center_lon]{config\_iso\_embayment\_center\_\-lon} & Longitude of the center of the embayment in the ISO. \\
    \hline
    \hyperref[subsec:nm_sec_config_iso_embayment_center_lat]{config\_iso\_embayment\_center\_\-lat} & Latitude of the center of the embayment in the ISO. \\
    \hline
    \hyperref[subsec:nm_sec_config_iso_embayment_radius]{config\_iso\_embayment\_radius} & Radius of the embayment in the ISO. \\
    \hline
    \hyperref[subsec:nm_sec_config_iso_embayment_depth]{config\_iso\_embayment\_depth} & Depth of the embayment in the ISO. \\
    \hline
    \hyperref[subsec:nm_sec_config_iso_depression_flag]{config\_iso\_depression\_flag} & Logical flag to add a depresseion between embayment and main channel. \\
    \hline
    \hyperref[subsec:nm_sec_config_iso_depression_center_lon]{config\_iso\_depression\_center\_\-lon} & Longitude of the center of the depression in the ISO. \\
    \hline
    \hyperref[subsec:nm_sec_config_iso_depression_south_lat]{config\_iso\_depression\_south\_\-lat} & Latitude of the south end of the depression in the ISO. \\
    \hline
    \hyperref[subsec:nm_sec_config_iso_depression_north_lat]{config\_iso\_depression\_north\_\-lat} & Latitude of the north end of the depression in the ISO. \\
    \hline
    \hyperref[subsec:nm_sec_config_iso_depression_width]{config\_iso\_depression\_width} & Width of the depression in the ISO. \\
    \hline
    \hyperref[subsec:nm_sec_config_iso_depression_depth]{config\_iso\_depression\_depth} & Depth of the depression in the ISO. \\
    \hline
    \hyperref[subsec:nm_sec_config_iso_salinity]{config\_iso\_salinity} & Salinity of the water in the ISO. \\
    \hline
    \hyperref[subsec:nm_sec_config_iso_wind_stress_max]{config\_iso\_wind\_stress\_max} & Maximum zonal windstress value. \\
    \hline
    \hyperref[subsec:nm_sec_config_iso_acc_wind]{config\_iso\_acc\_wind} & Maximum zonal windstress value over the Antarctic Circumpolar Current. \\
    \hline
    \hyperref[subsec:nm_sec_config_iso_asf_wind]{config\_iso\_asf\_wind} & Maximum zonal windstress value over the Antarctic Slope Front. \\
    \hline
    \hyperref[subsec:nm_sec_config_iso_wind_trans]{config\_iso\_wind\_trans} & Latitude of the transition region between easterly and westerly winds. \\
    \hline
    \hyperref[subsec:nm_sec_config_iso_heat_flux_south]{config\_iso\_heat\_flux\_south} & Heat flux into the ocean over the south side of the main channel. \\
    \hline
    \hyperref[subsec:nm_sec_config_iso_heat_flux_middle]{config\_iso\_heat\_flux\_middle} & Heat flux into the ocean over the middle of the main channel. \\
    \hline
    \hyperref[subsec:nm_sec_config_iso_heat_flux_north]{config\_iso\_heat\_flux\_north} & Heat flux into the ocean over the north side of the main channel. \\
    \hline
    \hyperref[subsec:nm_sec_config_iso_heat_flux_lat_ss]{config\_iso\_heat\_flux\_lat\_ss} & Latitude of southern point of heat flux region on the south. \\
    \hline
    \hyperref[subsec:nm_sec_config_iso_heat_flux_lat_sm]{config\_iso\_heat\_flux\_lat\_sm} & Latitude of transition point between heat flux regions on the south and middle. \\
    \hline
    \hyperref[subsec:nm_sec_config_iso_heat_flux_lat_mn]{config\_iso\_heat\_flux\_lat\_mn} & Latitude of transition point between heat flux regions on the middel and north. \\
    \hline
    \hyperref[subsec:nm_sec_config_iso_region1_center_lon]{config\_iso\_region1\_center\_lon} & Longitude of center region 1. \\
    \hline
    \hyperref[subsec:nm_sec_config_iso_region1_center_lat]{config\_iso\_region1\_center\_lat} & Latitude of center of region 1. \\
    \hline
    \hyperref[subsec:nm_sec_config_iso_region2_center_lon]{config\_iso\_region2\_center\_lon} & Longitude of center of region 2. \\
    \hline
    \hyperref[subsec:nm_sec_config_iso_region2_center_lat]{config\_iso\_region2\_center\_lat} & Latitude of center of region 2. \\
    \hline
    \hyperref[subsec:nm_sec_config_iso_region3_center_lon]{config\_iso\_region3\_center\_lon} & Longitude of center of region 3. \\
    \hline
    \hyperref[subsec:nm_sec_config_iso_region3_center_lat]{config\_iso\_region3\_center\_lat} & Latitude of center of region 3. \\
    \hline
    \hyperref[subsec:nm_sec_config_iso_region4_center_lon]{config\_iso\_region4\_center\_lon} & Longitude of center of region 4. \\
    \hline
    \hyperref[subsec:nm_sec_config_iso_region4_center_lat]{config\_iso\_region4\_center\_lat} & Latitude of center of region 2. \\
    \hline
    \hyperref[subsec:nm_sec_config_iso_heat_flux_region1_flag]{config\_iso\_heat\_flux\_region1\_\-flag} & Logical flag controlling use of heat flux in region 1. \\
    \hline
    \hyperref[subsec:nm_sec_config_iso_heat_flux_region1]{config\_iso\_heat\_flux\_region1} & Heat flux into of the ocean over a localized region 1. \\
    \hline
    \hyperref[subsec:nm_sec_config_iso_heat_flux_region1_radius]{config\_iso\_heat\_flux\_region1\_\-radius} & Radius of heat flux localized region 1. \\
    \hline
    \hyperref[subsec:nm_sec_config_iso_heat_flux_region2_flag]{config\_iso\_heat\_flux\_region2\_\-flag} & Logical flag controlling use of heat flux in region 2. \\
    \hline
    \hyperref[subsec:nm_sec_config_iso_heat_flux_region2]{config\_iso\_heat\_flux\_region2} & Heat flux into of the ocean over localized region 2. \\
    \hline
    \hyperref[subsec:nm_sec_config_iso_heat_flux_region2_radius]{config\_iso\_heat\_flux\_region2\_\-radius} & Radius of heat flux localized region 2. \\
    \hline
    \hyperref[subsec:nm_sec_config_iso_surface_temperature_piston_velocity]{config\_iso\_surface\_\-temperature\_piston\_velocity} & Surface temperature restoring piston velocity. \\
    \hline
    \hyperref[subsec:nm_sec_config_iso_initial_temp_t1]{config\_iso\_initial\_temp\_t1} & Maximum temperature parameter for the initial temperature profile. \\
    \hline
    \hyperref[subsec:nm_sec_config_iso_initial_temp_t2]{config\_iso\_initial\_temp\_t2} & Amplitude parameter for the initial temperature profile. \\
    \hline
    \hyperref[subsec:nm_sec_config_iso_initial_temp_h0]{config\_iso\_initial\_temp\_h0} & Depth parameter for the initial temperature profile. \\
    \hline
    \hyperref[subsec:nm_sec_config_iso_initial_temp_h1]{config\_iso\_initial\_temp\_h1} & Depth parameter for the initial temperature profile. \\
    \hline
    \hyperref[subsec:nm_sec_config_iso_initial_temp_mt]{config\_iso\_initial\_temp\_mt} & Slope parameter for the initial temperature profile. \\
    \hline
    \hyperref[subsec:nm_sec_config_iso_initial_temp_latS]{config\_iso\_initial\_temp\_latS} & Southern latitude used to linearly scale the initial temperature field in the horizontal. \\
    \hline
    \hyperref[subsec:nm_sec_config_iso_initial_temp_latN]{config\_iso\_initial\_temp\_latN} & Southern latitude used to linearly scale the initial temperature field in the horizontal. \\
    \hline
    \hyperref[subsec:nm_sec_config_iso_temperature_sponge_t1]{config\_iso\_temperature\_\-sponge\_t1} & Parameter for the sponge vertical temperature profile. \\
    \hline
    \hyperref[subsec:nm_sec_config_iso_temperature_sponge_h1]{config\_iso\_temperature\_\-sponge\_h1} & E-folding distance parameter for the sponge vertical temperature profile. \\
    \hline
    \hyperref[subsec:nm_sec_config_iso_temperature_sponge_l1]{config\_iso\_temperature\_\-sponge\_l1} & Horizontal e-folding distance parameter for the sponge weights. \\
    \hline
    \hyperref[subsec:nm_sec_config_iso_temperature_sponge_tau1]{config\_iso\_temperature\_\-sponge\_tau1} & Sponge layer restoring time scale, used to calculate interior restoring rate. \\
    \hline
    \hyperref[subsec:nm_sec_config_iso_temperature_restore_region1_flag]{config\_iso\_temperature\_\-restore\_region1\_flag} & Logical flag controlling use of temperature restoring in region 1. \\
    \hline
    \hyperref[subsec:nm_sec_config_iso_temperature_restore_t1]{config\_iso\_temperature\_\-restore\_t1} & Restoring temperature in region 1 \\
    \hline
    \hyperref[subsec:nm_sec_config_iso_temperature_restore_lcx1]{config\_iso\_temperature\_\-restore\_lcx1} & Zonal length scale of the restoring region 1 \\
    \hline
    \hyperref[subsec:nm_sec_config_iso_temperature_restore_lcy1]{config\_iso\_temperature\_\-restore\_lcy1} & Meridional length scale of the restoring region 1 \\
    \hline
    \hyperref[subsec:nm_sec_config_iso_temperature_restore_region2_flag]{config\_iso\_temperature\_\-restore\_region2\_flag} & Logical flag controlling use of temperature restoring in region 2. \\
    \hline
    \hyperref[subsec:nm_sec_config_iso_temperature_restore_t2]{config\_iso\_temperature\_\-restore\_t2} & Restoring temperature in region 2 \\
    \hline
    \hyperref[subsec:nm_sec_config_iso_temperature_restore_lcx2]{config\_iso\_temperature\_\-restore\_lcx2} & Zonal length scale of the restoring region 2 \\
    \hline
    \hyperref[subsec:nm_sec_config_iso_temperature_restore_lcy2]{config\_iso\_temperature\_\-restore\_lcy2} & Meridional length scale of the restoring region 2 \\
    \hline
    \hyperref[subsec:nm_sec_config_iso_temperature_restore_region3_flag]{config\_iso\_temperature\_\-restore\_region3\_flag} & Logical flag controlling use of temperature restoring in region 3. \\
    \hline
    \hyperref[subsec:nm_sec_config_iso_temperature_restore_t3]{config\_iso\_temperature\_\-restore\_t3} & Restoring temperature in region 3 \\
    \hline
    \hyperref[subsec:nm_sec_config_iso_temperature_restore_lcx3]{config\_iso\_temperature\_\-restore\_lcx3} & Zonal length scale of the restoring region 3 \\
    \hline
    \hyperref[subsec:nm_sec_config_iso_temperature_restore_lcy3]{config\_iso\_temperature\_\-restore\_lcy3} & Meridional length scale of the restoring region 3 \\
    \hline
    \hyperref[subsec:nm_sec_config_iso_temperature_restore_region4_flag]{config\_iso\_temperature\_\-restore\_region4\_flag} & Logical flag controlling use of temperature restoring in region 4. \\
    \hline
    \hyperref[subsec:nm_sec_config_iso_temperature_restore_t4]{config\_iso\_temperature\_\-restore\_t4} & Restoring temperature in region 4 \\
    \hline
    \hyperref[subsec:nm_sec_config_iso_temperature_restore_lcx4]{config\_iso\_temperature\_\-restore\_lcx4} & Zonal length scale of the restoring region 4 \\
    \hline
    \hyperref[subsec:nm_sec_config_iso_temperature_restore_lcy4]{config\_iso\_temperature\_\-restore\_lcy4} & Meridional length scale of the restoring region 4 \\
    \hline
\end{longtable}
\end{center}
}
\section[soma]{\hyperref[sec:nm_sec_soma]{soma}}
\label{sec:nm_tab_soma}
\vspace{0.5in}
{\small
\begin{center}
\begin{longtable}{| p{2.0in} || p{4.0in} |}
    \hline
    {\bf Name} & {\bf Description} \endfirsthead
    \hline 
    {\bf Name} & {\bf Description} (Continued) \endhead
    \hline
    \hline
    \hyperref[subsec:nm_sec_config_soma_vert_levels]{config\_soma\_vert\_levels} & Number of vertical levels in SOMA. \\
    \hline
    \hyperref[subsec:nm_sec_config_soma_domain_width]{config\_soma\_domain\_width} & Approximate radius of the SOMA domain. \\
    \hline
    \hyperref[subsec:nm_sec_config_soma_center_latitude]{config\_soma\_center\_latitude} & Latitude for the center of the SOMA basin. \\
    \hline
    \hyperref[subsec:nm_sec_config_soma_center_longitude]{config\_soma\_center\_longitude} & Longitude for the center of the SOMA basin. \\
    \hline
    \hyperref[subsec:nm_sec_config_soma_phi]{config\_soma\_phi} & Scale factor controlling width of continential slope. Typically around 0.1 \\
    \hline
    \hyperref[subsec:nm_sec_config_soma_bottom_depth]{config\_soma\_bottom\_depth} & Depth of the bottom of the ocean for the SOMA test case. \\
    \hline
    \hyperref[subsec:nm_sec_config_soma_shelf_width]{config\_soma\_shelf\_width} & Non-dimensional measure of continential shelf. Typically negative. \\
    \hline
    \hyperref[subsec:nm_sec_config_soma_shelf_depth]{config\_soma\_shelf\_depth} & Depth of the continential shelf. \\
    \hline
    \hyperref[subsec:nm_sec_config_soma_ref_density]{config\_soma\_ref\_density} & Reference density for the SOMA test case. \\
    \hline
    \hyperref[subsec:nm_sec_config_soma_density_difference]{config\_soma\_density\_difference} & Density difference between surface and bottom waters for the SOMA test case. \\
    \hline
    \hyperref[subsec:nm_sec_config_soma_thermocline_depth]{config\_soma\_thermocline\_depth} & Depth over which majority of initial stratification is placed. \\
    \hline
    \hyperref[subsec:nm_sec_config_soma_density_difference_linear]{config\_soma\_density\_\-difference\_linear} & Fraction of stratification put into linear profile extending from surface to bottom. \\
    \hline
    \hyperref[subsec:nm_sec_config_soma_surface_temperature]{config\_soma\_surface\_\-temperature} & Surface temperature value used in initial condition. \\
    \hline
    \hyperref[subsec:nm_sec_config_soma_surface_salinity]{config\_soma\_surface\_salinity} & Surface salinity value used in initial condition. \\
    \hline
    \hyperref[subsec:nm_sec_config_soma_use_surface_temp_restoring]{config\_soma\_use\_surface\_\-temp\_restoring} & Logical flag that determines if surface temperature restoring is to be used. \\
    \hline
    \hyperref[subsec:nm_sec_config_soma_surface_temp_restoring_at_center_latitude]{config\_soma\_surface\_temp\_\-restoring\_at\_center\_latitude} & Surface restoring temperature value at center latitutde. \\
    \hline
    \hyperref[subsec:nm_sec_config_soma_surface_temp_restoring_latitude_gradient]{config\_soma\_surface\_temp\_\-restoring\_latitude\_gradient} & Surface restoring temperature gradient in latitudal direction. \\
    \hline
    \hyperref[subsec:nm_sec_config_soma_restoring_temp_piston_vel]{config\_soma\_restoring\_temp\_\-piston\_vel} & Restoring piston velocity for surface temperature. \\
    \hline
\end{longtable}
\end{center}
}
\section[ziso]{\hyperref[sec:nm_sec_ziso]{ziso}}
\label{sec:nm_tab_ziso}
\vspace{0.5in}
{\small
\begin{center}
\begin{longtable}{| p{2.0in} || p{4.0in} |}
    \hline
    {\bf Name} & {\bf Description} \endfirsthead
    \hline 
    {\bf Name} & {\bf Description} (Continued) \endhead
    \hline
    \hline
    \hyperref[subsec:nm_sec_config_ziso_vert_levels]{config\_ziso\_vert\_levels} & Number of vertical levels in ziso. Typical value is 100. \\
    \hline
    \hyperref[subsec:nm_sec_config_ziso_add_easterly_wind_stress_ASF]{config\_ziso\_add\_easterly\_\-wind\_stress\_ASF} & Logical flag to determine if an easterly windstress is added \\
    \hline
    \hyperref[subsec:nm_sec_config_ziso_wind_transition_position]{config\_ziso\_wind\_transition\_\-position} & meridional position where windstress switches to easterly \\
    \hline
    \hyperref[subsec:nm_sec_config_ziso_antarctic_shelf_front_width]{config\_ziso\_antarctic\_shelf\_\-front\_width} & meridional extent over which the easterly wind stress is applied \\
    \hline
    \hyperref[subsec:nm_sec_config_ziso_wind_stress_shelf_front_max]{config\_ziso\_wind\_stress\_shelf\_\-front\_max} & Maximum zonal windstress value in the shelf front region, following Stewart et al. 2013 \\
    \hline
    \hyperref[subsec:nm_sec_config_ziso_use_slopping_bathymetry]{config\_ziso\_use\_slopping\_\-bathymetry} & Logical flag that determines if sloping bathymetery is used. \\
    \hline
    \hyperref[subsec:nm_sec_config_ziso_meridional_extent]{config\_ziso\_meridional\_extent} & Meridional extent of the domain ($L$). \\
    \hline
    \hyperref[subsec:nm_sec_config_ziso_zonal_extent]{config\_ziso\_zonal\_extent} & Zonal extent of the domain ($W$). \\
    \hline
    \hyperref[subsec:nm_sec_config_ziso_bottom_depth]{config\_ziso\_bottom\_depth} & Depth of the domain ($H$). \\
    \hline
    \hyperref[subsec:nm_sec_config_ziso_shelf_depth]{config\_ziso\_shelf\_depth} & Shelf depth in the domain ($H_s$). \\
    \hline
    \hyperref[subsec:nm_sec_config_ziso_slope_half_width]{config\_ziso\_slope\_half\_width} & Shelf half width ($W_s$). \\
    \hline
    \hyperref[subsec:nm_sec_config_ziso_slope_center_position]{config\_ziso\_slope\_center\_\-position} & Slope center posiiton ($Y_s$). \\
    \hline
    \hyperref[subsec:nm_sec_config_ziso_reference_coriolis]{config\_ziso\_reference\_coriolis} & Reference coriolis parameter $f_0$. Note $f = f_0 + \beta * y$. \\
    \hline
    \hyperref[subsec:nm_sec_config_ziso_coriolis_gradient]{config\_ziso\_coriolis\_gradient} & Meridional gradient of coriolis parameter $\beta$. \\
    \hline
    \hyperref[subsec:nm_sec_config_ziso_wind_stress_max]{config\_ziso\_wind\_stress\_max} & Maximum zonal windstress value $\tau_0$. \\
    \hline
    \hyperref[subsec:nm_sec_config_ziso_mean_restoring_temp]{config\_ziso\_mean\_restoring\_\-temp} & Mean restoring temperature $T_m$ in $T_r(y) = T_m + T_a \tanh\left(2\frac{y-L/2}{L/2}\right) + T_b \frac{y-L/2}{L/2}$. \\
    \hline
    \hyperref[subsec:nm_sec_config_ziso_restoring_temp_dev_ta]{config\_ziso\_restoring\_temp\_\-dev\_ta} & Temperature deviation $T_a$ in surface temp. $T_r(y) = T_m + T_a \tanh\left(2\frac{y-L/2}{L/2}\right) + T_b \frac{y-L/2}{L/2}$. \\
    \hline
    \hyperref[subsec:nm_sec_config_ziso_restoring_temp_dev_tb]{config\_ziso\_restoring\_temp\_\-dev\_tb} & Linear temperature deviation $T_b$ in surface temp. $T_r(y) = T_m + T_a \tanh\left(2\frac{y-L/2}{L/2}\right) + T_b \frac{y-L/2}{L/2}$. \\
    \hline
    \hyperref[subsec:nm_sec_config_ziso_restoring_temp_tau]{config\_ziso\_restoring\_temp\_tau} & Time scale for interior restoring of temperature. \\
    \hline
    \hyperref[subsec:nm_sec_config_ziso_restoring_temp_piston_vel]{config\_ziso\_restoring\_temp\_\-piston\_vel} & Restoring piston velocity for surface temperature. \\
    \hline
    \hyperref[subsec:nm_sec_config_ziso_restoring_temp_ze]{config\_ziso\_restoring\_temp\_ze} & Vertical $e-$folding scale in $T_s$ for northern wall: $T_s \exp(z/z_e)$. \\
    \hline
    \hyperref[subsec:nm_sec_config_ziso_restoring_sponge_l]{config\_ziso\_restoring\_sponge\_l} & E-folding distance parameter for the sponge vertical temperature profile. \\
    \hline
    \hyperref[subsec:nm_sec_config_ziso_initial_temp_t1]{config\_ziso\_initial\_temp\_t1} & Initial temperature profile constant $T_1$ in $T(z,t=0) = T_1 + T_2 \tanh(z/h_1) + m_T z$. \\
    \hline
    \hyperref[subsec:nm_sec_config_ziso_initial_temp_t2]{config\_ziso\_initial\_temp\_t2} & Initial temperature profile constant $T_2$ in $T(z,t=0) = T_1 + T_2 \tanh(z/h_1) + m_T z$. \\
    \hline
    \hyperref[subsec:nm_sec_config_ziso_initial_temp_h1]{config\_ziso\_initial\_temp\_h1} & Initial temperature profile constant $h_1$ in $T(z,t=0) = T_1 + T_2 \tanh(z/h_1) + m_T z$. \\
    \hline
    \hyperref[subsec:nm_sec_config_ziso_initial_temp_mt]{config\_ziso\_initial\_temp\_mt} & Initial temperature profile constant $m_T$ in $T(z,t=0) = T_1 + T_2 \tanh(z/h_1) + m_T z$. \\
    \hline
    \hyperref[subsec:nm_sec_config_ziso_frazil_enable]{config\_ziso\_frazil\_enable} & A logical to overload (and largely overwrite) this test case to evaluate frazil. In almost all uses of this test case, this configure option should be false. \\
    \hline
    \hyperref[subsec:nm_sec_config_ziso_frazil_temperature_anomaly]{config\_ziso\_frazil\_temperature\_\-anomaly} & Temperature anomaly to produce frazil \\
    \hline
\end{longtable}
\end{center}
}
\section[sub\_ice\_shelf\_2D]{\hyperref[sec:nm_sec_sub_ice_shelf_2D]{sub\_ice\_shelf\_2D}}
\label{sec:nm_tab_sub_ice_shelf_2D}
\vspace{0.5in}
{\small
\begin{center}
\begin{longtable}{| p{2.0in} || p{4.0in} |}
    \hline
    {\bf Name} & {\bf Description} \endfirsthead
    \hline 
    {\bf Name} & {\bf Description} (Continued) \endhead
    \hline
    \hline
    \hyperref[subsec:nm_sec_config_sub_ice_shelf_2D_vert_levels]{config\_sub\_ice\_shelf\_2D\_\-vert\_levels} & Number of vertical levels in sub\_ice\_shelf\_2D. Typical value is 22. \\
    \hline
    \hyperref[subsec:nm_sec_config_sub_ice_shelf_2D_bottom_depth]{config\_sub\_ice\_shelf\_2D\_\-bottom\_depth} & Depth of the bottom of the ocean for the this test case. \\
    \hline
    \hyperref[subsec:nm_sec_config_sub_ice_shelf_2D_cavity_thickness]{config\_sub\_ice\_shelf\_2D\_\-cavity\_thickness} & Vertical thickness of ocean sub-ice cavity. \\
    \hline
    \hyperref[subsec:nm_sec_config_sub_ice_shelf_2D_slope_height]{config\_sub\_ice\_shelf\_2D\_\-slope\_height} & Vertical thickness of fixed slope. \\
    \hline
    \hyperref[subsec:nm_sec_config_sub_ice_shelf_2D_edge_width]{config\_sub\_ice\_shelf\_2D\_\-edge\_width} & Horizontal width of angled part of the ice. \\
    \hline
    \hyperref[subsec:nm_sec_config_sub_ice_shelf_2D_y1]{config\_sub\_ice\_shelf\_2D\_y1} & cavity edge in y \\
    \hline
    \hyperref[subsec:nm_sec_config_sub_ice_shelf_2D_y2]{config\_sub\_ice\_shelf\_2D\_y2} & shelf edge in y \\
    \hline
    \hyperref[subsec:nm_sec_config_sub_ice_shelf_2D_temperature]{config\_sub\_ice\_shelf\_2D\_\-temperature} & Temperature of the surface in the northern half of the domain. \\
    \hline
    \hyperref[subsec:nm_sec_config_sub_ice_shelf_2D_surface_salinity]{config\_sub\_ice\_shelf\_2D\_\-surface\_salinity} & Salinity of the water in the entire domain. \\
    \hline
    \hyperref[subsec:nm_sec_config_sub_ice_shelf_2D_bottom_salinity]{config\_sub\_ice\_shelf\_2D\_\-bottom\_salinity} & Salinity of the water in the entire domain. \\
    \hline
\end{longtable}
\end{center}
}
\section[periodic\_planar]{\hyperref[sec:nm_sec_periodic_planar]{periodic\_planar}}
\label{sec:nm_tab_periodic_planar}
\vspace{0.5in}
{\small
\begin{center}
\begin{longtable}{| p{2.0in} || p{4.0in} |}
    \hline
    {\bf Name} & {\bf Description} \endfirsthead
    \hline 
    {\bf Name} & {\bf Description} (Continued) \endhead
    \hline
    \hline
    \hyperref[subsec:nm_sec_config_periodic_planar_vert_levels]{config\_periodic\_planar\_vert\_\-levels} & Number of vertical levels in periodic\_planar. Typical value is 1. \\
    \hline
    \hyperref[subsec:nm_sec_config_periodic_planar_bottom_depth]{config\_periodic\_planar\_\-bottom\_depth} & Bottom depth. \\
    \hline
    \hyperref[subsec:nm_sec_config_periodic_planar_velocity_strength]{config\_periodic\_planar\_\-velocity\_strength} & Strenght of velocity field. \\
    \hline
\end{longtable}
\end{center}
}
\section[ecosys\_column]{\hyperref[sec:nm_sec_ecosys_column]{ecosys\_column}}
\label{sec:nm_tab_ecosys_column}
\vspace{0.5in}
{\small
\begin{center}
\begin{longtable}{| p{2.0in} || p{4.0in} |}
    \hline
    {\bf Name} & {\bf Description} \endfirsthead
    \hline 
    {\bf Name} & {\bf Description} (Continued) \endhead
    \hline
    \hline
    \hyperref[subsec:nm_sec_config_ecosys_column_vert_levels]{config\_ecosys\_column\_vert\_\-levels} & Number of vertical levels in ecosys column unit test case. \\
    \hline
    \hyperref[subsec:nm_sec_config_ecosys_column_vertical_grid]{config\_ecosys\_column\_vertical\_\-grid} & prescription of setting the vertical resolution of the test case \\
    \hline
    \hyperref[subsec:nm_sec_config_ecosys_column_TS_filename]{config\_ecosys\_column\_TS\_\-filename} & Name of file containing column values of temperature and salinity \\
    \hline
    \hyperref[subsec:nm_sec_config_ecosys_column_ecosys_filename]{config\_ecosys\_column\_ecosys\_\-filename} & Name of file containing column values of ecosys variables \\
    \hline
    \hyperref[subsec:nm_sec_config_ecosys_column_bottom_depth]{config\_ecosys\_column\_bottom\_\-depth} & Depth of the bottom of the ocean for the ecosys column unit test case. \\
    \hline
\end{longtable}
\end{center}
}
\section[sea\_mount]{\hyperref[sec:nm_sec_sea_mount]{sea\_mount}}
\label{sec:nm_tab_sea_mount}
\vspace{0.5in}
{\small
\begin{center}
\begin{longtable}{| p{2.0in} || p{4.0in} |}
    \hline
    {\bf Name} & {\bf Description} \endfirsthead
    \hline 
    {\bf Name} & {\bf Description} (Continued) \endhead
    \hline
    \hline
    \hyperref[subsec:nm_sec_config_sea_mount_vert_levels]{config\_sea\_mount\_vert\_levels} & Number of vertical levels in sea mount test case. \\
    \hline
    \hyperref[subsec:nm_sec_config_sea_mount_layer_type]{config\_sea\_mount\_layer\_type} & Logical flag that controls the vertical coordinate initializaton \\
    \hline
    \hyperref[subsec:nm_sec_config_sea_mount_stratification_type]{config\_sea\_mount\_\-stratification\_type} & Logical flag that controls how the vertical profile of tracers.  See Beckmann and Haidvogel 1993 eqn 15-16. \\
    \hline
    \hyperref[subsec:nm_sec_config_sea_mount_density_coef_linear]{config\_sea\_mount\_density\_\-coef\_linear} & Density coefficient for linear vertical stratification \\
    \hline
    \hyperref[subsec:nm_sec_config_sea_mount_density_coef_exp]{config\_sea\_mount\_density\_\-coef\_exp} & Density coefficient for exponential vertical stratification \\
    \hline
    \hyperref[subsec:nm_sec_config_sea_mount_density_gradient_linear]{config\_sea\_mount\_density\_\-gradient\_linear} & Density gradient for linear vertical stratification, $\Delta_z \rho$ in Beckmann Haidvogel eqn 15 \\
    \hline
    \hyperref[subsec:nm_sec_config_sea_mount_density_gradient_exp]{config\_sea\_mount\_density\_\-gradient\_exp} & Density gradient for exponential vertical stratification, $\Delta_z \rho$ in Beckmann Haidvogel eqn 16 \\
    \hline
    \hyperref[subsec:nm_sec_config_sea_mount_density_depth_linear]{config\_sea\_mount\_density\_\-depth\_linear} & Density reference depth for linear vertical stratification \\
    \hline
    \hyperref[subsec:nm_sec_config_sea_mount_density_depth_exp]{config\_sea\_mount\_density\_\-depth\_exp} & Density reference depth for exponential vertical stratification \\
    \hline
    \hyperref[subsec:nm_sec_config_sea_mount_density_ref]{config\_sea\_mount\_density\_ref} & Density reference for eos to initialize temperature \\
    \hline
    \hyperref[subsec:nm_sec_config_sea_mount_density_Tref]{config\_sea\_mount\_density\_\-Tref} & Reference temperature for eos to initialize temperature \\
    \hline
    \hyperref[subsec:nm_sec_config_sea_mount_density_alpha]{config\_sea\_mount\_density\_\-alpha} & Linear thermal expansion coefficient to initialize temperature \\
    \hline
    \hyperref[subsec:nm_sec_config_sea_mount_bottom_depth]{config\_sea\_mount\_bottom\_\-depth} & Depth of the bottom of the ocean for the sea mount test case. \\
    \hline
    \hyperref[subsec:nm_sec_config_sea_mount_height]{config\_sea\_mount\_height} & Height of sea mount, $H_0$ \\
    \hline
    \hyperref[subsec:nm_sec_config_sea_mount_radius]{config\_sea\_mount\_radius} & Radius of sea mount \\
    \hline
    \hyperref[subsec:nm_sec_config_sea_mount_width]{config\_sea\_mount\_width} & Width parameter of sea mount, $L$. \\
    \hline
    \hyperref[subsec:nm_sec_config_sea_mount_salinity]{config\_sea\_mount\_salinity} & Salinity of the water in the entire domain. \\
    \hline
    \hyperref[subsec:nm_sec_config_sea_mount_coriolis_parameter]{config\_sea\_mount\_coriolis\_\-parameter} & Coriolis parameter for entrie domain. \\
    \hline
\end{longtable}
\end{center}
}
\section[isomip]{\hyperref[sec:nm_sec_isomip]{isomip}}
\label{sec:nm_tab_isomip}
\vspace{0.5in}
{\small
\begin{center}
\begin{longtable}{| p{2.0in} || p{4.0in} |}
    \hline
    {\bf Name} & {\bf Description} \endfirsthead
    \hline 
    {\bf Name} & {\bf Description} (Continued) \endhead
    \hline
    \hline
    \hyperref[subsec:nm_sec_config_isomip_vert_levels]{config\_isomip\_vert\_levels} & Number of vertical levels in test case. \\
    \hline
    \hyperref[subsec:nm_sec_config_isomip_vertical_level_distribution]{config\_isomip\_vertical\_level\_\-distribution} & The distribution of vertical levels, either constant (all equal thickness) or boundary layer (decreasing toward top and bottom). \\
    \hline
    \hyperref[subsec:nm_sec_config_isomip_bottom_depth]{config\_isomip\_bottom\_depth} & Depth of the ocean in the test case. \\
    \hline
    \hyperref[subsec:nm_sec_config_isomip_temperature]{config\_isomip\_temperature} & Temperature of the ocean for isomip initial conditions. \\
    \hline
    \hyperref[subsec:nm_sec_config_isomip_salinity]{config\_isomip\_salinity} & Salinity of the ocean for isomip initial conditions. \\
    \hline
    \hyperref[subsec:nm_sec_config_isomip_restoring_temperature]{config\_isomip\_restoring\_\-temperature} & Temperature for surface restoring. \\
    \hline
    \hyperref[subsec:nm_sec_config_isomip_temperature_piston_velocity]{config\_isomip\_temperature\_\-piston\_velocity} & Piston velocity for surface restoring of temperature \\
    \hline
    \hyperref[subsec:nm_sec_config_isomip_restoring_salinity]{config\_isomip\_restoring\_\-salinity} & Salinity for surface restoring. \\
    \hline
    \hyperref[subsec:nm_sec_config_isomip_salinity_piston_velocity]{config\_isomip\_salinity\_piston\_\-velocity} & Piston velocity for surface restoring of salinity \\
    \hline
    \hyperref[subsec:nm_sec_config_isomip_coriolis_parameter]{config\_isomip\_coriolis\_\-parameter} & Coriolis parameter for entrie domain. \\
    \hline
    \hyperref[subsec:nm_sec_config_isomip_southern_boundary]{config\_isomip\_southern\_\-boundary} & The y location of the southern boundary. \\
    \hline
    \hyperref[subsec:nm_sec_config_isomip_northern_boundary]{config\_isomip\_northern\_\-boundary} & The y location of the northern boundary. \\
    \hline
    \hyperref[subsec:nm_sec_config_isomip_western_boundary]{config\_isomip\_western\_\-boundary} & The x location of the western boundary. \\
    \hline
    \hyperref[subsec:nm_sec_config_isomip_eastern_boundary]{config\_isomip\_eastern\_\-boundary} & The x location of the eastern boundary. \\
    \hline
    \hyperref[subsec:nm_sec_config_isomip_y1]{config\_isomip\_y1} & The first y value in the piecewise linear ice draft. \\
    \hline
    \hyperref[subsec:nm_sec_config_isomip_z1]{config\_isomip\_z1} & The first z value in the piecewise linear ice draft. \\
    \hline
    \hyperref[subsec:nm_sec_config_isomip_ice_fraction1]{config\_isomip\_ice\_fraction1} & The first ice fraction value in the piecewise linear fit. \\
    \hline
    \hyperref[subsec:nm_sec_config_isomip_y2]{config\_isomip\_y2} & The second y value in the piecewise linear ice draft. \\
    \hline
    \hyperref[subsec:nm_sec_config_isomip_z2]{config\_isomip\_z2} & The second z value in the piecewise linear. \\
    \hline
    \hyperref[subsec:nm_sec_config_isomip_ice_fraction2]{config\_isomip\_ice\_fraction2} & The second ice fraction value in the piecewise linear fit. \\
    \hline
    \hyperref[subsec:nm_sec_config_isomip_y3]{config\_isomip\_y3} & The third y value in the piecewise linear ice draft. \\
    \hline
    \hyperref[subsec:nm_sec_config_isomip_z3]{config\_isomip\_z3} & The third z value in the piecewise linear. \\
    \hline
    \hyperref[subsec:nm_sec_config_isomip_ice_fraction3]{config\_isomip\_ice\_fraction3} & The third ice fraction value in the piecewise linear fit. \\
    \hline
\end{longtable}
\end{center}
}
\section[isomip\_plus]{\hyperref[sec:nm_sec_isomip_plus]{isomip\_plus}}
\label{sec:nm_tab_isomip_plus}
\vspace{0.5in}
{\small
\begin{center}
\begin{longtable}{| p{2.0in} || p{4.0in} |}
    \hline
    {\bf Name} & {\bf Description} \endfirsthead
    \hline 
    {\bf Name} & {\bf Description} (Continued) \endhead
    \hline
    \hline
    \hyperref[subsec:nm_sec_config_isomip_plus_vert_levels]{config\_isomip\_plus\_vert\_levels} & Number of vertical levels in test case. \\
    \hline
    \hyperref[subsec:nm_sec_config_isomip_plus_vertical_level_distribution]{config\_isomip\_plus\_vertical\_\-level\_distribution} & The distribution of vertical levels, currently only constant (all equal thickness). \\
    \hline
    \hyperref[subsec:nm_sec_config_isomip_plus_max_bottom_depth]{config\_isomip\_plus\_max\_\-bottom\_depth} & Maximum depth of the ocean in the test case. \\
    \hline
    \hyperref[subsec:nm_sec_config_isomip_plus_minimum_levels]{config\_isomip\_plus\_minimum\_\-levels} & Minimum number of vertical levels in a column. \\
    \hline
    \hyperref[subsec:nm_sec_config_isomip_plus_min_column_thickness]{config\_isomip\_plus\_min\_\-column\_thickness} & Minimum thickness of the inital ocean column (to prevent 'drying'). \\
    \hline
    \hyperref[subsec:nm_sec_config_isomip_plus_min_ocean_fraction]{config\_isomip\_plus\_min\_\-ocean\_fraction} & Minimum fraction of a cell that contains ocean (as opposed to land or grounded land ice) in order for it to be an active ocean cell. \\
    \hline
    \hyperref[subsec:nm_sec_config_isomip_plus_topography_file]{config\_isomip\_plus\_\-topography\_file} & Path to the topography initial condition file. \\
    \hline
    \hyperref[subsec:nm_sec_config_isomip_plus_init_top_temp]{config\_isomip\_plus\_init\_top\_\-temp} & Initial temperature at sea level. \\
    \hline
    \hyperref[subsec:nm_sec_config_isomip_plus_init_bot_temp]{config\_isomip\_plus\_init\_bot\_\-temp} & Initial temperature in deepest cells. \\
    \hline
    \hyperref[subsec:nm_sec_config_isomip_plus_init_top_sal]{config\_isomip\_plus\_init\_top\_\-sal} & Initial salinity at sea level. \\
    \hline
    \hyperref[subsec:nm_sec_config_isomip_plus_init_bot_sal]{config\_isomip\_plus\_init\_bot\_\-sal} & Initial salinity in deepest cells. \\
    \hline
    \hyperref[subsec:nm_sec_config_isomip_plus_restore_top_temp]{config\_isomip\_plus\_restore\_\-top\_temp} & Restoring temperature at sea level. \\
    \hline
    \hyperref[subsec:nm_sec_config_isomip_plus_restore_bot_temp]{config\_isomip\_plus\_restore\_\-bot\_temp} & Restoring temperature in deepest cells. \\
    \hline
    \hyperref[subsec:nm_sec_config_isomip_plus_restore_top_sal]{config\_isomip\_plus\_restore\_\-top\_sal} & Restoring salinity at sea level. \\
    \hline
    \hyperref[subsec:nm_sec_config_isomip_plus_restore_bot_sal]{config\_isomip\_plus\_restore\_\-bot\_sal} & Restoring salinity in deepest cells. \\
    \hline
    \hyperref[subsec:nm_sec_config_isomip_plus_restore_rate]{config\_isomip\_plus\_restore\_\-rate} & Restoring salinity in deepest cells. \\
    \hline
    \hyperref[subsec:nm_sec_config_isomip_plus_restore_evap_rate]{config\_isomip\_plus\_restore\_\-evap\_rate} & Evaporation rate used to maintain sea level near zero. \\
    \hline
    \hyperref[subsec:nm_sec_config_isomip_plus_restore_xMin]{config\_isomip\_plus\_restore\_x\-Min} & Southern boundary of restoring region. \\
    \hline
    \hyperref[subsec:nm_sec_config_isomip_plus_restore_xMax]{config\_isomip\_plus\_restore\_x\-Max} & Northern boundary of restoring region. \\
    \hline
    \hyperref[subsec:nm_sec_config_isomip_plus_coriolis_parameter]{config\_isomip\_plus\_coriolis\_\-parameter} & Coriolis parameter for entrie domain. \\
    \hline
    \hyperref[subsec:nm_sec_config_isomip_plus_effective_density]{config\_isomip\_plus\_effective\_\-density} & Initial value for the effective density for entrie domain. \\
    \hline
\end{longtable}
\end{center}
}
\section[tracer\_forcing\_activeTracers]{\hyperref[sec:nm_sec_tracer_forcing_activeTracers]{tracer\_forcing\_activeTracers}}
\label{sec:nm_tab_tracer_forcing_activeTracers}
\vspace{0.5in}
{\small
\begin{center}
\begin{longtable}{| p{2.0in} || p{4.0in} |}
    \hline
    {\bf Name} & {\bf Description} \endfirsthead
    \hline 
    {\bf Name} & {\bf Description} (Continued) \endhead
    \hline
    \hline
    \hyperref[subsec:nm_sec_config_use_activeTracers]{config\_use\_activeTracers} & if true, the 'activeTracers' category is enabled for the run \\
    \hline
    \hyperref[subsec:nm_sec_config_use_activeTracers_surface_bulk_forcing]{config\_use\_activeTracers\_\-surface\_bulk\_forcing} & if true, surface bulk forcing from coupler is added to surfaceTracerFlux in 'activeTracers' category \\
    \hline
    \hyperref[subsec:nm_sec_config_use_activeTracers_surface_restoring]{config\_use\_activeTracers\_\-surface\_restoring} & if true, surface restoring source is applied to tracers in 'activeTracers' category \\
    \hline
    \hyperref[subsec:nm_sec_config_use_activeTracers_interior_restoring]{config\_use\_activeTracers\_\-interior\_restoring} & if true, interior restoring source is applied to tracers in 'activeTracers' category \\
    \hline
    \hyperref[subsec:nm_sec_config_use_activeTracers_exponential_decay]{config\_use\_activeTracers\_\-exponential\_decay} & if true, exponential decay source is applied to tracers in 'activeTracers' category \\
    \hline
    \hyperref[subsec:nm_sec_config_use_activeTracers_idealAge_forcing]{config\_use\_activeTracers\_ideal\-Age\_forcing} & if true, idealAge forcing source is applied to tracers in 'activeTracers' category \\
    \hline
    \hyperref[subsec:nm_sec_config_use_activeTracers_ttd_forcing]{config\_use\_activeTracers\_ttd\_\-forcing} & if true, transit time distribution forcing source is applied to tracers in 'activeTracers' category \\
    \hline
    \hyperref[subsec:nm_sec_config_use_surface_salinity_monthly_restoring]{config\_use\_surface\_salinity\_\-monthly\_restoring} & If true, apply monthly salinity restoring using a uniform piston velocity, defined at run-time by config\_salinity\_restoring\_constant\_piston\_velocity.  When false, salinity piston velocity is specified in the input file by salinityPistonVelocity, which may be spatially variable. \\
    \hline
    \hyperref[subsec:nm_sec_config_surface_salinity_monthly_restoring_compute_interval]{config\_surface\_salinity\_\-monthly\_restoring\_compute\_\-interval} & Time interval to compute salinity restoring tendency. \\
    \hline
    \hyperref[subsec:nm_sec_config_salinity_restoring_constant_piston_velocity]{config\_salinity\_restoring\_\-constant\_piston\_velocity} & When config\_use\_surface\_salinity\_monthly\_restoring is true, this flag provides a run-time override of the salinityPistonVelocity variable in the input files.  It is uniform over the domain, and controls the rate at which salinity is restored to salinitySurfaceRestoringValue \\
    \hline
    \hyperref[subsec:nm_sec_config_salinity_restoring_max_difference]{config\_salinity\_restoring\_max\_\-difference} & Maximum allowable difference between surface salinity and climatology. \\
    \hline
    \hyperref[subsec:nm_sec_config_salinity_restoring_under_sea_ice]{config\_salinity\_restoring\_\-under\_sea\_ice} & Flag to enable salinity restoring under sea ice.  The default setting is false, where salinity restoring tapers from full restoring in the open ocean (iceFraction=0.0) to zero restoring below full sea ice coverage (iceFraction=1.0); below partial sea ice coverage, restoring is in proportion to iceFraction.  If true, full salinity restoring is used everywhere, regardless of iceFraction value \\
    \hline
\end{longtable}
\end{center}
}
\section[tracer\_forcing\_debugTracers]{\hyperref[sec:nm_sec_tracer_forcing_debugTracers]{tracer\_forcing\_debugTracers}}
\label{sec:nm_tab_tracer_forcing_debugTracers}
\vspace{0.5in}
{\small
\begin{center}
\begin{longtable}{| p{2.0in} || p{4.0in} |}
    \hline
    {\bf Name} & {\bf Description} \endfirsthead
    \hline 
    {\bf Name} & {\bf Description} (Continued) \endhead
    \hline
    \hline
    \hyperref[subsec:nm_sec_config_use_debugTracers]{config\_use\_debugTracers} & if true, the 'debugTracers' category is enabled for the run \\
    \hline
    \hyperref[subsec:nm_sec_config_use_debugTracers_surface_bulk_forcing]{config\_use\_debugTracers\_\-surface\_bulk\_forcing} & if true, surface bulk forcing from coupler is added to surfaceTracerFlux in 'debugTracers' category \\
    \hline
    \hyperref[subsec:nm_sec_config_use_debugTracers_surface_restoring]{config\_use\_debugTracers\_\-surface\_restoring} & if true, surface restoring source is applied to tracers in 'debugTracers' category \\
    \hline
    \hyperref[subsec:nm_sec_config_use_debugTracers_interior_restoring]{config\_use\_debugTracers\_\-interior\_restoring} & if true, interior restoring source is applied to tracers in 'debugTracers' category \\
    \hline
    \hyperref[subsec:nm_sec_config_use_debugTracers_exponential_decay]{config\_use\_debugTracers\_\-exponential\_decay} & if true, exponential decay source is applied to tracers in 'debugTracers' category \\
    \hline
    \hyperref[subsec:nm_sec_config_use_debugTracers_idealAge_forcing]{config\_use\_debugTracers\_ideal\-Age\_forcing} & if true, idealAge forcing source is applied to tracers in 'debugTracers' category \\
    \hline
    \hyperref[subsec:nm_sec_config_use_debugTracers_ttd_forcing]{config\_use\_debugTracers\_ttd\_\-forcing} & if true, transit time distribution forcing source is applied to tracers in 'debugTracers' category \\
    \hline
\end{longtable}
\end{center}
}
\section[tracer\_forcing\_ecosysTracers]{\hyperref[sec:nm_sec_tracer_forcing_ecosysTracers]{tracer\_forcing\_ecosysTracers}}
\label{sec:nm_tab_tracer_forcing_ecosysTracers}
\vspace{0.5in}
{\small
\begin{center}
\begin{longtable}{| p{2.0in} || p{4.0in} |}
    \hline
    {\bf Name} & {\bf Description} \endfirsthead
    \hline 
    {\bf Name} & {\bf Description} (Continued) \endhead
    \hline
    \hline
    \hyperref[subsec:nm_sec_config_use_ecosysTracers]{config\_use\_ecosysTracers} & if true, the 'ecosysGRP' category is enabled for the run \\
    \hline
    \hyperref[subsec:nm_sec_config_ecosys_atm_co2_option]{config\_ecosys\_atm\_co2\_option} & sets how atm co2 is set \\
    \hline
    \hyperref[subsec:nm_sec_config_ecosys_atm_alt_co2_option]{config\_ecosys\_atm\_alt\_co2\_\-option} & sets how alt atm co2 is set \\
    \hline
    \hyperref[subsec:nm_sec_config_ecosys_atm_alt_co2_use_eco]{config\_ecosys\_atm\_alt\_co2\_\-use\_eco} & determines whether DIC\_ALT is affected by ecosystem dynamics \\
    \hline
    \hyperref[subsec:nm_sec_config_ecosys_atm_co2_constant_value]{config\_ecosys\_atm\_co2\_\-constant\_value} & value of atm co2 when config\_ecosys\_atm\_co2\_option = constant \\
    \hline
    \hyperref[subsec:nm_sec_config_use_ecosysTracers_surface_bulk_forcing]{config\_use\_ecosysTracers\_\-surface\_bulk\_forcing} & if true, surface bulk forcing from coupler is added to surfaceTracerFlux in 'ecosysGRP' category \\
    \hline
    \hyperref[subsec:nm_sec_config_use_ecosysTracers_surface_restoring]{config\_use\_ecosysTracers\_\-surface\_restoring} & if true, surface restoring source is applied to tracers in 'ecosysGRP' category \\
    \hline
    \hyperref[subsec:nm_sec_config_use_ecosysTracers_interior_restoring]{config\_use\_ecosysTracers\_\-interior\_restoring} & if true, interior restoring source is applied to tracers in 'ecosysGRP' category \\
    \hline
    \hyperref[subsec:nm_sec_config_use_ecosysTracers_exponential_decay]{config\_use\_ecosysTracers\_\-exponential\_decay} & if true, exponential decay source is applied to tracers in 'ecosysGRP' category \\
    \hline
    \hyperref[subsec:nm_sec_config_use_ecosysTracers_idealAge_forcing]{config\_use\_ecosysTracers\_ideal\-Age\_forcing} & if true, idealAge forcing source is applied to tracers in 'ecosysGRP' category \\
    \hline
    \hyperref[subsec:nm_sec_config_use_ecosysTracers_ttd_forcing]{config\_use\_ecosysTracers\_ttd\_\-forcing} & if true, transit time distribution forcing source is applied to tracers in 'ecosysGRP' category \\
    \hline
    \hyperref[subsec:nm_sec_config_use_ecosysTracers_surface_value]{config\_use\_ecosysTracers\_\-surface\_value} & if true, surface value is computed for 'ecosysGRP' category \\
    \hline
    \hyperref[subsec:nm_sec_config_use_ecosysTracers_sea_ice_coupling]{config\_use\_ecosysTracers\_sea\_\-ice\_coupling} & if true, couple ecosys fields with sea ice \\
    \hline
    \hyperref[subsec:nm_sec_config_ecosysTracers_diagnostic_fields_level1]{config\_ecosysTracers\_\-diagnostic\_fields\_level1} & if true, make variables in ecosysDiagFieldsLevel1 available for output \\
    \hline
    \hyperref[subsec:nm_sec_config_ecosysTracers_diagnostic_fields_level2]{config\_ecosysTracers\_\-diagnostic\_fields\_level2} & if true, make variables in ecosysDiagFieldsLevel2 available for output \\
    \hline
    \hyperref[subsec:nm_sec_config_ecosysTracers_diagnostic_fields_level3]{config\_ecosysTracers\_\-diagnostic\_fields\_level3} & if true, make variables in ecosysDiagFieldsLevel3 available for output \\
    \hline
    \hyperref[subsec:nm_sec_config_ecosysTracers_diagnostic_fields_level4]{config\_ecosysTracers\_\-diagnostic\_fields\_level4} & if true, make variables in ecosysDiagFieldsLevel4 available for output \\
    \hline
    \hyperref[subsec:nm_sec_config_ecosysTracers_diagnostic_fields_level5]{config\_ecosysTracers\_\-diagnostic\_fields\_level5} & if true, make variables in ecosysDiagFieldsLevel5 available for output \\
    \hline
\end{longtable}
\end{center}
}
\section[tracer\_forcing\_DMSTracers]{\hyperref[sec:nm_sec_tracer_forcing_DMSTracers]{tracer\_forcing\_DMSTracers}}
\label{sec:nm_tab_tracer_forcing_DMSTracers}
\vspace{0.5in}
{\small
\begin{center}
\begin{longtable}{| p{2.0in} || p{4.0in} |}
    \hline
    {\bf Name} & {\bf Description} \endfirsthead
    \hline 
    {\bf Name} & {\bf Description} (Continued) \endhead
    \hline
    \hline
    \hyperref[subsec:nm_sec_config_use_DMSTracers]{config\_use\_DMSTracers} & if true, the 'DMSGRP' category is enabled for the run \\
    \hline
    \hyperref[subsec:nm_sec_config_use_DMSTracers_surface_bulk_forcing]{config\_use\_DMSTracers\_\-surface\_bulk\_forcing} & if true, surface bulk forcing from coupler is added to surfaceTracerFlux in 'DMSGRP' category \\
    \hline
    \hyperref[subsec:nm_sec_config_use_DMSTracers_surface_restoring]{config\_use\_DMSTracers\_\-surface\_restoring} & if true, surface restoring source is applied to tracers in 'DMSGRP' category \\
    \hline
    \hyperref[subsec:nm_sec_config_use_DMSTracers_interior_restoring]{config\_use\_DMSTracers\_\-interior\_restoring} & if true, interior restoring source is applied to tracers in 'DMSGRP' category \\
    \hline
    \hyperref[subsec:nm_sec_config_use_DMSTracers_exponential_decay]{config\_use\_DMSTracers\_\-exponential\_decay} & if true, exponential decay source is applied to tracers in 'DMSGRP' category \\
    \hline
    \hyperref[subsec:nm_sec_config_use_DMSTracers_idealAge_forcing]{config\_use\_DMSTracers\_ideal\-Age\_forcing} & if true, idealAge forcing source is applied to tracers in 'DMSGRP' category \\
    \hline
    \hyperref[subsec:nm_sec_config_use_DMSTracers_ttd_forcing]{config\_use\_DMSTracers\_ttd\_\-forcing} & if true, transit time distribution forcing source is applied to tracers in 'DMSGRP' category \\
    \hline
    \hyperref[subsec:nm_sec_config_use_DMSTracers_surface_value]{config\_use\_DMSTracers\_\-surface\_value} & if true, surface value is computed for 'DMSGRP' category \\
    \hline
    \hyperref[subsec:nm_sec_config_use_DMSTracers_sea_ice_coupling]{config\_use\_DMSTracers\_sea\_\-ice\_coupling} & if true, couple DMS fields with sea ice \\
    \hline
\end{longtable}
\end{center}
}
\section[tracer\_forcing\_MacroMoleculesTracers]{\hyperref[sec:nm_sec_tracer_forcing_MacroMoleculesTracers]{tracer\_forcing\_MacroMoleculesTracers}}
\label{sec:nm_tab_tracer_forcing_MacroMoleculesTracers}
\vspace{0.5in}
{\small
\begin{center}
\begin{longtable}{| p{2.0in} || p{4.0in} |}
    \hline
    {\bf Name} & {\bf Description} \endfirsthead
    \hline 
    {\bf Name} & {\bf Description} (Continued) \endhead
    \hline
    \hline
    \hyperref[subsec:nm_sec_config_use_MacroMoleculesTracers]{config\_use\_MacroMolecules\-Tracers} & if true, the 'MacroMoleculesGRP' category is enabled for the run \\
    \hline
    \hyperref[subsec:nm_sec_config_use_MacroMoleculesTracers_surface_bulk_forcing]{config\_use\_MacroMolecules\-Tracers\_surface\_bulk\_forcing} & if true, surface bulk forcing from coupler is added to surfaceTracerFlux in 'MacroMoleculesGRP' category \\
    \hline
    \hyperref[subsec:nm_sec_config_use_MacroMoleculesTracers_surface_restoring]{config\_use\_MacroMolecules\-Tracers\_surface\_restoring} & if true, surface restoring source is applied to tracers in 'MacroMoleculesGRP' category \\
    \hline
    \hyperref[subsec:nm_sec_config_use_MacroMoleculesTracers_interior_restoring]{config\_use\_MacroMolecules\-Tracers\_interior\_restoring} & if true, interior restoring source is applied to tracers in 'MacroMoleculesGRP' category \\
    \hline
    \hyperref[subsec:nm_sec_config_use_MacroMoleculesTracers_exponential_decay]{config\_use\_MacroMolecules\-Tracers\_exponential\_decay} & if true, exponential decay source is applied to tracers in 'MacroMoleculesGRP' category \\
    \hline
    \hyperref[subsec:nm_sec_config_use_MacroMoleculesTracers_idealAge_forcing]{config\_use\_MacroMolecules\-Tracers\_idealAge\_forcing} & if true, idealAge forcing source is applied to tracers in 'MacroMoleculesGRP' category \\
    \hline
    \hyperref[subsec:nm_sec_config_use_MacroMoleculesTracers_ttd_forcing]{config\_use\_MacroMolecules\-Tracers\_ttd\_forcing} & if true, transit time distribution forcing source is applied to tracers in 'MacroMoleculesGRP' category \\
    \hline
    \hyperref[subsec:nm_sec_config_use_MacroMoleculesTracers_surface_value]{config\_use\_MacroMolecules\-Tracers\_surface\_value} & if true, surface value is computed for 'MacroMoleculesGRP' category \\
    \hline
    \hyperref[subsec:nm_sec_config_use_MacroMoleculesTracers_sea_ice_coupling]{config\_use\_MacroMolecules\-Tracers\_sea\_ice\_coupling} & if true, couple MacroMolecules fields with sea ice \\
    \hline
\end{longtable}
\end{center}
}
\section[AM\_globalStats]{\hyperref[sec:nm_sec_AM_globalStats]{AM\_globalStats}}
\label{sec:nm_tab_AM_globalStats}
\vspace{0.5in}
{\small
\begin{center}
\begin{longtable}{| p{2.0in} || p{4.0in} |}
    \hline
    {\bf Name} & {\bf Description} \endfirsthead
    \hline 
    {\bf Name} & {\bf Description} (Continued) \endhead
    \hline
    \hline
    \hyperref[subsec:nm_sec_config_AM_globalStats_enable]{config\_AM\_globalStats\_enable} & If true, ocean analysis member global\_stats is called. \\
    \hline
    \hyperref[subsec:nm_sec_config_AM_globalStats_compute_interval]{config\_AM\_globalStats\_\-compute\_interval} & Timestamp determining how often analysis member computation should be performed. \\
    \hline
    \hyperref[subsec:nm_sec_config_AM_globalStats_compute_on_startup]{config\_AM\_globalStats\_\-compute\_on\_startup} & Logical flag determining if an analysis member computation occurs on start-up. \\
    \hline
    \hyperref[subsec:nm_sec_config_AM_globalStats_write_on_startup]{config\_AM\_globalStats\_write\_\-on\_startup} & Logical flag determining if an analysis member computation occurs on start-up. \\
    \hline
    \hyperref[subsec:nm_sec_config_AM_globalStats_text_file]{config\_AM\_globalStats\_text\_\-file} & If true, print global stats to a text file as well as streams. \\
    \hline
    \hyperref[subsec:nm_sec_config_AM_globalStats_directory]{config\_AM\_globalStats\_\-directory} & subdirectory to write eddy census text files \\
    \hline
    \hyperref[subsec:nm_sec_config_AM_globalStats_output_stream]{config\_AM\_globalStats\_\-output\_stream} & Name of the stream that the globalStats analysis member should get information from. \\
    \hline
\end{longtable}
\end{center}
}
\section[AM\_surfaceAreaWeightedAverages]{\hyperref[sec:nm_sec_AM_surfaceAreaWeightedAverages]{AM\_surfaceAreaWeightedAverages}}
\label{sec:nm_tab_AM_surfaceAreaWeightedAverages}
\vspace{0.5in}
{\small
\begin{center}
\begin{longtable}{| p{2.0in} || p{4.0in} |}
    \hline
    {\bf Name} & {\bf Description} \endfirsthead
    \hline 
    {\bf Name} & {\bf Description} (Continued) \endhead
    \hline
    \hline
    \hyperref[subsec:nm_sec_config_AM_surfaceAreaWeightedAverages_enable]{config\_AM\_surfaceArea\-WeightedAverages\_enable} & If true, ocean analysis member surface\_area\_weighted\_average is called. \\
    \hline
    \hyperref[subsec:nm_sec_config_AM_surfaceAreaWeightedAverages_compute_on_startup]{config\_AM\_surfaceArea\-WeightedAverages\_compute\_\-on\_startup} & Logical flag determining if an analysis member computation occurs on start-up. \\
    \hline
    \hyperref[subsec:nm_sec_config_AM_surfaceAreaWeightedAverages_write_on_startup]{config\_AM\_surfaceArea\-WeightedAverages\_write\_on\_\-startup} & Logical flag determining if an analysis member computation occurs on start-up. \\
    \hline
    \hyperref[subsec:nm_sec_config_AM_surfaceAreaWeightedAverages_compute_interval]{config\_AM\_surfaceArea\-WeightedAverages\_compute\_\-interval} & Time interval the determines how frequently the surface area weighted averages analysis member should be computed. \\
    \hline
    \hyperref[subsec:nm_sec_config_AM_surfaceAreaWeightedAverages_output_stream]{config\_AM\_surfaceArea\-WeightedAverages\_output\_\-stream} & Name of the stream the surface area weighted averages analysis member should be tied to. \\
    \hline
\end{longtable}
\end{center}
}
\section[AM\_waterMassCensus]{\hyperref[sec:nm_sec_AM_waterMassCensus]{AM\_waterMassCensus}}
\label{sec:nm_tab_AM_waterMassCensus}
\vspace{0.5in}
{\small
\begin{center}
\begin{longtable}{| p{2.0in} || p{4.0in} |}
    \hline
    {\bf Name} & {\bf Description} \endfirsthead
    \hline 
    {\bf Name} & {\bf Description} (Continued) \endhead
    \hline
    \hline
    \hyperref[subsec:nm_sec_config_AM_waterMassCensus_enable]{config\_AM\_waterMassCensus\_\-enable} & If true, ocean analysis member water mass census is called. \\
    \hline
    \hyperref[subsec:nm_sec_config_AM_waterMassCensus_compute_interval]{config\_AM\_waterMassCensus\_\-compute\_interval} & Timestamp determining how often analysis member computation should be performed. \\
    \hline
    \hyperref[subsec:nm_sec_config_AM_waterMassCensus_output_stream]{config\_AM\_waterMassCensus\_\-output\_stream} & Name of the stream the water mass census analysis member should be tied to. \\
    \hline
    \hyperref[subsec:nm_sec_config_AM_waterMassCensus_compute_on_startup]{config\_AM\_waterMassCensus\_\-compute\_on\_startup} & Logical flag determining if an analysis member computation occurs on start-up. \\
    \hline
    \hyperref[subsec:nm_sec_config_AM_waterMassCensus_write_on_startup]{config\_AM\_waterMassCensus\_\-write\_on\_startup} & Logical flag determining if an analysis member output occurs on start-up. \\
    \hline
    \hyperref[subsec:nm_sec_config_AM_waterMassCensus_minTemperature]{config\_AM\_waterMassCensus\_\-minTemperature} & minimum temperature used in water mass census \\
    \hline
    \hyperref[subsec:nm_sec_config_AM_waterMassCensus_maxTemperature]{config\_AM\_waterMassCensus\_\-maxTemperature} & maximum temperature used in water mass census \\
    \hline
    \hyperref[subsec:nm_sec_config_AM_waterMassCensus_minSalinity]{config\_AM\_waterMassCensus\_\-minSalinity} & minimum salinity used in water mass census \\
    \hline
    \hyperref[subsec:nm_sec_config_AM_waterMassCensus_maxSalinity]{config\_AM\_waterMassCensus\_\-maxSalinity} & maximum salinity used in water mass census \\
    \hline
    \hyperref[subsec:nm_sec_config_AM_waterMassCensus_compute_predefined_regions]{config\_AM\_waterMassCensus\_\-compute\_predefined\_regions} & Computes predefined regions. (Does not require a region mask file.) \\
    \hline
    \hyperref[subsec:nm_sec_config_AM_waterMassCensus_region_group]{config\_AM\_waterMassCensus\_\-region\_group} & The name of the region group, for which the WMC should be computed in addition to the existing WMC. \\
    \hline
\end{longtable}
\end{center}
}
\section[AM\_layerVolumeWeightedAverage]{\hyperref[sec:nm_sec_AM_layerVolumeWeightedAverage]{AM\_layerVolumeWeightedAverage}}
\label{sec:nm_tab_AM_layerVolumeWeightedAverage}
\vspace{0.5in}
{\small
\begin{center}
\begin{longtable}{| p{2.0in} || p{4.0in} |}
    \hline
    {\bf Name} & {\bf Description} \endfirsthead
    \hline 
    {\bf Name} & {\bf Description} (Continued) \endhead
    \hline
    \hline
    \hyperref[subsec:nm_sec_config_AM_layerVolumeWeightedAverage_enable]{config\_AM\_layerVolume\-WeightedAverage\_enable} & If true, ocean analysis member layer-volume weighted is called. \\
    \hline
    \hyperref[subsec:nm_sec_config_AM_layerVolumeWeightedAverage_compute_interval]{config\_AM\_layerVolume\-WeightedAverage\_compute\_\-interval} & Timestamp determining how often analysis member computation should be performed. \\
    \hline
    \hyperref[subsec:nm_sec_config_AM_layerVolumeWeightedAverage_compute_on_startup]{config\_AM\_layerVolume\-WeightedAverage\_compute\_on\_\-startup} & Logical flag determining if an analysis member computation occurs on start-up. \\
    \hline
    \hyperref[subsec:nm_sec_config_AM_layerVolumeWeightedAverage_write_on_startup]{config\_AM\_layerVolume\-WeightedAverage\_write\_on\_\-startup} & Logical flag determining if an analysis member output write occurs on start-up. \\
    \hline
    \hyperref[subsec:nm_sec_config_AM_layerVolumeWeightedAverage_output_stream]{config\_AM\_layerVolume\-WeightedAverage\_output\_\-stream} & Name of the string that should be tied to the layer volume weighted average analysis member \\
    \hline
\end{longtable}
\end{center}
}
\section[AM\_zonalMean]{\hyperref[sec:nm_sec_AM_zonalMean]{AM\_zonalMean}}
\label{sec:nm_tab_AM_zonalMean}
\vspace{0.5in}
{\small
\begin{center}
\begin{longtable}{| p{2.0in} || p{4.0in} |}
    \hline
    {\bf Name} & {\bf Description} \endfirsthead
    \hline 
    {\bf Name} & {\bf Description} (Continued) \endhead
    \hline
    \hline
    \hyperref[subsec:nm_sec_config_AM_zonalMean_enable]{config\_AM\_zonalMean\_enable} & If true, ocean analysis member zonal\_mean is called. \\
    \hline
    \hyperref[subsec:nm_sec_config_AM_zonalMean_compute_on_startup]{config\_AM\_zonalMean\_\-compute\_on\_startup} & Logical flag determining if an analysis member computation occurs on start-up. \\
    \hline
    \hyperref[subsec:nm_sec_config_AM_zonalMean_write_on_startup]{config\_AM\_zonalMean\_write\_\-on\_startup} & Logical flag determining if an analysis member output occurs on start-up. \\
    \hline
    \hyperref[subsec:nm_sec_config_AM_zonalMean_compute_interval]{config\_AM\_zonalMean\_\-compute\_interval} & Interval that determines frequency of computation for the zonal mean analysis member. \\
    \hline
    \hyperref[subsec:nm_sec_config_AM_zonalMean_output_stream]{config\_AM\_zonalMean\_\-output\_stream} & Name of stream the zonal mean analysis member should be tied to. \\
    \hline
    \hyperref[subsec:nm_sec_config_AM_zonalMean_num_bins]{config\_AM\_zonalMean\_num\_\-bins} & Number of bins used for zonal mean.  Must be less than or equal to the dimension nZonalMeanBins (set in Registry). \\
    \hline
    \hyperref[subsec:nm_sec_config_AM_zonalMean_min_bin]{config\_AM\_zonalMean\_min\_\-bin} & minimum bin boundary value.  If set to -1.0e34, the minimum value in the domain is found. \\
    \hline
    \hyperref[subsec:nm_sec_config_AM_zonalMean_max_bin]{config\_AM\_zonalMean\_max\_\-bin} & maximum bin boundary value.  If set to -1.0e34, the maximum value in the domain is found. \\
    \hline
\end{longtable}
\end{center}
}
\section[AM\_okuboWeiss]{\hyperref[sec:nm_sec_AM_okuboWeiss]{AM\_okuboWeiss}}
\label{sec:nm_tab_AM_okuboWeiss}
\vspace{0.5in}
{\small
\begin{center}
\begin{longtable}{| p{2.0in} || p{4.0in} |}
    \hline
    {\bf Name} & {\bf Description} \endfirsthead
    \hline 
    {\bf Name} & {\bf Description} (Continued) \endhead
    \hline
    \hline
    \hyperref[subsec:nm_sec_config_AM_okuboWeiss_enable]{config\_AM\_okuboWeiss\_enable} & If true, ocean analysis member okubo\_weiss is called. \\
    \hline
    \hyperref[subsec:nm_sec_config_AM_okuboWeiss_compute_on_startup]{config\_AM\_okuboWeiss\_\-compute\_on\_startup} & Logical flag determining if an analysis member computation occurs on start-up. \\
    \hline
    \hyperref[subsec:nm_sec_config_AM_okuboWeiss_write_on_startup]{config\_AM\_okuboWeiss\_write\_\-on\_startup} & Logical flag determining if an analysis member computation occurs on start-up. \\
    \hline
    \hyperref[subsec:nm_sec_config_AM_okuboWeiss_compute_interval]{config\_AM\_okuboWeiss\_\-compute\_interval} & Time stamp for frequency of computation of the okubo weiss analysis member. \\
    \hline
    \hyperref[subsec:nm_sec_config_AM_okuboWeiss_output_stream]{config\_AM\_okuboWeiss\_\-output\_stream} & Name of stream the okubo weiss analysis member should be tied to \\
    \hline
    \hyperref[subsec:nm_sec_config_AM_okuboWeiss_directory]{config\_AM\_okuboWeiss\_\-directory} & subdirectory to write eddy census text files \\
    \hline
    \hyperref[subsec:nm_sec_config_AM_okuboWeiss_threshold_value]{config\_AM\_okuboWeiss\_\-threshold\_value} & Threshold below which normalized OW values are counted as eddies, typically -0.2 \\
    \hline
    \hyperref[subsec:nm_sec_config_AM_okuboWeiss_normalization]{config\_AM\_okuboWeiss\_\-normalization} & Parameter by which the OW values are normalized, typically the standard deviation of OW \\
    \hline
    \hyperref[subsec:nm_sec_config_AM_okuboWeiss_lambda2_normalization]{config\_AM\_okuboWeiss\_\-lambda2\_normalization} & Parameter by which the lambda\_2 values are normalized, typically the standard deviation of lambda\_2 \\
    \hline
    \hyperref[subsec:nm_sec_config_AM_okuboWeiss_use_lat_lon_coords]{config\_AM\_okuboWeiss\_use\_\-lat\_lon\_coords} & If true, latitude/longitude coordinates are output for eddy census. Otherwise x/y/z coordinates are used. Ignored if not on a sphere. \\
    \hline
    \hyperref[subsec:nm_sec_config_AM_okuboWeiss_compute_eddy_census]{config\_AM\_okuboWeiss\_\-compute\_eddy\_census} & If true, connected components of thresholded OW values are computed, and used to compute an eddy census. \\
    \hline
    \hyperref[subsec:nm_sec_config_AM_okuboWeiss_eddy_min_cells]{config\_AM\_okuboWeiss\_eddy\_\-min\_cells} & Minimum number of cells that a connected component must contain to be considered an eddy. This needs to be scaled based on expected eddy size given a grid resolution. \\
    \hline
\end{longtable}
\end{center}
}
\section[AM\_meridionalHeatTransport]{\hyperref[sec:nm_sec_AM_meridionalHeatTransport]{AM\_meridionalHeatTransport}}
\label{sec:nm_tab_AM_meridionalHeatTransport}
\vspace{0.5in}
{\small
\begin{center}
\begin{longtable}{| p{2.0in} || p{4.0in} |}
    \hline
    {\bf Name} & {\bf Description} \endfirsthead
    \hline 
    {\bf Name} & {\bf Description} (Continued) \endhead
    \hline
    \hline
    \hyperref[subsec:nm_sec_config_AM_meridionalHeatTransport_enable]{config\_AM\_meridionalHeat\-Transport\_enable} & If true, ocean analysis member meridional\_heat\_transport is called. \\
    \hline
    \hyperref[subsec:nm_sec_config_AM_meridionalHeatTransport_compute_interval]{config\_AM\_meridionalHeat\-Transport\_compute\_interval} & Timestamp determining how often analysis member computation should be performed. \\
    \hline
    \hyperref[subsec:nm_sec_config_AM_meridionalHeatTransport_compute_on_startup]{config\_AM\_meridionalHeat\-Transport\_compute\_on\_startup} & Logical flag determining if an analysis member computation occurs on start-up. \\
    \hline
    \hyperref[subsec:nm_sec_config_AM_meridionalHeatTransport_write_on_startup]{config\_AM\_meridionalHeat\-Transport\_write\_on\_startup} & Logical flag determining if an analysis member output occurs on start-up. \\
    \hline
    \hyperref[subsec:nm_sec_config_AM_meridionalHeatTransport_output_stream]{config\_AM\_meridionalHeat\-Transport\_output\_stream} & Name of the stream that the meridional heat transport analysis member should be tied to. \\
    \hline
    \hyperref[subsec:nm_sec_config_AM_meridionalHeatTransport_num_bins]{config\_AM\_meridionalHeat\-Transport\_num\_bins} & Number of bins used for meridional heat transport. \\
    \hline
    \hyperref[subsec:nm_sec_config_AM_meridionalHeatTransport_min_bin]{config\_AM\_meridionalHeat\-Transport\_min\_bin} & minimum bin boundary value.  If set to -1.0e34, the minimum value in the domain is found. \\
    \hline
    \hyperref[subsec:nm_sec_config_AM_meridionalHeatTransport_max_bin]{config\_AM\_meridionalHeat\-Transport\_max\_bin} & maximum bin boundary value.  If set to -1.0e34, the maximum value in the domain is found. \\
    \hline
    \hyperref[subsec:nm_sec_config_AM_meridionalHeatTransport_region_group]{config\_AM\_meridionalHeat\-Transport\_region\_group} & The name of the region group, for which the MHT should be computed in addition to the global MHT. \\
    \hline
\end{longtable}
\end{center}
}
\section[AM\_testComputeInterval]{\hyperref[sec:nm_sec_AM_testComputeInterval]{AM\_testComputeInterval}}
\label{sec:nm_tab_AM_testComputeInterval}
\vspace{0.5in}
{\small
\begin{center}
\begin{longtable}{| p{2.0in} || p{4.0in} |}
    \hline
    {\bf Name} & {\bf Description} \endfirsthead
    \hline 
    {\bf Name} & {\bf Description} (Continued) \endhead
    \hline
    \hline
    \hyperref[subsec:nm_sec_config_AM_testComputeInterval_enable]{config\_AM\_testCompute\-Interval\_enable} & If true, ocean analysis member test\_compute\_interval is called. \\
    \hline
    \hyperref[subsec:nm_sec_config_AM_testComputeInterval_compute_interval]{config\_AM\_testCompute\-Interval\_compute\_interval} & Timestamp determining how often analysis member computation should be performed. \\
    \hline
    \hyperref[subsec:nm_sec_config_AM_testComputeInterval_compute_on_startup]{config\_AM\_testCompute\-Interval\_compute\_on\_startup} & Logical flag determining if an analysis member computation occurs on start-up. \\
    \hline
    \hyperref[subsec:nm_sec_config_AM_testComputeInterval_write_on_startup]{config\_AM\_testCompute\-Interval\_write\_on\_startup} & Logical flag determining if an analysis member write occurs on start-up. \\
    \hline
    \hyperref[subsec:nm_sec_config_AM_testComputeInterval_output_stream]{config\_AM\_testCompute\-Interval\_output\_stream} & Name of the stream that should be tied to the test\_compute\_interval analysis member \\
    \hline
\end{longtable}
\end{center}
}
\section[AM\_highFrequencyOutput]{\hyperref[sec:nm_sec_AM_highFrequencyOutput]{AM\_highFrequencyOutput}}
\label{sec:nm_tab_AM_highFrequencyOutput}
\vspace{0.5in}
{\small
\begin{center}
\begin{longtable}{| p{2.0in} || p{4.0in} |}
    \hline
    {\bf Name} & {\bf Description} \endfirsthead
    \hline 
    {\bf Name} & {\bf Description} (Continued) \endhead
    \hline
    \hline
    \hyperref[subsec:nm_sec_config_AM_highFrequencyOutput_enable]{config\_AM\_highFrequency\-Output\_enable} & If true, ocean analysis member highFrequencyOutput is called. \\
    \hline
    \hyperref[subsec:nm_sec_config_AM_highFrequencyOutput_compute_interval]{config\_AM\_highFrequency\-Output\_compute\_interval} & Timestamp determining how often analysis member computation should be performed. \\
    \hline
    \hyperref[subsec:nm_sec_config_AM_highFrequencyOutput_output_stream]{config\_AM\_highFrequency\-Output\_output\_stream} & Name of the stream that the highFrequencyOutput analysis member should be tied to. \\
    \hline
    \hyperref[subsec:nm_sec_config_AM_highFrequencyOutput_compute_on_startup]{config\_AM\_highFrequency\-Output\_compute\_on\_startup} & Logical flag determining if an analysis member computation occurs on start-up. \\
    \hline
    \hyperref[subsec:nm_sec_config_AM_highFrequencyOutput_write_on_startup]{config\_AM\_highFrequency\-Output\_write\_on\_startup} & Logical flag determining if an analysis member write occurs on start-up. \\
    \hline
\end{longtable}
\end{center}
}
\section[AM\_timeFilters]{\hyperref[sec:nm_sec_AM_timeFilters]{AM\_timeFilters}}
\label{sec:nm_tab_AM_timeFilters}
\vspace{0.5in}
{\small
\begin{center}
\begin{longtable}{| p{2.0in} || p{4.0in} |}
    \hline
    {\bf Name} & {\bf Description} \endfirsthead
    \hline 
    {\bf Name} & {\bf Description} (Continued) \endhead
    \hline
    \hline
    \hyperref[subsec:nm_sec_config_AM_timeFilters_enable]{config\_AM\_timeFilters\_enable} & If true, ocean analysis member timeFilters is called. \\
    \hline
    \hyperref[subsec:nm_sec_config_AM_timeFilters_compute_interval]{config\_AM\_timeFilters\_\-compute\_interval} & Timestamp determining how often analysis member computation should be performed. \\
    \hline
    \hyperref[subsec:nm_sec_config_AM_timeFilters_output_stream]{config\_AM\_timeFilters\_\-output\_stream} & Name of the stream that the timeFilters analysis member should be tied to. \\
    \hline
    \hyperref[subsec:nm_sec_config_AM_timeFilters_restart_stream]{config\_AM\_timeFilters\_\-restart\_stream} & Name of the stream that the timeFilters analysis member should use to perform restarts. \\
    \hline
    \hyperref[subsec:nm_sec_config_AM_timeFilters_compute_on_startup]{config\_AM\_timeFilters\_\-compute\_on\_startup} & Logical flag determining if an analysis member computation occurs on start-up. \\
    \hline
    \hyperref[subsec:nm_sec_config_AM_timeFilters_write_on_startup]{config\_AM\_timeFilters\_write\_\-on\_startup} & Logical flag determining if an analysis member write occurs on start-up. \\
    \hline
    \hyperref[subsec:nm_sec_config_AM_timeFilters_initialize_filters]{config\_AM\_timeFilters\_\-initialize\_filters} & Logical flag determining if filters should be initialized on start-up. \\
    \hline
    \hyperref[subsec:nm_sec_config_AM_timeFilters_tau]{config\_AM\_timeFilters\_tau} & Cutoff time scale $\tau$ for high and low pass filtering (default is 90 days). \\
    \hline
    \hyperref[subsec:nm_sec_config_AM_timeFilters_compute_cell_centered_values]{config\_AM\_timeFilters\_\-compute\_cell\_centered\_values} & Logical flag determining if cell centered values should be computed. \\
    \hline
\end{longtable}
\end{center}
}
\section[AM\_lagrPartTrack]{\hyperref[sec:nm_sec_AM_lagrPartTrack]{AM\_lagrPartTrack}}
\label{sec:nm_tab_AM_lagrPartTrack}
\vspace{0.5in}
{\small
\begin{center}
\begin{longtable}{| p{2.0in} || p{4.0in} |}
    \hline
    {\bf Name} & {\bf Description} \endfirsthead
    \hline 
    {\bf Name} & {\bf Description} (Continued) \endhead
    \hline
    \hline
    \hyperref[subsec:nm_sec_config_AM_lagrPartTrack_enable]{config\_AM\_lagrPartTrack\_\-enable} & If true, ocean analysis member lagrPartTrack is called. \\
    \hline
    \hyperref[subsec:nm_sec_config_AM_lagrPartTrack_compute_interval]{config\_AM\_lagrPartTrack\_\-compute\_interval} & Timestamp determining how often analysis member computation should be performed. \\
    \hline
    \hyperref[subsec:nm_sec_config_AM_lagrPartTrack_compute_on_startup]{config\_AM\_lagrPartTrack\_\-compute\_on\_startup} & Logical flag determining if an analysis member computation occurs on start-up. \\
    \hline
    \hyperref[subsec:nm_sec_config_AM_lagrPartTrack_output_stream]{config\_AM\_lagrPartTrack\_\-output\_stream} & Name of the stream that the lagrPartTrack analysis member should be tied to. \\
    \hline
    \hyperref[subsec:nm_sec_config_AM_lagrPartTrack_restart_stream]{config\_AM\_lagrPartTrack\_\-restart\_stream} & Name of the stream that the lagrPartTrack analysis member should use to perform restarts. \\
    \hline
    \hyperref[subsec:nm_sec_config_AM_lagrPartTrack_input_stream]{config\_AM\_lagrPartTrack\_\-input\_stream} & Name of the stream that the lagrPartTrack analysis member should read only in a non-restart run. \\
    \hline
    \hyperref[subsec:nm_sec_config_AM_lagrPartTrack_write_on_startup]{config\_AM\_lagrPartTrack\_\-write\_on\_startup} & Logical flag determining if an analysis member write occurs on start-up. \\
    \hline
    \hyperref[subsec:nm_sec_config_AM_lagrPartTrack_filter_number]{config\_AM\_lagrPartTrack\_\-filter\_number} & Number of times to apply filtering operation. \\
    \hline
    \hyperref[subsec:nm_sec_config_AM_lagrPartTrack_reset_criteria]{config\_AM\_lagrPartTrack\_\-reset\_criteria} & Specify whether particles should not be reset ('none'), be reset on a timer for each particle ('particle\_time'), be reset on config\_AM\_lagrPartTrack\_reset\_time\_globally value ('global\_time'), be reset based on regions ('region'), or be reset for all conditions ('all'). \\
    \hline
    \hyperref[subsec:nm_sec_config_AM_lagrPartTrack_reset_global_timestamp]{config\_AM\_lagrPartTrack\_\-reset\_global\_timestamp} & Specify reset global timestamp interval. \\
    \hline
    \hyperref[subsec:nm_sec_config_AM_lagrPartTrack_region_stream]{config\_AM\_lagrPartTrack\_\-region\_stream} & Name of the stream that has region arrays resetOutsideRegionMaskValue1 and resetInsideRegionMaskValue1 for region-based particle resets. \\
    \hline
    \hyperref[subsec:nm_sec_config_AM_lagrPartTrack_reset_if_outside_region]{config\_AM\_lagrPartTrack\_\-reset\_if\_outside\_region} & Specify whether particles should be reset when they leave the resetOutsideRegionMaskValue1 mask. \\
    \hline
    \hyperref[subsec:nm_sec_config_AM_lagrPartTrack_reset_if_inside_region]{config\_AM\_lagrPartTrack\_\-reset\_if\_inside\_region} & Specify whether particles should be reset when they enter the resetInsideRegionMaskValue1 mask. \\
    \hline
\end{longtable}
\end{center}
}
\section[AM\_eliassenPalm]{\hyperref[sec:nm_sec_AM_eliassenPalm]{AM\_eliassenPalm}}
\label{sec:nm_tab_AM_eliassenPalm}
\vspace{0.5in}
{\small
\begin{center}
\begin{longtable}{| p{2.0in} || p{4.0in} |}
    \hline
    {\bf Name} & {\bf Description} \endfirsthead
    \hline 
    {\bf Name} & {\bf Description} (Continued) \endhead
    \hline
    \hline
    \hyperref[subsec:nm_sec_config_AM_eliassenPalm_enable]{config\_AM\_eliassenPalm\_\-enable} & If true, ocean analysis member eliassenPalm is called. \\
    \hline
    \hyperref[subsec:nm_sec_config_AM_eliassenPalm_compute_interval]{config\_AM\_eliassenPalm\_\-compute\_interval} & Timestamp determining how often analysis member computation should be performed. \\
    \hline
    \hyperref[subsec:nm_sec_config_AM_eliassenPalm_output_stream]{config\_AM\_eliassenPalm\_\-output\_stream} & Name of the stream that the eliassenPalm analysis member should be tied to. \\
    \hline
    \hyperref[subsec:nm_sec_config_AM_eliassenPalm_restart_stream]{config\_AM\_eliassenPalm\_\-restart\_stream} & Name of the stream that the eliassenPalm analysis member will use to performing restarts. \\
    \hline
    \hyperref[subsec:nm_sec_config_AM_eliassenPalm_compute_on_startup]{config\_AM\_eliassenPalm\_\-compute\_on\_startup} & Logical flag determining if an analysis member computation occurs on start-up. \\
    \hline
    \hyperref[subsec:nm_sec_config_AM_eliassenPalm_write_on_startup]{config\_AM\_eliassenPalm\_\-write\_on\_startup} & Logical flag determining if an analysis member write occurs on start-up. \\
    \hline
    \hyperref[subsec:nm_sec_config_AM_eliassenPalm_debug]{config\_AM\_eliassenPalm\_debug} & If true, debugging code is turned on. \\
    \hline
    \hyperref[subsec:nm_sec_config_AM_eliassenPalm_nBuoyancyLayers]{config\_AM\_eliassenPalm\_n\-BuoyancyLayers} & Number of reference buoyancy layers. \\
    \hline
    \hyperref[subsec:nm_sec_config_AM_eliassenPalm_rhomin_buoycoor]{config\_AM\_eliassenPalm\_\-rhomin\_buoycoor} & Minimum density used in defining the first buoyancy coordinate layer \\
    \hline
    \hyperref[subsec:nm_sec_config_AM_eliassenPalm_rhomax_buoycoor]{config\_AM\_eliassenPalm\_\-rhomax\_buoycoor} & Maximum density used in defining the last buoyancy coordinate layer \\
    \hline
\end{longtable}
\end{center}
}
\section[AM\_mixedLayerDepths]{\hyperref[sec:nm_sec_AM_mixedLayerDepths]{AM\_mixedLayerDepths}}
\label{sec:nm_tab_AM_mixedLayerDepths}
\vspace{0.5in}
{\small
\begin{center}
\begin{longtable}{| p{2.0in} || p{4.0in} |}
    \hline
    {\bf Name} & {\bf Description} \endfirsthead
    \hline 
    {\bf Name} & {\bf Description} (Continued) \endhead
    \hline
    \hline
    \hyperref[subsec:nm_sec_config_AM_mixedLayerDepths_enable]{config\_AM\_mixedLayerDepths\_\-enable} & If true, ocean analysis member mixedLayerDepth is called. \\
    \hline
    \hyperref[subsec:nm_sec_config_AM_mixedLayerDepths_compute_interval]{config\_AM\_mixedLayerDepths\_\-compute\_interval} & Timestamp determining how often analysis member computation should be performed. \\
    \hline
    \hyperref[subsec:nm_sec_config_AM_mixedLayerDepths_output_stream]{config\_AM\_mixedLayerDepths\_\-output\_stream} & Name of the stream that the temPlate analysis member should be tied to. \\
    \hline
    \hyperref[subsec:nm_sec_config_AM_mixedLayerDepths_write_on_startup]{config\_AM\_mixedLayerDepths\_\-write\_on\_startup} & Logical flag determining if an analysis member write occurs on start-up. \\
    \hline
    \hyperref[subsec:nm_sec_config_AM_mixedLayerDepths_compute_on_startup]{config\_AM\_mixedLayerDepths\_\-compute\_on\_startup} & Logical flag determining if an analysis member computation occurs on start-up \\
    \hline
    \hyperref[subsec:nm_sec_config_AM_mixedLayerDepths_Tthreshold]{config\_AM\_mixedLayerDepths\_\-Tthreshold} & Logical flag that determines if MLDs are calculated using a critical temperature threshold \\
    \hline
    \hyperref[subsec:nm_sec_config_AM_mixedLayerDepths_Dthreshold]{config\_AM\_mixedLayerDepths\_\-Dthreshold} & Logical flag that determines if MLDs are calculated using a critical density threshold \\
    \hline
    \hyperref[subsec:nm_sec_config_AM_mixedLayerDepths_crit_temp_threshold]{config\_AM\_mixedLayerDepths\_\-crit\_temp\_threshold} & temperature change relative to surface for threshold method \\
    \hline
    \hyperref[subsec:nm_sec_config_AM_mixedLayerDepths_crit_dens_threshold]{config\_AM\_mixedLayerDepths\_\-crit\_dens\_threshold} & potential density change relative to surface for threshold method \\
    \hline
    \hyperref[subsec:nm_sec_config_AM_mixedLayerDepths_reference_pressure]{config\_AM\_mixedLayerDepths\_\-reference\_pressure} & reference pressure for threshold computation \\
    \hline
    \hyperref[subsec:nm_sec_config_AM_mixedLayerDepths_Tgradient]{config\_AM\_mixedLayerDepths\_\-Tgradient} & Logical flag controlling whether or not to compute MLDs via the temperature gradient \\
    \hline
    \hyperref[subsec:nm_sec_config_AM_mixedLayerDepths_Dgradient]{config\_AM\_mixedLayerDepths\_\-Dgradient} & Logical flag controlling whether or not to compute MLDs via the density gradient \\
    \hline
    \hyperref[subsec:nm_sec_config_AM_mixedLayerDepths_temp_gradient_threshold]{config\_AM\_mixedLayerDepths\_\-temp\_gradient\_threshold} & temp gradient crit value, if not exceeded max gradient used \\
    \hline
    \hyperref[subsec:nm_sec_config_AM_mixedLayerDepths_den_gradient_threshold]{config\_AM\_mixedLayerDepths\_\-den\_gradient\_threshold} & potential density gradient crit value.  If not exceeded max gradient used \\
    \hline
    \hyperref[subsec:nm_sec_config_AM_mixedLayerDepths_interp_method]{config\_AM\_mixedLayerDepths\_\-interp\_method} & flag specifying which interpolation method to use in computations \\
    \hline
\end{longtable}
\end{center}
}
\section[AM\_regionalStatsDaily]{\hyperref[sec:nm_sec_AM_regionalStatsDaily]{AM\_regionalStatsDaily}}
\label{sec:nm_tab_AM_regionalStatsDaily}
\vspace{0.5in}
{\small
\begin{center}
\begin{longtable}{| p{2.0in} || p{4.0in} |}
    \hline
    {\bf Name} & {\bf Description} \endfirsthead
    \hline 
    {\bf Name} & {\bf Description} (Continued) \endhead
    \hline
    \hline
    \hyperref[subsec:nm_sec_config_AM_regionalStatsDaily_enable]{config\_AM\_regionalStatsDaily\_\-enable} & If true, ocean analysis member regional stats is called. \\
    \hline
    \hyperref[subsec:nm_sec_config_AM_regionalStatsDaily_compute_on_startup]{config\_AM\_regionalStatsDaily\_\-compute\_on\_startup} & Logical flag determining if an analysis member computation occurs on start-up. \\
    \hline
    \hyperref[subsec:nm_sec_config_AM_regionalStatsDaily_write_on_startup]{config\_AM\_regionalStatsDaily\_\-write\_on\_startup} & Logical flag determining if an analysis member output occurs on start-up. \\
    \hline
    \hyperref[subsec:nm_sec_config_AM_regionalStatsDaily_compute_interval]{config\_AM\_regionalStatsDaily\_\-compute\_interval} & Interval that determines frequency of computation for the regional stats analysis member. \\
    \hline
    \hyperref[subsec:nm_sec_config_AM_regionalStatsDaily_output_stream]{config\_AM\_regionalStatsDaily\_\-output\_stream} & Name of stream the regional stats analysis member will operate on that contains the list of input fields (and will be modified to contain the output stats fields). \\
    \hline
    \hyperref[subsec:nm_sec_config_AM_regionalStatsDaily_restart_stream]{config\_AM\_regionalStatsDaily\_\-restart\_stream} & Name of stream the regional stats analysis member will use for the mask/region data. \\
    \hline
    \hyperref[subsec:nm_sec_config_AM_regionalStatsDaily_input_stream]{config\_AM\_regionalStatsDaily\_\-input\_stream} & Name of stream the regional stats analysis member will use for the mask/region data. \\
    \hline
    \hyperref[subsec:nm_sec_config_AM_regionalStatsDaily_operation]{config\_AM\_regionalStatsDaily\_\-operation} & An operation describing the statistic to apply to all variables in the output stream. \\
    \hline
    \hyperref[subsec:nm_sec_config_AM_regionalStatsDaily_region_type]{config\_AM\_regionalStatsDaily\_\-region\_type} & The reduced dimension of the region masks that will be used during the regional stats operation. Needs to be the last dimension, and the same dimension as all of the reduced fields, weight fields, and masks. \\
    \hline
    \hyperref[subsec:nm_sec_config_AM_regionalStatsDaily_region_group]{config\_AM\_regionalStatsDaily\_\-region\_group} & The name of the group of region masks that will be used to subset the mesh during the regional stats operation. \\
    \hline
    \hyperref[subsec:nm_sec_config_AM_regionalStatsDaily_1d_weighting_function]{config\_AM\_regionalStatsDaily\_\-1d\_weighting\_function} & An operation applied to every element in a region WITHOUT a vertical dimension, with a 1D weighting field, prior to an average operation. The average is normalized by the sum of the weight field in the region (divided by the sum of regional weight values). \\
    \hline
    \hyperref[subsec:nm_sec_config_AM_regionalStatsDaily_2d_weighting_function]{config\_AM\_regionalStatsDaily\_\-2d\_weighting\_function} & An operation applied to every element in a region WITH a vertical dimension, with a 2D weighting field, prior to an average operation. The average is normalized by the sum of the weight field in the region (divided by the sum of regional weight values). \\
    \hline
    \hyperref[subsec:nm_sec_config_AM_regionalStatsDaily_1d_weighting_field]{config\_AM\_regionalStatsDaily\_\-1d\_weighting\_field} & A 1D real field used in conjunction with the 1D weighting function, to be used as a weighting scale factor (like area). \\
    \hline
    \hyperref[subsec:nm_sec_config_AM_regionalStatsDaily_2d_weighting_field]{config\_AM\_regionalStatsDaily\_\-2d\_weighting\_field} & A 2D real field used in conjunction with the 2D weighting function, to be used as a weighting scale factor (like area). \\
    \hline
    \hyperref[subsec:nm_sec_config_AM_regionalStatsDaily_vertical_mask]{config\_AM\_regionalStatsDaily\_\-vertical\_mask} & An additional 2D vertical integer mask field, which is used in conjunction with the regional masks. Used in cases when an input field has a second dimension that matches the vertical mask dimension. \\
    \hline
    \hyperref[subsec:nm_sec_config_AM_regionalStatsDaily_vertical_dimension]{config\_AM\_regionalStatsDaily\_\-vertical\_dimension} & The second dimension to be used for additional vertical mask. \\
    \hline
\end{longtable}
\end{center}
}
\section[AM\_regionalStatsWeekly]{\hyperref[sec:nm_sec_AM_regionalStatsWeekly]{AM\_regionalStatsWeekly}}
\label{sec:nm_tab_AM_regionalStatsWeekly}
\vspace{0.5in}
{\small
\begin{center}
\begin{longtable}{| p{2.0in} || p{4.0in} |}
    \hline
    {\bf Name} & {\bf Description} \endfirsthead
    \hline 
    {\bf Name} & {\bf Description} (Continued) \endhead
    \hline
    \hline
    \hyperref[subsec:nm_sec_config_AM_regionalStatsWeekly_enable]{config\_AM\_regionalStats\-Weekly\_enable} & If true, ocean analysis member regional stats is called. \\
    \hline
    \hyperref[subsec:nm_sec_config_AM_regionalStatsWeekly_compute_on_startup]{config\_AM\_regionalStats\-Weekly\_compute\_on\_startup} & Logical flag determining if an analysis member computation occurs on start-up. \\
    \hline
    \hyperref[subsec:nm_sec_config_AM_regionalStatsWeekly_write_on_startup]{config\_AM\_regionalStats\-Weekly\_write\_on\_startup} & Logical flag determining if an analysis member output occurs on start-up. \\
    \hline
    \hyperref[subsec:nm_sec_config_AM_regionalStatsWeekly_compute_interval]{config\_AM\_regionalStats\-Weekly\_compute\_interval} & Interval that determines frequency of computation for the regional stats analysis member. \\
    \hline
    \hyperref[subsec:nm_sec_config_AM_regionalStatsWeekly_output_stream]{config\_AM\_regionalStats\-Weekly\_output\_stream} & Name of stream the regional stats analysis member will operate on that contains the list of input fields (and will be modified to contain the output stats fields). \\
    \hline
    \hyperref[subsec:nm_sec_config_AM_regionalStatsWeekly_restart_stream]{config\_AM\_regionalStats\-Weekly\_restart\_stream} & Name of stream the regional stats analysis member will use for the mask/region data. \\
    \hline
    \hyperref[subsec:nm_sec_config_AM_regionalStatsWeekly_input_stream]{config\_AM\_regionalStats\-Weekly\_input\_stream} & Name of stream the regional stats analysis member will use for the mask/region data. \\
    \hline
    \hyperref[subsec:nm_sec_config_AM_regionalStatsWeekly_operation]{config\_AM\_regionalStats\-Weekly\_operation} & An operation describing the statistic to apply to all variables in the output stream. \\
    \hline
    \hyperref[subsec:nm_sec_config_AM_regionalStatsWeekly_region_type]{config\_AM\_regionalStats\-Weekly\_region\_type} & The reduced dimension of the region masks that will be used during the regional stats operation. Needs to be the last dimension, and the same dimension as all of the reduced fields, weight fields, and masks. \\
    \hline
    \hyperref[subsec:nm_sec_config_AM_regionalStatsWeekly_region_group]{config\_AM\_regionalStats\-Weekly\_region\_group} & The name of the group of region masks that will be used to subset the mesh during the regional stats operation. \\
    \hline
    \hyperref[subsec:nm_sec_config_AM_regionalStatsWeekly_1d_weighting_function]{config\_AM\_regionalStats\-Weekly\_1d\_weighting\_function} & An operation applied to every element in a region WITHOUT a vertical dimension, with a 1D weighting field, prior to an average operation. The average is normalized by the sum of the weight field in the region (divided by the sum of regional weight values). \\
    \hline
    \hyperref[subsec:nm_sec_config_AM_regionalStatsWeekly_2d_weighting_function]{config\_AM\_regionalStats\-Weekly\_2d\_weighting\_function} & An operation applied to every element in a region WITH a vertical dimension, with a 2D weighting field, prior to an average operation. The average is normalized by the sum of the weight field in the region (divided by the sum of regional weight values). \\
    \hline
    \hyperref[subsec:nm_sec_config_AM_regionalStatsWeekly_1d_weighting_field]{config\_AM\_regionalStats\-Weekly\_1d\_weighting\_field} & A 1D real field used in conjunction with the 1D weighting function, to be used as a weighting scale factor (like area). \\
    \hline
    \hyperref[subsec:nm_sec_config_AM_regionalStatsWeekly_2d_weighting_field]{config\_AM\_regionalStats\-Weekly\_2d\_weighting\_field} & A 2D real field used in conjunction with the 2D weighting function, to be used as a weighting scale factor (like area). \\
    \hline
    \hyperref[subsec:nm_sec_config_AM_regionalStatsWeekly_vertical_mask]{config\_AM\_regionalStats\-Weekly\_vertical\_mask} & An additional 2D vertical integer mask field, which is used in conjunction with the regional masks. Used in cases when an input field has a second dimension that matches the vertical mask dimension. \\
    \hline
    \hyperref[subsec:nm_sec_config_AM_regionalStatsWeekly_vertical_dimension]{config\_AM\_regionalStats\-Weekly\_vertical\_dimension} & The second dimension to be used for additional vertical mask. \\
    \hline
\end{longtable}
\end{center}
}
\section[AM\_regionalStatsMonthly]{\hyperref[sec:nm_sec_AM_regionalStatsMonthly]{AM\_regionalStatsMonthly}}
\label{sec:nm_tab_AM_regionalStatsMonthly}
\vspace{0.5in}
{\small
\begin{center}
\begin{longtable}{| p{2.0in} || p{4.0in} |}
    \hline
    {\bf Name} & {\bf Description} \endfirsthead
    \hline 
    {\bf Name} & {\bf Description} (Continued) \endhead
    \hline
    \hline
    \hyperref[subsec:nm_sec_config_AM_regionalStatsMonthly_enable]{config\_AM\_regionalStats\-Monthly\_enable} & If true, ocean analysis member regional stats is called. \\
    \hline
    \hyperref[subsec:nm_sec_config_AM_regionalStatsMonthly_compute_on_startup]{config\_AM\_regionalStats\-Monthly\_compute\_on\_startup} & Logical flag determining if an analysis member computation occurs on start-up. \\
    \hline
    \hyperref[subsec:nm_sec_config_AM_regionalStatsMonthly_write_on_startup]{config\_AM\_regionalStats\-Monthly\_write\_on\_startup} & Logical flag determining if an analysis member output occurs on start-up. \\
    \hline
    \hyperref[subsec:nm_sec_config_AM_regionalStatsMonthly_compute_interval]{config\_AM\_regionalStats\-Monthly\_compute\_interval} & Interval that determines frequency of computation for the regional stats analysis member. \\
    \hline
    \hyperref[subsec:nm_sec_config_AM_regionalStatsMonthly_output_stream]{config\_AM\_regionalStats\-Monthly\_output\_stream} & Name of stream the regional stats analysis member will operate on that contains the list of input fields (and will be modified to contain the output stats fields). \\
    \hline
    \hyperref[subsec:nm_sec_config_AM_regionalStatsMonthly_restart_stream]{config\_AM\_regionalStats\-Monthly\_restart\_stream} & Name of stream the regional stats analysis member will use for the mask/region data. \\
    \hline
    \hyperref[subsec:nm_sec_config_AM_regionalStatsMonthly_input_stream]{config\_AM\_regionalStats\-Monthly\_input\_stream} & Name of stream the regional stats analysis member will use for the mask/region data. \\
    \hline
    \hyperref[subsec:nm_sec_config_AM_regionalStatsMonthly_operation]{config\_AM\_regionalStats\-Monthly\_operation} & An operation describing the statistic to apply to all variables in the output stream. \\
    \hline
    \hyperref[subsec:nm_sec_config_AM_regionalStatsMonthly_region_type]{config\_AM\_regionalStats\-Monthly\_region\_type} & The reduced dimension of the region masks that will be used during the regional stats operation. Needs to be the last dimension, and the same dimension as all of the reduced fields, weight fields, and masks. \\
    \hline
    \hyperref[subsec:nm_sec_config_AM_regionalStatsMonthly_region_group]{config\_AM\_regionalStats\-Monthly\_region\_group} & The name of the group of region masks that will be used to subset the mesh during the regional stats operation. \\
    \hline
    \hyperref[subsec:nm_sec_config_AM_regionalStatsMonthly_1d_weighting_function]{config\_AM\_regionalStats\-Monthly\_1d\_weighting\_\-function} & An operation applied to every element in a region WITHOUT a vertical dimension, with a 1D weighting field, prior to an average operation. The average is normalized by the sum of the weight field in the region (divided by the sum of regional weight values). \\
    \hline
    \hyperref[subsec:nm_sec_config_AM_regionalStatsMonthly_2d_weighting_function]{config\_AM\_regionalStats\-Monthly\_2d\_weighting\_\-function} & An operation applied to every element in a region WITH a vertical dimension, with a 2D weighting field, prior to an average operation. The average is normalized by the sum of the weight field in the region (divided by the sum of regional weight values). \\
    \hline
    \hyperref[subsec:nm_sec_config_AM_regionalStatsMonthly_1d_weighting_field]{config\_AM\_regionalStats\-Monthly\_1d\_weighting\_field} & A 1D real field used in conjunction with the 1D weighting function, to be used as a weighting scale factor (like area). \\
    \hline
    \hyperref[subsec:nm_sec_config_AM_regionalStatsMonthly_2d_weighting_field]{config\_AM\_regionalStats\-Monthly\_2d\_weighting\_field} & A 2D real field used in conjunction with the 2D weighting function, to be used as a weighting scale factor (like area). \\
    \hline
    \hyperref[subsec:nm_sec_config_AM_regionalStatsMonthly_vertical_mask]{config\_AM\_regionalStats\-Monthly\_vertical\_mask} & An additional 2D vertical integer mask field, which is used in conjunction with the regional masks. Used in cases when an input field has a second dimension that matches the vertical mask dimension. \\
    \hline
    \hyperref[subsec:nm_sec_config_AM_regionalStatsMonthly_vertical_dimension]{config\_AM\_regionalStats\-Monthly\_vertical\_dimension} & The second dimension to be used for additional vertical mask. \\
    \hline
\end{longtable}
\end{center}
}
\section[AM\_regionalStatsCustom]{\hyperref[sec:nm_sec_AM_regionalStatsCustom]{AM\_regionalStatsCustom}}
\label{sec:nm_tab_AM_regionalStatsCustom}
\vspace{0.5in}
{\small
\begin{center}
\begin{longtable}{| p{2.0in} || p{4.0in} |}
    \hline
    {\bf Name} & {\bf Description} \endfirsthead
    \hline 
    {\bf Name} & {\bf Description} (Continued) \endhead
    \hline
    \hline
    \hyperref[subsec:nm_sec_config_AM_regionalStatsCustom_enable]{config\_AM\_regionalStats\-Custom\_enable} & If true, ocean analysis member regional stats is called. \\
    \hline
    \hyperref[subsec:nm_sec_config_AM_regionalStatsCustom_compute_on_startup]{config\_AM\_regionalStats\-Custom\_compute\_on\_startup} & Logical flag determining if an analysis member computation occurs on start-up. \\
    \hline
    \hyperref[subsec:nm_sec_config_AM_regionalStatsCustom_write_on_startup]{config\_AM\_regionalStats\-Custom\_write\_on\_startup} & Logical flag determining if an analysis member output occurs on start-up. \\
    \hline
    \hyperref[subsec:nm_sec_config_AM_regionalStatsCustom_compute_interval]{config\_AM\_regionalStats\-Custom\_compute\_interval} & Interval that determines frequency of computation for the regional stats analysis member. \\
    \hline
    \hyperref[subsec:nm_sec_config_AM_regionalStatsCustom_output_stream]{config\_AM\_regionalStats\-Custom\_output\_stream} & Name of stream the regional stats analysis member will operate on that contains the list of input fields (and will be modified to contain the output stats fields). \\
    \hline
    \hyperref[subsec:nm_sec_config_AM_regionalStatsCustom_restart_stream]{config\_AM\_regionalStats\-Custom\_restart\_stream} & Name of stream the regional stats analysis member will use for the mask/region data. \\
    \hline
    \hyperref[subsec:nm_sec_config_AM_regionalStatsCustom_input_stream]{config\_AM\_regionalStats\-Custom\_input\_stream} & Name of stream the regional stats analysis member will use for the mask/region data. \\
    \hline
    \hyperref[subsec:nm_sec_config_AM_regionalStatsCustom_operation]{config\_AM\_regionalStats\-Custom\_operation} & An operation describing the statistic to apply to all variables in the output stream. \\
    \hline
    \hyperref[subsec:nm_sec_config_AM_regionalStatsCustom_region_type]{config\_AM\_regionalStats\-Custom\_region\_type} & The reduced dimension of the region masks that will be used during the regional stats operation. Needs to be the last dimension, and the same dimension as all of the reduced fields, weight fields, and masks. \\
    \hline
    \hyperref[subsec:nm_sec_config_AM_regionalStatsCustom_region_group]{config\_AM\_regionalStats\-Custom\_region\_group} & The name of the group of region masks that will be used to subset the mesh during the regional stats operation. \\
    \hline
    \hyperref[subsec:nm_sec_config_AM_regionalStatsCustom_1d_weighting_function]{config\_AM\_regionalStats\-Custom\_1d\_weighting\_function} & An operation applied to every element in a region WITHOUT a vertical dimension, with a 1D weighting field, prior to an average operation. The average is normalized by the sum of the weight field in the region (divided by the sum of regional weight values). \\
    \hline
    \hyperref[subsec:nm_sec_config_AM_regionalStatsCustom_2d_weighting_function]{config\_AM\_regionalStats\-Custom\_2d\_weighting\_function} & An operation applied to every element in a region WITH a vertical dimension, with a 2D weighting field, prior to an average operation. The average is normalized by the sum of the weight field in the region (divided by the sum of regional weight values). \\
    \hline
    \hyperref[subsec:nm_sec_config_AM_regionalStatsCustom_1d_weighting_field]{config\_AM\_regionalStats\-Custom\_1d\_weighting\_field} & A 1D real field used in conjunction with the 1D weighting function, to be used as a weighting scale factor (like area). \\
    \hline
    \hyperref[subsec:nm_sec_config_AM_regionalStatsCustom_2d_weighting_field]{config\_AM\_regionalStats\-Custom\_2d\_weighting\_field} & A 2D real field used in conjunction with the 2D weighting function, to be used as a weighting scale factor (like area). \\
    \hline
    \hyperref[subsec:nm_sec_config_AM_regionalStatsCustom_vertical_mask]{config\_AM\_regionalStats\-Custom\_vertical\_mask} & An additional 2D vertical integer mask field, which is used in conjunction with the regional masks. Used in cases when an input field has a second dimension that matches the vertical mask dimension. \\
    \hline
    \hyperref[subsec:nm_sec_config_AM_regionalStatsCustom_vertical_dimension]{config\_AM\_regionalStats\-Custom\_vertical\_dimension} & The second dimension to be used for additional vertical mask. \\
    \hline
\end{longtable}
\end{center}
}
\section[AM\_timeSeriesStatsDaily]{\hyperref[sec:nm_sec_AM_timeSeriesStatsDaily]{AM\_timeSeriesStatsDaily}}
\label{sec:nm_tab_AM_timeSeriesStatsDaily}
\vspace{0.5in}
{\small
\begin{center}
\begin{longtable}{| p{2.0in} || p{4.0in} |}
    \hline
    {\bf Name} & {\bf Description} \endfirsthead
    \hline 
    {\bf Name} & {\bf Description} (Continued) \endhead
    \hline
    \hline
    \hyperref[subsec:nm_sec_config_AM_timeSeriesStatsDaily_enable]{config\_AM\_timeSeriesStats\-Daily\_enable} & If true, ocean analysis member time series stats is called. \\
    \hline
    \hyperref[subsec:nm_sec_config_AM_timeSeriesStatsDaily_compute_on_startup]{config\_AM\_timeSeriesStats\-Daily\_compute\_on\_startup} & Logical flag determining if an analysis member computation occurs on start-up. You likely want this off for this (time series) analysis member because it will accumulate any state prior to time stepping (double counting the last time step). \\
    \hline
    \hyperref[subsec:nm_sec_config_AM_timeSeriesStatsDaily_write_on_startup]{config\_AM\_timeSeriesStats\-Daily\_write\_on\_startup} & Logical flag determining if an analysis member output occurs on start-up. \\
    \hline
    \hyperref[subsec:nm_sec_config_AM_timeSeriesStatsDaily_compute_interval]{config\_AM\_timeSeriesStats\-Daily\_compute\_interval} & Interval that determines frequency of computation for the time series stats analysis member. \\
    \hline
    \hyperref[subsec:nm_sec_config_AM_timeSeriesStatsDaily_output_stream]{config\_AM\_timeSeriesStats\-Daily\_output\_stream} & Name of stream the time series stats analysis member will operate on. \\
    \hline
    \hyperref[subsec:nm_sec_config_AM_timeSeriesStatsDaily_restart_stream]{config\_AM\_timeSeriesStats\-Daily\_restart\_stream} & Name of the restart stream the time series stats analysis member will use to initialize itself if restart is enabled. \\
    \hline
    \hyperref[subsec:nm_sec_config_AM_timeSeriesStatsDaily_operation]{config\_AM\_timeSeriesStats\-Daily\_operation} & An operation describing the statistic to apply to the time series for all variables in the output stream, reducing the time dimension. \\
    \hline
    \hyperref[subsec:nm_sec_config_AM_timeSeriesStatsDaily_reference_times]{config\_AM\_timeSeriesStats\-Daily\_reference\_times} & A list of absolute times describing when to start accumulating statistics. Each time indicates the start of one time window (time series statistic) per variable, in the output stream (i.e., provide four start times if you want quarterly climatologies, only one time is needed for monthly or daily averages, etc.) \\
    \hline
    \hyperref[subsec:nm_sec_config_AM_timeSeriesStatsDaily_duration_intervals]{config\_AM\_timeSeriesStats\-Daily\_duration\_intervals} & A list of time durations in d\_h:m:s describing how long to accumulate statistics in a time window for each repetition (repeat\_interval). It has to match the number of start time tokens in reference\_times. \\
    \hline
    \hyperref[subsec:nm_sec_config_AM_timeSeriesStatsDaily_repeat_intervals]{config\_AM\_timeSeriesStats\-Daily\_repeat\_intervals} & A list of time durations in d\_h:m:s describing the accumulation statistic temporal periodicity (time between beginning to accumulate again after it started - duration\_interval describes when to stop after starting/restarting). It has to match the number of tokens in reference\_times. \\
    \hline
    \hyperref[subsec:nm_sec_config_AM_timeSeriesStatsDaily_reset_intervals]{config\_AM\_timeSeriesStats\-Daily\_reset\_intervals} & A list of time durations in d\_h:m:s describing the statistic reset periodicity (how often to reset/clear/zero the accumulation). It has to match the number of tokens in reference\_times. \\
    \hline
    \hyperref[subsec:nm_sec_config_AM_timeSeriesStatsDaily_backward_output_offset]{config\_AM\_timeSeriesStats\-Daily\_backward\_output\_offset} & Backward offset for filename timestamps when writing the output stream \\
    \hline
\end{longtable}
\end{center}
}
\section[AM\_timeSeriesStatsMonthly]{\hyperref[sec:nm_sec_AM_timeSeriesStatsMonthly]{AM\_timeSeriesStatsMonthly}}
\label{sec:nm_tab_AM_timeSeriesStatsMonthly}
\vspace{0.5in}
{\small
\begin{center}
\begin{longtable}{| p{2.0in} || p{4.0in} |}
    \hline
    {\bf Name} & {\bf Description} \endfirsthead
    \hline 
    {\bf Name} & {\bf Description} (Continued) \endhead
    \hline
    \hline
    \hyperref[subsec:nm_sec_config_AM_timeSeriesStatsMonthly_enable]{config\_AM\_timeSeriesStats\-Monthly\_enable} & If true, ocean analysis member time series stats is called. \\
    \hline
    \hyperref[subsec:nm_sec_config_AM_timeSeriesStatsMonthly_compute_on_startup]{config\_AM\_timeSeriesStats\-Monthly\_compute\_on\_startup} & Logical flag determining if an analysis member computation occurs on start-up. You likely want this off for this (time series) analysis member because it will accumulate any state prior to time stepping (double counting the last time step). \\
    \hline
    \hyperref[subsec:nm_sec_config_AM_timeSeriesStatsMonthly_write_on_startup]{config\_AM\_timeSeriesStats\-Monthly\_write\_on\_startup} & Logical flag determining if an analysis member output occurs on start-up. \\
    \hline
    \hyperref[subsec:nm_sec_config_AM_timeSeriesStatsMonthly_compute_interval]{config\_AM\_timeSeriesStats\-Monthly\_compute\_interval} & Interval that determines frequency of computation for the time series stats analysis member. \\
    \hline
    \hyperref[subsec:nm_sec_config_AM_timeSeriesStatsMonthly_output_stream]{config\_AM\_timeSeriesStats\-Monthly\_output\_stream} & Name of stream the time series stats analysis member will operate on. \\
    \hline
    \hyperref[subsec:nm_sec_config_AM_timeSeriesStatsMonthly_restart_stream]{config\_AM\_timeSeriesStats\-Monthly\_restart\_stream} & Name of the restart stream the time series stats analysis member will use to initialize itself if restart is enabled. \\
    \hline
    \hyperref[subsec:nm_sec_config_AM_timeSeriesStatsMonthly_operation]{config\_AM\_timeSeriesStats\-Monthly\_operation} & An operation describing the statistic to apply to the time series for all variables in the output stream, reducing the time dimension. \\
    \hline
    \hyperref[subsec:nm_sec_config_AM_timeSeriesStatsMonthly_reference_times]{config\_AM\_timeSeriesStats\-Monthly\_reference\_times} & A list of absolute times describing when to start accumulating statistics. Each time indicates the start of one time window (time series statistic) per variable, in the output stream (i.e., provide four start times if you want quarterly climatologies, only one time is needed for monthly or daily averages, etc.) \\
    \hline
    \hyperref[subsec:nm_sec_config_AM_timeSeriesStatsMonthly_duration_intervals]{config\_AM\_timeSeriesStats\-Monthly\_duration\_intervals} & A list of time durations in d\_h:m:s describing how long to accumulate statistics in a time window for each repetition (repeat\_interval). It has to match the number of start time tokens in reference\_times. \\
    \hline
    \hyperref[subsec:nm_sec_config_AM_timeSeriesStatsMonthly_repeat_intervals]{config\_AM\_timeSeriesStats\-Monthly\_repeat\_intervals} & A list of time durations in d\_h:m:s describing the accumulation statistic temporal periodicity (time between beginning to accumulate again after it started - duration\_interval describes when to stop after starting/restarting). It has to match the number of tokens in reference\_times. \\
    \hline
    \hyperref[subsec:nm_sec_config_AM_timeSeriesStatsMonthly_reset_intervals]{config\_AM\_timeSeriesStats\-Monthly\_reset\_intervals} & A list of time durations in d\_h:m:s describing the statistic reset periodicity (how often to reset/clear/zero the accumulation). It has to match the number of tokens in reference\_times. \\
    \hline
    \hyperref[subsec:nm_sec_config_AM_timeSeriesStatsMonthly_backward_output_offset]{config\_AM\_timeSeriesStats\-Monthly\_backward\_output\_\-offset} & Backward offset for filename timestamps when writing the output stream \\
    \hline
\end{longtable}
\end{center}
}
\section[AM\_timeSeriesStatsClimatology]{\hyperref[sec:nm_sec_AM_timeSeriesStatsClimatology]{AM\_timeSeriesStatsClimatology}}
\label{sec:nm_tab_AM_timeSeriesStatsClimatology}
\vspace{0.5in}
{\small
\begin{center}
\begin{longtable}{| p{2.0in} || p{4.0in} |}
    \hline
    {\bf Name} & {\bf Description} \endfirsthead
    \hline 
    {\bf Name} & {\bf Description} (Continued) \endhead
    \hline
    \hline
    \hyperref[subsec:nm_sec_config_AM_timeSeriesStatsClimatology_enable]{config\_AM\_timeSeriesStats\-Climatology\_enable} & If true, ocean analysis member time series stats is called. \\
    \hline
    \hyperref[subsec:nm_sec_config_AM_timeSeriesStatsClimatology_compute_on_startup]{config\_AM\_timeSeriesStats\-Climatology\_compute\_on\_\-startup} & Logical flag determining if an analysis member computation occurs on start-up. You likely want this off for this (time series) analysis member because it will accumulate any state prior to time stepping (double counting the last time step). \\
    \hline
    \hyperref[subsec:nm_sec_config_AM_timeSeriesStatsClimatology_write_on_startup]{config\_AM\_timeSeriesStats\-Climatology\_write\_on\_startup} & Logical flag determining if an analysis member output occurs on start-up. \\
    \hline
    \hyperref[subsec:nm_sec_config_AM_timeSeriesStatsClimatology_compute_interval]{config\_AM\_timeSeriesStats\-Climatology\_compute\_interval} & Interval that determines frequency of computation for the time series stats analysis member. \\
    \hline
    \hyperref[subsec:nm_sec_config_AM_timeSeriesStatsClimatology_output_stream]{config\_AM\_timeSeriesStats\-Climatology\_output\_stream} & Name of stream the time series stats analysis member will operate on. \\
    \hline
    \hyperref[subsec:nm_sec_config_AM_timeSeriesStatsClimatology_restart_stream]{config\_AM\_timeSeriesStats\-Climatology\_restart\_stream} & Name of the restart stream the time series stats analysis member will use to initialize itself if restart is enabled. \\
    \hline
    \hyperref[subsec:nm_sec_config_AM_timeSeriesStatsClimatology_operation]{config\_AM\_timeSeriesStats\-Climatology\_operation} & An operation describing the statistic to apply to the time series for all variables in the output stream, reducing the time dimension. \\
    \hline
    \hyperref[subsec:nm_sec_config_AM_timeSeriesStatsClimatology_reference_times]{config\_AM\_timeSeriesStats\-Climatology\_reference\_times} & A list of absolute times describing when to start accumulating statistics. Each time indicates the start of one time window (time series statistic) per variable, in the output stream (i.e., provide four start times if you want quarterly climatologies, only one time is needed for monthly or daily averages, etc.) \\
    \hline
    \hyperref[subsec:nm_sec_config_AM_timeSeriesStatsClimatology_duration_intervals]{config\_AM\_timeSeriesStats\-Climatology\_duration\_intervals} & A list of time durations in d\_h:m:s describing how long to accumulate statistics in a time window for each repetition (repeat\_interval). It has to match the number of start time tokens in reference\_times. \\
    \hline
    \hyperref[subsec:nm_sec_config_AM_timeSeriesStatsClimatology_repeat_intervals]{config\_AM\_timeSeriesStats\-Climatology\_repeat\_intervals} & A list of time durations in d\_h:m:s describing the accumulation statistic temporal periodicity (time between beginning to accumulate again after it started - duration\_interval describes when to stop after starting/restarting). It has to match the number of tokens in reference\_times. \\
    \hline
    \hyperref[subsec:nm_sec_config_AM_timeSeriesStatsClimatology_reset_intervals]{config\_AM\_timeSeriesStats\-Climatology\_reset\_intervals} & A list of time durations in d\_h:m:s describing the statistic reset periodicity (how often to reset/clear/zero the accumulation). It has to match the number of tokens in reference\_times. \\
    \hline
    \hyperref[subsec:nm_sec_config_AM_timeSeriesStatsClimatology_backward_output_offset]{config\_AM\_timeSeriesStats\-Climatology\_backward\_output\_\-offset} & Backward offset for filename timestamps when writing the output stream \\
    \hline
\end{longtable}
\end{center}
}
\section[AM\_timeSeriesStatsCustom]{\hyperref[sec:nm_sec_AM_timeSeriesStatsCustom]{AM\_timeSeriesStatsCustom}}
\label{sec:nm_tab_AM_timeSeriesStatsCustom}
\vspace{0.5in}
{\small
\begin{center}
\begin{longtable}{| p{2.0in} || p{4.0in} |}
    \hline
    {\bf Name} & {\bf Description} \endfirsthead
    \hline 
    {\bf Name} & {\bf Description} (Continued) \endhead
    \hline
    \hline
    \hyperref[subsec:nm_sec_config_AM_timeSeriesStatsCustom_enable]{config\_AM\_timeSeriesStats\-Custom\_enable} & If true, ocean analysis member time series stats is called. \\
    \hline
    \hyperref[subsec:nm_sec_config_AM_timeSeriesStatsCustom_compute_on_startup]{config\_AM\_timeSeriesStats\-Custom\_compute\_on\_startup} & Logical flag determining if an analysis member computation occurs on start-up. You likely want this off for this (time series) analysis member because it will accumulate any state prior to time stepping (double counting the last time step). \\
    \hline
    \hyperref[subsec:nm_sec_config_AM_timeSeriesStatsCustom_write_on_startup]{config\_AM\_timeSeriesStats\-Custom\_write\_on\_startup} & Logical flag determining if an analysis member output occurs on start-up. \\
    \hline
    \hyperref[subsec:nm_sec_config_AM_timeSeriesStatsCustom_compute_interval]{config\_AM\_timeSeriesStats\-Custom\_compute\_interval} & Interval that determines frequency of computation for the time series stats analysis member. \\
    \hline
    \hyperref[subsec:nm_sec_config_AM_timeSeriesStatsCustom_output_stream]{config\_AM\_timeSeriesStats\-Custom\_output\_stream} & Name of stream the time series stats analysis member will operate on. \\
    \hline
    \hyperref[subsec:nm_sec_config_AM_timeSeriesStatsCustom_restart_stream]{config\_AM\_timeSeriesStats\-Custom\_restart\_stream} & Name of the restart stream the time series stats analysis member will use to initialize itself if restart is enabled. \\
    \hline
    \hyperref[subsec:nm_sec_config_AM_timeSeriesStatsCustom_operation]{config\_AM\_timeSeriesStats\-Custom\_operation} & An operation describing the statistic to apply to the time series for all variables in the output stream, reducing the time dimension. \\
    \hline
    \hyperref[subsec:nm_sec_config_AM_timeSeriesStatsCustom_reference_times]{config\_AM\_timeSeriesStats\-Custom\_reference\_times} & A list of absolute times describing when to start accumulating statistics. Each time indicates the start of one time window (time series statistic) per variable, in the output stream (i.e., provide four start times if you want quarterly climatologies, only one time is needed for monthly or daily averages, etc.) \\
    \hline
    \hyperref[subsec:nm_sec_config_AM_timeSeriesStatsCustom_duration_intervals]{config\_AM\_timeSeriesStats\-Custom\_duration\_intervals} & A list of time durations in d\_h:m:s describing how long to accumulate statistics in a time window for each repetition (repeat\_interval). It has to match the number of start time tokens in reference\_times. \\
    \hline
    \hyperref[subsec:nm_sec_config_AM_timeSeriesStatsCustom_repeat_intervals]{config\_AM\_timeSeriesStats\-Custom\_repeat\_intervals} & A list of time durations in d\_h:m:s describing the accumulation statistic temporal periodicity (time between beginning to accumulate again after it started - duration\_interval describes when to stop after starting/restarting). It has to match the number of tokens in reference\_times. \\
    \hline
    \hyperref[subsec:nm_sec_config_AM_timeSeriesStatsCustom_reset_intervals]{config\_AM\_timeSeriesStats\-Custom\_reset\_intervals} & A list of time durations in d\_h:m:s describing the statistic reset periodicity (how often to reset/clear/zero the accumulation). It has to match the number of tokens in reference\_times. \\
    \hline
    \hyperref[subsec:nm_sec_config_AM_timeSeriesStatsCustom_backward_output_offset]{config\_AM\_timeSeriesStats\-Custom\_backward\_output\_\-offset} & Backward offset for filename timestamps when writing the output stream \\
    \hline
\end{longtable}
\end{center}
}
\section[AM\_pointwiseStats]{\hyperref[sec:nm_sec_AM_pointwiseStats]{AM\_pointwiseStats}}
\label{sec:nm_tab_AM_pointwiseStats}
\vspace{0.5in}
{\small
\begin{center}
\begin{longtable}{| p{2.0in} || p{4.0in} |}
    \hline
    {\bf Name} & {\bf Description} \endfirsthead
    \hline 
    {\bf Name} & {\bf Description} (Continued) \endhead
    \hline
    \hline
    \hyperref[subsec:nm_sec_config_AM_pointwiseStats_enable]{config\_AM\_pointwiseStats\_\-enable} & If true, ocean analysis member pointwiseStats is called. \\
    \hline
    \hyperref[subsec:nm_sec_config_AM_pointwiseStats_compute_interval]{config\_AM\_pointwiseStats\_\-compute\_interval} & Timestamp determining how often analysis member computation should be performed. \\
    \hline
    \hyperref[subsec:nm_sec_config_AM_pointwiseStats_output_stream]{config\_AM\_pointwiseStats\_\-output\_stream} & Name of the stream that the pointwiseStats analysis member should be tied to. \\
    \hline
    \hyperref[subsec:nm_sec_config_AM_pointwiseStats_compute_on_startup]{config\_AM\_pointwiseStats\_\-compute\_on\_startup} & Logical flag determining if an analysis member computation occurs on start-up. \\
    \hline
    \hyperref[subsec:nm_sec_config_AM_pointwiseStats_write_on_startup]{config\_AM\_pointwiseStats\_\-write\_on\_startup} & Logical flag determining if an analysis member write occurs on start-up. \\
    \hline
\end{longtable}
\end{center}
}
\section[AM\_debugDiagnostics]{\hyperref[sec:nm_sec_AM_debugDiagnostics]{AM\_debugDiagnostics}}
\label{sec:nm_tab_AM_debugDiagnostics}
\vspace{0.5in}
{\small
\begin{center}
\begin{longtable}{| p{2.0in} || p{4.0in} |}
    \hline
    {\bf Name} & {\bf Description} \endfirsthead
    \hline 
    {\bf Name} & {\bf Description} (Continued) \endhead
    \hline
    \hline
    \hyperref[subsec:nm_sec_config_AM_debugDiagnostics_enable]{config\_AM\_debugDiagnostics\_\-enable} & If true, ocean analysis member debugDiagnostics is called. \\
    \hline
    \hyperref[subsec:nm_sec_config_AM_debugDiagnostics_compute_interval]{config\_AM\_debugDiagnostics\_\-compute\_interval} & Timestamp determining how often analysis member computation should be performed. \\
    \hline
    \hyperref[subsec:nm_sec_config_AM_debugDiagnostics_output_stream]{config\_AM\_debugDiagnostics\_\-output\_stream} & Name of the stream that the debugDiagnostics analysis member should be tied to. \\
    \hline
    \hyperref[subsec:nm_sec_config_AM_debugDiagnostics_compute_on_startup]{config\_AM\_debugDiagnostics\_\-compute\_on\_startup} & Logical flag determining if an analysis member computation occurs on start-up. \\
    \hline
    \hyperref[subsec:nm_sec_config_AM_debugDiagnostics_write_on_startup]{config\_AM\_debugDiagnostics\_\-write\_on\_startup} & Logical flag determining if an analysis member write occurs on start-up. \\
    \hline
    \hyperref[subsec:nm_sec_config_AM_debugDiagnostics_check_state]{config\_AM\_debugDiagnostics\_\-check\_state} & Logical flag determining if state checking happens when the debug diagnostics AM is called. \\
    \hline
\end{longtable}
\end{center}
}
\section[AM\_rpnCalculator]{\hyperref[sec:nm_sec_AM_rpnCalculator]{AM\_rpnCalculator}}
\label{sec:nm_tab_AM_rpnCalculator}
\vspace{0.5in}
{\small
\begin{center}
\begin{longtable}{| p{2.0in} || p{4.0in} |}
    \hline
    {\bf Name} & {\bf Description} \endfirsthead
    \hline 
    {\bf Name} & {\bf Description} (Continued) \endhead
    \hline
    \hline
    \hyperref[subsec:nm_sec_config_AM_rpnCalculator_enable]{config\_AM\_rpnCalculator\_\-enable} & If true, ocean analysis member RPN calculator is called. \\
    \hline
    \hyperref[subsec:nm_sec_config_AM_rpnCalculator_compute_on_startup]{config\_AM\_rpnCalculator\_\-compute\_on\_startup} & Logical flag determining if an analysis member computation occurs on start-up. \\
    \hline
    \hyperref[subsec:nm_sec_config_AM_rpnCalculator_write_on_startup]{config\_AM\_rpnCalculator\_\-write\_on\_startup} & Logical flag determining if an analysis member output occurs on start-up. \\
    \hline
    \hyperref[subsec:nm_sec_config_AM_rpnCalculator_compute_interval]{config\_AM\_rpnCalculator\_\-compute\_interval} & Interval that determines frequency of computation for the RPN calculator analysis member. \\
    \hline
    \hyperref[subsec:nm_sec_config_AM_rpnCalculator_output_stream]{config\_AM\_rpnCalculator\_\-output\_stream} & Name of stream the RPN calculator analysis member put output fields. \\
    \hline
    \hyperref[subsec:nm_sec_config_AM_rpnCalculator_variable_a]{config\_AM\_rpnCalculator\_\-variable\_a} & Name of a 0D or 1D real field that is bound to name 'a' in an RPN expression. \\
    \hline
    \hyperref[subsec:nm_sec_config_AM_rpnCalculator_variable_b]{config\_AM\_rpnCalculator\_\-variable\_b} & Name of a 0D or 1D real field that is bound to name 'b' in an RPN expression. \\
    \hline
    \hyperref[subsec:nm_sec_config_AM_rpnCalculator_variable_c]{config\_AM\_rpnCalculator\_\-variable\_c} & Name of a 0D or 1D real field that is bound to name 'c' in an RPN expression. \\
    \hline
    \hyperref[subsec:nm_sec_config_AM_rpnCalculator_variable_d]{config\_AM\_rpnCalculator\_\-variable\_d} & Name of a 0D or 1D real field that is bound to name 'd' in an RPN expression. \\
    \hline
    \hyperref[subsec:nm_sec_config_AM_rpnCalculator_variable_e]{config\_AM\_rpnCalculator\_\-variable\_e} & Name of a 0D or 1D real field that is bound to name 'e' in an RPN expression. \\
    \hline
    \hyperref[subsec:nm_sec_config_AM_rpnCalculator_variable_f]{config\_AM\_rpnCalculator\_\-variable\_f} & Name of a 0D or 1D real field that is bound to name 'f' in an RPN expression. \\
    \hline
    \hyperref[subsec:nm_sec_config_AM_rpnCalculator_variable_g]{config\_AM\_rpnCalculator\_\-variable\_g} & Name of a 0D or 1D real field that is bound to name 'g' in an RPN expression. \\
    \hline
    \hyperref[subsec:nm_sec_config_AM_rpnCalculator_variable_h]{config\_AM\_rpnCalculator\_\-variable\_h} & Name of a 0D or 1D real field that is bound to name 'h' in an RPN expression. \\
    \hline
    \hyperref[subsec:nm_sec_config_AM_rpnCalculator_expression_1]{config\_AM\_rpnCalculator\_\-expression\_1} & An RPN expression using fields bound to variable names. \\
    \hline
    \hyperref[subsec:nm_sec_config_AM_rpnCalculator_expression_2]{config\_AM\_rpnCalculator\_\-expression\_2} & An RPN expression using fields bound to variable names. \\
    \hline
    \hyperref[subsec:nm_sec_config_AM_rpnCalculator_expression_3]{config\_AM\_rpnCalculator\_\-expression\_3} & An RPN expression using fields bound to variable names. \\
    \hline
    \hyperref[subsec:nm_sec_config_AM_rpnCalculator_expression_4]{config\_AM\_rpnCalculator\_\-expression\_4} & An RPN expression using fields bound to variable names. \\
    \hline
    \hyperref[subsec:nm_sec_config_AM_rpnCalculator_output_name_1]{config\_AM\_rpnCalculator\_\-output\_name\_1} & The name of the output field resulting from RPN expression 1. \\
    \hline
    \hyperref[subsec:nm_sec_config_AM_rpnCalculator_output_name_2]{config\_AM\_rpnCalculator\_\-output\_name\_2} & The name of the output field resulting from RPN expression 2. \\
    \hline
    \hyperref[subsec:nm_sec_config_AM_rpnCalculator_output_name_3]{config\_AM\_rpnCalculator\_\-output\_name\_3} & The name of the output field resulting from RPN expression 3. \\
    \hline
    \hyperref[subsec:nm_sec_config_AM_rpnCalculator_output_name_4]{config\_AM\_rpnCalculator\_\-output\_name\_4} & The name of the output field resulting from RPN expression 4. \\
    \hline
\end{longtable}
\end{center}
}
\section[AM\_transectTransport]{\hyperref[sec:nm_sec_AM_transectTransport]{AM\_transectTransport}}
\label{sec:nm_tab_AM_transectTransport}
\vspace{0.5in}
{\small
\begin{center}
\begin{longtable}{| p{2.0in} || p{4.0in} |}
    \hline
    {\bf Name} & {\bf Description} \endfirsthead
    \hline 
    {\bf Name} & {\bf Description} (Continued) \endhead
    \hline
    \hline
    \hyperref[subsec:nm_sec_config_AM_transectTransport_enable]{config\_AM\_transectTransport\_\-enable} & If true, ocean analysis member transectTransport is called. \\
    \hline
    \hyperref[subsec:nm_sec_config_AM_transectTransport_compute_interval]{config\_AM\_transectTransport\_\-compute\_interval} & Timestamp determining how often analysis member computation should be performed. \\
    \hline
    \hyperref[subsec:nm_sec_config_AM_transectTransport_output_stream]{config\_AM\_transectTransport\_\-output\_stream} & Name of the stream that the transectTransport analysis member should be tied to. \\
    \hline
    \hyperref[subsec:nm_sec_config_AM_transectTransport_compute_on_startup]{config\_AM\_transectTransport\_\-compute\_on\_startup} & Logical flag determining if an analysis member computation occurs on start-up. \\
    \hline
    \hyperref[subsec:nm_sec_config_AM_transectTransport_write_on_startup]{config\_AM\_transectTransport\_\-write\_on\_startup} & Logical flag determining if an analysis member write occurs on start-up. \\
    \hline
    \hyperref[subsec:nm_sec_config_AM_transectTransport_transect_group]{config\_AM\_transectTransport\_\-transect\_group} & The name of the transect group that holds the transects for which the transport should be caclulated. \\
    \hline
\end{longtable}
\end{center}
}
\section[AM\_eddyProductVariables]{\hyperref[sec:nm_sec_AM_eddyProductVariables]{AM\_eddyProductVariables}}
\label{sec:nm_tab_AM_eddyProductVariables}
\vspace{0.5in}
{\small
\begin{center}
\begin{longtable}{| p{2.0in} || p{4.0in} |}
    \hline
    {\bf Name} & {\bf Description} \endfirsthead
    \hline 
    {\bf Name} & {\bf Description} (Continued) \endhead
    \hline
    \hline
    \hyperref[subsec:nm_sec_config_AM_eddyProductVariables_enable]{config\_AM\_eddyProduct\-Variables\_enable} & If true, ocean analysis member eddyProductVariables is called. \\
    \hline
    \hyperref[subsec:nm_sec_config_AM_eddyProductVariables_compute_interval]{config\_AM\_eddyProduct\-Variables\_compute\_interval} & Timestamp determining how often analysis member computation should be performed. \\
    \hline
    \hyperref[subsec:nm_sec_config_AM_eddyProductVariables_output_stream]{config\_AM\_eddyProduct\-Variables\_output\_stream} & Name of the stream that the eddyProductVariables analysis member should be tied to. \\
    \hline
    \hyperref[subsec:nm_sec_config_AM_eddyProductVariables_compute_on_startup]{config\_AM\_eddyProduct\-Variables\_compute\_on\_startup} & Logical flag determining if an analysis member computation occurs on start-up. \\
    \hline
    \hyperref[subsec:nm_sec_config_AM_eddyProductVariables_write_on_startup]{config\_AM\_eddyProduct\-Variables\_write\_on\_startup} & Logical flag determining if an analysis member write occurs on start-up. \\
    \hline
\end{longtable}
\end{center}
}
\section[AM\_mocStreamfunction]{\hyperref[sec:nm_sec_AM_mocStreamfunction]{AM\_mocStreamfunction}}
\label{sec:nm_tab_AM_mocStreamfunction}
\vspace{0.5in}
{\small
\begin{center}
\begin{longtable}{| p{2.0in} || p{4.0in} |}
    \hline
    {\bf Name} & {\bf Description} \endfirsthead
    \hline 
    {\bf Name} & {\bf Description} (Continued) \endhead
    \hline
    \hline
    \hyperref[subsec:nm_sec_config_AM_mocStreamfunction_enable]{config\_AM\_moc\-Streamfunction\_enable} & If true, ocean analysis member MOC streamfunction is called. \\
    \hline
    \hyperref[subsec:nm_sec_config_AM_mocStreamfunction_compute_interval]{config\_AM\_moc\-Streamfunction\_compute\_\-interval} & Timestamp determining how often analysis member computation should be performed. \\
    \hline
    \hyperref[subsec:nm_sec_config_AM_mocStreamfunction_output_stream]{config\_AM\_moc\-Streamfunction\_output\_stream} & Name of the stream that the mocStreamfunction analysis member should be tied to. \\
    \hline
    \hyperref[subsec:nm_sec_config_AM_mocStreamfunction_compute_on_startup]{config\_AM\_moc\-Streamfunction\_compute\_on\_\-startup} & Logical flag determining if an analysis member computation occurs on start-up. \\
    \hline
    \hyperref[subsec:nm_sec_config_AM_mocStreamfunction_write_on_startup]{config\_AM\_moc\-Streamfunction\_write\_on\_\-startup} & Logical flag determining if an analysis member write occurs on start-up. \\
    \hline
    \hyperref[subsec:nm_sec_config_AM_mocStreamfunction_min_bin]{config\_AM\_moc\-Streamfunction\_min\_bin} & minimum bin boundary value.  If set to -1.0e34, the minimum value in the domain is found. \\
    \hline
    \hyperref[subsec:nm_sec_config_AM_mocStreamfunction_max_bin]{config\_AM\_moc\-Streamfunction\_max\_bin} & maximum bin boundary value.  If set to -1.0e34, the maximum value in the domain is found. \\
    \hline
    \hyperref[subsec:nm_sec_config_AM_mocStreamfunction_num_bins]{config\_AM\_moc\-Streamfunction\_num\_bins} & Number of bins in South-to-North direction used for moc streamfunction calculation. \\
    \hline
    \hyperref[subsec:nm_sec_config_AM_mocStreamfunction_vertical_velocity_value]{config\_AM\_moc\-Streamfunction\_vertical\_\-velocity\_value} & The vertical velocity variable used for the computation of the MOC Streamfunction. \\
    \hline
    \hyperref[subsec:nm_sec_config_AM_mocStreamfunction_normal_velocity_value]{config\_AM\_moc\-Streamfunction\_normal\_\-velocity\_value} & The normal velocity variable used for the computation of the MOC Streamfunction. \\
    \hline
    \hyperref[subsec:nm_sec_config_AM_mocStreamfunction_region_group]{config\_AM\_moc\-Streamfunction\_region\_group} & The name of the region group, for which the moc should be computed in addition to the global MOC. \\
    \hline
    \hyperref[subsec:nm_sec_config_AM_mocStreamfunction_transect_group]{config\_AM\_moc\-Streamfunction\_transect\_group} & The name of the transect group that holds the boundaries for the regions in the region group, configured in 'config\_AM\_mocStreamfunction\_region\_group'. Please note, that these two should have the same amount of entries. \\
    \hline
\end{longtable}
\end{center}
}
