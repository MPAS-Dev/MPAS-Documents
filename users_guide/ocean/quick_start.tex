\chapter{MPAS-Ocean Quick Start Guide}
\label{chap:quick_start}

This chapter provides MPAS-Ocean users with a quick start description of how to
build and run the model. It is meant merely as a brief overview of the process,
while the more detailed descriptions of each step are provided in later
sections.

In general, the build process follows the following steps.

\begin{enumerate}
	\item Build MPI Layer (OpenMPI, MVAPICH2, etc.)
	\item Build serial NetCDF library (v3.6.3, v4.1.3, etc.)
	\item Build Parallel-NetCDF library (v1.2.1, v1.3.0, etc.)
	\item Build Parallel I/O library (v1.4.1, v1.6.1, etc.)
	\item (Optional) Build METIS library and executables (v4.0, v5.0.2, etc.)
	\item Checkout MPAS-Ocean from repository
	\item Build ocean core
\end{enumerate}

After step 7, an executable should be created called ocean\_model.exe. Once the executable is built, one can begin the run process as follows:

\begin{enumerate}
	\item Create run directory.
	\item Copy executable to run directory.
	\item Copy namelist.input, input and graph files into run directory.  See test cases, Chapter \ref{chap:\core_test_cases}.  The standard first test case is the 10km-resolution baroclinic channel, available at\\
 \url{http://mpas-dev.github.com/ocean/release_\version/release_\version.html} \\
as {\tt overflow\_10km\_40layer.tgz}.
	\item Edit namelist.input to have the proper parameters. \\
		  If step 4 was skipped, ensure paths to input and graph files are appropriately set.
	\item (Optional) Create graph files, using METIS executable (pmetis or gpmetis depending on version).  A graph file is required for each processor count you want to use.  See Section \ref{sec:metis}
	\item Run MPAS-Ocean (e.g.\ {\tt mpirun -np 8 ocean\_model}).
	\item Visualize output file, and perform analysis.  See Chapters \ref{chap:mpas_visualization} and \ref{chap:ocean_visualization}.
\end{enumerate}
