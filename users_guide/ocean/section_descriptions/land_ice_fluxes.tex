MPAS-Ocean supports the option to compute fluxes across
the land ice-ocean interface in either ``standalone'' or ``coupled''
mode.

In either mode, quadratic top drag is applied as a surface stress
in (\ref{ocean:\mode_mom_surf_forcing}), where
\begin{align}
\tau = & \rho_0 C_\textrm{D,top} \overline{\left(|u| u\right)}^\textrm{BL},
\end{align}
and where $C_\textrm{D,top}$ is the dimensionless top drag coefficient
and the operator $\overline{\left(\cdot\right)}^\textrm{BL}$ indicates
the vertical average over the boundary layer below land ice, which
has a depth $H_\textrm{BL}$.  Several additional diagnostics are computed
in either mode:
\begin{align}
u_* = & \sqrt{C_\textrm{D,top} \overline{\left(|u|^2+u_\textrm{tidal}^2\right)}^\textrm{BL}}, \\
\Theta_\textrm{BL} = & \overline{\left(\Theta\right)}^\textrm{BL}, \\
S_\textrm{BL} = & \overline{\left(S\right)}^\textrm{BL}.
\end{align}

The boundary conditions at the ice-ocean interface arise
from conservation of enthalpy and salt across the interface and the requirement that the interface
be at the pressure- and salinity-dependent potential freezing point:
\begin{align}
  \rho_\textrm{fw} L m_w & = \mathcal{F}_\textrm{H,ice} - \rho_\textrm{sw} c_p \gamma_\Theta \left(\Theta_b - \Theta_\textrm{BL} \right), \label{ocean:\mode_land_ice_enthalpy_balance}\\
  \rho_\textrm{fw} m_w S_b & = - \rho_\textrm{sw} \gamma_S \left(S_b - S_\textrm{BL}\right),  \\
  \Theta_b & = \Theta_f(S,p), \\
\end{align}
where $\rho_\textrm{fw}$ is the density of freshwater, $L$ is the latent heat of fusion of
water, $\mathcal{F}_\textrm{H,ice}$ is the sensible heat flux into the ice (always negative),
$\rho_\textrm{sw}$ is the reference denstiy of seawater $c_p$ is the heat capacity of seawater
$\gamma_\Theta$ and $\gamma_S$ are thermal- and salt-transfer velocities,
$\Theta_f$ is the freezing potential temperature and $\Theta_b$, $S_b$ and $m_w$ are the
unknown potential temperature and salinity at the boundary and the melt rate (expressed in
water-equivalent).  The associated surface freshwater, heat and salt fluxes are
\begin{align}
  \mathcal{F}_\textrm{fw} & = \rho_\textrm{fw} m_w, \\
  \mathcal{F}_\textrm{H} & = c_p \left[F_\textrm{fw} \Theta_b + \rho_\textrm{sw} \gamma_\Theta \left(\Theta_b - \Theta_\textrm{BL}\right)\right], \\
  \mathcal{F}_\textrm{S} & = 0.
\end{align}

In ``coupled'' mode, the time-averaged values of $\Theta_\textrm{BL}$, $S_\textrm{BL}$,
$\gamma_\Theta$ and $\gamma_S$ are computed in MPAS-Ocean, and the fluxes are computed
in the coupler.

In ``standalone'' mode, the boundary conditions are solved and the fluxes computed in
MPAS-Ocean.  Three flux formulations are supported, ``isomip'', ``jenkins'' and
``hollandJenkins'', each with its own definition of the freezing potential temperature
and the transfer velocities.

\hfill \break \noindent \textit{isomip:}

\noindent The ISOMIP formulation \citep{Hunter2006} is the simplest, using the
boundary-layer salinity to compute the freezing potential temperature and a
velocity independent heat-transfer velocity:
\begin{align}
  \Theta_f & = \lambda_1 S_\textrm{BL} + \lambda_2 + \lambda_3 p_b, \\
  \gamma_\Theta & = \gamma_\textrm{ISO}.
\end{align}
The salt-transfer velocity $\gamma_S$ and the salinity at the boundary $S_b$
are not needed for the flux computation and are not computed for this
formulation.

\hfill \break \noindent \textit{jenkins:}

\noindent The \citet{Jenkins2010} boundary conditions are more sophisticated,
using the interface salinity in the freezing potential temperature
and including the spatially variable friction velocity in the computation of
heat and salt transfer across the boundary layer:
\begin{align}
  \Theta_f & = \lambda_1 S_b + \lambda_2 + \lambda_3 p_b, \\
  \gamma_\Theta & = u_* \Gamma_\Theta, \\
  \gamma_S & = u_* \Gamma_S,
\end{align}
where $\Gamma_\Theta$ and $\Gamma_S$ are constants calibrated from observations
\citep[e.g.][]{Jenkins2010}.

\hfill \break \noindent \textit{holland-jenkins:}

\noindent The \citet{Holland1999} formulation includes slightly nonlinear velocity
dependence in heat and salt transfer velocities, and is commonly used in many studies.
The boundary conditions are as in the ``jenkins'' case except that $\Gamma_\Theta$ and
$\Gamma_S$ are not constants, but are given by
\begin{align}
  \Gamma_{\Theta,S} & = \frac{1}{\Lambda_\textrm{turb} + \Lambda_\textrm{mol}^{\Theta,S}}, \\
  \Lambda_\textrm{turb} & = \frac{1}{k} \textrm{ln}\left(\frac{u_* \xi_N}{f h_\nu}\right) + \frac{1}{2 \xi_N} - \frac{1}{k}, \\
  \Lambda_\textrm{mol}^{\Theta,S} & = 12.5~(\textrm{Pr},\textrm{Sc})^{2/3} - 6, \\
  h_\nu & = 5 \frac{\nu}{u_*},
\end{align}
where the various parameters $k$, $\xi_n$, $h_\nu$, $\textrm{Pr}$ and $\textrm{Sc}$ have the values defined
in \citet{Holland1999}.

The sensible heat flux $\mathcal{F}_\textrm{H,ice}$ into the ice in
(\ref{ocean:\mode_land_ice_enthalpy_balance}) is not known in ``standalone'' mode.  This flux
is generally small, with the dominant balance between latent heat released through melting
and ocean sensible heat fluxes.  The default assumption is that
$\mathcal{F}_\textrm{H,ice} = 0$, meaning that the ice is perfectly insulating.  An alternative
is to use the advection-diffusion method of \citet{Holland1999}, where the ice is assumed
to advect vertically at the melt rate and its temperature is assumed
to be at a reference surface temperature under melting conditions and at
the freezing point for freezing conditions.

