The Lagrangian In-situ Global High-performance particle Tracking (LIGHT) \citep{Wolfram_di15jpo}
analysis member computes particle trajectories on-line, providing a 
Lagrangian description of the flow that is comparable to the flow computed 
with the Eulerian dynamic core.  Interpolation schemes and time integration used to
advect particles are described in \citet{Wolfram_di15jpo}.  

\subsection{Different particle transport modes}

There are several different particle modes which are specified by assigning values to the \inlineCode{verticalTreatment} particle
variable.  These modes are used to determine the method used to assign the horizontal velocity for each particle's location,
relative to its horizontal location and associated cell:

\begin{enumerate}
  \item \inlineCode{indexLevel} -- specifies that particles are constrained to a specified \inlineCode{indexLevel} variable.
  \item \inlineCode{fixedZLevel} -- specifies that particles are constrained to a specified z-level (\inlineCode{fixedZLevel}).
  \item \inlineCode{passiveFloat} -- particles are advected by the full three-dimensional velocity field and are passive floats.
  \item \inlineCode{buoyancySurface} -- particles are constrained to buoyancy surfaces designated by the \inlineCode{buoyancyParticle} potential density surface.  For example, this approach was used in \citet{Wolfram_di15jpo}.
\end{enumerate}

The horizontal interpolation scheme is specified by the
\inlineCode{vertexReconstMethod} and \inlineCode{horizontalTreatment} variables. Currently
all horizontal interpolations are performed by interpolating cell-centered
Radial Basis Function (RBF) values onto cell vertices via linear interpolation
(specified by \inlineCode{vertexReconstMethod}) with Wachspress interpolation used to
interpolate vertex-located velocities onto arbitrary points within each cell
(specified by the \inlineCode{horizontalTreatment} variable).  

\subsection{Parallel decomposition}

Particle transport is dependent upon {\it a priori} knowledge of the block
decomposition used in the host Eulerian dynamic core and this information is
specified via the \inlineCode{currentBlock} variable which is used at run-time to
determine the destination block for particle computations.  Input/output blocks
are automatically assigned based on a uniform decomposition of particle across blocks.
The number of times each particle is transferred across computational blocks is 
stored via the \inlineCode{transfered} variable.

\subsection{Particle time stepping}

Particle time stepping is accomplished via sub-steps in between Eulerian dynamic core time steps
and is specified by \inlineCode{dtParticle}.  

\subsection{Storage of buoyancy-surface velocities and depth}

LIGHT can also interpolate the Eulerian dynamical core velocity field onto buoyancy surfaces.
The number of buoyancy surfaces is specified by the dimension \inlineCode{nBuoyancySurfaces} and 
the values of the buoyancy surfaces designated by \inlineCode{buoyancySurfaceValues}.  Velocities are 
returned via \inlineCode{buoyancySurfaceVelocityZonal} and \inlineCode{buoyancySurfaceVelocityMeridional} and 
buoyancy surface depth in \inlineCode{buoyancySurfaceDepth}.

\subsection{Pre-filtering of the Eulerian velocity field}

Unstructured second-order Shapiro filters are used to low-pass filter the Eulerian velocity fields 
\citep{Wolfram_di15jpo, wolfram2013mitigating}.  The number of filter passes, which determines the 
attenuation of high-frequencies, is specified via the 

{\noindent} \inlineCode{config{\_}lagrangian{\_}particle{\_}tracking{\_}filter{\_}number} namelist configuration option.  Filtered velocities are stored in the \inlineCode{filteredVelocityU} and \inlineCode{filteredVelocityV}
variables.

%Provide a table for 2dx, 4dx, 8dx filtering?

\subsection{Computation of Lagrangian mean, eddy velocities, and integral timescales}
LIGHT also provides the capability to store summed velocities and velocity
products, useful in determining mean velocities, eddy velocities, and
Lagrangian timescales: \inlineCode{sumU}, \inlineCode{sumV}, \inlineCode{sumUU}, \inlineCode{sumUV},
and \inlineCode{sumVV} corresponding to the time-integrated $u$, $v$, $uu$, $uv$, and
$vv$ velocity components.

