The ``del2'', or Laplacian, turbulence closures are
\begin{eqnarray}
\label{ocean:h_mom_del2}
& {\bf D}^u_h=\displaystyle\frac{\nu_h}{\rho_m^{3/4}} \nabla^2 {\bf u} 
= \displaystyle\frac{\nu_h}{\rho_m^{3/4}}(\nabla \delta + {\bf 
k}\times \nabla \eta),\\
\label{ocean:h_tr_del2}
& D^\varphi_h = \nabla\cdot\left(h 
   \displaystyle\frac{\kappa_h}{\rho_m^{3/4}} \nabla\varphi \right)
\end{eqnarray}
for momentum and tracers, respectively.  Variable definitions appear in Tables \ref{oceanTable:variables} and \ref{oceanTable:variables_Greek}.  The momentum diffusion is in divergence-vorticity form because it is a natural discretization of the vector Laplacian operator with a C-grid staggering.  

The Laplacian operator smooths the momentum and 
tracer fields, and smooths more strongly at small scales than at large 
scales.  This operator is the two dimensional form of the heat equation, 
$u_t=\nu u_{xx}$, described in introductory books on partial 
differential equations.  The strength of mixing is controlled by the 
viscosity, $\nu_h$, for the momentum equation, and the diffusion, 
$\kappa_h$, for the tracer equation.
