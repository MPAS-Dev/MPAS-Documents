There are several choices of horizontal mixing schemes available for the 
momentum and tracer equations.  Each of these is a turbulence closure, 
and attempts to account for subgrid-scale mixing and diffusion.  These 
schemes have the practical effect of reducing grid-scale noise in the 
velocity and tracer fields, and improving numerical stability.

Each horizontal mixing scheme has its own namelist, and may be turned
on with the \verb|_use_| logical configuration flags.  Multiple
schemes may be run simultaneously.  The horizontal mixing terms in the
governing equations (\ref{ocean:momentum},
\ref{ocean:tracer}) are ${\bf D}^u_h$ for momentum and
$D^\varphi_h$ for tracers.  No horizontal mixing is applied to the
thickness equation.

All horizontal mixing coefficients can be set to scale with the mesh as $\rho_m^{-3/4}$ in equations (\ref{ocean:h_mom_del2}, \ref{ocean:h_tr_del2}, \ref{ocean:h_mom_del4}, \ref{ocean:h_tr_del4}).  The mesh density, $\rho_m$, is a variable in the input and restart file.  It can vary between zero and one, and is one in the highest resolution region.  Scaling with the mesh can be turned off, as described in the options below.

The anticipated potential Vorticity (APV) method is a parameterization of the effects of subgrid or unresolved scales on those explicitly resolved \citep{Vallis_Hua88jas}.  It contributes an upstream bias to the vorticity in the del2 and del4 momentum terms as follows,
\begin{equation}
\eta_{apv} = \eta - c_{apv} dt \left({\bf u}\cdot \nabla \eta\right),
\end{equation}
where the altered vorticity $\eta_{apv}$ is used in equations (\ref{ocean:h_mom_del2}, \ref{ocean:h_mom_del4a}, \ref{ocean:h_mom_del4b}).
