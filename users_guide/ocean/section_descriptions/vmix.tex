There are several choices of vertical mixing schemes available for the 
momentum and tracer equations.  These are all applied with an implicit solve at the end of each time step using an operator splitting scheme.  

Implicit vertical mixing is required in ocean models because vertical mixing between unstably stratified layers occurs at timescales faster that other model processes.  The timestep requirement for explicit timestepping is usually set by the horizontal advective CFL condition.  In order to include realistic vertical mixing without very small time steps, we use operator splitting so that the vertical momentum and tracer diffusion terms are treated with an implicit timestep, while the remaining terms of the momentum and tracer equations use an explicit method.

Each vertical mixing scheme has its own namelist, and may be turned
on with the \verb|_use_| logical configuration flags.  Multiple
schemes may be run simultaneously.  The vertical mixing terms in the
governing equations (\ref{ocean:momentum}, \ref{ocean:tracer}) are 
\begin{eqnarray} 
\label{ocean:\mode_v_mom_diff}
& \displaystyle {\bf D}^u_v=\frac{\partial }{\partial z} 
\left( \nu_v \frac{\partial {\bf u}}{\partial z} \right), \\
\label{\mode_v_tr_diff}
& \displaystyle D^\varphi_v = \rho \frac{\partial }{\partial z} 
  \left( \kappa_v \frac{\partial \varphi}{\partial z} \right),
\end{eqnarray}
for momentum and tracers, respectively.  No vertical mixing is applied to the
thickness equation.
