There are several choices of vertical mixing schemes available for the momentum and tracer equations.  The CVMix namelist record is intended to control the Community Vertical Mixing package \url{https://github.com/CVMix}. CVMix is a collection of individual vertical mixing parameterizations intended to model background, convective, shear, tidal, double diffusion and ocean boundary layer processes. The output of each CVMix parameterization is a vertical profile of viscosity/diffusivity for a specific process. The CVMix parameterizations are constructed and implemented as ``stand-alone'' modules, so each parameterization can be toggled on/off without directly impacting the use of other parameterizations. The net viscosity/diffusivity used by the ocean model is the sum of all of the individual viscosity/diffusivity profiles. The one exception to this rule is when KPP is used when config\_cvmix\_kpp\_matching is set to ``MatchBoth''; in this instance the KPP-computed viscosity/diffusivity is the only contribution to mixing from the surface to the bottom of the ocean boundary layer.  

All supported schemes are part of the Community Vertical Mixing (CVMix) library.  A few legacy MPAS specific options are included, but are not officially supported and may be depracated in a future release.  The use of CVMix is strongly encouraged and most legacy mixing schemes can be reproduced within CVMix. 

Implicit vertical mixing is required in ocean models because vertical mixing between unstably stratified layers occurs at timescales faster that other model processes.  The timestep requirement for explicit timestepping is usually set by the horizontal advective CFL condition.  In order to include realistic vertical mixing without very small time steps, we use operator splitting so that the vertical momentum and tracer diffusion terms are treated with an implicit timestep, while the remaining terms of the momentum and tracer equations use an explicit method.

Each vertical mixing scheme has its own namelist, and may be turned
on with the \verb|_use_| logical configuration flags.  Multiple
schemes may be run simultaneously.  The vertical mixing terms in the
governing equations (\ref{ocean:momentum}, \ref{ocean:tracer}) are 
\begin{eqnarray} 
\label{ocean:\mode_v_mom_diff}
& \displaystyle {\bf D}^u_v=\frac{\partial }{\partial z} 
\left( \nu_v \frac{\partial {\bf u}}{\partial z} \right), \\
\label{\mode_v_tr_diff}
& \displaystyle D^\varphi_v = \rho \frac{\partial }{\partial z} 
\left( \kappa_v \frac{\partial \varphi}{\partial z} \right),
\end{eqnarray}
for momentum and tracers, respectively.  No vertical mixing is applied to the thickness equation.

Note that some CVMix functionality is not yet available.  A few namelist options exist as placeholders for these future additions, \textit{i.e.}, config\_use\_cvmix\_tidal\_mixing and config\_use\_cvmix\_double\_diffusion.  Finally, config\_cvmix\_kpp\_stop\_OBL\_search has no impact on the behavior of CVMix.

