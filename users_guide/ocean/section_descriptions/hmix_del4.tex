The ``del4'', or biharmonic, turbulence closures are
\begin{eqnarray}
\label{ocean:h_mom_del4}
& {\bf D}^u_h=-\displaystyle\frac{\nu_h}{\rho_m^{3/4}} \nabla^4 {\bf u} \\
\label{ocean:h_tr_del4}
& D^\varphi_h = -\nabla\cdot\left(h \displaystyle\frac{\kappa_h}{\rho_m^{3/4}} \nabla \left[\nabla\cdot\left(h \nabla\varphi \right) \right] \right)
\end{eqnarray}
for momentum and tracers  These are both computed by applying the Laplacian operator twice.  For momentum, this can be written in terms of divergence and vorticity as
\begin{eqnarray}
&\delta=\nabla\cdot{\bf u}\\
&\eta={\bf k} \cdot \left( \nabla \times {\bf u} \right)+f\\
\label{ocean:h_mom_del4a}
&\nabla^2{\bf u}=(\nabla \delta + {\bf k}\times \nabla \eta) \\
&\delta_2=\nabla\cdot(\nabla^2{\bf u})\\
&\eta_2={\bf k} \cdot \left( \nabla \times (\nabla^2{\bf u}) \right)+f\\
\label{ocean:h_mom_del4b}
& {\bf D}^u_h= \displaystyle\frac{\nu_h}{\rho_m^{3/4}} (\nabla \delta_2 + {\bf k}\times \nabla \eta_2).
\end{eqnarray}
The biharmonic operator is similar to the Laplacian operator, but smooths more strongly at high wavenumbers.  
