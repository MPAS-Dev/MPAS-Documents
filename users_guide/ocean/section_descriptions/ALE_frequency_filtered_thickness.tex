The high-frequency thickness alteration, $h^{hf}$, in (\ref{ocean:\mode_desired thickness}) allows the thicknesses to oscillate so that high-frequency motions, such as internal gravity waves, are treated in a Lagrangian manner.  Low-freqency motions, such as seasonal changes or slow motions of water masses, are treated in an Eulerian manner.  This is the ``z-tilde'' scheme of \citet{Leclair_Madec11om}, which generally reduces spurious vertical mixing and preserves water mass properties.  Two additional prognostic equations are solved when config\_use\_freq\_filtered\_thickness is true,
\begin{eqnarray}
\label{ocean:\mode_Dlf}
 & \displaystyle
  \frac{\partial D^{lf}_k}{\partial t} = - \frac{2\pi}{\tau_{Dlf}} \left( D^{lf}_k - D'_k \right), 
\\ & \displaystyle
\label{ocean:\mode_hhf}
\frac{\partial h^{hf}_k}{\partial t} =  - D^{hf}_k - \frac{2\pi}{\tau_{hhf}} h^{hf}_k + \nabla_h\cdot \left( \kappa_{hhf} \nabla_h h^{hf}_k \right) 
\end{eqnarray}
where $\tau_{Dlf}$ is the filter timescale and other variables are defined in Table \ref{oceanTable:\mode_ALE_variables}.  This may be used in addition to any of the z-type or sigma-type vertical coordinates in Table \ref{oceanTable:\mode_vertical coordinates}.  Some combination of thickness restoring and diffusion are recommended to avoid long-term drift of $h^{hf}$ away from zero.



