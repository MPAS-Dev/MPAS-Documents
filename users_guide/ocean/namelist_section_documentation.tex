\chapter[Namelist options]{\hyperref[chap:namelist_tables]{Namelist options}}
\label{chap:namelist_sections}
Embedded links point to information in chapter \ref{chap:namelist_tables}
\section[run\_modes]{\hyperref[sec:nm_tab_run_modes]{run\_modes}}
\label{sec:nm_sec_run_modes}
\subsection[config\_ocean\_run\_mode]{\hyperref[sec:nm_tab_run_modes]{config\_ocean\_run\_mode}}
\label{subsec:nm_sec_config_ocean_run_mode}
\begin{center}
\begin{longtable}{| p{2.0in} || p{4.0in} |}
    \hline
    Type: & character \\
    \hline
    Units: & -- \\
    \hline
    Default Value: & forward \\
    \hline
    Possible Values: & 'forward' and 'analysis' \\
    \hline
    \caption{config\_ocean\_run\_mode: Determines which run mode will be used for the ocean model.}
\end{longtable}
\end{center}
\section[time\_management]{\hyperref[sec:nm_tab_time_management]{time\_management}}
\label{sec:nm_sec_time_management}
\subsection[config\_do\_restart]{\hyperref[sec:nm_tab_time_management]{config\_do\_restart}}
\label{subsec:nm_sec_config_do_restart}
\begin{center}
\begin{longtable}{| p{2.0in} || p{4.0in} |}
    \hline
    Type: & logical \\
    \hline
    Units: & -- \\
    \hline
    Default Value: & .false. \\
    \hline
    Possible Values: & .true. or .false. \\
    \hline
    \caption{config\_do\_restart: Determines if the initial conditions should be read from a restart file, or an input file.}
\end{longtable}
\end{center}
\subsection[config\_restart\_timestamp\_name]{\hyperref[sec:nm_tab_time_management]{config\_restart\_timestamp\_name}}
\label{subsec:nm_sec_config_restart_timestamp_name}
\begin{center}
\begin{longtable}{| p{2.0in} || p{4.0in} |}
    \hline
    Type: & character \\
    \hline
    Units: & -- \\
    \hline
    Default Value: & Restart\_timestamp \\
    \hline
    Possible Values: & Path to a file. \\
    \hline
    \caption{config\_restart\_timestamp\_name: Path to the filename for restart timestamps to be read and written from.}
\end{longtable}
\end{center}
\subsection[config\_start\_time]{\hyperref[sec:nm_tab_time_management]{config\_start\_time}}
\label{subsec:nm_sec_config_start_time}
\begin{center}
\begin{longtable}{| p{2.0in} || p{4.0in} |}
    \hline
    Type: & character \\
    \hline
    Units: & -- \\
    \hline
    Default Value: & 0001-01-01\_00:00:00 \\
    \hline
    Possible Values: & 'YYYY-MM-DD\_HH:MM:SS' or 'file' \\
    \hline
    \caption{config\_start\_time: Timestamp describing the initial time of the simulation. If it is set to 'file', the initial time is read from restart\_timestamp.}
\end{longtable}
\end{center}
\subsection[config\_stop\_time]{\hyperref[sec:nm_tab_time_management]{config\_stop\_time}}
\label{subsec:nm_sec_config_stop_time}
\begin{center}
\begin{longtable}{| p{2.0in} || p{4.0in} |}
    \hline
    Type: & character \\
    \hline
    Units: & -- \\
    \hline
    Default Value: & none \\
    \hline
    Possible Values: & 'YYYY-MM-DD\_HH:MM:SS' or 'none' \\
    \hline
    \caption{config\_stop\_time: Timestamp descriping the final time of the simulation. If it is set to 'none' the final time is determined from config\_start\_time and config\_run\_duration.}
\end{longtable}
\end{center}
\subsection[config\_run\_duration]{\hyperref[sec:nm_tab_time_management]{config\_run\_duration}}
\label{subsec:nm_sec_config_run_duration}
\begin{center}
\begin{longtable}{| p{2.0in} || p{4.0in} |}
    \hline
    Type: & character \\
    \hline
    Units: & -- \\
    \hline
    Default Value: & 0010\_00:00:00 \\
    \hline
    Possible Values: & 'DDDD\_HH:MM:SS' or 'none' \\
    \hline
    \caption{config\_run\_duration: Timestamp describing the length of the simulation. If it is set to 'none' the duration is determined from config\_start\_time and config\_stop\_time. config\_run\_duration overrides inconsistent values of config\_stop\_time.}
\end{longtable}
\end{center}
\subsection[config\_calendar\_type]{\hyperref[sec:nm_tab_time_management]{config\_calendar\_type}}
\label{subsec:nm_sec_config_calendar_type}
\begin{center}
\begin{longtable}{| p{2.0in} || p{4.0in} |}
    \hline
    Type: & character \\
    \hline
    Units: & -- \\
    \hline
    Default Value: & noleap \\
    \hline
    Possible Values: & 'gregorian', 'noleap' \\
    \hline
    \caption{config\_calendar\_type: Selection of the type of calendar that should be used in the simulation.}
\end{longtable}
\end{center}
\subsection[config\_output\_reference\_time]{\hyperref[sec:nm_tab_time_management]{config\_output\_reference\_time}}
\label{subsec:nm_sec_config_output_reference_time}
\begin{center}
\begin{longtable}{| p{2.0in} || p{4.0in} |}
    \hline
    Type: & character \\
    \hline
    Units: & -- \\
    \hline
    Default Value: & 0001-01-01\_00:00:00 \\
    \hline
    Possible Values: & 'YYYY-MM-DD\_HH:MM:SS' \\
    \hline
    \caption{config\_output\_reference\_time: Reference time used in the units attribute of Time in output files.}
\end{longtable}
\end{center}
\section[io]{\hyperref[sec:nm_tab_io]{io}}
\label{sec:nm_sec_io}
\subsection[config\_write\_output\_on\_startup]{\hyperref[sec:nm_tab_io]{config\_write\_output\_on\_startup}}
\label{subsec:nm_sec_config_write_output_on_startup}
\begin{center}
\begin{longtable}{| p{2.0in} || p{4.0in} |}
    \hline
    Type: & logical \\
    \hline
    Units: & -- \\
    \hline
    Default Value: & .true. \\
    \hline
    Possible Values: & .true. or .false. \\
    \hline
    \caption{config\_write\_output\_on\_startup: Logical flag determining if an output file should be written prior to the first time step.}
\end{longtable}
\end{center}
\subsection[config\_pio\_num\_iotasks]{\hyperref[sec:nm_tab_io]{config\_pio\_num\_iotasks}}
\label{subsec:nm_sec_config_pio_num_iotasks}
\begin{center}
\begin{longtable}{| p{2.0in} || p{4.0in} |}
    \hline
    Type: & integer \\
    \hline
    Units: & -- \\
    \hline
    Default Value: & 0 \\
    \hline
    Possible Values: & Any positive integer value greater than or equal to 0. \\
    \hline
    \caption{config\_pio\_num\_iotasks: Integer specifying how many IO tasks should be used within the PIO library. A value of 0 causes all MPI tasks to also be IO tasks. IO tasks are requried to write contiguous blocks of data to a file.}
\end{longtable}
\end{center}
\subsection[config\_pio\_stride]{\hyperref[sec:nm_tab_io]{config\_pio\_stride}}
\label{subsec:nm_sec_config_pio_stride}
\begin{center}
\begin{longtable}{| p{2.0in} || p{4.0in} |}
    \hline
    Type: & integer \\
    \hline
    Units: & -- \\
    \hline
    Default Value: & 1 \\
    \hline
    Possible Values: & Any positive integer value greater than 0. \\
    \hline
    \caption{config\_pio\_stride: Integer specifying the stride of each IO task.}
\end{longtable}
\end{center}
\section[decomposition]{\hyperref[sec:nm_tab_decomposition]{decomposition}}
\label{sec:nm_sec_decomposition}
\subsection[config\_num\_halos]{\hyperref[sec:nm_tab_decomposition]{config\_num\_halos}}
\label{subsec:nm_sec_config_num_halos}
\begin{center}
\begin{longtable}{| p{2.0in} || p{4.0in} |}
    \hline
    Type: & integer \\
    \hline
    Units: & -- \\
    \hline
    Default Value: & 3 \\
    \hline
    Possible Values: & Any positive integer value. \\
    \hline
    \caption{config\_num\_halos: Determines the number of halo cells extending from a blocks owned cells (Called the 0-Halo). The default of 3 is the minimum that can be used with monotonic advection.}
\end{longtable}
\end{center}
\subsection[config\_block\_decomp\_file\_prefix]{\hyperref[sec:nm_tab_decomposition]{config\_block\_decomp\_file\_prefix}}
\label{subsec:nm_sec_config_block_decomp_file_prefix}
\begin{center}
\begin{longtable}{| p{2.0in} || p{4.0in} |}
    \hline
    Type: & character \\
    \hline
    Units: & -- \\
    \hline
    Default Value: & graph.info.part. \\
    \hline
    Possible Values: & Any path/prefix to a block decomposition file. \\
    \hline
    \caption{config\_block\_decomp\_file\_prefix: Defines the prefix for the block decomposition file. Can include a path. The number of blocks is appended to the end of the prefix at run-time.}
\end{longtable}
\end{center}
\subsection[config\_number\_of\_blocks]{\hyperref[sec:nm_tab_decomposition]{config\_number\_of\_blocks}}
\label{subsec:nm_sec_config_number_of_blocks}
\begin{center}
\begin{longtable}{| p{2.0in} || p{4.0in} |}
    \hline
    Type: & integer \\
    \hline
    Units: & -- \\
    \hline
    Default Value: & 0 \\
    \hline
    Possible Values: & Any integer greater than or equal to 0. \\
    \hline
    \caption{config\_number\_of\_blocks: Determines the number of blocks a simulation should be run with. If it is set to 0, the number of blocks is the same as the number of MPI tasks at run-time.}
\end{longtable}
\end{center}
\subsection[config\_explicit\_proc\_decomp]{\hyperref[sec:nm_tab_decomposition]{config\_explicit\_proc\_decomp}}
\label{subsec:nm_sec_config_explicit_proc_decomp}
\begin{center}
\begin{longtable}{| p{2.0in} || p{4.0in} |}
    \hline
    Type: & logical \\
    \hline
    Units: & -- \\
    \hline
    Default Value: & .false. \\
    \hline
    Possible Values: & .true. or .false. \\
    \hline
    \caption{config\_explicit\_proc\_decomp: Determines if an explicit processor decomposition should be used. This is only useful if multiple blocks per processor are used.}
\end{longtable}
\end{center}
\subsection[config\_proc\_decomp\_file\_prefix]{\hyperref[sec:nm_tab_decomposition]{config\_proc\_decomp\_file\_prefix}}
\label{subsec:nm_sec_config_proc_decomp_file_prefix}
\begin{center}
\begin{longtable}{| p{2.0in} || p{4.0in} |}
    \hline
    Type: & character \\
    \hline
    Units: & -- \\
    \hline
    Default Value: & graph.info.part. \\
    \hline
    Possible Values: & Any path/prefix to a processor decomposition file. \\
    \hline
    \caption{config\_proc\_decomp\_file\_prefix: Defines the prefix for the processor decomposition file. This file is only read if config\_explicit\_proc\_decomp is .true. The number of processors is appended to the end of the prefix at run-time.}
\end{longtable}
\end{center}
\section[time\_integration]{\hyperref[sec:nm_tab_time_integration]{time\_integration}}
\label{sec:nm_sec_time_integration}
\subsection[config\_dt]{\hyperref[sec:nm_tab_time_integration]{config\_dt}}
\label{subsec:nm_sec_config_dt}
\begin{center}
\begin{longtable}{| p{2.0in} || p{4.0in} |}
    \hline
    Type: & character \\
    \hline
    Units: & -- \\
    \hline
    Default Value: & 00:05:00 \\
    \hline
    Possible Values: & Any time stamp in 'YYYY-MM-DD\_hh:mm:ss' format. Items can be removed from the left if they are unused. \\
    \hline
    \caption{config\_dt: Length of model time-step.}
\end{longtable}
\end{center}
\subsection[config\_time\_integrator]{\hyperref[sec:nm_tab_time_integration]{config\_time\_integrator}}
\label{subsec:nm_sec_config_time_integrator}
\begin{center}
\begin{longtable}{| p{2.0in} || p{4.0in} |}
    \hline
    Type: & character \\
    \hline
    Units: & -- \\
    \hline
    Default Value: & split\_explicit\_ab2 \\
    \hline
    Possible Values: & 'split\_explicit', 'RK4', 'unsplit\_explicit', 'split\_implicit', 'LTS', 'split\_explicit\_ab2' \\
    \hline
    \caption{config\_time\_integrator: Time integration method.}
\end{longtable}
\end{center}
\subsection[config\_number\_of\_time\_levels]{\hyperref[sec:nm_tab_time_integration]{config\_number\_of\_time\_levels}}
\label{subsec:nm_sec_config_number_of_time_levels}
\begin{center}
\begin{longtable}{| p{2.0in} || p{4.0in} |}
    \hline
    Type: & integer \\
    \hline
    Units: & -- \\
    \hline
    Default Value: & 2 \\
    \hline
    Possible Values: & Any integer greater than or equal to 2. \\
    \hline
    \caption{config\_number\_of\_time\_levels: The number of time levels in the time-stepping scheme. This is used for array allocation.}
\end{longtable}
\end{center}
\section[hmix]{\hyperref[sec:nm_tab_hmix]{hmix}}
\label{sec:nm_sec_hmix}
\subsection[config\_hmix\_scaleWithMesh]{\hyperref[sec:nm_tab_hmix]{config\_hmix\_scaleWithMesh}}
\label{subsec:nm_sec_config_hmix_scaleWithMesh}
\begin{center}
\begin{longtable}{| p{2.0in} || p{4.0in} |}
    \hline
    Type: & logical \\
    \hline
    Units: & -- \\
    \hline
    Default Value: & .false. \\
    \hline
    Possible Values: & .true. or .false. \\
    \hline
    \caption{config\_hmix\_scaleWithMesh: If false, del2 and del4 coefficients are constant throughout the mesh (equivalent to setting $\rho_m=1$ throughout the mesh).  If true, these coefficients scale as mesh density to the -3/4 power.}
\end{longtable}
\end{center}
\subsection[config\_maxMeshDensity]{\hyperref[sec:nm_tab_hmix]{config\_maxMeshDensity}}
\label{subsec:nm_sec_config_maxMeshDensity}
\begin{center}
\begin{longtable}{| p{2.0in} || p{4.0in} |}
    \hline
    Type: & real \\
    \hline
    Units: & -- \\
    \hline
    Default Value: & -1.0 \\
    \hline
    Possible Values: & Any positive real number. If set any negative real number, config\_maxMeshDensity is computed during the initialization of each simulation. \\
    \hline
    \caption{config\_maxMeshDensity: Global maximum of the mesh density}
\end{longtable}
\end{center}
\subsection[config\_hmix\_use\_ref\_cell\_width]{\hyperref[sec:nm_tab_hmix]{config\_hmix\_use\_ref\_cell\_width}}
\label{subsec:nm_sec_config_hmix_use_ref_cell_width}
\begin{center}
\begin{longtable}{| p{2.0in} || p{4.0in} |}
    \hline
    Type: & logical \\
    \hline
    Units: & -- \\
    \hline
    Default Value: & .false. \\
    \hline
    Possible Values: & .true. or .false. \\
    \hline
    \caption{config\_hmix\_use\_ref\_cell\_width: If true, hmix coefficient values are set with reference to config\_hmix\_ref\_cell\_width. If false, hmix coefficient values are referenced to smallest gridcell in the mesh. The false setting is for backwards compatilibity. When false, hmix coefficient flags must be adjusted for every new mesh with a different minimum-sized cell.}
\end{longtable}
\end{center}
\subsection[config\_hmix\_ref\_cell\_width]{\hyperref[sec:nm_tab_hmix]{config\_hmix\_ref\_cell\_width}}
\label{subsec:nm_sec_config_hmix_ref_cell_width}
\begin{center}
\begin{longtable}{| p{2.0in} || p{4.0in} |}
    \hline
    Type: & real \\
    \hline
    Units: & \si{m} \\
    \hline
    Default Value: & 30.0e3 \\
    \hline
    Possible Values: & Any positive real number, but typically a resolution number such as 30km. \\
    \hline
    \caption{config\_hmix\_ref\_cell\_width: Reference cell width. If config\_hmix\_use\_ref\_cell\_width = .true., then hmix coefficients are set to be $nu_{2h}$ = config\_mom\_del2*(cellWidth / config\_hmix\_use\_ref\_cell\_width) and $nu_{4h}$ = config\_mom\_del4*(cellWidth / config\_hmix\_use\_ref\_cell\_width)$^3$ where cellWidth is the effective cell width computed as 2*sqrt(areaCell/pi). See Hoch et al 2020 JAMES eq 1,2. This relation applies within a simulation, but also generally among multiple simulations, so the parameters config\_mom\_del2, config\_mom\_del4, and config\_hmix\_use\_ref\_cell\_width can generally remain at their standard values, and just be adjusted for fine tuning.}
\end{longtable}
\end{center}
\subsection[config\_apvm\_scale\_factor]{\hyperref[sec:nm_tab_hmix]{config\_apvm\_scale\_factor}}
\label{subsec:nm_sec_config_apvm_scale_factor}
\begin{center}
\begin{longtable}{| p{2.0in} || p{4.0in} |}
    \hline
    Type: & real \\
    \hline
    Units: & -- \\
    \hline
    Default Value: & 0.0 \\
    \hline
    Possible Values: & Any non-negative number, typically between zero and one. \\
    \hline
    \caption{config\_apvm\_scale\_factor: Anticipated potential vorticity (APV) method scale factor, $c_{apv}$. When zero, APV is off.}
\end{longtable}
\end{center}
\section[hmix\_del2]{\hyperref[sec:nm_tab_hmix_del2]{hmix\_del2}}
\label{sec:nm_sec_hmix_del2}
\subsection[config\_use\_mom\_del2]{\hyperref[sec:nm_tab_hmix_del2]{config\_use\_mom\_del2}}
\label{subsec:nm_sec_config_use_mom_del2}
\begin{center}
\begin{longtable}{| p{2.0in} || p{4.0in} |}
    \hline
    Type: & logical \\
    \hline
    Units: & -- \\
    \hline
    Default Value: & .false. \\
    \hline
    Possible Values: & .true. or .false. \\
    \hline
    \caption{config\_use\_mom\_del2: If true, Laplacian horizontal mixing is used on the momentum equation.}
\end{longtable}
\end{center}
\subsection[config\_mom\_del2]{\hyperref[sec:nm_tab_hmix_del2]{config\_mom\_del2}}
\label{subsec:nm_sec_config_mom_del2}
\begin{center}
\begin{longtable}{| p{2.0in} || p{4.0in} |}
    \hline
    Type: & real \\
    \hline
    Units: & \si{m^2.s^-1} \\
    \hline
    Default Value: & 1.0e3 \\
    \hline
    Possible Values: & any positive real \\
    \hline
    \caption{config\_mom\_del2: Horizontal viscosity, $\nu_{2h}$. If config\_hmix\_use\_ref\_cell\_width = .true. then $\nu_h$ = config\_mom\_del2*(cellWidth / config\_hmix\_use\_ref\_cell\_width). If config\_hmix\_use\_ref\_cell\_width = .false. then it is referenced to the smallest cell.}
\end{longtable}
\end{center}
\subsection[config\_use\_tracer\_del2]{\hyperref[sec:nm_tab_hmix_del2]{config\_use\_tracer\_del2}}
\label{subsec:nm_sec_config_use_tracer_del2}
\begin{center}
\begin{longtable}{| p{2.0in} || p{4.0in} |}
    \hline
    Type: & logical \\
    \hline
    Units: & -- \\
    \hline
    Default Value: & .false. \\
    \hline
    Possible Values: & .true. or .false. \\
    \hline
    \caption{config\_use\_tracer\_del2: If true, Laplacian horizontal mixing is used on the tracer equation.}
\end{longtable}
\end{center}
\subsection[config\_tracer\_del2]{\hyperref[sec:nm_tab_hmix_del2]{config\_tracer\_del2}}
\label{subsec:nm_sec_config_tracer_del2}
\begin{center}
\begin{longtable}{| p{2.0in} || p{4.0in} |}
    \hline
    Type: & real \\
    \hline
    Units: & \si{m^2.s^-1} \\
    \hline
    Default Value: & 10.0 \\
    \hline
    Possible Values: & any positive real \\
    \hline
    \caption{config\_tracer\_del2: Horizontal diffusion, $\kappa_{2h}$. If config\_hmix\_use\_ref\_cell\_width = .true. then $\kappa_h$ = config\_tracer\_del2*(cellWidth / config\_hmix\_use\_ref\_cell\_width). If config\_hmix\_use\_ref\_cell\_width = .false. then it is referenced to the smallest cell.}
\end{longtable}
\end{center}
\section[hmix\_del4]{\hyperref[sec:nm_tab_hmix_del4]{hmix\_del4}}
\label{sec:nm_sec_hmix_del4}
\subsection[config\_use\_mom\_del4]{\hyperref[sec:nm_tab_hmix_del4]{config\_use\_mom\_del4}}
\label{subsec:nm_sec_config_use_mom_del4}
\begin{center}
\begin{longtable}{| p{2.0in} || p{4.0in} |}
    \hline
    Type: & logical \\
    \hline
    Units: & -- \\
    \hline
    Default Value: & .false. \\
    \hline
    Possible Values: & .true. or .false. \\
    \hline
    \caption{config\_use\_mom\_del4: If true, biharmonic horizontal mixing is used on the momentum equation.}
\end{longtable}
\end{center}
\subsection[config\_mom\_del4]{\hyperref[sec:nm_tab_hmix_del4]{config\_mom\_del4}}
\label{subsec:nm_sec_config_mom_del4}
\begin{center}
\begin{longtable}{| p{2.0in} || p{4.0in} |}
    \hline
    Type: & real \\
    \hline
    Units: & \si{m^4.s^-1} \\
    \hline
    Default Value: & 1.2e11 \\
    \hline
    Possible Values: & any positive real \\
    \hline
    \caption{config\_mom\_del4: Coefficient for horizontal biharmonic operator on momentum.  If config\_hmix\_use\_ref\_cell\_width = .true. then $\nu_{4h}$ = config\_mom\_del4*(cellWidth / config\_hmix\_use\_ref\_cell\_width)$^3$. If config\_hmix\_use\_ref\_cell\_width = .false. then it is referenced to the smallest cell.}
\end{longtable}
\end{center}
\subsection[config\_mom\_del4\_div\_factor]{\hyperref[sec:nm_tab_hmix_del4]{config\_mom\_del4\_div\_factor}}
\label{subsec:nm_sec_config_mom_del4_div_factor}
\begin{center}
\begin{longtable}{| p{2.0in} || p{4.0in} |}
    \hline
    Type: & real \\
    \hline
    Units: & \si{non-dimensional} \\
    \hline
    Default Value: & 1.0 \\
    \hline
    Possible Values: & any positive real \\
    \hline
    \caption{config\_mom\_del4\_div\_factor: The divergence portion of the del4 operator is scaled by this factor.}
\end{longtable}
\end{center}
\subsection[config\_use\_tracer\_del4]{\hyperref[sec:nm_tab_hmix_del4]{config\_use\_tracer\_del4}}
\label{subsec:nm_sec_config_use_tracer_del4}
\begin{center}
\begin{longtable}{| p{2.0in} || p{4.0in} |}
    \hline
    Type: & logical \\
    \hline
    Units: & -- \\
    \hline
    Default Value: & .false. \\
    \hline
    Possible Values: & .true. or .false. \\
    \hline
    \caption{config\_use\_tracer\_del4: If true, biharmonic horizontal mixing is used on the tracer equation.}
\end{longtable}
\end{center}
\subsection[config\_tracer\_del4]{\hyperref[sec:nm_tab_hmix_del4]{config\_tracer\_del4}}
\label{subsec:nm_sec_config_tracer_del4}
\begin{center}
\begin{longtable}{| p{2.0in} || p{4.0in} |}
    \hline
    Type: & real \\
    \hline
    Units: & \si{m^4.s^-1} \\
    \hline
    Default Value: & 0.0 \\
    \hline
    Possible Values: & any positive real \\
    \hline
    \caption{config\_tracer\_del4: Coefficient for horizontal biharmonic operator on tracers.  If config\_hmix\_use\_ref\_cell\_width = .true. then $\nu_{4h}$ = config\_tracer\_del4*(cellWidth / config\_hmix\_use\_ref\_cell\_width)$^3$. If config\_hmix\_use\_ref\_cell\_width = .false. then it is referenced to the smallest cell.}
\end{longtable}
\end{center}
\section[hmix\_Leith]{\hyperref[sec:nm_tab_hmix_Leith]{hmix\_Leith}}
\label{sec:nm_sec_hmix_Leith}
\subsection[config\_use\_Leith\_del2]{\hyperref[sec:nm_tab_hmix_Leith]{config\_use\_Leith\_del2}}
\label{subsec:nm_sec_config_use_Leith_del2}
\begin{center}
\begin{longtable}{| p{2.0in} || p{4.0in} |}
    \hline
    Type: & logical \\
    \hline
    Units: & -- \\
    \hline
    Default Value: & .false. \\
    \hline
    Possible Values: & .true. or .false. \\
    \hline
    \caption{config\_use\_Leith\_del2: If true, the Leith enstrophy-cascade closure is turned on}
\end{longtable}
\end{center}
\subsection[config\_Leith\_parameter]{\hyperref[sec:nm_tab_hmix_Leith]{config\_Leith\_parameter}}
\label{subsec:nm_sec_config_Leith_parameter}
\begin{center}
\begin{longtable}{| p{2.0in} || p{4.0in} |}
    \hline
    Type: & real \\
    \hline
    Units: & \si{non-dimensional} \\
    \hline
    Default Value: & 1.0 \\
    \hline
    Possible Values: & any positive real \\
    \hline
    \caption{config\_Leith\_parameter: Non-dimensional Leith closure parameter}
\end{longtable}
\end{center}
\subsection[config\_Leith\_dx]{\hyperref[sec:nm_tab_hmix_Leith]{config\_Leith\_dx}}
\label{subsec:nm_sec_config_Leith_dx}
\begin{center}
\begin{longtable}{| p{2.0in} || p{4.0in} |}
    \hline
    Type: & real \\
    \hline
    Units: & \si{m} \\
    \hline
    Default Value: & 15000.0 \\
    \hline
    Possible Values: & any positive real \\
    \hline
    \caption{config\_Leith\_dx: Characteristic length scale, usually the smallest dx in the mesh}
\end{longtable}
\end{center}
\subsection[config\_Leith\_visc2\_max]{\hyperref[sec:nm_tab_hmix_Leith]{config\_Leith\_visc2\_max}}
\label{subsec:nm_sec_config_Leith_visc2_max}
\begin{center}
\begin{longtable}{| p{2.0in} || p{4.0in} |}
    \hline
    Type: & real \\
    \hline
    Units: & \si{m^2.s^-1} \\
    \hline
    Default Value: & 2.5e3 \\
    \hline
    Possible Values: & any positive real \\
    \hline
    \caption{config\_Leith\_visc2\_max: Upper bound on the allowable value of Leith-computed viscosity}
\end{longtable}
\end{center}
\section[Redi\_isopycnal\_mixing]{\hyperref[sec:nm_tab_Redi_isopycnal_mixing]{Redi\_isopycnal\_mixing}}
\label{sec:nm_sec_Redi_isopycnal_mixing}
\subsection[config\_use\_Redi]{\hyperref[sec:nm_tab_Redi_isopycnal_mixing]{config\_use\_Redi}}
\label{subsec:nm_sec_config_use_Redi}
\begin{center}
\begin{longtable}{| p{2.0in} || p{4.0in} |}
    \hline
    Type: & logical \\
    \hline
    Units: & -- \\
    \hline
    Default Value: & .false. \\
    \hline
    Possible Values: & .true. or .false. \\
    \hline
    \caption{config\_use\_Redi: If true, Redi isopycnal mixing is turned on}
\end{longtable}
\end{center}
\subsection[config\_Redi\_closure]{\hyperref[sec:nm_tab_Redi_isopycnal_mixing]{config\_Redi\_closure}}
\label{subsec:nm_sec_config_Redi_closure}
\begin{center}
\begin{longtable}{| p{2.0in} || p{4.0in} |}
    \hline
    Type: & character \\
    \hline
    Units: & -- \\
    \hline
    Default Value: & constant \\
    \hline
    Possible Values: & 'constant', 'equalGM', 'data' \\
    \hline
    \caption{config\_Redi\_closure: Control what type of function is used for Redi $\kappa$. For 'equalGM', RediKappa is set to gmBolusKappa, so picks up the closure used by GM. Note that equalGM should only be used with 2D GM schemes (e.g. config\_GM\_closure=constant or Visbeck), not with EdenGreatbatch.}
\end{longtable}
\end{center}
\subsection[config\_Redi\_constant\_kappa]{\hyperref[sec:nm_tab_Redi_isopycnal_mixing]{config\_Redi\_constant\_kappa}}
\label{subsec:nm_sec_config_Redi_constant_kappa}
\begin{center}
\begin{longtable}{| p{2.0in} || p{4.0in} |}
    \hline
    Type: & real \\
    \hline
    Units: & \si{m^2.s^-1} \\
    \hline
    Default Value: & 600.0 \\
    \hline
    Possible Values: & Positive real numbers. \\
    \hline
    \caption{config\_Redi\_constant\_kappa: The Redi diffusion coefficient. Only used when config\_Redi\_closure = 'constant'.}
\end{longtable}
\end{center}
\subsection[config\_Redi\_maximum\_slope]{\hyperref[sec:nm_tab_Redi_isopycnal_mixing]{config\_Redi\_maximum\_slope}}
\label{subsec:nm_sec_config_Redi_maximum_slope}
\begin{center}
\begin{longtable}{| p{2.0in} || p{4.0in} |}
    \hline
    Type: & real \\
    \hline
    Units: & \si{non-dimensional} \\
    \hline
    Default Value: & 0.3 \\
    \hline
    Possible Values: & positive real numbers, but small \\
    \hline
    \caption{config\_Redi\_maximum\_slope: value of maximum allowed isopycnal slope from Danabasoglu et al 2008 equation (2)}
\end{longtable}
\end{center}
\subsection[config\_Redi\_use\_slope\_taper]{\hyperref[sec:nm_tab_Redi_isopycnal_mixing]{config\_Redi\_use\_slope\_taper}}
\label{subsec:nm_sec_config_Redi_use_slope_taper}
\begin{center}
\begin{longtable}{| p{2.0in} || p{4.0in} |}
    \hline
    Type: & logical \\
    \hline
    Units: & -- \\
    \hline
    Default Value: & .true. \\
    \hline
    Possible Values: & .true. or .false. \\
    \hline
    \caption{config\_Redi\_use\_slope\_taper: If true, Redi is tapered based on Danabasoglu and McWilliams 1995 (slope tapering)}
\end{longtable}
\end{center}
\subsection[config\_Redi\_use\_surface\_taper]{\hyperref[sec:nm_tab_Redi_isopycnal_mixing]{config\_Redi\_use\_surface\_taper}}
\label{subsec:nm_sec_config_Redi_use_surface_taper}
\begin{center}
\begin{longtable}{| p{2.0in} || p{4.0in} |}
    \hline
    Type: & logical \\
    \hline
    Units: & -- \\
    \hline
    Default Value: & .true. \\
    \hline
    Possible Values: & .true. or .false. \\
    \hline
    \caption{config\_Redi\_use\_surface\_taper: If true, Redi slope is tapered near sfc based on Large et al 1997}
\end{longtable}
\end{center}
\subsection[config\_Redi\_limit\_term1]{\hyperref[sec:nm_tab_Redi_isopycnal_mixing]{config\_Redi\_limit\_term1}}
\label{subsec:nm_sec_config_Redi_limit_term1}
\begin{center}
\begin{longtable}{| p{2.0in} || p{4.0in} |}
    \hline
    Type: & logical \\
    \hline
    Units: & -- \\
    \hline
    Default Value: & .true. \\
    \hline
    Possible Values: & .true. or .false. \\
    \hline
    \caption{config\_Redi\_limit\_term1: If true, the N2 limiting is applied to the horizontal diffusion term}
\end{longtable}
\end{center}
\subsection[config\_Redi\_use\_quasi\_monotone\_limiter]{\hyperref[sec:nm_tab_Redi_isopycnal_mixing]{config\_Redi\_use\_quasi\_monotone\_limiter}}
\label{subsec:nm_sec_config_Redi_use_quasi_monotone_limiter}
\begin{center}
\begin{longtable}{| p{2.0in} || p{4.0in} |}
    \hline
    Type: & logical \\
    \hline
    Units: & -- \\
    \hline
    Default Value: & .true. \\
    \hline
    Possible Values: & .true. or .false. \\
    \hline
    \caption{config\_Redi\_use\_quasi\_monotone\_limiter: If true, fluxes are reduced to prevent tracers from violating monotonicity. Cross-term fluxes are scaled toward zero to prevent tracers from under/overshooting the min/max values in adjacent cells and layers}
\end{longtable}
\end{center}
\subsection[config\_Redi\_quasi\_monotone\_safety\_factor]{\hyperref[sec:nm_tab_Redi_isopycnal_mixing]{config\_Redi\_quasi\_monotone\_safety\_factor}}
\label{subsec:nm_sec_config_Redi_quasi_monotone_safety_factor}
\begin{center}
\begin{longtable}{| p{2.0in} || p{4.0in} |}
    \hline
    Type: & real \\
    \hline
    Units: & -- \\
    \hline
    Default Value: & 0.9 \\
    \hline
    Possible Values: & A value between 0 and 1 \\
    \hline
    \caption{config\_Redi\_quasi\_monotone\_safety\_factor: A safety factor applied to flux scaling when monotonicity is violated. Smaller values scale fluxes toward zero more aggressively.}
\end{longtable}
\end{center}
\subsection[config\_Redi\_min\_layers\_diag\_terms]{\hyperref[sec:nm_tab_Redi_isopycnal_mixing]{config\_Redi\_min\_layers\_diag\_terms}}
\label{subsec:nm_sec_config_Redi_min_layers_diag_terms}
\begin{center}
\begin{longtable}{| p{2.0in} || p{4.0in} |}
    \hline
    Type: & integer \\
    \hline
    Units: & -- \\
    \hline
    Default Value: & 6 \\
    \hline
    Possible Values: & any integer between 0 (all layers on) and nVertLevels (all layers off) \\
    \hline
    \caption{config\_Redi\_min\_layers\_diag\_terms: Redi diagonal terms (2 and 3) are turned off from layer 1 through config\_Redi\_min\_layers\_diag\_terms-1, and on from config\_Redi\_min\_layers\_diag\_terms to nVertLevels. The Redi diagonal terms are not guaranteed to produce bounded tracer fields, and in practice produce growing temperatures in a few columns with fewer than 5 vertical cells. Redi is meant for isopycnal mixing in the deep ocean, so not applying Redi diagonal terms in very shallow regions is an acceptable solution.}
\end{longtable}
\end{center}
\subsection[config\_Redi\_horizontal\_taper]{\hyperref[sec:nm_tab_Redi_isopycnal_mixing]{config\_Redi\_horizontal\_taper}}
\label{subsec:nm_sec_config_Redi_horizontal_taper}
\begin{center}
\begin{longtable}{| p{2.0in} || p{4.0in} |}
    \hline
    Type: & character \\
    \hline
    Units: & -- \\
    \hline
    Default Value: & ramp \\
    \hline
    Possible Values: & 'none', 'ramp', 'RossbyRadius' \\
    \hline
    \caption{config\_Redi\_horizontal\_taper: Control how the Redi $\kappa$ value varies as a function of horizontal resolution. 'none' is constant, 'ramp' is strictly based on resolution, 'RossbyRadius' follows Hallberg (2013) - https://doi.org/10.1016/j.ocemod.2013.08.007}
\end{longtable}
\end{center}
\subsection[config\_Redi\_horizontal\_ramp\_min]{\hyperref[sec:nm_tab_Redi_isopycnal_mixing]{config\_Redi\_horizontal\_ramp\_min}}
\label{subsec:nm_sec_config_Redi_horizontal_ramp_min}
\begin{center}
\begin{longtable}{| p{2.0in} || p{4.0in} |}
    \hline
    Type: & real \\
    \hline
    Units: & \si{m} \\
    \hline
    Default Value: & 20e3 \\
    \hline
    Possible Values: & Any positive real value. \\
    \hline
    \caption{config\_Redi\_horizontal\_ramp\_min: Minimum value in grid cell size for Redi $\kappa$ ramp function.  Here cell size refers to dcEdge. Used when config\_Redi\_horizontal\_taper is set to ramp.}
\end{longtable}
\end{center}
\subsection[config\_Redi\_horizontal\_ramp\_max]{\hyperref[sec:nm_tab_Redi_isopycnal_mixing]{config\_Redi\_horizontal\_ramp\_max}}
\label{subsec:nm_sec_config_Redi_horizontal_ramp_max}
\begin{center}
\begin{longtable}{| p{2.0in} || p{4.0in} |}
    \hline
    Type: & real \\
    \hline
    Units: & \si{m} \\
    \hline
    Default Value: & 30e3 \\
    \hline
    Possible Values: & Any positive real value. \\
    \hline
    \caption{config\_Redi\_horizontal\_ramp\_max: Maximum value in grid cell size for Redi $\kappa$ ramp function.  Here cell size refers to dcEdge. Used when config\_Redi\_horizontal\_taper is set to ramp.}
\end{longtable}
\end{center}
\section[submesoscale\_eddy\_parameterization]{\hyperref[sec:nm_tab_submesoscale_eddy_parameterization]{submesoscale\_eddy\_parameterization}}
\label{sec:nm_sec_submesoscale_eddy_parameterization}
\subsection[config\_submesoscale\_enable]{\hyperref[sec:nm_tab_submesoscale_eddy_parameterization]{config\_submesoscale\_enable}}
\label{subsec:nm_sec_config_submesoscale_enable}
\begin{center}
\begin{longtable}{| p{2.0in} || p{4.0in} |}
    \hline
    Type: & logical \\
    \hline
    Units: & -- \\
    \hline
    Default Value: & .false. \\
    \hline
    Possible Values: & .true. or .false. \\
    \hline
    \caption{config\_submesoscale\_enable: flag to enable the FK2011 parameterization for submesoscale eddies}
\end{longtable}
\end{center}
\subsection[config\_submesoscale\_tau]{\hyperref[sec:nm_tab_submesoscale_eddy_parameterization]{config\_submesoscale\_tau}}
\label{subsec:nm_sec_config_submesoscale_tau}
\begin{center}
\begin{longtable}{| p{2.0in} || p{4.0in} |}
    \hline
    Type: & real \\
    \hline
    Units: & \si{s} \\
    \hline
    Default Value: & 172800 \\
    \hline
    Possible Values: & positive reals, between 1-10 days \\
    \hline
    \caption{config\_submesoscale\_tau: timescale for frictional slumping of front (in seconds)}
\end{longtable}
\end{center}
\subsection[config\_submesoscale\_Ce]{\hyperref[sec:nm_tab_submesoscale_eddy_parameterization]{config\_submesoscale\_Ce}}
\label{subsec:nm_sec_config_submesoscale_Ce}
\begin{center}
\begin{longtable}{| p{2.0in} || p{4.0in} |}
    \hline
    Type: & real \\
    \hline
    Units: & -- \\
    \hline
    Default Value: & 0.06 \\
    \hline
    Possible Values: & 0.06 - 0.08 \\
    \hline
    \caption{config\_submesoscale\_Ce: efficiency of submesoscale eddies}
\end{longtable}
\end{center}
\subsection[config\_submesoscale\_Lfmin]{\hyperref[sec:nm_tab_submesoscale_eddy_parameterization]{config\_submesoscale\_Lfmin}}
\label{subsec:nm_sec_config_submesoscale_Lfmin}
\begin{center}
\begin{longtable}{| p{2.0in} || p{4.0in} |}
    \hline
    Type: & real \\
    \hline
    Units: & \si{m} \\
    \hline
    Default Value: & 1000.0 \\
    \hline
    Possible Values: & between 200 and 5000m \\
    \hline
    \caption{config\_submesoscale\_Lfmin: minimum frontal width (meters)}
\end{longtable}
\end{center}
\subsection[config\_submesoscale\_ds\_max]{\hyperref[sec:nm_tab_submesoscale_eddy_parameterization]{config\_submesoscale\_ds\_max}}
\label{subsec:nm_sec_config_submesoscale_ds_max}
\begin{center}
\begin{longtable}{| p{2.0in} || p{4.0in} |}
    \hline
    Type: & real \\
    \hline
    Units: & \si{m} \\
    \hline
    Default Value: & 100000.0 \\
    \hline
    Possible Values: & around 1 degree \\
    \hline
    \caption{config\_submesoscale\_ds\_max: maximum grid scale to scale up buoyancy gradient}
\end{longtable}
\end{center}
\section[GM\_eddy\_parameterization]{\hyperref[sec:nm_tab_GM_eddy_parameterization]{GM\_eddy\_parameterization}}
\label{sec:nm_sec_GM_eddy_parameterization}
\subsection[config\_use\_GM]{\hyperref[sec:nm_tab_GM_eddy_parameterization]{config\_use\_GM}}
\label{subsec:nm_sec_config_use_GM}
\begin{center}
\begin{longtable}{| p{2.0in} || p{4.0in} |}
    \hline
    Type: & logical \\
    \hline
    Units: & -- \\
    \hline
    Default Value: & .false. \\
    \hline
    Possible Values: & .true. or .false. \\
    \hline
    \caption{config\_use\_GM: If true, the standard GM for the tracer advection and mixing is turned on.}
\end{longtable}
\end{center}
\subsection[config\_GM\_closure]{\hyperref[sec:nm_tab_GM_eddy_parameterization]{config\_GM\_closure}}
\label{subsec:nm_sec_config_GM_closure}
\begin{center}
\begin{longtable}{| p{2.0in} || p{4.0in} |}
    \hline
    Type: & character \\
    \hline
    Units: & -- \\
    \hline
    Default Value: & EdenGreatbatch \\
    \hline
    Possible Values: & 'constant', 'N2\_dependent', 'Visbeck', 'EdenGreatbatch' \\
    \hline
    \caption{config\_GM\_closure: Control what method used to compute GM $\kappa$. Both 'constant' and 'N2\_dependent' use the method in Ferrari et al. 2010 (https://doi.org/10.1016/j.ocemod.2010.01.004). 'constant' uses a constant kappa in eqn 16a, while 'N2\_dependent' varies kappa in the vertical according to Danabasoglu and Marshall 2007 (https://doi.org/10.1016/j.ocemod.2007.03.006). 'Visbeck' implements a horizontally varying diffusivity of Visbeck et al 1997. EdenGreatbatch implements a simplified form of the EKE scheme in Eden and Greatbatch (2008) Ocean modeling}
\end{longtable}
\end{center}
\subsection[config\_GM\_constant\_kappa]{\hyperref[sec:nm_tab_GM_eddy_parameterization]{config\_GM\_constant\_kappa}}
\label{subsec:nm_sec_config_GM_constant_kappa}
\begin{center}
\begin{longtable}{| p{2.0in} || p{4.0in} |}
    \hline
    Type: & real \\
    \hline
    Units: & \si{m^2.s^-1} \\
    \hline
    Default Value: & 600.0 \\
    \hline
    Possible Values: & {\bf \color{red} MISSING} \\
    \hline
    \caption{config\_GM\_constant\_kappa: Coefficient of standard GM parametrization of eddy transport (Bolus component), $\kappa$. Only used when config\_GM\_closure is set to constant.}
\end{longtable}
\end{center}
\subsection[config\_GM\_constant\_bclModeSpeed]{\hyperref[sec:nm_tab_GM_eddy_parameterization]{config\_GM\_constant\_bclModeSpeed}}
\label{subsec:nm_sec_config_GM_constant_bclModeSpeed}
\begin{center}
\begin{longtable}{| p{2.0in} || p{4.0in} |}
    \hline
    Type: & real \\
    \hline
    Units: & \si{m.s^-1} \\
    \hline
    Default Value: & 0.3 \\
    \hline
    Possible Values: & Positive real numbers \\
    \hline
    \caption{config\_GM\_constant\_bclModeSpeed: The parameter setting for the first baroclinic mode speed for the vertical stream function boundary value problem. This appears as $c$ in eqn 16a of Ferrari et al. 2010 (https://doi.org/10.1016/j.ocemod.2010.01.004).}
\end{longtable}
\end{center}
\subsection[config\_GM\_minBclModeSpeed\_method]{\hyperref[sec:nm_tab_GM_eddy_parameterization]{config\_GM\_minBclModeSpeed\_method}}
\label{subsec:nm_sec_config_GM_minBclModeSpeed_method}
\begin{center}
\begin{longtable}{| p{2.0in} || p{4.0in} |}
    \hline
    Type: & character \\
    \hline
    Units: & -- \\
    \hline
    Default Value: & constant \\
    \hline
    Possible Values: & 'constant' and 'computed' \\
    \hline
    \caption{config\_GM\_minBclModeSpeed\_method: Determines how the GM setting for the minimum of the first baroclinic mode speed is computed. If 'constant' then use config\_GM\_constant\_bclModeSpeed. If 'computed' then compute at every edge at every time step using the Brunt-Vaisala frequency}
\end{longtable}
\end{center}
\subsection[config\_GM\_spatially\_variable\_min\_kappa]{\hyperref[sec:nm_tab_GM_eddy_parameterization]{config\_GM\_spatially\_variable\_min\_kappa}}
\label{subsec:nm_sec_config_GM_spatially_variable_min_kappa}
\begin{center}
\begin{longtable}{| p{2.0in} || p{4.0in} |}
    \hline
    Type: & real \\
    \hline
    Units: & \si{m^2.s^-1} \\
    \hline
    Default Value: & 300.0 \\
    \hline
    Possible Values: & values around 100s \\
    \hline
    \caption{config\_GM\_spatially\_variable\_min\_kappa: minimum value of bolus diffusivity for spatially variable GM schemes. Used for all choices of config\_GM\_closure other than 'constant'.}
\end{longtable}
\end{center}
\subsection[config\_GM\_spatially\_variable\_max\_kappa]{\hyperref[sec:nm_tab_GM_eddy_parameterization]{config\_GM\_spatially\_variable\_max\_kappa}}
\label{subsec:nm_sec_config_GM_spatially_variable_max_kappa}
\begin{center}
\begin{longtable}{| p{2.0in} || p{4.0in} |}
    \hline
    Type: & real \\
    \hline
    Units: & \si{m^2.s^-1} \\
    \hline
    Default Value: & 1800.0 \\
    \hline
    Possible Values: & values around 100s \\
    \hline
    \caption{config\_GM\_spatially\_variable\_max\_kappa: minimum value of bolus diffusivity for spatially variable GM schemes. Used for all choices of config\_GM\_closure other than 'constant'.}
\end{longtable}
\end{center}
\subsection[config\_GM\_spatially\_variable\_baroclinic\_mode]{\hyperref[sec:nm_tab_GM_eddy_parameterization]{config\_GM\_spatially\_variable\_baroclinic\_mode}}
\label{subsec:nm_sec_config_GM_spatially_variable_baroclinic_mode}
\begin{center}
\begin{longtable}{| p{2.0in} || p{4.0in} |}
    \hline
    Type: & real \\
    \hline
    Units: & -- \\
    \hline
    Default Value: & 1.0 \\
    \hline
    Possible Values: & small positive numbers \\
    \hline
    \caption{config\_GM\_spatially\_variable\_baroclinic\_mode: baroclinic wave mode chosen for the Ferrari et al 2010 calculation. Used for all choices of config\_GM\_closure other than 'constant'.}
\end{longtable}
\end{center}
\subsection[config\_GM\_Visbeck\_alpha]{\hyperref[sec:nm_tab_GM_eddy_parameterization]{config\_GM\_Visbeck\_alpha}}
\label{subsec:nm_sec_config_GM_Visbeck_alpha}
\begin{center}
\begin{longtable}{| p{2.0in} || p{4.0in} |}
    \hline
    Type: & real \\
    \hline
    Units: & -- \\
    \hline
    Default Value: & 0.005 \\
    \hline
    Possible Values: & small positive numbers \\
    \hline
    \caption{config\_GM\_Visbeck\_alpha: scaling factor on the Visbeck diffusivity parameterization}
\end{longtable}
\end{center}
\subsection[config\_GM\_Visbeck\_max\_depth]{\hyperref[sec:nm_tab_GM_eddy_parameterization]{config\_GM\_Visbeck\_max\_depth}}
\label{subsec:nm_sec_config_GM_Visbeck_max_depth}
\begin{center}
\begin{longtable}{| p{2.0in} || p{4.0in} |}
    \hline
    Type: & real \\
    \hline
    Units: & \si{m} \\
    \hline
    Default Value: & 1000.0 \\
    \hline
    Possible Values: & values between zero and bottom depth \\
    \hline
    \caption{config\_GM\_Visbeck\_max\_depth: minimum depth for calculation of vertical average}
\end{longtable}
\end{center}
\subsection[config\_GM\_EG\_riMin]{\hyperref[sec:nm_tab_GM_eddy_parameterization]{config\_GM\_EG\_riMin}}
\label{subsec:nm_sec_config_GM_EG_riMin}
\begin{center}
\begin{longtable}{| p{2.0in} || p{4.0in} |}
    \hline
    Type: & real \\
    \hline
    Units: & -- \\
    \hline
    Default Value: & 200.0 \\
    \hline
    Possible Values: & numbers greater than zero \\
    \hline
    \caption{config\_GM\_EG\_riMin: minimum Richardson number to prevent overly large bolus Kappa values}
\end{longtable}
\end{center}
\subsection[config\_GM\_EG\_kappa\_factor]{\hyperref[sec:nm_tab_GM_eddy_parameterization]{config\_GM\_EG\_kappa\_factor}}
\label{subsec:nm_sec_config_GM_EG_kappa_factor}
\begin{center}
\begin{longtable}{| p{2.0in} || p{4.0in} |}
    \hline
    Type: & real \\
    \hline
    Units: & -- \\
    \hline
    Default Value: & 3.0 \\
    \hline
    Possible Values: & small positive reals \\
    \hline
    \caption{config\_GM\_EG\_kappa\_factor: factor to scale diffusivity for Eden Greatbach scheme}
\end{longtable}
\end{center}
\subsection[config\_GM\_EG\_Rossby\_factor]{\hyperref[sec:nm_tab_GM_eddy_parameterization]{config\_GM\_EG\_Rossby\_factor}}
\label{subsec:nm_sec_config_GM_EG_Rossby_factor}
\begin{center}
\begin{longtable}{| p{2.0in} || p{4.0in} |}
    \hline
    Type: & real \\
    \hline
    Units: & -- \\
    \hline
    Default Value: & 2.0 \\
    \hline
    Possible Values: & small values greater than or equal to one \\
    \hline
    \caption{config\_GM\_EG\_Rossby\_factor: factor multiplying the Rossby length in the scheme from Eden Greatbatch (2008) Ocean Modeling -- Equation (28)}
\end{longtable}
\end{center}
\subsection[config\_GM\_EG\_Rhines\_factor]{\hyperref[sec:nm_tab_GM_eddy_parameterization]{config\_GM\_EG\_Rhines\_factor}}
\label{subsec:nm_sec_config_GM_EG_Rhines_factor}
\begin{center}
\begin{longtable}{| p{2.0in} || p{4.0in} |}
    \hline
    Type: & real \\
    \hline
    Units: & -- \\
    \hline
    Default Value: & 0.3 \\
    \hline
    Possible Values: & small positive values less than equal to one \\
    \hline
    \caption{config\_GM\_EG\_Rhines\_factor: factor multiplying the Rhines length in the scheme from Eden Greatbatch (2008) Ocean Modeling -- Equation (28)}
\end{longtable}
\end{center}
\subsection[config\_GM\_horizontal\_taper]{\hyperref[sec:nm_tab_GM_eddy_parameterization]{config\_GM\_horizontal\_taper}}
\label{subsec:nm_sec_config_GM_horizontal_taper}
\begin{center}
\begin{longtable}{| p{2.0in} || p{4.0in} |}
    \hline
    Type: & character \\
    \hline
    Units: & -- \\
    \hline
    Default Value: & ramp \\
    \hline
    Possible Values: & 'none', 'ramp', 'RossbyRadius' \\
    \hline
    \caption{config\_GM\_horizontal\_taper: Control how the GM Bolus value varies as a function of horizontal resolution. 'none' is constant, 'ramp' is strictly based on resolution, 'RossbyRadius' follows Hallberg (2013) - https://doi.org/10.1016/j.ocemod.2013.08.007}
\end{longtable}
\end{center}
\subsection[config\_GM\_horizontal\_ramp\_min]{\hyperref[sec:nm_tab_GM_eddy_parameterization]{config\_GM\_horizontal\_ramp\_min}}
\label{subsec:nm_sec_config_GM_horizontal_ramp_min}
\begin{center}
\begin{longtable}{| p{2.0in} || p{4.0in} |}
    \hline
    Type: & real \\
    \hline
    Units: & \si{m} \\
    \hline
    Default Value: & 20e3 \\
    \hline
    Possible Values: & Any positive real value. \\
    \hline
    \caption{config\_GM\_horizontal\_ramp\_min: Minimum value in grid cell size for GM $\kappa$ ramp function.  Here cell size refers to dcEdge. Used when config\_GM\_horizontal\_taper is set to ramp.}
\end{longtable}
\end{center}
\subsection[config\_GM\_horizontal\_ramp\_max]{\hyperref[sec:nm_tab_GM_eddy_parameterization]{config\_GM\_horizontal\_ramp\_max}}
\label{subsec:nm_sec_config_GM_horizontal_ramp_max}
\begin{center}
\begin{longtable}{| p{2.0in} || p{4.0in} |}
    \hline
    Type: & real \\
    \hline
    Units: & \si{m} \\
    \hline
    Default Value: & 30e3 \\
    \hline
    Possible Values: & Any positive real value. \\
    \hline
    \caption{config\_GM\_horizontal\_ramp\_max: Maximum value in grid cell size for GM $\kappa$ ramp function.  Here cell size refers to dcEdge. Used when config\_GM\_horizontal\_taper is set to ramp.}
\end{longtable}
\end{center}
\subsection[config\_GMRedi\_Rossby\_ramp\_min]{\hyperref[sec:nm_tab_GM_eddy_parameterization]{config\_GMRedi\_Rossby\_ramp\_min}}
\label{subsec:nm_sec_config_GMRedi_Rossby_ramp_min}
\begin{center}
\begin{longtable}{| p{2.0in} || p{4.0in} |}
    \hline
    Type: & real \\
    \hline
    Units: & -- \\
    \hline
    Default Value: & 0.5 \\
    \hline
    Possible Values: & Any positive real value. \\
    \hline
    \caption{config\_GMRedi\_Rossby\_ramp\_min: Minimum value of the ratio between grid-cell size (dcEdge) and Rossby radius for GM and Redi $\kappa$ ramp functions. Used when config\_GM\_horizontal\_taper and/or config\_Redi\_horizontal\_taper are set to RossbyRadius.}
\end{longtable}
\end{center}
\subsection[config\_GMRedi\_Rossby\_ramp\_max]{\hyperref[sec:nm_tab_GM_eddy_parameterization]{config\_GMRedi\_Rossby\_ramp\_max}}
\label{subsec:nm_sec_config_GMRedi_Rossby_ramp_max}
\begin{center}
\begin{longtable}{| p{2.0in} || p{4.0in} |}
    \hline
    Type: & real \\
    \hline
    Units: & -- \\
    \hline
    Default Value: & 3.0 \\
    \hline
    Possible Values: & Any positive real value. \\
    \hline
    \caption{config\_GMRedi\_Rossby\_ramp\_max: Maximum value of the ratio between grid-cell size (dcEdge) and Rossby radius for GM and Redi $\kappa$ ramp functions. Used when config\_GM\_horizontal\_taper and/or config\_Redi\_horizontal\_taper are set to RossbyRadius.}
\end{longtable}
\end{center}
\section[eddy\_parameterization]{\hyperref[sec:nm_tab_eddy_parameterization]{eddy\_parameterization}}
\label{sec:nm_sec_eddy_parameterization}
\subsection[config\_eddyMLD\_dens\_threshold]{\hyperref[sec:nm_tab_eddy_parameterization]{config\_eddyMLD\_dens\_threshold}}
\label{subsec:nm_sec_config_eddyMLD_dens_threshold}
\begin{center}
\begin{longtable}{| p{2.0in} || p{4.0in} |}
    \hline
    Type: & real \\
    \hline
    Units: & \si{kg.m^-3} \\
    \hline
    Default Value: & 0.03 \\
    \hline
    Possible Values: & suggested range 0.01 less than or equal to thresh less than or equal to 0.5 \\
    \hline
    \caption{config\_eddyMLD\_dens\_threshold: potential density change relative to surface for mixed layer depth threshold method.  This calculation is used for the Redi tapering, GM N2\_dependent bolus kappa, and the submesoscale eddy parameterization}
\end{longtable}
\end{center}
\subsection[config\_eddyMLD\_reference\_depth]{\hyperref[sec:nm_tab_eddy_parameterization]{config\_eddyMLD\_reference\_depth}}
\label{subsec:nm_sec_config_eddyMLD_reference_depth}
\begin{center}
\begin{longtable}{| p{2.0in} || p{4.0in} |}
    \hline
    Type: & real \\
    \hline
    Units: & \si{m} \\
    \hline
    Default Value: & 10 \\
    \hline
    Possible Values: & any positive real, near 10 \\
    \hline
    \caption{config\_eddyMLD\_reference\_depth: reference depth for threshold computation}
\end{longtable}
\end{center}
\subsection[config\_eddyMLD\_reference\_pressure]{\hyperref[sec:nm_tab_eddy_parameterization]{config\_eddyMLD\_reference\_pressure}}
\label{subsec:nm_sec_config_eddyMLD_reference_pressure}
\begin{center}
\begin{longtable}{| p{2.0in} || p{4.0in} |}
    \hline
    Type: & real \\
    \hline
    Units: & \si{Pa} \\
    \hline
    Default Value: & 1.0e5 \\
    \hline
    Possible Values: & positive reals around 1.0e5 \\
    \hline
    \caption{config\_eddyMLD\_reference\_pressure: reference pressure for original mixed layer depth calculation}
\end{longtable}
\end{center}
\subsection[config\_eddyMLD\_use\_old]{\hyperref[sec:nm_tab_eddy_parameterization]{config\_eddyMLD\_use\_old}}
\label{subsec:nm_sec_config_eddyMLD_use_old}
\begin{center}
\begin{longtable}{| p{2.0in} || p{4.0in} |}
    \hline
    Type: & logical \\
    \hline
    Units: & -- \\
    \hline
    Default Value: & .true. \\
    \hline
    Possible Values: & .true. or .false. \\
    \hline
    \caption{config\_eddyMLD\_use\_old: switches from old dThreshMLD calculation to new (fixed one)}
\end{longtable}
\end{center}
\section[cvmix]{\hyperref[sec:nm_tab_cvmix]{cvmix}}
\label{sec:nm_sec_cvmix}
\subsection[config\_use\_cvmix]{\hyperref[sec:nm_tab_cvmix]{config\_use\_cvmix}}
\label{subsec:nm_sec_config_use_cvmix}
\begin{center}
\begin{longtable}{| p{2.0in} || p{4.0in} |}
    \hline
    Type: & logical \\
    \hline
    Units: & -- \\
    \hline
    Default Value: & .true. \\
    \hline
    Possible Values: & True or False \\
    \hline
    \caption{config\_use\_cvmix: If true, use the Community Vertical MIXing routines to compute vertical diffusivity and viscosity}
\end{longtable}
\end{center}
\subsection[config\_cvmix\_prandtl\_number]{\hyperref[sec:nm_tab_cvmix]{config\_cvmix\_prandtl\_number}}
\label{subsec:nm_sec_config_cvmix_prandtl_number}
\begin{center}
\begin{longtable}{| p{2.0in} || p{4.0in} |}
    \hline
    Type: & real \\
    \hline
    Units: & \si{non-dimensional} \\
    \hline
    Default Value: & 1.0 \\
    \hline
    Possible Values: & Any non-negative real value. \\
    \hline
    \caption{config\_cvmix\_prandtl\_number: Prandtl number to be used within the CVMix parameterization suite}
\end{longtable}
\end{center}
\subsection[config\_cvmix\_background\_scheme]{\hyperref[sec:nm_tab_cvmix]{config\_cvmix\_background\_scheme}}
\label{subsec:nm_sec_config_cvmix_background_scheme}
\begin{center}
\begin{longtable}{| p{2.0in} || p{4.0in} |}
    \hline
    Type: & character \\
    \hline
    Units: & -- \\
    \hline
    Default Value: & constant \\
    \hline
    Possible Values: & 'constant', 'BryanLewis', and 'none' \\
    \hline
    \caption{config\_cvmix\_background\_scheme: Scheme for background diffusivity, 'constant' for constant with depth and space, 'BryanLewis' for vertically variable, 'none' for no background diffusivity}
\end{longtable}
\end{center}
\subsection[config\_cvmix\_background\_diffusion]{\hyperref[sec:nm_tab_cvmix]{config\_cvmix\_background\_diffusion}}
\label{subsec:nm_sec_config_cvmix_background_diffusion}
\begin{center}
\begin{longtable}{| p{2.0in} || p{4.0in} |}
    \hline
    Type: & real \\
    \hline
    Units: & \si{m^2.s^-1} \\
    \hline
    Default Value: & 1.0e-5 \\
    \hline
    Possible Values: & Any positive real value. \\
    \hline
    \caption{config\_cvmix\_background\_diffusion: Background vertical diffusion applied to tracer quantities}
\end{longtable}
\end{center}
\subsection[config\_cvmix\_background\_viscosity]{\hyperref[sec:nm_tab_cvmix]{config\_cvmix\_background\_viscosity}}
\label{subsec:nm_sec_config_cvmix_background_viscosity}
\begin{center}
\begin{longtable}{| p{2.0in} || p{4.0in} |}
    \hline
    Type: & real \\
    \hline
    Units: & \si{m^2.s^-1} \\
    \hline
    Default Value: & 1.0e-4 \\
    \hline
    Possible Values: & Any positive real value. \\
    \hline
    \caption{config\_cvmix\_background\_viscosity: Background vertical viscosity applied to horizontal velocity}
\end{longtable}
\end{center}
\subsection[config\_cvmix\_BryanLewis\_bl1]{\hyperref[sec:nm_tab_cvmix]{config\_cvmix\_BryanLewis\_bl1}}
\label{subsec:nm_sec_config_cvmix_BryanLewis_bl1}
\begin{center}
\begin{longtable}{| p{2.0in} || p{4.0in} |}
    \hline
    Type: & real \\
    \hline
    Units: & \si{m^2/s} \\
    \hline
    Default Value: & 8.0e-5 \\
    \hline
    Possible Values: & small positive real numbers \\
    \hline
    \caption{config\_cvmix\_BryanLewis\_bl1: near surface diffusivity for the Bryan and Lewis (1979) profile}
\end{longtable}
\end{center}
\subsection[config\_cvmix\_BryanLewis\_bl2]{\hyperref[sec:nm_tab_cvmix]{config\_cvmix\_BryanLewis\_bl2}}
\label{subsec:nm_sec_config_cvmix_BryanLewis_bl2}
\begin{center}
\begin{longtable}{| p{2.0in} || p{4.0in} |}
    \hline
    Type: & real \\
    \hline
    Units: & \si{m^2/s} \\
    \hline
    Default Value: & 1.05E-4 \\
    \hline
    Possible Values: & small positive real numbers \\
    \hline
    \caption{config\_cvmix\_BryanLewis\_bl2: increase in diffusivity at depth for Bryan Lewis (1979) scheme}
\end{longtable}
\end{center}
\subsection[config\_cvmix\_BryanLewis\_transitionDepth]{\hyperref[sec:nm_tab_cvmix]{config\_cvmix\_BryanLewis\_transitionDepth}}
\label{subsec:nm_sec_config_cvmix_BryanLewis_transitionDepth}
\begin{center}
\begin{longtable}{| p{2.0in} || p{4.0in} |}
    \hline
    Type: & real \\
    \hline
    Units: & \si{m} \\
    \hline
    Default Value: & 2500 \\
    \hline
    Possible Values: & positive real numbers \\
    \hline
    \caption{config\_cvmix\_BryanLewis\_transitionDepth: depth at which the diffusivity transitions to the higher value}
\end{longtable}
\end{center}
\subsection[config\_cvmix\_BryanLewis\_transitionWidth]{\hyperref[sec:nm_tab_cvmix]{config\_cvmix\_BryanLewis\_transitionWidth}}
\label{subsec:nm_sec_config_cvmix_BryanLewis_transitionWidth}
\begin{center}
\begin{longtable}{| p{2.0in} || p{4.0in} |}
    \hline
    Type: & real \\
    \hline
    Units: & \si{m} \\
    \hline
    Default Value: & 222. \\
    \hline
    Possible Values: & positive real numbers \\
    \hline
    \caption{config\_cvmix\_BryanLewis\_transitionWidth: width of transition in Bryan Lewis (1979) scheme}
\end{longtable}
\end{center}
\subsection[config\_use\_cvmix\_convection]{\hyperref[sec:nm_tab_cvmix]{config\_use\_cvmix\_convection}}
\label{subsec:nm_sec_config_use_cvmix_convection}
\begin{center}
\begin{longtable}{| p{2.0in} || p{4.0in} |}
    \hline
    Type: & logical \\
    \hline
    Units: & -- \\
    \hline
    Default Value: & .false. \\
    \hline
    Possible Values: & True or False \\
    \hline
    \caption{config\_use\_cvmix\_convection: If true, convective diffusivity and viscosity is computed using CVMix}
\end{longtable}
\end{center}
\subsection[config\_cvmix\_convective\_diffusion]{\hyperref[sec:nm_tab_cvmix]{config\_cvmix\_convective\_diffusion}}
\label{subsec:nm_sec_config_cvmix_convective_diffusion}
\begin{center}
\begin{longtable}{| p{2.0in} || p{4.0in} |}
    \hline
    Type: & real \\
    \hline
    Units: & \si{m^2.s^-1} \\
    \hline
    Default Value: & 1.0 \\
    \hline
    Possible Values: & Any positive real value. \\
    \hline
    \caption{config\_cvmix\_convective\_diffusion: Convective vertical diffusion applied to tracer quantities}
\end{longtable}
\end{center}
\subsection[config\_cvmix\_convective\_viscosity]{\hyperref[sec:nm_tab_cvmix]{config\_cvmix\_convective\_viscosity}}
\label{subsec:nm_sec_config_cvmix_convective_viscosity}
\begin{center}
\begin{longtable}{| p{2.0in} || p{4.0in} |}
    \hline
    Type: & real \\
    \hline
    Units: & \si{m^2.s^-1} \\
    \hline
    Default Value: & 1.0 \\
    \hline
    Possible Values: & Any positive real value. \\
    \hline
    \caption{config\_cvmix\_convective\_viscosity: Convective vertical viscosity applied to horizontal velocity components}
\end{longtable}
\end{center}
\subsection[config\_cvmix\_convective\_basedOnBVF]{\hyperref[sec:nm_tab_cvmix]{config\_cvmix\_convective\_basedOnBVF}}
\label{subsec:nm_sec_config_cvmix_convective_basedOnBVF}
\begin{center}
\begin{longtable}{| p{2.0in} || p{4.0in} |}
    \hline
    Type: & logical \\
    \hline
    Units: & -- \\
    \hline
    Default Value: & .true. \\
    \hline
    Possible Values: & True or False \\
    \hline
    \caption{config\_cvmix\_convective\_basedOnBVF: If true, convection is triggered based on value of config\_cvmix\_convective\_triggerBVF}
\end{longtable}
\end{center}
\subsection[config\_cvmix\_convective\_triggerBVF]{\hyperref[sec:nm_tab_cvmix]{config\_cvmix\_convective\_triggerBVF}}
\label{subsec:nm_sec_config_cvmix_convective_triggerBVF}
\begin{center}
\begin{longtable}{| p{2.0in} || p{4.0in} |}
    \hline
    Type: & real \\
    \hline
    Units: & \si{s^-2} \\
    \hline
    Default Value: & 0.0 \\
    \hline
    Possible Values: & Any real value \\
    \hline
    \caption{config\_cvmix\_convective\_triggerBVF: Value of Brunt Viasala frequency squared below which convective mixing is triggered}
\end{longtable}
\end{center}
\subsection[config\_use\_cvmix\_shear]{\hyperref[sec:nm_tab_cvmix]{config\_use\_cvmix\_shear}}
\label{subsec:nm_sec_config_use_cvmix_shear}
\begin{center}
\begin{longtable}{| p{2.0in} || p{4.0in} |}
    \hline
    Type: & logical \\
    \hline
    Units: & -- \\
    \hline
    Default Value: & .false. \\
    \hline
    Possible Values: & True or False \\
    \hline
    \caption{config\_use\_cvmix\_shear: If true, shear-based mixing is computed using CVMix}
\end{longtable}
\end{center}
\subsection[config\_cvmix\_num\_ri\_smooth\_loops]{\hyperref[sec:nm_tab_cvmix]{config\_cvmix\_num\_ri\_smooth\_loops}}
\label{subsec:nm_sec_config_cvmix_num_ri_smooth_loops}
\begin{center}
\begin{longtable}{| p{2.0in} || p{4.0in} |}
    \hline
    Type: & integer \\
    \hline
    Units: & -- \\
    \hline
    Default Value: & 2 \\
    \hline
    Possible Values: & any integer \\
    \hline
    \caption{config\_cvmix\_num\_ri\_smooth\_loops: Number of smoothing passes over RiTopOfCell for LMD94 shear instability mixing}
\end{longtable}
\end{center}
\subsection[config\_cvmix\_use\_BLD\_smoothing]{\hyperref[sec:nm_tab_cvmix]{config\_cvmix\_use\_BLD\_smoothing}}
\label{subsec:nm_sec_config_cvmix_use_BLD_smoothing}
\begin{center}
\begin{longtable}{| p{2.0in} || p{4.0in} |}
    \hline
    Type: & logical \\
    \hline
    Units: & -- \\
    \hline
    Default Value: & .true. \\
    \hline
    Possible Values: & .true. or .false. \\
    \hline
    \caption{config\_cvmix\_use\_BLD\_smoothing: If true KPP bld is smoothed with a laplacian filter}
\end{longtable}
\end{center}
\subsection[config\_cvmix\_shear\_mixing\_scheme]{\hyperref[sec:nm_tab_cvmix]{config\_cvmix\_shear\_mixing\_scheme}}
\label{subsec:nm_sec_config_cvmix_shear_mixing_scheme}
\begin{center}
\begin{longtable}{| p{2.0in} || p{4.0in} |}
    \hline
    Type: & character \\
    \hline
    Units: & -- \\
    \hline
    Default Value: & PP \\
    \hline
    Possible Values: & PP or KPP \\
    \hline
    \caption{config\_cvmix\_shear\_mixing\_scheme: Choose between Pacanowski/Philander or Large et al. shear mixing}
\end{longtable}
\end{center}
\subsection[config\_cvmix\_shear\_PP\_nu\_zero]{\hyperref[sec:nm_tab_cvmix]{config\_cvmix\_shear\_PP\_nu\_zero}}
\label{subsec:nm_sec_config_cvmix_shear_PP_nu_zero}
\begin{center}
\begin{longtable}{| p{2.0in} || p{4.0in} |}
    \hline
    Type: & real \\
    \hline
    Units: & \si{m^2.s^-1} \\
    \hline
    Default Value: & 0.005 \\
    \hline
    Possible Values: & Any positive real value \\
    \hline
    \caption{config\_cvmix\_shear\_PP\_nu\_zero: Numerator of Pacanowski and Philander (1981) Eq (1)}
\end{longtable}
\end{center}
\subsection[config\_cvmix\_shear\_PP\_alpha]{\hyperref[sec:nm_tab_cvmix]{config\_cvmix\_shear\_PP\_alpha}}
\label{subsec:nm_sec_config_cvmix_shear_PP_alpha}
\begin{center}
\begin{longtable}{| p{2.0in} || p{4.0in} |}
    \hline
    Type: & real \\
    \hline
    Units: & -- \\
    \hline
    Default Value: & 5.0 \\
    \hline
    Possible Values: & Any positive real value \\
    \hline
    \caption{config\_cvmix\_shear\_PP\_alpha: Alpha values used in Pacanowski and Philander (1981) Eqs (1) and (2)}
\end{longtable}
\end{center}
\subsection[config\_cvmix\_shear\_PP\_exp]{\hyperref[sec:nm_tab_cvmix]{config\_cvmix\_shear\_PP\_exp}}
\label{subsec:nm_sec_config_cvmix_shear_PP_exp}
\begin{center}
\begin{longtable}{| p{2.0in} || p{4.0in} |}
    \hline
    Type: & real \\
    \hline
    Units: & -- \\
    \hline
    Default Value: & 2.0 \\
    \hline
    Possible Values: & Any positive real value \\
    \hline
    \caption{config\_cvmix\_shear\_PP\_exp: Exponent used in denominator of Pacanowski and Philander (1981) Eqs (1)}
\end{longtable}
\end{center}
\subsection[config\_cvmix\_shear\_KPP\_nu\_zero]{\hyperref[sec:nm_tab_cvmix]{config\_cvmix\_shear\_KPP\_nu\_zero}}
\label{subsec:nm_sec_config_cvmix_shear_KPP_nu_zero}
\begin{center}
\begin{longtable}{| p{2.0in} || p{4.0in} |}
    \hline
    Type: & real \\
    \hline
    Units: & \si{m^2.s^-1} \\
    \hline
    Default Value: & 0.005 \\
    \hline
    Possible Values: & Any positive real value \\
    \hline
    \caption{config\_cvmix\_shear\_KPP\_nu\_zero: Maximum diffusivity produced by shear-generated mixing}
\end{longtable}
\end{center}
\subsection[config\_cvmix\_shear\_KPP\_Ri\_zero]{\hyperref[sec:nm_tab_cvmix]{config\_cvmix\_shear\_KPP\_Ri\_zero}}
\label{subsec:nm_sec_config_cvmix_shear_KPP_Ri_zero}
\begin{center}
\begin{longtable}{| p{2.0in} || p{4.0in} |}
    \hline
    Type: & real \\
    \hline
    Units: & \si{non-dimensional} \\
    \hline
    Default Value: & 0.7 \\
    \hline
    Possible Values: & Any positive real value \\
    \hline
    \caption{config\_cvmix\_shear\_KPP\_Ri\_zero: Theshold gradient Richardson number to produced enhanced diffusivities, See Large et al. (1994) Eq (28a,b,c)}
\end{longtable}
\end{center}
\subsection[config\_cvmix\_shear\_KPP\_exp]{\hyperref[sec:nm_tab_cvmix]{config\_cvmix\_shear\_KPP\_exp}}
\label{subsec:nm_sec_config_cvmix_shear_KPP_exp}
\begin{center}
\begin{longtable}{| p{2.0in} || p{4.0in} |}
    \hline
    Type: & real \\
    \hline
    Units: & -- \\
    \hline
    Default Value: & 3 \\
    \hline
    Possible Values: & Any positive real value \\
    \hline
    \caption{config\_cvmix\_shear\_KPP\_exp: Exponent relating diffusivities to $Ri_g$. Referred to as $p_1$ in Large et al. (1994) Eq (28b)}
\end{longtable}
\end{center}
\subsection[config\_use\_cvmix\_tidal\_mixing]{\hyperref[sec:nm_tab_cvmix]{config\_use\_cvmix\_tidal\_mixing}}
\label{subsec:nm_sec_config_use_cvmix_tidal_mixing}
\begin{center}
\begin{longtable}{| p{2.0in} || p{4.0in} |}
    \hline
    Type: & logical \\
    \hline
    Units: & -- \\
    \hline
    Default Value: & .false. \\
    \hline
    Possible Values: & True or False \\
    \hline
    \caption{config\_use\_cvmix\_tidal\_mixing: If true, diffusivity and viscosity is computed using CVMix tidal mixing}
\end{longtable}
\end{center}
\subsection[config\_use\_cvmix\_double\_diffusion]{\hyperref[sec:nm_tab_cvmix]{config\_use\_cvmix\_double\_diffusion}}
\label{subsec:nm_sec_config_use_cvmix_double_diffusion}
\begin{center}
\begin{longtable}{| p{2.0in} || p{4.0in} |}
    \hline
    Type: & logical \\
    \hline
    Units: & -- \\
    \hline
    Default Value: & .false. \\
    \hline
    Possible Values: & True or False \\
    \hline
    \caption{config\_use\_cvmix\_double\_diffusion: If true, diffusivity and viscosity is computed using CVMix double diffusion}
\end{longtable}
\end{center}
\subsection[config\_use\_cvmix\_kpp]{\hyperref[sec:nm_tab_cvmix]{config\_use\_cvmix\_kpp}}
\label{subsec:nm_sec_config_use_cvmix_kpp}
\begin{center}
\begin{longtable}{| p{2.0in} || p{4.0in} |}
    \hline
    Type: & logical \\
    \hline
    Units: & -- \\
    \hline
    Default Value: & .false. \\
    \hline
    Possible Values: & True or False \\
    \hline
    \caption{config\_use\_cvmix\_kpp: If true, diffusivity and viscosity is computed using CVMix KPP}
\end{longtable}
\end{center}
\subsection[config\_use\_cvmix\_fixed\_boundary\_layer]{\hyperref[sec:nm_tab_cvmix]{config\_use\_cvmix\_fixed\_boundary\_layer}}
\label{subsec:nm_sec_config_use_cvmix_fixed_boundary_layer}
\begin{center}
\begin{longtable}{| p{2.0in} || p{4.0in} |}
    \hline
    Type: & logical \\
    \hline
    Units: & -- \\
    \hline
    Default Value: & .false. \\
    \hline
    Possible Values: & True or False \\
    \hline
    \caption{config\_use\_cvmix\_fixed\_boundary\_layer: If true, boundary layer depth is specified as config\_cvmix\_kpp\_boundary\_layer\_depth}
\end{longtable}
\end{center}
\subsection[config\_cvmix\_kpp\_boundary\_layer\_depth]{\hyperref[sec:nm_tab_cvmix]{config\_cvmix\_kpp\_boundary\_layer\_depth}}
\label{subsec:nm_sec_config_cvmix_kpp_boundary_layer_depth}
\begin{center}
\begin{longtable}{| p{2.0in} || p{4.0in} |}
    \hline
    Type: & real \\
    \hline
    Units: & \si{m} \\
    \hline
    Default Value: & 30.0 \\
    \hline
    Possible Values: & Any positive real value. \\
    \hline
    \caption{config\_cvmix\_kpp\_boundary\_layer\_depth: If config\_use\_cvmix\_fixed\_boundary\_layer, then KPP OBL calculation is overwritten with this value}
\end{longtable}
\end{center}
\subsection[config\_cvmix\_kpp\_criticalBulkRichardsonNumber]{\hyperref[sec:nm_tab_cvmix]{config\_cvmix\_kpp\_criticalBulkRichardsonNumber}}
\label{subsec:nm_sec_config_cvmix_kpp_criticalBulkRichardsonNumber}
\begin{center}
\begin{longtable}{| p{2.0in} || p{4.0in} |}
    \hline
    Type: & real \\
    \hline
    Units: & \si{non-dimensional} \\
    \hline
    Default Value: & 0.25 \\
    \hline
    Possible Values: & Any positive real value. \\
    \hline
    \caption{config\_cvmix\_kpp\_criticalBulkRichardsonNumber: Critical bulk Richardson number used to determine bottom of ocean mixed layer}
\end{longtable}
\end{center}
\subsection[config\_cvmix\_kpp\_matching]{\hyperref[sec:nm_tab_cvmix]{config\_cvmix\_kpp\_matching}}
\label{subsec:nm_sec_config_cvmix_kpp_matching}
\begin{center}
\begin{longtable}{| p{2.0in} || p{4.0in} |}
    \hline
    Type: & character \\
    \hline
    Units: & -- \\
    \hline
    Default Value: & SimpleShapes \\
    \hline
    Possible Values: & MatchBoth, MatchGradient, SimpleShapes \\
    \hline
    \caption{config\_cvmix\_kpp\_matching: Determines how the KPP diffusivities are matched to values at base of boundary layer}
\end{longtable}
\end{center}
\subsection[config\_cvmix\_kpp\_EkmanOBL]{\hyperref[sec:nm_tab_cvmix]{config\_cvmix\_kpp\_EkmanOBL}}
\label{subsec:nm_sec_config_cvmix_kpp_EkmanOBL}
\begin{center}
\begin{longtable}{| p{2.0in} || p{4.0in} |}
    \hline
    Type: & logical \\
    \hline
    Units: & -- \\
    \hline
    Default Value: & .false. \\
    \hline
    Possible Values: & True or False \\
    \hline
    \caption{config\_cvmix\_kpp\_EkmanOBL: If true, boundary layer depth is limited by Ekman layer depth}
\end{longtable}
\end{center}
\subsection[config\_cvmix\_kpp\_MonObOBL]{\hyperref[sec:nm_tab_cvmix]{config\_cvmix\_kpp\_MonObOBL}}
\label{subsec:nm_sec_config_cvmix_kpp_MonObOBL}
\begin{center}
\begin{longtable}{| p{2.0in} || p{4.0in} |}
    \hline
    Type: & logical \\
    \hline
    Units: & -- \\
    \hline
    Default Value: & .false. \\
    \hline
    Possible Values: & True or False \\
    \hline
    \caption{config\_cvmix\_kpp\_MonObOBL: If true, boundary layer depth is limited by Monin-Obukhov layer depth}
\end{longtable}
\end{center}
\subsection[config\_cvmix\_kpp\_interpolationOMLType]{\hyperref[sec:nm_tab_cvmix]{config\_cvmix\_kpp\_interpolationOMLType}}
\label{subsec:nm_sec_config_cvmix_kpp_interpolationOMLType}
\begin{center}
\begin{longtable}{| p{2.0in} || p{4.0in} |}
    \hline
    Type: & character \\
    \hline
    Units: & -- \\
    \hline
    Default Value: & quadratic \\
    \hline
    Possible Values: & linear, quadratic, cubic \\
    \hline
    \caption{config\_cvmix\_kpp\_interpolationOMLType: Determine bottom of ocean mixed layer using linear, quadratic or cubic interpolation}
\end{longtable}
\end{center}
\subsection[config\_cvmix\_kpp\_surface\_layer\_extent]{\hyperref[sec:nm_tab_cvmix]{config\_cvmix\_kpp\_surface\_layer\_extent}}
\label{subsec:nm_sec_config_cvmix_kpp_surface_layer_extent}
\begin{center}
\begin{longtable}{| p{2.0in} || p{4.0in} |}
    \hline
    Type: & real \\
    \hline
    Units: & \si{non-dimensional} \\
    \hline
    Default Value: & 0.1 \\
    \hline
    Possible Values: & Any value between 0 and 1 \\
    \hline
    \caption{config\_cvmix\_kpp\_surface\_layer\_extent: The non-dimensional extent of the surface layer, measured as fraction of boundary layer depth}
\end{longtable}
\end{center}
\subsection[config\_cvmix\_kpp\_surface\_layer\_averaging]{\hyperref[sec:nm_tab_cvmix]{config\_cvmix\_kpp\_surface\_layer\_averaging}}
\label{subsec:nm_sec_config_cvmix_kpp_surface_layer_averaging}
\begin{center}
\begin{longtable}{| p{2.0in} || p{4.0in} |}
    \hline
    Type: & real \\
    \hline
    Units: & \si{m} \\
    \hline
    Default Value: & 5.0 \\
    \hline
    Possible Values: & Any positive real value, but typically should be between 1 and 20 meters \\
    \hline
    \caption{config\_cvmix\_kpp\_surface\_layer\_averaging: The thickness over which to average when computing surface-averaged velocity and buoyancy}
\end{longtable}
\end{center}
\subsection[configure\_cvmix\_kpp\_minimum\_OBL\_under\_sea\_ice]{\hyperref[sec:nm_tab_cvmix]{configure\_cvmix\_kpp\_minimum\_OBL\_under\_sea\_ice}}
\label{subsec:nm_sec_configure_cvmix_kpp_minimum_OBL_under_sea_ice}
\begin{center}
\begin{longtable}{| p{2.0in} || p{4.0in} |}
    \hline
    Type: & real \\
    \hline
    Units: & \si{m} \\
    \hline
    Default Value: & 10.0 \\
    \hline
    Possible Values: & Any positive real value, but typically should be between 1 and 20 meters \\
    \hline
    \caption{configure\_cvmix\_kpp\_minimum\_OBL\_under\_sea\_ice: The minimum allowable boundary layer depth with sea-ice is present}
\end{longtable}
\end{center}
\subsection[config\_cvmix\_kpp\_stop\_OBL\_search]{\hyperref[sec:nm_tab_cvmix]{config\_cvmix\_kpp\_stop\_OBL\_search}}
\label{subsec:nm_sec_config_cvmix_kpp_stop_OBL_search}
\begin{center}
\begin{longtable}{| p{2.0in} || p{4.0in} |}
    \hline
    Type: & real \\
    \hline
    Units: & \si{non-dimensional} \\
    \hline
    Default Value: & 100.0 \\
    \hline
    Possible Values: & Any positive value \\
    \hline
    \caption{config\_cvmix\_kpp\_stop\_OBL\_search: The search for boundary layer depth is terminated when bulk Richardson number is greater than config\_cvmix\_kpp\_stop\_OBL\_search*config\_cvmix\_kpp\_criticalBulkRichardsonNumber}
\end{longtable}
\end{center}
\subsection[config\_cvmix\_kpp\_use\_enhanced\_diff]{\hyperref[sec:nm_tab_cvmix]{config\_cvmix\_kpp\_use\_enhanced\_diff}}
\label{subsec:nm_sec_config_cvmix_kpp_use_enhanced_diff}
\begin{center}
\begin{longtable}{| p{2.0in} || p{4.0in} |}
    \hline
    Type: & logical \\
    \hline
    Units: & -- \\
    \hline
    Default Value: & .true. \\
    \hline
    Possible Values: & .true. or .false. \\
    \hline
    \caption{config\_cvmix\_kpp\_use\_enhanced\_diff: Flag for use of enhanced diffusion at boundary layer base as in Large et al (1994)}
\end{longtable}
\end{center}
\subsection[config\_cvmix\_kpp\_nonlocal\_with\_implicit\_mix]{\hyperref[sec:nm_tab_cvmix]{config\_cvmix\_kpp\_nonlocal\_with\_implicit\_mix}}
\label{subsec:nm_sec_config_cvmix_kpp_nonlocal_with_implicit_mix}
\begin{center}
\begin{longtable}{| p{2.0in} || p{4.0in} |}
    \hline
    Type: & logical \\
    \hline
    Units: & -- \\
    \hline
    Default Value: & .false. \\
    \hline
    Possible Values: & .true. or .false. \\
    \hline
    \caption{config\_cvmix\_kpp\_nonlocal\_with\_implicit\_mix: flag that moves the non-local computation and application of tendency to after main timestep loop}
\end{longtable}
\end{center}
\subsection[config\_cvmix\_kpp\_use\_theory\_wave]{\hyperref[sec:nm_tab_cvmix]{config\_cvmix\_kpp\_use\_theory\_wave}}
\label{subsec:nm_sec_config_cvmix_kpp_use_theory_wave}
\begin{center}
\begin{longtable}{| p{2.0in} || p{4.0in} |}
    \hline
    Type: & logical \\
    \hline
    Units: & -- \\
    \hline
    Default Value: & .false. \\
    \hline
    Possible Values: & .true. or .false. \\
    \hline
    \caption{config\_cvmix\_kpp\_use\_theory\_wave: Flag for use of theory-wave in Li et al. (2017) to approximate the Langmuir number and enhancement factor}
\end{longtable}
\end{center}
\subsection[config\_cvmix\_kpp\_langmuir\_mixing\_opt]{\hyperref[sec:nm_tab_cvmix]{config\_cvmix\_kpp\_langmuir\_mixing\_opt}}
\label{subsec:nm_sec_config_cvmix_kpp_langmuir_mixing_opt}
\begin{center}
\begin{longtable}{| p{2.0in} || p{4.0in} |}
    \hline
    Type: & character \\
    \hline
    Units: & -- \\
    \hline
    Default Value: & NONE \\
    \hline
    Possible Values: & NONE, LWF16, RWHGK16 \\
    \hline
    \caption{config\_cvmix\_kpp\_langmuir\_mixing\_opt: Option of Langmuir enhanced mixing parameterization}
\end{longtable}
\end{center}
\subsection[config\_cvmix\_kpp\_langmuir\_entrainment\_opt]{\hyperref[sec:nm_tab_cvmix]{config\_cvmix\_kpp\_langmuir\_entrainment\_opt}}
\label{subsec:nm_sec_config_cvmix_kpp_langmuir_entrainment_opt}
\begin{center}
\begin{longtable}{| p{2.0in} || p{4.0in} |}
    \hline
    Type: & character \\
    \hline
    Units: & -- \\
    \hline
    Default Value: & NONE \\
    \hline
    Possible Values: & NONE, LWF16, LF17, RWHGK16 \\
    \hline
    \caption{config\_cvmix\_kpp\_langmuir\_entrainment\_opt: Option of Langmuir enhanced entrainment parameterization}
\end{longtable}
\end{center}
\subsection[config\_cvmix\_kpp\_use\_active\_wave]{\hyperref[sec:nm_tab_cvmix]{config\_cvmix\_kpp\_use\_active\_wave}}
\label{subsec:nm_sec_config_cvmix_kpp_use_active_wave}
\begin{center}
\begin{longtable}{| p{2.0in} || p{4.0in} |}
    \hline
    Type: & logical \\
    \hline
    Units: & -- \\
    \hline
    Default Value: & .false. \\
    \hline
    Possible Values: & .true. or .false. \\
    \hline
    \caption{config\_cvmix\_kpp\_use\_active\_wave: Flag for Langmuir enchancement factor using prognostic waves. Requires config\_use\_active\_wave = .true.}
\end{longtable}
\end{center}
\section[wave\_coupling]{\hyperref[sec:nm_tab_wave_coupling]{wave\_coupling}}
\label{sec:nm_sec_wave_coupling}
\subsection[config\_use\_active\_wave]{\hyperref[sec:nm_tab_wave_coupling]{config\_use\_active\_wave}}
\label{subsec:nm_sec_config_use_active_wave}
\begin{center}
\begin{longtable}{| p{2.0in} || p{4.0in} |}
    \hline
    Type: & logical \\
    \hline
    Units: & -- \\
    \hline
    Default Value: & .false. \\
    \hline
    Possible Values: & .true. or .false. \\
    \hline
    \caption{config\_use\_active\_wave: Flag for using prognostic waves. Controls the allocation of wave arrays and computation of Stokes drift profiles.}
\end{longtable}
\end{center}
\subsection[config\_n\_stokes\_drift\_wavenumber\_partitions]{\hyperref[sec:nm_tab_wave_coupling]{config\_n\_stokes\_drift\_wavenumber\_partitions}}
\label{subsec:nm_sec_config_n_stokes_drift_wavenumber_partitions}
\begin{center}
\begin{longtable}{| p{2.0in} || p{4.0in} |}
    \hline
    Type: & integer \\
    \hline
    Units: & -- \\
    \hline
    Default Value: & 6 \\
    \hline
    Possible Values: & 3,4,6 \\
    \hline
    \caption{config\_n\_stokes\_drift\_wavenumber\_partitions: Number of wavenumber partitions to be used in reconstructing wave-induced Stokes drift profile}
\end{longtable}
\end{center}
\section[gotm]{\hyperref[sec:nm_tab_gotm]{gotm}}
\label{sec:nm_sec_gotm}
\subsection[config\_use\_gotm]{\hyperref[sec:nm_tab_gotm]{config\_use\_gotm}}
\label{subsec:nm_sec_config_use_gotm}
\begin{center}
\begin{longtable}{| p{2.0in} || p{4.0in} |}
    \hline
    Type: & logical \\
    \hline
    Units: & -- \\
    \hline
    Default Value: & .false. \\
    \hline
    Possible Values: & True or False \\
    \hline
    \caption{config\_use\_gotm: If true, use the General Ocean Turbulence Model routines to compute vertical diffusivity and viscosity}
\end{longtable}
\end{center}
\subsection[config\_gotm\_namelist\_file]{\hyperref[sec:nm_tab_gotm]{config\_gotm\_namelist\_file}}
\label{subsec:nm_sec_config_gotm_namelist_file}
\begin{center}
\begin{longtable}{| p{2.0in} || p{4.0in} |}
    \hline
    Type: & character \\
    \hline
    Units: & -- \\
    \hline
    Default Value: & gotmturb.nml \\
    \hline
    Possible Values: & gotmturb.nml \\
    \hline
    \caption{config\_gotm\_namelist\_file: File name of GOTM turbulence namelist}
\end{longtable}
\end{center}
\subsection[config\_gotm\_constant\_surface\_roughness\_length]{\hyperref[sec:nm_tab_gotm]{config\_gotm\_constant\_surface\_roughness\_length}}
\label{subsec:nm_sec_config_gotm_constant_surface_roughness_length}
\begin{center}
\begin{longtable}{| p{2.0in} || p{4.0in} |}
    \hline
    Type: & real \\
    \hline
    Units: & \si{m} \\
    \hline
    Default Value: & 0.02 \\
    \hline
    Possible Values: & Any positive real number. \\
    \hline
    \caption{config\_gotm\_constant\_surface\_roughness\_length: The constant surface roughness length scale.}
\end{longtable}
\end{center}
\subsection[config\_gotm\_constant\_bottom\_roughness\_length]{\hyperref[sec:nm_tab_gotm]{config\_gotm\_constant\_bottom\_roughness\_length}}
\label{subsec:nm_sec_config_gotm_constant_bottom_roughness_length}
\begin{center}
\begin{longtable}{| p{2.0in} || p{4.0in} |}
    \hline
    Type: & real \\
    \hline
    Units: & \si{m} \\
    \hline
    Default Value: & 0.0015 \\
    \hline
    Possible Values: & Any positive real number. \\
    \hline
    \caption{config\_gotm\_constant\_bottom\_roughness\_length: The constant bottom roughness length scale.}
\end{longtable}
\end{center}
\subsection[config\_gotm\_constant\_bottom\_drag\_coeff]{\hyperref[sec:nm_tab_gotm]{config\_gotm\_constant\_bottom\_drag\_coeff}}
\label{subsec:nm_sec_config_gotm_constant_bottom_drag_coeff}
\begin{center}
\begin{longtable}{| p{2.0in} || p{4.0in} |}
    \hline
    Type: & real \\
    \hline
    Units: & -- \\
    \hline
    Default Value: & 1.e-3 \\
    \hline
    Possible Values: & Any positive real number. \\
    \hline
    \caption{config\_gotm\_constant\_bottom\_drag\_coeff: The constant bottom drag coefficient.}
\end{longtable}
\end{center}
\section[forcing]{\hyperref[sec:nm_tab_forcing]{forcing}}
\label{sec:nm_sec_forcing}
\subsection[config\_use\_variable\_drag]{\hyperref[sec:nm_tab_forcing]{config\_use\_variable\_drag}}
\label{subsec:nm_sec_config_use_variable_drag}
\begin{center}
\begin{longtable}{| p{2.0in} || p{4.0in} |}
    \hline
    Type: & logical \\
    \hline
    Units: & -- \\
    \hline
    Default Value: & .false. \\
    \hline
    Possible Values: & .true. or .false. \\
    \hline
    \caption{config\_use\_variable\_drag: Controls if variable drag is enabled.}
\end{longtable}
\end{center}
\subsection[config\_use\_bulk\_wind\_stress]{\hyperref[sec:nm_tab_forcing]{config\_use\_bulk\_wind\_stress}}
\label{subsec:nm_sec_config_use_bulk_wind_stress}
\begin{center}
\begin{longtable}{| p{2.0in} || p{4.0in} |}
    \hline
    Type: & logical \\
    \hline
    Units: & -- \\
    \hline
    Default Value: & .false. \\
    \hline
    Possible Values: & .true. or .false. \\
    \hline
    \caption{config\_use\_bulk\_wind\_stress: Controls if zonal and meridional components of windstress are used to build surface wind stress.}
\end{longtable}
\end{center}
\subsection[config\_use\_bulk\_thickness\_flux]{\hyperref[sec:nm_tab_forcing]{config\_use\_bulk\_thickness\_flux}}
\label{subsec:nm_sec_config_use_bulk_thickness_flux}
\begin{center}
\begin{longtable}{| p{2.0in} || p{4.0in} |}
    \hline
    Type: & logical \\
    \hline
    Units: & -- \\
    \hline
    Default Value: & .false. \\
    \hline
    Possible Values: & .true. or .false. \\
    \hline
    \caption{config\_use\_bulk\_thickness\_flux: Controls if a bulk thickness flux will be computed for surface forcing.}
\end{longtable}
\end{center}
\subsection[config\_flux\_attenuation\_coefficient]{\hyperref[sec:nm_tab_forcing]{config\_flux\_attenuation\_coefficient}}
\label{subsec:nm_sec_config_flux_attenuation_coefficient}
\begin{center}
\begin{longtable}{| p{2.0in} || p{4.0in} |}
    \hline
    Type: & real \\
    \hline
    Units: & \si{m} \\
    \hline
    Default Value: & 0.001 \\
    \hline
    Possible Values: & Any positive real number. \\
    \hline
    \caption{config\_flux\_attenuation\_coefficient: The length scale of exponential decay of surface fluxes. Fluxes are multiplied by $e^{z/\gamma}$, where this coefficient is $\gamma$.}
\end{longtable}
\end{center}
\subsection[config\_flux\_attenuation\_coefficient\_runoff]{\hyperref[sec:nm_tab_forcing]{config\_flux\_attenuation\_coefficient\_runoff}}
\label{subsec:nm_sec_config_flux_attenuation_coefficient_runoff}
\begin{center}
\begin{longtable}{| p{2.0in} || p{4.0in} |}
    \hline
    Type: & real \\
    \hline
    Units: & \si{m} \\
    \hline
    Default Value: & 0.001 \\
    \hline
    Possible Values: & Any positive real number. \\
    \hline
    \caption{config\_flux\_attenuation\_coefficient\_runoff: The length scale of exponential decay of river runoff. Fluxes are multiplied by $e^{z/\gamma}$, where this coefficient is $\gamma$.}
\end{longtable}
\end{center}
\section[time\_varying\_forcing]{\hyperref[sec:nm_tab_time_varying_forcing]{time\_varying\_forcing}}
\label{sec:nm_sec_time_varying_forcing}
\subsection[config\_use\_time\_varying\_atmospheric\_forcing]{\hyperref[sec:nm_tab_time_varying_forcing]{config\_use\_time\_varying\_atmospheric\_forcing}}
\label{subsec:nm_sec_config_use_time_varying_atmospheric_forcing}
\begin{center}
\begin{longtable}{| p{2.0in} || p{4.0in} |}
    \hline
    Type: & logical \\
    \hline
    Units: & -- \\
    \hline
    Default Value: & .false. \\
    \hline
    Possible Values: & .true. or .false. \\
    \hline
    \caption{config\_use\_time\_varying\_atmospheric\_forcing: If true calculate input forcing fields.}
\end{longtable}
\end{center}
\subsection[config\_time\_varying\_atmospheric\_forcing\_type]{\hyperref[sec:nm_tab_time_varying_forcing]{config\_time\_varying\_atmospheric\_forcing\_type}}
\label{subsec:nm_sec_config_time_varying_atmospheric_forcing_type}
\begin{center}
\begin{longtable}{| p{2.0in} || p{4.0in} |}
    \hline
    Type: & character \\
    \hline
    Units: & -- \\
    \hline
    Default Value: & WINDPRES \\
    \hline
    Possible Values: & 'WINDPRES' \\
    \hline
    \caption{config\_time\_varying\_atmospheric\_forcing\_type: Atmospheric forcing type.}
\end{longtable}
\end{center}
\subsection[config\_time\_varying\_atmospheric\_forcing\_start\_time]{\hyperref[sec:nm_tab_time_varying_forcing]{config\_time\_varying\_atmospheric\_forcing\_start\_time}}
\label{subsec:nm_sec_config_time_varying_atmospheric_forcing_start_time}
\begin{center}
\begin{longtable}{| p{2.0in} || p{4.0in} |}
    \hline
    Type: & character \\
    \hline
    Units: & -- \\
    \hline
    Default Value: & 0001-01-01\_00:00:00 \\
    \hline
    Possible Values: & 'YYYY-MM-DD\_HH:MM:SS \\
    \hline
    \caption{config\_time\_varying\_atmospheric\_forcing\_start\_time: Forcing time to use at the simulation start time}
\end{longtable}
\end{center}
\subsection[config\_time\_varying\_atmospheric\_forcing\_reference\_time]{\hyperref[sec:nm_tab_time_varying_forcing]{config\_time\_varying\_atmospheric\_forcing\_reference\_time}}
\label{subsec:nm_sec_config_time_varying_atmospheric_forcing_reference_time}
\begin{center}
\begin{longtable}{| p{2.0in} || p{4.0in} |}
    \hline
    Type: & character \\
    \hline
    Units: & -- \\
    \hline
    Default Value: & 0001-01-01\_00:00:00 \\
    \hline
    Possible Values: & 'YYYY-MM-DD\_HH:MM:SS \\
    \hline
    \caption{config\_time\_varying\_atmospheric\_forcing\_reference\_time: Reference time for the forcing}
\end{longtable}
\end{center}
\subsection[config\_time\_varying\_atmospheric\_forcing\_cycle\_start]{\hyperref[sec:nm_tab_time_varying_forcing]{config\_time\_varying\_atmospheric\_forcing\_cycle\_start}}
\label{subsec:nm_sec_config_time_varying_atmospheric_forcing_cycle_start}
\begin{center}
\begin{longtable}{| p{2.0in} || p{4.0in} |}
    \hline
    Type: & character \\
    \hline
    Units: & -- \\
    \hline
    Default Value: & 0001-01-01\_00:00:00 \\
    \hline
    Possible Values: & 'YYYY-MM-DD\_HH:MM:SS \\
    \hline
    \caption{config\_time\_varying\_atmospheric\_forcing\_cycle\_start: Start time for the forcing cycle.}
\end{longtable}
\end{center}
\subsection[config\_time\_varying\_atmospheric\_forcing\_cycle\_duration]{\hyperref[sec:nm_tab_time_varying_forcing]{config\_time\_varying\_atmospheric\_forcing\_cycle\_duration}}
\label{subsec:nm_sec_config_time_varying_atmospheric_forcing_cycle_duration}
\begin{center}
\begin{longtable}{| p{2.0in} || p{4.0in} |}
    \hline
    Type: & character \\
    \hline
    Units: & -- \\
    \hline
    Default Value: & 2-00-00\_00:00:00 \\
    \hline
    Possible Values: & 'YYYY-MM-DD\_HH:MM:SS \\
    \hline
    \caption{config\_time\_varying\_atmospheric\_forcing\_cycle\_duration: Duration of the forcing cycle.}
\end{longtable}
\end{center}
\subsection[config\_time\_varying\_atmospheric\_forcing\_interval]{\hyperref[sec:nm_tab_time_varying_forcing]{config\_time\_varying\_atmospheric\_forcing\_interval}}
\label{subsec:nm_sec_config_time_varying_atmospheric_forcing_interval}
\begin{center}
\begin{longtable}{| p{2.0in} || p{4.0in} |}
    \hline
    Type: & character \\
    \hline
    Units: & -- \\
    \hline
    Default Value: & 01:00:00 \\
    \hline
    Possible Values: & 'YYYY-MM-DD\_HH:MM:SS \\
    \hline
    \caption{config\_time\_varying\_atmospheric\_forcing\_interval: Time between forcing inputs}
\end{longtable}
\end{center}
\subsection[config\_time\_varying\_atmospheric\_forcing\_ramp]{\hyperref[sec:nm_tab_time_varying_forcing]{config\_time\_varying\_atmospheric\_forcing\_ramp}}
\label{subsec:nm_sec_config_time_varying_atmospheric_forcing_ramp}
\begin{center}
\begin{longtable}{| p{2.0in} || p{4.0in} |}
    \hline
    Type: & real \\
    \hline
    Units: & \si{days} \\
    \hline
    Default Value: & 10.0 \\
    \hline
    Possible Values: & Any positive real number \\
    \hline
    \caption{config\_time\_varying\_atmospheric\_forcing\_ramp: Number of days to ramp up time varying forcing}
\end{longtable}
\end{center}
\subsection[config\_time\_varying\_atmospheric\_forcing\_ramp\_delay]{\hyperref[sec:nm_tab_time_varying_forcing]{config\_time\_varying\_atmospheric\_forcing\_ramp\_delay}}
\label{subsec:nm_sec_config_time_varying_atmospheric_forcing_ramp_delay}
\begin{center}
\begin{longtable}{| p{2.0in} || p{4.0in} |}
    \hline
    Type: & real \\
    \hline
    Units: & \si{days} \\
    \hline
    Default Value: & 0.0 \\
    \hline
    Possible Values: & Any positive real number \\
    \hline
    \caption{config\_time\_varying\_atmospheric\_forcing\_ramp\_delay: Number of days to delay ramp time varying forcing}
\end{longtable}
\end{center}
\subsection[config\_use\_time\_varying\_land\_ice\_forcing]{\hyperref[sec:nm_tab_time_varying_forcing]{config\_use\_time\_varying\_land\_ice\_forcing}}
\label{subsec:nm_sec_config_use_time_varying_land_ice_forcing}
\begin{center}
\begin{longtable}{| p{2.0in} || p{4.0in} |}
    \hline
    Type: & logical \\
    \hline
    Units: & -- \\
    \hline
    Default Value: & .false. \\
    \hline
    Possible Values: & .true. or .false. \\
    \hline
    \caption{config\_use\_time\_varying\_land\_ice\_forcing: If true calculate input forcing fields.}
\end{longtable}
\end{center}
\subsection[config\_time\_varying\_land\_ice\_forcing\_start\_time]{\hyperref[sec:nm_tab_time_varying_forcing]{config\_time\_varying\_land\_ice\_forcing\_start\_time}}
\label{subsec:nm_sec_config_time_varying_land_ice_forcing_start_time}
\begin{center}
\begin{longtable}{| p{2.0in} || p{4.0in} |}
    \hline
    Type: & character \\
    \hline
    Units: & -- \\
    \hline
    Default Value: & 0001-01-01\_00:00:00 \\
    \hline
    Possible Values: & 'YYYY-MM-DD\_HH:MM:SS \\
    \hline
    \caption{config\_time\_varying\_land\_ice\_forcing\_start\_time: Forcing time to use at the simulation start time}
\end{longtable}
\end{center}
\subsection[config\_time\_varying\_land\_ice\_forcing\_reference\_time]{\hyperref[sec:nm_tab_time_varying_forcing]{config\_time\_varying\_land\_ice\_forcing\_reference\_time}}
\label{subsec:nm_sec_config_time_varying_land_ice_forcing_reference_time}
\begin{center}
\begin{longtable}{| p{2.0in} || p{4.0in} |}
    \hline
    Type: & character \\
    \hline
    Units: & -- \\
    \hline
    Default Value: & 0001-01-01\_00:00:00 \\
    \hline
    Possible Values: & 'YYYY-MM-DD\_HH:MM:SS \\
    \hline
    \caption{config\_time\_varying\_land\_ice\_forcing\_reference\_time: Reference time for the forcing}
\end{longtable}
\end{center}
\subsection[config\_time\_varying\_land\_ice\_forcing\_cycle\_start]{\hyperref[sec:nm_tab_time_varying_forcing]{config\_time\_varying\_land\_ice\_forcing\_cycle\_start}}
\label{subsec:nm_sec_config_time_varying_land_ice_forcing_cycle_start}
\begin{center}
\begin{longtable}{| p{2.0in} || p{4.0in} |}
    \hline
    Type: & character \\
    \hline
    Units: & -- \\
    \hline
    Default Value: & 0001-01-01\_00:00:00 \\
    \hline
    Possible Values: & 'YYYY-MM-DD\_HH:MM:SS \\
    \hline
    \caption{config\_time\_varying\_land\_ice\_forcing\_cycle\_start: Start time for the forcing cycle.}
\end{longtable}
\end{center}
\subsection[config\_time\_varying\_land\_ice\_forcing\_cycle\_duration]{\hyperref[sec:nm_tab_time_varying_forcing]{config\_time\_varying\_land\_ice\_forcing\_cycle\_duration}}
\label{subsec:nm_sec_config_time_varying_land_ice_forcing_cycle_duration}
\begin{center}
\begin{longtable}{| p{2.0in} || p{4.0in} |}
    \hline
    Type: & character \\
    \hline
    Units: & -- \\
    \hline
    Default Value: & 2-00-00\_00:00:00 \\
    \hline
    Possible Values: & 'YYYY-MM-DD\_HH:MM:SS \\
    \hline
    \caption{config\_time\_varying\_land\_ice\_forcing\_cycle\_duration: Duration of the forcing cycle.}
\end{longtable}
\end{center}
\subsection[config\_time\_varying\_land\_ice\_forcing\_interval]{\hyperref[sec:nm_tab_time_varying_forcing]{config\_time\_varying\_land\_ice\_forcing\_interval}}
\label{subsec:nm_sec_config_time_varying_land_ice_forcing_interval}
\begin{center}
\begin{longtable}{| p{2.0in} || p{4.0in} |}
    \hline
    Type: & character \\
    \hline
    Units: & -- \\
    \hline
    Default Value: & 01:00:00 \\
    \hline
    Possible Values: & 'YYYY-MM-DD\_HH:MM:SS \\
    \hline
    \caption{config\_time\_varying\_land\_ice\_forcing\_interval: Time between forcing inputs}
\end{longtable}
\end{center}
\section[coupling]{\hyperref[sec:nm_tab_coupling]{coupling}}
\label{sec:nm_sec_coupling}
\subsection[config\_ssh\_grad\_relax\_timescale]{\hyperref[sec:nm_tab_coupling]{config\_ssh\_grad\_relax\_timescale}}
\label{subsec:nm_sec_config_ssh_grad_relax_timescale}
\begin{center}
\begin{longtable}{| p{2.0in} || p{4.0in} |}
    \hline
    Type: & real \\
    \hline
    Units: & \si{seconds} \\
    \hline
    Default Value: & 0.0 \\
    \hline
    Possible Values: & Any positive real number. \\
    \hline
    \caption{config\_ssh\_grad\_relax\_timescale: Timescale for relaxation of the ssh gradient for coupling. A value of 0.0 (default) removes any relaxation and gives instantaneous response.}
\end{longtable}
\end{center}
\subsection[config\_remove\_AIS\_coupler\_runoff]{\hyperref[sec:nm_tab_coupling]{config\_remove\_AIS\_coupler\_runoff}}
\label{subsec:nm_sec_config_remove_AIS_coupler_runoff}
\begin{center}
\begin{longtable}{| p{2.0in} || p{4.0in} |}
    \hline
    Type: & logical \\
    \hline
    Units: & -- \\
    \hline
    Default Value: & .false. \\
    \hline
    Possible Values: & .true. or .false. \\
    \hline
    \caption{config\_remove\_AIS\_coupler\_runoff: If true, solid and liquid runoff from the Antarctic Ice Sheet (below 60S latitude) coming from the coupled is zeroed in the coupler import routines.  To be used with data iceberg fluxes coming from the sea ice model.}
\end{longtable}
\end{center}
\section[shortwaveRadiation]{\hyperref[sec:nm_tab_shortwaveRadiation]{shortwaveRadiation}}
\label{sec:nm_sec_shortwaveRadiation}
\subsection[config\_sw\_absorption\_type]{\hyperref[sec:nm_tab_shortwaveRadiation]{config\_sw\_absorption\_type}}
\label{subsec:nm_sec_config_sw_absorption_type}
\begin{center}
\begin{longtable}{| p{2.0in} || p{4.0in} |}
    \hline
    Type: & character \\
    \hline
    Units: & -- \\
    \hline
    Default Value: & none \\
    \hline
    Possible Values: & 'jerlov' or 'ohlmann00' or 'none' \\
    \hline
    \caption{config\_sw\_absorption\_type: Name of shortwave absorption type used in simulation. }
\end{longtable}
\end{center}
\subsection[config\_jerlov\_water\_type]{\hyperref[sec:nm_tab_shortwaveRadiation]{config\_jerlov\_water\_type}}
\label{subsec:nm_sec_config_jerlov_water_type}
\begin{center}
\begin{longtable}{| p{2.0in} || p{4.0in} |}
    \hline
    Type: & integer \\
    \hline
    Units: & -- \\
    \hline
    Default Value: & 3 \\
    \hline
    Possible Values: & Integer values between 1 and 5 \\
    \hline
    \caption{config\_jerlov\_water\_type: Integer value defining the water type used in Jerlov short wave absorption.}
\end{longtable}
\end{center}
\subsection[config\_surface\_buoyancy\_depth]{\hyperref[sec:nm_tab_shortwaveRadiation]{config\_surface\_buoyancy\_depth}}
\label{subsec:nm_sec_config_surface_buoyancy_depth}
\begin{center}
\begin{longtable}{| p{2.0in} || p{4.0in} |}
    \hline
    Type: & real \\
    \hline
    Units: & \si{m} \\
    \hline
    Default Value: & 1 \\
    \hline
    Possible Values: & Real Values greater than zero less than bottomDepth \\
    \hline
    \caption{config\_surface\_buoyancy\_depth: Depth over which to apply penetrating SW to sfcBuoyancyFlux}
\end{longtable}
\end{center}
\subsection[config\_enable\_shortwave\_energy\_fixer]{\hyperref[sec:nm_tab_shortwaveRadiation]{config\_enable\_shortwave\_energy\_fixer}}
\label{subsec:nm_sec_config_enable_shortwave_energy_fixer}
\begin{center}
\begin{longtable}{| p{2.0in} || p{4.0in} |}
    \hline
    Type: & logical \\
    \hline
    Units: & -- \\
    \hline
    Default Value: & .false. \\
    \hline
    Possible Values: & .true. or .false. \\
    \hline
    \caption{config\_enable\_shortwave\_energy\_fixer: Flag to enable the shortwave energy fixer for shallow ocean cells}
\end{longtable}
\end{center}
\section[tidal\_forcing]{\hyperref[sec:nm_tab_tidal_forcing]{tidal\_forcing}}
\label{sec:nm_sec_tidal_forcing}
\subsection[config\_use\_tidal\_forcing]{\hyperref[sec:nm_tab_tidal_forcing]{config\_use\_tidal\_forcing}}
\label{subsec:nm_sec_config_use_tidal_forcing}
\begin{center}
\begin{longtable}{| p{2.0in} || p{4.0in} |}
    \hline
    Type: & logical \\
    \hline
    Units: & -- \\
    \hline
    Default Value: & .false. \\
    \hline
    Possible Values: & .true. or .false. \\
    \hline
    \caption{config\_use\_tidal\_forcing: Controls if tidal forcing is used.}
\end{longtable}
\end{center}
\subsection[config\_use\_tidal\_forcing\_tau]{\hyperref[sec:nm_tab_tidal_forcing]{config\_use\_tidal\_forcing\_tau}}
\label{subsec:nm_sec_config_use_tidal_forcing_tau}
\begin{center}
\begin{longtable}{| p{2.0in} || p{4.0in} |}
    \hline
    Type: & real \\
    \hline
    Units: & \si{s} \\
    \hline
    Default Value: & 10000 \\
    \hline
    Possible Values: & Real non-zero value. \\
    \hline
    \caption{config\_use\_tidal\_forcing\_tau: Controls time scale for relaxation of tidal forcing.}
\end{longtable}
\end{center}
\subsection[config\_tidal\_forcing\_type]{\hyperref[sec:nm_tab_tidal_forcing]{config\_tidal\_forcing\_type}}
\label{subsec:nm_sec_config_tidal_forcing_type}
\begin{center}
\begin{longtable}{| p{2.0in} || p{4.0in} |}
    \hline
    Type: & character \\
    \hline
    Units: & -- \\
    \hline
    Default Value: & off \\
    \hline
    Possible Values: & 'thickness\_source','direct' \\
    \hline
    \caption{config\_tidal\_forcing\_type: Selects the way tidal forcing is applied.}
\end{longtable}
\end{center}
\subsection[config\_tidal\_forcing\_model]{\hyperref[sec:nm_tab_tidal_forcing]{config\_tidal\_forcing\_model}}
\label{subsec:nm_sec_config_tidal_forcing_model}
\begin{center}
\begin{longtable}{| p{2.0in} || p{4.0in} |}
    \hline
    Type: & character \\
    \hline
    Units: & -- \\
    \hline
    Default Value: & off \\
    \hline
    Possible Values: & 'off','monochromatic' \\
    \hline
    \caption{config\_tidal\_forcing\_model: Selects the mode in which tidal forcing is computed.}
\end{longtable}
\end{center}
\subsection[config\_tidal\_forcing\_monochromatic\_amp]{\hyperref[sec:nm_tab_tidal_forcing]{config\_tidal\_forcing\_monochromatic\_amp}}
\label{subsec:nm_sec_config_tidal_forcing_monochromatic_amp}
\begin{center}
\begin{longtable}{| p{2.0in} || p{4.0in} |}
    \hline
    Type: & real \\
    \hline
    Units: & \si{m} \\
    \hline
    Default Value: & 2.0 \\
    \hline
    Possible Values: & Any positive real number. \\
    \hline
    \caption{config\_tidal\_forcing\_monochromatic\_amp: Value of amplitude of monochromatic tide.}
\end{longtable}
\end{center}
\subsection[config\_tidal\_forcing\_monochromatic\_period]{\hyperref[sec:nm_tab_tidal_forcing]{config\_tidal\_forcing\_monochromatic\_period}}
\label{subsec:nm_sec_config_tidal_forcing_monochromatic_period}
\begin{center}
\begin{longtable}{| p{2.0in} || p{4.0in} |}
    \hline
    Type: & real \\
    \hline
    Units: & \si{days} \\
    \hline
    Default Value: & 0.5 \\
    \hline
    Possible Values: & Any positive real number. \\
    \hline
    \caption{config\_tidal\_forcing\_monochromatic\_period: Value of period of monochromatic tide.}
\end{longtable}
\end{center}
\subsection[config\_tidal\_forcing\_monochromatic\_phaseLag]{\hyperref[sec:nm_tab_tidal_forcing]{config\_tidal\_forcing\_monochromatic\_phaseLag}}
\label{subsec:nm_sec_config_tidal_forcing_monochromatic_phaseLag}
\begin{center}
\begin{longtable}{| p{2.0in} || p{4.0in} |}
    \hline
    Type: & real \\
    \hline
    Units: & -- \\
    \hline
    Default Value: & 0.0 \\
    \hline
    Possible Values: & Any real number between. \\
    \hline
    \caption{config\_tidal\_forcing\_monochromatic\_phaseLag: Value of phase of monochromatic tide.}
\end{longtable}
\end{center}
\subsection[config\_tidal\_forcing\_monochromatic\_baseline]{\hyperref[sec:nm_tab_tidal_forcing]{config\_tidal\_forcing\_monochromatic\_baseline}}
\label{subsec:nm_sec_config_tidal_forcing_monochromatic_baseline}
\begin{center}
\begin{longtable}{| p{2.0in} || p{4.0in} |}
    \hline
    Type: & real \\
    \hline
    Units: & \si{days} \\
    \hline
    Default Value: & 0.0 \\
    \hline
    Possible Values: & Any positive real number. \\
    \hline
    \caption{config\_tidal\_forcing\_monochromatic\_baseline: Value of baseline monochromatic tide, e.g., sea level rise.}
\end{longtable}
\end{center}
\section[self\_attraction\_loading]{\hyperref[sec:nm_tab_self_attraction_loading]{self\_attraction\_loading}}
\label{sec:nm_sec_self_attraction_loading}
\subsection[config\_use\_self\_attraction\_loading]{\hyperref[sec:nm_tab_self_attraction_loading]{config\_use\_self\_attraction\_loading}}
\label{subsec:nm_sec_config_use_self_attraction_loading}
\begin{center}
\begin{longtable}{| p{2.0in} || p{4.0in} |}
    \hline
    Type: & logical \\
    \hline
    Units: & -- \\
    \hline
    Default Value: & .false. \\
    \hline
    Possible Values: & .true. or .false. \\
    \hline
    \caption{config\_use\_self\_attraction\_loading: Controls if self-attraction and loading is applied to ssh}
\end{longtable}
\end{center}
\subsection[config\_self\_attraction\_loading\_smoothing\_width]{\hyperref[sec:nm_tab_self_attraction_loading]{config\_self\_attraction\_loading\_smoothing\_width}}
\label{subsec:nm_sec_config_self_attraction_loading_smoothing_width}
\begin{center}
\begin{longtable}{| p{2.0in} || p{4.0in} |}
    \hline
    Type: & real \\
    \hline
    Units: & \si{km} \\
    \hline
    Default Value: & 1.0 \\
    \hline
    Possible Values: & Any positive real number. \\
    \hline
    \caption{config\_self\_attraction\_loading\_smoothing\_width: Defines region over which ssh is smoothed to zero at coasts for SAL calculation.}
\end{longtable}
\end{center}
\subsection[config\_mpas\_to\_grid\_weights\_file]{\hyperref[sec:nm_tab_self_attraction_loading]{config\_mpas\_to\_grid\_weights\_file}}
\label{subsec:nm_sec_config_mpas_to_grid_weights_file}
\begin{center}
\begin{longtable}{| p{2.0in} || p{4.0in} |}
    \hline
    Type: & character \\
    \hline
    Units: & -- \\
    \hline
    Default Value: & mpas\_to\_grid.nc \\
    \hline
    Possible Values: & Any file name string \\
    \hline
    \caption{config\_mpas\_to\_grid\_weights\_file: Location of the file containing the interpolation weights for transformation from the MPAS mesh to a Gaussian Grid.}
\end{longtable}
\end{center}
\subsection[config\_grid\_to\_mpas\_weights\_file]{\hyperref[sec:nm_tab_self_attraction_loading]{config\_grid\_to\_mpas\_weights\_file}}
\label{subsec:nm_sec_config_grid_to_mpas_weights_file}
\begin{center}
\begin{longtable}{| p{2.0in} || p{4.0in} |}
    \hline
    Type: & character \\
    \hline
    Units: & -- \\
    \hline
    Default Value: & grid\_to\_mpas.nc \\
    \hline
    Possible Values: & Any file name string \\
    \hline
    \caption{config\_grid\_to\_mpas\_weights\_file: Location of the file containing the interpolation weights for transformation from a Gaussian Grid to the MPAS mesh.}
\end{longtable}
\end{center}
\subsection[config\_self\_attraction\_loading\_compute\_interval]{\hyperref[sec:nm_tab_self_attraction_loading]{config\_self\_attraction\_loading\_compute\_interval}}
\label{subsec:nm_sec_config_self_attraction_loading_compute_interval}
\begin{center}
\begin{longtable}{| p{2.0in} || p{4.0in} |}
    \hline
    Type: & character \\
    \hline
    Units: & -- \\
    \hline
    Default Value: & 0000-00-00\_00:30:00 \\
    \hline
    Possible Values: & 'YYYY-MM-DD\_HH:MM:SS' \\
    \hline
    \caption{config\_self\_attraction\_loading\_compute\_interval: Interval for computing full SAL.}
\end{longtable}
\end{center}
\subsection[config\_nLatitude]{\hyperref[sec:nm_tab_self_attraction_loading]{config\_nLatitude}}
\label{subsec:nm_sec_config_nLatitude}
\begin{center}
\begin{longtable}{| p{2.0in} || p{4.0in} |}
    \hline
    Type: & integer \\
    \hline
    Units: & -- \\
    \hline
    Default Value: & 128 \\
    \hline
    Possible Values: & Any positive integer value. \\
    \hline
    \caption{config\_nLatitude: Numer of latitude points in the Gaussian Grid.}
\end{longtable}
\end{center}
\subsection[config\_nLongitude]{\hyperref[sec:nm_tab_self_attraction_loading]{config\_nLongitude}}
\label{subsec:nm_sec_config_nLongitude}
\begin{center}
\begin{longtable}{| p{2.0in} || p{4.0in} |}
    \hline
    Type: & integer \\
    \hline
    Units: & -- \\
    \hline
    Default Value: & 256 \\
    \hline
    Possible Values: & Any positive integer value. \\
    \hline
    \caption{config\_nLongitude: Numer of longitude points in the Gaussian Grid.}
\end{longtable}
\end{center}
\subsection[config\_use\_parallel\_self\_attraction\_loading]{\hyperref[sec:nm_tab_self_attraction_loading]{config\_use\_parallel\_self\_attraction\_loading}}
\label{subsec:nm_sec_config_use_parallel_self_attraction_loading}
\begin{center}
\begin{longtable}{| p{2.0in} || p{4.0in} |}
    \hline
    Type: & logical \\
    \hline
    Units: & -- \\
    \hline
    Default Value: & .false. \\
    \hline
    Possible Values: & .true. or .false. \\
    \hline
    \caption{config\_use\_parallel\_self\_attraction\_loading: Controls if self-attraction and loading is computed with parallel or serial algorithm}
\end{longtable}
\end{center}
\subsection[config\_parallel\_self\_attraction\_loading\_order]{\hyperref[sec:nm_tab_self_attraction_loading]{config\_parallel\_self\_attraction\_loading\_order}}
\label{subsec:nm_sec_config_parallel_self_attraction_loading_order}
\begin{center}
\begin{longtable}{| p{2.0in} || p{4.0in} |}
    \hline
    Type: & integer \\
    \hline
    Units: & -- \\
    \hline
    Default Value: & 10 \\
    \hline
    Possible Values: & Any positive integer value. \\
    \hline
    \caption{config\_parallel\_self\_attraction\_loading\_order: Controls the order of the sperical harmonic expansion used in the parallel self attraction and loading algorithm}
\end{longtable}
\end{center}
\subsection[config\_parallel\_self\_attraction\_loading\_n\_cells\_per\_block]{\hyperref[sec:nm_tab_self_attraction_loading]{config\_parallel\_self\_attraction\_loading\_n\_cells\_per\_block}}
\label{subsec:nm_sec_config_parallel_self_attraction_loading_n_cells_per_block}
\begin{center}
\begin{longtable}{| p{2.0in} || p{4.0in} |}
    \hline
    Type: & integer \\
    \hline
    Units: & -- \\
    \hline
    Default Value: & 600 \\
    \hline
    Possible Values: & Any positive integer value. \\
    \hline
    \caption{config\_parallel\_self\_attraction\_loading\_n\_cells\_per\_block: Controls the number of blocks used for spherical harmonics calculation}
\end{longtable}
\end{center}
\subsection[config\_parallel\_self\_attraction\_loading\_bfb]{\hyperref[sec:nm_tab_self_attraction_loading]{config\_parallel\_self\_attraction\_loading\_bfb}}
\label{subsec:nm_sec_config_parallel_self_attraction_loading_bfb}
\begin{center}
\begin{longtable}{| p{2.0in} || p{4.0in} |}
    \hline
    Type: & logical \\
    \hline
    Units: & -- \\
    \hline
    Default Value: & .false. \\
    \hline
    Possible Values: & .true. or .false. \\
    \hline
    \caption{config\_parallel\_self\_attraction\_loading\_bfb: Controls whether a reproducible sum is used for the parallel spherical harmonics calculations}
\end{longtable}
\end{center}
\section[tidal\_potential\_forcing]{\hyperref[sec:nm_tab_tidal_potential_forcing]{tidal\_potential\_forcing}}
\label{sec:nm_sec_tidal_potential_forcing}
\subsection[config\_use\_tidal\_potential\_forcing]{\hyperref[sec:nm_tab_tidal_potential_forcing]{config\_use\_tidal\_potential\_forcing}}
\label{subsec:nm_sec_config_use_tidal_potential_forcing}
\begin{center}
\begin{longtable}{| p{2.0in} || p{4.0in} |}
    \hline
    Type: & logical \\
    \hline
    Units: & -- \\
    \hline
    Default Value: & .false. \\
    \hline
    Possible Values: & .true. or .false. \\
    \hline
    \caption{config\_use\_tidal\_potential\_forcing: Controls if tidal potential forcing is used.}
\end{longtable}
\end{center}
\subsection[config\_tidal\_potential\_reference\_time]{\hyperref[sec:nm_tab_tidal_potential_forcing]{config\_tidal\_potential\_reference\_time}}
\label{subsec:nm_sec_config_tidal_potential_reference_time}
\begin{center}
\begin{longtable}{| p{2.0in} || p{4.0in} |}
    \hline
    Type: & character \\
    \hline
    Units: & -- \\
    \hline
    Default Value: & 0001-01-01\_00:00:00 \\
    \hline
    Possible Values: & 'YYYY-MM-DD\_HH:MM:SS' \\
    \hline
    \caption{config\_tidal\_potential\_reference\_time: Timestamp describing the time used to initialize nodal factors.}
\end{longtable}
\end{center}
\subsection[config\_use\_tidal\_potential\_forcing\_M2]{\hyperref[sec:nm_tab_tidal_potential_forcing]{config\_use\_tidal\_potential\_forcing\_M2}}
\label{subsec:nm_sec_config_use_tidal_potential_forcing_M2}
\begin{center}
\begin{longtable}{| p{2.0in} || p{4.0in} |}
    \hline
    Type: & logical \\
    \hline
    Units: & -- \\
    \hline
    Default Value: & .true. \\
    \hline
    Possible Values: & .true. or .false. \\
    \hline
    \caption{config\_use\_tidal\_potential\_forcing\_M2: Controls if tidal potential forcing for the M2 constituent is used.}
\end{longtable}
\end{center}
\subsection[config\_use\_tidal\_potential\_forcing\_S2]{\hyperref[sec:nm_tab_tidal_potential_forcing]{config\_use\_tidal\_potential\_forcing\_S2}}
\label{subsec:nm_sec_config_use_tidal_potential_forcing_S2}
\begin{center}
\begin{longtable}{| p{2.0in} || p{4.0in} |}
    \hline
    Type: & logical \\
    \hline
    Units: & -- \\
    \hline
    Default Value: & .true. \\
    \hline
    Possible Values: & .true. or .false. \\
    \hline
    \caption{config\_use\_tidal\_potential\_forcing\_S2: Controls if tidal potential forcing for the S2 constituent is used.}
\end{longtable}
\end{center}
\subsection[config\_use\_tidal\_potential\_forcing\_N2]{\hyperref[sec:nm_tab_tidal_potential_forcing]{config\_use\_tidal\_potential\_forcing\_N2}}
\label{subsec:nm_sec_config_use_tidal_potential_forcing_N2}
\begin{center}
\begin{longtable}{| p{2.0in} || p{4.0in} |}
    \hline
    Type: & logical \\
    \hline
    Units: & -- \\
    \hline
    Default Value: & .true. \\
    \hline
    Possible Values: & .true. or .false. \\
    \hline
    \caption{config\_use\_tidal\_potential\_forcing\_N2: Controls if tidal potential forcing for the N2 constituent is used.}
\end{longtable}
\end{center}
\subsection[config\_use\_tidal\_potential\_forcing\_K2]{\hyperref[sec:nm_tab_tidal_potential_forcing]{config\_use\_tidal\_potential\_forcing\_K2}}
\label{subsec:nm_sec_config_use_tidal_potential_forcing_K2}
\begin{center}
\begin{longtable}{| p{2.0in} || p{4.0in} |}
    \hline
    Type: & logical \\
    \hline
    Units: & -- \\
    \hline
    Default Value: & .true. \\
    \hline
    Possible Values: & .true. or .false. \\
    \hline
    \caption{config\_use\_tidal\_potential\_forcing\_K2: Controls if tidal potential forcing for the K2 constituent is used.}
\end{longtable}
\end{center}
\subsection[config\_use\_tidal\_potential\_forcing\_K1]{\hyperref[sec:nm_tab_tidal_potential_forcing]{config\_use\_tidal\_potential\_forcing\_K1}}
\label{subsec:nm_sec_config_use_tidal_potential_forcing_K1}
\begin{center}
\begin{longtable}{| p{2.0in} || p{4.0in} |}
    \hline
    Type: & logical \\
    \hline
    Units: & -- \\
    \hline
    Default Value: & .true. \\
    \hline
    Possible Values: & .true. or .false. \\
    \hline
    \caption{config\_use\_tidal\_potential\_forcing\_K1: Controls if tidal potential forcing for the K1 constituent is used.}
\end{longtable}
\end{center}
\subsection[config\_use\_tidal\_potential\_forcing\_O1]{\hyperref[sec:nm_tab_tidal_potential_forcing]{config\_use\_tidal\_potential\_forcing\_O1}}
\label{subsec:nm_sec_config_use_tidal_potential_forcing_O1}
\begin{center}
\begin{longtable}{| p{2.0in} || p{4.0in} |}
    \hline
    Type: & logical \\
    \hline
    Units: & -- \\
    \hline
    Default Value: & .true. \\
    \hline
    Possible Values: & .true. or .false. \\
    \hline
    \caption{config\_use\_tidal\_potential\_forcing\_O1: Controls if tidal potential forcing for the O1 constituent is used.}
\end{longtable}
\end{center}
\subsection[config\_use\_tidal\_potential\_forcing\_Q1]{\hyperref[sec:nm_tab_tidal_potential_forcing]{config\_use\_tidal\_potential\_forcing\_Q1}}
\label{subsec:nm_sec_config_use_tidal_potential_forcing_Q1}
\begin{center}
\begin{longtable}{| p{2.0in} || p{4.0in} |}
    \hline
    Type: & logical \\
    \hline
    Units: & -- \\
    \hline
    Default Value: & .true. \\
    \hline
    Possible Values: & .true. or .false. \\
    \hline
    \caption{config\_use\_tidal\_potential\_forcing\_Q1: Controls if tidal potential forcing for the Q1 constituent is used.}
\end{longtable}
\end{center}
\subsection[config\_use\_tidal\_potential\_forcing\_P1]{\hyperref[sec:nm_tab_tidal_potential_forcing]{config\_use\_tidal\_potential\_forcing\_P1}}
\label{subsec:nm_sec_config_use_tidal_potential_forcing_P1}
\begin{center}
\begin{longtable}{| p{2.0in} || p{4.0in} |}
    \hline
    Type: & logical \\
    \hline
    Units: & -- \\
    \hline
    Default Value: & .true. \\
    \hline
    Possible Values: & .true. or .false. \\
    \hline
    \caption{config\_use\_tidal\_potential\_forcing\_P1: Controls if tidal potential forcing for the P1 constituent is used.}
\end{longtable}
\end{center}
\subsection[config\_tidal\_potential\_ramp]{\hyperref[sec:nm_tab_tidal_potential_forcing]{config\_tidal\_potential\_ramp}}
\label{subsec:nm_sec_config_tidal_potential_ramp}
\begin{center}
\begin{longtable}{| p{2.0in} || p{4.0in} |}
    \hline
    Type: & real \\
    \hline
    Units: & \si{days} \\
    \hline
    Default Value: & 10.0 \\
    \hline
    Possible Values: & Any positive real number \\
    \hline
    \caption{config\_tidal\_potential\_ramp: Number of days over which the tidal potential forcing is ramped}
\end{longtable}
\end{center}
\subsection[config\_self\_attraction\_and\_loading\_beta]{\hyperref[sec:nm_tab_tidal_potential_forcing]{config\_self\_attraction\_and\_loading\_beta}}
\label{subsec:nm_sec_config_self_attraction_and_loading_beta}
\begin{center}
\begin{longtable}{| p{2.0in} || p{4.0in} |}
    \hline
    Type: & real \\
    \hline
    Units: & -- \\
    \hline
    Default Value: & 0.09 \\
    \hline
    Possible Values: & 0.0 to turn off, 0.09 is typical value to use for scalar approximation \\
    \hline
    \caption{config\_self\_attraction\_and\_loading\_beta: Coefficient for SAL scalar approximation}
\end{longtable}
\end{center}
\section[frazil\_ice]{\hyperref[sec:nm_tab_frazil_ice]{frazil\_ice}}
\label{sec:nm_sec_frazil_ice}
\subsection[config\_use\_frazil\_ice\_formation]{\hyperref[sec:nm_tab_frazil_ice]{config\_use\_frazil\_ice\_formation}}
\label{subsec:nm_sec_config_use_frazil_ice_formation}
\begin{center}
\begin{longtable}{| p{2.0in} || p{4.0in} |}
    \hline
    Type: & logical \\
    \hline
    Units: & -- \\
    \hline
    Default Value: & .false. \\
    \hline
    Possible Values: & .true. or .false. \\
    \hline
    \caption{config\_use\_frazil\_ice\_formation: Controls if fluxes related to frazil ice process are computed.}
\end{longtable}
\end{center}
\subsection[config\_frazil\_in\_open\_ocean]{\hyperref[sec:nm_tab_frazil_ice]{config\_frazil\_in\_open\_ocean}}
\label{subsec:nm_sec_config_frazil_in_open_ocean}
\begin{center}
\begin{longtable}{| p{2.0in} || p{4.0in} |}
    \hline
    Type: & logical \\
    \hline
    Units: & -- \\
    \hline
    Default Value: & .true. \\
    \hline
    Possible Values: & .true. or .false. \\
    \hline
    \caption{config\_frazil\_in\_open\_ocean: If frazil formation is used, controls if frazil fluxes are computed in the open ocean (as opposed to under land ice).}
\end{longtable}
\end{center}
\subsection[config\_frazil\_under\_land\_ice]{\hyperref[sec:nm_tab_frazil_ice]{config\_frazil\_under\_land\_ice}}
\label{subsec:nm_sec_config_frazil_under_land_ice}
\begin{center}
\begin{longtable}{| p{2.0in} || p{4.0in} |}
    \hline
    Type: & logical \\
    \hline
    Units: & -- \\
    \hline
    Default Value: & .true. \\
    \hline
    Possible Values: & .true. or .false. \\
    \hline
    \caption{config\_frazil\_under\_land\_ice: If frazil formation is used, controls if frazil fluxes are computed under land ice.}
\end{longtable}
\end{center}
\subsection[config\_frazil\_heat\_of\_fusion]{\hyperref[sec:nm_tab_frazil_ice]{config\_frazil\_heat\_of\_fusion}}
\label{subsec:nm_sec_config_frazil_heat_of_fusion}
\begin{center}
\begin{longtable}{| p{2.0in} || p{4.0in} |}
    \hline
    Type: & real \\
    \hline
    Units: & \si{J.kg^-1} \\
    \hline
    Default Value: & 3.34e5 \\
    \hline
    Possible Values: & Any positive real number. \\
    \hline
    \caption{config\_frazil\_heat\_of\_fusion: Energy per kilogram released when sea water freezes. NOTE: test and make consistent with E3SM.}
\end{longtable}
\end{center}
\subsection[config\_frazil\_ice\_density]{\hyperref[sec:nm_tab_frazil_ice]{config\_frazil\_ice\_density}}
\label{subsec:nm_sec_config_frazil_ice_density}
\begin{center}
\begin{longtable}{| p{2.0in} || p{4.0in} |}
    \hline
    Type: & real \\
    \hline
    Units: & \si{kg.m^-3} \\
    \hline
    Default Value: & 1000.0 \\
    \hline
    Possible Values: & Any positive real number. \\
    \hline
    \caption{config\_frazil\_ice\_density: Assumed density of frazil. NOTE: test and make consistent with E3SM.}
\end{longtable}
\end{center}
\subsection[config\_frazil\_fractional\_thickness\_limit]{\hyperref[sec:nm_tab_frazil_ice]{config\_frazil\_fractional\_thickness\_limit}}
\label{subsec:nm_sec_config_frazil_fractional_thickness_limit}
\begin{center}
\begin{longtable}{| p{2.0in} || p{4.0in} |}
    \hline
    Type: & real \\
    \hline
    Units: & \si{non-dimensional} \\
    \hline
    Default Value: & 0.1 \\
    \hline
    Possible Values: & Any positive real number between 0 and 1. \\
    \hline
    \caption{config\_frazil\_fractional\_thickness\_limit: maximum fraction of layer thickness than can be used or created at an instant by frazil.}
\end{longtable}
\end{center}
\subsection[config\_specific\_heat\_sea\_water]{\hyperref[sec:nm_tab_frazil_ice]{config\_specific\_heat\_sea\_water}}
\label{subsec:nm_sec_config_specific_heat_sea_water}
\begin{center}
\begin{longtable}{| p{2.0in} || p{4.0in} |}
    \hline
    Type: & real \\
    \hline
    Units: & \si{J.kg^-1.C^-1} \\
    \hline
    Default Value: & 3985.0 \\
    \hline
    Possible Values: & Any positive real number. \\
    \hline
    \caption{config\_specific\_heat\_sea\_water: Energy per kilogram per C needed to raise ocean temperature 1 C. NOTE: test and make consistent with E3SM.}
\end{longtable}
\end{center}
\subsection[config\_frazil\_maximum\_depth]{\hyperref[sec:nm_tab_frazil_ice]{config\_frazil\_maximum\_depth}}
\label{subsec:nm_sec_config_frazil_maximum_depth}
\begin{center}
\begin{longtable}{| p{2.0in} || p{4.0in} |}
    \hline
    Type: & real \\
    \hline
    Units: & \si{m} \\
    \hline
    Default Value: & 100.0 \\
    \hline
    Possible Values: & Any positive real number. \\
    \hline
    \caption{config\_frazil\_maximum\_depth: maximum depth for the formation of frazil}
\end{longtable}
\end{center}
\subsection[config\_frazil\_sea\_ice\_reference\_salinity]{\hyperref[sec:nm_tab_frazil_ice]{config\_frazil\_sea\_ice\_reference\_salinity}}
\label{subsec:nm_sec_config_frazil_sea_ice_reference_salinity}
\begin{center}
\begin{longtable}{| p{2.0in} || p{4.0in} |}
    \hline
    Type: & real \\
    \hline
    Units: & \si{1.e-3} \\
    \hline
    Default Value: & 4.0 \\
    \hline
    Possible Values: & Any positive real number. \\
    \hline
    \caption{config\_frazil\_sea\_ice\_reference\_salinity: assumed salinity of frazil ice in the open ocean.}
\end{longtable}
\end{center}
\subsection[config\_frazil\_land\_ice\_reference\_salinity]{\hyperref[sec:nm_tab_frazil_ice]{config\_frazil\_land\_ice\_reference\_salinity}}
\label{subsec:nm_sec_config_frazil_land_ice_reference_salinity}
\begin{center}
\begin{longtable}{| p{2.0in} || p{4.0in} |}
    \hline
    Type: & real \\
    \hline
    Units: & \si{1.e-3} \\
    \hline
    Default Value: & 0.0 \\
    \hline
    Possible Values: & Any non-negative real number. \\
    \hline
    \caption{config\_frazil\_land\_ice\_reference\_salinity: assumed salinity of frazil ice under land ice.}
\end{longtable}
\end{center}
\subsection[config\_frazil\_maximum\_freezing\_temperature]{\hyperref[sec:nm_tab_frazil_ice]{config\_frazil\_maximum\_freezing\_temperature}}
\label{subsec:nm_sec_config_frazil_maximum_freezing_temperature}
\begin{center}
\begin{longtable}{| p{2.0in} || p{4.0in} |}
    \hline
    Type: & real \\
    \hline
    Units: & \si{C} \\
    \hline
    Default Value: & 0.0 \\
    \hline
    Possible Values: & Any real number. \\
    \hline
    \caption{config\_frazil\_maximum\_freezing\_temperature: Maximum freezing temperature for the creation of frazil}
\end{longtable}
\end{center}
\subsection[config\_frazil\_use\_surface\_pressure]{\hyperref[sec:nm_tab_frazil_ice]{config\_frazil\_use\_surface\_pressure}}
\label{subsec:nm_sec_config_frazil_use_surface_pressure}
\begin{center}
\begin{longtable}{| p{2.0in} || p{4.0in} |}
    \hline
    Type: & logical \\
    \hline
    Units: & -- \\
    \hline
    Default Value: & .true. \\
    \hline
    Possible Values: & .true. or .false. \\
    \hline
    \caption{config\_frazil\_use\_surface\_pressure: Flag that controls if frazil formation will exert a surface pressure as it is formed.}
\end{longtable}
\end{center}
\section[land\_ice\_fluxes]{\hyperref[sec:nm_tab_land_ice_fluxes]{land\_ice\_fluxes}}
\label{sec:nm_sec_land_ice_fluxes}
\subsection[config\_land\_ice\_flux\_mode]{\hyperref[sec:nm_tab_land_ice_fluxes]{config\_land\_ice\_flux\_mode}}
\label{subsec:nm_sec_config_land_ice_flux_mode}
\begin{center}
\begin{longtable}{| p{2.0in} || p{4.0in} |}
    \hline
    Type: & character \\
    \hline
    Units: & -- \\
    \hline
    Default Value: & off \\
    \hline
    Possible Values: & 'off','pressure\_only', 'data', 'standalone','coupled' \\
    \hline
    \caption{config\_land\_ice\_flux\_mode: Selects the mode in which land-ice fluxes are computed.}
\end{longtable}
\end{center}
\subsection[config\_land\_ice\_flux\_formulation]{\hyperref[sec:nm_tab_land_ice_fluxes]{config\_land\_ice\_flux\_formulation}}
\label{subsec:nm_sec_config_land_ice_flux_formulation}
\begin{center}
\begin{longtable}{| p{2.0in} || p{4.0in} |}
    \hline
    Type: & character \\
    \hline
    Units: & -- \\
    \hline
    Default Value: & Jenkins \\
    \hline
    Possible Values: & 'ISOMIP', 'Jenkins', 'HollandJenkins' \\
    \hline
    \caption{config\_land\_ice\_flux\_formulation: Name of land-ice flux formulation.}
\end{longtable}
\end{center}
\subsection[config\_land\_ice\_flux\_useHollandJenkinsAdvDiff]{\hyperref[sec:nm_tab_land_ice_fluxes]{config\_land\_ice\_flux\_useHollandJenkinsAdvDiff}}
\label{subsec:nm_sec_config_land_ice_flux_useHollandJenkinsAdvDiff}
\begin{center}
\begin{longtable}{| p{2.0in} || p{4.0in} |}
    \hline
    Type: & logical \\
    \hline
    Units: & -- \\
    \hline
    Default Value: & .false. \\
    \hline
    Possible Values: & .true. and .false. \\
    \hline
    \caption{config\_land\_ice\_flux\_useHollandJenkinsAdvDiff: If .true. then uses the advection/diffusion scheme of Holland and Jenkins (1999) for ice-shelf heat fluxes}
\end{longtable}
\end{center}
\subsection[config\_land\_ice\_flux\_attenuation\_coefficient]{\hyperref[sec:nm_tab_land_ice_fluxes]{config\_land\_ice\_flux\_attenuation\_coefficient}}
\label{subsec:nm_sec_config_land_ice_flux_attenuation_coefficient}
\begin{center}
\begin{longtable}{| p{2.0in} || p{4.0in} |}
    \hline
    Type: & real \\
    \hline
    Units: & \si{m} \\
    \hline
    Default Value: & 10.0 \\
    \hline
    Possible Values: & Any positive real number. \\
    \hline
    \caption{config\_land\_ice\_flux\_attenuation\_coefficient: The vertical length scale of exponential decay for surface fluxes under land ice.}
\end{longtable}
\end{center}
\subsection[config\_land\_ice\_flux\_boundaryLayerThickness]{\hyperref[sec:nm_tab_land_ice_fluxes]{config\_land\_ice\_flux\_boundaryLayerThickness}}
\label{subsec:nm_sec_config_land_ice_flux_boundaryLayerThickness}
\begin{center}
\begin{longtable}{| p{2.0in} || p{4.0in} |}
    \hline
    Type: & real \\
    \hline
    Units: & \si{m} \\
    \hline
    Default Value: & 10.0 \\
    \hline
    Possible Values: & Any non-negative real number.  A value of 0 means that T and S are taken top level. \\
    \hline
    \caption{config\_land\_ice\_flux\_boundaryLayerThickness: The thickness of the sub-ice-shelf boundary layer, over which T and S will be averaged.}
\end{longtable}
\end{center}
\subsection[config\_land\_ice\_flux\_boundaryLayerNeighborWeight]{\hyperref[sec:nm_tab_land_ice_fluxes]{config\_land\_ice\_flux\_boundaryLayerNeighborWeight}}
\label{subsec:nm_sec_config_land_ice_flux_boundaryLayerNeighborWeight}
\begin{center}
\begin{longtable}{| p{2.0in} || p{4.0in} |}
    \hline
    Type: & real \\
    \hline
    Units: & -- \\
    \hline
    Default Value: & 0.0 \\
    \hline
    Possible Values: & Most likely a value between 0 (no smoothing) and 1 (neighbors get same weight as this cell). \\
    \hline
    \caption{config\_land\_ice\_flux\_boundaryLayerNeighborWeight: The for horizontal neighbors used to horizontally smooth boundary layer T and S.}
\end{longtable}
\end{center}
\subsection[config\_land\_ice\_flux\_cp\_ice]{\hyperref[sec:nm_tab_land_ice_fluxes]{config\_land\_ice\_flux\_cp\_ice}}
\label{subsec:nm_sec_config_land_ice_flux_cp_ice}
\begin{center}
\begin{longtable}{| p{2.0in} || p{4.0in} |}
    \hline
    Type: & real \\
    \hline
    Units: & \si{J.C^-1.kg^-1} \\
    \hline
    Default Value: & 2.009e3 \\
    \hline
    Possible Values: & Any positive real number \\
    \hline
    \caption{config\_land\_ice\_flux\_cp\_ice: The specific heat capacity for ice.}
\end{longtable}
\end{center}
\subsection[config\_land\_ice\_flux\_rho\_ice]{\hyperref[sec:nm_tab_land_ice_fluxes]{config\_land\_ice\_flux\_rho\_ice}}
\label{subsec:nm_sec_config_land_ice_flux_rho_ice}
\begin{center}
\begin{longtable}{| p{2.0in} || p{4.0in} |}
    \hline
    Type: & real \\
    \hline
    Units: & \si{kg.m^-3} \\
    \hline
    Default Value: & 918 \\
    \hline
    Possible Values: & Any positive real number \\
    \hline
    \caption{config\_land\_ice\_flux\_rho\_ice: The density of land ice.}
\end{longtable}
\end{center}
\subsection[config\_land\_ice\_flux\_explicit\_topDragCoeff]{\hyperref[sec:nm_tab_land_ice_fluxes]{config\_land\_ice\_flux\_explicit\_topDragCoeff}}
\label{subsec:nm_sec_config_land_ice_flux_explicit_topDragCoeff}
\begin{center}
\begin{longtable}{| p{2.0in} || p{4.0in} |}
    \hline
    Type: & real \\
    \hline
    Units: & -- \\
    \hline
    Default Value: & 2.5e-3 \\
    \hline
    Possible Values: & Any positive real number \\
    \hline
    \caption{config\_land\_ice\_flux\_explicit\_topDragCoeff: The top drag coefficient if config\_use\_implicit\_top\_drag\_coeff is false.}
\end{longtable}
\end{center}
\subsection[config\_land\_ice\_flux\_ISOMIP\_gammaT]{\hyperref[sec:nm_tab_land_ice_fluxes]{config\_land\_ice\_flux\_ISOMIP\_gammaT}}
\label{subsec:nm_sec_config_land_ice_flux_ISOMIP_gammaT}
\begin{center}
\begin{longtable}{| p{2.0in} || p{4.0in} |}
    \hline
    Type: & real \\
    \hline
    Units: & \si{m.s^-1} \\
    \hline
    Default Value: & 1e-4 \\
    \hline
    Possible Values: & Any positive real number \\
    \hline
    \caption{config\_land\_ice\_flux\_ISOMIP\_gammaT: The constant heat transport velocity through the boundary layer under land ice used in the ISOMIP test cases.}
\end{longtable}
\end{center}
\subsection[config\_land\_ice\_flux\_rms\_tidal\_velocity]{\hyperref[sec:nm_tab_land_ice_fluxes]{config\_land\_ice\_flux\_rms\_tidal\_velocity}}
\label{subsec:nm_sec_config_land_ice_flux_rms_tidal_velocity}
\begin{center}
\begin{longtable}{| p{2.0in} || p{4.0in} |}
    \hline
    Type: & real \\
    \hline
    Units: & \si{m.s^-1} \\
    \hline
    Default Value: & 5e-2 \\
    \hline
    Possible Values: & Any non-negative real number \\
    \hline
    \caption{config\_land\_ice\_flux\_rms\_tidal\_velocity: Parameterization of tidal velocity used in computing the sub-ice-shelf friction velocity}
\end{longtable}
\end{center}
\subsection[config\_land\_ice\_flux\_jenkins\_heat\_transfer\_coefficient]{\hyperref[sec:nm_tab_land_ice_fluxes]{config\_land\_ice\_flux\_jenkins\_heat\_transfer\_coefficient}}
\label{subsec:nm_sec_config_land_ice_flux_jenkins_heat_transfer_coefficient}
\begin{center}
\begin{longtable}{| p{2.0in} || p{4.0in} |}
    \hline
    Type: & real \\
    \hline
    Units: & -- \\
    \hline
    Default Value: & 0.011 \\
    \hline
    Possible Values: & Any positive real number \\
    \hline
    \caption{config\_land\_ice\_flux\_jenkins\_heat\_transfer\_coefficient: constant nondimensional heat transfer coefficient across the ice-ocean boundary layer}
\end{longtable}
\end{center}
\subsection[config\_land\_ice\_flux\_jenkins\_salt\_transfer\_coefficient]{\hyperref[sec:nm_tab_land_ice_fluxes]{config\_land\_ice\_flux\_jenkins\_salt\_transfer\_coefficient}}
\label{subsec:nm_sec_config_land_ice_flux_jenkins_salt_transfer_coefficient}
\begin{center}
\begin{longtable}{| p{2.0in} || p{4.0in} |}
    \hline
    Type: & real \\
    \hline
    Units: & -- \\
    \hline
    Default Value: & 3.1e-4 \\
    \hline
    Possible Values: & Any positive real number \\
    \hline
    \caption{config\_land\_ice\_flux\_jenkins\_salt\_transfer\_coefficient: constant nondimensional salt transfer coefficient across the ice-ocean boundary layer}
\end{longtable}
\end{center}
\section[advection]{\hyperref[sec:nm_tab_advection]{advection}}
\label{sec:nm_sec_advection}
\subsection[config\_vert\_advection\_method]{\hyperref[sec:nm_tab_advection]{config\_vert\_advection\_method}}
\label{subsec:nm_sec_config_vert_advection_method}
\begin{center}
\begin{longtable}{| p{2.0in} || p{4.0in} |}
    \hline
    Type: & character \\
    \hline
    Units: & -- \\
    \hline
    Default Value: & flux-form \\
    \hline
    Possible Values: & 'flux-form' and 'remap' \\
    \hline
    \caption{config\_vert\_advection\_method: Method for advecting tracers, momentum, and thickness vertically.}
\end{longtable}
\end{center}
\subsection[config\_vert\_remap\_order]{\hyperref[sec:nm_tab_advection]{config\_vert\_remap\_order}}
\label{subsec:nm_sec_config_vert_remap_order}
\begin{center}
\begin{longtable}{| p{2.0in} || p{4.0in} |}
    \hline
    Type: & integer \\
    \hline
    Units: & -- \\
    \hline
    Default Value: & 3 \\
    \hline
    Possible Values: & 1, 2, 3 and 5 \\
    \hline
    \caption{config\_vert\_remap\_order: Order of remapping method used for momentum and tracer advection}
\end{longtable}
\end{center}
\subsection[config\_vert\_remap\_interval]{\hyperref[sec:nm_tab_advection]{config\_vert\_remap\_interval}}
\label{subsec:nm_sec_config_vert_remap_interval}
\begin{center}
\begin{longtable}{| p{2.0in} || p{4.0in} |}
    \hline
    Type: & integer \\
    \hline
    Units: & -- \\
    \hline
    Default Value: & 0 \\
    \hline
    Possible Values: & Any integer greater than or equal to 0 \\
    \hline
    \caption{config\_vert\_remap\_interval: Number of timesteps between each remapping. If 0, remapping occurs every timestep}
\end{longtable}
\end{center}
\subsection[config\_vert\_tracer\_adv\_flux\_order]{\hyperref[sec:nm_tab_advection]{config\_vert\_tracer\_adv\_flux\_order}}
\label{subsec:nm_sec_config_vert_tracer_adv_flux_order}
\begin{center}
\begin{longtable}{| p{2.0in} || p{4.0in} |}
    \hline
    Type: & integer \\
    \hline
    Units: & -- \\
    \hline
    Default Value: & 3 \\
    \hline
    Possible Values: & 2, 3 and 4 \\
    \hline
    \caption{config\_vert\_tracer\_adv\_flux\_order: Order of polynomial used for tracer reconstruction at layer edges for flux-form method}
\end{longtable}
\end{center}
\subsection[config\_horiz\_tracer\_adv\_order]{\hyperref[sec:nm_tab_advection]{config\_horiz\_tracer\_adv\_order}}
\label{subsec:nm_sec_config_horiz_tracer_adv_order}
\begin{center}
\begin{longtable}{| p{2.0in} || p{4.0in} |}
    \hline
    Type: & integer \\
    \hline
    Units: & -- \\
    \hline
    Default Value: & 3 \\
    \hline
    Possible Values: & 2, 3 and 4 \\
    \hline
    \caption{config\_horiz\_tracer\_adv\_order: Order of polynomial used for tracer reconstruction at cell edges}
\end{longtable}
\end{center}
\subsection[config\_coef\_3rd\_order]{\hyperref[sec:nm_tab_advection]{config\_coef\_3rd\_order}}
\label{subsec:nm_sec_config_coef_3rd_order}
\begin{center}
\begin{longtable}{| p{2.0in} || p{4.0in} |}
    \hline
    Type: & real \\
    \hline
    Units: & -- \\
    \hline
    Default Value: & 0.25 \\
    \hline
    Possible Values: & any real between 0 and 1 \\
    \hline
    \caption{config\_coef\_3rd\_order: Reconstruction of 3rd-order reconstruction to blend with 4th-order reconstuction}
\end{longtable}
\end{center}
\subsection[config\_flux\_limiter]{\hyperref[sec:nm_tab_advection]{config\_flux\_limiter}}
\label{subsec:nm_sec_config_flux_limiter}
\begin{center}
\begin{longtable}{| p{2.0in} || p{4.0in} |}
    \hline
    Type: & character \\
    \hline
    Units: & -- \\
    \hline
    Default Value: & monotonic \\
    \hline
    Possible Values: & 'none','monotonic' \\
    \hline
    \caption{config\_flux\_limiter: Slope limiter for the flux-form advection scheme.}
\end{longtable}
\end{center}
\subsection[config\_remap\_limiter]{\hyperref[sec:nm_tab_advection]{config\_remap\_limiter}}
\label{subsec:nm_sec_config_remap_limiter}
\begin{center}
\begin{longtable}{| p{2.0in} || p{4.0in} |}
    \hline
    Type: & character \\
    \hline
    Units: & -- \\
    \hline
    Default Value: & monotonic \\
    \hline
    Possible Values: & 'none','monotonic','weno' \\
    \hline
    \caption{config\_remap\_limiter: Slope limiter for the vertical remap advection scheme.}
\end{longtable}
\end{center}
\subsection[config\_thickness\_flux\_type]{\hyperref[sec:nm_tab_advection]{config\_thickness\_flux\_type}}
\label{subsec:nm_sec_config_thickness_flux_type}
\begin{center}
\begin{longtable}{| p{2.0in} || p{4.0in} |}
    \hline
    Type: & character \\
    \hline
    Units: & -- \\
    \hline
    Default Value: & centered \\
    \hline
    Possible Values: & 'upwind', 'centered', 'constant' \\
    \hline
    \caption{config\_thickness\_flux\_type: If 'upwind', use upwind to evaluate the edge-value for layerThickness, i.e., layerThickEdgeFlux.  The standard MPAS-O approach is 'centered'. For 'constant', uses constant thickness in time from restingThickness, for linear test problems. Note that these two flags are set together for linearized test cases: config\_thickness\_flux\_type = 'constant' linearizes the thickness equation, and config\_disable\_vel\_hadv = .true. linearizes the momentum equation if there is no assumed mean background velocity.}
\end{longtable}
\end{center}
\section[bottom\_drag]{\hyperref[sec:nm_tab_bottom_drag]{bottom\_drag}}
\label{sec:nm_sec_bottom_drag}
\subsection[config\_bottom\_drag\_mode]{\hyperref[sec:nm_tab_bottom_drag]{config\_bottom\_drag\_mode}}
\label{subsec:nm_sec_config_bottom_drag_mode}
\begin{center}
\begin{longtable}{| p{2.0in} || p{4.0in} |}
    \hline
    Type: & character \\
    \hline
    Units: & -- \\
    \hline
    Default Value: & implicit \\
    \hline
    Possible Values: & 'implicit','explicit' \\
    \hline
    \caption{config\_bottom\_drag\_mode: Formulation of the bottom drag.}
\end{longtable}
\end{center}
\subsection[config\_implicit\_bottom\_drag\_type]{\hyperref[sec:nm_tab_bottom_drag]{config\_implicit\_bottom\_drag\_type}}
\label{subsec:nm_sec_config_implicit_bottom_drag_type}
\begin{center}
\begin{longtable}{| p{2.0in} || p{4.0in} |}
    \hline
    Type: & character \\
    \hline
    Units: & -- \\
    \hline
    Default Value: & constant \\
    \hline
    Possible Values: & 'constant','constant\_and\_rayleigh','loglaw','spatially-variable','mannings' \\
    \hline
    \caption{config\_implicit\_bottom\_drag\_type: Type of implicit bottom drag used.}
\end{longtable}
\end{center}
\subsection[config\_implicit\_constant\_bottom\_drag\_coeff]{\hyperref[sec:nm_tab_bottom_drag]{config\_implicit\_constant\_bottom\_drag\_coeff}}
\label{subsec:nm_sec_config_implicit_constant_bottom_drag_coeff}
\begin{center}
\begin{longtable}{| p{2.0in} || p{4.0in} |}
    \hline
    Type: & real \\
    \hline
    Units: & -- \\
    \hline
    Default Value: & 1.0e-3 \\
    \hline
    Possible Values: & any positive real, typically 1.0e-3 \\
    \hline
    \caption{config\_implicit\_constant\_bottom\_drag\_coeff: Dimensionless bottom drag coefficient, $c_{drag}$.}
\end{longtable}
\end{center}
\subsection[config\_use\_implicit\_top\_drag]{\hyperref[sec:nm_tab_bottom_drag]{config\_use\_implicit\_top\_drag}}
\label{subsec:nm_sec_config_use_implicit_top_drag}
\begin{center}
\begin{longtable}{| p{2.0in} || p{4.0in} |}
    \hline
    Type: & logical \\
    \hline
    Units: & -- \\
    \hline
    Default Value: & .false. \\
    \hline
    Possible Values: & .true. or .false. \\
    \hline
    \caption{config\_use\_implicit\_top\_drag: If true, implicit top drag is used on the momentum equation.}
\end{longtable}
\end{center}
\subsection[config\_implicit\_top\_drag\_coeff]{\hyperref[sec:nm_tab_bottom_drag]{config\_implicit\_top\_drag\_coeff}}
\label{subsec:nm_sec_config_implicit_top_drag_coeff}
\begin{center}
\begin{longtable}{| p{2.0in} || p{4.0in} |}
    \hline
    Type: & real \\
    \hline
    Units: & -- \\
    \hline
    Default Value: & 1.0e-3 \\
    \hline
    Possible Values: & any positive real, typically 1.0e-3 \\
    \hline
    \caption{config\_implicit\_top\_drag\_coeff: Dimensionless top drag coefficient, $c_{drag}$.}
\end{longtable}
\end{center}
\subsection[config\_loglaw\_bottom\_roughness]{\hyperref[sec:nm_tab_bottom_drag]{config\_loglaw\_bottom\_roughness}}
\label{subsec:nm_sec_config_loglaw_bottom_roughness}
\begin{center}
\begin{longtable}{| p{2.0in} || p{4.0in} |}
    \hline
    Type: & real \\
    \hline
    Units: & \si{m} \\
    \hline
    Default Value: & 1.0e-3 \\
    \hline
    Possible Values: & any positive real, typically 1.0e-3 \\
    \hline
    \caption{config\_loglaw\_bottom\_roughness: Bottom roughness, $z_0$, in meters.}
\end{longtable}
\end{center}
\subsection[config\_loglaw\_layer\_depth\_max]{\hyperref[sec:nm_tab_bottom_drag]{config\_loglaw\_layer\_depth\_max}}
\label{subsec:nm_sec_config_loglaw_layer_depth_max}
\begin{center}
\begin{longtable}{| p{2.0in} || p{4.0in} |}
    \hline
    Type: & real \\
    \hline
    Units: & \si{m} \\
    \hline
    Default Value: & 10.0 \\
    \hline
    Possible Values: & any positive real, typically 10m \\
    \hline
    \caption{config\_loglaw\_layer\_depth\_max: Maximum distance above the seafloor at which log-law drag is applied.}
\end{longtable}
\end{center}
\subsection[config\_loglaw\_bottom\_drag\_min]{\hyperref[sec:nm_tab_bottom_drag]{config\_loglaw\_bottom\_drag\_min}}
\label{subsec:nm_sec_config_loglaw_bottom_drag_min}
\begin{center}
\begin{longtable}{| p{2.0in} || p{4.0in} |}
    \hline
    Type: & real \\
    \hline
    Units: & -- \\
    \hline
    Default Value: & 2.5e-3 \\
    \hline
    Possible Values: & any positive real, typically 2.5e-3 \\
    \hline
    \caption{config\_loglaw\_bottom\_drag\_min: Dimensionless bottom drag minimum used in log-law parameterization.}
\end{longtable}
\end{center}
\subsection[config\_loglaw\_bottom\_drag\_max]{\hyperref[sec:nm_tab_bottom_drag]{config\_loglaw\_bottom\_drag\_max}}
\label{subsec:nm_sec_config_loglaw_bottom_drag_max}
\begin{center}
\begin{longtable}{| p{2.0in} || p{4.0in} |}
    \hline
    Type: & real \\
    \hline
    Units: & -- \\
    \hline
    Default Value: & 1.0e-1 \\
    \hline
    Possible Values: & any positive real, typically 1.0e-1 \\
    \hline
    \caption{config\_loglaw\_bottom\_drag\_max: Dimensionless bottom drag maximum used in log-law parameterization.}
\end{longtable}
\end{center}
\subsection[config\_explicit\_bottom\_drag\_coeff]{\hyperref[sec:nm_tab_bottom_drag]{config\_explicit\_bottom\_drag\_coeff}}
\label{subsec:nm_sec_config_explicit_bottom_drag_coeff}
\begin{center}
\begin{longtable}{| p{2.0in} || p{4.0in} |}
    \hline
    Type: & real \\
    \hline
    Units: & -- \\
    \hline
    Default Value: & 1.0e-3 \\
    \hline
    Possible Values: & any positive real, typically 1.0e-3 \\
    \hline
    \caption{config\_explicit\_bottom\_drag\_coeff: Dimensionless explicit bottom drag coefficient, $c_{drag}$.}
\end{longtable}
\end{center}
\subsection[config\_use\_topographic\_wave\_drag]{\hyperref[sec:nm_tab_bottom_drag]{config\_use\_topographic\_wave\_drag}}
\label{subsec:nm_sec_config_use_topographic_wave_drag}
\begin{center}
\begin{longtable}{| p{2.0in} || p{4.0in} |}
    \hline
    Type: & logical \\
    \hline
    Units: & -- \\
    \hline
    Default Value: & .false. \\
    \hline
    Possible Values: & .true. or .false. \\
    \hline
    \caption{config\_use\_topographic\_wave\_drag: If true, topographic wave drag is used on the momentum equation.}
\end{longtable}
\end{center}
\subsection[config\_topographic\_wave\_drag\_coeff]{\hyperref[sec:nm_tab_bottom_drag]{config\_topographic\_wave\_drag\_coeff}}
\label{subsec:nm_sec_config_topographic_wave_drag_coeff}
\begin{center}
\begin{longtable}{| p{2.0in} || p{4.0in} |}
    \hline
    Type: & real \\
    \hline
    Units: & -- \\
    \hline
    Default Value: & 5.0e-4 \\
    \hline
    Possible Values: & any positive real, typically 5.0e-4 \\
    \hline
    \caption{config\_topographic\_wave\_drag\_coeff: Dimensionless topographic wave drag coefficient, $c_{topo}$.}
\end{longtable}
\end{center}
\subsection[config\_thickness\_drag\_type]{\hyperref[sec:nm_tab_bottom_drag]{config\_thickness\_drag\_type}}
\label{subsec:nm_sec_config_thickness_drag_type}
\begin{center}
\begin{longtable}{| p{2.0in} || p{4.0in} |}
    \hline
    Type: & character \\
    \hline
    Units: & -- \\
    \hline
    Default Value: & centered \\
    \hline
    Possible Values: & 'harmonic-mean', 'centered' \\
    \hline
    \caption{config\_thickness\_drag\_type: The type of layerThickness averaging to use on the drag term. The standard MPAS-O approach is 'centered'.}
\end{longtable}
\end{center}
\section[Rayleigh\_damping]{\hyperref[sec:nm_tab_Rayleigh_damping]{Rayleigh\_damping}}
\label{sec:nm_sec_Rayleigh_damping}
\subsection[config\_Rayleigh\_damping\_coeff]{\hyperref[sec:nm_tab_Rayleigh_damping]{config\_Rayleigh\_damping\_coeff}}
\label{subsec:nm_sec_config_Rayleigh_damping_coeff}
\begin{center}
\begin{longtable}{| p{2.0in} || p{4.0in} |}
    \hline
    Type: & real \\
    \hline
    Units: & \si{s^-1} \\
    \hline
    Default Value: & 1.0e-4 \\
    \hline
    Possible Values: & Any positive real value. \\
    \hline
    \caption{config\_Rayleigh\_damping\_coeff: Inverse-time coefficient for the Rayleigh damping term, $c_R$, applied at every depth level.}
\end{longtable}
\end{center}
\subsection[config\_Rayleigh\_damping\_depth\_variable]{\hyperref[sec:nm_tab_Rayleigh_damping]{config\_Rayleigh\_damping\_depth\_variable}}
\label{subsec:nm_sec_config_Rayleigh_damping_depth_variable}
\begin{center}
\begin{longtable}{| p{2.0in} || p{4.0in} |}
    \hline
    Type: & logical \\
    \hline
    Units: & -- \\
    \hline
    Default Value: & .false. \\
    \hline
    Possible Values: & .true. or .false. \\
    \hline
    \caption{config\_Rayleigh\_damping\_depth\_variable: If true applies r h$^-1$ instead of just r.}
\end{longtable}
\end{center}
\subsection[config\_Rayleigh\_bottom\_friction]{\hyperref[sec:nm_tab_Rayleigh_damping]{config\_Rayleigh\_bottom\_friction}}
\label{subsec:nm_sec_config_Rayleigh_bottom_friction}
\begin{center}
\begin{longtable}{| p{2.0in} || p{4.0in} |}
    \hline
    Type: & logical \\
    \hline
    Units: & -- \\
    \hline
    Default Value: & .false. \\
    \hline
    Possible Values: & .true. or .false. \\
    \hline
    \caption{config\_Rayleigh\_bottom\_friction: If true, Rayleigh friction is only applied to the bottom}
\end{longtable}
\end{center}
\subsection[config\_Rayleigh\_bottom\_damping\_coeff]{\hyperref[sec:nm_tab_Rayleigh_damping]{config\_Rayleigh\_bottom\_damping\_coeff}}
\label{subsec:nm_sec_config_Rayleigh_bottom_damping_coeff}
\begin{center}
\begin{longtable}{| p{2.0in} || p{4.0in} |}
    \hline
    Type: & real \\
    \hline
    Units: & \si{s^-1} \\
    \hline
    Default Value: & 1.0e-4 \\
    \hline
    Possible Values: & Any positive real value. \\
    \hline
    \caption{config\_Rayleigh\_bottom\_damping\_coeff: Inverse-time coefficient for the Rayleigh damping term, $c_R$, only applied at the bottom.}
\end{longtable}
\end{center}
\section[vegetation\_drag]{\hyperref[sec:nm_tab_vegetation_drag]{vegetation\_drag}}
\label{sec:nm_sec_vegetation_drag}
\subsection[config\_use\_vegetation\_drag]{\hyperref[sec:nm_tab_vegetation_drag]{config\_use\_vegetation\_drag}}
\label{subsec:nm_sec_config_use_vegetation_drag}
\begin{center}
\begin{longtable}{| p{2.0in} || p{4.0in} |}
    \hline
    Type: & logical \\
    \hline
    Units: & -- \\
    \hline
    Default Value: & .false. \\
    \hline
    Possible Values: & .true. or .false. \\
    \hline
    \caption{config\_use\_vegetation\_drag: Controls if vegetation\_drag is used to compute Manning's roughness coefficient.}
\end{longtable}
\end{center}
\subsection[config\_vegetation\_drag\_coefficient]{\hyperref[sec:nm_tab_vegetation_drag]{config\_vegetation\_drag\_coefficient}}
\label{subsec:nm_sec_config_vegetation_drag_coefficient}
\begin{center}
\begin{longtable}{| p{2.0in} || p{4.0in} |}
    \hline
    Type: & real \\
    \hline
    Units: & -- \\
    \hline
    Default Value: & 1.09 \\
    \hline
    Possible Values: & O(1) \\
    \hline
    \caption{config\_vegetation\_drag\_coefficient: Vegetation drag coefficient}
\end{longtable}
\end{center}
\section[wetting\_drying]{\hyperref[sec:nm_tab_wetting_drying]{wetting\_drying}}
\label{sec:nm_sec_wetting_drying}
\subsection[config\_use\_wetting\_drying]{\hyperref[sec:nm_tab_wetting_drying]{config\_use\_wetting\_drying}}
\label{subsec:nm_sec_config_use_wetting_drying}
\begin{center}
\begin{longtable}{| p{2.0in} || p{4.0in} |}
    \hline
    Type: & logical \\
    \hline
    Units: & -- \\
    \hline
    Default Value: & .false. \\
    \hline
    Possible Values: & .true. or .false. \\
    \hline
    \caption{config\_use\_wetting\_drying: If true, use wetting and drying algorithm to allow for dry cells to config\_min\_cell\_height.}
\end{longtable}
\end{center}
\subsection[config\_prevent\_drying]{\hyperref[sec:nm_tab_wetting_drying]{config\_prevent\_drying}}
\label{subsec:nm_sec_config_prevent_drying}
\begin{center}
\begin{longtable}{| p{2.0in} || p{4.0in} |}
    \hline
    Type: & logical \\
    \hline
    Units: & -- \\
    \hline
    Default Value: & .false. \\
    \hline
    Possible Values: & .true. or .false. \\
    \hline
    \caption{config\_prevent\_drying: If true, prevent cells from drying past config\_min\_cell\_height.}
\end{longtable}
\end{center}
\subsection[config\_drying\_min\_cell\_height]{\hyperref[sec:nm_tab_wetting_drying]{config\_drying\_min\_cell\_height}}
\label{subsec:nm_sec_config_drying_min_cell_height}
\begin{center}
\begin{longtable}{| p{2.0in} || p{4.0in} |}
    \hline
    Type: & real \\
    \hline
    Units: & \si{m} \\
    \hline
    Default Value: & 1.0e-3 \\
    \hline
    Possible Values: & any positive real, typically 1.0e-3 \\
    \hline
    \caption{config\_drying\_min\_cell\_height: Minimum allowable cell height under drying.  Cell to be kept wet to at least this thickness. When ramp is applied this is the min edge height}
\end{longtable}
\end{center}
\subsection[config\_zero\_drying\_velocity]{\hyperref[sec:nm_tab_wetting_drying]{config\_zero\_drying\_velocity}}
\label{subsec:nm_sec_config_zero_drying_velocity}
\begin{center}
\begin{longtable}{| p{2.0in} || p{4.0in} |}
    \hline
    Type: & logical \\
    \hline
    Units: & -- \\
    \hline
    Default Value: & .false. \\
    \hline
    Possible Values: & .true. or .false. \\
    \hline
    \caption{config\_zero\_drying\_velocity: If true, just zero out velocity that is contributing to drying for cell that is drying.  This option can be used to estimate acceptable minimum thicknesses for a run.}
\end{longtable}
\end{center}
\subsection[config\_zero\_drying\_velocity\_ramp]{\hyperref[sec:nm_tab_wetting_drying]{config\_zero\_drying\_velocity\_ramp}}
\label{subsec:nm_sec_config_zero_drying_velocity_ramp}
\begin{center}
\begin{longtable}{| p{2.0in} || p{4.0in} |}
    \hline
    Type: & logical \\
    \hline
    Units: & -- \\
    \hline
    Default Value: & .false. \\
    \hline
    Possible Values: & .true. or .false. \\
    \hline
    \caption{config\_zero\_drying\_velocity\_ramp: If true, ramp velocities and tendencies to zero rather than applying a simple on/off switch.}
\end{longtable}
\end{center}
\subsection[config\_zero\_drying\_velocity\_ramp\_hmin]{\hyperref[sec:nm_tab_wetting_drying]{config\_zero\_drying\_velocity\_ramp\_hmin}}
\label{subsec:nm_sec_config_zero_drying_velocity_ramp_hmin}
\begin{center}
\begin{longtable}{| p{2.0in} || p{4.0in} |}
    \hline
    Type: & real \\
    \hline
    Units: & -- \\
    \hline
    Default Value: & 1e-3 \\
    \hline
    Possible Values: & Any positive real \\
    \hline
    \caption{config\_zero\_drying\_velocity\_ramp\_hmin: Minimum layer thickness at which velocities and tendencies are ramped toward zero. Recommended value equal to config\_drying\_min\_cell\_height.}
\end{longtable}
\end{center}
\subsection[config\_zero\_drying\_velocity\_ramp\_hmax]{\hyperref[sec:nm_tab_wetting_drying]{config\_zero\_drying\_velocity\_ramp\_hmax}}
\label{subsec:nm_sec_config_zero_drying_velocity_ramp_hmax}
\begin{center}
\begin{longtable}{| p{2.0in} || p{4.0in} |}
    \hline
    Type: & real \\
    \hline
    Units: & -- \\
    \hline
    Default Value: & 2e-3 \\
    \hline
    Possible Values: & Any positive real \\
    \hline
    \caption{config\_zero\_drying\_velocity\_ramp\_hmax: Maximum layer thickness at which velocities and tendencies are ramped toward zero. Recommended values between 2x and 10x config\_drying\_min\_cell\_height.}
\end{longtable}
\end{center}
\subsection[config\_verify\_not\_dry]{\hyperref[sec:nm_tab_wetting_drying]{config\_verify\_not\_dry}}
\label{subsec:nm_sec_config_verify_not_dry}
\begin{center}
\begin{longtable}{| p{2.0in} || p{4.0in} |}
    \hline
    Type: & logical \\
    \hline
    Units: & \si{unitless} \\
    \hline
    Default Value: & .false. \\
    \hline
    Possible Values: & .true. or .false. \\
    \hline
    \caption{config\_verify\_not\_dry: If true, verify that cells are at least config\_min\_cell\_height thick.}
\end{longtable}
\end{center}
\subsection[config\_drying\_safety\_height]{\hyperref[sec:nm_tab_wetting_drying]{config\_drying\_safety\_height}}
\label{subsec:nm_sec_config_drying_safety_height}
\begin{center}
\begin{longtable}{| p{2.0in} || p{4.0in} |}
    \hline
    Type: & real \\
    \hline
    Units: & \si{m} \\
    \hline
    Default Value: & 1.0e-10 \\
    \hline
    Possible Values: & Real value greater or equal to 0. \\
    \hline
    \caption{config\_drying\_safety\_height: Safety factor on minimum cell height to ensure the minimum height is not violated due to round-off.}
\end{longtable}
\end{center}
\section[ocean\_constants]{\hyperref[sec:nm_tab_ocean_constants]{ocean\_constants}}
\label{sec:nm_sec_ocean_constants}
\subsection[config\_density0]{\hyperref[sec:nm_tab_ocean_constants]{config\_density0}}
\label{subsec:nm_sec_config_density0}
\begin{center}
\begin{longtable}{| p{2.0in} || p{4.0in} |}
    \hline
    Type: & real \\
    \hline
    Units: & \si{kg.m^-3} \\
    \hline
    Default Value: & 1026.0 \\
    \hline
    Possible Values: & any positive real, but typically 1000-1035 \\
    \hline
    \caption{config\_density0: Density used as a coefficient of the pressure gradient terms, $\rho_0$. This is a constant due to the Boussinesq approximation.}
\end{longtable}
\end{center}
\section[lts]{\hyperref[sec:nm_tab_lts]{lts}}
\label{sec:nm_sec_lts}
\subsection[config\_dt\_scaling\_LTS]{\hyperref[sec:nm_tab_lts]{config\_dt\_scaling\_LTS}}
\label{subsec:nm_sec_config_dt_scaling_LTS}
\begin{center}
\begin{longtable}{| p{2.0in} || p{4.0in} |}
    \hline
    Type: & integer \\
    \hline
    Units: & -- \\
    \hline
    Default Value: & 1 \\
    \hline
    Possible Values: & Any positive integer greater than or equal to one. A value of one employs the same dt in all regions. \\
    \hline
    \caption{config\_dt\_scaling\_LTS: The ratio between the dt on the coarse region and the dt on the fine region. Specifically, it is the positive integer M that defines dtFine for LTS, dtFine = dt / M.}
\end{longtable}
\end{center}
\section[pressure\_gradient]{\hyperref[sec:nm_tab_pressure_gradient]{pressure\_gradient}}
\label{sec:nm_sec_pressure_gradient}
\subsection[config\_pressure\_gradient\_type]{\hyperref[sec:nm_tab_pressure_gradient]{config\_pressure\_gradient\_type}}
\label{subsec:nm_sec_config_pressure_gradient_type}
\begin{center}
\begin{longtable}{| p{2.0in} || p{4.0in} |}
    \hline
    Type: & character \\
    \hline
    Units: & -- \\
    \hline
    Default Value: & pressure\_and\_zmid \\
    \hline
    Possible Values: & 'ssh\_gradient', 'pressure\_and\_zmid' or 'Jacobian\_from\_density' or 'Jacobian\_from\_TS' or 'MontgomeryPotential' or 'constant\_forced' \\
    \hline
    \caption{config\_pressure\_gradient\_type: Form of pressure gradient terms in momentum equation. For most applications, the gradient of pressure and layer mid-depth are appropriate.  For isopycnal coordinates, one may use the gradient of the Montgomery potential.  The sea surface height gradient (ssh\_gradient) option is for barotropic, depth-averaged pressure.}
\end{longtable}
\end{center}
\subsection[config\_common\_level\_weight]{\hyperref[sec:nm_tab_pressure_gradient]{config\_common\_level\_weight}}
\label{subsec:nm_sec_config_common_level_weight}
\begin{center}
\begin{longtable}{| p{2.0in} || p{4.0in} |}
    \hline
    Type: & real \\
    \hline
    Units: & -- \\
    \hline
    Default Value: & 0.5 \\
    \hline
    Possible Values: & any real between 0 and 1 \\
    \hline
    \caption{config\_common\_level\_weight: The weight between standard Jacobian and weighted Jacobian, $\gamma$.}
\end{longtable}
\end{center}
\subsection[config\_zonal\_ssh\_grad]{\hyperref[sec:nm_tab_pressure_gradient]{config\_zonal\_ssh\_grad}}
\label{subsec:nm_sec_config_zonal_ssh_grad}
\begin{center}
\begin{longtable}{| p{2.0in} || p{4.0in} |}
    \hline
    Type: & real \\
    \hline
    Units: & -- \\
    \hline
    Default Value: & 0.0 \\
    \hline
    Possible Values: & any real \\
    \hline
    \caption{config\_zonal\_ssh\_grad: The zonal (x) ssh gradient to be applied.}
\end{longtable}
\end{center}
\subsection[config\_meridional\_ssh\_grad]{\hyperref[sec:nm_tab_pressure_gradient]{config\_meridional\_ssh\_grad}}
\label{subsec:nm_sec_config_meridional_ssh_grad}
\begin{center}
\begin{longtable}{| p{2.0in} || p{4.0in} |}
    \hline
    Type: & real \\
    \hline
    Units: & -- \\
    \hline
    Default Value: & 0.0 \\
    \hline
    Possible Values: & any real \\
    \hline
    \caption{config\_meridional\_ssh\_grad: The meridional (y) ssh gradient to be applied.}
\end{longtable}
\end{center}
\section[eos]{\hyperref[sec:nm_tab_eos]{eos}}
\label{sec:nm_sec_eos}
\subsection[config\_eos\_type]{\hyperref[sec:nm_tab_eos]{config\_eos\_type}}
\label{subsec:nm_sec_config_eos_type}
\begin{center}
\begin{longtable}{| p{2.0in} || p{4.0in} |}
    \hline
    Type: & character \\
    \hline
    Units: & -- \\
    \hline
    Default Value: & linear \\
    \hline
    Possible Values: & Jackett McDougall EOS = 'jm', Wright = 'wright', and Linear EOS = 'linear' \\
    \hline
    \caption{config\_eos\_type: Character string to choose EOS formulation}
\end{longtable}
\end{center}
\subsection[config\_open\_ocean\_freezing\_temperature\_coeff\_0]{\hyperref[sec:nm_tab_eos]{config\_open\_ocean\_freezing\_temperature\_coeff\_0}}
\label{subsec:nm_sec_config_open_ocean_freezing_temperature_coeff_0}
\begin{center}
\begin{longtable}{| p{2.0in} || p{4.0in} |}
    \hline
    Type: & real \\
    \hline
    Units: & \si{C} \\
    \hline
    Default Value: & 0.0 \\
    \hline
    Possible Values: & Any real number \\
    \hline
    \caption{config\_open\_ocean\_freezing\_temperature\_coeff\_0: The freezing temperature at zero pressure and salinity in open ocean.}
\end{longtable}
\end{center}
\subsection[config\_open\_ocean\_freezing\_temperature\_coeff\_S]{\hyperref[sec:nm_tab_eos]{config\_open\_ocean\_freezing\_temperature\_coeff\_S}}
\label{subsec:nm_sec_config_open_ocean_freezing_temperature_coeff_S}
\begin{center}
\begin{longtable}{| p{2.0in} || p{4.0in} |}
    \hline
    Type: & real \\
    \hline
    Units: & \si{1.e3.C} \\
    \hline
    Default Value: & 0.0 \\
    \hline
    Possible Values: & Any real number \\
    \hline
    \caption{config\_open\_ocean\_freezing\_temperature\_coeff\_S: The coefficient for the term proportional to salinity in the freezing temperature in the open ocean.}
\end{longtable}
\end{center}
\subsection[config\_open\_ocean\_freezing\_temperature\_coeff\_p]{\hyperref[sec:nm_tab_eos]{config\_open\_ocean\_freezing\_temperature\_coeff\_p}}
\label{subsec:nm_sec_config_open_ocean_freezing_temperature_coeff_p}
\begin{center}
\begin{longtable}{| p{2.0in} || p{4.0in} |}
    \hline
    Type: & real \\
    \hline
    Units: & \si{C.Pa^-1} \\
    \hline
    Default Value: & 0.0 \\
    \hline
    Possible Values: & Any real number \\
    \hline
    \caption{config\_open\_ocean\_freezing\_temperature\_coeff\_p: The coefficient for the term proportional to the pressure in the freezing temperature in the open ocean.}
\end{longtable}
\end{center}
\subsection[config\_open\_ocean\_freezing\_temperature\_coeff\_pS]{\hyperref[sec:nm_tab_eos]{config\_open\_ocean\_freezing\_temperature\_coeff\_pS}}
\label{subsec:nm_sec_config_open_ocean_freezing_temperature_coeff_pS}
\begin{center}
\begin{longtable}{| p{2.0in} || p{4.0in} |}
    \hline
    Type: & real \\
    \hline
    Units: & \si{1.e3.C.Pa^-1} \\
    \hline
    Default Value: & 0.0 \\
    \hline
    Possible Values: & Any real number \\
    \hline
    \caption{config\_open\_ocean\_freezing\_temperature\_coeff\_pS: The coefficient for the term proportional to salinity times pressure in the freezing temperature in the open ocean.}
\end{longtable}
\end{center}
\subsection[config\_open\_ocean\_freezing\_temperature\_coeff\_mushy\_az1\_liq]{\hyperref[sec:nm_tab_eos]{config\_open\_ocean\_freezing\_temperature\_coeff\_mushy\_az1\_liq}}
\label{subsec:nm_sec_config_open_ocean_freezing_temperature_coeff_mushy_az1_liq}
\begin{center}
\begin{longtable}{| p{2.0in} || p{4.0in} |}
    \hline
    Type: & real \\
    \hline
    Units: & \si{1.e-3.C^-1} \\
    \hline
    Default Value: & -18.48 \\
    \hline
    Possible Values: & Any non-positive number \\
    \hline
    \caption{config\_open\_ocean\_freezing\_temperature\_coeff\_mushy\_az1\_liq: The coefficient for the mushy sea-ice physics term az1\_liq in the open ocean.}
\end{longtable}
\end{center}
\subsection[config\_land\_ice\_cavity\_freezing\_temperature\_coeff\_0]{\hyperref[sec:nm_tab_eos]{config\_land\_ice\_cavity\_freezing\_temperature\_coeff\_0}}
\label{subsec:nm_sec_config_land_ice_cavity_freezing_temperature_coeff_0}
\begin{center}
\begin{longtable}{| p{2.0in} || p{4.0in} |}
    \hline
    Type: & real \\
    \hline
    Units: & \si{C} \\
    \hline
    Default Value: & 6.22e-2 \\
    \hline
    Possible Values: & Any real number \\
    \hline
    \caption{config\_land\_ice\_cavity\_freezing\_temperature\_coeff\_0: The freezing temperature at zero pressure and salinity in land-ice cavities.}
\end{longtable}
\end{center}
\subsection[config\_land\_ice\_cavity\_freezing\_temperature\_coeff\_S]{\hyperref[sec:nm_tab_eos]{config\_land\_ice\_cavity\_freezing\_temperature\_coeff\_S}}
\label{subsec:nm_sec_config_land_ice_cavity_freezing_temperature_coeff_S}
\begin{center}
\begin{longtable}{| p{2.0in} || p{4.0in} |}
    \hline
    Type: & real \\
    \hline
    Units: & \si{1.e3.C} \\
    \hline
    Default Value: & -5.63e-2 \\
    \hline
    Possible Values: & Any real number \\
    \hline
    \caption{config\_land\_ice\_cavity\_freezing\_temperature\_coeff\_S: The coefficient for the term proportional to salinity in the freezing temperature in land-ice cavities.}
\end{longtable}
\end{center}
\subsection[config\_land\_ice\_cavity\_freezing\_temperature\_coeff\_p]{\hyperref[sec:nm_tab_eos]{config\_land\_ice\_cavity\_freezing\_temperature\_coeff\_p}}
\label{subsec:nm_sec_config_land_ice_cavity_freezing_temperature_coeff_p}
\begin{center}
\begin{longtable}{| p{2.0in} || p{4.0in} |}
    \hline
    Type: & real \\
    \hline
    Units: & \si{C.Pa^-1} \\
    \hline
    Default Value: & -7.43e-8 \\
    \hline
    Possible Values: & Any real number \\
    \hline
    \caption{config\_land\_ice\_cavity\_freezing\_temperature\_coeff\_p: The coefficient for the term proportional to the pressure in the freezing temperature in land-ice cavities.}
\end{longtable}
\end{center}
\subsection[config\_land\_ice\_cavity\_freezing\_temperature\_coeff\_pS]{\hyperref[sec:nm_tab_eos]{config\_land\_ice\_cavity\_freezing\_temperature\_coeff\_pS}}
\label{subsec:nm_sec_config_land_ice_cavity_freezing_temperature_coeff_pS}
\begin{center}
\begin{longtable}{| p{2.0in} || p{4.0in} |}
    \hline
    Type: & real \\
    \hline
    Units: & \si{1.e3.C.Pa^-1} \\
    \hline
    Default Value: & -1.74e-10 \\
    \hline
    Possible Values: & Any real number \\
    \hline
    \caption{config\_land\_ice\_cavity\_freezing\_temperature\_coeff\_pS: The coefficient for the term proportional to salinity times pressure in the freezing temperature in land-ice cavities.}
\end{longtable}
\end{center}
\section[eos\_linear]{\hyperref[sec:nm_tab_eos_linear]{eos\_linear}}
\label{sec:nm_sec_eos_linear}
\subsection[config\_eos\_linear\_alpha]{\hyperref[sec:nm_tab_eos_linear]{config\_eos\_linear\_alpha}}
\label{subsec:nm_sec_config_eos_linear_alpha}
\begin{center}
\begin{longtable}{| p{2.0in} || p{4.0in} |}
    \hline
    Type: & real \\
    \hline
    Units: & \si{kg.m^-3.C^-1} \\
    \hline
    Default Value: & 0.2 \\
    \hline
    Possible Values: & any positive real \\
    \hline
    \caption{config\_eos\_linear\_alpha: Linear thermal expansion coefficient}
\end{longtable}
\end{center}
\subsection[config\_eos\_linear\_beta]{\hyperref[sec:nm_tab_eos_linear]{config\_eos\_linear\_beta}}
\label{subsec:nm_sec_config_eos_linear_beta}
\begin{center}
\begin{longtable}{| p{2.0in} || p{4.0in} |}
    \hline
    Type: & real \\
    \hline
    Units: & \si{1.e3.kg.m^-3} \\
    \hline
    Default Value: & 0.8 \\
    \hline
    Possible Values: & any positive real \\
    \hline
    \caption{config\_eos\_linear\_beta: Linear haline contraction coefficient}
\end{longtable}
\end{center}
\subsection[config\_eos\_linear\_Tref]{\hyperref[sec:nm_tab_eos_linear]{config\_eos\_linear\_Tref}}
\label{subsec:nm_sec_config_eos_linear_Tref}
\begin{center}
\begin{longtable}{| p{2.0in} || p{4.0in} |}
    \hline
    Type: & real \\
    \hline
    Units: & \si{C} \\
    \hline
    Default Value: & 5.0 \\
    \hline
    Possible Values: & any real \\
    \hline
    \caption{config\_eos\_linear\_Tref: Reference temperature}
\end{longtable}
\end{center}
\subsection[config\_eos\_linear\_Sref]{\hyperref[sec:nm_tab_eos_linear]{config\_eos\_linear\_Sref}}
\label{subsec:nm_sec_config_eos_linear_Sref}
\begin{center}
\begin{longtable}{| p{2.0in} || p{4.0in} |}
    \hline
    Type: & real \\
    \hline
    Units: & \si{1.e-3} \\
    \hline
    Default Value: & 35.0 \\
    \hline
    Possible Values: & any real \\
    \hline
    \caption{config\_eos\_linear\_Sref: Reference salinity}
\end{longtable}
\end{center}
\subsection[config\_eos\_linear\_densityref]{\hyperref[sec:nm_tab_eos_linear]{config\_eos\_linear\_densityref}}
\label{subsec:nm_sec_config_eos_linear_densityref}
\begin{center}
\begin{longtable}{| p{2.0in} || p{4.0in} |}
    \hline
    Type: & real \\
    \hline
    Units: & \si{kg.m^-3} \\
    \hline
    Default Value: & 1000.0 \\
    \hline
    Possible Values: & any positive real \\
    \hline
    \caption{config\_eos\_linear\_densityref: Reference density, i.e. density when T=Tref and S=Sref}
\end{longtable}
\end{center}
\section[eos\_wright]{\hyperref[sec:nm_tab_eos_wright]{eos\_wright}}
\label{sec:nm_sec_eos_wright}
\subsection[config\_eos\_wright\_ref\_pressure]{\hyperref[sec:nm_tab_eos_wright]{config\_eos\_wright\_ref\_pressure}}
\label{subsec:nm_sec_config_eos_wright_ref_pressure}
\begin{center}
\begin{longtable}{| p{2.0in} || p{4.0in} |}
    \hline
    Type: & real \\
    \hline
    Units: & \si{N.m^-2} \\
    \hline
    Default Value: & 0.0 \\
    \hline
    Possible Values: & any positive real \\
    \hline
    \caption{config\_eos\_wright\_ref\_pressure: Reference pressure for potential density}
\end{longtable}
\end{center}
\section[split\_timestep\_share]{\hyperref[sec:nm_tab_split_timestep_share]{split\_timestep\_share}}
\label{sec:nm_sec_split_timestep_share}
\subsection[config\_n\_ts\_iter]{\hyperref[sec:nm_tab_split_timestep_share]{config\_n\_ts\_iter}}
\label{subsec:nm_sec_config_n_ts_iter}
\begin{center}
\begin{longtable}{| p{2.0in} || p{4.0in} |}
    \hline
    Type: & integer \\
    \hline
    Units: & -- \\
    \hline
    Default Value: & 2 \\
    \hline
    Possible Values: & any positive integer, but typically 1, 2, or 3 \\
    \hline
    \caption{config\_n\_ts\_iter: number of large iterations over stages 1-3; For the split\_explicit\_ab2 time integrator, this value only affects the first time step when it is not a restart run. For restart runs, this value has no effect on the split\_explicit\_ab2 time integrator.}
\end{longtable}
\end{center}
\subsection[config\_n\_bcl\_iter\_beg]{\hyperref[sec:nm_tab_split_timestep_share]{config\_n\_bcl\_iter\_beg}}
\label{subsec:nm_sec_config_n_bcl_iter_beg}
\begin{center}
\begin{longtable}{| p{2.0in} || p{4.0in} |}
    \hline
    Type: & integer \\
    \hline
    Units: & -- \\
    \hline
    Default Value: & 1 \\
    \hline
    Possible Values: & any positive integer, but typically 1, 2, or 3 \\
    \hline
    \caption{config\_n\_bcl\_iter\_beg: number of iterations of stage 1 (baroclinic solve) on the first split timestepping iteration}
\end{longtable}
\end{center}
\subsection[config\_n\_bcl\_iter\_mid]{\hyperref[sec:nm_tab_split_timestep_share]{config\_n\_bcl\_iter\_mid}}
\label{subsec:nm_sec_config_n_bcl_iter_mid}
\begin{center}
\begin{longtable}{| p{2.0in} || p{4.0in} |}
    \hline
    Type: & integer \\
    \hline
    Units: & -- \\
    \hline
    Default Value: & 2 \\
    \hline
    Possible Values: & any positive integer, but typically 1, 2, or 3 \\
    \hline
    \caption{config\_n\_bcl\_iter\_mid: number of iterations of stage 1 (baroclinic solve) on any split timestepping iterations between first and last}
\end{longtable}
\end{center}
\subsection[config\_n\_bcl\_iter\_end]{\hyperref[sec:nm_tab_split_timestep_share]{config\_n\_bcl\_iter\_end}}
\label{subsec:nm_sec_config_n_bcl_iter_end}
\begin{center}
\begin{longtable}{| p{2.0in} || p{4.0in} |}
    \hline
    Type: & integer \\
    \hline
    Units: & -- \\
    \hline
    Default Value: & 2 \\
    \hline
    Possible Values: & any positive integer, but typically 1, 2, or 3 \\
    \hline
    \caption{config\_n\_bcl\_iter\_end: number of iterations of stage 1 (baroclinic solve) on the last split timestepping iteration}
\end{longtable}
\end{center}
\section[split\_explicit\_ts]{\hyperref[sec:nm_tab_split_explicit_ts]{split\_explicit\_ts}}
\label{sec:nm_sec_split_explicit_ts}
\subsection[config\_btr\_dt]{\hyperref[sec:nm_tab_split_explicit_ts]{config\_btr\_dt}}
\label{subsec:nm_sec_config_btr_dt}
\begin{center}
\begin{longtable}{| p{2.0in} || p{4.0in} |}
    \hline
    Type: & character \\
    \hline
    Units: & -- \\
    \hline
    Default Value: & 0000\_00:00:15 \\
    \hline
    Possible Values: & Any time stamp in 'YYYY-MM-DD\_hh:mm:ss' format. Items can be removed from the left if they are unused. \\
    \hline
    \caption{config\_btr\_dt: Timestep to use for the barotropic mode in the split explicit time integrator}
\end{longtable}
\end{center}
\subsection[config\_n\_btr\_cor\_iter]{\hyperref[sec:nm_tab_split_explicit_ts]{config\_n\_btr\_cor\_iter}}
\label{subsec:nm_sec_config_n_btr_cor_iter}
\begin{center}
\begin{longtable}{| p{2.0in} || p{4.0in} |}
    \hline
    Type: & integer \\
    \hline
    Units: & -- \\
    \hline
    Default Value: & 2 \\
    \hline
    Possible Values: & any positive integer, but typically 1, 2, or 3 \\
    \hline
    \caption{config\_n\_btr\_cor\_iter: number of iterations of the velocity corrector step in stage 2}
\end{longtable}
\end{center}
\subsection[config\_vel\_correction]{\hyperref[sec:nm_tab_split_explicit_ts]{config\_vel\_correction}}
\label{subsec:nm_sec_config_vel_correction}
\begin{center}
\begin{longtable}{| p{2.0in} || p{4.0in} |}
    \hline
    Type: & logical \\
    \hline
    Units: & -- \\
    \hline
    Default Value: & .true. \\
    \hline
    Possible Values: & .true. or .false. \\
    \hline
    \caption{config\_vel\_correction: If true, the velocity correction term is included in the horizontal advection of thickness and tracers}
\end{longtable}
\end{center}
\subsection[config\_btr\_subcycle\_loop\_factor]{\hyperref[sec:nm_tab_split_explicit_ts]{config\_btr\_subcycle\_loop\_factor}}
\label{subsec:nm_sec_config_btr_subcycle_loop_factor}
\begin{center}
\begin{longtable}{| p{2.0in} || p{4.0in} |}
    \hline
    Type: & integer \\
    \hline
    Units: & -- \\
    \hline
    Default Value: & 2 \\
    \hline
    Possible Values: & Any positive integer, but typically 1 or 2 \\
    \hline
    \caption{config\_btr\_subcycle\_loop\_factor: Barotropic subcycles proceed from $t$ to $t+n\Delta t$, where $n$ is this configuration option.}
\end{longtable}
\end{center}
\subsection[config\_btr\_gam1\_velWt1]{\hyperref[sec:nm_tab_split_explicit_ts]{config\_btr\_gam1\_velWt1}}
\label{subsec:nm_sec_config_btr_gam1_velWt1}
\begin{center}
\begin{longtable}{| p{2.0in} || p{4.0in} |}
    \hline
    Type: & real \\
    \hline
    Units: & -- \\
    \hline
    Default Value: & 0.5333 \\
    \hline
    Possible Values: & between 0 and 1 \\
    \hline
    \caption{config\_btr\_gam1\_velWt1: Weighting of velocity in the SSH predictor step in stage 2. When zero, previous subcycle time is used; when one, new subcycle time is used.}
\end{longtable}
\end{center}
\subsection[config\_btr\_gam2\_SSHWt1]{\hyperref[sec:nm_tab_split_explicit_ts]{config\_btr\_gam2\_SSHWt1}}
\label{subsec:nm_sec_config_btr_gam2_SSHWt1}
\begin{center}
\begin{longtable}{| p{2.0in} || p{4.0in} |}
    \hline
    Type: & real \\
    \hline
    Units: & -- \\
    \hline
    Default Value: & 0.5333 \\
    \hline
    Possible Values: & between 0 and 1 \\
    \hline
    \caption{config\_btr\_gam2\_SSHWt1: Weighting of SSH in the velocity corrector step in stage 2. When zero, previous subcycle time is used; when one, new subcycle time is used.}
\end{longtable}
\end{center}
\subsection[config\_btr\_gam3\_velWt2]{\hyperref[sec:nm_tab_split_explicit_ts]{config\_btr\_gam3\_velWt2}}
\label{subsec:nm_sec_config_btr_gam3_velWt2}
\begin{center}
\begin{longtable}{| p{2.0in} || p{4.0in} |}
    \hline
    Type: & real \\
    \hline
    Units: & -- \\
    \hline
    Default Value: & 1.0 \\
    \hline
    Possible Values: & between 0 and 1 \\
    \hline
    \caption{config\_btr\_gam3\_velWt2: Weighting of velocity in the SSH corrector step in stage 2. When zero, previous subcycle time is used; when one, new subcycle time is used.}
\end{longtable}
\end{center}
\subsection[config\_btr\_solve\_SSH2]{\hyperref[sec:nm_tab_split_explicit_ts]{config\_btr\_solve\_SSH2}}
\label{subsec:nm_sec_config_btr_solve_SSH2}
\begin{center}
\begin{longtable}{| p{2.0in} || p{4.0in} |}
    \hline
    Type: & logical \\
    \hline
    Units: & -- \\
    \hline
    Default Value: & .false. \\
    \hline
    Possible Values: & .true. or .false. \\
    \hline
    \caption{config\_btr\_solve\_SSH2: If true, execute the SSH corrector step in stage 2}
\end{longtable}
\end{center}
\section[split\_implicit\_ts]{\hyperref[sec:nm_tab_split_implicit_ts]{split\_implicit\_ts}}
\label{sec:nm_sec_split_implicit_ts}
\subsection[config\_btr\_si\_preconditioner]{\hyperref[sec:nm_tab_split_implicit_ts]{config\_btr\_si\_preconditioner}}
\label{subsec:nm_sec_config_btr_si_preconditioner}
\begin{center}
\begin{longtable}{| p{2.0in} || p{4.0in} |}
    \hline
    Type: & character \\
    \hline
    Units: & -- \\
    \hline
    Default Value: & ras \\
    \hline
    Possible Values: & ras, block\_jacobi, jacobi, none \\
    \hline
    \caption{config\_btr\_si\_preconditioner: Type of preconditioner for the barotropic mode solver}
\end{longtable}
\end{center}
\subsection[config\_btr\_si\_tolerance]{\hyperref[sec:nm_tab_split_implicit_ts]{config\_btr\_si\_tolerance}}
\label{subsec:nm_sec_config_btr_si_tolerance}
\begin{center}
\begin{longtable}{| p{2.0in} || p{4.0in} |}
    \hline
    Type: & real \\
    \hline
    Units: & -- \\
    \hline
    Default Value: & 1.0e-9 \\
    \hline
    Possible Values: & any positive real, but typically less than 1.0e-9 \\
    \hline
    \caption{config\_btr\_si\_tolerance: Tolerance for the barotropic mode solver}
\end{longtable}
\end{center}
\subsection[config\_n\_btr\_si\_large\_iter]{\hyperref[sec:nm_tab_split_implicit_ts]{config\_n\_btr\_si\_large\_iter}}
\label{subsec:nm_sec_config_n_btr_si_large_iter}
\begin{center}
\begin{longtable}{| p{2.0in} || p{4.0in} |}
    \hline
    Type: & integer \\
    \hline
    Units: & -- \\
    \hline
    Default Value: & 1 \\
    \hline
    Possible Values: & any positive integer, but typically 1 and less than 2 \\
    \hline
    \caption{config\_n\_btr\_si\_large\_iter: number of large barotropic system iterations}
\end{longtable}
\end{center}
\subsection[config\_btr\_si\_partition\_match\_mode]{\hyperref[sec:nm_tab_split_implicit_ts]{config\_btr\_si\_partition\_match\_mode}}
\label{subsec:nm_sec_config_btr_si_partition_match_mode}
\begin{center}
\begin{longtable}{| p{2.0in} || p{4.0in} |}
    \hline
    Type: & logical \\
    \hline
    Units: & -- \\
    \hline
    Default Value: & .false. \\
    \hline
    Possible Values: & .true. or .false. \\
    \hline
    \caption{config\_btr\_si\_partition\_match\_mode: If true, the split-implicit method uses the Jacobi preconditioner with the bit-for-bit all-reduce. This is less performant, so should only be used for testing.}
\end{longtable}
\end{center}
\section[ALE\_vertical\_grid]{\hyperref[sec:nm_tab_ALE_vertical_grid]{ALE\_vertical\_grid}}
\label{sec:nm_sec_ALE_vertical_grid}
\subsection[config\_vert\_coord\_movement]{\hyperref[sec:nm_tab_ALE_vertical_grid]{config\_vert\_coord\_movement}}
\label{subsec:nm_sec_config_vert_coord_movement}
\begin{center}
\begin{longtable}{| p{2.0in} || p{4.0in} |}
    \hline
    Type: & character \\
    \hline
    Units: & -- \\
    \hline
    Default Value: & uniform\_stretching \\
    \hline
    Possible Values: & 'uniform\_stretching', 'fixed', 'user\_specified', 'impermeable\_interfaces', 'tapered' \\
    \hline
    \caption{config\_vert\_coord\_movement: Determines the vertical coordinate movement type. 'uniform\_stretching' distributes SSH perturbations through all vertical levels (z-star vertical coordinate); 'fixed' places them all in the top level (z-level vertical coordinate); 'user\_specified' allows the input file to determine the distribution using the variable vertCoordMovementWeights (weighted z-star vertical coordinate); and 'impermeable\_interfaces' makes the vertical transport between layers zero, i.e. $w^t=0$ (idealized isopycnal).}
\end{longtable}
\end{center}
\subsection[config\_ALE\_thickness\_proportionality]{\hyperref[sec:nm_tab_ALE_vertical_grid]{config\_ALE\_thickness\_proportionality}}
\label{subsec:nm_sec_config_ALE_thickness_proportionality}
\begin{center}
\begin{longtable}{| p{2.0in} || p{4.0in} |}
    \hline
    Type: & character \\
    \hline
    Units: & -- \\
    \hline
    Default Value: & restingThickness\_times\_weights \\
    \hline
    Possible Values: & 'restingThickness\_times\_weights' or 'weights\_only' \\
    \hline
    \caption{config\_ALE\_thickness\_proportionality: When config\_vert\_coord\_movement='uniform\_stretching' (z-star type coordinate), determines whether ALE layer thickness is proportional to the resting thickness times weights, or just the weights. The first is standard for global runs and is what is specified in Petersen et al 2015 eqns 4 and 6. The second is useful for wetting/drying test cases where resting thickness may be zero at the coastlines.}
\end{longtable}
\end{center}
\subsection[config\_vert\_taper\_weight\_depth\_1]{\hyperref[sec:nm_tab_ALE_vertical_grid]{config\_vert\_taper\_weight\_depth\_1}}
\label{subsec:nm_sec_config_vert_taper_weight_depth_1}
\begin{center}
\begin{longtable}{| p{2.0in} || p{4.0in} |}
    \hline
    Type: & real \\
    \hline
    Units: & \si{m} \\
    \hline
    Default Value: & 250.0 \\
    \hline
    Possible Values: & any positive real value, but typically 100 to 1000 m. \\
    \hline
    \caption{config\_vert\_taper\_weight\_depth\_1: Vertical coordinate taper weight is one above this depth, linearly decreases to zero below.}
\end{longtable}
\end{center}
\subsection[config\_vert\_taper\_weight\_depth\_2]{\hyperref[sec:nm_tab_ALE_vertical_grid]{config\_vert\_taper\_weight\_depth\_2}}
\label{subsec:nm_sec_config_vert_taper_weight_depth_2}
\begin{center}
\begin{longtable}{| p{2.0in} || p{4.0in} |}
    \hline
    Type: & real \\
    \hline
    Units: & \si{m} \\
    \hline
    Default Value: & 500.0 \\
    \hline
    Possible Values: & any positive real value, but typically 100 to 1000 m and greater than config\_vert\_taper\_weight\_depth\_1. \\
    \hline
    \caption{config\_vert\_taper\_weight\_depth\_2: Vertical coordinate taper weight is zero below this depth, linearly increases to one above.}
\end{longtable}
\end{center}
\subsection[config\_use\_min\_max\_thickness]{\hyperref[sec:nm_tab_ALE_vertical_grid]{config\_use\_min\_max\_thickness}}
\label{subsec:nm_sec_config_use_min_max_thickness}
\begin{center}
\begin{longtable}{| p{2.0in} || p{4.0in} |}
    \hline
    Type: & logical \\
    \hline
    Units: & -- \\
    \hline
    Default Value: & .false. \\
    \hline
    Possible Values: & .true. or .false. \\
    \hline
    \caption{config\_use\_min\_max\_thickness: If true, a minimum and maximum thicknesses are enforced to prevent massless and very thick layers.}
\end{longtable}
\end{center}
\subsection[config\_min\_thickness]{\hyperref[sec:nm_tab_ALE_vertical_grid]{config\_min\_thickness}}
\label{subsec:nm_sec_config_min_thickness}
\begin{center}
\begin{longtable}{| p{2.0in} || p{4.0in} |}
    \hline
    Type: & real \\
    \hline
    Units: & \si{m} \\
    \hline
    Default Value: & 1.0 \\
    \hline
    Possible Values: & any positive real value, but typically 0.1 to 1 m. \\
    \hline
    \caption{config\_min\_thickness: Minimum thickness allowed.}
\end{longtable}
\end{center}
\subsection[config\_max\_thickness\_factor]{\hyperref[sec:nm_tab_ALE_vertical_grid]{config\_max\_thickness\_factor}}
\label{subsec:nm_sec_config_max_thickness_factor}
\begin{center}
\begin{longtable}{| p{2.0in} || p{4.0in} |}
    \hline
    Type: & real \\
    \hline
    Units: & -- \\
    \hline
    Default Value: & 6.0 \\
    \hline
    Possible Values: & any positive real value, but typically 2-4. \\
    \hline
    \caption{config\_max\_thickness\_factor: Maximum thickness allowed. This is a factor times the resting thickness, i.e., maximum thickness = config\_max\_thickness\_factor*$h^{rest}$.}
\end{longtable}
\end{center}
\subsection[config\_dzdk\_positive]{\hyperref[sec:nm_tab_ALE_vertical_grid]{config\_dzdk\_positive}}
\label{subsec:nm_sec_config_dzdk_positive}
\begin{center}
\begin{longtable}{| p{2.0in} || p{4.0in} |}
    \hline
    Type: & logical \\
    \hline
    Units: & -- \\
    \hline
    Default Value: & .false. \\
    \hline
    Possible Values: & .true. or .false. \\
    \hline
    \caption{config\_dzdk\_positive: Determines if the positive Z axis is aligned with the positive K index direction.}
\end{longtable}
\end{center}
\section[ALE\_frequency\_filtered\_thickness]{\hyperref[sec:nm_tab_ALE_frequency_filtered_thickness]{ALE\_frequency\_filtered\_thickness}}
\label{sec:nm_sec_ALE_frequency_filtered_thickness}
\subsection[config\_use\_freq\_filtered\_thickness]{\hyperref[sec:nm_tab_ALE_frequency_filtered_thickness]{config\_use\_freq\_filtered\_thickness}}
\label{subsec:nm_sec_config_use_freq_filtered_thickness}
\begin{center}
\begin{longtable}{| p{2.0in} || p{4.0in} |}
    \hline
    Type: & logical \\
    \hline
    Units: & -- \\
    \hline
    Default Value: & .false. \\
    \hline
    Possible Values: & .true. or .false. \\
    \hline
    \caption{config\_use\_freq\_filtered\_thickness: If true, $h^{hf}$ is included in the desired ALE thickness, and the prognostic equations for $D^{lf}$ and $h^{hf}$ are integrated in the code.}
\end{longtable}
\end{center}
\subsection[config\_thickness\_filter\_timescale]{\hyperref[sec:nm_tab_ALE_frequency_filtered_thickness]{config\_thickness\_filter\_timescale}}
\label{subsec:nm_sec_config_thickness_filter_timescale}
\begin{center}
\begin{longtable}{| p{2.0in} || p{4.0in} |}
    \hline
    Type: & real \\
    \hline
    Units: & \si{days} \\
    \hline
    Default Value: & 5.0 \\
    \hline
    Possible Values: & any positive real value, but typically 5 days. \\
    \hline
    \caption{config\_thickness\_filter\_timescale: Filter time scale for the low-frequency baroclinic divergence, $\tau_{Dlf}$.}
\end{longtable}
\end{center}
\subsection[config\_use\_highFreqThick\_restore]{\hyperref[sec:nm_tab_ALE_frequency_filtered_thickness]{config\_use\_highFreqThick\_restore}}
\label{subsec:nm_sec_config_use_highFreqThick_restore}
\begin{center}
\begin{longtable}{| p{2.0in} || p{4.0in} |}
    \hline
    Type: & logical \\
    \hline
    Units: & -- \\
    \hline
    Default Value: & .false. \\
    \hline
    Possible Values: & .true. or .false. \\
    \hline
    \caption{config\_use\_highFreqThick\_restore: If true, the high frequency thickness prognostic equation ($h^{hf}$) includes term 2 on the RHS, the restoring term.  The high frequency thickness is restored to zero with time scale $\tau_{hhf}$.}
\end{longtable}
\end{center}
\subsection[config\_highFreqThick\_restore\_time]{\hyperref[sec:nm_tab_ALE_frequency_filtered_thickness]{config\_highFreqThick\_restore\_time}}
\label{subsec:nm_sec_config_highFreqThick_restore_time}
\begin{center}
\begin{longtable}{| p{2.0in} || p{4.0in} |}
    \hline
    Type: & real \\
    \hline
    Units: & \si{days} \\
    \hline
    Default Value: & 30.0 \\
    \hline
    Possible Values: & any positive real value, but typically 5-30 days. \\
    \hline
    \caption{config\_highFreqThick\_restore\_time: Restoring time scale for high-frequency thickness, $\tau_{hhf}$.}
\end{longtable}
\end{center}
\subsection[config\_use\_highFreqThick\_del2]{\hyperref[sec:nm_tab_ALE_frequency_filtered_thickness]{config\_use\_highFreqThick\_del2}}
\label{subsec:nm_sec_config_use_highFreqThick_del2}
\begin{center}
\begin{longtable}{| p{2.0in} || p{4.0in} |}
    \hline
    Type: & logical \\
    \hline
    Units: & -- \\
    \hline
    Default Value: & .false. \\
    \hline
    Possible Values: & .true. or .false. \\
    \hline
    \caption{config\_use\_highFreqThick\_del2: If true, high frequency thickness prognostic equation ($h^{hf}$) includes term 3 on the RHS, the diffusion term.}
\end{longtable}
\end{center}
\subsection[config\_highFreqThick\_del2]{\hyperref[sec:nm_tab_ALE_frequency_filtered_thickness]{config\_highFreqThick\_del2}}
\label{subsec:nm_sec_config_highFreqThick_del2}
\begin{center}
\begin{longtable}{| p{2.0in} || p{4.0in} |}
    \hline
    Type: & real \\
    \hline
    Units: & \si{m^2.s^-1} \\
    \hline
    Default Value: & 100.0 \\
    \hline
    Possible Values: & any positive real \\
    \hline
    \caption{config\_highFreqThick\_del2: Horizontal high frequency thickness diffusion, $\kappa_{hhf}$.}
\end{longtable}
\end{center}
\section[debug]{\hyperref[sec:nm_tab_debug]{debug}}
\label{sec:nm_sec_debug}
\subsection[config\_check\_zlevel\_consistency]{\hyperref[sec:nm_tab_debug]{config\_check\_zlevel\_consistency}}
\label{subsec:nm_sec_config_check_zlevel_consistency}
\begin{center}
\begin{longtable}{| p{2.0in} || p{4.0in} |}
    \hline
    Type: & logical \\
    \hline
    Units: & -- \\
    \hline
    Default Value: & .false. \\
    \hline
    Possible Values: & .true. or .false. \\
    \hline
    \caption{config\_check\_zlevel\_consistency: Enables a run-time check for consistency for a zlevel grid. Ensures relevant variables correctly define the bottom of the ocean.}
\end{longtable}
\end{center}
\subsection[config\_check\_ssh\_consistency]{\hyperref[sec:nm_tab_debug]{config\_check\_ssh\_consistency}}
\label{subsec:nm_sec_config_check_ssh_consistency}
\begin{center}
\begin{longtable}{| p{2.0in} || p{4.0in} |}
    \hline
    Type: & logical \\
    \hline
    Units: & -- \\
    \hline
    Default Value: & .true. \\
    \hline
    Possible Values: & .true. or .false. \\
    \hline
    \caption{config\_check\_ssh\_consistency: Enables a run-time check to determine if the SSH is within 2m of the surface.  See equation for $\zeta_i$.}
\end{longtable}
\end{center}
\subsection[config\_filter\_btr\_mode]{\hyperref[sec:nm_tab_debug]{config\_filter\_btr\_mode}}
\label{subsec:nm_sec_config_filter_btr_mode}
\begin{center}
\begin{longtable}{| p{2.0in} || p{4.0in} |}
    \hline
    Type: & logical \\
    \hline
    Units: & -- \\
    \hline
    Default Value: & .false. \\
    \hline
    Possible Values: & .true. or .false. \\
    \hline
    \caption{config\_filter\_btr\_mode: Enables filtering of the barotropic mode.}
\end{longtable}
\end{center}
\subsection[config\_prescribe\_velocity]{\hyperref[sec:nm_tab_debug]{config\_prescribe\_velocity}}
\label{subsec:nm_sec_config_prescribe_velocity}
\begin{center}
\begin{longtable}{| p{2.0in} || p{4.0in} |}
    \hline
    Type: & logical \\
    \hline
    Units: & -- \\
    \hline
    Default Value: & .false. \\
    \hline
    Possible Values: & .true. or .false. \\
    \hline
    \caption{config\_prescribe\_velocity: Enables a prescribed velocity field. This velocity field is read on input, and remains constant through a simulation.}
\end{longtable}
\end{center}
\subsection[config\_prescribe\_thickness]{\hyperref[sec:nm_tab_debug]{config\_prescribe\_thickness}}
\label{subsec:nm_sec_config_prescribe_thickness}
\begin{center}
\begin{longtable}{| p{2.0in} || p{4.0in} |}
    \hline
    Type: & logical \\
    \hline
    Units: & -- \\
    \hline
    Default Value: & .false. \\
    \hline
    Possible Values: & .true. or .false. \\
    \hline
    \caption{config\_prescribe\_thickness: Enables a prescribed thickness field. This thickness field is read on input, and remains constant through a simulation.}
\end{longtable}
\end{center}
\subsection[config\_include\_KE\_vertex]{\hyperref[sec:nm_tab_debug]{config\_include\_KE\_vertex}}
\label{subsec:nm_sec_config_include_KE_vertex}
\begin{center}
\begin{longtable}{| p{2.0in} || p{4.0in} |}
    \hline
    Type: & logical \\
    \hline
    Units: & -- \\
    \hline
    Default Value: & .false. \\
    \hline
    Possible Values: & .true. or .false. \\
    \hline
    \caption{config\_include\_KE\_vertex: If true, the kinetic energy in each cell is computed by blending cell-based and vertex-based values of kinetic energy.}
\end{longtable}
\end{center}
\subsection[config\_check\_tracer\_monotonicity]{\hyperref[sec:nm_tab_debug]{config\_check\_tracer\_monotonicity}}
\label{subsec:nm_sec_config_check_tracer_monotonicity}
\begin{center}
\begin{longtable}{| p{2.0in} || p{4.0in} |}
    \hline
    Type: & logical \\
    \hline
    Units: & -- \\
    \hline
    Default Value: & .false. \\
    \hline
    Possible Values: & .true. or .false. \\
    \hline
    \caption{config\_check\_tracer\_monotonicity: Enables a change on tracer monotonicity at the end of the monotonic advection routine. Only used if config\_flux\_limiter is set to monotonic}
\end{longtable}
\end{center}
\subsection[config\_compute\_active\_tracer\_budgets]{\hyperref[sec:nm_tab_debug]{config\_compute\_active\_tracer\_budgets}}
\label{subsec:nm_sec_config_compute_active_tracer_budgets}
\begin{center}
\begin{longtable}{| p{2.0in} || p{4.0in} |}
    \hline
    Type: & logical \\
    \hline
    Units: & -- \\
    \hline
    Default Value: & .true. \\
    \hline
    Possible Values: & .true. or .false. \\
    \hline
    \caption{config\_compute\_active\_tracer\_budgets: Enables the computation of tracer budget terms}
\end{longtable}
\end{center}
\subsection[config\_disable\_thick\_all\_tend]{\hyperref[sec:nm_tab_debug]{config\_disable\_thick\_all\_tend}}
\label{subsec:nm_sec_config_disable_thick_all_tend}
\begin{center}
\begin{longtable}{| p{2.0in} || p{4.0in} |}
    \hline
    Type: & logical \\
    \hline
    Units: & -- \\
    \hline
    Default Value: & .false. \\
    \hline
    Possible Values: & .true. or .false. \\
    \hline
    \caption{config\_disable\_thick\_all\_tend: Disables all tendencies on the thickness field.}
\end{longtable}
\end{center}
\subsection[config\_disable\_thick\_hadv]{\hyperref[sec:nm_tab_debug]{config\_disable\_thick\_hadv}}
\label{subsec:nm_sec_config_disable_thick_hadv}
\begin{center}
\begin{longtable}{| p{2.0in} || p{4.0in} |}
    \hline
    Type: & logical \\
    \hline
    Units: & -- \\
    \hline
    Default Value: & .false. \\
    \hline
    Possible Values: & .true. or .false. \\
    \hline
    \caption{config\_disable\_thick\_hadv: Disable tendencies on the thickness field from horizontal advection.}
\end{longtable}
\end{center}
\subsection[config\_disable\_thick\_vadv]{\hyperref[sec:nm_tab_debug]{config\_disable\_thick\_vadv}}
\label{subsec:nm_sec_config_disable_thick_vadv}
\begin{center}
\begin{longtable}{| p{2.0in} || p{4.0in} |}
    \hline
    Type: & logical \\
    \hline
    Units: & -- \\
    \hline
    Default Value: & .false. \\
    \hline
    Possible Values: & .true. or .false. \\
    \hline
    \caption{config\_disable\_thick\_vadv: Disables tendencies on the thickness field from vertical advection.}
\end{longtable}
\end{center}
\subsection[config\_disable\_thick\_sflux]{\hyperref[sec:nm_tab_debug]{config\_disable\_thick\_sflux}}
\label{subsec:nm_sec_config_disable_thick_sflux}
\begin{center}
\begin{longtable}{| p{2.0in} || p{4.0in} |}
    \hline
    Type: & logical \\
    \hline
    Units: & -- \\
    \hline
    Default Value: & .false. \\
    \hline
    Possible Values: & .true. or .false. \\
    \hline
    \caption{config\_disable\_thick\_sflux: Disables tendencies on the thickness field from surface fluxes.}
\end{longtable}
\end{center}
\subsection[config\_disable\_vel\_all\_tend]{\hyperref[sec:nm_tab_debug]{config\_disable\_vel\_all\_tend}}
\label{subsec:nm_sec_config_disable_vel_all_tend}
\begin{center}
\begin{longtable}{| p{2.0in} || p{4.0in} |}
    \hline
    Type: & logical \\
    \hline
    Units: & -- \\
    \hline
    Default Value: & .false. \\
    \hline
    Possible Values: & .true. or .false. \\
    \hline
    \caption{config\_disable\_vel\_all\_tend: Disables all tendencies on the velocity field.}
\end{longtable}
\end{center}
\subsection[config\_disable\_vel\_hadv]{\hyperref[sec:nm_tab_debug]{config\_disable\_vel\_hadv}}
\label{subsec:nm_sec_config_disable_vel_hadv}
\begin{center}
\begin{longtable}{| p{2.0in} || p{4.0in} |}
    \hline
    Type: & logical \\
    \hline
    Units: & -- \\
    \hline
    Default Value: & .false. \\
    \hline
    Possible Values: & .true. or .false. \\
    \hline
    \caption{config\_disable\_vel\_hadv: Disables tendencies on the velocity field from the horizontal momentum advection. Note that these two flags are set together for linearized test cases: config\_thickness\_flux\_type = 'constant' linearizes the thickness equation, and config\_disable\_vel\_hadv = .true. linearizes the momentum equation if there is no assumed mean background velocity.}
\end{longtable}
\end{center}
\subsection[config\_disable\_vel\_coriolis]{\hyperref[sec:nm_tab_debug]{config\_disable\_vel\_coriolis}}
\label{subsec:nm_sec_config_disable_vel_coriolis}
\begin{center}
\begin{longtable}{| p{2.0in} || p{4.0in} |}
    \hline
    Type: & logical \\
    \hline
    Units: & -- \\
    \hline
    Default Value: & .false. \\
    \hline
    Possible Values: & .true. or .false. \\
    \hline
    \caption{config\_disable\_vel\_coriolis: Disables tendencies on the velocity field from the Coriolis force.}
\end{longtable}
\end{center}
\subsection[config\_disable\_vel\_pgrad]{\hyperref[sec:nm_tab_debug]{config\_disable\_vel\_pgrad}}
\label{subsec:nm_sec_config_disable_vel_pgrad}
\begin{center}
\begin{longtable}{| p{2.0in} || p{4.0in} |}
    \hline
    Type: & logical \\
    \hline
    Units: & -- \\
    \hline
    Default Value: & .false. \\
    \hline
    Possible Values: & .true. or .false. \\
    \hline
    \caption{config\_disable\_vel\_pgrad: Disables tendencies on the velocity field from the horizontal pressure gradient.}
\end{longtable}
\end{center}
\subsection[config\_disable\_vel\_hmix]{\hyperref[sec:nm_tab_debug]{config\_disable\_vel\_hmix}}
\label{subsec:nm_sec_config_disable_vel_hmix}
\begin{center}
\begin{longtable}{| p{2.0in} || p{4.0in} |}
    \hline
    Type: & logical \\
    \hline
    Units: & -- \\
    \hline
    Default Value: & .false. \\
    \hline
    Possible Values: & .true. or .false. \\
    \hline
    \caption{config\_disable\_vel\_hmix: Disables tendencies on the velocity field from horizontal mixing.}
\end{longtable}
\end{center}
\subsection[config\_disable\_vel\_surface\_stress]{\hyperref[sec:nm_tab_debug]{config\_disable\_vel\_surface\_stress}}
\label{subsec:nm_sec_config_disable_vel_surface_stress}
\begin{center}
\begin{longtable}{| p{2.0in} || p{4.0in} |}
    \hline
    Type: & logical \\
    \hline
    Units: & -- \\
    \hline
    Default Value: & .false. \\
    \hline
    Possible Values: & .true. or .false. \\
    \hline
    \caption{config\_disable\_vel\_surface\_stress: Disables tendencies on the velocity field from horizontal surface stresses (e.g. wind stress and top drag).}
\end{longtable}
\end{center}
\subsection[config\_disable\_vel\_topographic\_wave\_drag]{\hyperref[sec:nm_tab_debug]{config\_disable\_vel\_topographic\_wave\_drag}}
\label{subsec:nm_sec_config_disable_vel_topographic_wave_drag}
\begin{center}
\begin{longtable}{| p{2.0in} || p{4.0in} |}
    \hline
    Type: & logical \\
    \hline
    Units: & -- \\
    \hline
    Default Value: & .false. \\
    \hline
    Possible Values: & .true. or .false. \\
    \hline
    \caption{config\_disable\_vel\_topographic\_wave\_drag: Disables tendencies on the velocity field from topographic wave drag}
\end{longtable}
\end{center}
\subsection[config\_disable\_vel\_explicit\_bottom\_drag]{\hyperref[sec:nm_tab_debug]{config\_disable\_vel\_explicit\_bottom\_drag}}
\label{subsec:nm_sec_config_disable_vel_explicit_bottom_drag}
\begin{center}
\begin{longtable}{| p{2.0in} || p{4.0in} |}
    \hline
    Type: & logical \\
    \hline
    Units: & -- \\
    \hline
    Default Value: & .false. \\
    \hline
    Possible Values: & .true. or .false. \\
    \hline
    \caption{config\_disable\_vel\_explicit\_bottom\_drag: Disables tendencies on the velocity field from explicit bottom drag}
\end{longtable}
\end{center}
\subsection[config\_disable\_vel\_vmix]{\hyperref[sec:nm_tab_debug]{config\_disable\_vel\_vmix}}
\label{subsec:nm_sec_config_disable_vel_vmix}
\begin{center}
\begin{longtable}{| p{2.0in} || p{4.0in} |}
    \hline
    Type: & logical \\
    \hline
    Units: & -- \\
    \hline
    Default Value: & .false. \\
    \hline
    Possible Values: & .true. or .false. \\
    \hline
    \caption{config\_disable\_vel\_vmix: Disables tendencies on the velocity field from vertical mixing.}
\end{longtable}
\end{center}
\subsection[config\_disable\_vel\_vadv]{\hyperref[sec:nm_tab_debug]{config\_disable\_vel\_vadv}}
\label{subsec:nm_sec_config_disable_vel_vadv}
\begin{center}
\begin{longtable}{| p{2.0in} || p{4.0in} |}
    \hline
    Type: & logical \\
    \hline
    Units: & -- \\
    \hline
    Default Value: & .false. \\
    \hline
    Possible Values: & .true. or .false. \\
    \hline
    \caption{config\_disable\_vel\_vadv: Disables tendencies on the velocity field from vertical advection.}
\end{longtable}
\end{center}
\subsection[config\_disable\_tr\_all\_tend]{\hyperref[sec:nm_tab_debug]{config\_disable\_tr\_all\_tend}}
\label{subsec:nm_sec_config_disable_tr_all_tend}
\begin{center}
\begin{longtable}{| p{2.0in} || p{4.0in} |}
    \hline
    Type: & logical \\
    \hline
    Units: & -- \\
    \hline
    Default Value: & .false. \\
    \hline
    Possible Values: & .true. or .false. \\
    \hline
    \caption{config\_disable\_tr\_all\_tend: Disables all tendencies on tracer fields.}
\end{longtable}
\end{center}
\subsection[config\_disable\_tr\_adv]{\hyperref[sec:nm_tab_debug]{config\_disable\_tr\_adv}}
\label{subsec:nm_sec_config_disable_tr_adv}
\begin{center}
\begin{longtable}{| p{2.0in} || p{4.0in} |}
    \hline
    Type: & logical \\
    \hline
    Units: & -- \\
    \hline
    Default Value: & .false. \\
    \hline
    Possible Values: & .true. or .false. \\
    \hline
    \caption{config\_disable\_tr\_adv: Disables tendencies on tracer fields from advection, both horizontal and vertical.}
\end{longtable}
\end{center}
\subsection[config\_disable\_tr\_hmix]{\hyperref[sec:nm_tab_debug]{config\_disable\_tr\_hmix}}
\label{subsec:nm_sec_config_disable_tr_hmix}
\begin{center}
\begin{longtable}{| p{2.0in} || p{4.0in} |}
    \hline
    Type: & logical \\
    \hline
    Units: & -- \\
    \hline
    Default Value: & .false. \\
    \hline
    Possible Values: & .true. or .false. \\
    \hline
    \caption{config\_disable\_tr\_hmix: Disables tendencies on tracer fields from horizontal mixing.}
\end{longtable}
\end{center}
\subsection[config\_disable\_tr\_vmix]{\hyperref[sec:nm_tab_debug]{config\_disable\_tr\_vmix}}
\label{subsec:nm_sec_config_disable_tr_vmix}
\begin{center}
\begin{longtable}{| p{2.0in} || p{4.0in} |}
    \hline
    Type: & logical \\
    \hline
    Units: & -- \\
    \hline
    Default Value: & .false. \\
    \hline
    Possible Values: & .true. or .false. \\
    \hline
    \caption{config\_disable\_tr\_vmix: Disables tendencies on tracer fields from vertical mixing.}
\end{longtable}
\end{center}
\subsection[config\_disable\_tr\_sflux]{\hyperref[sec:nm_tab_debug]{config\_disable\_tr\_sflux}}
\label{subsec:nm_sec_config_disable_tr_sflux}
\begin{center}
\begin{longtable}{| p{2.0in} || p{4.0in} |}
    \hline
    Type: & logical \\
    \hline
    Units: & -- \\
    \hline
    Default Value: & .false. \\
    \hline
    Possible Values: & .true. or .false. \\
    \hline
    \caption{config\_disable\_tr\_sflux: Disables tendencies on tracer fields from surface fluxes.}
\end{longtable}
\end{center}
\subsection[config\_disable\_tr\_nonlocalflux]{\hyperref[sec:nm_tab_debug]{config\_disable\_tr\_nonlocalflux}}
\label{subsec:nm_sec_config_disable_tr_nonlocalflux}
\begin{center}
\begin{longtable}{| p{2.0in} || p{4.0in} |}
    \hline
    Type: & logical \\
    \hline
    Units: & -- \\
    \hline
    Default Value: & .false. \\
    \hline
    Possible Values: & .true. or .false. \\
    \hline
    \caption{config\_disable\_tr\_nonlocalflux: Disables tendencies on the tracer fields from CVMix/KPP nonlocal fluxes.}
\end{longtable}
\end{center}
\subsection[config\_disable\_redi\_k33]{\hyperref[sec:nm_tab_debug]{config\_disable\_redi\_k33}}
\label{subsec:nm_sec_config_disable_redi_k33}
\begin{center}
\begin{longtable}{| p{2.0in} || p{4.0in} |}
    \hline
    Type: & logical \\
    \hline
    Units: & -- \\
    \hline
    Default Value: & .false. \\
    \hline
    Possible Values: & .true. or .false. \\
    \hline
    \caption{config\_disable\_redi\_k33: If true, disables k33 portion of Redi neutral surface mixing.}
\end{longtable}
\end{center}
\subsection[config\_read\_nearest\_restart]{\hyperref[sec:nm_tab_debug]{config\_read\_nearest\_restart}}
\label{subsec:nm_sec_config_read_nearest_restart}
\begin{center}
\begin{longtable}{| p{2.0in} || p{4.0in} |}
    \hline
    Type: & logical \\
    \hline
    Units: & -- \\
    \hline
    Default Value: & .false. \\
    \hline
    Possible Values: & .true. or .false. \\
    \hline
    \caption{config\_read\_nearest\_restart: This flag is intended for the expert user.  If false, forward model will error out if time given by config\_start\_time (or Restart\_timestamp file if config\_start\_time='file') does not match any xtime strings in the restart file.  If true, forward model will read in record with xtime nearest to config\_start\_time.  Note that the restart file name is still given by config\_start\_time (or Restart\_timestamp file), regardless of the state of this flag.}
\end{longtable}
\end{center}
\section[testing]{\hyperref[sec:nm_tab_testing]{testing}}
\label{sec:nm_sec_testing}
\subsection[config\_conduct\_tests]{\hyperref[sec:nm_tab_testing]{config\_conduct\_tests}}
\label{subsec:nm_sec_config_conduct_tests}
\begin{center}
\begin{longtable}{| p{2.0in} || p{4.0in} |}
    \hline
    Type: & logical \\
    \hline
    Units: & -- \\
    \hline
    Default Value: & .false. \\
    \hline
    Possible Values: & .true. or .false. \\
    \hline
    \caption{config\_conduct\_tests: If true, run testing suite. This is the overarching control on the test suite. Individual flags must be set to true below to conduct each test.}
\end{longtable}
\end{center}
\subsection[config\_test\_tensors]{\hyperref[sec:nm_tab_testing]{config\_test\_tensors}}
\label{subsec:nm_sec_config_test_tensors}
\begin{center}
\begin{longtable}{| p{2.0in} || p{4.0in} |}
    \hline
    Type: & logical \\
    \hline
    Units: & -- \\
    \hline
    Default Value: & .false. \\
    \hline
    Possible Values: & .true. or .false. \\
    \hline
    \caption{config\_test\_tensors: If true, tensor operations are tested upon start-up.}
\end{longtable}
\end{center}
\subsection[config\_tensor\_test\_function]{\hyperref[sec:nm_tab_testing]{config\_tensor\_test\_function}}
\label{subsec:nm_sec_config_tensor_test_function}
\begin{center}
\begin{longtable}{| p{2.0in} || p{4.0in} |}
    \hline
    Type: & character \\
    \hline
    Units: & -- \\
    \hline
    Default Value: & sph\_uCosCos \\
    \hline
    Possible Values: & 'linear\_x', 'linear\_y', 'linear\_arb\_rot', 'power\_x', 'power\_y', 'power\_arb\_rot', 'sin\_arb\_rot', 'sph\_solid\_body', 'sph\_Williamson', 'sph\_uCosCos', 'sph\_vCosCos', 'sph\_ELonLon\_CosCos', 'sph\_ELatLat\_CosCos', 'sph\_ELonLat\_CosCos' \\
    \hline
    \caption{config\_tensor\_test\_function: Character string to choose tensor test fuction}
\end{longtable}
\end{center}
\section[transport\_tests]{\hyperref[sec:nm_tab_transport_tests]{transport\_tests}}
\label{sec:nm_sec_transport_tests}
\subsection[config\_transport\_tests\_vert\_levels]{\hyperref[sec:nm_tab_transport_tests]{config\_transport\_tests\_vert\_levels}}
\label{subsec:nm_sec_config_transport_tests_vert_levels}
\begin{center}
\begin{longtable}{| p{2.0in} || p{4.0in} |}
    \hline
    Type: & integer \\
    \hline
    Units: & -- \\
    \hline
    Default Value: & 3 \\
    \hline
    Possible Values: & Any positive integer number greater than 0. \\
    \hline
    \caption{config\_transport\_tests\_vert\_levels: Number of vertical levels in transport\_tests. Typical value is 3 for 2D tests.}
\end{longtable}
\end{center}
\subsection[config\_transport\_tests\_temperature]{\hyperref[sec:nm_tab_transport_tests]{config\_transport\_tests\_temperature}}
\label{subsec:nm_sec_config_transport_tests_temperature}
\begin{center}
\begin{longtable}{| p{2.0in} || p{4.0in} |}
    \hline
    Type: & real \\
    \hline
    Units: & \si{deg.C} \\
    \hline
    Default Value: & 15.0 \\
    \hline
    Possible Values: & Any real number \\
    \hline
    \caption{config\_transport\_tests\_temperature: Temperature of the ocean.}
\end{longtable}
\end{center}
\subsection[config\_transport\_tests\_salinity]{\hyperref[sec:nm_tab_transport_tests]{config\_transport\_tests\_salinity}}
\label{subsec:nm_sec_config_transport_tests_salinity}
\begin{center}
\begin{longtable}{| p{2.0in} || p{4.0in} |}
    \hline
    Type: & real \\
    \hline
    Units: & \si{1.e-3} \\
    \hline
    Default Value: & 35.0 \\
    \hline
    Possible Values: & Any real number \\
    \hline
    \caption{config\_transport\_tests\_salinity: Salinity of the ocean.}
\end{longtable}
\end{center}
\subsection[config\_transport\_tests\_flow\_id]{\hyperref[sec:nm_tab_transport_tests]{config\_transport\_tests\_flow\_id}}
\label{subsec:nm_sec_config_transport_tests_flow_id}
\begin{center}
\begin{longtable}{| p{2.0in} || p{4.0in} |}
    \hline
    Type: & integer \\
    \hline
    Units: & -- \\
    \hline
    Default Value: & 0 \\
    \hline
    Possible Values: & 1 = rotation, 2 = nondivergent2D, 3 = divergent2D, 4 = correlatedTracers2D \\
    \hline
    \caption{config\_transport\_tests\_flow\_id: integer id of transport test.}
\end{longtable}
\end{center}
\section[init\_mode\_vert\_levels]{\hyperref[sec:nm_tab_init_mode_vert_levels]{init\_mode\_vert\_levels}}
\label{sec:nm_sec_init_mode_vert_levels}
\subsection[config\_vert\_levels]{\hyperref[sec:nm_tab_init_mode_vert_levels]{config\_vert\_levels}}
\label{subsec:nm_sec_config_vert_levels}
\begin{center}
\begin{longtable}{| p{2.0in} || p{4.0in} |}
    \hline
    Type: & integer \\
    \hline
    Units: & -- \\
    \hline
    Default Value: & -1 \\
    \hline
    Possible Values: & Any positive non-zero integer. A value of -1 causes this to be overwritten with the configurations vertical level definition. \\
    \hline
    \caption{config\_vert\_levels: Number of vertical levels to create within the configuration.}
\end{longtable}
\end{center}
\section[manufactured\_solution]{\hyperref[sec:nm_tab_manufactured_solution]{manufactured\_solution}}
\label{sec:nm_sec_manufactured_solution}
\subsection[config\_use\_manufactured\_solution]{\hyperref[sec:nm_tab_manufactured_solution]{config\_use\_manufactured\_solution}}
\label{subsec:nm_sec_config_use_manufactured_solution}
\begin{center}
\begin{longtable}{| p{2.0in} || p{4.0in} |}
    \hline
    Type: & logical \\
    \hline
    Units: & -- \\
    \hline
    Default Value: & .false. \\
    \hline
    Possible Values: & .true. or .false. \\
    \hline
    \caption{config\_use\_manufactured\_solution: This flag includes additional thickness and velocity tendencies necessary for testing with a manufactured solution.}
\end{longtable}
\end{center}
\subsection[config\_manufactured\_solution\_wavelength\_x]{\hyperref[sec:nm_tab_manufactured_solution]{config\_manufactured\_solution\_wavelength\_x}}
\label{subsec:nm_sec_config_manufactured_solution_wavelength_x}
\begin{center}
\begin{longtable}{| p{2.0in} || p{4.0in} |}
    \hline
    Type: & real \\
    \hline
    Units: & \si{m} \\
    \hline
    Default Value: & 2000000.0 \\
    \hline
    Possible Values: & Any positive real number \\
    \hline
    \caption{config\_manufactured\_solution\_wavelength\_x: Wavelength of manufactured solution in the x direction}
\end{longtable}
\end{center}
\subsection[config\_manufactured\_solution\_wavelength\_y]{\hyperref[sec:nm_tab_manufactured_solution]{config\_manufactured\_solution\_wavelength\_y}}
\label{subsec:nm_sec_config_manufactured_solution_wavelength_y}
\begin{center}
\begin{longtable}{| p{2.0in} || p{4.0in} |}
    \hline
    Type: & real \\
    \hline
    Units: & \si{m} \\
    \hline
    Default Value: & 2000000.0 \\
    \hline
    Possible Values: & Any positive real number \\
    \hline
    \caption{config\_manufactured\_solution\_wavelength\_y: Wavelength of manufactured solution in the y direction}
\end{longtable}
\end{center}
\subsection[config\_manufactured\_solution\_amplitude]{\hyperref[sec:nm_tab_manufactured_solution]{config\_manufactured\_solution\_amplitude}}
\label{subsec:nm_sec_config_manufactured_solution_amplitude}
\begin{center}
\begin{longtable}{| p{2.0in} || p{4.0in} |}
    \hline
    Type: & real \\
    \hline
    Units: & \si{m} \\
    \hline
    Default Value: & 1 \\
    \hline
    Possible Values: & Any positive real number \\
    \hline
    \caption{config\_manufactured\_solution\_amplitude: Amplitude of the manufactured solution}
\end{longtable}
\end{center}
\section[tracer\_forcing\_activeTracers]{\hyperref[sec:nm_tab_tracer_forcing_activeTracers]{tracer\_forcing\_activeTracers}}
\label{sec:nm_sec_tracer_forcing_activeTracers}
\subsection[config\_use\_activeTracers]{\hyperref[sec:nm_tab_tracer_forcing_activeTracers]{config\_use\_activeTracers}}
\label{subsec:nm_sec_config_use_activeTracers}
\begin{center}
\begin{longtable}{| p{2.0in} || p{4.0in} |}
    \hline
    Type: & logical \\
    \hline
    Units: & -- \\
    \hline
    Default Value: & .true. \\
    \hline
    Possible Values: & .true. or .false. \\
    \hline
    \caption{config\_use\_activeTracers: if true, the 'activeTracers' category is enabled for the run}
\end{longtable}
\end{center}
\subsection[config\_use\_activeTracers\_surface\_bulk\_forcing]{\hyperref[sec:nm_tab_tracer_forcing_activeTracers]{config\_use\_activeTracers\_surface\_bulk\_forcing}}
\label{subsec:nm_sec_config_use_activeTracers_surface_bulk_forcing}
\begin{center}
\begin{longtable}{| p{2.0in} || p{4.0in} |}
    \hline
    Type: & logical \\
    \hline
    Units: & -- \\
    \hline
    Default Value: & .false. \\
    \hline
    Possible Values: & .true. or .false. \\
    \hline
    \caption{config\_use\_activeTracers\_surface\_bulk\_forcing: if true, surface bulk forcing from coupler is added to surfaceTracerFlux in 'activeTracers' category}
\end{longtable}
\end{center}
\subsection[config\_use\_activeTracers\_surface\_restoring]{\hyperref[sec:nm_tab_tracer_forcing_activeTracers]{config\_use\_activeTracers\_surface\_restoring}}
\label{subsec:nm_sec_config_use_activeTracers_surface_restoring}
\begin{center}
\begin{longtable}{| p{2.0in} || p{4.0in} |}
    \hline
    Type: & logical \\
    \hline
    Units: & -- \\
    \hline
    Default Value: & .false. \\
    \hline
    Possible Values: & .true. or .false. \\
    \hline
    \caption{config\_use\_activeTracers\_surface\_restoring: if true, surface restoring source is applied to tracers in 'activeTracers' category}
\end{longtable}
\end{center}
\subsection[config\_use\_activeTracers\_interior\_restoring]{\hyperref[sec:nm_tab_tracer_forcing_activeTracers]{config\_use\_activeTracers\_interior\_restoring}}
\label{subsec:nm_sec_config_use_activeTracers_interior_restoring}
\begin{center}
\begin{longtable}{| p{2.0in} || p{4.0in} |}
    \hline
    Type: & logical \\
    \hline
    Units: & -- \\
    \hline
    Default Value: & .false. \\
    \hline
    Possible Values: & .true. or .false. \\
    \hline
    \caption{config\_use\_activeTracers\_interior\_restoring: if true, interior restoring source is applied to tracers in 'activeTracers' category}
\end{longtable}
\end{center}
\subsection[config\_use\_activeTracers\_exponential\_decay]{\hyperref[sec:nm_tab_tracer_forcing_activeTracers]{config\_use\_activeTracers\_exponential\_decay}}
\label{subsec:nm_sec_config_use_activeTracers_exponential_decay}
\begin{center}
\begin{longtable}{| p{2.0in} || p{4.0in} |}
    \hline
    Type: & logical \\
    \hline
    Units: & -- \\
    \hline
    Default Value: & .false. \\
    \hline
    Possible Values: & .true. or .false. \\
    \hline
    \caption{config\_use\_activeTracers\_exponential\_decay: if true, exponential decay source is applied to tracers in 'activeTracers' category}
\end{longtable}
\end{center}
\subsection[config\_use\_activeTracers\_idealAge\_forcing]{\hyperref[sec:nm_tab_tracer_forcing_activeTracers]{config\_use\_activeTracers\_idealAge\_forcing}}
\label{subsec:nm_sec_config_use_activeTracers_idealAge_forcing}
\begin{center}
\begin{longtable}{| p{2.0in} || p{4.0in} |}
    \hline
    Type: & logical \\
    \hline
    Units: & -- \\
    \hline
    Default Value: & .false. \\
    \hline
    Possible Values: & .true. or .false. \\
    \hline
    \caption{config\_use\_activeTracers\_idealAge\_forcing: if true, idealAge forcing source is applied to tracers in 'activeTracers' category}
\end{longtable}
\end{center}
\subsection[config\_use\_activeTracers\_ttd\_forcing]{\hyperref[sec:nm_tab_tracer_forcing_activeTracers]{config\_use\_activeTracers\_ttd\_forcing}}
\label{subsec:nm_sec_config_use_activeTracers_ttd_forcing}
\begin{center}
\begin{longtable}{| p{2.0in} || p{4.0in} |}
    \hline
    Type: & logical \\
    \hline
    Units: & -- \\
    \hline
    Default Value: & .false. \\
    \hline
    Possible Values: & .true. or .false. \\
    \hline
    \caption{config\_use\_activeTracers\_ttd\_forcing: if true, transit time distribution forcing source is applied to tracers in 'activeTracers' category}
\end{longtable}
\end{center}
\subsection[config\_use\_surface\_salinity\_monthly\_restoring]{\hyperref[sec:nm_tab_tracer_forcing_activeTracers]{config\_use\_surface\_salinity\_monthly\_restoring}}
\label{subsec:nm_sec_config_use_surface_salinity_monthly_restoring}
\begin{center}
\begin{longtable}{| p{2.0in} || p{4.0in} |}
    \hline
    Type: & logical \\
    \hline
    Units: & -- \\
    \hline
    Default Value: & .false. \\
    \hline
    Possible Values: & .true. or .false. \\
    \hline
    \caption{config\_use\_surface\_salinity\_monthly\_restoring: If true, apply monthly salinity restoring using a uniform piston velocity, defined at run-time by config\_salinity\_restoring\_constant\_piston\_velocity.  When false, salinity piston velocity is specified in the input file by salinityPistonVelocity, which may be spatially variable.}
\end{longtable}
\end{center}
\subsection[config\_surface\_salinity\_monthly\_restoring\_compute\_interval]{\hyperref[sec:nm_tab_tracer_forcing_activeTracers]{config\_surface\_salinity\_monthly\_restoring\_compute\_interval}}
\label{subsec:nm_sec_config_surface_salinity_monthly_restoring_compute_interval}
\begin{center}
\begin{longtable}{| p{2.0in} || p{4.0in} |}
    \hline
    Type: & character \\
    \hline
    Units: & -- \\
    \hline
    Default Value: & 0000-00-01\_00:00:00 \\
    \hline
    Possible Values: & Any valid time stamp or 'dt' \\
    \hline
    \caption{config\_surface\_salinity\_monthly\_restoring\_compute\_interval: Time interval to compute salinity restoring tendency.}
\end{longtable}
\end{center}
\subsection[config\_salinity\_restoring\_constant\_piston\_velocity]{\hyperref[sec:nm_tab_tracer_forcing_activeTracers]{config\_salinity\_restoring\_constant\_piston\_velocity}}
\label{subsec:nm_sec_config_salinity_restoring_constant_piston_velocity}
\begin{center}
\begin{longtable}{| p{2.0in} || p{4.0in} |}
    \hline
    Type: & real \\
    \hline
    Units: & \si{m/year} \\
    \hline
    Default Value: & 1.585e-5 \\
    \hline
    Possible Values: & any non-negative number \\
    \hline
    \caption{config\_salinity\_restoring\_constant\_piston\_velocity: When config\_use\_surface\_salinity\_monthly\_restoring is true, this flag provides a run-time override of the salinityPistonVelocity variable in the input files.  It is uniform over the domain, and controls the rate at which salinity is restored to salinitySurfaceRestoringValue}
\end{longtable}
\end{center}
\subsection[config\_salinity\_restoring\_max\_difference]{\hyperref[sec:nm_tab_tracer_forcing_activeTracers]{config\_salinity\_restoring\_max\_difference}}
\label{subsec:nm_sec_config_salinity_restoring_max_difference}
\begin{center}
\begin{longtable}{| p{2.0in} || p{4.0in} |}
    \hline
    Type: & real \\
    \hline
    Units: & \si{1.e-3} \\
    \hline
    Default Value: & 100.0 \\
    \hline
    Possible Values: & any non-negative number \\
    \hline
    \caption{config\_salinity\_restoring\_max\_difference: Maximum allowable difference between surface salinity and climatology, in grams salt per kilogram seawater.}
\end{longtable}
\end{center}
\subsection[config\_salinity\_restoring\_under\_sea\_ice]{\hyperref[sec:nm_tab_tracer_forcing_activeTracers]{config\_salinity\_restoring\_under\_sea\_ice}}
\label{subsec:nm_sec_config_salinity_restoring_under_sea_ice}
\begin{center}
\begin{longtable}{| p{2.0in} || p{4.0in} |}
    \hline
    Type: & logical \\
    \hline
    Units: & -- \\
    \hline
    Default Value: & .false. \\
    \hline
    Possible Values: & .true. or .false. \\
    \hline
    \caption{config\_salinity\_restoring\_under\_sea\_ice: Flag to enable salinity restoring under sea ice.  The default setting is false, where salinity restoring tapers from full restoring in the open ocean (iceFraction=0.0) to zero restoring below full sea ice coverage (iceFraction=1.0); below partial sea ice coverage, restoring is in proportion to iceFraction.  If true, full salinity restoring is used everywhere, regardless of iceFraction value}
\end{longtable}
\end{center}
\section[tracer\_forcing\_debugTracers]{\hyperref[sec:nm_tab_tracer_forcing_debugTracers]{tracer\_forcing\_debugTracers}}
\label{sec:nm_sec_tracer_forcing_debugTracers}
\subsection[config\_use\_debugTracers]{\hyperref[sec:nm_tab_tracer_forcing_debugTracers]{config\_use\_debugTracers}}
\label{subsec:nm_sec_config_use_debugTracers}
\begin{center}
\begin{longtable}{| p{2.0in} || p{4.0in} |}
    \hline
    Type: & logical \\
    \hline
    Units: & -- \\
    \hline
    Default Value: & .false. \\
    \hline
    Possible Values: & .true. or .false. \\
    \hline
    \caption{config\_use\_debugTracers: if true, the 'debugTracers' category is enabled for the run}
\end{longtable}
\end{center}
\subsection[config\_reset\_debugTracers\_near\_surface]{\hyperref[sec:nm_tab_tracer_forcing_debugTracers]{config\_reset\_debugTracers\_near\_surface}}
\label{subsec:nm_sec_config_reset_debugTracers_near_surface}
\begin{center}
\begin{longtable}{| p{2.0in} || p{4.0in} |}
    \hline
    Type: & logical \\
    \hline
    Units: & -- \\
    \hline
    Default Value: & .false. \\
    \hline
    Possible Values: & .true. or .false. \\
    \hline
    \caption{config\_reset\_debugTracers\_near\_surface: if true, the reset 'debugTracers' in the top n layers, where n is set by config\_reset\_debugTracers\_top\_nLayers}
\end{longtable}
\end{center}
\subsection[config\_reset\_debugTracers\_top\_nLayers]{\hyperref[sec:nm_tab_tracer_forcing_debugTracers]{config\_reset\_debugTracers\_top\_nLayers}}
\label{subsec:nm_sec_config_reset_debugTracers_top_nLayers}
\begin{center}
\begin{longtable}{| p{2.0in} || p{4.0in} |}
    \hline
    Type: & integer \\
    \hline
    Units: & -- \\
    \hline
    Default Value: & 20 \\
    \hline
    Possible Values: & Any positive integer value greater than or equal to 0. \\
    \hline
    \caption{config\_reset\_debugTracers\_top\_nLayers: Integer specifying number of layers at top to reset 2 at end of each timestep. Default is 20, chosen to be near a typical mixed layer depth of 200m.}
\end{longtable}
\end{center}
\subsection[config\_use\_debugTracers\_surface\_bulk\_forcing]{\hyperref[sec:nm_tab_tracer_forcing_debugTracers]{config\_use\_debugTracers\_surface\_bulk\_forcing}}
\label{subsec:nm_sec_config_use_debugTracers_surface_bulk_forcing}
\begin{center}
\begin{longtable}{| p{2.0in} || p{4.0in} |}
    \hline
    Type: & logical \\
    \hline
    Units: & -- \\
    \hline
    Default Value: & .false. \\
    \hline
    Possible Values: & .true. or .false. \\
    \hline
    \caption{config\_use\_debugTracers\_surface\_bulk\_forcing: if true, surface bulk forcing from coupler is added to surfaceTracerFlux in 'debugTracers' category}
\end{longtable}
\end{center}
\subsection[config\_use\_debugTracers\_surface\_restoring]{\hyperref[sec:nm_tab_tracer_forcing_debugTracers]{config\_use\_debugTracers\_surface\_restoring}}
\label{subsec:nm_sec_config_use_debugTracers_surface_restoring}
\begin{center}
\begin{longtable}{| p{2.0in} || p{4.0in} |}
    \hline
    Type: & logical \\
    \hline
    Units: & -- \\
    \hline
    Default Value: & .false. \\
    \hline
    Possible Values: & .true. or .false. \\
    \hline
    \caption{config\_use\_debugTracers\_surface\_restoring: if true, surface restoring source is applied to tracers in 'debugTracers' category}
\end{longtable}
\end{center}
\subsection[config\_use\_debugTracers\_interior\_restoring]{\hyperref[sec:nm_tab_tracer_forcing_debugTracers]{config\_use\_debugTracers\_interior\_restoring}}
\label{subsec:nm_sec_config_use_debugTracers_interior_restoring}
\begin{center}
\begin{longtable}{| p{2.0in} || p{4.0in} |}
    \hline
    Type: & logical \\
    \hline
    Units: & -- \\
    \hline
    Default Value: & .false. \\
    \hline
    Possible Values: & .true. or .false. \\
    \hline
    \caption{config\_use\_debugTracers\_interior\_restoring: if true, interior restoring source is applied to tracers in 'debugTracers' category}
\end{longtable}
\end{center}
\subsection[config\_use\_debugTracers\_exponential\_decay]{\hyperref[sec:nm_tab_tracer_forcing_debugTracers]{config\_use\_debugTracers\_exponential\_decay}}
\label{subsec:nm_sec_config_use_debugTracers_exponential_decay}
\begin{center}
\begin{longtable}{| p{2.0in} || p{4.0in} |}
    \hline
    Type: & logical \\
    \hline
    Units: & -- \\
    \hline
    Default Value: & .false. \\
    \hline
    Possible Values: & .true. or .false. \\
    \hline
    \caption{config\_use\_debugTracers\_exponential\_decay: if true, exponential decay source is applied to tracers in 'debugTracers' category}
\end{longtable}
\end{center}
\subsection[config\_use\_debugTracers\_idealAge\_forcing]{\hyperref[sec:nm_tab_tracer_forcing_debugTracers]{config\_use\_debugTracers\_idealAge\_forcing}}
\label{subsec:nm_sec_config_use_debugTracers_idealAge_forcing}
\begin{center}
\begin{longtable}{| p{2.0in} || p{4.0in} |}
    \hline
    Type: & logical \\
    \hline
    Units: & -- \\
    \hline
    Default Value: & .false. \\
    \hline
    Possible Values: & .true. or .false. \\
    \hline
    \caption{config\_use\_debugTracers\_idealAge\_forcing: if true, idealAge forcing source is applied to tracers in 'debugTracers' category}
\end{longtable}
\end{center}
\subsection[config\_use\_debugTracers\_ttd\_forcing]{\hyperref[sec:nm_tab_tracer_forcing_debugTracers]{config\_use\_debugTracers\_ttd\_forcing}}
\label{subsec:nm_sec_config_use_debugTracers_ttd_forcing}
\begin{center}
\begin{longtable}{| p{2.0in} || p{4.0in} |}
    \hline
    Type: & logical \\
    \hline
    Units: & -- \\
    \hline
    Default Value: & .false. \\
    \hline
    Possible Values: & .true. or .false. \\
    \hline
    \caption{config\_use\_debugTracers\_ttd\_forcing: if true, transit time distribution forcing source is applied to tracers in 'debugTracers' category}
\end{longtable}
\end{center}
\section[tracer\_forcing\_ecosysTracers]{\hyperref[sec:nm_tab_tracer_forcing_ecosysTracers]{tracer\_forcing\_ecosysTracers}}
\label{sec:nm_sec_tracer_forcing_ecosysTracers}
\subsection[config\_use\_ecosysTracers]{\hyperref[sec:nm_tab_tracer_forcing_ecosysTracers]{config\_use\_ecosysTracers}}
\label{subsec:nm_sec_config_use_ecosysTracers}
\begin{center}
\begin{longtable}{| p{2.0in} || p{4.0in} |}
    \hline
    Type: & logical \\
    \hline
    Units: & \si{unitless} \\
    \hline
    Default Value: & .false. \\
    \hline
    Possible Values: & .true. or .false. \\
    \hline
    \caption{config\_use\_ecosysTracers: if true, the 'ecosysGRP' category is enabled for the run}
\end{longtable}
\end{center}
\subsection[config\_ecosys\_atm\_co2\_option]{\hyperref[sec:nm_tab_tracer_forcing_ecosysTracers]{config\_ecosys\_atm\_co2\_option}}
\label{subsec:nm_sec_config_ecosys_atm_co2_option}
\begin{center}
\begin{longtable}{| p{2.0in} || p{4.0in} |}
    \hline
    Type: & character \\
    \hline
    Units: & \si{unitless} \\
    \hline
    Default Value: & none \\
    \hline
    Possible Values: & none,constant,prognostic,diagnostic \\
    \hline
    \caption{config\_ecosys\_atm\_co2\_option: sets how atm co2 is set}
\end{longtable}
\end{center}
\subsection[config\_ecosys\_atm\_alt\_co2\_option]{\hyperref[sec:nm_tab_tracer_forcing_ecosysTracers]{config\_ecosys\_atm\_alt\_co2\_option}}
\label{subsec:nm_sec_config_ecosys_atm_alt_co2_option}
\begin{center}
\begin{longtable}{| p{2.0in} || p{4.0in} |}
    \hline
    Type: & character \\
    \hline
    Units: & \si{unitless} \\
    \hline
    Default Value: & none \\
    \hline
    Possible Values: & none,constant,prognostic,diagnostic \\
    \hline
    \caption{config\_ecosys\_atm\_alt\_co2\_option: sets how alt atm co2 is set}
\end{longtable}
\end{center}
\subsection[config\_ecosys\_atm\_alt\_co2\_use\_eco]{\hyperref[sec:nm_tab_tracer_forcing_ecosysTracers]{config\_ecosys\_atm\_alt\_co2\_use\_eco}}
\label{subsec:nm_sec_config_ecosys_atm_alt_co2_use_eco}
\begin{center}
\begin{longtable}{| p{2.0in} || p{4.0in} |}
    \hline
    Type: & logical \\
    \hline
    Units: & \si{unitless} \\
    \hline
    Default Value: & .true. \\
    \hline
    Possible Values: & .true. or .false. \\
    \hline
    \caption{config\_ecosys\_atm\_alt\_co2\_use\_eco: determines whether DIC\_ALT is affected by ecosystem dynamics}
\end{longtable}
\end{center}
\subsection[config\_ecosys\_atm\_co2\_constant\_value]{\hyperref[sec:nm_tab_tracer_forcing_ecosysTracers]{config\_ecosys\_atm\_co2\_constant\_value}}
\label{subsec:nm_sec_config_ecosys_atm_co2_constant_value}
\begin{center}
\begin{longtable}{| p{2.0in} || p{4.0in} |}
    \hline
    Type: & real \\
    \hline
    Units: & \si{ppmv} \\
    \hline
    Default Value: & 379.0 \\
    \hline
    Possible Values: & positive real number \\
    \hline
    \caption{config\_ecosys\_atm\_co2\_constant\_value: value of atm co2 when config\_ecosys\_atm\_co2\_option = constant}
\end{longtable}
\end{center}
\subsection[config\_use\_ecosysTracers\_surface\_bulk\_forcing]{\hyperref[sec:nm_tab_tracer_forcing_ecosysTracers]{config\_use\_ecosysTracers\_surface\_bulk\_forcing}}
\label{subsec:nm_sec_config_use_ecosysTracers_surface_bulk_forcing}
\begin{center}
\begin{longtable}{| p{2.0in} || p{4.0in} |}
    \hline
    Type: & logical \\
    \hline
    Units: & \si{unitless} \\
    \hline
    Default Value: & .false. \\
    \hline
    Possible Values: & .true. or .false. \\
    \hline
    \caption{config\_use\_ecosysTracers\_surface\_bulk\_forcing: if true, surface bulk forcing from coupler is added to surfaceTracerFlux in 'ecosysGRP' category}
\end{longtable}
\end{center}
\subsection[config\_use\_ecosysTracers\_surface\_restoring]{\hyperref[sec:nm_tab_tracer_forcing_ecosysTracers]{config\_use\_ecosysTracers\_surface\_restoring}}
\label{subsec:nm_sec_config_use_ecosysTracers_surface_restoring}
\begin{center}
\begin{longtable}{| p{2.0in} || p{4.0in} |}
    \hline
    Type: & logical \\
    \hline
    Units: & \si{unitless} \\
    \hline
    Default Value: & .false. \\
    \hline
    Possible Values: & .true. or .false. \\
    \hline
    \caption{config\_use\_ecosysTracers\_surface\_restoring: if true, surface restoring source is applied to tracers in 'ecosysGRP' category}
\end{longtable}
\end{center}
\subsection[config\_use\_ecosysTracers\_interior\_restoring]{\hyperref[sec:nm_tab_tracer_forcing_ecosysTracers]{config\_use\_ecosysTracers\_interior\_restoring}}
\label{subsec:nm_sec_config_use_ecosysTracers_interior_restoring}
\begin{center}
\begin{longtable}{| p{2.0in} || p{4.0in} |}
    \hline
    Type: & logical \\
    \hline
    Units: & \si{unitless} \\
    \hline
    Default Value: & .false. \\
    \hline
    Possible Values: & .true. or .false. \\
    \hline
    \caption{config\_use\_ecosysTracers\_interior\_restoring: if true, interior restoring source is applied to tracers in 'ecosysGRP' category}
\end{longtable}
\end{center}
\subsection[config\_use\_ecosysTracers\_exponential\_decay]{\hyperref[sec:nm_tab_tracer_forcing_ecosysTracers]{config\_use\_ecosysTracers\_exponential\_decay}}
\label{subsec:nm_sec_config_use_ecosysTracers_exponential_decay}
\begin{center}
\begin{longtable}{| p{2.0in} || p{4.0in} |}
    \hline
    Type: & logical \\
    \hline
    Units: & \si{unitless} \\
    \hline
    Default Value: & .false. \\
    \hline
    Possible Values: & .true. or .false. \\
    \hline
    \caption{config\_use\_ecosysTracers\_exponential\_decay: if true, exponential decay source is applied to tracers in 'ecosysGRP' category}
\end{longtable}
\end{center}
\subsection[config\_use\_ecosysTracers\_idealAge\_forcing]{\hyperref[sec:nm_tab_tracer_forcing_ecosysTracers]{config\_use\_ecosysTracers\_idealAge\_forcing}}
\label{subsec:nm_sec_config_use_ecosysTracers_idealAge_forcing}
\begin{center}
\begin{longtable}{| p{2.0in} || p{4.0in} |}
    \hline
    Type: & logical \\
    \hline
    Units: & \si{unitless} \\
    \hline
    Default Value: & .false. \\
    \hline
    Possible Values: & .true. or .false. \\
    \hline
    \caption{config\_use\_ecosysTracers\_idealAge\_forcing: if true, idealAge forcing source is applied to tracers in 'ecosysGRP' category}
\end{longtable}
\end{center}
\subsection[config\_use\_ecosysTracers\_ttd\_forcing]{\hyperref[sec:nm_tab_tracer_forcing_ecosysTracers]{config\_use\_ecosysTracers\_ttd\_forcing}}
\label{subsec:nm_sec_config_use_ecosysTracers_ttd_forcing}
\begin{center}
\begin{longtable}{| p{2.0in} || p{4.0in} |}
    \hline
    Type: & logical \\
    \hline
    Units: & \si{unitless} \\
    \hline
    Default Value: & .false. \\
    \hline
    Possible Values: & .true. or .false. \\
    \hline
    \caption{config\_use\_ecosysTracers\_ttd\_forcing: if true, transit time distribution forcing source is applied to tracers in 'ecosysGRP' category}
\end{longtable}
\end{center}
\subsection[config\_use\_ecosysTracers\_surface\_value]{\hyperref[sec:nm_tab_tracer_forcing_ecosysTracers]{config\_use\_ecosysTracers\_surface\_value}}
\label{subsec:nm_sec_config_use_ecosysTracers_surface_value}
\begin{center}
\begin{longtable}{| p{2.0in} || p{4.0in} |}
    \hline
    Type: & logical \\
    \hline
    Units: & \si{unitless} \\
    \hline
    Default Value: & .false. \\
    \hline
    Possible Values: & .true. or .false. \\
    \hline
    \caption{config\_use\_ecosysTracers\_surface\_value: if true, surface value is computed for 'ecosysGRP' category}
\end{longtable}
\end{center}
\subsection[config\_use\_ecosysTracers\_river\_inputs\_from\_coupler]{\hyperref[sec:nm_tab_tracer_forcing_ecosysTracers]{config\_use\_ecosysTracers\_river\_inputs\_from\_coupler}}
\label{subsec:nm_sec_config_use_ecosysTracers_river_inputs_from_coupler}
\begin{center}
\begin{longtable}{| p{2.0in} || p{4.0in} |}
    \hline
    Type: & logical \\
    \hline
    Units: & \si{unitless} \\
    \hline
    Default Value: & .false. \\
    \hline
    Possible Values: & .true. or .false. \\
    \hline
    \caption{config\_use\_ecosysTracers\_river\_inputs\_from\_coupler: if true, get river nutrient inputs from the coupler, else from ecosys monthly forcing file}
\end{longtable}
\end{center}
\subsection[config\_use\_ecosysTracers\_sea\_ice\_coupling]{\hyperref[sec:nm_tab_tracer_forcing_ecosysTracers]{config\_use\_ecosysTracers\_sea\_ice\_coupling}}
\label{subsec:nm_sec_config_use_ecosysTracers_sea_ice_coupling}
\begin{center}
\begin{longtable}{| p{2.0in} || p{4.0in} |}
    \hline
    Type: & logical \\
    \hline
    Units: & \si{unitless} \\
    \hline
    Default Value: & .false. \\
    \hline
    Possible Values: & .true. or .false. \\
    \hline
    \caption{config\_use\_ecosysTracers\_sea\_ice\_coupling: if true, couple ecosys fields with sea ice}
\end{longtable}
\end{center}
\subsection[config\_ecosysTracers\_diagnostic\_fields\_level1]{\hyperref[sec:nm_tab_tracer_forcing_ecosysTracers]{config\_ecosysTracers\_diagnostic\_fields\_level1}}
\label{subsec:nm_sec_config_ecosysTracers_diagnostic_fields_level1}
\begin{center}
\begin{longtable}{| p{2.0in} || p{4.0in} |}
    \hline
    Type: & logical \\
    \hline
    Units: & \si{unitless} \\
    \hline
    Default Value: & .false. \\
    \hline
    Possible Values: & .true. or .false. \\
    \hline
    \caption{config\_ecosysTracers\_diagnostic\_fields\_level1: if true, make variables in ecosysDiagFieldsLevel1 available for output}
\end{longtable}
\end{center}
\subsection[config\_ecosysTracers\_diagnostic\_fields\_level2]{\hyperref[sec:nm_tab_tracer_forcing_ecosysTracers]{config\_ecosysTracers\_diagnostic\_fields\_level2}}
\label{subsec:nm_sec_config_ecosysTracers_diagnostic_fields_level2}
\begin{center}
\begin{longtable}{| p{2.0in} || p{4.0in} |}
    \hline
    Type: & logical \\
    \hline
    Units: & \si{unitless} \\
    \hline
    Default Value: & .false. \\
    \hline
    Possible Values: & .true. or .false. \\
    \hline
    \caption{config\_ecosysTracers\_diagnostic\_fields\_level2: if true, make variables in ecosysDiagFieldsLevel2 available for output}
\end{longtable}
\end{center}
\subsection[config\_ecosysTracers\_diagnostic\_fields\_level3]{\hyperref[sec:nm_tab_tracer_forcing_ecosysTracers]{config\_ecosysTracers\_diagnostic\_fields\_level3}}
\label{subsec:nm_sec_config_ecosysTracers_diagnostic_fields_level3}
\begin{center}
\begin{longtable}{| p{2.0in} || p{4.0in} |}
    \hline
    Type: & logical \\
    \hline
    Units: & \si{unitless} \\
    \hline
    Default Value: & .false. \\
    \hline
    Possible Values: & .true. or .false. \\
    \hline
    \caption{config\_ecosysTracers\_diagnostic\_fields\_level3: if true, make variables in ecosysDiagFieldsLevel3 available for output}
\end{longtable}
\end{center}
\subsection[config\_ecosysTracers\_diagnostic\_fields\_level4]{\hyperref[sec:nm_tab_tracer_forcing_ecosysTracers]{config\_ecosysTracers\_diagnostic\_fields\_level4}}
\label{subsec:nm_sec_config_ecosysTracers_diagnostic_fields_level4}
\begin{center}
\begin{longtable}{| p{2.0in} || p{4.0in} |}
    \hline
    Type: & logical \\
    \hline
    Units: & \si{unitless} \\
    \hline
    Default Value: & .false. \\
    \hline
    Possible Values: & .true. or .false. \\
    \hline
    \caption{config\_ecosysTracers\_diagnostic\_fields\_level4: if true, make variables in ecosysDiagFieldsLevel4 available for output}
\end{longtable}
\end{center}
\subsection[config\_ecosysTracers\_diagnostic\_fields\_level5]{\hyperref[sec:nm_tab_tracer_forcing_ecosysTracers]{config\_ecosysTracers\_diagnostic\_fields\_level5}}
\label{subsec:nm_sec_config_ecosysTracers_diagnostic_fields_level5}
\begin{center}
\begin{longtable}{| p{2.0in} || p{4.0in} |}
    \hline
    Type: & logical \\
    \hline
    Units: & \si{unitless} \\
    \hline
    Default Value: & .false. \\
    \hline
    Possible Values: & .true. or .false. \\
    \hline
    \caption{config\_ecosysTracers\_diagnostic\_fields\_level5: if true, make variables in ecosysDiagFieldsLevel5 available for output}
\end{longtable}
\end{center}
\section[tracer\_forcing\_DMSTracers]{\hyperref[sec:nm_tab_tracer_forcing_DMSTracers]{tracer\_forcing\_DMSTracers}}
\label{sec:nm_sec_tracer_forcing_DMSTracers}
\subsection[config\_use\_DMSTracers]{\hyperref[sec:nm_tab_tracer_forcing_DMSTracers]{config\_use\_DMSTracers}}
\label{subsec:nm_sec_config_use_DMSTracers}
\begin{center}
\begin{longtable}{| p{2.0in} || p{4.0in} |}
    \hline
    Type: & logical \\
    \hline
    Units: & \si{unitless} \\
    \hline
    Default Value: & .false. \\
    \hline
    Possible Values: & .true. or .false. \\
    \hline
    \caption{config\_use\_DMSTracers: if true, the 'DMSGRP' category is enabled for the run}
\end{longtable}
\end{center}
\subsection[config\_use\_DMSTracers\_surface\_bulk\_forcing]{\hyperref[sec:nm_tab_tracer_forcing_DMSTracers]{config\_use\_DMSTracers\_surface\_bulk\_forcing}}
\label{subsec:nm_sec_config_use_DMSTracers_surface_bulk_forcing}
\begin{center}
\begin{longtable}{| p{2.0in} || p{4.0in} |}
    \hline
    Type: & logical \\
    \hline
    Units: & \si{unitless} \\
    \hline
    Default Value: & .false. \\
    \hline
    Possible Values: & .true. or .false. \\
    \hline
    \caption{config\_use\_DMSTracers\_surface\_bulk\_forcing: if true, surface bulk forcing from coupler is added to surfaceTracerFlux in 'DMSGRP' category}
\end{longtable}
\end{center}
\subsection[config\_use\_DMSTracers\_surface\_restoring]{\hyperref[sec:nm_tab_tracer_forcing_DMSTracers]{config\_use\_DMSTracers\_surface\_restoring}}
\label{subsec:nm_sec_config_use_DMSTracers_surface_restoring}
\begin{center}
\begin{longtable}{| p{2.0in} || p{4.0in} |}
    \hline
    Type: & logical \\
    \hline
    Units: & \si{unitless} \\
    \hline
    Default Value: & .false. \\
    \hline
    Possible Values: & .true. or .false. \\
    \hline
    \caption{config\_use\_DMSTracers\_surface\_restoring: if true, surface restoring source is applied to tracers in 'DMSGRP' category}
\end{longtable}
\end{center}
\subsection[config\_use\_DMSTracers\_interior\_restoring]{\hyperref[sec:nm_tab_tracer_forcing_DMSTracers]{config\_use\_DMSTracers\_interior\_restoring}}
\label{subsec:nm_sec_config_use_DMSTracers_interior_restoring}
\begin{center}
\begin{longtable}{| p{2.0in} || p{4.0in} |}
    \hline
    Type: & logical \\
    \hline
    Units: & \si{unitless} \\
    \hline
    Default Value: & .false. \\
    \hline
    Possible Values: & .true. or .false. \\
    \hline
    \caption{config\_use\_DMSTracers\_interior\_restoring: if true, interior restoring source is applied to tracers in 'DMSGRP' category}
\end{longtable}
\end{center}
\subsection[config\_use\_DMSTracers\_exponential\_decay]{\hyperref[sec:nm_tab_tracer_forcing_DMSTracers]{config\_use\_DMSTracers\_exponential\_decay}}
\label{subsec:nm_sec_config_use_DMSTracers_exponential_decay}
\begin{center}
\begin{longtable}{| p{2.0in} || p{4.0in} |}
    \hline
    Type: & logical \\
    \hline
    Units: & \si{unitless} \\
    \hline
    Default Value: & .false. \\
    \hline
    Possible Values: & .true. or .false. \\
    \hline
    \caption{config\_use\_DMSTracers\_exponential\_decay: if true, exponential decay source is applied to tracers in 'DMSGRP' category}
\end{longtable}
\end{center}
\subsection[config\_use\_DMSTracers\_idealAge\_forcing]{\hyperref[sec:nm_tab_tracer_forcing_DMSTracers]{config\_use\_DMSTracers\_idealAge\_forcing}}
\label{subsec:nm_sec_config_use_DMSTracers_idealAge_forcing}
\begin{center}
\begin{longtable}{| p{2.0in} || p{4.0in} |}
    \hline
    Type: & logical \\
    \hline
    Units: & \si{unitless} \\
    \hline
    Default Value: & .false. \\
    \hline
    Possible Values: & .true. or .false. \\
    \hline
    \caption{config\_use\_DMSTracers\_idealAge\_forcing: if true, idealAge forcing source is applied to tracers in 'DMSGRP' category}
\end{longtable}
\end{center}
\subsection[config\_use\_DMSTracers\_ttd\_forcing]{\hyperref[sec:nm_tab_tracer_forcing_DMSTracers]{config\_use\_DMSTracers\_ttd\_forcing}}
\label{subsec:nm_sec_config_use_DMSTracers_ttd_forcing}
\begin{center}
\begin{longtable}{| p{2.0in} || p{4.0in} |}
    \hline
    Type: & logical \\
    \hline
    Units: & \si{unitless} \\
    \hline
    Default Value: & .false. \\
    \hline
    Possible Values: & .true. or .false. \\
    \hline
    \caption{config\_use\_DMSTracers\_ttd\_forcing: if true, transit time distribution forcing source is applied to tracers in 'DMSGRP' category}
\end{longtable}
\end{center}
\subsection[config\_use\_DMSTracers\_surface\_value]{\hyperref[sec:nm_tab_tracer_forcing_DMSTracers]{config\_use\_DMSTracers\_surface\_value}}
\label{subsec:nm_sec_config_use_DMSTracers_surface_value}
\begin{center}
\begin{longtable}{| p{2.0in} || p{4.0in} |}
    \hline
    Type: & logical \\
    \hline
    Units: & \si{unitless} \\
    \hline
    Default Value: & .false. \\
    \hline
    Possible Values: & .true. or .false. \\
    \hline
    \caption{config\_use\_DMSTracers\_surface\_value: if true, surface value is computed for 'DMSGRP' category}
\end{longtable}
\end{center}
\subsection[config\_use\_DMSTracers\_sea\_ice\_coupling]{\hyperref[sec:nm_tab_tracer_forcing_DMSTracers]{config\_use\_DMSTracers\_sea\_ice\_coupling}}
\label{subsec:nm_sec_config_use_DMSTracers_sea_ice_coupling}
\begin{center}
\begin{longtable}{| p{2.0in} || p{4.0in} |}
    \hline
    Type: & logical \\
    \hline
    Units: & \si{unitless} \\
    \hline
    Default Value: & .false. \\
    \hline
    Possible Values: & .true. or .false. \\
    \hline
    \caption{config\_use\_DMSTracers\_sea\_ice\_coupling: if true, couple DMS fields with sea ice}
\end{longtable}
\end{center}
\section[tracer\_forcing\_MacroMoleculesTracers]{\hyperref[sec:nm_tab_tracer_forcing_MacroMoleculesTracers]{tracer\_forcing\_MacroMoleculesTracers}}
\label{sec:nm_sec_tracer_forcing_MacroMoleculesTracers}
\subsection[config\_use\_MacroMoleculesTracers]{\hyperref[sec:nm_tab_tracer_forcing_MacroMoleculesTracers]{config\_use\_MacroMoleculesTracers}}
\label{subsec:nm_sec_config_use_MacroMoleculesTracers}
\begin{center}
\begin{longtable}{| p{2.0in} || p{4.0in} |}
    \hline
    Type: & logical \\
    \hline
    Units: & \si{unitless} \\
    \hline
    Default Value: & .false. \\
    \hline
    Possible Values: & .true. or .false. \\
    \hline
    \caption{config\_use\_MacroMoleculesTracers: if true, the 'MacroMoleculesGRP' category is enabled for the run}
\end{longtable}
\end{center}
\subsection[config\_use\_MacroMoleculesTracers\_surface\_bulk\_forcing]{\hyperref[sec:nm_tab_tracer_forcing_MacroMoleculesTracers]{config\_use\_MacroMoleculesTracers\_surface\_bulk\_forcing}}
\label{subsec:nm_sec_config_use_MacroMoleculesTracers_surface_bulk_forcing}
\begin{center}
\begin{longtable}{| p{2.0in} || p{4.0in} |}
    \hline
    Type: & logical \\
    \hline
    Units: & \si{unitless} \\
    \hline
    Default Value: & .false. \\
    \hline
    Possible Values: & .true. or .false. \\
    \hline
    \caption{config\_use\_MacroMoleculesTracers\_surface\_bulk\_forcing: if true, surface bulk forcing from coupler is added to surfaceTracerFlux in 'MacroMoleculesGRP' category}
\end{longtable}
\end{center}
\subsection[config\_use\_MacroMoleculesTracers\_surface\_restoring]{\hyperref[sec:nm_tab_tracer_forcing_MacroMoleculesTracers]{config\_use\_MacroMoleculesTracers\_surface\_restoring}}
\label{subsec:nm_sec_config_use_MacroMoleculesTracers_surface_restoring}
\begin{center}
\begin{longtable}{| p{2.0in} || p{4.0in} |}
    \hline
    Type: & logical \\
    \hline
    Units: & \si{unitless} \\
    \hline
    Default Value: & .false. \\
    \hline
    Possible Values: & .true. or .false. \\
    \hline
    \caption{config\_use\_MacroMoleculesTracers\_surface\_restoring: if true, surface restoring source is applied to tracers in 'MacroMoleculesGRP' category}
\end{longtable}
\end{center}
\subsection[config\_use\_MacroMoleculesTracers\_interior\_restoring]{\hyperref[sec:nm_tab_tracer_forcing_MacroMoleculesTracers]{config\_use\_MacroMoleculesTracers\_interior\_restoring}}
\label{subsec:nm_sec_config_use_MacroMoleculesTracers_interior_restoring}
\begin{center}
\begin{longtable}{| p{2.0in} || p{4.0in} |}
    \hline
    Type: & logical \\
    \hline
    Units: & \si{unitless} \\
    \hline
    Default Value: & .false. \\
    \hline
    Possible Values: & .true. or .false. \\
    \hline
    \caption{config\_use\_MacroMoleculesTracers\_interior\_restoring: if true, interior restoring source is applied to tracers in 'MacroMoleculesGRP' category}
\end{longtable}
\end{center}
\subsection[config\_use\_MacroMoleculesTracers\_exponential\_decay]{\hyperref[sec:nm_tab_tracer_forcing_MacroMoleculesTracers]{config\_use\_MacroMoleculesTracers\_exponential\_decay}}
\label{subsec:nm_sec_config_use_MacroMoleculesTracers_exponential_decay}
\begin{center}
\begin{longtable}{| p{2.0in} || p{4.0in} |}
    \hline
    Type: & logical \\
    \hline
    Units: & \si{unitless} \\
    \hline
    Default Value: & .false. \\
    \hline
    Possible Values: & .true. or .false. \\
    \hline
    \caption{config\_use\_MacroMoleculesTracers\_exponential\_decay: if true, exponential decay source is applied to tracers in 'MacroMoleculesGRP' category}
\end{longtable}
\end{center}
\subsection[config\_use\_MacroMoleculesTracers\_idealAge\_forcing]{\hyperref[sec:nm_tab_tracer_forcing_MacroMoleculesTracers]{config\_use\_MacroMoleculesTracers\_idealAge\_forcing}}
\label{subsec:nm_sec_config_use_MacroMoleculesTracers_idealAge_forcing}
\begin{center}
\begin{longtable}{| p{2.0in} || p{4.0in} |}
    \hline
    Type: & logical \\
    \hline
    Units: & \si{unitless} \\
    \hline
    Default Value: & .false. \\
    \hline
    Possible Values: & .true. or .false. \\
    \hline
    \caption{config\_use\_MacroMoleculesTracers\_idealAge\_forcing: if true, idealAge forcing source is applied to tracers in 'MacroMoleculesGRP' category}
\end{longtable}
\end{center}
\subsection[config\_use\_MacroMoleculesTracers\_ttd\_forcing]{\hyperref[sec:nm_tab_tracer_forcing_MacroMoleculesTracers]{config\_use\_MacroMoleculesTracers\_ttd\_forcing}}
\label{subsec:nm_sec_config_use_MacroMoleculesTracers_ttd_forcing}
\begin{center}
\begin{longtable}{| p{2.0in} || p{4.0in} |}
    \hline
    Type: & logical \\
    \hline
    Units: & \si{unitless} \\
    \hline
    Default Value: & .false. \\
    \hline
    Possible Values: & .true. or .false. \\
    \hline
    \caption{config\_use\_MacroMoleculesTracers\_ttd\_forcing: if true, transit time distribution forcing source is applied to tracers in 'MacroMoleculesGRP' category}
\end{longtable}
\end{center}
\subsection[config\_use\_MacroMoleculesTracers\_surface\_value]{\hyperref[sec:nm_tab_tracer_forcing_MacroMoleculesTracers]{config\_use\_MacroMoleculesTracers\_surface\_value}}
\label{subsec:nm_sec_config_use_MacroMoleculesTracers_surface_value}
\begin{center}
\begin{longtable}{| p{2.0in} || p{4.0in} |}
    \hline
    Type: & logical \\
    \hline
    Units: & \si{unitless} \\
    \hline
    Default Value: & .false. \\
    \hline
    Possible Values: & .true. or .false. \\
    \hline
    \caption{config\_use\_MacroMoleculesTracers\_surface\_value: if true, surface value is computed for 'MacroMoleculesGRP' category}
\end{longtable}
\end{center}
\subsection[config\_use\_MacroMoleculesTracers\_sea\_ice\_coupling]{\hyperref[sec:nm_tab_tracer_forcing_MacroMoleculesTracers]{config\_use\_MacroMoleculesTracers\_sea\_ice\_coupling}}
\label{subsec:nm_sec_config_use_MacroMoleculesTracers_sea_ice_coupling}
\begin{center}
\begin{longtable}{| p{2.0in} || p{4.0in} |}
    \hline
    Type: & logical \\
    \hline
    Units: & \si{unitless} \\
    \hline
    Default Value: & .false. \\
    \hline
    Possible Values: & .true. or .false. \\
    \hline
    \caption{config\_use\_MacroMoleculesTracers\_sea\_ice\_coupling: if true, couple MacroMolecules fields with sea ice}
\end{longtable}
\end{center}
\section[tracer\_forcing\_idealAgeTracers]{\hyperref[sec:nm_tab_tracer_forcing_idealAgeTracers]{tracer\_forcing\_idealAgeTracers}}
\label{sec:nm_sec_tracer_forcing_idealAgeTracers}
\subsection[config\_use\_idealAgeTracers]{\hyperref[sec:nm_tab_tracer_forcing_idealAgeTracers]{config\_use\_idealAgeTracers}}
\label{subsec:nm_sec_config_use_idealAgeTracers}
\begin{center}
\begin{longtable}{| p{2.0in} || p{4.0in} |}
    \hline
    Type: & logical \\
    \hline
    Units: & \si{unitless} \\
    \hline
    Default Value: & .false. \\
    \hline
    Possible Values: & .true. or .false. \\
    \hline
    \caption{config\_use\_idealAgeTracers: if true, the 'idealAgeTracers' category is enabled for the run}
\end{longtable}
\end{center}
\subsection[config\_use\_idealAgeTracers\_surface\_bulk\_forcing]{\hyperref[sec:nm_tab_tracer_forcing_idealAgeTracers]{config\_use\_idealAgeTracers\_surface\_bulk\_forcing}}
\label{subsec:nm_sec_config_use_idealAgeTracers_surface_bulk_forcing}
\begin{center}
\begin{longtable}{| p{2.0in} || p{4.0in} |}
    \hline
    Type: & logical \\
    \hline
    Units: & \si{unitless} \\
    \hline
    Default Value: & .false. \\
    \hline
    Possible Values: & .true. or .false. \\
    \hline
    \caption{config\_use\_idealAgeTracers\_surface\_bulk\_forcing: if true, surface bulk forcing from coupler is added to surfaceTracerFlux in 'idealAgeTracers' category}
\end{longtable}
\end{center}
\subsection[config\_use\_idealAgeTracers\_surface\_restoring]{\hyperref[sec:nm_tab_tracer_forcing_idealAgeTracers]{config\_use\_idealAgeTracers\_surface\_restoring}}
\label{subsec:nm_sec_config_use_idealAgeTracers_surface_restoring}
\begin{center}
\begin{longtable}{| p{2.0in} || p{4.0in} |}
    \hline
    Type: & logical \\
    \hline
    Units: & \si{unitless} \\
    \hline
    Default Value: & .false. \\
    \hline
    Possible Values: & .true. or .false. \\
    \hline
    \caption{config\_use\_idealAgeTracers\_surface\_restoring: if true, surface restoring source is applied to tracers in 'idealAgeTracers' category}
\end{longtable}
\end{center}
\subsection[config\_use\_idealAgeTracers\_interior\_restoring]{\hyperref[sec:nm_tab_tracer_forcing_idealAgeTracers]{config\_use\_idealAgeTracers\_interior\_restoring}}
\label{subsec:nm_sec_config_use_idealAgeTracers_interior_restoring}
\begin{center}
\begin{longtable}{| p{2.0in} || p{4.0in} |}
    \hline
    Type: & logical \\
    \hline
    Units: & \si{unitless} \\
    \hline
    Default Value: & .false. \\
    \hline
    Possible Values: & .true. or .false. \\
    \hline
    \caption{config\_use\_idealAgeTracers\_interior\_restoring: if true, interior restoring source is applied to tracers in 'idealAgeTracers' category}
\end{longtable}
\end{center}
\subsection[config\_use\_idealAgeTracers\_exponential\_decay]{\hyperref[sec:nm_tab_tracer_forcing_idealAgeTracers]{config\_use\_idealAgeTracers\_exponential\_decay}}
\label{subsec:nm_sec_config_use_idealAgeTracers_exponential_decay}
\begin{center}
\begin{longtable}{| p{2.0in} || p{4.0in} |}
    \hline
    Type: & logical \\
    \hline
    Units: & \si{unitless} \\
    \hline
    Default Value: & .false. \\
    \hline
    Possible Values: & .true. or .false. \\
    \hline
    \caption{config\_use\_idealAgeTracers\_exponential\_decay: if true, exponential decay source is applied to tracers in 'idealAgeTracers' category}
\end{longtable}
\end{center}
\subsection[config\_use\_idealAgeTracers\_idealAge\_forcing]{\hyperref[sec:nm_tab_tracer_forcing_idealAgeTracers]{config\_use\_idealAgeTracers\_idealAge\_forcing}}
\label{subsec:nm_sec_config_use_idealAgeTracers_idealAge_forcing}
\begin{center}
\begin{longtable}{| p{2.0in} || p{4.0in} |}
    \hline
    Type: & logical \\
    \hline
    Units: & \si{unitless} \\
    \hline
    Default Value: & .true. \\
    \hline
    Possible Values: & .true. or .false. \\
    \hline
    \caption{config\_use\_idealAgeTracers\_idealAge\_forcing: if true, idealAge forcing source is applied to tracers in 'idealAgeTracers' category}
\end{longtable}
\end{center}
\subsection[config\_use\_idealAgeTracers\_ttd\_forcing]{\hyperref[sec:nm_tab_tracer_forcing_idealAgeTracers]{config\_use\_idealAgeTracers\_ttd\_forcing}}
\label{subsec:nm_sec_config_use_idealAgeTracers_ttd_forcing}
\begin{center}
\begin{longtable}{| p{2.0in} || p{4.0in} |}
    \hline
    Type: & logical \\
    \hline
    Units: & \si{unitless} \\
    \hline
    Default Value: & .false. \\
    \hline
    Possible Values: & .true. or .false. \\
    \hline
    \caption{config\_use\_idealAgeTracers\_ttd\_forcing: if true, transit time distribution forcing source is applied to tracers in 'idealAgeTracers' category}
\end{longtable}
\end{center}
\section[tracer\_forcing\_CFCTracers]{\hyperref[sec:nm_tab_tracer_forcing_CFCTracers]{tracer\_forcing\_CFCTracers}}
\label{sec:nm_sec_tracer_forcing_CFCTracers}
\subsection[config\_use\_CFCTracers]{\hyperref[sec:nm_tab_tracer_forcing_CFCTracers]{config\_use\_CFCTracers}}
\label{subsec:nm_sec_config_use_CFCTracers}
\begin{center}
\begin{longtable}{| p{2.0in} || p{4.0in} |}
    \hline
    Type: & logical \\
    \hline
    Units: & \si{unitless} \\
    \hline
    Default Value: & .false. \\
    \hline
    Possible Values: & .true. or .false. \\
    \hline
    \caption{config\_use\_CFCTracers: if true, the 'CFCGRP' category is enabled for the run}
\end{longtable}
\end{center}
\subsection[config\_use\_CFCTracers\_surface\_bulk\_forcing]{\hyperref[sec:nm_tab_tracer_forcing_CFCTracers]{config\_use\_CFCTracers\_surface\_bulk\_forcing}}
\label{subsec:nm_sec_config_use_CFCTracers_surface_bulk_forcing}
\begin{center}
\begin{longtable}{| p{2.0in} || p{4.0in} |}
    \hline
    Type: & logical \\
    \hline
    Units: & \si{unitless} \\
    \hline
    Default Value: & .false. \\
    \hline
    Possible Values: & .true. or .false. \\
    \hline
    \caption{config\_use\_CFCTracers\_surface\_bulk\_forcing: if true, surface bulk forcing from coupler is added to surfaceTracerFlux in 'CFCGRP' category}
\end{longtable}
\end{center}
\subsection[config\_use\_CFCTracers\_surface\_restoring]{\hyperref[sec:nm_tab_tracer_forcing_CFCTracers]{config\_use\_CFCTracers\_surface\_restoring}}
\label{subsec:nm_sec_config_use_CFCTracers_surface_restoring}
\begin{center}
\begin{longtable}{| p{2.0in} || p{4.0in} |}
    \hline
    Type: & logical \\
    \hline
    Units: & \si{unitless} \\
    \hline
    Default Value: & .false. \\
    \hline
    Possible Values: & .true. or .false. \\
    \hline
    \caption{config\_use\_CFCTracers\_surface\_restoring: if true, surface restoring source is applied to tracers in 'CFCGRP' category}
\end{longtable}
\end{center}
\subsection[config\_use\_CFCTracers\_interior\_restoring]{\hyperref[sec:nm_tab_tracer_forcing_CFCTracers]{config\_use\_CFCTracers\_interior\_restoring}}
\label{subsec:nm_sec_config_use_CFCTracers_interior_restoring}
\begin{center}
\begin{longtable}{| p{2.0in} || p{4.0in} |}
    \hline
    Type: & logical \\
    \hline
    Units: & \si{unitless} \\
    \hline
    Default Value: & .false. \\
    \hline
    Possible Values: & .true. or .false. \\
    \hline
    \caption{config\_use\_CFCTracers\_interior\_restoring: if true, interior restoring source is applied to tracers in 'CFCGRP' category}
\end{longtable}
\end{center}
\subsection[config\_use\_CFCTracers\_exponential\_decay]{\hyperref[sec:nm_tab_tracer_forcing_CFCTracers]{config\_use\_CFCTracers\_exponential\_decay}}
\label{subsec:nm_sec_config_use_CFCTracers_exponential_decay}
\begin{center}
\begin{longtable}{| p{2.0in} || p{4.0in} |}
    \hline
    Type: & logical \\
    \hline
    Units: & \si{unitless} \\
    \hline
    Default Value: & .false. \\
    \hline
    Possible Values: & .true. or .false. \\
    \hline
    \caption{config\_use\_CFCTracers\_exponential\_decay: if true, exponential decay source is applied to tracers in 'CFCGRP' category}
\end{longtable}
\end{center}
\subsection[config\_use\_CFCTracers\_idealAge\_forcing]{\hyperref[sec:nm_tab_tracer_forcing_CFCTracers]{config\_use\_CFCTracers\_idealAge\_forcing}}
\label{subsec:nm_sec_config_use_CFCTracers_idealAge_forcing}
\begin{center}
\begin{longtable}{| p{2.0in} || p{4.0in} |}
    \hline
    Type: & logical \\
    \hline
    Units: & \si{unitless} \\
    \hline
    Default Value: & .false. \\
    \hline
    Possible Values: & .true. or .false. \\
    \hline
    \caption{config\_use\_CFCTracers\_idealAge\_forcing: if true, idealAge forcing source is applied to tracers in 'CFCGRP' category}
\end{longtable}
\end{center}
\subsection[config\_use\_CFCTracers\_ttd\_forcing]{\hyperref[sec:nm_tab_tracer_forcing_CFCTracers]{config\_use\_CFCTracers\_ttd\_forcing}}
\label{subsec:nm_sec_config_use_CFCTracers_ttd_forcing}
\begin{center}
\begin{longtable}{| p{2.0in} || p{4.0in} |}
    \hline
    Type: & logical \\
    \hline
    Units: & \si{unitless} \\
    \hline
    Default Value: & .false. \\
    \hline
    Possible Values: & .true. or .false. \\
    \hline
    \caption{config\_use\_CFCTracers\_ttd\_forcing: if true, transit time distribution forcing source is applied to tracers in 'CFCGRP' category}
\end{longtable}
\end{center}
\subsection[config\_use\_CFC11]{\hyperref[sec:nm_tab_tracer_forcing_CFCTracers]{config\_use\_CFC11}}
\label{subsec:nm_sec_config_use_CFC11}
\begin{center}
\begin{longtable}{| p{2.0in} || p{4.0in} |}
    \hline
    Type: & logical \\
    \hline
    Units: & \si{unitless} \\
    \hline
    Default Value: & .true. \\
    \hline
    Possible Values: & .true. or .false. \\
    \hline
    \caption{config\_use\_CFC11: if true, CFC11 is enabled for the run}
\end{longtable}
\end{center}
\subsection[config\_use\_CFC12]{\hyperref[sec:nm_tab_tracer_forcing_CFCTracers]{config\_use\_CFC12}}
\label{subsec:nm_sec_config_use_CFC12}
\begin{center}
\begin{longtable}{| p{2.0in} || p{4.0in} |}
    \hline
    Type: & logical \\
    \hline
    Units: & \si{unitless} \\
    \hline
    Default Value: & .true. \\
    \hline
    Possible Values: & .true. or .false. \\
    \hline
    \caption{config\_use\_CFC12: if true, CFC12 is enabled for the run}
\end{longtable}
\end{center}
\section[AM\_globalStats]{\hyperref[sec:nm_tab_AM_globalStats]{AM\_globalStats}}
\label{sec:nm_sec_AM_globalStats}
\subsection[config\_AM\_globalStats\_enable]{\hyperref[sec:nm_tab_AM_globalStats]{config\_AM\_globalStats\_enable}}
\label{subsec:nm_sec_config_AM_globalStats_enable}
\begin{center}
\begin{longtable}{| p{2.0in} || p{4.0in} |}
    \hline
    Type: & logical \\
    \hline
    Units: & -- \\
    \hline
    Default Value: & .false. \\
    \hline
    Possible Values: & .true. or .false. \\
    \hline
    \caption{config\_AM\_globalStats\_enable: If true, ocean analysis member global\_stats is called.}
\end{longtable}
\end{center}
\subsection[config\_AM\_globalStats\_compute\_interval]{\hyperref[sec:nm_tab_AM_globalStats]{config\_AM\_globalStats\_compute\_interval}}
\label{subsec:nm_sec_config_AM_globalStats_compute_interval}
\begin{center}
\begin{longtable}{| p{2.0in} || p{4.0in} |}
    \hline
    Type: & character \\
    \hline
    Units: & -- \\
    \hline
    Default Value: & output\_interval \\
    \hline
    Possible Values: & 'DDDD\_HH:MM:SS', 'dt', 'output\_interval' \\
    \hline
    \caption{config\_AM\_globalStats\_compute\_interval: Timestamp determining how often analysis member computation should be performed.}
\end{longtable}
\end{center}
\subsection[config\_AM\_globalStats\_compute\_on\_startup]{\hyperref[sec:nm_tab_AM_globalStats]{config\_AM\_globalStats\_compute\_on\_startup}}
\label{subsec:nm_sec_config_AM_globalStats_compute_on_startup}
\begin{center}
\begin{longtable}{| p{2.0in} || p{4.0in} |}
    \hline
    Type: & logical \\
    \hline
    Units: & -- \\
    \hline
    Default Value: & .false. \\
    \hline
    Possible Values: & .true. or .false. \\
    \hline
    \caption{config\_AM\_globalStats\_compute\_on\_startup: Logical flag determining if an analysis member computation occurs on start-up.}
\end{longtable}
\end{center}
\subsection[config\_AM\_globalStats\_write\_on\_startup]{\hyperref[sec:nm_tab_AM_globalStats]{config\_AM\_globalStats\_write\_on\_startup}}
\label{subsec:nm_sec_config_AM_globalStats_write_on_startup}
\begin{center}
\begin{longtable}{| p{2.0in} || p{4.0in} |}
    \hline
    Type: & logical \\
    \hline
    Units: & -- \\
    \hline
    Default Value: & .false. \\
    \hline
    Possible Values: & .true. or .false. \\
    \hline
    \caption{config\_AM\_globalStats\_write\_on\_startup: Logical flag determining if an analysis member computation occurs on start-up.}
\end{longtable}
\end{center}
\subsection[config\_AM\_globalStats\_text\_file]{\hyperref[sec:nm_tab_AM_globalStats]{config\_AM\_globalStats\_text\_file}}
\label{subsec:nm_sec_config_AM_globalStats_text_file}
\begin{center}
\begin{longtable}{| p{2.0in} || p{4.0in} |}
    \hline
    Type: & logical \\
    \hline
    Units: & -- \\
    \hline
    Default Value: & .false. \\
    \hline
    Possible Values: & .true. or .false. \\
    \hline
    \caption{config\_AM\_globalStats\_text\_file: If true, print global stats to a text file as well as streams.}
\end{longtable}
\end{center}
\subsection[config\_AM\_globalStats\_directory]{\hyperref[sec:nm_tab_AM_globalStats]{config\_AM\_globalStats\_directory}}
\label{subsec:nm_sec_config_AM_globalStats_directory}
\begin{center}
\begin{longtable}{| p{2.0in} || p{4.0in} |}
    \hline
    Type: & character \\
    \hline
    Units: & -- \\
    \hline
    Default Value: & analysis\_members \\
    \hline
    Possible Values: & any valid directory name \\
    \hline
    \caption{config\_AM\_globalStats\_directory: subdirectory to write eddy census text files}
\end{longtable}
\end{center}
\subsection[config\_AM\_globalStats\_output\_stream]{\hyperref[sec:nm_tab_AM_globalStats]{config\_AM\_globalStats\_output\_stream}}
\label{subsec:nm_sec_config_AM_globalStats_output_stream}
\begin{center}
\begin{longtable}{| p{2.0in} || p{4.0in} |}
    \hline
    Type: & character \\
    \hline
    Units: & -- \\
    \hline
    Default Value: & globalStatsOutput \\
    \hline
    Possible Values: & Any existing stream, or 'none' \\
    \hline
    \caption{config\_AM\_globalStats\_output\_stream: Name of the stream that the globalStats analysis member should get information from.}
\end{longtable}
\end{center}
\section[AM\_surfaceAreaWeightedAverages]{\hyperref[sec:nm_tab_AM_surfaceAreaWeightedAverages]{AM\_surfaceAreaWeightedAverages}}
\label{sec:nm_sec_AM_surfaceAreaWeightedAverages}
\subsection[config\_AM\_surfaceAreaWeightedAverages\_enable]{\hyperref[sec:nm_tab_AM_surfaceAreaWeightedAverages]{config\_AM\_surfaceAreaWeightedAverages\_enable}}
\label{subsec:nm_sec_config_AM_surfaceAreaWeightedAverages_enable}
\begin{center}
\begin{longtable}{| p{2.0in} || p{4.0in} |}
    \hline
    Type: & logical \\
    \hline
    Units: & -- \\
    \hline
    Default Value: & .false. \\
    \hline
    Possible Values: & .true. or .false. \\
    \hline
    \caption{config\_AM\_surfaceAreaWeightedAverages\_enable: If true, ocean analysis member surface\_area\_weighted\_average is called.}
\end{longtable}
\end{center}
\subsection[config\_AM\_surfaceAreaWeightedAverages\_compute\_on\_startup]{\hyperref[sec:nm_tab_AM_surfaceAreaWeightedAverages]{config\_AM\_surfaceAreaWeightedAverages\_compute\_on\_startup}}
\label{subsec:nm_sec_config_AM_surfaceAreaWeightedAverages_compute_on_startup}
\begin{center}
\begin{longtable}{| p{2.0in} || p{4.0in} |}
    \hline
    Type: & logical \\
    \hline
    Units: & -- \\
    \hline
    Default Value: & .true. \\
    \hline
    Possible Values: & .true. or .false. \\
    \hline
    \caption{config\_AM\_surfaceAreaWeightedAverages\_compute\_on\_startup: Logical flag determining if an analysis member computation occurs on start-up.}
\end{longtable}
\end{center}
\subsection[config\_AM\_surfaceAreaWeightedAverages\_write\_on\_startup]{\hyperref[sec:nm_tab_AM_surfaceAreaWeightedAverages]{config\_AM\_surfaceAreaWeightedAverages\_write\_on\_startup}}
\label{subsec:nm_sec_config_AM_surfaceAreaWeightedAverages_write_on_startup}
\begin{center}
\begin{longtable}{| p{2.0in} || p{4.0in} |}
    \hline
    Type: & logical \\
    \hline
    Units: & -- \\
    \hline
    Default Value: & .true. \\
    \hline
    Possible Values: & .true. or .false. \\
    \hline
    \caption{config\_AM\_surfaceAreaWeightedAverages\_write\_on\_startup: Logical flag determining if an analysis member computation occurs on start-up.}
\end{longtable}
\end{center}
\subsection[config\_AM\_surfaceAreaWeightedAverages\_compute\_interval]{\hyperref[sec:nm_tab_AM_surfaceAreaWeightedAverages]{config\_AM\_surfaceAreaWeightedAverages\_compute\_interval}}
\label{subsec:nm_sec_config_AM_surfaceAreaWeightedAverages_compute_interval}
\begin{center}
\begin{longtable}{| p{2.0in} || p{4.0in} |}
    \hline
    Type: & character \\
    \hline
    Units: & -- \\
    \hline
    Default Value: & output\_interval \\
    \hline
    Possible Values: & Any valid time stamp, 'dt', or 'output\_interval' \\
    \hline
    \caption{config\_AM\_surfaceAreaWeightedAverages\_compute\_interval: Time interval the determines how frequently the surface area weighted averages analysis member should be computed.}
\end{longtable}
\end{center}
\subsection[config\_AM\_surfaceAreaWeightedAverages\_output\_stream]{\hyperref[sec:nm_tab_AM_surfaceAreaWeightedAverages]{config\_AM\_surfaceAreaWeightedAverages\_output\_stream}}
\label{subsec:nm_sec_config_AM_surfaceAreaWeightedAverages_output_stream}
\begin{center}
\begin{longtable}{| p{2.0in} || p{4.0in} |}
    \hline
    Type: & character \\
    \hline
    Units: & -- \\
    \hline
    Default Value: & surfaceAreaWeightedAveragesOutput \\
    \hline
    Possible Values: & Any existing stream or 'none' \\
    \hline
    \caption{config\_AM\_surfaceAreaWeightedAverages\_output\_stream: Name of the stream the surface area weighted averages analysis member should be tied to.}
\end{longtable}
\end{center}
\section[AM\_waterMassCensus]{\hyperref[sec:nm_tab_AM_waterMassCensus]{AM\_waterMassCensus}}
\label{sec:nm_sec_AM_waterMassCensus}
\subsection[config\_AM\_waterMassCensus\_enable]{\hyperref[sec:nm_tab_AM_waterMassCensus]{config\_AM\_waterMassCensus\_enable}}
\label{subsec:nm_sec_config_AM_waterMassCensus_enable}
\begin{center}
\begin{longtable}{| p{2.0in} || p{4.0in} |}
    \hline
    Type: & logical \\
    \hline
    Units: & -- \\
    \hline
    Default Value: & .false. \\
    \hline
    Possible Values: & .true. or .false. \\
    \hline
    \caption{config\_AM\_waterMassCensus\_enable: If true, ocean analysis member water mass census is called.}
\end{longtable}
\end{center}
\subsection[config\_AM\_waterMassCensus\_compute\_interval]{\hyperref[sec:nm_tab_AM_waterMassCensus]{config\_AM\_waterMassCensus\_compute\_interval}}
\label{subsec:nm_sec_config_AM_waterMassCensus_compute_interval}
\begin{center}
\begin{longtable}{| p{2.0in} || p{4.0in} |}
    \hline
    Type: & character \\
    \hline
    Units: & -- \\
    \hline
    Default Value: & output\_interval \\
    \hline
    Possible Values: & Any valid time stamp, 'dt', or 'output\_interval' \\
    \hline
    \caption{config\_AM\_waterMassCensus\_compute\_interval: Timestamp determining how often analysis member computation should be performed.}
\end{longtable}
\end{center}
\subsection[config\_AM\_waterMassCensus\_output\_stream]{\hyperref[sec:nm_tab_AM_waterMassCensus]{config\_AM\_waterMassCensus\_output\_stream}}
\label{subsec:nm_sec_config_AM_waterMassCensus_output_stream}
\begin{center}
\begin{longtable}{| p{2.0in} || p{4.0in} |}
    \hline
    Type: & character \\
    \hline
    Units: & -- \\
    \hline
    Default Value: & waterMassCensusOutput \\
    \hline
    Possible Values: & Any existing stream name or 'none' \\
    \hline
    \caption{config\_AM\_waterMassCensus\_output\_stream: Name of the stream the water mass census analysis member should be tied to.}
\end{longtable}
\end{center}
\subsection[config\_AM\_waterMassCensus\_compute\_on\_startup]{\hyperref[sec:nm_tab_AM_waterMassCensus]{config\_AM\_waterMassCensus\_compute\_on\_startup}}
\label{subsec:nm_sec_config_AM_waterMassCensus_compute_on_startup}
\begin{center}
\begin{longtable}{| p{2.0in} || p{4.0in} |}
    \hline
    Type: & logical \\
    \hline
    Units: & -- \\
    \hline
    Default Value: & .false. \\
    \hline
    Possible Values: & .true. or .false. \\
    \hline
    \caption{config\_AM\_waterMassCensus\_compute\_on\_startup: Logical flag determining if an analysis member computation occurs on start-up.}
\end{longtable}
\end{center}
\subsection[config\_AM\_waterMassCensus\_write\_on\_startup]{\hyperref[sec:nm_tab_AM_waterMassCensus]{config\_AM\_waterMassCensus\_write\_on\_startup}}
\label{subsec:nm_sec_config_AM_waterMassCensus_write_on_startup}
\begin{center}
\begin{longtable}{| p{2.0in} || p{4.0in} |}
    \hline
    Type: & logical \\
    \hline
    Units: & -- \\
    \hline
    Default Value: & .false. \\
    \hline
    Possible Values: & .true. or .false. \\
    \hline
    \caption{config\_AM\_waterMassCensus\_write\_on\_startup: Logical flag determining if an analysis member output occurs on start-up.}
\end{longtable}
\end{center}
\subsection[config\_AM\_waterMassCensus\_minTemperature]{\hyperref[sec:nm_tab_AM_waterMassCensus]{config\_AM\_waterMassCensus\_minTemperature}}
\label{subsec:nm_sec_config_AM_waterMassCensus_minTemperature}
\begin{center}
\begin{longtable}{| p{2.0in} || p{4.0in} |}
    \hline
    Type: & real \\
    \hline
    Units: & \si{C} \\
    \hline
    Default Value: & -2.0 \\
    \hline
    Possible Values: & any real number smaller than config\_AM\_waterMassCensus\_maxTemperature \\
    \hline
    \caption{config\_AM\_waterMassCensus\_minTemperature: minimum temperature used in water mass census}
\end{longtable}
\end{center}
\subsection[config\_AM\_waterMassCensus\_maxTemperature]{\hyperref[sec:nm_tab_AM_waterMassCensus]{config\_AM\_waterMassCensus\_maxTemperature}}
\label{subsec:nm_sec_config_AM_waterMassCensus_maxTemperature}
\begin{center}
\begin{longtable}{| p{2.0in} || p{4.0in} |}
    \hline
    Type: & real \\
    \hline
    Units: & \si{C} \\
    \hline
    Default Value: & 30.0 \\
    \hline
    Possible Values: & any real number greater than config\_AM\_waterMassCensus\_minTemperature \\
    \hline
    \caption{config\_AM\_waterMassCensus\_maxTemperature: maximum temperature used in water mass census}
\end{longtable}
\end{center}
\subsection[config\_AM\_waterMassCensus\_minSalinity]{\hyperref[sec:nm_tab_AM_waterMassCensus]{config\_AM\_waterMassCensus\_minSalinity}}
\label{subsec:nm_sec_config_AM_waterMassCensus_minSalinity}
\begin{center}
\begin{longtable}{| p{2.0in} || p{4.0in} |}
    \hline
    Type: & real \\
    \hline
    Units: & \si{1.e-3} \\
    \hline
    Default Value: & 32.0 \\
    \hline
    Possible Values: & any real number smaller than config\_AM\_waterMassCensus\_maxSalinity \\
    \hline
    \caption{config\_AM\_waterMassCensus\_minSalinity: minimum salinity used in water mass census}
\end{longtable}
\end{center}
\subsection[config\_AM\_waterMassCensus\_maxSalinity]{\hyperref[sec:nm_tab_AM_waterMassCensus]{config\_AM\_waterMassCensus\_maxSalinity}}
\label{subsec:nm_sec_config_AM_waterMassCensus_maxSalinity}
\begin{center}
\begin{longtable}{| p{2.0in} || p{4.0in} |}
    \hline
    Type: & real \\
    \hline
    Units: & \si{1.e-3} \\
    \hline
    Default Value: & 37.0 \\
    \hline
    Possible Values: & any real number greater than config\_AM\_waterMassCensus\_minSalinity \\
    \hline
    \caption{config\_AM\_waterMassCensus\_maxSalinity: maximum salinity used in water mass census}
\end{longtable}
\end{center}
\subsection[config\_AM\_waterMassCensus\_compute\_predefined\_regions]{\hyperref[sec:nm_tab_AM_waterMassCensus]{config\_AM\_waterMassCensus\_compute\_predefined\_regions}}
\label{subsec:nm_sec_config_AM_waterMassCensus_compute_predefined_regions}
\begin{center}
\begin{longtable}{| p{2.0in} || p{4.0in} |}
    \hline
    Type: & logical \\
    \hline
    Units: & -- \\
    \hline
    Default Value: & .true. \\
    \hline
    Possible Values: & .true. or .false. \\
    \hline
    \caption{config\_AM\_waterMassCensus\_compute\_predefined\_regions: Computes predefined regions. (Does not require a region mask file.)}
\end{longtable}
\end{center}
\subsection[config\_AM\_waterMassCensus\_region\_group]{\hyperref[sec:nm_tab_AM_waterMassCensus]{config\_AM\_waterMassCensus\_region\_group}}
\label{subsec:nm_sec_config_AM_waterMassCensus_region_group}
\begin{center}
\begin{longtable}{| p{2.0in} || p{4.0in} |}
    \hline
    Type: & character \\
    \hline
    Units: & -- \\
    \hline
    Default Value: & {\bf \color{red} MISSING} \\
    \hline
    Possible Values: & 'all', '', or the name of a region group. \\
    \hline
    \caption{config\_AM\_waterMassCensus\_region\_group: The name of the region group, for which the WMC should be computed in addition to the existing WMC.}
\end{longtable}
\end{center}
\section[AM\_layerVolumeWeightedAverage]{\hyperref[sec:nm_tab_AM_layerVolumeWeightedAverage]{AM\_layerVolumeWeightedAverage}}
\label{sec:nm_sec_AM_layerVolumeWeightedAverage}
\subsection[config\_AM\_layerVolumeWeightedAverage\_enable]{\hyperref[sec:nm_tab_AM_layerVolumeWeightedAverage]{config\_AM\_layerVolumeWeightedAverage\_enable}}
\label{subsec:nm_sec_config_AM_layerVolumeWeightedAverage_enable}
\begin{center}
\begin{longtable}{| p{2.0in} || p{4.0in} |}
    \hline
    Type: & logical \\
    \hline
    Units: & -- \\
    \hline
    Default Value: & .false. \\
    \hline
    Possible Values: & .true. or .false. \\
    \hline
    \caption{config\_AM\_layerVolumeWeightedAverage\_enable: If true, ocean analysis member layer-volume weighted is called.}
\end{longtable}
\end{center}
\subsection[config\_AM\_layerVolumeWeightedAverage\_compute\_interval]{\hyperref[sec:nm_tab_AM_layerVolumeWeightedAverage]{config\_AM\_layerVolumeWeightedAverage\_compute\_interval}}
\label{subsec:nm_sec_config_AM_layerVolumeWeightedAverage_compute_interval}
\begin{center}
\begin{longtable}{| p{2.0in} || p{4.0in} |}
    \hline
    Type: & character \\
    \hline
    Units: & -- \\
    \hline
    Default Value: & output\_interval \\
    \hline
    Possible Values: & 'DDDD\_HH:MM:SS' \\
    \hline
    \caption{config\_AM\_layerVolumeWeightedAverage\_compute\_interval: Timestamp determining how often analysis member computation should be performed.}
\end{longtable}
\end{center}
\subsection[config\_AM\_layerVolumeWeightedAverage\_compute\_on\_startup]{\hyperref[sec:nm_tab_AM_layerVolumeWeightedAverage]{config\_AM\_layerVolumeWeightedAverage\_compute\_on\_startup}}
\label{subsec:nm_sec_config_AM_layerVolumeWeightedAverage_compute_on_startup}
\begin{center}
\begin{longtable}{| p{2.0in} || p{4.0in} |}
    \hline
    Type: & logical \\
    \hline
    Units: & -- \\
    \hline
    Default Value: & .false. \\
    \hline
    Possible Values: & .true. or .false. \\
    \hline
    \caption{config\_AM\_layerVolumeWeightedAverage\_compute\_on\_startup: Logical flag determining if an analysis member computation occurs on start-up.}
\end{longtable}
\end{center}
\subsection[config\_AM\_layerVolumeWeightedAverage\_write\_on\_startup]{\hyperref[sec:nm_tab_AM_layerVolumeWeightedAverage]{config\_AM\_layerVolumeWeightedAverage\_write\_on\_startup}}
\label{subsec:nm_sec_config_AM_layerVolumeWeightedAverage_write_on_startup}
\begin{center}
\begin{longtable}{| p{2.0in} || p{4.0in} |}
    \hline
    Type: & logical \\
    \hline
    Units: & -- \\
    \hline
    Default Value: & .false. \\
    \hline
    Possible Values: & .true. or .false. \\
    \hline
    \caption{config\_AM\_layerVolumeWeightedAverage\_write\_on\_startup: Logical flag determining if an analysis member output write occurs on start-up.}
\end{longtable}
\end{center}
\subsection[config\_AM\_layerVolumeWeightedAverage\_output\_stream]{\hyperref[sec:nm_tab_AM_layerVolumeWeightedAverage]{config\_AM\_layerVolumeWeightedAverage\_output\_stream}}
\label{subsec:nm_sec_config_AM_layerVolumeWeightedAverage_output_stream}
\begin{center}
\begin{longtable}{| p{2.0in} || p{4.0in} |}
    \hline
    Type: & character \\
    \hline
    Units: & -- \\
    \hline
    Default Value: & layerVolumeWeightedAverageOutput \\
    \hline
    Possible Values: & Any existing stream name or 'none' \\
    \hline
    \caption{config\_AM\_layerVolumeWeightedAverage\_output\_stream: Name of the string that should be tied to the layer volume weighted average analysis member}
\end{longtable}
\end{center}
\section[AM\_zonalMean]{\hyperref[sec:nm_tab_AM_zonalMean]{AM\_zonalMean}}
\label{sec:nm_sec_AM_zonalMean}
\subsection[config\_AM\_zonalMean\_enable]{\hyperref[sec:nm_tab_AM_zonalMean]{config\_AM\_zonalMean\_enable}}
\label{subsec:nm_sec_config_AM_zonalMean_enable}
\begin{center}
\begin{longtable}{| p{2.0in} || p{4.0in} |}
    \hline
    Type: & logical \\
    \hline
    Units: & -- \\
    \hline
    Default Value: & .false. \\
    \hline
    Possible Values: & .true. or .false. \\
    \hline
    \caption{config\_AM\_zonalMean\_enable: If true, ocean analysis member zonal\_mean is called.}
\end{longtable}
\end{center}
\subsection[config\_AM\_zonalMean\_compute\_on\_startup]{\hyperref[sec:nm_tab_AM_zonalMean]{config\_AM\_zonalMean\_compute\_on\_startup}}
\label{subsec:nm_sec_config_AM_zonalMean_compute_on_startup}
\begin{center}
\begin{longtable}{| p{2.0in} || p{4.0in} |}
    \hline
    Type: & logical \\
    \hline
    Units: & -- \\
    \hline
    Default Value: & .true. \\
    \hline
    Possible Values: & .true. or .false. \\
    \hline
    \caption{config\_AM\_zonalMean\_compute\_on\_startup: Logical flag determining if an analysis member computation occurs on start-up.}
\end{longtable}
\end{center}
\subsection[config\_AM\_zonalMean\_write\_on\_startup]{\hyperref[sec:nm_tab_AM_zonalMean]{config\_AM\_zonalMean\_write\_on\_startup}}
\label{subsec:nm_sec_config_AM_zonalMean_write_on_startup}
\begin{center}
\begin{longtable}{| p{2.0in} || p{4.0in} |}
    \hline
    Type: & logical \\
    \hline
    Units: & -- \\
    \hline
    Default Value: & .true. \\
    \hline
    Possible Values: & .true. or .false. \\
    \hline
    \caption{config\_AM\_zonalMean\_write\_on\_startup: Logical flag determining if an analysis member output occurs on start-up.}
\end{longtable}
\end{center}
\subsection[config\_AM\_zonalMean\_compute\_interval]{\hyperref[sec:nm_tab_AM_zonalMean]{config\_AM\_zonalMean\_compute\_interval}}
\label{subsec:nm_sec_config_AM_zonalMean_compute_interval}
\begin{center}
\begin{longtable}{| p{2.0in} || p{4.0in} |}
    \hline
    Type: & character \\
    \hline
    Units: & -- \\
    \hline
    Default Value: & output\_interval \\
    \hline
    Possible Values: & Any valid time stamp, 'dt', or 'output\_interval' \\
    \hline
    \caption{config\_AM\_zonalMean\_compute\_interval: Interval that determines frequency of computation for the zonal mean analysis member.}
\end{longtable}
\end{center}
\subsection[config\_AM\_zonalMean\_output\_stream]{\hyperref[sec:nm_tab_AM_zonalMean]{config\_AM\_zonalMean\_output\_stream}}
\label{subsec:nm_sec_config_AM_zonalMean_output_stream}
\begin{center}
\begin{longtable}{| p{2.0in} || p{4.0in} |}
    \hline
    Type: & character \\
    \hline
    Units: & -- \\
    \hline
    Default Value: & zonalMeanOutput \\
    \hline
    Possible Values: & Any existing stream or 'none'. \\
    \hline
    \caption{config\_AM\_zonalMean\_output\_stream: Name of stream the zonal mean analysis member should be tied to.}
\end{longtable}
\end{center}
\subsection[config\_AM\_zonalMean\_num\_bins]{\hyperref[sec:nm_tab_AM_zonalMean]{config\_AM\_zonalMean\_num\_bins}}
\label{subsec:nm_sec_config_AM_zonalMean_num_bins}
\begin{center}
\begin{longtable}{| p{2.0in} || p{4.0in} |}
    \hline
    Type: & integer \\
    \hline
    Units: & -- \\
    \hline
    Default Value: & 180 \\
    \hline
    Possible Values: & Any positive integer value less than or equal to nZonalMeanBins. \\
    \hline
    \caption{config\_AM\_zonalMean\_num\_bins: Number of bins used for zonal mean.  Must be less than or equal to the dimension nZonalMeanBins (set in Registry).}
\end{longtable}
\end{center}
\subsection[config\_AM\_zonalMean\_min\_bin]{\hyperref[sec:nm_tab_AM_zonalMean]{config\_AM\_zonalMean\_min\_bin}}
\label{subsec:nm_sec_config_AM_zonalMean_min_bin}
\begin{center}
\begin{longtable}{| p{2.0in} || p{4.0in} |}
    \hline
    Type: & real \\
    \hline
    Units: & \si{varies} \\
    \hline
    Default Value: & -1.0e34 \\
    \hline
    Possible Values: & Any real number. \\
    \hline
    \caption{config\_AM\_zonalMean\_min\_bin: minimum bin boundary value.  If set to -1.0e34, the minimum value in the domain is found.}
\end{longtable}
\end{center}
\subsection[config\_AM\_zonalMean\_max\_bin]{\hyperref[sec:nm_tab_AM_zonalMean]{config\_AM\_zonalMean\_max\_bin}}
\label{subsec:nm_sec_config_AM_zonalMean_max_bin}
\begin{center}
\begin{longtable}{| p{2.0in} || p{4.0in} |}
    \hline
    Type: & real \\
    \hline
    Units: & \si{varies} \\
    \hline
    Default Value: & -1.0e34 \\
    \hline
    Possible Values: & Any real number. \\
    \hline
    \caption{config\_AM\_zonalMean\_max\_bin: maximum bin boundary value.  If set to -1.0e34, the maximum value in the domain is found.}
\end{longtable}
\end{center}
\section[AM\_okuboWeiss]{\hyperref[sec:nm_tab_AM_okuboWeiss]{AM\_okuboWeiss}}
\label{sec:nm_sec_AM_okuboWeiss}
\subsection[config\_AM\_okuboWeiss\_enable]{\hyperref[sec:nm_tab_AM_okuboWeiss]{config\_AM\_okuboWeiss\_enable}}
\label{subsec:nm_sec_config_AM_okuboWeiss_enable}
\begin{center}
\begin{longtable}{| p{2.0in} || p{4.0in} |}
    \hline
    Type: & logical \\
    \hline
    Units: & -- \\
    \hline
    Default Value: & .false. \\
    \hline
    Possible Values: & .true. or .false. \\
    \hline
    \caption{config\_AM\_okuboWeiss\_enable: If true, ocean analysis member okubo\_weiss is called.}
\end{longtable}
\end{center}
\subsection[config\_AM\_okuboWeiss\_compute\_on\_startup]{\hyperref[sec:nm_tab_AM_okuboWeiss]{config\_AM\_okuboWeiss\_compute\_on\_startup}}
\label{subsec:nm_sec_config_AM_okuboWeiss_compute_on_startup}
\begin{center}
\begin{longtable}{| p{2.0in} || p{4.0in} |}
    \hline
    Type: & logical \\
    \hline
    Units: & -- \\
    \hline
    Default Value: & .true. \\
    \hline
    Possible Values: & .true. or .false. \\
    \hline
    \caption{config\_AM\_okuboWeiss\_compute\_on\_startup: Logical flag determining if an analysis member computation occurs on start-up.}
\end{longtable}
\end{center}
\subsection[config\_AM\_okuboWeiss\_write\_on\_startup]{\hyperref[sec:nm_tab_AM_okuboWeiss]{config\_AM\_okuboWeiss\_write\_on\_startup}}
\label{subsec:nm_sec_config_AM_okuboWeiss_write_on_startup}
\begin{center}
\begin{longtable}{| p{2.0in} || p{4.0in} |}
    \hline
    Type: & logical \\
    \hline
    Units: & -- \\
    \hline
    Default Value: & .true. \\
    \hline
    Possible Values: & .true. or .false. \\
    \hline
    \caption{config\_AM\_okuboWeiss\_write\_on\_startup: Logical flag determining if an analysis member computation occurs on start-up.}
\end{longtable}
\end{center}
\subsection[config\_AM\_okuboWeiss\_compute\_interval]{\hyperref[sec:nm_tab_AM_okuboWeiss]{config\_AM\_okuboWeiss\_compute\_interval}}
\label{subsec:nm_sec_config_AM_okuboWeiss_compute_interval}
\begin{center}
\begin{longtable}{| p{2.0in} || p{4.0in} |}
    \hline
    Type: & character \\
    \hline
    Units: & -- \\
    \hline
    Default Value: & output\_interval \\
    \hline
    Possible Values: & Any time stamp, 'dt', or 'output\_interval' \\
    \hline
    \caption{config\_AM\_okuboWeiss\_compute\_interval: Time stamp for frequency of computation of the okubo weiss analysis member.}
\end{longtable}
\end{center}
\subsection[config\_AM\_okuboWeiss\_output\_stream]{\hyperref[sec:nm_tab_AM_okuboWeiss]{config\_AM\_okuboWeiss\_output\_stream}}
\label{subsec:nm_sec_config_AM_okuboWeiss_output_stream}
\begin{center}
\begin{longtable}{| p{2.0in} || p{4.0in} |}
    \hline
    Type: & character \\
    \hline
    Units: & -- \\
    \hline
    Default Value: & okuboWeissOutput \\
    \hline
    Possible Values: & Any existing stream name or 'none' \\
    \hline
    \caption{config\_AM\_okuboWeiss\_output\_stream: Name of stream the okubo weiss analysis member should be tied to}
\end{longtable}
\end{center}
\subsection[config\_AM\_okuboWeiss\_directory]{\hyperref[sec:nm_tab_AM_okuboWeiss]{config\_AM\_okuboWeiss\_directory}}
\label{subsec:nm_sec_config_AM_okuboWeiss_directory}
\begin{center}
\begin{longtable}{| p{2.0in} || p{4.0in} |}
    \hline
    Type: & character \\
    \hline
    Units: & -- \\
    \hline
    Default Value: & analysis\_members \\
    \hline
    Possible Values: & any valid directory name \\
    \hline
    \caption{config\_AM\_okuboWeiss\_directory: subdirectory to write eddy census text files}
\end{longtable}
\end{center}
\subsection[config\_AM\_okuboWeiss\_threshold\_value]{\hyperref[sec:nm_tab_AM_okuboWeiss]{config\_AM\_okuboWeiss\_threshold\_value}}
\label{subsec:nm_sec_config_AM_okuboWeiss_threshold_value}
\begin{center}
\begin{longtable}{| p{2.0in} || p{4.0in} |}
    \hline
    Type: & real \\
    \hline
    Units: & \si{s^-2} \\
    \hline
    Default Value: & -0.2 \\
    \hline
    Possible Values: & any negative real value \\
    \hline
    \caption{config\_AM\_okuboWeiss\_threshold\_value: Threshold below which normalized OW values are counted as eddies, typically -0.2}
\end{longtable}
\end{center}
\subsection[config\_AM\_okuboWeiss\_normalization]{\hyperref[sec:nm_tab_AM_okuboWeiss]{config\_AM\_okuboWeiss\_normalization}}
\label{subsec:nm_sec_config_AM_okuboWeiss_normalization}
\begin{center}
\begin{longtable}{| p{2.0in} || p{4.0in} |}
    \hline
    Type: & real \\
    \hline
    Units: & -- \\
    \hline
    Default Value: & 1e-10 \\
    \hline
    Possible Values: & any positive real value \\
    \hline
    \caption{config\_AM\_okuboWeiss\_normalization: Parameter by which the OW values are normalized, typically the standard deviation of OW}
\end{longtable}
\end{center}
\subsection[config\_AM\_okuboWeiss\_lambda2\_normalization]{\hyperref[sec:nm_tab_AM_okuboWeiss]{config\_AM\_okuboWeiss\_lambda2\_normalization}}
\label{subsec:nm_sec_config_AM_okuboWeiss_lambda2_normalization}
\begin{center}
\begin{longtable}{| p{2.0in} || p{4.0in} |}
    \hline
    Type: & real \\
    \hline
    Units: & -- \\
    \hline
    Default Value: & 1e-10 \\
    \hline
    Possible Values: & any positive real value \\
    \hline
    \caption{config\_AM\_okuboWeiss\_lambda2\_normalization: Parameter by which the lambda\_2 values are normalized, typically the standard deviation of lambda\_2}
\end{longtable}
\end{center}
\subsection[config\_AM\_okuboWeiss\_use\_lat\_lon\_coords]{\hyperref[sec:nm_tab_AM_okuboWeiss]{config\_AM\_okuboWeiss\_use\_lat\_lon\_coords}}
\label{subsec:nm_sec_config_AM_okuboWeiss_use_lat_lon_coords}
\begin{center}
\begin{longtable}{| p{2.0in} || p{4.0in} |}
    \hline
    Type: & logical \\
    \hline
    Units: & -- \\
    \hline
    Default Value: & .true. \\
    \hline
    Possible Values: & .true. or .false. \\
    \hline
    \caption{config\_AM\_okuboWeiss\_use\_lat\_lon\_coords: If true, latitude/longitude coordinates are output for eddy census. Otherwise x/y/z coordinates are used. Ignored if not on a sphere.}
\end{longtable}
\end{center}
\subsection[config\_AM\_okuboWeiss\_compute\_eddy\_census]{\hyperref[sec:nm_tab_AM_okuboWeiss]{config\_AM\_okuboWeiss\_compute\_eddy\_census}}
\label{subsec:nm_sec_config_AM_okuboWeiss_compute_eddy_census}
\begin{center}
\begin{longtable}{| p{2.0in} || p{4.0in} |}
    \hline
    Type: & logical \\
    \hline
    Units: & -- \\
    \hline
    Default Value: & .true. \\
    \hline
    Possible Values: & .true. or .false. \\
    \hline
    \caption{config\_AM\_okuboWeiss\_compute\_eddy\_census: If true, connected components of thresholded OW values are computed, and used to compute an eddy census.}
\end{longtable}
\end{center}
\subsection[config\_AM\_okuboWeiss\_eddy\_min\_cells]{\hyperref[sec:nm_tab_AM_okuboWeiss]{config\_AM\_okuboWeiss\_eddy\_min\_cells}}
\label{subsec:nm_sec_config_AM_okuboWeiss_eddy_min_cells}
\begin{center}
\begin{longtable}{| p{2.0in} || p{4.0in} |}
    \hline
    Type: & integer \\
    \hline
    Units: & -- \\
    \hline
    Default Value: & 20 \\
    \hline
    Possible Values: & any positive integer value \\
    \hline
    \caption{config\_AM\_okuboWeiss\_eddy\_min\_cells: Minimum number of cells that a connected component must contain to be considered an eddy. This needs to be scaled based on expected eddy size given a grid resolution.}
\end{longtable}
\end{center}
\section[AM\_meridionalHeatTransport]{\hyperref[sec:nm_tab_AM_meridionalHeatTransport]{AM\_meridionalHeatTransport}}
\label{sec:nm_sec_AM_meridionalHeatTransport}
\subsection[config\_AM\_meridionalHeatTransport\_enable]{\hyperref[sec:nm_tab_AM_meridionalHeatTransport]{config\_AM\_meridionalHeatTransport\_enable}}
\label{subsec:nm_sec_config_AM_meridionalHeatTransport_enable}
\begin{center}
\begin{longtable}{| p{2.0in} || p{4.0in} |}
    \hline
    Type: & logical \\
    \hline
    Units: & -- \\
    \hline
    Default Value: & .false. \\
    \hline
    Possible Values: & .true. or .false. \\
    \hline
    \caption{config\_AM\_meridionalHeatTransport\_enable: If true, ocean analysis member meridional\_heat\_transport is called.}
\end{longtable}
\end{center}
\subsection[config\_AM\_meridionalHeatTransport\_compute\_interval]{\hyperref[sec:nm_tab_AM_meridionalHeatTransport]{config\_AM\_meridionalHeatTransport\_compute\_interval}}
\label{subsec:nm_sec_config_AM_meridionalHeatTransport_compute_interval}
\begin{center}
\begin{longtable}{| p{2.0in} || p{4.0in} |}
    \hline
    Type: & character \\
    \hline
    Units: & -- \\
    \hline
    Default Value: & output\_interval \\
    \hline
    Possible Values: & Any valid time stamp, 'dt', or 'output\_interval' \\
    \hline
    \caption{config\_AM\_meridionalHeatTransport\_compute\_interval: Timestamp determining how often analysis member computation should be performed.}
\end{longtable}
\end{center}
\subsection[config\_AM\_meridionalHeatTransport\_compute\_on\_startup]{\hyperref[sec:nm_tab_AM_meridionalHeatTransport]{config\_AM\_meridionalHeatTransport\_compute\_on\_startup}}
\label{subsec:nm_sec_config_AM_meridionalHeatTransport_compute_on_startup}
\begin{center}
\begin{longtable}{| p{2.0in} || p{4.0in} |}
    \hline
    Type: & logical \\
    \hline
    Units: & -- \\
    \hline
    Default Value: & .true. \\
    \hline
    Possible Values: & .true. or .false. \\
    \hline
    \caption{config\_AM\_meridionalHeatTransport\_compute\_on\_startup: Logical flag determining if an analysis member computation occurs on start-up.}
\end{longtable}
\end{center}
\subsection[config\_AM\_meridionalHeatTransport\_write\_on\_startup]{\hyperref[sec:nm_tab_AM_meridionalHeatTransport]{config\_AM\_meridionalHeatTransport\_write\_on\_startup}}
\label{subsec:nm_sec_config_AM_meridionalHeatTransport_write_on_startup}
\begin{center}
\begin{longtable}{| p{2.0in} || p{4.0in} |}
    \hline
    Type: & logical \\
    \hline
    Units: & -- \\
    \hline
    Default Value: & .true. \\
    \hline
    Possible Values: & .true. or .false. \\
    \hline
    \caption{config\_AM\_meridionalHeatTransport\_write\_on\_startup: Logical flag determining if an analysis member output occurs on start-up.}
\end{longtable}
\end{center}
\subsection[config\_AM\_meridionalHeatTransport\_output\_stream]{\hyperref[sec:nm_tab_AM_meridionalHeatTransport]{config\_AM\_meridionalHeatTransport\_output\_stream}}
\label{subsec:nm_sec_config_AM_meridionalHeatTransport_output_stream}
\begin{center}
\begin{longtable}{| p{2.0in} || p{4.0in} |}
    \hline
    Type: & character \\
    \hline
    Units: & -- \\
    \hline
    Default Value: & meridionalHeatTransportOutput \\
    \hline
    Possible Values: & Any existing stream name or 'none' \\
    \hline
    \caption{config\_AM\_meridionalHeatTransport\_output\_stream: Name of the stream that the meridional heat transport analysis member should be tied to.}
\end{longtable}
\end{center}
\subsection[config\_AM\_meridionalHeatTransport\_num\_bins]{\hyperref[sec:nm_tab_AM_meridionalHeatTransport]{config\_AM\_meridionalHeatTransport\_num\_bins}}
\label{subsec:nm_sec_config_AM_meridionalHeatTransport_num_bins}
\begin{center}
\begin{longtable}{| p{2.0in} || p{4.0in} |}
    \hline
    Type: & integer \\
    \hline
    Units: & -- \\
    \hline
    Default Value: & 180 \\
    \hline
    Possible Values: & Any positive integer value less than or equal to nMerHeatTransBins. \\
    \hline
    \caption{config\_AM\_meridionalHeatTransport\_num\_bins: Number of bins used for meridional heat transport.}
\end{longtable}
\end{center}
\subsection[config\_AM\_meridionalHeatTransport\_min\_bin]{\hyperref[sec:nm_tab_AM_meridionalHeatTransport]{config\_AM\_meridionalHeatTransport\_min\_bin}}
\label{subsec:nm_sec_config_AM_meridionalHeatTransport_min_bin}
\begin{center}
\begin{longtable}{| p{2.0in} || p{4.0in} |}
    \hline
    Type: & real \\
    \hline
    Units: & \si{varies} \\
    \hline
    Default Value: & -1.0e34 \\
    \hline
    Possible Values: & Any real number. \\
    \hline
    \caption{config\_AM\_meridionalHeatTransport\_min\_bin: minimum bin boundary value.  If set to -1.0e34, the minimum value in the domain is found.}
\end{longtable}
\end{center}
\subsection[config\_AM\_meridionalHeatTransport\_max\_bin]{\hyperref[sec:nm_tab_AM_meridionalHeatTransport]{config\_AM\_meridionalHeatTransport\_max\_bin}}
\label{subsec:nm_sec_config_AM_meridionalHeatTransport_max_bin}
\begin{center}
\begin{longtable}{| p{2.0in} || p{4.0in} |}
    \hline
    Type: & real \\
    \hline
    Units: & \si{varies} \\
    \hline
    Default Value: & -1.0e34 \\
    \hline
    Possible Values: & Any real number. \\
    \hline
    \caption{config\_AM\_meridionalHeatTransport\_max\_bin: maximum bin boundary value.  If set to -1.0e34, the maximum value in the domain is found.}
\end{longtable}
\end{center}
\subsection[config\_AM\_meridionalHeatTransport\_region\_group]{\hyperref[sec:nm_tab_AM_meridionalHeatTransport]{config\_AM\_meridionalHeatTransport\_region\_group}}
\label{subsec:nm_sec_config_AM_meridionalHeatTransport_region_group}
\begin{center}
\begin{longtable}{| p{2.0in} || p{4.0in} |}
    \hline
    Type: & character \\
    \hline
    Units: & -- \\
    \hline
    Default Value: & {\bf \color{red} MISSING} \\
    \hline
    Possible Values: & 'all', '', or the name of a region group. \\
    \hline
    \caption{config\_AM\_meridionalHeatTransport\_region\_group: The name of the region group, for which the MHT should be computed in addition to the global MHT.}
\end{longtable}
\end{center}
\section[AM\_testComputeInterval]{\hyperref[sec:nm_tab_AM_testComputeInterval]{AM\_testComputeInterval}}
\label{sec:nm_sec_AM_testComputeInterval}
\subsection[config\_AM\_testComputeInterval\_enable]{\hyperref[sec:nm_tab_AM_testComputeInterval]{config\_AM\_testComputeInterval\_enable}}
\label{subsec:nm_sec_config_AM_testComputeInterval_enable}
\begin{center}
\begin{longtable}{| p{2.0in} || p{4.0in} |}
    \hline
    Type: & logical \\
    \hline
    Units: & -- \\
    \hline
    Default Value: & .false. \\
    \hline
    Possible Values: & .true. or .false. \\
    \hline
    \caption{config\_AM\_testComputeInterval\_enable: If true, ocean analysis member test\_compute\_interval is called.}
\end{longtable}
\end{center}
\subsection[config\_AM\_testComputeInterval\_compute\_interval]{\hyperref[sec:nm_tab_AM_testComputeInterval]{config\_AM\_testComputeInterval\_compute\_interval}}
\label{subsec:nm_sec_config_AM_testComputeInterval_compute_interval}
\begin{center}
\begin{longtable}{| p{2.0in} || p{4.0in} |}
    \hline
    Type: & character \\
    \hline
    Units: & -- \\
    \hline
    Default Value: & 00-00-01\_00:00:00 \\
    \hline
    Possible Values: & Any valid time stamp, 'dt', 'output\_interval' \\
    \hline
    \caption{config\_AM\_testComputeInterval\_compute\_interval: Timestamp determining how often analysis member computation should be performed.}
\end{longtable}
\end{center}
\subsection[config\_AM\_testComputeInterval\_compute\_on\_startup]{\hyperref[sec:nm_tab_AM_testComputeInterval]{config\_AM\_testComputeInterval\_compute\_on\_startup}}
\label{subsec:nm_sec_config_AM_testComputeInterval_compute_on_startup}
\begin{center}
\begin{longtable}{| p{2.0in} || p{4.0in} |}
    \hline
    Type: & logical \\
    \hline
    Units: & -- \\
    \hline
    Default Value: & .true. \\
    \hline
    Possible Values: & .true. or .false. \\
    \hline
    \caption{config\_AM\_testComputeInterval\_compute\_on\_startup: Logical flag determining if an analysis member computation occurs on start-up.}
\end{longtable}
\end{center}
\subsection[config\_AM\_testComputeInterval\_write\_on\_startup]{\hyperref[sec:nm_tab_AM_testComputeInterval]{config\_AM\_testComputeInterval\_write\_on\_startup}}
\label{subsec:nm_sec_config_AM_testComputeInterval_write_on_startup}
\begin{center}
\begin{longtable}{| p{2.0in} || p{4.0in} |}
    \hline
    Type: & logical \\
    \hline
    Units: & -- \\
    \hline
    Default Value: & .true. \\
    \hline
    Possible Values: & .true. or .false. \\
    \hline
    \caption{config\_AM\_testComputeInterval\_write\_on\_startup: Logical flag determining if an analysis member write occurs on start-up.}
\end{longtable}
\end{center}
\subsection[config\_AM\_testComputeInterval\_output\_stream]{\hyperref[sec:nm_tab_AM_testComputeInterval]{config\_AM\_testComputeInterval\_output\_stream}}
\label{subsec:nm_sec_config_AM_testComputeInterval_output_stream}
\begin{center}
\begin{longtable}{| p{2.0in} || p{4.0in} |}
    \hline
    Type: & character \\
    \hline
    Units: & -- \\
    \hline
    Default Value: & testComputeIntervalOutput \\
    \hline
    Possible Values: & Any existing stream name or 'none' \\
    \hline
    \caption{config\_AM\_testComputeInterval\_output\_stream: Name of the stream that should be tied to the test\_compute\_interval analysis member}
\end{longtable}
\end{center}
\section[AM\_highFrequencyOutput]{\hyperref[sec:nm_tab_AM_highFrequencyOutput]{AM\_highFrequencyOutput}}
\label{sec:nm_sec_AM_highFrequencyOutput}
\subsection[config\_AM\_highFrequencyOutput\_enable]{\hyperref[sec:nm_tab_AM_highFrequencyOutput]{config\_AM\_highFrequencyOutput\_enable}}
\label{subsec:nm_sec_config_AM_highFrequencyOutput_enable}
\begin{center}
\begin{longtable}{| p{2.0in} || p{4.0in} |}
    \hline
    Type: & logical \\
    \hline
    Units: & -- \\
    \hline
    Default Value: & .false. \\
    \hline
    Possible Values: & .true. or .false. \\
    \hline
    \caption{config\_AM\_highFrequencyOutput\_enable: If true, ocean analysis member highFrequencyOutput is called.}
\end{longtable}
\end{center}
\subsection[config\_AM\_highFrequencyOutput\_compute\_interval]{\hyperref[sec:nm_tab_AM_highFrequencyOutput]{config\_AM\_highFrequencyOutput\_compute\_interval}}
\label{subsec:nm_sec_config_AM_highFrequencyOutput_compute_interval}
\begin{center}
\begin{longtable}{| p{2.0in} || p{4.0in} |}
    \hline
    Type: & character \\
    \hline
    Units: & -- \\
    \hline
    Default Value: & output\_interval \\
    \hline
    Possible Values: & Any valid time stamp, 'dt', or 'output\_interval' \\
    \hline
    \caption{config\_AM\_highFrequencyOutput\_compute\_interval: Timestamp determining how often analysis member computation should be performed.}
\end{longtable}
\end{center}
\subsection[config\_AM\_highFrequencyOutput\_output\_stream]{\hyperref[sec:nm_tab_AM_highFrequencyOutput]{config\_AM\_highFrequencyOutput\_output\_stream}}
\label{subsec:nm_sec_config_AM_highFrequencyOutput_output_stream}
\begin{center}
\begin{longtable}{| p{2.0in} || p{4.0in} |}
    \hline
    Type: & character \\
    \hline
    Units: & -- \\
    \hline
    Default Value: & highFrequencyOutput \\
    \hline
    Possible Values: & Any existing stream name or 'none' \\
    \hline
    \caption{config\_AM\_highFrequencyOutput\_output\_stream: Name of the stream that the highFrequencyOutput analysis member should be tied to.}
\end{longtable}
\end{center}
\subsection[config\_AM\_highFrequencyOutput\_compute\_on\_startup]{\hyperref[sec:nm_tab_AM_highFrequencyOutput]{config\_AM\_highFrequencyOutput\_compute\_on\_startup}}
\label{subsec:nm_sec_config_AM_highFrequencyOutput_compute_on_startup}
\begin{center}
\begin{longtable}{| p{2.0in} || p{4.0in} |}
    \hline
    Type: & logical \\
    \hline
    Units: & -- \\
    \hline
    Default Value: & .true. \\
    \hline
    Possible Values: & .true. or .false. \\
    \hline
    \caption{config\_AM\_highFrequencyOutput\_compute\_on\_startup: Logical flag determining if an analysis member computation occurs on start-up.}
\end{longtable}
\end{center}
\subsection[config\_AM\_highFrequencyOutput\_write\_on\_startup]{\hyperref[sec:nm_tab_AM_highFrequencyOutput]{config\_AM\_highFrequencyOutput\_write\_on\_startup}}
\label{subsec:nm_sec_config_AM_highFrequencyOutput_write_on_startup}
\begin{center}
\begin{longtable}{| p{2.0in} || p{4.0in} |}
    \hline
    Type: & logical \\
    \hline
    Units: & -- \\
    \hline
    Default Value: & .true. \\
    \hline
    Possible Values: & .true. or .false. \\
    \hline
    \caption{config\_AM\_highFrequencyOutput\_write\_on\_startup: Logical flag determining if an analysis member write occurs on start-up.}
\end{longtable}
\end{center}
\section[AM\_timeFilters]{\hyperref[sec:nm_tab_AM_timeFilters]{AM\_timeFilters}}
\label{sec:nm_sec_AM_timeFilters}
\subsection[config\_AM\_timeFilters\_enable]{\hyperref[sec:nm_tab_AM_timeFilters]{config\_AM\_timeFilters\_enable}}
\label{subsec:nm_sec_config_AM_timeFilters_enable}
\begin{center}
\begin{longtable}{| p{2.0in} || p{4.0in} |}
    \hline
    Type: & logical \\
    \hline
    Units: & -- \\
    \hline
    Default Value: & .false. \\
    \hline
    Possible Values: & .true. or .false. \\
    \hline
    \caption{config\_AM\_timeFilters\_enable: If true, ocean analysis member timeFilters is called.}
\end{longtable}
\end{center}
\subsection[config\_AM\_timeFilters\_compute\_interval]{\hyperref[sec:nm_tab_AM_timeFilters]{config\_AM\_timeFilters\_compute\_interval}}
\label{subsec:nm_sec_config_AM_timeFilters_compute_interval}
\begin{center}
\begin{longtable}{| p{2.0in} || p{4.0in} |}
    \hline
    Type: & character \\
    \hline
    Units: & -- \\
    \hline
    Default Value: & dt \\
    \hline
    Possible Values: & 'dt' because filtering should be performed at each time step. \\
    \hline
    \caption{config\_AM\_timeFilters\_compute\_interval: Timestamp determining how often analysis member computation should be performed.}
\end{longtable}
\end{center}
\subsection[config\_AM\_timeFilters\_output\_stream]{\hyperref[sec:nm_tab_AM_timeFilters]{config\_AM\_timeFilters\_output\_stream}}
\label{subsec:nm_sec_config_AM_timeFilters_output_stream}
\begin{center}
\begin{longtable}{| p{2.0in} || p{4.0in} |}
    \hline
    Type: & character \\
    \hline
    Units: & -- \\
    \hline
    Default Value: & timeFiltersOutput \\
    \hline
    Possible Values: & Any existing stream name or 'none' \\
    \hline
    \caption{config\_AM\_timeFilters\_output\_stream: Name of the stream that the timeFilters analysis member should be tied to.}
\end{longtable}
\end{center}
\subsection[config\_AM\_timeFilters\_restart\_stream]{\hyperref[sec:nm_tab_AM_timeFilters]{config\_AM\_timeFilters\_restart\_stream}}
\label{subsec:nm_sec_config_AM_timeFilters_restart_stream}
\begin{center}
\begin{longtable}{| p{2.0in} || p{4.0in} |}
    \hline
    Type: & character \\
    \hline
    Units: & -- \\
    \hline
    Default Value: & timeFiltersRestart \\
    \hline
    Possible Values: & Any existing stream name or 'none' \\
    \hline
    \caption{config\_AM\_timeFilters\_restart\_stream: Name of the stream that the timeFilters analysis member should use to perform restarts.}
\end{longtable}
\end{center}
\subsection[config\_AM\_timeFilters\_compute\_on\_startup]{\hyperref[sec:nm_tab_AM_timeFilters]{config\_AM\_timeFilters\_compute\_on\_startup}}
\label{subsec:nm_sec_config_AM_timeFilters_compute_on_startup}
\begin{center}
\begin{longtable}{| p{2.0in} || p{4.0in} |}
    \hline
    Type: & logical \\
    \hline
    Units: & -- \\
    \hline
    Default Value: & .true. \\
    \hline
    Possible Values: & .true. or .false. \\
    \hline
    \caption{config\_AM\_timeFilters\_compute\_on\_startup: Logical flag determining if an analysis member computation occurs on start-up.}
\end{longtable}
\end{center}
\subsection[config\_AM\_timeFilters\_write\_on\_startup]{\hyperref[sec:nm_tab_AM_timeFilters]{config\_AM\_timeFilters\_write\_on\_startup}}
\label{subsec:nm_sec_config_AM_timeFilters_write_on_startup}
\begin{center}
\begin{longtable}{| p{2.0in} || p{4.0in} |}
    \hline
    Type: & logical \\
    \hline
    Units: & -- \\
    \hline
    Default Value: & .true. \\
    \hline
    Possible Values: & .true. or .false. \\
    \hline
    \caption{config\_AM\_timeFilters\_write\_on\_startup: Logical flag determining if an analysis member write occurs on start-up.}
\end{longtable}
\end{center}
\subsection[config\_AM\_timeFilters\_initialize\_filters]{\hyperref[sec:nm_tab_AM_timeFilters]{config\_AM\_timeFilters\_initialize\_filters}}
\label{subsec:nm_sec_config_AM_timeFilters_initialize_filters}
\begin{center}
\begin{longtable}{| p{2.0in} || p{4.0in} |}
    \hline
    Type: & logical \\
    \hline
    Units: & -- \\
    \hline
    Default Value: & .true. \\
    \hline
    Possible Values: & .true. or .false. \\
    \hline
    \caption{config\_AM\_timeFilters\_initialize\_filters: Logical flag determining if filters should be initialized on start-up.}
\end{longtable}
\end{center}
\subsection[config\_AM\_timeFilters\_tau]{\hyperref[sec:nm_tab_AM_timeFilters]{config\_AM\_timeFilters\_tau}}
\label{subsec:nm_sec_config_AM_timeFilters_tau}
\begin{center}
\begin{longtable}{| p{2.0in} || p{4.0in} |}
    \hline
    Type: & character \\
    \hline
    Units: & -- \\
    \hline
    Default Value: & 90\_00:00:00 \\
    \hline
    Possible Values: & Any time stamp in 'DD\_hh:mm:ss' format. Items can be removed from the left if they are unused. \\
    \hline
    \caption{config\_AM\_timeFilters\_tau: Cutoff time scale $\tau$ for high and low pass filtering (default is 90 days).}
\end{longtable}
\end{center}
\subsection[config\_AM\_timeFilters\_compute\_cell\_centered\_values]{\hyperref[sec:nm_tab_AM_timeFilters]{config\_AM\_timeFilters\_compute\_cell\_centered\_values}}
\label{subsec:nm_sec_config_AM_timeFilters_compute_cell_centered_values}
\begin{center}
\begin{longtable}{| p{2.0in} || p{4.0in} |}
    \hline
    Type: & logical \\
    \hline
    Units: & -- \\
    \hline
    Default Value: & .true. \\
    \hline
    Possible Values: & .true. or .false. \\
    \hline
    \caption{config\_AM\_timeFilters\_compute\_cell\_centered\_values: Logical flag determining if cell centered values should be computed.}
\end{longtable}
\end{center}
\section[AM\_lagrPartTrack]{\hyperref[sec:nm_tab_AM_lagrPartTrack]{AM\_lagrPartTrack}}
\label{sec:nm_sec_AM_lagrPartTrack}
\subsection[config\_AM\_lagrPartTrack\_enable]{\hyperref[sec:nm_tab_AM_lagrPartTrack]{config\_AM\_lagrPartTrack\_enable}}
\label{subsec:nm_sec_config_AM_lagrPartTrack_enable}
\begin{center}
\begin{longtable}{| p{2.0in} || p{4.0in} |}
    \hline
    Type: & logical \\
    \hline
    Units: & -- \\
    \hline
    Default Value: & .false. \\
    \hline
    Possible Values: & .true. or .false. \\
    \hline
    \caption{config\_AM\_lagrPartTrack\_enable: If true, ocean analysis member lagrPartTrack is called.}
\end{longtable}
\end{center}
\subsection[config\_AM\_lagrPartTrack\_compute\_interval]{\hyperref[sec:nm_tab_AM_lagrPartTrack]{config\_AM\_lagrPartTrack\_compute\_interval}}
\label{subsec:nm_sec_config_AM_lagrPartTrack_compute_interval}
\begin{center}
\begin{longtable}{| p{2.0in} || p{4.0in} |}
    \hline
    Type: & character \\
    \hline
    Units: & -- \\
    \hline
    Default Value: & dt \\
    \hline
    Possible Values: & 'DDDD\_HH:MM:SS', 'output', 'dt' \\
    \hline
    \caption{config\_AM\_lagrPartTrack\_compute\_interval: Timestamp determining how often analysis member computation should be performed.}
\end{longtable}
\end{center}
\subsection[config\_AM\_lagrPartTrack\_compute\_on\_startup]{\hyperref[sec:nm_tab_AM_lagrPartTrack]{config\_AM\_lagrPartTrack\_compute\_on\_startup}}
\label{subsec:nm_sec_config_AM_lagrPartTrack_compute_on_startup}
\begin{center}
\begin{longtable}{| p{2.0in} || p{4.0in} |}
    \hline
    Type: & logical \\
    \hline
    Units: & -- \\
    \hline
    Default Value: & .false. \\
    \hline
    Possible Values: & .true. or .false. \\
    \hline
    \caption{config\_AM\_lagrPartTrack\_compute\_on\_startup: Logical flag determining if an analysis member computation occurs on start-up.}
\end{longtable}
\end{center}
\subsection[config\_AM\_lagrPartTrack\_output\_stream]{\hyperref[sec:nm_tab_AM_lagrPartTrack]{config\_AM\_lagrPartTrack\_output\_stream}}
\label{subsec:nm_sec_config_AM_lagrPartTrack_output_stream}
\begin{center}
\begin{longtable}{| p{2.0in} || p{4.0in} |}
    \hline
    Type: & character \\
    \hline
    Units: & -- \\
    \hline
    Default Value: & lagrPartTrackOutput \\
    \hline
    Possible Values: & Any existing stream name or 'none' \\
    \hline
    \caption{config\_AM\_lagrPartTrack\_output\_stream: Name of the stream that the lagrPartTrack analysis member should be tied to.}
\end{longtable}
\end{center}
\subsection[config\_AM\_lagrPartTrack\_restart\_stream]{\hyperref[sec:nm_tab_AM_lagrPartTrack]{config\_AM\_lagrPartTrack\_restart\_stream}}
\label{subsec:nm_sec_config_AM_lagrPartTrack_restart_stream}
\begin{center}
\begin{longtable}{| p{2.0in} || p{4.0in} |}
    \hline
    Type: & character \\
    \hline
    Units: & -- \\
    \hline
    Default Value: & lagrPartTrackRestart \\
    \hline
    Possible Values: & Any existing stream name or 'none' \\
    \hline
    \caption{config\_AM\_lagrPartTrack\_restart\_stream: Name of the stream that the lagrPartTrack analysis member should use to perform restarts.}
\end{longtable}
\end{center}
\subsection[config\_AM\_lagrPartTrack\_input\_stream]{\hyperref[sec:nm_tab_AM_lagrPartTrack]{config\_AM\_lagrPartTrack\_input\_stream}}
\label{subsec:nm_sec_config_AM_lagrPartTrack_input_stream}
\begin{center}
\begin{longtable}{| p{2.0in} || p{4.0in} |}
    \hline
    Type: & character \\
    \hline
    Units: & -- \\
    \hline
    Default Value: & lagrPartTrackInput \\
    \hline
    Possible Values: & Any existing stream name or 'none' \\
    \hline
    \caption{config\_AM\_lagrPartTrack\_input\_stream: Name of the stream that the lagrPartTrack analysis member should read only in a non-restart run.}
\end{longtable}
\end{center}
\subsection[config\_AM\_lagrPartTrack\_write\_on\_startup]{\hyperref[sec:nm_tab_AM_lagrPartTrack]{config\_AM\_lagrPartTrack\_write\_on\_startup}}
\label{subsec:nm_sec_config_AM_lagrPartTrack_write_on_startup}
\begin{center}
\begin{longtable}{| p{2.0in} || p{4.0in} |}
    \hline
    Type: & logical \\
    \hline
    Units: & -- \\
    \hline
    Default Value: & .true. \\
    \hline
    Possible Values: & .true. or .false. \\
    \hline
    \caption{config\_AM\_lagrPartTrack\_write\_on\_startup: Logical flag determining if an analysis member write occurs on start-up.}
\end{longtable}
\end{center}
\subsection[config\_AM\_lagrPartTrack\_filter\_number]{\hyperref[sec:nm_tab_AM_lagrPartTrack]{config\_AM\_lagrPartTrack\_filter\_number}}
\label{subsec:nm_sec_config_AM_lagrPartTrack_filter_number}
\begin{center}
\begin{longtable}{| p{2.0in} || p{4.0in} |}
    \hline
    Type: & integer \\
    \hline
    Units: & -- \\
    \hline
    Default Value: & 0 \\
    \hline
    Possible Values: & 0, 1, 2, ... \\
    \hline
    \caption{config\_AM\_lagrPartTrack\_filter\_number: Number of times to apply filtering operation.}
\end{longtable}
\end{center}
\subsection[config\_AM\_lagrPartTrack\_timeIntegration]{\hyperref[sec:nm_tab_AM_lagrPartTrack]{config\_AM\_lagrPartTrack\_timeIntegration}}
\label{subsec:nm_sec_config_AM_lagrPartTrack_timeIntegration}
\begin{center}
\begin{longtable}{| p{2.0in} || p{4.0in} |}
    \hline
    Type: & integer \\
    \hline
    Units: & -- \\
    \hline
    Default Value: & 2 \\
    \hline
    Possible Values: & 1, 2, 4,... \\
    \hline
    \caption{config\_AM\_lagrPartTrack\_timeIntegration: type of temporal interpolation with possible\_values='EE(1), RK2(2), RK4(4)' as ENUMs}
\end{longtable}
\end{center}
\subsection[config\_AM\_lagrPartTrack\_reset\_criteria]{\hyperref[sec:nm_tab_AM_lagrPartTrack]{config\_AM\_lagrPartTrack\_reset\_criteria}}
\label{subsec:nm_sec_config_AM_lagrPartTrack_reset_criteria}
\begin{center}
\begin{longtable}{| p{2.0in} || p{4.0in} |}
    \hline
    Type: & character \\
    \hline
    Units: & -- \\
    \hline
    Default Value: & none \\
    \hline
    Possible Values: & 'none','particle\_time','global\_time', 'region','all' \\
    \hline
    \caption{config\_AM\_lagrPartTrack\_reset\_criteria: Specify whether particles should not be reset ('none'), be reset on a timer for each particle ('particle\_time'), be reset on config\_AM\_lagrPartTrack\_reset\_time\_globally value ('global\_time'), be reset based on regions ('region'), or be reset for all conditions ('all').}
\end{longtable}
\end{center}
\subsection[config\_AM\_lagrPartTrack\_reset\_global\_timestamp]{\hyperref[sec:nm_tab_AM_lagrPartTrack]{config\_AM\_lagrPartTrack\_reset\_global\_timestamp}}
\label{subsec:nm_sec_config_AM_lagrPartTrack_reset_global_timestamp}
\begin{center}
\begin{longtable}{| p{2.0in} || p{4.0in} |}
    \hline
    Type: & character \\
    \hline
    Units: & -- \\
    \hline
    Default Value: & 0000\_00:00:00 \\
    \hline
    Possible Values: & timestamps of form 0000\_00:00:00 \\
    \hline
    \caption{config\_AM\_lagrPartTrack\_reset\_global\_timestamp: Specify reset global timestamp interval.}
\end{longtable}
\end{center}
\subsection[config\_AM\_lagrPartTrack\_region\_stream]{\hyperref[sec:nm_tab_AM_lagrPartTrack]{config\_AM\_lagrPartTrack\_region\_stream}}
\label{subsec:nm_sec_config_AM_lagrPartTrack_region_stream}
\begin{center}
\begin{longtable}{| p{2.0in} || p{4.0in} |}
    \hline
    Type: & character \\
    \hline
    Units: & -- \\
    \hline
    Default Value: & lagrPartTrackRegions \\
    \hline
    Possible Values: & Any existing stream name or 'none' \\
    \hline
    \caption{config\_AM\_lagrPartTrack\_region\_stream: Name of the stream that has region arrays resetOutsideRegionMaskValue1 and resetInsideRegionMaskValue1 for region-based particle resets.}
\end{longtable}
\end{center}
\subsection[config\_AM\_lagrPartTrack\_reset\_if\_outside\_region]{\hyperref[sec:nm_tab_AM_lagrPartTrack]{config\_AM\_lagrPartTrack\_reset\_if\_outside\_region}}
\label{subsec:nm_sec_config_AM_lagrPartTrack_reset_if_outside_region}
\begin{center}
\begin{longtable}{| p{2.0in} || p{4.0in} |}
    \hline
    Type: & logical \\
    \hline
    Units: & -- \\
    \hline
    Default Value: & .false. \\
    \hline
    Possible Values: & .false. or .true. \\
    \hline
    \caption{config\_AM\_lagrPartTrack\_reset\_if\_outside\_region: Specify whether particles should be reset when they leave the resetOutsideRegionMaskValue1 mask.}
\end{longtable}
\end{center}
\subsection[config\_AM\_lagrPartTrack\_reset\_if\_inside\_region]{\hyperref[sec:nm_tab_AM_lagrPartTrack]{config\_AM\_lagrPartTrack\_reset\_if\_inside\_region}}
\label{subsec:nm_sec_config_AM_lagrPartTrack_reset_if_inside_region}
\begin{center}
\begin{longtable}{| p{2.0in} || p{4.0in} |}
    \hline
    Type: & logical \\
    \hline
    Units: & -- \\
    \hline
    Default Value: & .false. \\
    \hline
    Possible Values: & .false. or .true. \\
    \hline
    \caption{config\_AM\_lagrPartTrack\_reset\_if\_inside\_region: Specify whether particles should be reset when they enter the resetInsideRegionMaskValue1 mask.}
\end{longtable}
\end{center}
\subsection[config\_AM\_lagrPartTrack\_sample\_horizontal\_interp]{\hyperref[sec:nm_tab_AM_lagrPartTrack]{config\_AM\_lagrPartTrack\_sample\_horizontal\_interp}}
\label{subsec:nm_sec_config_AM_lagrPartTrack_sample_horizontal_interp}
\begin{center}
\begin{longtable}{| p{2.0in} || p{4.0in} |}
    \hline
    Type: & logical \\
    \hline
    Units: & -- \\
    \hline
    Default Value: & .true. \\
    \hline
    Possible Values: & .true. or .false. \\
    \hline
    \caption{config\_AM\_lagrPartTrack\_sample\_horizontal\_interp: If true, particles horizontally interpolate sample quantities.}
\end{longtable}
\end{center}
\subsection[config\_AM\_lagrPartTrack\_sample\_temperature]{\hyperref[sec:nm_tab_AM_lagrPartTrack]{config\_AM\_lagrPartTrack\_sample\_temperature}}
\label{subsec:nm_sec_config_AM_lagrPartTrack_sample_temperature}
\begin{center}
\begin{longtable}{| p{2.0in} || p{4.0in} |}
    \hline
    Type: & logical \\
    \hline
    Units: & -- \\
    \hline
    Default Value: & .true. \\
    \hline
    Possible Values: & .true. or .false. \\
    \hline
    \caption{config\_AM\_lagrPartTrack\_sample\_temperature: If true, particles sample temperature.}
\end{longtable}
\end{center}
\subsection[config\_AM\_lagrPartTrack\_sample\_salinity]{\hyperref[sec:nm_tab_AM_lagrPartTrack]{config\_AM\_lagrPartTrack\_sample\_salinity}}
\label{subsec:nm_sec_config_AM_lagrPartTrack_sample_salinity}
\begin{center}
\begin{longtable}{| p{2.0in} || p{4.0in} |}
    \hline
    Type: & logical \\
    \hline
    Units: & -- \\
    \hline
    Default Value: & .true. \\
    \hline
    Possible Values: & .true. or .false. \\
    \hline
    \caption{config\_AM\_lagrPartTrack\_sample\_salinity: If true, particles sample salinity.}
\end{longtable}
\end{center}
\subsection[config\_AM\_lagrPartTrack\_sample\_DIC]{\hyperref[sec:nm_tab_AM_lagrPartTrack]{config\_AM\_lagrPartTrack\_sample\_DIC}}
\label{subsec:nm_sec_config_AM_lagrPartTrack_sample_DIC}
\begin{center}
\begin{longtable}{| p{2.0in} || p{4.0in} |}
    \hline
    Type: & logical \\
    \hline
    Units: & -- \\
    \hline
    Default Value: & .false. \\
    \hline
    Possible Values: & .true. or .false. \\
    \hline
    \caption{config\_AM\_lagrPartTrack\_sample\_DIC: If true, particles sample DIC.}
\end{longtable}
\end{center}
\subsection[config\_AM\_lagrPartTrack\_sample\_ALK]{\hyperref[sec:nm_tab_AM_lagrPartTrack]{config\_AM\_lagrPartTrack\_sample\_ALK}}
\label{subsec:nm_sec_config_AM_lagrPartTrack_sample_ALK}
\begin{center}
\begin{longtable}{| p{2.0in} || p{4.0in} |}
    \hline
    Type: & logical \\
    \hline
    Units: & -- \\
    \hline
    Default Value: & .false. \\
    \hline
    Possible Values: & .true. or .false. \\
    \hline
    \caption{config\_AM\_lagrPartTrack\_sample\_ALK: If true, particles sample ALK.}
\end{longtable}
\end{center}
\subsection[config\_AM\_lagrPartTrack\_sample\_PO4]{\hyperref[sec:nm_tab_AM_lagrPartTrack]{config\_AM\_lagrPartTrack\_sample\_PO4}}
\label{subsec:nm_sec_config_AM_lagrPartTrack_sample_PO4}
\begin{center}
\begin{longtable}{| p{2.0in} || p{4.0in} |}
    \hline
    Type: & logical \\
    \hline
    Units: & -- \\
    \hline
    Default Value: & .false. \\
    \hline
    Possible Values: & .true. or .false. \\
    \hline
    \caption{config\_AM\_lagrPartTrack\_sample\_PO4: If true, particles sample PO4.}
\end{longtable}
\end{center}
\subsection[config\_AM\_lagrPartTrack\_sample\_NO3]{\hyperref[sec:nm_tab_AM_lagrPartTrack]{config\_AM\_lagrPartTrack\_sample\_NO3}}
\label{subsec:nm_sec_config_AM_lagrPartTrack_sample_NO3}
\begin{center}
\begin{longtable}{| p{2.0in} || p{4.0in} |}
    \hline
    Type: & logical \\
    \hline
    Units: & -- \\
    \hline
    Default Value: & .false. \\
    \hline
    Possible Values: & .true. or .false. \\
    \hline
    \caption{config\_AM\_lagrPartTrack\_sample\_NO3: If true, particles sample NO3.}
\end{longtable}
\end{center}
\subsection[config\_AM\_lagrPartTrack\_sample\_SiO3]{\hyperref[sec:nm_tab_AM_lagrPartTrack]{config\_AM\_lagrPartTrack\_sample\_SiO3}}
\label{subsec:nm_sec_config_AM_lagrPartTrack_sample_SiO3}
\begin{center}
\begin{longtable}{| p{2.0in} || p{4.0in} |}
    \hline
    Type: & logical \\
    \hline
    Units: & -- \\
    \hline
    Default Value: & .false. \\
    \hline
    Possible Values: & .true. or .false. \\
    \hline
    \caption{config\_AM\_lagrPartTrack\_sample\_SiO3: If true, particles sample SiO3.}
\end{longtable}
\end{center}
\subsection[config\_AM\_lagrPartTrack\_sample\_NH4]{\hyperref[sec:nm_tab_AM_lagrPartTrack]{config\_AM\_lagrPartTrack\_sample\_NH4}}
\label{subsec:nm_sec_config_AM_lagrPartTrack_sample_NH4}
\begin{center}
\begin{longtable}{| p{2.0in} || p{4.0in} |}
    \hline
    Type: & logical \\
    \hline
    Units: & -- \\
    \hline
    Default Value: & .false. \\
    \hline
    Possible Values: & .true. or .false. \\
    \hline
    \caption{config\_AM\_lagrPartTrack\_sample\_NH4: If true, particles sample NH4.}
\end{longtable}
\end{center}
\subsection[config\_AM\_lagrPartTrack\_sample\_Fe]{\hyperref[sec:nm_tab_AM_lagrPartTrack]{config\_AM\_lagrPartTrack\_sample\_Fe}}
\label{subsec:nm_sec_config_AM_lagrPartTrack_sample_Fe}
\begin{center}
\begin{longtable}{| p{2.0in} || p{4.0in} |}
    \hline
    Type: & logical \\
    \hline
    Units: & -- \\
    \hline
    Default Value: & .false. \\
    \hline
    Possible Values: & .true. or .false. \\
    \hline
    \caption{config\_AM\_lagrPartTrack\_sample\_Fe: If true, particles sample Fe.}
\end{longtable}
\end{center}
\subsection[config\_AM\_lagrPartTrack\_sample\_O2]{\hyperref[sec:nm_tab_AM_lagrPartTrack]{config\_AM\_lagrPartTrack\_sample\_O2}}
\label{subsec:nm_sec_config_AM_lagrPartTrack_sample_O2}
\begin{center}
\begin{longtable}{| p{2.0in} || p{4.0in} |}
    \hline
    Type: & logical \\
    \hline
    Units: & -- \\
    \hline
    Default Value: & .false. \\
    \hline
    Possible Values: & .true. or .false. \\
    \hline
    \caption{config\_AM\_lagrPartTrack\_sample\_O2: If true, particles sample O2.}
\end{longtable}
\end{center}
\section[AM\_eliassenPalm]{\hyperref[sec:nm_tab_AM_eliassenPalm]{AM\_eliassenPalm}}
\label{sec:nm_sec_AM_eliassenPalm}
\subsection[config\_AM\_eliassenPalm\_enable]{\hyperref[sec:nm_tab_AM_eliassenPalm]{config\_AM\_eliassenPalm\_enable}}
\label{subsec:nm_sec_config_AM_eliassenPalm_enable}
\begin{center}
\begin{longtable}{| p{2.0in} || p{4.0in} |}
    \hline
    Type: & logical \\
    \hline
    Units: & -- \\
    \hline
    Default Value: & .false. \\
    \hline
    Possible Values: & .true. or .false. \\
    \hline
    \caption{config\_AM\_eliassenPalm\_enable: If true, ocean analysis member eliassenPalm is called.}
\end{longtable}
\end{center}
\subsection[config\_AM\_eliassenPalm\_compute\_interval]{\hyperref[sec:nm_tab_AM_eliassenPalm]{config\_AM\_eliassenPalm\_compute\_interval}}
\label{subsec:nm_sec_config_AM_eliassenPalm_compute_interval}
\begin{center}
\begin{longtable}{| p{2.0in} || p{4.0in} |}
    \hline
    Type: & character \\
    \hline
    Units: & -- \\
    \hline
    Default Value: & output\_interval \\
    \hline
    Possible Values: & Any valid time stamp, 'dt', or 'output\_interval' \\
    \hline
    \caption{config\_AM\_eliassenPalm\_compute\_interval: Timestamp determining how often analysis member computation should be performed.}
\end{longtable}
\end{center}
\subsection[config\_AM\_eliassenPalm\_output\_stream]{\hyperref[sec:nm_tab_AM_eliassenPalm]{config\_AM\_eliassenPalm\_output\_stream}}
\label{subsec:nm_sec_config_AM_eliassenPalm_output_stream}
\begin{center}
\begin{longtable}{| p{2.0in} || p{4.0in} |}
    \hline
    Type: & character \\
    \hline
    Units: & -- \\
    \hline
    Default Value: & eliassenPalmOutput \\
    \hline
    Possible Values: & Any existing stream name or 'none' \\
    \hline
    \caption{config\_AM\_eliassenPalm\_output\_stream: Name of the stream that the eliassenPalm analysis member should be tied to.}
\end{longtable}
\end{center}
\subsection[config\_AM\_eliassenPalm\_restart\_stream]{\hyperref[sec:nm_tab_AM_eliassenPalm]{config\_AM\_eliassenPalm\_restart\_stream}}
\label{subsec:nm_sec_config_AM_eliassenPalm_restart_stream}
\begin{center}
\begin{longtable}{| p{2.0in} || p{4.0in} |}
    \hline
    Type: & character \\
    \hline
    Units: & -- \\
    \hline
    Default Value: & eliassenPalmRestart \\
    \hline
    Possible Values: & Any existing stream name or 'none' \\
    \hline
    \caption{config\_AM\_eliassenPalm\_restart\_stream: Name of the stream that the eliassenPalm analysis member will use to performing restarts.}
\end{longtable}
\end{center}
\subsection[config\_AM\_eliassenPalm\_compute\_on\_startup]{\hyperref[sec:nm_tab_AM_eliassenPalm]{config\_AM\_eliassenPalm\_compute\_on\_startup}}
\label{subsec:nm_sec_config_AM_eliassenPalm_compute_on_startup}
\begin{center}
\begin{longtable}{| p{2.0in} || p{4.0in} |}
    \hline
    Type: & logical \\
    \hline
    Units: & -- \\
    \hline
    Default Value: & .true. \\
    \hline
    Possible Values: & .true. or .false. \\
    \hline
    \caption{config\_AM\_eliassenPalm\_compute\_on\_startup: Logical flag determining if an analysis member computation occurs on start-up.}
\end{longtable}
\end{center}
\subsection[config\_AM\_eliassenPalm\_write\_on\_startup]{\hyperref[sec:nm_tab_AM_eliassenPalm]{config\_AM\_eliassenPalm\_write\_on\_startup}}
\label{subsec:nm_sec_config_AM_eliassenPalm_write_on_startup}
\begin{center}
\begin{longtable}{| p{2.0in} || p{4.0in} |}
    \hline
    Type: & logical \\
    \hline
    Units: & -- \\
    \hline
    Default Value: & .true. \\
    \hline
    Possible Values: & .true. or .false. \\
    \hline
    \caption{config\_AM\_eliassenPalm\_write\_on\_startup: Logical flag determining if an analysis member write occurs on start-up.}
\end{longtable}
\end{center}
\subsection[config\_AM\_eliassenPalm\_debug]{\hyperref[sec:nm_tab_AM_eliassenPalm]{config\_AM\_eliassenPalm\_debug}}
\label{subsec:nm_sec_config_AM_eliassenPalm_debug}
\begin{center}
\begin{longtable}{| p{2.0in} || p{4.0in} |}
    \hline
    Type: & logical \\
    \hline
    Units: & -- \\
    \hline
    Default Value: & .false. \\
    \hline
    Possible Values: & .true. or .false. \\
    \hline
    \caption{config\_AM\_eliassenPalm\_debug: If true, debugging code is turned on.}
\end{longtable}
\end{center}
\subsection[config\_AM\_eliassenPalm\_nBuoyancyLayers]{\hyperref[sec:nm_tab_AM_eliassenPalm]{config\_AM\_eliassenPalm\_nBuoyancyLayers}}
\label{subsec:nm_sec_config_AM_eliassenPalm_nBuoyancyLayers}
\begin{center}
\begin{longtable}{| p{2.0in} || p{4.0in} |}
    \hline
    Type: & integer \\
    \hline
    Units: & \si{-} \\
    \hline
    Default Value: & 45 \\
    \hline
    Possible Values: & any positive real value. \\
    \hline
    \caption{config\_AM\_eliassenPalm\_nBuoyancyLayers: Number of reference buoyancy layers.}
\end{longtable}
\end{center}
\subsection[config\_AM\_eliassenPalm\_rhomin\_buoycoor]{\hyperref[sec:nm_tab_AM_eliassenPalm]{config\_AM\_eliassenPalm\_rhomin\_buoycoor}}
\label{subsec:nm_sec_config_AM_eliassenPalm_rhomin_buoycoor}
\begin{center}
\begin{longtable}{| p{2.0in} || p{4.0in} |}
    \hline
    Type: & real \\
    \hline
    Units: & \si{kg.m^-3} \\
    \hline
    Default Value: & 900 \\
    \hline
    Possible Values: & any positive real value. \\
    \hline
    \caption{config\_AM\_eliassenPalm\_rhomin\_buoycoor: Minimum density used in defining the first buoyancy coordinate layer}
\end{longtable}
\end{center}
\subsection[config\_AM\_eliassenPalm\_rhomax\_buoycoor]{\hyperref[sec:nm_tab_AM_eliassenPalm]{config\_AM\_eliassenPalm\_rhomax\_buoycoor}}
\label{subsec:nm_sec_config_AM_eliassenPalm_rhomax_buoycoor}
\begin{center}
\begin{longtable}{| p{2.0in} || p{4.0in} |}
    \hline
    Type: & real \\
    \hline
    Units: & \si{kg.m^-3} \\
    \hline
    Default Value: & 1080 \\
    \hline
    Possible Values: & any positive real value. \\
    \hline
    \caption{config\_AM\_eliassenPalm\_rhomax\_buoycoor: Maximum density used in defining the last buoyancy coordinate layer}
\end{longtable}
\end{center}
\section[AM\_mixedLayerDepths]{\hyperref[sec:nm_tab_AM_mixedLayerDepths]{AM\_mixedLayerDepths}}
\label{sec:nm_sec_AM_mixedLayerDepths}
\subsection[config\_AM\_mixedLayerDepths\_enable]{\hyperref[sec:nm_tab_AM_mixedLayerDepths]{config\_AM\_mixedLayerDepths\_enable}}
\label{subsec:nm_sec_config_AM_mixedLayerDepths_enable}
\begin{center}
\begin{longtable}{| p{2.0in} || p{4.0in} |}
    \hline
    Type: & logical \\
    \hline
    Units: & -- \\
    \hline
    Default Value: & .false. \\
    \hline
    Possible Values: & .true. or .false. \\
    \hline
    \caption{config\_AM\_mixedLayerDepths\_enable: If true, ocean analysis member mixedLayerDepth is called.}
\end{longtable}
\end{center}
\subsection[config\_AM\_mixedLayerDepths\_compute\_interval]{\hyperref[sec:nm_tab_AM_mixedLayerDepths]{config\_AM\_mixedLayerDepths\_compute\_interval}}
\label{subsec:nm_sec_config_AM_mixedLayerDepths_compute_interval}
\begin{center}
\begin{longtable}{| p{2.0in} || p{4.0in} |}
    \hline
    Type: & character \\
    \hline
    Units: & -- \\
    \hline
    Default Value: & output\_interval \\
    \hline
    Possible Values: & Any valid time stamp, 'dt', or 'output\_interval' \\
    \hline
    \caption{config\_AM\_mixedLayerDepths\_compute\_interval: Timestamp determining how often analysis member computation should be performed.}
\end{longtable}
\end{center}
\subsection[config\_AM\_mixedLayerDepths\_output\_stream]{\hyperref[sec:nm_tab_AM_mixedLayerDepths]{config\_AM\_mixedLayerDepths\_output\_stream}}
\label{subsec:nm_sec_config_AM_mixedLayerDepths_output_stream}
\begin{center}
\begin{longtable}{| p{2.0in} || p{4.0in} |}
    \hline
    Type: & character \\
    \hline
    Units: & -- \\
    \hline
    Default Value: & mixedLayerDepthsOutput \\
    \hline
    Possible Values: & Any existing stream name or 'none' \\
    \hline
    \caption{config\_AM\_mixedLayerDepths\_output\_stream: Name of the stream that the temPlate analysis member should be tied to.}
\end{longtable}
\end{center}
\subsection[config\_AM\_mixedLayerDepths\_write\_on\_startup]{\hyperref[sec:nm_tab_AM_mixedLayerDepths]{config\_AM\_mixedLayerDepths\_write\_on\_startup}}
\label{subsec:nm_sec_config_AM_mixedLayerDepths_write_on_startup}
\begin{center}
\begin{longtable}{| p{2.0in} || p{4.0in} |}
    \hline
    Type: & logical \\
    \hline
    Units: & -- \\
    \hline
    Default Value: & .true. \\
    \hline
    Possible Values: & .true. or .false. \\
    \hline
    \caption{config\_AM\_mixedLayerDepths\_write\_on\_startup: Logical flag determining if an analysis member write occurs on start-up.}
\end{longtable}
\end{center}
\subsection[config\_AM\_mixedLayerDepths\_compute\_on\_startup]{\hyperref[sec:nm_tab_AM_mixedLayerDepths]{config\_AM\_mixedLayerDepths\_compute\_on\_startup}}
\label{subsec:nm_sec_config_AM_mixedLayerDepths_compute_on_startup}
\begin{center}
\begin{longtable}{| p{2.0in} || p{4.0in} |}
    \hline
    Type: & logical \\
    \hline
    Units: & -- \\
    \hline
    Default Value: & .true. \\
    \hline
    Possible Values: & .true. or .false. \\
    \hline
    \caption{config\_AM\_mixedLayerDepths\_compute\_on\_startup: Logical flag determining if an analysis member computation occurs on start-up}
\end{longtable}
\end{center}
\subsection[config\_AM\_mixedLayerDepths\_Tthreshold]{\hyperref[sec:nm_tab_AM_mixedLayerDepths]{config\_AM\_mixedLayerDepths\_Tthreshold}}
\label{subsec:nm_sec_config_AM_mixedLayerDepths_Tthreshold}
\begin{center}
\begin{longtable}{| p{2.0in} || p{4.0in} |}
    \hline
    Type: & logical \\
    \hline
    Units: & -- \\
    \hline
    Default Value: & .true. \\
    \hline
    Possible Values: & .true. or .false. \\
    \hline
    \caption{config\_AM\_mixedLayerDepths\_Tthreshold: Logical flag that determines if MLDs are calculated using a critical temperature threshold}
\end{longtable}
\end{center}
\subsection[config\_AM\_mixedLayerDepths\_crit\_temp\_threshold]{\hyperref[sec:nm_tab_AM_mixedLayerDepths]{config\_AM\_mixedLayerDepths\_crit\_temp\_threshold}}
\label{subsec:nm_sec_config_AM_mixedLayerDepths_crit_temp_threshold}
\begin{center}
\begin{longtable}{| p{2.0in} || p{4.0in} |}
    \hline
    Type: & real \\
    \hline
    Units: & \si{C} \\
    \hline
    Default Value: & 0.2 \\
    \hline
    Possible Values: & all real positive values, suggested range 0.2 less than or equal to thresh less than or equal to 1 \\
    \hline
    \caption{config\_AM\_mixedLayerDepths\_crit\_temp\_threshold: temperature change relative to surface for threshold method}
\end{longtable}
\end{center}
\subsection[config\_AM\_mixedLayerDepths\_reference\_pressure]{\hyperref[sec:nm_tab_AM_mixedLayerDepths]{config\_AM\_mixedLayerDepths\_reference\_pressure}}
\label{subsec:nm_sec_config_AM_mixedLayerDepths_reference_pressure}
\begin{center}
\begin{longtable}{| p{2.0in} || p{4.0in} |}
    \hline
    Type: & real \\
    \hline
    Units: & \si{N.m^-2} \\
    \hline
    Default Value: & 1.0E5 \\
    \hline
    Possible Values: & any positive real, suggested range .25E5 less than or equal to value less than or equal to 2E5 \\
    \hline
    \caption{config\_AM\_mixedLayerDepths\_reference\_pressure: reference pressure for threshold computation}
\end{longtable}
\end{center}
\subsection[config\_AM\_mixedLayerDepths\_Tgradient]{\hyperref[sec:nm_tab_AM_mixedLayerDepths]{config\_AM\_mixedLayerDepths\_Tgradient}}
\label{subsec:nm_sec_config_AM_mixedLayerDepths_Tgradient}
\begin{center}
\begin{longtable}{| p{2.0in} || p{4.0in} |}
    \hline
    Type: & logical \\
    \hline
    Units: & -- \\
    \hline
    Default Value: & .false. \\
    \hline
    Possible Values: & .true. or .false. \\
    \hline
    \caption{config\_AM\_mixedLayerDepths\_Tgradient: Logical flag controlling whether or not to compute MLDs via the temperature gradient}
\end{longtable}
\end{center}
\subsection[config\_AM\_mixedLayerDepths\_Dgradient]{\hyperref[sec:nm_tab_AM_mixedLayerDepths]{config\_AM\_mixedLayerDepths\_Dgradient}}
\label{subsec:nm_sec_config_AM_mixedLayerDepths_Dgradient}
\begin{center}
\begin{longtable}{| p{2.0in} || p{4.0in} |}
    \hline
    Type: & logical \\
    \hline
    Units: & -- \\
    \hline
    Default Value: & .false. \\
    \hline
    Possible Values: & .true. or .false. \\
    \hline
    \caption{config\_AM\_mixedLayerDepths\_Dgradient: Logical flag controlling whether or not to compute MLDs via the density gradient}
\end{longtable}
\end{center}
\subsection[config\_AM\_mixedLayerDepths\_temp\_gradient\_threshold]{\hyperref[sec:nm_tab_AM_mixedLayerDepths]{config\_AM\_mixedLayerDepths\_temp\_gradient\_threshold}}
\label{subsec:nm_sec_config_AM_mixedLayerDepths_temp_gradient_threshold}
\begin{center}
\begin{longtable}{| p{2.0in} || p{4.0in} |}
    \hline
    Type: & real \\
    \hline
    Units: & \si{C.Pa^-1} \\
    \hline
    Default Value: & 5E-7 \\
    \hline
    Possible Values: & all positive reals \\
    \hline
    \caption{config\_AM\_mixedLayerDepths\_temp\_gradient\_threshold: temp gradient crit value, if not exceeded max gradient used}
\end{longtable}
\end{center}
\subsection[config\_AM\_mixedLayerDepths\_den\_gradient\_threshold]{\hyperref[sec:nm_tab_AM_mixedLayerDepths]{config\_AM\_mixedLayerDepths\_den\_gradient\_threshold}}
\label{subsec:nm_sec_config_AM_mixedLayerDepths_den_gradient_threshold}
\begin{center}
\begin{longtable}{| p{2.0in} || p{4.0in} |}
    \hline
    Type: & real \\
    \hline
    Units: & \si{C.Pa^-1} \\
    \hline
    Default Value: & 5E-8 \\
    \hline
    Possible Values: & all positive reals \\
    \hline
    \caption{config\_AM\_mixedLayerDepths\_den\_gradient\_threshold: potential density gradient crit value.  If not exceeded max gradient used}
\end{longtable}
\end{center}
\subsection[config\_AM\_mixedLayerDepths\_interp\_method]{\hyperref[sec:nm_tab_AM_mixedLayerDepths]{config\_AM\_mixedLayerDepths\_interp\_method}}
\label{subsec:nm_sec_config_AM_mixedLayerDepths_interp_method}
\begin{center}
\begin{longtable}{| p{2.0in} || p{4.0in} |}
    \hline
    Type: & integer \\
    \hline
    Units: & -- \\
    \hline
    Default Value: & 1 \\
    \hline
    Possible Values: & 1=linear or 2=quadratic or 3=cubic \\
    \hline
    \caption{config\_AM\_mixedLayerDepths\_interp\_method: flag specifying which interpolation method to use in computations}
\end{longtable}
\end{center}
\section[AM\_regionalStatsDaily]{\hyperref[sec:nm_tab_AM_regionalStatsDaily]{AM\_regionalStatsDaily}}
\label{sec:nm_sec_AM_regionalStatsDaily}
\subsection[config\_AM\_regionalStatsDaily\_enable]{\hyperref[sec:nm_tab_AM_regionalStatsDaily]{config\_AM\_regionalStatsDaily\_enable}}
\label{subsec:nm_sec_config_AM_regionalStatsDaily_enable}
\begin{center}
\begin{longtable}{| p{2.0in} || p{4.0in} |}
    \hline
    Type: & logical \\
    \hline
    Units: & -- \\
    \hline
    Default Value: & .false. \\
    \hline
    Possible Values: & .true. or .false. \\
    \hline
    \caption{config\_AM\_regionalStatsDaily\_enable: If true, ocean analysis member regional stats is called.}
\end{longtable}
\end{center}
\subsection[config\_AM\_regionalStatsDaily\_compute\_on\_startup]{\hyperref[sec:nm_tab_AM_regionalStatsDaily]{config\_AM\_regionalStatsDaily\_compute\_on\_startup}}
\label{subsec:nm_sec_config_AM_regionalStatsDaily_compute_on_startup}
\begin{center}
\begin{longtable}{| p{2.0in} || p{4.0in} |}
    \hline
    Type: & logical \\
    \hline
    Units: & -- \\
    \hline
    Default Value: & .false. \\
    \hline
    Possible Values: & .true. or .false. \\
    \hline
    \caption{config\_AM\_regionalStatsDaily\_compute\_on\_startup: Logical flag determining if an analysis member computation occurs on start-up.}
\end{longtable}
\end{center}
\subsection[config\_AM\_regionalStatsDaily\_write\_on\_startup]{\hyperref[sec:nm_tab_AM_regionalStatsDaily]{config\_AM\_regionalStatsDaily\_write\_on\_startup}}
\label{subsec:nm_sec_config_AM_regionalStatsDaily_write_on_startup}
\begin{center}
\begin{longtable}{| p{2.0in} || p{4.0in} |}
    \hline
    Type: & logical \\
    \hline
    Units: & -- \\
    \hline
    Default Value: & .false. \\
    \hline
    Possible Values: & .true. or .false. \\
    \hline
    \caption{config\_AM\_regionalStatsDaily\_write\_on\_startup: Logical flag determining if an analysis member output occurs on start-up.}
\end{longtable}
\end{center}
\subsection[config\_AM\_regionalStatsDaily\_compute\_interval]{\hyperref[sec:nm_tab_AM_regionalStatsDaily]{config\_AM\_regionalStatsDaily\_compute\_interval}}
\label{subsec:nm_sec_config_AM_regionalStatsDaily_compute_interval}
\begin{center}
\begin{longtable}{| p{2.0in} || p{4.0in} |}
    \hline
    Type: & character \\
    \hline
    Units: & -- \\
    \hline
    Default Value: & output\_interval \\
    \hline
    Possible Values: & Any valid time stamp, 'output\_interval', or 'dt'. \\
    \hline
    \caption{config\_AM\_regionalStatsDaily\_compute\_interval: Interval that determines frequency of computation for the regional stats analysis member.}
\end{longtable}
\end{center}
\subsection[config\_AM\_regionalStatsDaily\_output\_stream]{\hyperref[sec:nm_tab_AM_regionalStatsDaily]{config\_AM\_regionalStatsDaily\_output\_stream}}
\label{subsec:nm_sec_config_AM_regionalStatsDaily_output_stream}
\begin{center}
\begin{longtable}{| p{2.0in} || p{4.0in} |}
    \hline
    Type: & character \\
    \hline
    Units: & -- \\
    \hline
    Default Value: & regionalStatsDailyOutput \\
    \hline
    Possible Values: & An existing output stream that will be read for input fields. Cannot be 'none', like other analysis members. \\
    \hline
    \caption{config\_AM\_regionalStatsDaily\_output\_stream: Name of stream the regional stats analysis member will operate on that contains the list of input fields (and will be modified to contain the output stats fields).}
\end{longtable}
\end{center}
\subsection[config\_AM\_regionalStatsDaily\_restart\_stream]{\hyperref[sec:nm_tab_AM_regionalStatsDaily]{config\_AM\_regionalStatsDaily\_restart\_stream}}
\label{subsec:nm_sec_config_AM_regionalStatsDaily_restart_stream}
\begin{center}
\begin{longtable}{| p{2.0in} || p{4.0in} |}
    \hline
    Type: & character \\
    \hline
    Units: & -- \\
    \hline
    Default Value: & regionalMasksInput \\
    \hline
    Possible Values: & An existing input stream that will be read for regions/masks. Cannot be 'none', like other analysis members, and should be the same as input\_stream to ensure the masks are read on start or restart. \\
    \hline
    \caption{config\_AM\_regionalStatsDaily\_restart\_stream: Name of stream the regional stats analysis member will use for the mask/region data.}
\end{longtable}
\end{center}
\subsection[config\_AM\_regionalStatsDaily\_input\_stream]{\hyperref[sec:nm_tab_AM_regionalStatsDaily]{config\_AM\_regionalStatsDaily\_input\_stream}}
\label{subsec:nm_sec_config_AM_regionalStatsDaily_input_stream}
\begin{center}
\begin{longtable}{| p{2.0in} || p{4.0in} |}
    \hline
    Type: & character \\
    \hline
    Units: & -- \\
    \hline
    Default Value: & regionalMasksInput \\
    \hline
    Possible Values: & An existing input stream that will be read for regions/masks. Cannot be 'none', like other analysis members, and should be the same as restart\_stream to ensure the masks are read on start or restart. \\
    \hline
    \caption{config\_AM\_regionalStatsDaily\_input\_stream: Name of stream the regional stats analysis member will use for the mask/region data.}
\end{longtable}
\end{center}
\subsection[config\_AM\_regionalStatsDaily\_operation]{\hyperref[sec:nm_tab_AM_regionalStatsDaily]{config\_AM\_regionalStatsDaily\_operation}}
\label{subsec:nm_sec_config_AM_regionalStatsDaily_operation}
\begin{center}
\begin{longtable}{| p{2.0in} || p{4.0in} |}
    \hline
    Type: & character \\
    \hline
    Units: & -- \\
    \hline
    Default Value: & avg \\
    \hline
    Possible Values: & An operation, where it can be 'avg', 'min', or 'max', 'sum', or 'sos' (sum of squares). \\
    \hline
    \caption{config\_AM\_regionalStatsDaily\_operation: An operation describing the statistic to apply to all variables in the output stream.}
\end{longtable}
\end{center}
\subsection[config\_AM\_regionalStatsDaily\_region\_type]{\hyperref[sec:nm_tab_AM_regionalStatsDaily]{config\_AM\_regionalStatsDaily\_region\_type}}
\label{subsec:nm_sec_config_AM_regionalStatsDaily_region_type}
\begin{center}
\begin{longtable}{| p{2.0in} || p{4.0in} |}
    \hline
    Type: & character \\
    \hline
    Units: & -- \\
    \hline
    Default Value: & cell \\
    \hline
    Possible Values: & The mask reduction dimension, where it can be 'cell' or 'vertex' \\
    \hline
    \caption{config\_AM\_regionalStatsDaily\_region\_type: The reduced dimension of the region masks that will be used during the regional stats operation. Needs to be the last dimension, and the same dimension as all of the reduced fields, weight fields, and masks.}
\end{longtable}
\end{center}
\subsection[config\_AM\_regionalStatsDaily\_region\_group]{\hyperref[sec:nm_tab_AM_regionalStatsDaily]{config\_AM\_regionalStatsDaily\_region\_group}}
\label{subsec:nm_sec_config_AM_regionalStatsDaily_region_group}
\begin{center}
\begin{longtable}{| p{2.0in} || p{4.0in} |}
    \hline
    Type: & character \\
    \hline
    Units: & -- \\
    \hline
    Default Value: & all \\
    \hline
    Possible Values: & a valid name in the region group names \\
    \hline
    \caption{config\_AM\_regionalStatsDaily\_region\_group: The name of the group of region masks that will be used to subset the mesh during the regional stats operation.}
\end{longtable}
\end{center}
\subsection[config\_AM\_regionalStatsDaily\_1d\_weighting\_function]{\hyperref[sec:nm_tab_AM_regionalStatsDaily]{config\_AM\_regionalStatsDaily\_1d\_weighting\_function}}
\label{subsec:nm_sec_config_AM_regionalStatsDaily_1d_weighting_function}
\begin{center}
\begin{longtable}{| p{2.0in} || p{4.0in} |}
    \hline
    Type: & character \\
    \hline
    Units: & -- \\
    \hline
    Default Value: & mul \\
    \hline
    Possible Values: & A weight operation, where it can be 'id' (for 'min', 'max', or 'avg') or 'mul' (where 'mul' is restricted to 'avg'). \\
    \hline
    \caption{config\_AM\_regionalStatsDaily\_1d\_weighting\_function: An operation applied to every element in a region WITHOUT a vertical dimension, with a 1D weighting field, prior to an average operation. The average is normalized by the sum of the weight field in the region (divided by the sum of regional weight values).}
\end{longtable}
\end{center}
\subsection[config\_AM\_regionalStatsDaily\_2d\_weighting\_function]{\hyperref[sec:nm_tab_AM_regionalStatsDaily]{config\_AM\_regionalStatsDaily\_2d\_weighting\_function}}
\label{subsec:nm_sec_config_AM_regionalStatsDaily_2d_weighting_function}
\begin{center}
\begin{longtable}{| p{2.0in} || p{4.0in} |}
    \hline
    Type: & character \\
    \hline
    Units: & -- \\
    \hline
    Default Value: & mul \\
    \hline
    Possible Values: & A weight operation, where it can be 'id' (for 'min', 'max', or 'avg') or 'mul' (where 'mul' is restricted to 'avg'). \\
    \hline
    \caption{config\_AM\_regionalStatsDaily\_2d\_weighting\_function: An operation applied to every element in a region WITH a vertical dimension, with a 2D weighting field, prior to an average operation. The average is normalized by the sum of the weight field in the region (divided by the sum of regional weight values).}
\end{longtable}
\end{center}
\subsection[config\_AM\_regionalStatsDaily\_1d\_weighting\_field]{\hyperref[sec:nm_tab_AM_regionalStatsDaily]{config\_AM\_regionalStatsDaily\_1d\_weighting\_field}}
\label{subsec:nm_sec_config_AM_regionalStatsDaily_1d_weighting_field}
\begin{center}
\begin{longtable}{| p{2.0in} || p{4.0in} |}
    \hline
    Type: & character \\
    \hline
    Units: & -- \\
    \hline
    Default Value: & areaCell \\
    \hline
    Possible Values: & 'none' if 1D weighting function is 'id', otherwise any valid 1D real field with the same horizontal dimensions as the region masks \\
    \hline
    \caption{config\_AM\_regionalStatsDaily\_1d\_weighting\_field: A 1D real field used in conjunction with the 1D weighting function, to be used as a weighting scale factor (like area).}
\end{longtable}
\end{center}
\subsection[config\_AM\_regionalStatsDaily\_2d\_weighting\_field]{\hyperref[sec:nm_tab_AM_regionalStatsDaily]{config\_AM\_regionalStatsDaily\_2d\_weighting\_field}}
\label{subsec:nm_sec_config_AM_regionalStatsDaily_2d_weighting_field}
\begin{center}
\begin{longtable}{| p{2.0in} || p{4.0in} |}
    \hline
    Type: & character \\
    \hline
    Units: & -- \\
    \hline
    Default Value: & volumeCell \\
    \hline
    Possible Values: & 'none' if weighting function is 'id', otherwise any valid 2D real field with the same horizontal dimensions as the region masks and the vertical mask (requires that there is a vertical mask and vertical dimension) \\
    \hline
    \caption{config\_AM\_regionalStatsDaily\_2d\_weighting\_field: A 2D real field used in conjunction with the 2D weighting function, to be used as a weighting scale factor (like area).}
\end{longtable}
\end{center}
\subsection[config\_AM\_regionalStatsDaily\_vertical\_mask]{\hyperref[sec:nm_tab_AM_regionalStatsDaily]{config\_AM\_regionalStatsDaily\_vertical\_mask}}
\label{subsec:nm_sec_config_AM_regionalStatsDaily_vertical_mask}
\begin{center}
\begin{longtable}{| p{2.0in} || p{4.0in} |}
    \hline
    Type: & character \\
    \hline
    Units: & -- \\
    \hline
    Default Value: & cellMask \\
    \hline
    Possible Values: & 'none' if no vertical mask field is to be used, otherwise any integer 2D field with the configured second vertical dimension \\
    \hline
    \caption{config\_AM\_regionalStatsDaily\_vertical\_mask: An additional 2D vertical integer mask field, which is used in conjunction with the regional masks. Used in cases when an input field has a second dimension that matches the vertical mask dimension.}
\end{longtable}
\end{center}
\subsection[config\_AM\_regionalStatsDaily\_vertical\_dimension]{\hyperref[sec:nm_tab_AM_regionalStatsDaily]{config\_AM\_regionalStatsDaily\_vertical\_dimension}}
\label{subsec:nm_sec_config_AM_regionalStatsDaily_vertical_dimension}
\begin{center}
\begin{longtable}{| p{2.0in} || p{4.0in} |}
    \hline
    Type: & character \\
    \hline
    Units: & -- \\
    \hline
    Default Value: & nVertLevels \\
    \hline
    Possible Values: & 'none' if no vertical mask field is to be used, otherwise the name of the second dimension in the vertical mask field (where the first has to be the element dimension) \\
    \hline
    \caption{config\_AM\_regionalStatsDaily\_vertical\_dimension: The second dimension to be used for additional vertical mask.}
\end{longtable}
\end{center}
\section[AM\_regionalStatsWeekly]{\hyperref[sec:nm_tab_AM_regionalStatsWeekly]{AM\_regionalStatsWeekly}}
\label{sec:nm_sec_AM_regionalStatsWeekly}
\subsection[config\_AM\_regionalStatsWeekly\_enable]{\hyperref[sec:nm_tab_AM_regionalStatsWeekly]{config\_AM\_regionalStatsWeekly\_enable}}
\label{subsec:nm_sec_config_AM_regionalStatsWeekly_enable}
\begin{center}
\begin{longtable}{| p{2.0in} || p{4.0in} |}
    \hline
    Type: & logical \\
    \hline
    Units: & -- \\
    \hline
    Default Value: & .false. \\
    \hline
    Possible Values: & .true. or .false. \\
    \hline
    \caption{config\_AM\_regionalStatsWeekly\_enable: If true, ocean analysis member regional stats is called.}
\end{longtable}
\end{center}
\subsection[config\_AM\_regionalStatsWeekly\_compute\_on\_startup]{\hyperref[sec:nm_tab_AM_regionalStatsWeekly]{config\_AM\_regionalStatsWeekly\_compute\_on\_startup}}
\label{subsec:nm_sec_config_AM_regionalStatsWeekly_compute_on_startup}
\begin{center}
\begin{longtable}{| p{2.0in} || p{4.0in} |}
    \hline
    Type: & logical \\
    \hline
    Units: & -- \\
    \hline
    Default Value: & .false. \\
    \hline
    Possible Values: & .true. or .false. \\
    \hline
    \caption{config\_AM\_regionalStatsWeekly\_compute\_on\_startup: Logical flag determining if an analysis member computation occurs on start-up.}
\end{longtable}
\end{center}
\subsection[config\_AM\_regionalStatsWeekly\_write\_on\_startup]{\hyperref[sec:nm_tab_AM_regionalStatsWeekly]{config\_AM\_regionalStatsWeekly\_write\_on\_startup}}
\label{subsec:nm_sec_config_AM_regionalStatsWeekly_write_on_startup}
\begin{center}
\begin{longtable}{| p{2.0in} || p{4.0in} |}
    \hline
    Type: & logical \\
    \hline
    Units: & -- \\
    \hline
    Default Value: & .false. \\
    \hline
    Possible Values: & .true. or .false. \\
    \hline
    \caption{config\_AM\_regionalStatsWeekly\_write\_on\_startup: Logical flag determining if an analysis member output occurs on start-up.}
\end{longtable}
\end{center}
\subsection[config\_AM\_regionalStatsWeekly\_compute\_interval]{\hyperref[sec:nm_tab_AM_regionalStatsWeekly]{config\_AM\_regionalStatsWeekly\_compute\_interval}}
\label{subsec:nm_sec_config_AM_regionalStatsWeekly_compute_interval}
\begin{center}
\begin{longtable}{| p{2.0in} || p{4.0in} |}
    \hline
    Type: & character \\
    \hline
    Units: & -- \\
    \hline
    Default Value: & output\_interval \\
    \hline
    Possible Values: & Any valid time stamp, 'output\_interval', or 'dt'. \\
    \hline
    \caption{config\_AM\_regionalStatsWeekly\_compute\_interval: Interval that determines frequency of computation for the regional stats analysis member.}
\end{longtable}
\end{center}
\subsection[config\_AM\_regionalStatsWeekly\_output\_stream]{\hyperref[sec:nm_tab_AM_regionalStatsWeekly]{config\_AM\_regionalStatsWeekly\_output\_stream}}
\label{subsec:nm_sec_config_AM_regionalStatsWeekly_output_stream}
\begin{center}
\begin{longtable}{| p{2.0in} || p{4.0in} |}
    \hline
    Type: & character \\
    \hline
    Units: & -- \\
    \hline
    Default Value: & regionalStatsWeeklyOutput \\
    \hline
    Possible Values: & An existing output stream that will be read for input fields. Cannot be 'none', like other analysis members. \\
    \hline
    \caption{config\_AM\_regionalStatsWeekly\_output\_stream: Name of stream the regional stats analysis member will operate on that contains the list of input fields (and will be modified to contain the output stats fields).}
\end{longtable}
\end{center}
\subsection[config\_AM\_regionalStatsWeekly\_restart\_stream]{\hyperref[sec:nm_tab_AM_regionalStatsWeekly]{config\_AM\_regionalStatsWeekly\_restart\_stream}}
\label{subsec:nm_sec_config_AM_regionalStatsWeekly_restart_stream}
\begin{center}
\begin{longtable}{| p{2.0in} || p{4.0in} |}
    \hline
    Type: & character \\
    \hline
    Units: & -- \\
    \hline
    Default Value: & regionalMasksInput \\
    \hline
    Possible Values: & An existing input stream that will be read for regions/masks. Cannot be 'none', like other analysis members, and should be the same as input\_stream to ensure the masks are read on start or restart. \\
    \hline
    \caption{config\_AM\_regionalStatsWeekly\_restart\_stream: Name of stream the regional stats analysis member will use for the mask/region data.}
\end{longtable}
\end{center}
\subsection[config\_AM\_regionalStatsWeekly\_input\_stream]{\hyperref[sec:nm_tab_AM_regionalStatsWeekly]{config\_AM\_regionalStatsWeekly\_input\_stream}}
\label{subsec:nm_sec_config_AM_regionalStatsWeekly_input_stream}
\begin{center}
\begin{longtable}{| p{2.0in} || p{4.0in} |}
    \hline
    Type: & character \\
    \hline
    Units: & -- \\
    \hline
    Default Value: & regionalMasksInput \\
    \hline
    Possible Values: & An existing input stream that will be read for regions/masks. Cannot be 'none', like other analysis members, and should be the same as restart\_stream to ensure the masks are read on start or restart. \\
    \hline
    \caption{config\_AM\_regionalStatsWeekly\_input\_stream: Name of stream the regional stats analysis member will use for the mask/region data.}
\end{longtable}
\end{center}
\subsection[config\_AM\_regionalStatsWeekly\_operation]{\hyperref[sec:nm_tab_AM_regionalStatsWeekly]{config\_AM\_regionalStatsWeekly\_operation}}
\label{subsec:nm_sec_config_AM_regionalStatsWeekly_operation}
\begin{center}
\begin{longtable}{| p{2.0in} || p{4.0in} |}
    \hline
    Type: & character \\
    \hline
    Units: & -- \\
    \hline
    Default Value: & avg \\
    \hline
    Possible Values: & An operation, where it can be 'avg', 'min', or 'max', 'sum', or 'sos' (sum of squares). \\
    \hline
    \caption{config\_AM\_regionalStatsWeekly\_operation: An operation describing the statistic to apply to all variables in the output stream.}
\end{longtable}
\end{center}
\subsection[config\_AM\_regionalStatsWeekly\_region\_type]{\hyperref[sec:nm_tab_AM_regionalStatsWeekly]{config\_AM\_regionalStatsWeekly\_region\_type}}
\label{subsec:nm_sec_config_AM_regionalStatsWeekly_region_type}
\begin{center}
\begin{longtable}{| p{2.0in} || p{4.0in} |}
    \hline
    Type: & character \\
    \hline
    Units: & -- \\
    \hline
    Default Value: & cell \\
    \hline
    Possible Values: & The mask reduction dimension, where it can be 'cell' or 'vertex' \\
    \hline
    \caption{config\_AM\_regionalStatsWeekly\_region\_type: The reduced dimension of the region masks that will be used during the regional stats operation. Needs to be the last dimension, and the same dimension as all of the reduced fields, weight fields, and masks.}
\end{longtable}
\end{center}
\subsection[config\_AM\_regionalStatsWeekly\_region\_group]{\hyperref[sec:nm_tab_AM_regionalStatsWeekly]{config\_AM\_regionalStatsWeekly\_region\_group}}
\label{subsec:nm_sec_config_AM_regionalStatsWeekly_region_group}
\begin{center}
\begin{longtable}{| p{2.0in} || p{4.0in} |}
    \hline
    Type: & character \\
    \hline
    Units: & -- \\
    \hline
    Default Value: & all \\
    \hline
    Possible Values: & a valid name in the region group names \\
    \hline
    \caption{config\_AM\_regionalStatsWeekly\_region\_group: The name of the group of region masks that will be used to subset the mesh during the regional stats operation.}
\end{longtable}
\end{center}
\subsection[config\_AM\_regionalStatsWeekly\_1d\_weighting\_function]{\hyperref[sec:nm_tab_AM_regionalStatsWeekly]{config\_AM\_regionalStatsWeekly\_1d\_weighting\_function}}
\label{subsec:nm_sec_config_AM_regionalStatsWeekly_1d_weighting_function}
\begin{center}
\begin{longtable}{| p{2.0in} || p{4.0in} |}
    \hline
    Type: & character \\
    \hline
    Units: & -- \\
    \hline
    Default Value: & mul \\
    \hline
    Possible Values: & A weight operation, where it can be 'id' (for 'min', 'max', or 'avg') or 'mul' (where 'mul' is restricted to 'avg'). \\
    \hline
    \caption{config\_AM\_regionalStatsWeekly\_1d\_weighting\_function: An operation applied to every element in a region WITHOUT a vertical dimension, with a 1D weighting field, prior to an average operation. The average is normalized by the sum of the weight field in the region (divided by the sum of regional weight values).}
\end{longtable}
\end{center}
\subsection[config\_AM\_regionalStatsWeekly\_2d\_weighting\_function]{\hyperref[sec:nm_tab_AM_regionalStatsWeekly]{config\_AM\_regionalStatsWeekly\_2d\_weighting\_function}}
\label{subsec:nm_sec_config_AM_regionalStatsWeekly_2d_weighting_function}
\begin{center}
\begin{longtable}{| p{2.0in} || p{4.0in} |}
    \hline
    Type: & character \\
    \hline
    Units: & -- \\
    \hline
    Default Value: & mul \\
    \hline
    Possible Values: & A weight operation, where it can be 'id' (for 'min', 'max', or 'avg') or 'mul' (where 'mul' is restricted to 'avg'). \\
    \hline
    \caption{config\_AM\_regionalStatsWeekly\_2d\_weighting\_function: An operation applied to every element in a region WITH a vertical dimension, with a 2D weighting field, prior to an average operation. The average is normalized by the sum of the weight field in the region (divided by the sum of regional weight values).}
\end{longtable}
\end{center}
\subsection[config\_AM\_regionalStatsWeekly\_1d\_weighting\_field]{\hyperref[sec:nm_tab_AM_regionalStatsWeekly]{config\_AM\_regionalStatsWeekly\_1d\_weighting\_field}}
\label{subsec:nm_sec_config_AM_regionalStatsWeekly_1d_weighting_field}
\begin{center}
\begin{longtable}{| p{2.0in} || p{4.0in} |}
    \hline
    Type: & character \\
    \hline
    Units: & -- \\
    \hline
    Default Value: & areaCell \\
    \hline
    Possible Values: & 'none' if 1D weighting function is 'id', otherwise any valid 1D real field with the same horizontal dimensions as the region masks \\
    \hline
    \caption{config\_AM\_regionalStatsWeekly\_1d\_weighting\_field: A 1D real field used in conjunction with the 1D weighting function, to be used as a weighting scale factor (like area).}
\end{longtable}
\end{center}
\subsection[config\_AM\_regionalStatsWeekly\_2d\_weighting\_field]{\hyperref[sec:nm_tab_AM_regionalStatsWeekly]{config\_AM\_regionalStatsWeekly\_2d\_weighting\_field}}
\label{subsec:nm_sec_config_AM_regionalStatsWeekly_2d_weighting_field}
\begin{center}
\begin{longtable}{| p{2.0in} || p{4.0in} |}
    \hline
    Type: & character \\
    \hline
    Units: & -- \\
    \hline
    Default Value: & volumeCell \\
    \hline
    Possible Values: & 'none' if weighting function is 'id', otherwise any valid 2D real field with the same horizontal dimensions as the region masks and the vertical mask (requires that there is a vertical mask and vertical dimension) \\
    \hline
    \caption{config\_AM\_regionalStatsWeekly\_2d\_weighting\_field: A 2D real field used in conjunction with the 2D weighting function, to be used as a weighting scale factor (like area).}
\end{longtable}
\end{center}
\subsection[config\_AM\_regionalStatsWeekly\_vertical\_mask]{\hyperref[sec:nm_tab_AM_regionalStatsWeekly]{config\_AM\_regionalStatsWeekly\_vertical\_mask}}
\label{subsec:nm_sec_config_AM_regionalStatsWeekly_vertical_mask}
\begin{center}
\begin{longtable}{| p{2.0in} || p{4.0in} |}
    \hline
    Type: & character \\
    \hline
    Units: & -- \\
    \hline
    Default Value: & cellMask \\
    \hline
    Possible Values: & 'none' if no vertical mask field is to be used, otherwise any integer 2D field with the configured second vertical dimension \\
    \hline
    \caption{config\_AM\_regionalStatsWeekly\_vertical\_mask: An additional 2D vertical integer mask field, which is used in conjunction with the regional masks. Used in cases when an input field has a second dimension that matches the vertical mask dimension.}
\end{longtable}
\end{center}
\subsection[config\_AM\_regionalStatsWeekly\_vertical\_dimension]{\hyperref[sec:nm_tab_AM_regionalStatsWeekly]{config\_AM\_regionalStatsWeekly\_vertical\_dimension}}
\label{subsec:nm_sec_config_AM_regionalStatsWeekly_vertical_dimension}
\begin{center}
\begin{longtable}{| p{2.0in} || p{4.0in} |}
    \hline
    Type: & character \\
    \hline
    Units: & -- \\
    \hline
    Default Value: & nVertLevels \\
    \hline
    Possible Values: & 'none' if no vertical mask field is to be used, otherwise the name of the second dimension in the vertical mask field (where the first has to be the element dimension) \\
    \hline
    \caption{config\_AM\_regionalStatsWeekly\_vertical\_dimension: The second dimension to be used for additional vertical mask.}
\end{longtable}
\end{center}
\section[AM\_regionalStatsMonthly]{\hyperref[sec:nm_tab_AM_regionalStatsMonthly]{AM\_regionalStatsMonthly}}
\label{sec:nm_sec_AM_regionalStatsMonthly}
\subsection[config\_AM\_regionalStatsMonthly\_enable]{\hyperref[sec:nm_tab_AM_regionalStatsMonthly]{config\_AM\_regionalStatsMonthly\_enable}}
\label{subsec:nm_sec_config_AM_regionalStatsMonthly_enable}
\begin{center}
\begin{longtable}{| p{2.0in} || p{4.0in} |}
    \hline
    Type: & logical \\
    \hline
    Units: & -- \\
    \hline
    Default Value: & .false. \\
    \hline
    Possible Values: & .true. or .false. \\
    \hline
    \caption{config\_AM\_regionalStatsMonthly\_enable: If true, ocean analysis member regional stats is called.}
\end{longtable}
\end{center}
\subsection[config\_AM\_regionalStatsMonthly\_compute\_on\_startup]{\hyperref[sec:nm_tab_AM_regionalStatsMonthly]{config\_AM\_regionalStatsMonthly\_compute\_on\_startup}}
\label{subsec:nm_sec_config_AM_regionalStatsMonthly_compute_on_startup}
\begin{center}
\begin{longtable}{| p{2.0in} || p{4.0in} |}
    \hline
    Type: & logical \\
    \hline
    Units: & -- \\
    \hline
    Default Value: & .false. \\
    \hline
    Possible Values: & .true. or .false. \\
    \hline
    \caption{config\_AM\_regionalStatsMonthly\_compute\_on\_startup: Logical flag determining if an analysis member computation occurs on start-up.}
\end{longtable}
\end{center}
\subsection[config\_AM\_regionalStatsMonthly\_write\_on\_startup]{\hyperref[sec:nm_tab_AM_regionalStatsMonthly]{config\_AM\_regionalStatsMonthly\_write\_on\_startup}}
\label{subsec:nm_sec_config_AM_regionalStatsMonthly_write_on_startup}
\begin{center}
\begin{longtable}{| p{2.0in} || p{4.0in} |}
    \hline
    Type: & logical \\
    \hline
    Units: & -- \\
    \hline
    Default Value: & .false. \\
    \hline
    Possible Values: & .true. or .false. \\
    \hline
    \caption{config\_AM\_regionalStatsMonthly\_write\_on\_startup: Logical flag determining if an analysis member output occurs on start-up.}
\end{longtable}
\end{center}
\subsection[config\_AM\_regionalStatsMonthly\_compute\_interval]{\hyperref[sec:nm_tab_AM_regionalStatsMonthly]{config\_AM\_regionalStatsMonthly\_compute\_interval}}
\label{subsec:nm_sec_config_AM_regionalStatsMonthly_compute_interval}
\begin{center}
\begin{longtable}{| p{2.0in} || p{4.0in} |}
    \hline
    Type: & character \\
    \hline
    Units: & -- \\
    \hline
    Default Value: & output\_interval \\
    \hline
    Possible Values: & Any valid time stamp, 'output\_interval', or 'dt'. \\
    \hline
    \caption{config\_AM\_regionalStatsMonthly\_compute\_interval: Interval that determines frequency of computation for the regional stats analysis member.}
\end{longtable}
\end{center}
\subsection[config\_AM\_regionalStatsMonthly\_output\_stream]{\hyperref[sec:nm_tab_AM_regionalStatsMonthly]{config\_AM\_regionalStatsMonthly\_output\_stream}}
\label{subsec:nm_sec_config_AM_regionalStatsMonthly_output_stream}
\begin{center}
\begin{longtable}{| p{2.0in} || p{4.0in} |}
    \hline
    Type: & character \\
    \hline
    Units: & -- \\
    \hline
    Default Value: & regionalStatsMonthlyOutput \\
    \hline
    Possible Values: & An existing output stream that will be read for input fields. Cannot be 'none', like other analysis members. \\
    \hline
    \caption{config\_AM\_regionalStatsMonthly\_output\_stream: Name of stream the regional stats analysis member will operate on that contains the list of input fields (and will be modified to contain the output stats fields).}
\end{longtable}
\end{center}
\subsection[config\_AM\_regionalStatsMonthly\_restart\_stream]{\hyperref[sec:nm_tab_AM_regionalStatsMonthly]{config\_AM\_regionalStatsMonthly\_restart\_stream}}
\label{subsec:nm_sec_config_AM_regionalStatsMonthly_restart_stream}
\begin{center}
\begin{longtable}{| p{2.0in} || p{4.0in} |}
    \hline
    Type: & character \\
    \hline
    Units: & -- \\
    \hline
    Default Value: & regionalMasksInput \\
    \hline
    Possible Values: & An existing input stream that will be read for regions/masks. Cannot be 'none', like other analysis members, and should be the same as input\_stream to ensure the masks are read on start or restart. \\
    \hline
    \caption{config\_AM\_regionalStatsMonthly\_restart\_stream: Name of stream the regional stats analysis member will use for the mask/region data.}
\end{longtable}
\end{center}
\subsection[config\_AM\_regionalStatsMonthly\_input\_stream]{\hyperref[sec:nm_tab_AM_regionalStatsMonthly]{config\_AM\_regionalStatsMonthly\_input\_stream}}
\label{subsec:nm_sec_config_AM_regionalStatsMonthly_input_stream}
\begin{center}
\begin{longtable}{| p{2.0in} || p{4.0in} |}
    \hline
    Type: & character \\
    \hline
    Units: & -- \\
    \hline
    Default Value: & regionalMasksInput \\
    \hline
    Possible Values: & An existing input stream that will be read for regions/masks. Cannot be 'none', like other analysis members, and should be the same as restart\_stream to ensure the masks are read on start or restart. \\
    \hline
    \caption{config\_AM\_regionalStatsMonthly\_input\_stream: Name of stream the regional stats analysis member will use for the mask/region data.}
\end{longtable}
\end{center}
\subsection[config\_AM\_regionalStatsMonthly\_operation]{\hyperref[sec:nm_tab_AM_regionalStatsMonthly]{config\_AM\_regionalStatsMonthly\_operation}}
\label{subsec:nm_sec_config_AM_regionalStatsMonthly_operation}
\begin{center}
\begin{longtable}{| p{2.0in} || p{4.0in} |}
    \hline
    Type: & character \\
    \hline
    Units: & -- \\
    \hline
    Default Value: & avg \\
    \hline
    Possible Values: & An operation, where it can be 'avg', 'min', or 'max', 'sum', or 'sos' (sum of squares). \\
    \hline
    \caption{config\_AM\_regionalStatsMonthly\_operation: An operation describing the statistic to apply to all variables in the output stream.}
\end{longtable}
\end{center}
\subsection[config\_AM\_regionalStatsMonthly\_region\_type]{\hyperref[sec:nm_tab_AM_regionalStatsMonthly]{config\_AM\_regionalStatsMonthly\_region\_type}}
\label{subsec:nm_sec_config_AM_regionalStatsMonthly_region_type}
\begin{center}
\begin{longtable}{| p{2.0in} || p{4.0in} |}
    \hline
    Type: & character \\
    \hline
    Units: & -- \\
    \hline
    Default Value: & cell \\
    \hline
    Possible Values: & The mask reduction dimension, where it can be 'cell' or 'vertex' \\
    \hline
    \caption{config\_AM\_regionalStatsMonthly\_region\_type: The reduced dimension of the region masks that will be used during the regional stats operation. Needs to be the last dimension, and the same dimension as all of the reduced fields, weight fields, and masks.}
\end{longtable}
\end{center}
\subsection[config\_AM\_regionalStatsMonthly\_region\_group]{\hyperref[sec:nm_tab_AM_regionalStatsMonthly]{config\_AM\_regionalStatsMonthly\_region\_group}}
\label{subsec:nm_sec_config_AM_regionalStatsMonthly_region_group}
\begin{center}
\begin{longtable}{| p{2.0in} || p{4.0in} |}
    \hline
    Type: & character \\
    \hline
    Units: & -- \\
    \hline
    Default Value: & all \\
    \hline
    Possible Values: & a valid name in the region group names \\
    \hline
    \caption{config\_AM\_regionalStatsMonthly\_region\_group: The name of the group of region masks that will be used to subset the mesh during the regional stats operation.}
\end{longtable}
\end{center}
\subsection[config\_AM\_regionalStatsMonthly\_1d\_weighting\_function]{\hyperref[sec:nm_tab_AM_regionalStatsMonthly]{config\_AM\_regionalStatsMonthly\_1d\_weighting\_function}}
\label{subsec:nm_sec_config_AM_regionalStatsMonthly_1d_weighting_function}
\begin{center}
\begin{longtable}{| p{2.0in} || p{4.0in} |}
    \hline
    Type: & character \\
    \hline
    Units: & -- \\
    \hline
    Default Value: & mul \\
    \hline
    Possible Values: & A weight operation, where it can be 'id' (for 'min', 'max', or 'avg') or 'mul' (where 'mul' is restricted to 'avg'). \\
    \hline
    \caption{config\_AM\_regionalStatsMonthly\_1d\_weighting\_function: An operation applied to every element in a region WITHOUT a vertical dimension, with a 1D weighting field, prior to an average operation. The average is normalized by the sum of the weight field in the region (divided by the sum of regional weight values).}
\end{longtable}
\end{center}
\subsection[config\_AM\_regionalStatsMonthly\_2d\_weighting\_function]{\hyperref[sec:nm_tab_AM_regionalStatsMonthly]{config\_AM\_regionalStatsMonthly\_2d\_weighting\_function}}
\label{subsec:nm_sec_config_AM_regionalStatsMonthly_2d_weighting_function}
\begin{center}
\begin{longtable}{| p{2.0in} || p{4.0in} |}
    \hline
    Type: & character \\
    \hline
    Units: & -- \\
    \hline
    Default Value: & mul \\
    \hline
    Possible Values: & A weight operation, where it can be 'id' (for 'min', 'max', or 'avg') or 'mul' (where 'mul' is restricted to 'avg'). \\
    \hline
    \caption{config\_AM\_regionalStatsMonthly\_2d\_weighting\_function: An operation applied to every element in a region WITH a vertical dimension, with a 2D weighting field, prior to an average operation. The average is normalized by the sum of the weight field in the region (divided by the sum of regional weight values).}
\end{longtable}
\end{center}
\subsection[config\_AM\_regionalStatsMonthly\_1d\_weighting\_field]{\hyperref[sec:nm_tab_AM_regionalStatsMonthly]{config\_AM\_regionalStatsMonthly\_1d\_weighting\_field}}
\label{subsec:nm_sec_config_AM_regionalStatsMonthly_1d_weighting_field}
\begin{center}
\begin{longtable}{| p{2.0in} || p{4.0in} |}
    \hline
    Type: & character \\
    \hline
    Units: & -- \\
    \hline
    Default Value: & areaCell \\
    \hline
    Possible Values: & 'none' if 1D weighting function is 'id', otherwise any valid 1D real field with the same horizontal dimensions as the region masks \\
    \hline
    \caption{config\_AM\_regionalStatsMonthly\_1d\_weighting\_field: A 1D real field used in conjunction with the 1D weighting function, to be used as a weighting scale factor (like area).}
\end{longtable}
\end{center}
\subsection[config\_AM\_regionalStatsMonthly\_2d\_weighting\_field]{\hyperref[sec:nm_tab_AM_regionalStatsMonthly]{config\_AM\_regionalStatsMonthly\_2d\_weighting\_field}}
\label{subsec:nm_sec_config_AM_regionalStatsMonthly_2d_weighting_field}
\begin{center}
\begin{longtable}{| p{2.0in} || p{4.0in} |}
    \hline
    Type: & character \\
    \hline
    Units: & -- \\
    \hline
    Default Value: & volumeCell \\
    \hline
    Possible Values: & 'none' if weighting function is 'id', otherwise any valid 2D real field with the same horizontal dimensions as the region masks and the vertical mask (requires that there is a vertical mask and vertical dimension) \\
    \hline
    \caption{config\_AM\_regionalStatsMonthly\_2d\_weighting\_field: A 2D real field used in conjunction with the 2D weighting function, to be used as a weighting scale factor (like area).}
\end{longtable}
\end{center}
\subsection[config\_AM\_regionalStatsMonthly\_vertical\_mask]{\hyperref[sec:nm_tab_AM_regionalStatsMonthly]{config\_AM\_regionalStatsMonthly\_vertical\_mask}}
\label{subsec:nm_sec_config_AM_regionalStatsMonthly_vertical_mask}
\begin{center}
\begin{longtable}{| p{2.0in} || p{4.0in} |}
    \hline
    Type: & character \\
    \hline
    Units: & -- \\
    \hline
    Default Value: & cellMask \\
    \hline
    Possible Values: & 'none' if no vertical mask field is to be used, otherwise any integer 2D field with the configured second vertical dimension \\
    \hline
    \caption{config\_AM\_regionalStatsMonthly\_vertical\_mask: An additional 2D vertical integer mask field, which is used in conjunction with the regional masks. Used in cases when an input field has a second dimension that matches the vertical mask dimension.}
\end{longtable}
\end{center}
\subsection[config\_AM\_regionalStatsMonthly\_vertical\_dimension]{\hyperref[sec:nm_tab_AM_regionalStatsMonthly]{config\_AM\_regionalStatsMonthly\_vertical\_dimension}}
\label{subsec:nm_sec_config_AM_regionalStatsMonthly_vertical_dimension}
\begin{center}
\begin{longtable}{| p{2.0in} || p{4.0in} |}
    \hline
    Type: & character \\
    \hline
    Units: & -- \\
    \hline
    Default Value: & nVertLevels \\
    \hline
    Possible Values: & 'none' if no vertical mask field is to be used, otherwise the name of the second dimension in the vertical mask field (where the first has to be the element dimension) \\
    \hline
    \caption{config\_AM\_regionalStatsMonthly\_vertical\_dimension: The second dimension to be used for additional vertical mask.}
\end{longtable}
\end{center}
\section[AM\_regionalStatsCustom]{\hyperref[sec:nm_tab_AM_regionalStatsCustom]{AM\_regionalStatsCustom}}
\label{sec:nm_sec_AM_regionalStatsCustom}
\subsection[config\_AM\_regionalStatsCustom\_enable]{\hyperref[sec:nm_tab_AM_regionalStatsCustom]{config\_AM\_regionalStatsCustom\_enable}}
\label{subsec:nm_sec_config_AM_regionalStatsCustom_enable}
\begin{center}
\begin{longtable}{| p{2.0in} || p{4.0in} |}
    \hline
    Type: & logical \\
    \hline
    Units: & -- \\
    \hline
    Default Value: & .false. \\
    \hline
    Possible Values: & .true. or .false. \\
    \hline
    \caption{config\_AM\_regionalStatsCustom\_enable: If true, ocean analysis member regional stats is called.}
\end{longtable}
\end{center}
\subsection[config\_AM\_regionalStatsCustom\_compute\_on\_startup]{\hyperref[sec:nm_tab_AM_regionalStatsCustom]{config\_AM\_regionalStatsCustom\_compute\_on\_startup}}
\label{subsec:nm_sec_config_AM_regionalStatsCustom_compute_on_startup}
\begin{center}
\begin{longtable}{| p{2.0in} || p{4.0in} |}
    \hline
    Type: & logical \\
    \hline
    Units: & -- \\
    \hline
    Default Value: & .false. \\
    \hline
    Possible Values: & .true. or .false. \\
    \hline
    \caption{config\_AM\_regionalStatsCustom\_compute\_on\_startup: Logical flag determining if an analysis member computation occurs on start-up.}
\end{longtable}
\end{center}
\subsection[config\_AM\_regionalStatsCustom\_write\_on\_startup]{\hyperref[sec:nm_tab_AM_regionalStatsCustom]{config\_AM\_regionalStatsCustom\_write\_on\_startup}}
\label{subsec:nm_sec_config_AM_regionalStatsCustom_write_on_startup}
\begin{center}
\begin{longtable}{| p{2.0in} || p{4.0in} |}
    \hline
    Type: & logical \\
    \hline
    Units: & -- \\
    \hline
    Default Value: & .false. \\
    \hline
    Possible Values: & .true. or .false. \\
    \hline
    \caption{config\_AM\_regionalStatsCustom\_write\_on\_startup: Logical flag determining if an analysis member output occurs on start-up.}
\end{longtable}
\end{center}
\subsection[config\_AM\_regionalStatsCustom\_compute\_interval]{\hyperref[sec:nm_tab_AM_regionalStatsCustom]{config\_AM\_regionalStatsCustom\_compute\_interval}}
\label{subsec:nm_sec_config_AM_regionalStatsCustom_compute_interval}
\begin{center}
\begin{longtable}{| p{2.0in} || p{4.0in} |}
    \hline
    Type: & character \\
    \hline
    Units: & -- \\
    \hline
    Default Value: & output\_interval \\
    \hline
    Possible Values: & Any valid time stamp, 'output\_interval', or 'dt'. \\
    \hline
    \caption{config\_AM\_regionalStatsCustom\_compute\_interval: Interval that determines frequency of computation for the regional stats analysis member.}
\end{longtable}
\end{center}
\subsection[config\_AM\_regionalStatsCustom\_output\_stream]{\hyperref[sec:nm_tab_AM_regionalStatsCustom]{config\_AM\_regionalStatsCustom\_output\_stream}}
\label{subsec:nm_sec_config_AM_regionalStatsCustom_output_stream}
\begin{center}
\begin{longtable}{| p{2.0in} || p{4.0in} |}
    \hline
    Type: & character \\
    \hline
    Units: & -- \\
    \hline
    Default Value: & regionalStatsCustomOutput \\
    \hline
    Possible Values: & An existing output stream that will be read for input fields. Cannot be 'none', like other analysis members. \\
    \hline
    \caption{config\_AM\_regionalStatsCustom\_output\_stream: Name of stream the regional stats analysis member will operate on that contains the list of input fields (and will be modified to contain the output stats fields).}
\end{longtable}
\end{center}
\subsection[config\_AM\_regionalStatsCustom\_restart\_stream]{\hyperref[sec:nm_tab_AM_regionalStatsCustom]{config\_AM\_regionalStatsCustom\_restart\_stream}}
\label{subsec:nm_sec_config_AM_regionalStatsCustom_restart_stream}
\begin{center}
\begin{longtable}{| p{2.0in} || p{4.0in} |}
    \hline
    Type: & character \\
    \hline
    Units: & -- \\
    \hline
    Default Value: & regionalMasksInput \\
    \hline
    Possible Values: & An existing input stream that will be read for regions/masks. Cannot be 'none', like other analysis members, and should be the same as input\_stream to ensure the masks are read on start or restart. \\
    \hline
    \caption{config\_AM\_regionalStatsCustom\_restart\_stream: Name of stream the regional stats analysis member will use for the mask/region data.}
\end{longtable}
\end{center}
\subsection[config\_AM\_regionalStatsCustom\_input\_stream]{\hyperref[sec:nm_tab_AM_regionalStatsCustom]{config\_AM\_regionalStatsCustom\_input\_stream}}
\label{subsec:nm_sec_config_AM_regionalStatsCustom_input_stream}
\begin{center}
\begin{longtable}{| p{2.0in} || p{4.0in} |}
    \hline
    Type: & character \\
    \hline
    Units: & -- \\
    \hline
    Default Value: & regionalMasksInput \\
    \hline
    Possible Values: & An existing input stream that will be read for regions/masks. Cannot be 'none', like other analysis members, and should be the same as restart\_stream to ensure the masks are read on start or restart. \\
    \hline
    \caption{config\_AM\_regionalStatsCustom\_input\_stream: Name of stream the regional stats analysis member will use for the mask/region data.}
\end{longtable}
\end{center}
\subsection[config\_AM\_regionalStatsCustom\_operation]{\hyperref[sec:nm_tab_AM_regionalStatsCustom]{config\_AM\_regionalStatsCustom\_operation}}
\label{subsec:nm_sec_config_AM_regionalStatsCustom_operation}
\begin{center}
\begin{longtable}{| p{2.0in} || p{4.0in} |}
    \hline
    Type: & character \\
    \hline
    Units: & -- \\
    \hline
    Default Value: & avg \\
    \hline
    Possible Values: & An operation, where it can be 'avg', 'min', or 'max', 'sum', or 'sos' (sum of squares). \\
    \hline
    \caption{config\_AM\_regionalStatsCustom\_operation: An operation describing the statistic to apply to all variables in the output stream.}
\end{longtable}
\end{center}
\subsection[config\_AM\_regionalStatsCustom\_region\_type]{\hyperref[sec:nm_tab_AM_regionalStatsCustom]{config\_AM\_regionalStatsCustom\_region\_type}}
\label{subsec:nm_sec_config_AM_regionalStatsCustom_region_type}
\begin{center}
\begin{longtable}{| p{2.0in} || p{4.0in} |}
    \hline
    Type: & character \\
    \hline
    Units: & -- \\
    \hline
    Default Value: & cell \\
    \hline
    Possible Values: & The mask reduction dimension, where it can be 'cell' or 'vertex' \\
    \hline
    \caption{config\_AM\_regionalStatsCustom\_region\_type: The reduced dimension of the region masks that will be used during the regional stats operation. Needs to be the last dimension, and the same dimension as all of the reduced fields, weight fields, and masks.}
\end{longtable}
\end{center}
\subsection[config\_AM\_regionalStatsCustom\_region\_group]{\hyperref[sec:nm_tab_AM_regionalStatsCustom]{config\_AM\_regionalStatsCustom\_region\_group}}
\label{subsec:nm_sec_config_AM_regionalStatsCustom_region_group}
\begin{center}
\begin{longtable}{| p{2.0in} || p{4.0in} |}
    \hline
    Type: & character \\
    \hline
    Units: & -- \\
    \hline
    Default Value: & all \\
    \hline
    Possible Values: & a valid name in the region group names \\
    \hline
    \caption{config\_AM\_regionalStatsCustom\_region\_group: The name of the group of region masks that will be used to subset the mesh during the regional stats operation.}
\end{longtable}
\end{center}
\subsection[config\_AM\_regionalStatsCustom\_1d\_weighting\_function]{\hyperref[sec:nm_tab_AM_regionalStatsCustom]{config\_AM\_regionalStatsCustom\_1d\_weighting\_function}}
\label{subsec:nm_sec_config_AM_regionalStatsCustom_1d_weighting_function}
\begin{center}
\begin{longtable}{| p{2.0in} || p{4.0in} |}
    \hline
    Type: & character \\
    \hline
    Units: & -- \\
    \hline
    Default Value: & mul \\
    \hline
    Possible Values: & A weight operation, where it can be 'id' (for 'min', 'max', or 'avg') or 'mul' (where 'mul' is restricted to 'avg'). \\
    \hline
    \caption{config\_AM\_regionalStatsCustom\_1d\_weighting\_function: An operation applied to every element in a region WITHOUT a vertical dimension, with a 1D weighting field, prior to an average operation. The average is normalized by the sum of the weight field in the region (divided by the sum of regional weight values).}
\end{longtable}
\end{center}
\subsection[config\_AM\_regionalStatsCustom\_2d\_weighting\_function]{\hyperref[sec:nm_tab_AM_regionalStatsCustom]{config\_AM\_regionalStatsCustom\_2d\_weighting\_function}}
\label{subsec:nm_sec_config_AM_regionalStatsCustom_2d_weighting_function}
\begin{center}
\begin{longtable}{| p{2.0in} || p{4.0in} |}
    \hline
    Type: & character \\
    \hline
    Units: & -- \\
    \hline
    Default Value: & mul \\
    \hline
    Possible Values: & A weight operation, where it can be 'id' (for 'min', 'max', or 'avg') or 'mul' (where 'mul' is restricted to 'avg'). \\
    \hline
    \caption{config\_AM\_regionalStatsCustom\_2d\_weighting\_function: An operation applied to every element in a region WITH a vertical dimension, with a 2D weighting field, prior to an average operation. The average is normalized by the sum of the weight field in the region (divided by the sum of regional weight values).}
\end{longtable}
\end{center}
\subsection[config\_AM\_regionalStatsCustom\_1d\_weighting\_field]{\hyperref[sec:nm_tab_AM_regionalStatsCustom]{config\_AM\_regionalStatsCustom\_1d\_weighting\_field}}
\label{subsec:nm_sec_config_AM_regionalStatsCustom_1d_weighting_field}
\begin{center}
\begin{longtable}{| p{2.0in} || p{4.0in} |}
    \hline
    Type: & character \\
    \hline
    Units: & -- \\
    \hline
    Default Value: & areaCell \\
    \hline
    Possible Values: & 'none' if 1D weighting function is 'id', otherwise any valid 1D real field with the same horizontal dimensions as the region masks \\
    \hline
    \caption{config\_AM\_regionalStatsCustom\_1d\_weighting\_field: A 1D real field used in conjunction with the 1D weighting function, to be used as a weighting scale factor (like area).}
\end{longtable}
\end{center}
\subsection[config\_AM\_regionalStatsCustom\_2d\_weighting\_field]{\hyperref[sec:nm_tab_AM_regionalStatsCustom]{config\_AM\_regionalStatsCustom\_2d\_weighting\_field}}
\label{subsec:nm_sec_config_AM_regionalStatsCustom_2d_weighting_field}
\begin{center}
\begin{longtable}{| p{2.0in} || p{4.0in} |}
    \hline
    Type: & character \\
    \hline
    Units: & -- \\
    \hline
    Default Value: & volumeCell \\
    \hline
    Possible Values: & 'none' if weighting function is 'id', otherwise any valid 2D real field with the same horizontal dimensions as the region masks and the vertical mask (requires that there is a vertical mask and vertical dimension) \\
    \hline
    \caption{config\_AM\_regionalStatsCustom\_2d\_weighting\_field: A 2D real field used in conjunction with the 2D weighting function, to be used as a weighting scale factor (like area).}
\end{longtable}
\end{center}
\subsection[config\_AM\_regionalStatsCustom\_vertical\_mask]{\hyperref[sec:nm_tab_AM_regionalStatsCustom]{config\_AM\_regionalStatsCustom\_vertical\_mask}}
\label{subsec:nm_sec_config_AM_regionalStatsCustom_vertical_mask}
\begin{center}
\begin{longtable}{| p{2.0in} || p{4.0in} |}
    \hline
    Type: & character \\
    \hline
    Units: & -- \\
    \hline
    Default Value: & cellMask \\
    \hline
    Possible Values: & 'none' if no vertical mask field is to be used, otherwise any integer 2D field with the configured second vertical dimension \\
    \hline
    \caption{config\_AM\_regionalStatsCustom\_vertical\_mask: An additional 2D vertical integer mask field, which is used in conjunction with the regional masks. Used in cases when an input field has a second dimension that matches the vertical mask dimension.}
\end{longtable}
\end{center}
\subsection[config\_AM\_regionalStatsCustom\_vertical\_dimension]{\hyperref[sec:nm_tab_AM_regionalStatsCustom]{config\_AM\_regionalStatsCustom\_vertical\_dimension}}
\label{subsec:nm_sec_config_AM_regionalStatsCustom_vertical_dimension}
\begin{center}
\begin{longtable}{| p{2.0in} || p{4.0in} |}
    \hline
    Type: & character \\
    \hline
    Units: & -- \\
    \hline
    Default Value: & nVertLevels \\
    \hline
    Possible Values: & 'none' if no vertical mask field is to be used, otherwise the name of the second dimension in the vertical mask field (where the first has to be the element dimension) \\
    \hline
    \caption{config\_AM\_regionalStatsCustom\_vertical\_dimension: The second dimension to be used for additional vertical mask.}
\end{longtable}
\end{center}
\section[AM\_timeSeriesStatsDaily]{\hyperref[sec:nm_tab_AM_timeSeriesStatsDaily]{AM\_timeSeriesStatsDaily}}
\label{sec:nm_sec_AM_timeSeriesStatsDaily}
\subsection[config\_AM\_timeSeriesStatsDaily\_enable]{\hyperref[sec:nm_tab_AM_timeSeriesStatsDaily]{config\_AM\_timeSeriesStatsDaily\_enable}}
\label{subsec:nm_sec_config_AM_timeSeriesStatsDaily_enable}
\begin{center}
\begin{longtable}{| p{2.0in} || p{4.0in} |}
    \hline
    Type: & logical \\
    \hline
    Units: & -- \\
    \hline
    Default Value: & .false. \\
    \hline
    Possible Values: & .true. or .false. \\
    \hline
    \caption{config\_AM\_timeSeriesStatsDaily\_enable: If true, ocean analysis member time series stats is called.}
\end{longtable}
\end{center}
\subsection[config\_AM\_timeSeriesStatsDaily\_compute\_on\_startup]{\hyperref[sec:nm_tab_AM_timeSeriesStatsDaily]{config\_AM\_timeSeriesStatsDaily\_compute\_on\_startup}}
\label{subsec:nm_sec_config_AM_timeSeriesStatsDaily_compute_on_startup}
\begin{center}
\begin{longtable}{| p{2.0in} || p{4.0in} |}
    \hline
    Type: & logical \\
    \hline
    Units: & -- \\
    \hline
    Default Value: & .false. \\
    \hline
    Possible Values: & .true. or .false. \\
    \hline
    \caption{config\_AM\_timeSeriesStatsDaily\_compute\_on\_startup: Logical flag determining if an analysis member computation occurs on start-up. You likely want this off for this (time series) analysis member because it will accumulate any state prior to time stepping (double counting the last time step).}
\end{longtable}
\end{center}
\subsection[config\_AM\_timeSeriesStatsDaily\_write\_on\_startup]{\hyperref[sec:nm_tab_AM_timeSeriesStatsDaily]{config\_AM\_timeSeriesStatsDaily\_write\_on\_startup}}
\label{subsec:nm_sec_config_AM_timeSeriesStatsDaily_write_on_startup}
\begin{center}
\begin{longtable}{| p{2.0in} || p{4.0in} |}
    \hline
    Type: & logical \\
    \hline
    Units: & -- \\
    \hline
    Default Value: & .false. \\
    \hline
    Possible Values: & .true. or .false. \\
    \hline
    \caption{config\_AM\_timeSeriesStatsDaily\_write\_on\_startup: Logical flag determining if an analysis member output occurs on start-up.}
\end{longtable}
\end{center}
\subsection[config\_AM\_timeSeriesStatsDaily\_compute\_interval]{\hyperref[sec:nm_tab_AM_timeSeriesStatsDaily]{config\_AM\_timeSeriesStatsDaily\_compute\_interval}}
\label{subsec:nm_sec_config_AM_timeSeriesStatsDaily_compute_interval}
\begin{center}
\begin{longtable}{| p{2.0in} || p{4.0in} |}
    \hline
    Type: & character \\
    \hline
    Units: & -- \\
    \hline
    Default Value: & 00-00-00\_01:00:00 \\
    \hline
    Possible Values: & Any valid time stamp or 'dt'. This must also be less than or equal to output\_interval / 2 (i.e., requires at least two samples in a series). \\
    \hline
    \caption{config\_AM\_timeSeriesStatsDaily\_compute\_interval: Interval that determines frequency of computation for the time series stats analysis member.}
\end{longtable}
\end{center}
\subsection[config\_AM\_timeSeriesStatsDaily\_output\_stream]{\hyperref[sec:nm_tab_AM_timeSeriesStatsDaily]{config\_AM\_timeSeriesStatsDaily\_output\_stream}}
\label{subsec:nm_sec_config_AM_timeSeriesStatsDaily_output_stream}
\begin{center}
\begin{longtable}{| p{2.0in} || p{4.0in} |}
    \hline
    Type: & character \\
    \hline
    Units: & -- \\
    \hline
    Default Value: & timeSeriesStatsDailyOutput \\
    \hline
    Possible Values: & An existing stream that will be modified (existing real fields removed and new time series stats versions added) with time series stats outputs. Cannot be 'none', like other analysis members. \\
    \hline
    \caption{config\_AM\_timeSeriesStatsDaily\_output\_stream: Name of stream the time series stats analysis member will operate on.}
\end{longtable}
\end{center}
\subsection[config\_AM\_timeSeriesStatsDaily\_restart\_stream]{\hyperref[sec:nm_tab_AM_timeSeriesStatsDaily]{config\_AM\_timeSeriesStatsDaily\_restart\_stream}}
\label{subsec:nm_sec_config_AM_timeSeriesStatsDaily_restart_stream}
\begin{center}
\begin{longtable}{| p{2.0in} || p{4.0in} |}
    \hline
    Type: & character \\
    \hline
    Units: & -- \\
    \hline
    Default Value: & timeSeriesStatsDailyRestart \\
    \hline
    Possible Values: & A restart stream with state of the time series stats. \\
    \hline
    \caption{config\_AM\_timeSeriesStatsDaily\_restart\_stream: Name of the restart stream the time series stats analysis member will use to initialize itself if restart is enabled.}
\end{longtable}
\end{center}
\subsection[config\_AM\_timeSeriesStatsDaily\_operation]{\hyperref[sec:nm_tab_AM_timeSeriesStatsDaily]{config\_AM\_timeSeriesStatsDaily\_operation}}
\label{subsec:nm_sec_config_AM_timeSeriesStatsDaily_operation}
\begin{center}
\begin{longtable}{| p{2.0in} || p{4.0in} |}
    \hline
    Type: & character \\
    \hline
    Units: & -- \\
    \hline
    Default Value: & avg \\
    \hline
    Possible Values: & An operation, where it can be 'avg', 'min', or 'max', 'sum', or 'sos' (sum of squares). \\
    \hline
    \caption{config\_AM\_timeSeriesStatsDaily\_operation: An operation describing the statistic to apply to the time series for all variables in the output stream, reducing the time dimension.}
\end{longtable}
\end{center}
\subsection[config\_AM\_timeSeriesStatsDaily\_reference\_times]{\hyperref[sec:nm_tab_AM_timeSeriesStatsDaily]{config\_AM\_timeSeriesStatsDaily\_reference\_times}}
\label{subsec:nm_sec_config_AM_timeSeriesStatsDaily_reference_times}
\begin{center}
\begin{longtable}{| p{2.0in} || p{4.0in} |}
    \hline
    Type: & character \\
    \hline
    Units: & -- \\
    \hline
    Default Value: & initial\_time \\
    \hline
    Possible Values: & A list of absolute times or 'initial\_time's, separated by ;. \\
    \hline
    \caption{config\_AM\_timeSeriesStatsDaily\_reference\_times: A list of absolute times describing when to start accumulating statistics. Each time indicates the start of one time window (time series statistic) per variable, in the output stream (i.e., provide four start times if you want quarterly climatologies, only one time is needed for monthly or daily averages, etc.)}
\end{longtable}
\end{center}
\subsection[config\_AM\_timeSeriesStatsDaily\_duration\_intervals]{\hyperref[sec:nm_tab_AM_timeSeriesStatsDaily]{config\_AM\_timeSeriesStatsDaily\_duration\_intervals}}
\label{subsec:nm_sec_config_AM_timeSeriesStatsDaily_duration_intervals}
\begin{center}
\begin{longtable}{| p{2.0in} || p{4.0in} |}
    \hline
    Type: & character \\
    \hline
    Units: & -- \\
    \hline
    Default Value: & repeat\_interval \\
    \hline
    Possible Values: & A list of time durations in d\_h:m:s or 'repeat\_interval's, separated by ;. Each must be greater than or equal to compute\_interval * 2 and less than or equal to repeat\_interval. duration\_intervals less than repeat\_intervals allow for repeated statistics within the repeat\_interval (i.e., for climatologies) \\
    \hline
    \caption{config\_AM\_timeSeriesStatsDaily\_duration\_intervals: A list of time durations in d\_h:m:s describing how long to accumulate statistics in a time window for each repetition (repeat\_interval). It has to match the number of start time tokens in reference\_times.}
\end{longtable}
\end{center}
\subsection[config\_AM\_timeSeriesStatsDaily\_repeat\_intervals]{\hyperref[sec:nm_tab_AM_timeSeriesStatsDaily]{config\_AM\_timeSeriesStatsDaily\_repeat\_intervals}}
\label{subsec:nm_sec_config_AM_timeSeriesStatsDaily_repeat_intervals}
\begin{center}
\begin{longtable}{| p{2.0in} || p{4.0in} |}
    \hline
    Type: & character \\
    \hline
    Units: & -- \\
    \hline
    Default Value: & reset\_interval \\
    \hline
    Possible Values: & {\bf \color{red} MISSING} \\
    \hline
    \caption{config\_AM\_timeSeriesStatsDaily\_repeat\_intervals: A list of time durations in d\_h:m:s describing the accumulation statistic temporal periodicity (time between beginning to accumulate again after it started - duration\_interval describes when to stop after starting/restarting). It has to match the number of tokens in reference\_times.}
\end{longtable}
\end{center}
\subsection[config\_AM\_timeSeriesStatsDaily\_reset\_intervals]{\hyperref[sec:nm_tab_AM_timeSeriesStatsDaily]{config\_AM\_timeSeriesStatsDaily\_reset\_intervals}}
\label{subsec:nm_sec_config_AM_timeSeriesStatsDaily_reset_intervals}
\begin{center}
\begin{longtable}{| p{2.0in} || p{4.0in} |}
    \hline
    Type: & character \\
    \hline
    Units: & -- \\
    \hline
    Default Value: & 00-00-01\_00:00:00 \\
    \hline
    Possible Values: & A list of time durations in d\_h:m:s, separated by ;. Ought to be greater than or equal to output\_interval (not verified by the analysis member). \\
    \hline
    \caption{config\_AM\_timeSeriesStatsDaily\_reset\_intervals: A list of time durations in d\_h:m:s describing the statistic reset periodicity (how often to reset/clear/zero the accumulation). It has to match the number of tokens in reference\_times.}
\end{longtable}
\end{center}
\subsection[config\_AM\_timeSeriesStatsDaily\_backward\_output\_offset]{\hyperref[sec:nm_tab_AM_timeSeriesStatsDaily]{config\_AM\_timeSeriesStatsDaily\_backward\_output\_offset}}
\label{subsec:nm_sec_config_AM_timeSeriesStatsDaily_backward_output_offset}
\begin{center}
\begin{longtable}{| p{2.0in} || p{4.0in} |}
    \hline
    Type: & character \\
    \hline
    Units: & -- \\
    \hline
    Default Value: & 00-00-01\_00:00:00 \\
    \hline
    Possible Values: & A time interval in YYYY-MM-DD\_hh:mm:ss. \\
    \hline
    \caption{config\_AM\_timeSeriesStatsDaily\_backward\_output\_offset: Backward offset for filename timestamps when writing the output stream}
\end{longtable}
\end{center}
\section[AM\_timeSeriesStatsMonthly]{\hyperref[sec:nm_tab_AM_timeSeriesStatsMonthly]{AM\_timeSeriesStatsMonthly}}
\label{sec:nm_sec_AM_timeSeriesStatsMonthly}
\subsection[config\_AM\_timeSeriesStatsMonthly\_enable]{\hyperref[sec:nm_tab_AM_timeSeriesStatsMonthly]{config\_AM\_timeSeriesStatsMonthly\_enable}}
\label{subsec:nm_sec_config_AM_timeSeriesStatsMonthly_enable}
\begin{center}
\begin{longtable}{| p{2.0in} || p{4.0in} |}
    \hline
    Type: & logical \\
    \hline
    Units: & -- \\
    \hline
    Default Value: & .false. \\
    \hline
    Possible Values: & .true. or .false. \\
    \hline
    \caption{config\_AM\_timeSeriesStatsMonthly\_enable: If true, ocean analysis member time series stats is called.}
\end{longtable}
\end{center}
\subsection[config\_AM\_timeSeriesStatsMonthly\_compute\_on\_startup]{\hyperref[sec:nm_tab_AM_timeSeriesStatsMonthly]{config\_AM\_timeSeriesStatsMonthly\_compute\_on\_startup}}
\label{subsec:nm_sec_config_AM_timeSeriesStatsMonthly_compute_on_startup}
\begin{center}
\begin{longtable}{| p{2.0in} || p{4.0in} |}
    \hline
    Type: & logical \\
    \hline
    Units: & -- \\
    \hline
    Default Value: & .false. \\
    \hline
    Possible Values: & .true. or .false. \\
    \hline
    \caption{config\_AM\_timeSeriesStatsMonthly\_compute\_on\_startup: Logical flag determining if an analysis member computation occurs on start-up. You likely want this off for this (time series) analysis member because it will accumulate any state prior to time stepping (double counting the last time step).}
\end{longtable}
\end{center}
\subsection[config\_AM\_timeSeriesStatsMonthly\_write\_on\_startup]{\hyperref[sec:nm_tab_AM_timeSeriesStatsMonthly]{config\_AM\_timeSeriesStatsMonthly\_write\_on\_startup}}
\label{subsec:nm_sec_config_AM_timeSeriesStatsMonthly_write_on_startup}
\begin{center}
\begin{longtable}{| p{2.0in} || p{4.0in} |}
    \hline
    Type: & logical \\
    \hline
    Units: & -- \\
    \hline
    Default Value: & .false. \\
    \hline
    Possible Values: & .true. or .false. \\
    \hline
    \caption{config\_AM\_timeSeriesStatsMonthly\_write\_on\_startup: Logical flag determining if an analysis member output occurs on start-up.}
\end{longtable}
\end{center}
\subsection[config\_AM\_timeSeriesStatsMonthly\_compute\_interval]{\hyperref[sec:nm_tab_AM_timeSeriesStatsMonthly]{config\_AM\_timeSeriesStatsMonthly\_compute\_interval}}
\label{subsec:nm_sec_config_AM_timeSeriesStatsMonthly_compute_interval}
\begin{center}
\begin{longtable}{| p{2.0in} || p{4.0in} |}
    \hline
    Type: & character \\
    \hline
    Units: & -- \\
    \hline
    Default Value: & 00-00-00\_01:00:00 \\
    \hline
    Possible Values: & Any valid time stamp or 'dt'. This must also be less than or equal to output\_interval / 2 (i.e., requires at least two samples in a series). \\
    \hline
    \caption{config\_AM\_timeSeriesStatsMonthly\_compute\_interval: Interval that determines frequency of computation for the time series stats analysis member.}
\end{longtable}
\end{center}
\subsection[config\_AM\_timeSeriesStatsMonthly\_output\_stream]{\hyperref[sec:nm_tab_AM_timeSeriesStatsMonthly]{config\_AM\_timeSeriesStatsMonthly\_output\_stream}}
\label{subsec:nm_sec_config_AM_timeSeriesStatsMonthly_output_stream}
\begin{center}
\begin{longtable}{| p{2.0in} || p{4.0in} |}
    \hline
    Type: & character \\
    \hline
    Units: & -- \\
    \hline
    Default Value: & timeSeriesStatsMonthlyOutput \\
    \hline
    Possible Values: & An existing stream that will be modified (existing real fields removed and new time series stats versions added) with time series stats outputs. Cannot be 'none', like other analysis members. \\
    \hline
    \caption{config\_AM\_timeSeriesStatsMonthly\_output\_stream: Name of stream the time series stats analysis member will operate on.}
\end{longtable}
\end{center}
\subsection[config\_AM\_timeSeriesStatsMonthly\_restart\_stream]{\hyperref[sec:nm_tab_AM_timeSeriesStatsMonthly]{config\_AM\_timeSeriesStatsMonthly\_restart\_stream}}
\label{subsec:nm_sec_config_AM_timeSeriesStatsMonthly_restart_stream}
\begin{center}
\begin{longtable}{| p{2.0in} || p{4.0in} |}
    \hline
    Type: & character \\
    \hline
    Units: & -- \\
    \hline
    Default Value: & timeSeriesStatsMonthlyRestart \\
    \hline
    Possible Values: & A restart stream with state of the time series stats. \\
    \hline
    \caption{config\_AM\_timeSeriesStatsMonthly\_restart\_stream: Name of the restart stream the time series stats analysis member will use to initialize itself if restart is enabled.}
\end{longtable}
\end{center}
\subsection[config\_AM\_timeSeriesStatsMonthly\_operation]{\hyperref[sec:nm_tab_AM_timeSeriesStatsMonthly]{config\_AM\_timeSeriesStatsMonthly\_operation}}
\label{subsec:nm_sec_config_AM_timeSeriesStatsMonthly_operation}
\begin{center}
\begin{longtable}{| p{2.0in} || p{4.0in} |}
    \hline
    Type: & character \\
    \hline
    Units: & -- \\
    \hline
    Default Value: & avg \\
    \hline
    Possible Values: & An operation, where it can be 'avg', 'min', or 'max', 'sum', or 'sos' (sum of squares). \\
    \hline
    \caption{config\_AM\_timeSeriesStatsMonthly\_operation: An operation describing the statistic to apply to the time series for all variables in the output stream, reducing the time dimension.}
\end{longtable}
\end{center}
\subsection[config\_AM\_timeSeriesStatsMonthly\_reference\_times]{\hyperref[sec:nm_tab_AM_timeSeriesStatsMonthly]{config\_AM\_timeSeriesStatsMonthly\_reference\_times}}
\label{subsec:nm_sec_config_AM_timeSeriesStatsMonthly_reference_times}
\begin{center}
\begin{longtable}{| p{2.0in} || p{4.0in} |}
    \hline
    Type: & character \\
    \hline
    Units: & -- \\
    \hline
    Default Value: & initial\_time \\
    \hline
    Possible Values: & A list of absolute times or 'initial\_time's, separated by ;. \\
    \hline
    \caption{config\_AM\_timeSeriesStatsMonthly\_reference\_times: A list of absolute times describing when to start accumulating statistics. Each time indicates the start of one time window (time series statistic) per variable, in the output stream (i.e., provide four start times if you want quarterly climatologies, only one time is needed for monthly or daily averages, etc.)}
\end{longtable}
\end{center}
\subsection[config\_AM\_timeSeriesStatsMonthly\_duration\_intervals]{\hyperref[sec:nm_tab_AM_timeSeriesStatsMonthly]{config\_AM\_timeSeriesStatsMonthly\_duration\_intervals}}
\label{subsec:nm_sec_config_AM_timeSeriesStatsMonthly_duration_intervals}
\begin{center}
\begin{longtable}{| p{2.0in} || p{4.0in} |}
    \hline
    Type: & character \\
    \hline
    Units: & -- \\
    \hline
    Default Value: & repeat\_interval \\
    \hline
    Possible Values: & A list of time durations in d\_h:m:s or 'repeat\_interval's, separated by ;. Each must be greater than or equal to compute\_interval * 2 and less than or equal to repeat\_interval. duration\_intervals less than repeat\_intervals allow for repeated statistics within the repeat\_interval (i.e., for climatologies) \\
    \hline
    \caption{config\_AM\_timeSeriesStatsMonthly\_duration\_intervals: A list of time durations in d\_h:m:s describing how long to accumulate statistics in a time window for each repetition (repeat\_interval). It has to match the number of start time tokens in reference\_times.}
\end{longtable}
\end{center}
\subsection[config\_AM\_timeSeriesStatsMonthly\_repeat\_intervals]{\hyperref[sec:nm_tab_AM_timeSeriesStatsMonthly]{config\_AM\_timeSeriesStatsMonthly\_repeat\_intervals}}
\label{subsec:nm_sec_config_AM_timeSeriesStatsMonthly_repeat_intervals}
\begin{center}
\begin{longtable}{| p{2.0in} || p{4.0in} |}
    \hline
    Type: & character \\
    \hline
    Units: & -- \\
    \hline
    Default Value: & reset\_interval \\
    \hline
    Possible Values: & {\bf \color{red} MISSING} \\
    \hline
    \caption{config\_AM\_timeSeriesStatsMonthly\_repeat\_intervals: A list of time durations in d\_h:m:s describing the accumulation statistic temporal periodicity (time between beginning to accumulate again after it started - duration\_interval describes when to stop after starting/restarting). It has to match the number of tokens in reference\_times.}
\end{longtable}
\end{center}
\subsection[config\_AM\_timeSeriesStatsMonthly\_reset\_intervals]{\hyperref[sec:nm_tab_AM_timeSeriesStatsMonthly]{config\_AM\_timeSeriesStatsMonthly\_reset\_intervals}}
\label{subsec:nm_sec_config_AM_timeSeriesStatsMonthly_reset_intervals}
\begin{center}
\begin{longtable}{| p{2.0in} || p{4.0in} |}
    \hline
    Type: & character \\
    \hline
    Units: & -- \\
    \hline
    Default Value: & 00-01-00\_00:00:00 \\
    \hline
    Possible Values: & A list of time durations in d\_h:m:s, separated by ;. Ought to be greater than or equal to output\_interval (not verified by the analysis member). \\
    \hline
    \caption{config\_AM\_timeSeriesStatsMonthly\_reset\_intervals: A list of time durations in d\_h:m:s describing the statistic reset periodicity (how often to reset/clear/zero the accumulation). It has to match the number of tokens in reference\_times.}
\end{longtable}
\end{center}
\subsection[config\_AM\_timeSeriesStatsMonthly\_backward\_output\_offset]{\hyperref[sec:nm_tab_AM_timeSeriesStatsMonthly]{config\_AM\_timeSeriesStatsMonthly\_backward\_output\_offset}}
\label{subsec:nm_sec_config_AM_timeSeriesStatsMonthly_backward_output_offset}
\begin{center}
\begin{longtable}{| p{2.0in} || p{4.0in} |}
    \hline
    Type: & character \\
    \hline
    Units: & -- \\
    \hline
    Default Value: & 00-01-00\_00:00:00 \\
    \hline
    Possible Values: & A time interval in YYYY-MM-DD\_hh:mm:ss. \\
    \hline
    \caption{config\_AM\_timeSeriesStatsMonthly\_backward\_output\_offset: Backward offset for filename timestamps when writing the output stream}
\end{longtable}
\end{center}
\section[AM\_timeSeriesStatsClimatology]{\hyperref[sec:nm_tab_AM_timeSeriesStatsClimatology]{AM\_timeSeriesStatsClimatology}}
\label{sec:nm_sec_AM_timeSeriesStatsClimatology}
\subsection[config\_AM\_timeSeriesStatsClimatology\_enable]{\hyperref[sec:nm_tab_AM_timeSeriesStatsClimatology]{config\_AM\_timeSeriesStatsClimatology\_enable}}
\label{subsec:nm_sec_config_AM_timeSeriesStatsClimatology_enable}
\begin{center}
\begin{longtable}{| p{2.0in} || p{4.0in} |}
    \hline
    Type: & logical \\
    \hline
    Units: & -- \\
    \hline
    Default Value: & .false. \\
    \hline
    Possible Values: & .true. or .false. \\
    \hline
    \caption{config\_AM\_timeSeriesStatsClimatology\_enable: If true, ocean analysis member time series stats is called.}
\end{longtable}
\end{center}
\subsection[config\_AM\_timeSeriesStatsClimatology\_compute\_on\_startup]{\hyperref[sec:nm_tab_AM_timeSeriesStatsClimatology]{config\_AM\_timeSeriesStatsClimatology\_compute\_on\_startup}}
\label{subsec:nm_sec_config_AM_timeSeriesStatsClimatology_compute_on_startup}
\begin{center}
\begin{longtable}{| p{2.0in} || p{4.0in} |}
    \hline
    Type: & logical \\
    \hline
    Units: & -- \\
    \hline
    Default Value: & .false. \\
    \hline
    Possible Values: & .true. or .false. \\
    \hline
    \caption{config\_AM\_timeSeriesStatsClimatology\_compute\_on\_startup: Logical flag determining if an analysis member computation occurs on start-up. You likely want this off for this (time series) analysis member because it will accumulate any state prior to time stepping (double counting the last time step).}
\end{longtable}
\end{center}
\subsection[config\_AM\_timeSeriesStatsClimatology\_write\_on\_startup]{\hyperref[sec:nm_tab_AM_timeSeriesStatsClimatology]{config\_AM\_timeSeriesStatsClimatology\_write\_on\_startup}}
\label{subsec:nm_sec_config_AM_timeSeriesStatsClimatology_write_on_startup}
\begin{center}
\begin{longtable}{| p{2.0in} || p{4.0in} |}
    \hline
    Type: & logical \\
    \hline
    Units: & -- \\
    \hline
    Default Value: & .false. \\
    \hline
    Possible Values: & .true. or .false. \\
    \hline
    \caption{config\_AM\_timeSeriesStatsClimatology\_write\_on\_startup: Logical flag determining if an analysis member output occurs on start-up.}
\end{longtable}
\end{center}
\subsection[config\_AM\_timeSeriesStatsClimatology\_compute\_interval]{\hyperref[sec:nm_tab_AM_timeSeriesStatsClimatology]{config\_AM\_timeSeriesStatsClimatology\_compute\_interval}}
\label{subsec:nm_sec_config_AM_timeSeriesStatsClimatology_compute_interval}
\begin{center}
\begin{longtable}{| p{2.0in} || p{4.0in} |}
    \hline
    Type: & character \\
    \hline
    Units: & -- \\
    \hline
    Default Value: & 00-00-00\_01:00:00 \\
    \hline
    Possible Values: & Any valid time stamp or 'dt'. This must also be less than or equal to output\_interval / 2 (i.e., requires at least two samples in a series). \\
    \hline
    \caption{config\_AM\_timeSeriesStatsClimatology\_compute\_interval: Interval that determines frequency of computation for the time series stats analysis member.}
\end{longtable}
\end{center}
\subsection[config\_AM\_timeSeriesStatsClimatology\_output\_stream]{\hyperref[sec:nm_tab_AM_timeSeriesStatsClimatology]{config\_AM\_timeSeriesStatsClimatology\_output\_stream}}
\label{subsec:nm_sec_config_AM_timeSeriesStatsClimatology_output_stream}
\begin{center}
\begin{longtable}{| p{2.0in} || p{4.0in} |}
    \hline
    Type: & character \\
    \hline
    Units: & -- \\
    \hline
    Default Value: & timeSeriesStatsClimatologyOutput \\
    \hline
    Possible Values: & An existing stream that will be modified (existing real fields removed and new time series stats versions added) with time series stats outputs. Cannot be 'none', like other analysis members. \\
    \hline
    \caption{config\_AM\_timeSeriesStatsClimatology\_output\_stream: Name of stream the time series stats analysis member will operate on.}
\end{longtable}
\end{center}
\subsection[config\_AM\_timeSeriesStatsClimatology\_restart\_stream]{\hyperref[sec:nm_tab_AM_timeSeriesStatsClimatology]{config\_AM\_timeSeriesStatsClimatology\_restart\_stream}}
\label{subsec:nm_sec_config_AM_timeSeriesStatsClimatology_restart_stream}
\begin{center}
\begin{longtable}{| p{2.0in} || p{4.0in} |}
    \hline
    Type: & character \\
    \hline
    Units: & -- \\
    \hline
    Default Value: & timeSeriesStatsClimatologyRestart \\
    \hline
    Possible Values: & A restart stream with state of the time series stats. \\
    \hline
    \caption{config\_AM\_timeSeriesStatsClimatology\_restart\_stream: Name of the restart stream the time series stats analysis member will use to initialize itself if restart is enabled.}
\end{longtable}
\end{center}
\subsection[config\_AM\_timeSeriesStatsClimatology\_operation]{\hyperref[sec:nm_tab_AM_timeSeriesStatsClimatology]{config\_AM\_timeSeriesStatsClimatology\_operation}}
\label{subsec:nm_sec_config_AM_timeSeriesStatsClimatology_operation}
\begin{center}
\begin{longtable}{| p{2.0in} || p{4.0in} |}
    \hline
    Type: & character \\
    \hline
    Units: & -- \\
    \hline
    Default Value: & avg \\
    \hline
    Possible Values: & An operation, where it can be 'avg', 'min', or 'max', 'sum', or 'sos' (sum of squares). \\
    \hline
    \caption{config\_AM\_timeSeriesStatsClimatology\_operation: An operation describing the statistic to apply to the time series for all variables in the output stream, reducing the time dimension.}
\end{longtable}
\end{center}
\subsection[config\_AM\_timeSeriesStatsClimatology\_reference\_times]{\hyperref[sec:nm_tab_AM_timeSeriesStatsClimatology]{config\_AM\_timeSeriesStatsClimatology\_reference\_times}}
\label{subsec:nm_sec_config_AM_timeSeriesStatsClimatology_reference_times}
\begin{center}
\begin{longtable}{| p{2.0in} || p{4.0in} |}
    \hline
    Type: & character \\
    \hline
    Units: & -- \\
    \hline
    Default Value: & 00-03-01\_00:00:00;00-06-01\_00:00:00;00-09-01\_00:00:00;00-12-01\_00:00:00 \\
    \hline
    Possible Values: & A list of absolute times or 'initial\_time's, separated by ;. \\
    \hline
    \caption{config\_AM\_timeSeriesStatsClimatology\_reference\_times: A list of absolute times describing when to start accumulating statistics. Each time indicates the start of one time window (time series statistic) per variable, in the output stream (i.e., provide four start times if you want quarterly climatologies, only one time is needed for monthly or daily averages, etc.)}
\end{longtable}
\end{center}
\subsection[config\_AM\_timeSeriesStatsClimatology\_duration\_intervals]{\hyperref[sec:nm_tab_AM_timeSeriesStatsClimatology]{config\_AM\_timeSeriesStatsClimatology\_duration\_intervals}}
\label{subsec:nm_sec_config_AM_timeSeriesStatsClimatology_duration_intervals}
\begin{center}
\begin{longtable}{| p{2.0in} || p{4.0in} |}
    \hline
    Type: & character \\
    \hline
    Units: & -- \\
    \hline
    Default Value: & 00-03-00\_00:00:00;00-03-00\_00:00:00;00-03-00\_00:00:00;00-03-00\_00:00:00 \\
    \hline
    Possible Values: & A list of time durations in d\_h:m:s or 'repeat\_interval's, separated by ;. Each must be greater than or equal to compute\_interval * 2 and less than or equal to repeat\_interval. duration\_intervals less than repeat\_intervals allow for repeated statistics within the repeat\_interval (i.e., for climatologies) \\
    \hline
    \caption{config\_AM\_timeSeriesStatsClimatology\_duration\_intervals: A list of time durations in d\_h:m:s describing how long to accumulate statistics in a time window for each repetition (repeat\_interval). It has to match the number of start time tokens in reference\_times.}
\end{longtable}
\end{center}
\subsection[config\_AM\_timeSeriesStatsClimatology\_repeat\_intervals]{\hyperref[sec:nm_tab_AM_timeSeriesStatsClimatology]{config\_AM\_timeSeriesStatsClimatology\_repeat\_intervals}}
\label{subsec:nm_sec_config_AM_timeSeriesStatsClimatology_repeat_intervals}
\begin{center}
\begin{longtable}{| p{2.0in} || p{4.0in} |}
    \hline
    Type: & character \\
    \hline
    Units: & -- \\
    \hline
    Default Value: & 01-00-00\_00:00:00;01-00-00\_00:00:00;01-00-00\_00:00:00;01-00-00\_00:00:00 \\
    \hline
    Possible Values: & {\bf \color{red} MISSING} \\
    \hline
    \caption{config\_AM\_timeSeriesStatsClimatology\_repeat\_intervals: A list of time durations in d\_h:m:s describing the accumulation statistic temporal periodicity (time between beginning to accumulate again after it started - duration\_interval describes when to stop after starting/restarting). It has to match the number of tokens in reference\_times.}
\end{longtable}
\end{center}
\subsection[config\_AM\_timeSeriesStatsClimatology\_reset\_intervals]{\hyperref[sec:nm_tab_AM_timeSeriesStatsClimatology]{config\_AM\_timeSeriesStatsClimatology\_reset\_intervals}}
\label{subsec:nm_sec_config_AM_timeSeriesStatsClimatology_reset_intervals}
\begin{center}
\begin{longtable}{| p{2.0in} || p{4.0in} |}
    \hline
    Type: & character \\
    \hline
    Units: & -- \\
    \hline
    Default Value: & 1000-00-00\_00:00:00;1000-00-00\_00:00:00;1000-00-00\_00:00:00;1000-00-00\_00:00:00 \\
    \hline
    Possible Values: & A list of time durations in d\_h:m:s, separated by ;. Ought to be greater than or equal to output\_interval (not verified by the analysis member). \\
    \hline
    \caption{config\_AM\_timeSeriesStatsClimatology\_reset\_intervals: A list of time durations in d\_h:m:s describing the statistic reset periodicity (how often to reset/clear/zero the accumulation). It has to match the number of tokens in reference\_times.}
\end{longtable}
\end{center}
\subsection[config\_AM\_timeSeriesStatsClimatology\_backward\_output\_offset]{\hyperref[sec:nm_tab_AM_timeSeriesStatsClimatology]{config\_AM\_timeSeriesStatsClimatology\_backward\_output\_offset}}
\label{subsec:nm_sec_config_AM_timeSeriesStatsClimatology_backward_output_offset}
\begin{center}
\begin{longtable}{| p{2.0in} || p{4.0in} |}
    \hline
    Type: & character \\
    \hline
    Units: & -- \\
    \hline
    Default Value: & 00-03-00\_00:00:00 \\
    \hline
    Possible Values: & A time interval in YYYY-MM-DD\_hh:mm:ss. \\
    \hline
    \caption{config\_AM\_timeSeriesStatsClimatology\_backward\_output\_offset: Backward offset for filename timestamps when writing the output stream}
\end{longtable}
\end{center}
\section[AM\_timeSeriesStatsMonthlyMax]{\hyperref[sec:nm_tab_AM_timeSeriesStatsMonthlyMax]{AM\_timeSeriesStatsMonthlyMax}}
\label{sec:nm_sec_AM_timeSeriesStatsMonthlyMax}
\subsection[config\_AM\_timeSeriesStatsMonthlyMax\_enable]{\hyperref[sec:nm_tab_AM_timeSeriesStatsMonthlyMax]{config\_AM\_timeSeriesStatsMonthlyMax\_enable}}
\label{subsec:nm_sec_config_AM_timeSeriesStatsMonthlyMax_enable}
\begin{center}
\begin{longtable}{| p{2.0in} || p{4.0in} |}
    \hline
    Type: & logical \\
    \hline
    Units: & -- \\
    \hline
    Default Value: & .false. \\
    \hline
    Possible Values: & .true. or .false. \\
    \hline
    \caption{config\_AM\_timeSeriesStatsMonthlyMax\_enable: If true, ocean analysis member time series stats is called.}
\end{longtable}
\end{center}
\subsection[config\_AM\_timeSeriesStatsMonthlyMax\_compute\_on\_startup]{\hyperref[sec:nm_tab_AM_timeSeriesStatsMonthlyMax]{config\_AM\_timeSeriesStatsMonthlyMax\_compute\_on\_startup}}
\label{subsec:nm_sec_config_AM_timeSeriesStatsMonthlyMax_compute_on_startup}
\begin{center}
\begin{longtable}{| p{2.0in} || p{4.0in} |}
    \hline
    Type: & logical \\
    \hline
    Units: & -- \\
    \hline
    Default Value: & .false. \\
    \hline
    Possible Values: & .true. or .false. \\
    \hline
    \caption{config\_AM\_timeSeriesStatsMonthlyMax\_compute\_on\_startup: Logical flag determining if an analysis member computation occurs on start-up. You likely want this off for this (time series) analysis member because it will accumulate any state prior to time stepping (double counting the last time step).}
\end{longtable}
\end{center}
\subsection[config\_AM\_timeSeriesStatsMonthlyMax\_write\_on\_startup]{\hyperref[sec:nm_tab_AM_timeSeriesStatsMonthlyMax]{config\_AM\_timeSeriesStatsMonthlyMax\_write\_on\_startup}}
\label{subsec:nm_sec_config_AM_timeSeriesStatsMonthlyMax_write_on_startup}
\begin{center}
\begin{longtable}{| p{2.0in} || p{4.0in} |}
    \hline
    Type: & logical \\
    \hline
    Units: & -- \\
    \hline
    Default Value: & .false. \\
    \hline
    Possible Values: & .true. or .false. \\
    \hline
    \caption{config\_AM\_timeSeriesStatsMonthlyMax\_write\_on\_startup: Logical flag determining if an analysis member output occurs on start-up.}
\end{longtable}
\end{center}
\subsection[config\_AM\_timeSeriesStatsMonthlyMax\_compute\_interval]{\hyperref[sec:nm_tab_AM_timeSeriesStatsMonthlyMax]{config\_AM\_timeSeriesStatsMonthlyMax\_compute\_interval}}
\label{subsec:nm_sec_config_AM_timeSeriesStatsMonthlyMax_compute_interval}
\begin{center}
\begin{longtable}{| p{2.0in} || p{4.0in} |}
    \hline
    Type: & character \\
    \hline
    Units: & -- \\
    \hline
    Default Value: & 00-00-00\_01:00:00 \\
    \hline
    Possible Values: & Any valid time stamp or 'dt'. This must also be less than or equal to output\_interval / 2 (i.e., requires at least two samples in a series). \\
    \hline
    \caption{config\_AM\_timeSeriesStatsMonthlyMax\_compute\_interval: Interval that determines frequency of computation for the time series stats analysis member.}
\end{longtable}
\end{center}
\subsection[config\_AM\_timeSeriesStatsMonthlyMax\_output\_stream]{\hyperref[sec:nm_tab_AM_timeSeriesStatsMonthlyMax]{config\_AM\_timeSeriesStatsMonthlyMax\_output\_stream}}
\label{subsec:nm_sec_config_AM_timeSeriesStatsMonthlyMax_output_stream}
\begin{center}
\begin{longtable}{| p{2.0in} || p{4.0in} |}
    \hline
    Type: & character \\
    \hline
    Units: & -- \\
    \hline
    Default Value: & timeSeriesStatsMonthlyMaxOutput \\
    \hline
    Possible Values: & An existing stream that will be modified (existing real fields removed and new time series stats versions added) with time series stats outputs. Cannot be 'none', like other analysis members. \\
    \hline
    \caption{config\_AM\_timeSeriesStatsMonthlyMax\_output\_stream: Name of stream the time series stats analysis member will operate on.}
\end{longtable}
\end{center}
\subsection[config\_AM\_timeSeriesStatsMonthlyMax\_restart\_stream]{\hyperref[sec:nm_tab_AM_timeSeriesStatsMonthlyMax]{config\_AM\_timeSeriesStatsMonthlyMax\_restart\_stream}}
\label{subsec:nm_sec_config_AM_timeSeriesStatsMonthlyMax_restart_stream}
\begin{center}
\begin{longtable}{| p{2.0in} || p{4.0in} |}
    \hline
    Type: & character \\
    \hline
    Units: & -- \\
    \hline
    Default Value: & timeSeriesStatsMonthlyMaxRestart \\
    \hline
    Possible Values: & A restart stream with state of the time series stats. \\
    \hline
    \caption{config\_AM\_timeSeriesStatsMonthlyMax\_restart\_stream: Name of the restart stream the time series stats analysis member will use to initialize itself if restart is enabled.}
\end{longtable}
\end{center}
\subsection[config\_AM\_timeSeriesStatsMonthlyMax\_operation]{\hyperref[sec:nm_tab_AM_timeSeriesStatsMonthlyMax]{config\_AM\_timeSeriesStatsMonthlyMax\_operation}}
\label{subsec:nm_sec_config_AM_timeSeriesStatsMonthlyMax_operation}
\begin{center}
\begin{longtable}{| p{2.0in} || p{4.0in} |}
    \hline
    Type: & character \\
    \hline
    Units: & -- \\
    \hline
    Default Value: & max \\
    \hline
    Possible Values: & An operation, where it can be 'avg', 'min', or 'max', 'sum', or 'sos' (sum of squares). \\
    \hline
    \caption{config\_AM\_timeSeriesStatsMonthlyMax\_operation: An operation describing the statistic to apply to the time series for all variables in the output stream, reducing the time dimension.}
\end{longtable}
\end{center}
\subsection[config\_AM\_timeSeriesStatsMonthlyMax\_reference\_times]{\hyperref[sec:nm_tab_AM_timeSeriesStatsMonthlyMax]{config\_AM\_timeSeriesStatsMonthlyMax\_reference\_times}}
\label{subsec:nm_sec_config_AM_timeSeriesStatsMonthlyMax_reference_times}
\begin{center}
\begin{longtable}{| p{2.0in} || p{4.0in} |}
    \hline
    Type: & character \\
    \hline
    Units: & -- \\
    \hline
    Default Value: & initial\_time \\
    \hline
    Possible Values: & A list of absolute times or 'initial\_time's, separated by ;. \\
    \hline
    \caption{config\_AM\_timeSeriesStatsMonthlyMax\_reference\_times: A list of absolute times describing when to start accumulating statistics. Each time indicates the start of one time window (time series statistic) per variable, in the output stream (i.e., provide four start times if you want quarterly climatologies, only one time is needed for monthly or daily averages, etc.)}
\end{longtable}
\end{center}
\subsection[config\_AM\_timeSeriesStatsMonthlyMax\_duration\_intervals]{\hyperref[sec:nm_tab_AM_timeSeriesStatsMonthlyMax]{config\_AM\_timeSeriesStatsMonthlyMax\_duration\_intervals}}
\label{subsec:nm_sec_config_AM_timeSeriesStatsMonthlyMax_duration_intervals}
\begin{center}
\begin{longtable}{| p{2.0in} || p{4.0in} |}
    \hline
    Type: & character \\
    \hline
    Units: & -- \\
    \hline
    Default Value: & repeat\_interval \\
    \hline
    Possible Values: & A list of time durations in d\_h:m:s or 'repeat\_interval's, separated by ;. Each must be greater than or equal to compute\_interval * 2 and less than or equal to repeat\_interval. duration\_intervals less than repeat\_intervals allow for repeated statistics within the repeat\_interval (i.e., for climatologies) \\
    \hline
    \caption{config\_AM\_timeSeriesStatsMonthlyMax\_duration\_intervals: A list of time durations in d\_h:m:s describing how long to accumulate statistics in a time window for each repetition (repeat\_interval). It has to match the number of start time tokens in reference\_times.}
\end{longtable}
\end{center}
\subsection[config\_AM\_timeSeriesStatsMonthlyMax\_repeat\_intervals]{\hyperref[sec:nm_tab_AM_timeSeriesStatsMonthlyMax]{config\_AM\_timeSeriesStatsMonthlyMax\_repeat\_intervals}}
\label{subsec:nm_sec_config_AM_timeSeriesStatsMonthlyMax_repeat_intervals}
\begin{center}
\begin{longtable}{| p{2.0in} || p{4.0in} |}
    \hline
    Type: & character \\
    \hline
    Units: & -- \\
    \hline
    Default Value: & reset\_interval \\
    \hline
    Possible Values: & {\bf \color{red} MISSING} \\
    \hline
    \caption{config\_AM\_timeSeriesStatsMonthlyMax\_repeat\_intervals: A list of time durations in d\_h:m:s describing the accumulation statistic temporal periodicity (time between beginning to accumulate again after it started - duration\_interval describes when to stop after starting/restarting). It has to match the number of tokens in reference\_times.}
\end{longtable}
\end{center}
\subsection[config\_AM\_timeSeriesStatsMonthlyMax\_reset\_intervals]{\hyperref[sec:nm_tab_AM_timeSeriesStatsMonthlyMax]{config\_AM\_timeSeriesStatsMonthlyMax\_reset\_intervals}}
\label{subsec:nm_sec_config_AM_timeSeriesStatsMonthlyMax_reset_intervals}
\begin{center}
\begin{longtable}{| p{2.0in} || p{4.0in} |}
    \hline
    Type: & character \\
    \hline
    Units: & -- \\
    \hline
    Default Value: & 00-01-00\_00:00:00 \\
    \hline
    Possible Values: & A list of time durations in d\_h:m:s, separated by ;. Ought to be greater than or equal to output\_interval (not verified by the analysis member). \\
    \hline
    \caption{config\_AM\_timeSeriesStatsMonthlyMax\_reset\_intervals: A list of time durations in d\_h:m:s describing the statistic reset periodicity (how often to reset/clear/zero the accumulation). It has to match the number of tokens in reference\_times.}
\end{longtable}
\end{center}
\subsection[config\_AM\_timeSeriesStatsMonthlyMax\_backward\_output\_offset]{\hyperref[sec:nm_tab_AM_timeSeriesStatsMonthlyMax]{config\_AM\_timeSeriesStatsMonthlyMax\_backward\_output\_offset}}
\label{subsec:nm_sec_config_AM_timeSeriesStatsMonthlyMax_backward_output_offset}
\begin{center}
\begin{longtable}{| p{2.0in} || p{4.0in} |}
    \hline
    Type: & character \\
    \hline
    Units: & -- \\
    \hline
    Default Value: & 00-01-00\_00:00:00 \\
    \hline
    Possible Values: & A time interval in YYYY-MM-DD\_hh:mm:ss. \\
    \hline
    \caption{config\_AM\_timeSeriesStatsMonthlyMax\_backward\_output\_offset: Backward offset for filename timestamps when writing the output stream}
\end{longtable}
\end{center}
\section[AM\_timeSeriesStatsMonthlyMin]{\hyperref[sec:nm_tab_AM_timeSeriesStatsMonthlyMin]{AM\_timeSeriesStatsMonthlyMin}}
\label{sec:nm_sec_AM_timeSeriesStatsMonthlyMin}
\subsection[config\_AM\_timeSeriesStatsMonthlyMin\_enable]{\hyperref[sec:nm_tab_AM_timeSeriesStatsMonthlyMin]{config\_AM\_timeSeriesStatsMonthlyMin\_enable}}
\label{subsec:nm_sec_config_AM_timeSeriesStatsMonthlyMin_enable}
\begin{center}
\begin{longtable}{| p{2.0in} || p{4.0in} |}
    \hline
    Type: & logical \\
    \hline
    Units: & -- \\
    \hline
    Default Value: & .false. \\
    \hline
    Possible Values: & .true. or .false. \\
    \hline
    \caption{config\_AM\_timeSeriesStatsMonthlyMin\_enable: If true, ocean analysis member time series stats is called.}
\end{longtable}
\end{center}
\subsection[config\_AM\_timeSeriesStatsMonthlyMin\_compute\_on\_startup]{\hyperref[sec:nm_tab_AM_timeSeriesStatsMonthlyMin]{config\_AM\_timeSeriesStatsMonthlyMin\_compute\_on\_startup}}
\label{subsec:nm_sec_config_AM_timeSeriesStatsMonthlyMin_compute_on_startup}
\begin{center}
\begin{longtable}{| p{2.0in} || p{4.0in} |}
    \hline
    Type: & logical \\
    \hline
    Units: & -- \\
    \hline
    Default Value: & .false. \\
    \hline
    Possible Values: & .true. or .false. \\
    \hline
    \caption{config\_AM\_timeSeriesStatsMonthlyMin\_compute\_on\_startup: Logical flag determining if an analysis member computation occurs on start-up. You likely want this off for this (time series) analysis member because it will accumulate any state prior to time stepping (double counting the last time step).}
\end{longtable}
\end{center}
\subsection[config\_AM\_timeSeriesStatsMonthlyMin\_write\_on\_startup]{\hyperref[sec:nm_tab_AM_timeSeriesStatsMonthlyMin]{config\_AM\_timeSeriesStatsMonthlyMin\_write\_on\_startup}}
\label{subsec:nm_sec_config_AM_timeSeriesStatsMonthlyMin_write_on_startup}
\begin{center}
\begin{longtable}{| p{2.0in} || p{4.0in} |}
    \hline
    Type: & logical \\
    \hline
    Units: & -- \\
    \hline
    Default Value: & .false. \\
    \hline
    Possible Values: & .true. or .false. \\
    \hline
    \caption{config\_AM\_timeSeriesStatsMonthlyMin\_write\_on\_startup: Logical flag determining if an analysis member output occurs on start-up.}
\end{longtable}
\end{center}
\subsection[config\_AM\_timeSeriesStatsMonthlyMin\_compute\_interval]{\hyperref[sec:nm_tab_AM_timeSeriesStatsMonthlyMin]{config\_AM\_timeSeriesStatsMonthlyMin\_compute\_interval}}
\label{subsec:nm_sec_config_AM_timeSeriesStatsMonthlyMin_compute_interval}
\begin{center}
\begin{longtable}{| p{2.0in} || p{4.0in} |}
    \hline
    Type: & character \\
    \hline
    Units: & -- \\
    \hline
    Default Value: & 00-00-00\_01:00:00 \\
    \hline
    Possible Values: & Any valid time stamp or 'dt'. This must also be less than or equal to output\_interval / 2 (i.e., requires at least two samples in a series). \\
    \hline
    \caption{config\_AM\_timeSeriesStatsMonthlyMin\_compute\_interval: Interval that determines frequency of computation for the time series stats analysis member.}
\end{longtable}
\end{center}
\subsection[config\_AM\_timeSeriesStatsMonthlyMin\_output\_stream]{\hyperref[sec:nm_tab_AM_timeSeriesStatsMonthlyMin]{config\_AM\_timeSeriesStatsMonthlyMin\_output\_stream}}
\label{subsec:nm_sec_config_AM_timeSeriesStatsMonthlyMin_output_stream}
\begin{center}
\begin{longtable}{| p{2.0in} || p{4.0in} |}
    \hline
    Type: & character \\
    \hline
    Units: & -- \\
    \hline
    Default Value: & timeSeriesStatsMonthlyMinOutput \\
    \hline
    Possible Values: & An existing stream that will be modified (existing real fields removed and new time series stats versions added) with time series stats outputs. Cannot be 'none', like other analysis members. \\
    \hline
    \caption{config\_AM\_timeSeriesStatsMonthlyMin\_output\_stream: Name of stream the time series stats analysis member will operate on.}
\end{longtable}
\end{center}
\subsection[config\_AM\_timeSeriesStatsMonthlyMin\_restart\_stream]{\hyperref[sec:nm_tab_AM_timeSeriesStatsMonthlyMin]{config\_AM\_timeSeriesStatsMonthlyMin\_restart\_stream}}
\label{subsec:nm_sec_config_AM_timeSeriesStatsMonthlyMin_restart_stream}
\begin{center}
\begin{longtable}{| p{2.0in} || p{4.0in} |}
    \hline
    Type: & character \\
    \hline
    Units: & -- \\
    \hline
    Default Value: & timeSeriesStatsMonthlyMinRestart \\
    \hline
    Possible Values: & A restart stream with state of the time series stats. \\
    \hline
    \caption{config\_AM\_timeSeriesStatsMonthlyMin\_restart\_stream: Name of the restart stream the time series stats analysis member will use to initialize itself if restart is enabled.}
\end{longtable}
\end{center}
\subsection[config\_AM\_timeSeriesStatsMonthlyMin\_operation]{\hyperref[sec:nm_tab_AM_timeSeriesStatsMonthlyMin]{config\_AM\_timeSeriesStatsMonthlyMin\_operation}}
\label{subsec:nm_sec_config_AM_timeSeriesStatsMonthlyMin_operation}
\begin{center}
\begin{longtable}{| p{2.0in} || p{4.0in} |}
    \hline
    Type: & character \\
    \hline
    Units: & -- \\
    \hline
    Default Value: & min \\
    \hline
    Possible Values: & An operation, where it can be 'avg', 'min', or 'max', 'sum', or 'sos' (sum of squares). \\
    \hline
    \caption{config\_AM\_timeSeriesStatsMonthlyMin\_operation: An operation describing the statistic to apply to the time series for all variables in the output stream, reducing the time dimension.}
\end{longtable}
\end{center}
\subsection[config\_AM\_timeSeriesStatsMonthlyMin\_reference\_times]{\hyperref[sec:nm_tab_AM_timeSeriesStatsMonthlyMin]{config\_AM\_timeSeriesStatsMonthlyMin\_reference\_times}}
\label{subsec:nm_sec_config_AM_timeSeriesStatsMonthlyMin_reference_times}
\begin{center}
\begin{longtable}{| p{2.0in} || p{4.0in} |}
    \hline
    Type: & character \\
    \hline
    Units: & -- \\
    \hline
    Default Value: & initial\_time \\
    \hline
    Possible Values: & A list of absolute times or 'initial\_time's, separated by ;. \\
    \hline
    \caption{config\_AM\_timeSeriesStatsMonthlyMin\_reference\_times: A list of absolute times describing when to start accumulating statistics. Each time indicates the start of one time window (time series statistic) per variable, in the output stream (i.e., provide four start times if you want quarterly climatologies, only one time is needed for monthly or daily averages, etc.)}
\end{longtable}
\end{center}
\subsection[config\_AM\_timeSeriesStatsMonthlyMin\_duration\_intervals]{\hyperref[sec:nm_tab_AM_timeSeriesStatsMonthlyMin]{config\_AM\_timeSeriesStatsMonthlyMin\_duration\_intervals}}
\label{subsec:nm_sec_config_AM_timeSeriesStatsMonthlyMin_duration_intervals}
\begin{center}
\begin{longtable}{| p{2.0in} || p{4.0in} |}
    \hline
    Type: & character \\
    \hline
    Units: & -- \\
    \hline
    Default Value: & repeat\_interval \\
    \hline
    Possible Values: & A list of time durations in d\_h:m:s or 'repeat\_interval's, separated by ;. Each must be greater than or equal to compute\_interval * 2 and less than or equal to repeat\_interval. duration\_intervals less than repeat\_intervals allow for repeated statistics within the repeat\_interval (i.e., for climatologies) \\
    \hline
    \caption{config\_AM\_timeSeriesStatsMonthlyMin\_duration\_intervals: A list of time durations in d\_h:m:s describing how long to accumulate statistics in a time window for each repetition (repeat\_interval). It has to match the number of start time tokens in reference\_times.}
\end{longtable}
\end{center}
\subsection[config\_AM\_timeSeriesStatsMonthlyMin\_repeat\_intervals]{\hyperref[sec:nm_tab_AM_timeSeriesStatsMonthlyMin]{config\_AM\_timeSeriesStatsMonthlyMin\_repeat\_intervals}}
\label{subsec:nm_sec_config_AM_timeSeriesStatsMonthlyMin_repeat_intervals}
\begin{center}
\begin{longtable}{| p{2.0in} || p{4.0in} |}
    \hline
    Type: & character \\
    \hline
    Units: & -- \\
    \hline
    Default Value: & reset\_interval \\
    \hline
    Possible Values: & {\bf \color{red} MISSING} \\
    \hline
    \caption{config\_AM\_timeSeriesStatsMonthlyMin\_repeat\_intervals: A list of time durations in d\_h:m:s describing the accumulation statistic temporal periodicity (time between beginning to accumulate again after it started - duration\_interval describes when to stop after starting/restarting). It has to match the number of tokens in reference\_times.}
\end{longtable}
\end{center}
\subsection[config\_AM\_timeSeriesStatsMonthlyMin\_reset\_intervals]{\hyperref[sec:nm_tab_AM_timeSeriesStatsMonthlyMin]{config\_AM\_timeSeriesStatsMonthlyMin\_reset\_intervals}}
\label{subsec:nm_sec_config_AM_timeSeriesStatsMonthlyMin_reset_intervals}
\begin{center}
\begin{longtable}{| p{2.0in} || p{4.0in} |}
    \hline
    Type: & character \\
    \hline
    Units: & -- \\
    \hline
    Default Value: & 00-01-00\_00:00:00 \\
    \hline
    Possible Values: & A list of time durations in d\_h:m:s, separated by ;. Ought to be greater than or equal to output\_interval (not verified by the analysis member). \\
    \hline
    \caption{config\_AM\_timeSeriesStatsMonthlyMin\_reset\_intervals: A list of time durations in d\_h:m:s describing the statistic reset periodicity (how often to reset/clear/zero the accumulation). It has to match the number of tokens in reference\_times.}
\end{longtable}
\end{center}
\subsection[config\_AM\_timeSeriesStatsMonthlyMin\_backward\_output\_offset]{\hyperref[sec:nm_tab_AM_timeSeriesStatsMonthlyMin]{config\_AM\_timeSeriesStatsMonthlyMin\_backward\_output\_offset}}
\label{subsec:nm_sec_config_AM_timeSeriesStatsMonthlyMin_backward_output_offset}
\begin{center}
\begin{longtable}{| p{2.0in} || p{4.0in} |}
    \hline
    Type: & character \\
    \hline
    Units: & -- \\
    \hline
    Default Value: & 00-01-00\_00:00:00 \\
    \hline
    Possible Values: & A time interval in YYYY-MM-DD\_hh:mm:ss. \\
    \hline
    \caption{config\_AM\_timeSeriesStatsMonthlyMin\_backward\_output\_offset: Backward offset for filename timestamps when writing the output stream}
\end{longtable}
\end{center}
\section[AM\_timeSeriesStatsCustom]{\hyperref[sec:nm_tab_AM_timeSeriesStatsCustom]{AM\_timeSeriesStatsCustom}}
\label{sec:nm_sec_AM_timeSeriesStatsCustom}
\subsection[config\_AM\_timeSeriesStatsCustom\_enable]{\hyperref[sec:nm_tab_AM_timeSeriesStatsCustom]{config\_AM\_timeSeriesStatsCustom\_enable}}
\label{subsec:nm_sec_config_AM_timeSeriesStatsCustom_enable}
\begin{center}
\begin{longtable}{| p{2.0in} || p{4.0in} |}
    \hline
    Type: & logical \\
    \hline
    Units: & -- \\
    \hline
    Default Value: & .false. \\
    \hline
    Possible Values: & .true. or .false. \\
    \hline
    \caption{config\_AM\_timeSeriesStatsCustom\_enable: If true, ocean analysis member time series stats is called.}
\end{longtable}
\end{center}
\subsection[config\_AM\_timeSeriesStatsCustom\_compute\_on\_startup]{\hyperref[sec:nm_tab_AM_timeSeriesStatsCustom]{config\_AM\_timeSeriesStatsCustom\_compute\_on\_startup}}
\label{subsec:nm_sec_config_AM_timeSeriesStatsCustom_compute_on_startup}
\begin{center}
\begin{longtable}{| p{2.0in} || p{4.0in} |}
    \hline
    Type: & logical \\
    \hline
    Units: & -- \\
    \hline
    Default Value: & .false. \\
    \hline
    Possible Values: & .true. or .false. \\
    \hline
    \caption{config\_AM\_timeSeriesStatsCustom\_compute\_on\_startup: Logical flag determining if an analysis member computation occurs on start-up. You likely want this off for this (time series) analysis member because it will accumulate any state prior to time stepping (double counting the last time step).}
\end{longtable}
\end{center}
\subsection[config\_AM\_timeSeriesStatsCustom\_write\_on\_startup]{\hyperref[sec:nm_tab_AM_timeSeriesStatsCustom]{config\_AM\_timeSeriesStatsCustom\_write\_on\_startup}}
\label{subsec:nm_sec_config_AM_timeSeriesStatsCustom_write_on_startup}
\begin{center}
\begin{longtable}{| p{2.0in} || p{4.0in} |}
    \hline
    Type: & logical \\
    \hline
    Units: & -- \\
    \hline
    Default Value: & .false. \\
    \hline
    Possible Values: & .true. or .false. \\
    \hline
    \caption{config\_AM\_timeSeriesStatsCustom\_write\_on\_startup: Logical flag determining if an analysis member output occurs on start-up.}
\end{longtable}
\end{center}
\subsection[config\_AM\_timeSeriesStatsCustom\_compute\_interval]{\hyperref[sec:nm_tab_AM_timeSeriesStatsCustom]{config\_AM\_timeSeriesStatsCustom\_compute\_interval}}
\label{subsec:nm_sec_config_AM_timeSeriesStatsCustom_compute_interval}
\begin{center}
\begin{longtable}{| p{2.0in} || p{4.0in} |}
    \hline
    Type: & character \\
    \hline
    Units: & -- \\
    \hline
    Default Value: & 00-00-00\_01:00:00 \\
    \hline
    Possible Values: & Any valid time stamp or 'dt'. This must also be less than or equal to output\_interval / 2 (i.e., requires at least two samples in a series). \\
    \hline
    \caption{config\_AM\_timeSeriesStatsCustom\_compute\_interval: Interval that determines frequency of computation for the time series stats analysis member.}
\end{longtable}
\end{center}
\subsection[config\_AM\_timeSeriesStatsCustom\_output\_stream]{\hyperref[sec:nm_tab_AM_timeSeriesStatsCustom]{config\_AM\_timeSeriesStatsCustom\_output\_stream}}
\label{subsec:nm_sec_config_AM_timeSeriesStatsCustom_output_stream}
\begin{center}
\begin{longtable}{| p{2.0in} || p{4.0in} |}
    \hline
    Type: & character \\
    \hline
    Units: & -- \\
    \hline
    Default Value: & timeSeriesStatsCustomOutput \\
    \hline
    Possible Values: & An existing stream that will be modified (existing real fields removed and new time series stats versions added) with time series stats outputs. Cannot be 'none', like other analysis members. \\
    \hline
    \caption{config\_AM\_timeSeriesStatsCustom\_output\_stream: Name of stream the time series stats analysis member will operate on.}
\end{longtable}
\end{center}
\subsection[config\_AM\_timeSeriesStatsCustom\_restart\_stream]{\hyperref[sec:nm_tab_AM_timeSeriesStatsCustom]{config\_AM\_timeSeriesStatsCustom\_restart\_stream}}
\label{subsec:nm_sec_config_AM_timeSeriesStatsCustom_restart_stream}
\begin{center}
\begin{longtable}{| p{2.0in} || p{4.0in} |}
    \hline
    Type: & character \\
    \hline
    Units: & -- \\
    \hline
    Default Value: & timeSeriesStatsCustomRestart \\
    \hline
    Possible Values: & A restart stream with state of the time series stats. \\
    \hline
    \caption{config\_AM\_timeSeriesStatsCustom\_restart\_stream: Name of the restart stream the time series stats analysis member will use to initialize itself if restart is enabled.}
\end{longtable}
\end{center}
\subsection[config\_AM\_timeSeriesStatsCustom\_operation]{\hyperref[sec:nm_tab_AM_timeSeriesStatsCustom]{config\_AM\_timeSeriesStatsCustom\_operation}}
\label{subsec:nm_sec_config_AM_timeSeriesStatsCustom_operation}
\begin{center}
\begin{longtable}{| p{2.0in} || p{4.0in} |}
    \hline
    Type: & character \\
    \hline
    Units: & -- \\
    \hline
    Default Value: & avg \\
    \hline
    Possible Values: & An operation, where it can be 'avg', 'min', or 'max', 'sum', or 'sos' (sum of squares). \\
    \hline
    \caption{config\_AM\_timeSeriesStatsCustom\_operation: An operation describing the statistic to apply to the time series for all variables in the output stream, reducing the time dimension.}
\end{longtable}
\end{center}
\subsection[config\_AM\_timeSeriesStatsCustom\_reference\_times]{\hyperref[sec:nm_tab_AM_timeSeriesStatsCustom]{config\_AM\_timeSeriesStatsCustom\_reference\_times}}
\label{subsec:nm_sec_config_AM_timeSeriesStatsCustom_reference_times}
\begin{center}
\begin{longtable}{| p{2.0in} || p{4.0in} |}
    \hline
    Type: & character \\
    \hline
    Units: & -- \\
    \hline
    Default Value: & initial\_time \\
    \hline
    Possible Values: & A list of absolute times or 'initial\_time's, separated by ;. \\
    \hline
    \caption{config\_AM\_timeSeriesStatsCustom\_reference\_times: A list of absolute times describing when to start accumulating statistics. Each time indicates the start of one time window (time series statistic) per variable, in the output stream (i.e., provide four start times if you want quarterly climatologies, only one time is needed for monthly or daily averages, etc.)}
\end{longtable}
\end{center}
\subsection[config\_AM\_timeSeriesStatsCustom\_duration\_intervals]{\hyperref[sec:nm_tab_AM_timeSeriesStatsCustom]{config\_AM\_timeSeriesStatsCustom\_duration\_intervals}}
\label{subsec:nm_sec_config_AM_timeSeriesStatsCustom_duration_intervals}
\begin{center}
\begin{longtable}{| p{2.0in} || p{4.0in} |}
    \hline
    Type: & character \\
    \hline
    Units: & -- \\
    \hline
    Default Value: & repeat\_interval \\
    \hline
    Possible Values: & A list of time durations in d\_h:m:s or 'repeat\_interval's, separated by ;. Each must be greater than or equal to compute\_interval * 2 and less than or equal to repeat\_interval. duration\_intervals less than repeat\_intervals allow for repeated statistics within the repeat\_interval (i.e., for climatologies) \\
    \hline
    \caption{config\_AM\_timeSeriesStatsCustom\_duration\_intervals: A list of time durations in d\_h:m:s describing how long to accumulate statistics in a time window for each repetition (repeat\_interval). It has to match the number of start time tokens in reference\_times.}
\end{longtable}
\end{center}
\subsection[config\_AM\_timeSeriesStatsCustom\_repeat\_intervals]{\hyperref[sec:nm_tab_AM_timeSeriesStatsCustom]{config\_AM\_timeSeriesStatsCustom\_repeat\_intervals}}
\label{subsec:nm_sec_config_AM_timeSeriesStatsCustom_repeat_intervals}
\begin{center}
\begin{longtable}{| p{2.0in} || p{4.0in} |}
    \hline
    Type: & character \\
    \hline
    Units: & -- \\
    \hline
    Default Value: & reset\_interval \\
    \hline
    Possible Values: & {\bf \color{red} MISSING} \\
    \hline
    \caption{config\_AM\_timeSeriesStatsCustom\_repeat\_intervals: A list of time durations in d\_h:m:s describing the accumulation statistic temporal periodicity (time between beginning to accumulate again after it started - duration\_interval describes when to stop after starting/restarting). It has to match the number of tokens in reference\_times.}
\end{longtable}
\end{center}
\subsection[config\_AM\_timeSeriesStatsCustom\_reset\_intervals]{\hyperref[sec:nm_tab_AM_timeSeriesStatsCustom]{config\_AM\_timeSeriesStatsCustom\_reset\_intervals}}
\label{subsec:nm_sec_config_AM_timeSeriesStatsCustom_reset_intervals}
\begin{center}
\begin{longtable}{| p{2.0in} || p{4.0in} |}
    \hline
    Type: & character \\
    \hline
    Units: & -- \\
    \hline
    Default Value: & 00-00-07\_00:00:00 \\
    \hline
    Possible Values: & A list of time durations in d\_h:m:s, separated by ;. Ought to be greater than or equal to output\_interval (not verified by the analysis member). \\
    \hline
    \caption{config\_AM\_timeSeriesStatsCustom\_reset\_intervals: A list of time durations in d\_h:m:s describing the statistic reset periodicity (how often to reset/clear/zero the accumulation). It has to match the number of tokens in reference\_times.}
\end{longtable}
\end{center}
\subsection[config\_AM\_timeSeriesStatsCustom\_backward\_output\_offset]{\hyperref[sec:nm_tab_AM_timeSeriesStatsCustom]{config\_AM\_timeSeriesStatsCustom\_backward\_output\_offset}}
\label{subsec:nm_sec_config_AM_timeSeriesStatsCustom_backward_output_offset}
\begin{center}
\begin{longtable}{| p{2.0in} || p{4.0in} |}
    \hline
    Type: & character \\
    \hline
    Units: & -- \\
    \hline
    Default Value: & 00-00-01\_00:00:00 \\
    \hline
    Possible Values: & A time interval in YYYY-MM-DD\_hh:mm:ss. \\
    \hline
    \caption{config\_AM\_timeSeriesStatsCustom\_backward\_output\_offset: Backward offset for filename timestamps when writing the output stream}
\end{longtable}
\end{center}
\section[AM\_pointwiseStats]{\hyperref[sec:nm_tab_AM_pointwiseStats]{AM\_pointwiseStats}}
\label{sec:nm_sec_AM_pointwiseStats}
\subsection[config\_AM\_pointwiseStats\_enable]{\hyperref[sec:nm_tab_AM_pointwiseStats]{config\_AM\_pointwiseStats\_enable}}
\label{subsec:nm_sec_config_AM_pointwiseStats_enable}
\begin{center}
\begin{longtable}{| p{2.0in} || p{4.0in} |}
    \hline
    Type: & logical \\
    \hline
    Units: & -- \\
    \hline
    Default Value: & .false. \\
    \hline
    Possible Values: & .true. or .false. \\
    \hline
    \caption{config\_AM\_pointwiseStats\_enable: If true, ocean analysis member pointwiseStats is called.}
\end{longtable}
\end{center}
\subsection[config\_AM\_pointwiseStats\_compute\_interval]{\hyperref[sec:nm_tab_AM_pointwiseStats]{config\_AM\_pointwiseStats\_compute\_interval}}
\label{subsec:nm_sec_config_AM_pointwiseStats_compute_interval}
\begin{center}
\begin{longtable}{| p{2.0in} || p{4.0in} |}
    \hline
    Type: & character \\
    \hline
    Units: & -- \\
    \hline
    Default Value: & output\_interval \\
    \hline
    Possible Values: & Any valid time stamp, 'dt', or 'output\_interval' \\
    \hline
    \caption{config\_AM\_pointwiseStats\_compute\_interval: Timestamp determining how often analysis member computation should be performed.}
\end{longtable}
\end{center}
\subsection[config\_AM\_pointwiseStats\_output\_stream]{\hyperref[sec:nm_tab_AM_pointwiseStats]{config\_AM\_pointwiseStats\_output\_stream}}
\label{subsec:nm_sec_config_AM_pointwiseStats_output_stream}
\begin{center}
\begin{longtable}{| p{2.0in} || p{4.0in} |}
    \hline
    Type: & character \\
    \hline
    Units: & -- \\
    \hline
    Default Value: & pointwiseStatsOutput \\
    \hline
    Possible Values: & Any existing stream name or 'none' \\
    \hline
    \caption{config\_AM\_pointwiseStats\_output\_stream: Name of the stream that the pointwiseStats analysis member should be tied to.}
\end{longtable}
\end{center}
\subsection[config\_AM\_pointwiseStats\_compute\_on\_startup]{\hyperref[sec:nm_tab_AM_pointwiseStats]{config\_AM\_pointwiseStats\_compute\_on\_startup}}
\label{subsec:nm_sec_config_AM_pointwiseStats_compute_on_startup}
\begin{center}
\begin{longtable}{| p{2.0in} || p{4.0in} |}
    \hline
    Type: & logical \\
    \hline
    Units: & -- \\
    \hline
    Default Value: & .true. \\
    \hline
    Possible Values: & .true. or .false. \\
    \hline
    \caption{config\_AM\_pointwiseStats\_compute\_on\_startup: Logical flag determining if an analysis member computation occurs on start-up.}
\end{longtable}
\end{center}
\subsection[config\_AM\_pointwiseStats\_write\_on\_startup]{\hyperref[sec:nm_tab_AM_pointwiseStats]{config\_AM\_pointwiseStats\_write\_on\_startup}}
\label{subsec:nm_sec_config_AM_pointwiseStats_write_on_startup}
\begin{center}
\begin{longtable}{| p{2.0in} || p{4.0in} |}
    \hline
    Type: & logical \\
    \hline
    Units: & -- \\
    \hline
    Default Value: & .true. \\
    \hline
    Possible Values: & .true. or .false. \\
    \hline
    \caption{config\_AM\_pointwiseStats\_write\_on\_startup: Logical flag determining if an analysis member write occurs on start-up.}
\end{longtable}
\end{center}
\section[AM\_debugDiagnostics]{\hyperref[sec:nm_tab_AM_debugDiagnostics]{AM\_debugDiagnostics}}
\label{sec:nm_sec_AM_debugDiagnostics}
\subsection[config\_AM\_debugDiagnostics\_enable]{\hyperref[sec:nm_tab_AM_debugDiagnostics]{config\_AM\_debugDiagnostics\_enable}}
\label{subsec:nm_sec_config_AM_debugDiagnostics_enable}
\begin{center}
\begin{longtable}{| p{2.0in} || p{4.0in} |}
    \hline
    Type: & logical \\
    \hline
    Units: & -- \\
    \hline
    Default Value: & .false. \\
    \hline
    Possible Values: & .true. or .false. \\
    \hline
    \caption{config\_AM\_debugDiagnostics\_enable: If true, ocean analysis member debugDiagnostics is called.}
\end{longtable}
\end{center}
\subsection[config\_AM\_debugDiagnostics\_compute\_interval]{\hyperref[sec:nm_tab_AM_debugDiagnostics]{config\_AM\_debugDiagnostics\_compute\_interval}}
\label{subsec:nm_sec_config_AM_debugDiagnostics_compute_interval}
\begin{center}
\begin{longtable}{| p{2.0in} || p{4.0in} |}
    \hline
    Type: & character \\
    \hline
    Units: & -- \\
    \hline
    Default Value: & output\_interval \\
    \hline
    Possible Values: & Any valid time stamp, 'dt', or 'output\_interval' \\
    \hline
    \caption{config\_AM\_debugDiagnostics\_compute\_interval: Timestamp determining how often analysis member computation should be performed.}
\end{longtable}
\end{center}
\subsection[config\_AM\_debugDiagnostics\_output\_stream]{\hyperref[sec:nm_tab_AM_debugDiagnostics]{config\_AM\_debugDiagnostics\_output\_stream}}
\label{subsec:nm_sec_config_AM_debugDiagnostics_output_stream}
\begin{center}
\begin{longtable}{| p{2.0in} || p{4.0in} |}
    \hline
    Type: & character \\
    \hline
    Units: & -- \\
    \hline
    Default Value: & debugDiagnosticsOutput \\
    \hline
    Possible Values: & Any existing stream name or 'none' \\
    \hline
    \caption{config\_AM\_debugDiagnostics\_output\_stream: Name of the stream that the debugDiagnostics analysis member should be tied to.}
\end{longtable}
\end{center}
\subsection[config\_AM\_debugDiagnostics\_compute\_on\_startup]{\hyperref[sec:nm_tab_AM_debugDiagnostics]{config\_AM\_debugDiagnostics\_compute\_on\_startup}}
\label{subsec:nm_sec_config_AM_debugDiagnostics_compute_on_startup}
\begin{center}
\begin{longtable}{| p{2.0in} || p{4.0in} |}
    \hline
    Type: & logical \\
    \hline
    Units: & -- \\
    \hline
    Default Value: & .true. \\
    \hline
    Possible Values: & .true. or .false. \\
    \hline
    \caption{config\_AM\_debugDiagnostics\_compute\_on\_startup: Logical flag determining if an analysis member computation occurs on start-up.}
\end{longtable}
\end{center}
\subsection[config\_AM\_debugDiagnostics\_write\_on\_startup]{\hyperref[sec:nm_tab_AM_debugDiagnostics]{config\_AM\_debugDiagnostics\_write\_on\_startup}}
\label{subsec:nm_sec_config_AM_debugDiagnostics_write_on_startup}
\begin{center}
\begin{longtable}{| p{2.0in} || p{4.0in} |}
    \hline
    Type: & logical \\
    \hline
    Units: & -- \\
    \hline
    Default Value: & .true. \\
    \hline
    Possible Values: & .true. or .false. \\
    \hline
    \caption{config\_AM\_debugDiagnostics\_write\_on\_startup: Logical flag determining if an analysis member write occurs on start-up.}
\end{longtable}
\end{center}
\subsection[config\_AM\_debugDiagnostics\_check\_state]{\hyperref[sec:nm_tab_AM_debugDiagnostics]{config\_AM\_debugDiagnostics\_check\_state}}
\label{subsec:nm_sec_config_AM_debugDiagnostics_check_state}
\begin{center}
\begin{longtable}{| p{2.0in} || p{4.0in} |}
    \hline
    Type: & logical \\
    \hline
    Units: & -- \\
    \hline
    Default Value: & .false. \\
    \hline
    Possible Values: & .true. or .false. \\
    \hline
    \caption{config\_AM\_debugDiagnostics\_check\_state: Logical flag determining if state checking happens when the debug diagnostics AM is called.}
\end{longtable}
\end{center}
\section[AM\_rpnCalculator]{\hyperref[sec:nm_tab_AM_rpnCalculator]{AM\_rpnCalculator}}
\label{sec:nm_sec_AM_rpnCalculator}
\subsection[config\_AM\_rpnCalculator\_enable]{\hyperref[sec:nm_tab_AM_rpnCalculator]{config\_AM\_rpnCalculator\_enable}}
\label{subsec:nm_sec_config_AM_rpnCalculator_enable}
\begin{center}
\begin{longtable}{| p{2.0in} || p{4.0in} |}
    \hline
    Type: & logical \\
    \hline
    Units: & -- \\
    \hline
    Default Value: & .false. \\
    \hline
    Possible Values: & .true. or .false. \\
    \hline
    \caption{config\_AM\_rpnCalculator\_enable: If true, ocean analysis member RPN calculator is called.}
\end{longtable}
\end{center}
\subsection[config\_AM\_rpnCalculator\_compute\_on\_startup]{\hyperref[sec:nm_tab_AM_rpnCalculator]{config\_AM\_rpnCalculator\_compute\_on\_startup}}
\label{subsec:nm_sec_config_AM_rpnCalculator_compute_on_startup}
\begin{center}
\begin{longtable}{| p{2.0in} || p{4.0in} |}
    \hline
    Type: & logical \\
    \hline
    Units: & -- \\
    \hline
    Default Value: & .true. \\
    \hline
    Possible Values: & .true. or .false. \\
    \hline
    \caption{config\_AM\_rpnCalculator\_compute\_on\_startup: Logical flag determining if an analysis member computation occurs on start-up.}
\end{longtable}
\end{center}
\subsection[config\_AM\_rpnCalculator\_write\_on\_startup]{\hyperref[sec:nm_tab_AM_rpnCalculator]{config\_AM\_rpnCalculator\_write\_on\_startup}}
\label{subsec:nm_sec_config_AM_rpnCalculator_write_on_startup}
\begin{center}
\begin{longtable}{| p{2.0in} || p{4.0in} |}
    \hline
    Type: & logical \\
    \hline
    Units: & -- \\
    \hline
    Default Value: & .false. \\
    \hline
    Possible Values: & .true. or .false. \\
    \hline
    \caption{config\_AM\_rpnCalculator\_write\_on\_startup: Logical flag determining if an analysis member output occurs on start-up.}
\end{longtable}
\end{center}
\subsection[config\_AM\_rpnCalculator\_compute\_interval]{\hyperref[sec:nm_tab_AM_rpnCalculator]{config\_AM\_rpnCalculator\_compute\_interval}}
\label{subsec:nm_sec_config_AM_rpnCalculator_compute_interval}
\begin{center}
\begin{longtable}{| p{2.0in} || p{4.0in} |}
    \hline
    Type: & character \\
    \hline
    Units: & -- \\
    \hline
    Default Value: & 0010-00-00\_00:00:00 \\
    \hline
    Possible Values: & Any valid time stamp, 'output\_interval', or 'dt'. \\
    \hline
    \caption{config\_AM\_rpnCalculator\_compute\_interval: Interval that determines frequency of computation for the RPN calculator analysis member.}
\end{longtable}
\end{center}
\subsection[config\_AM\_rpnCalculator\_output\_stream]{\hyperref[sec:nm_tab_AM_rpnCalculator]{config\_AM\_rpnCalculator\_output\_stream}}
\label{subsec:nm_sec_config_AM_rpnCalculator_output_stream}
\begin{center}
\begin{longtable}{| p{2.0in} || p{4.0in} |}
    \hline
    Type: & character \\
    \hline
    Units: & -- \\
    \hline
    Default Value: & none \\
    \hline
    Possible Values: & 'none' or the name of an output stream \\
    \hline
    \caption{config\_AM\_rpnCalculator\_output\_stream: Name of stream the RPN calculator analysis member put output fields.}
\end{longtable}
\end{center}
\subsection[config\_AM\_rpnCalculator\_variable\_a]{\hyperref[sec:nm_tab_AM_rpnCalculator]{config\_AM\_rpnCalculator\_variable\_a}}
\label{subsec:nm_sec_config_AM_rpnCalculator_variable_a}
\begin{center}
\begin{longtable}{| p{2.0in} || p{4.0in} |}
    \hline
    Type: & character \\
    \hline
    Units: & -- \\
    \hline
    Default Value: & layerThickness \\
    \hline
    Possible Values: & 'none' or the name of a valid MPAS 0D or 1D real field \\
    \hline
    \caption{config\_AM\_rpnCalculator\_variable\_a: Name of a 0D or 1D real field that is bound to name 'a' in an RPN expression.}
\end{longtable}
\end{center}
\subsection[config\_AM\_rpnCalculator\_variable\_b]{\hyperref[sec:nm_tab_AM_rpnCalculator]{config\_AM\_rpnCalculator\_variable\_b}}
\label{subsec:nm_sec_config_AM_rpnCalculator_variable_b}
\begin{center}
\begin{longtable}{| p{2.0in} || p{4.0in} |}
    \hline
    Type: & character \\
    \hline
    Units: & -- \\
    \hline
    Default Value: & areaCell \\
    \hline
    Possible Values: & 'none' or the name of a valid MPAS 0D or 1D real field \\
    \hline
    \caption{config\_AM\_rpnCalculator\_variable\_b: Name of a 0D or 1D real field that is bound to name 'b' in an RPN expression.}
\end{longtable}
\end{center}
\subsection[config\_AM\_rpnCalculator\_variable\_c]{\hyperref[sec:nm_tab_AM_rpnCalculator]{config\_AM\_rpnCalculator\_variable\_c}}
\label{subsec:nm_sec_config_AM_rpnCalculator_variable_c}
\begin{center}
\begin{longtable}{| p{2.0in} || p{4.0in} |}
    \hline
    Type: & character \\
    \hline
    Units: & -- \\
    \hline
    Default Value: & none \\
    \hline
    Possible Values: & 'none' or the name of a valid MPAS 0D or 1D real field \\
    \hline
    \caption{config\_AM\_rpnCalculator\_variable\_c: Name of a 0D or 1D real field that is bound to name 'c' in an RPN expression.}
\end{longtable}
\end{center}
\subsection[config\_AM\_rpnCalculator\_variable\_d]{\hyperref[sec:nm_tab_AM_rpnCalculator]{config\_AM\_rpnCalculator\_variable\_d}}
\label{subsec:nm_sec_config_AM_rpnCalculator_variable_d}
\begin{center}
\begin{longtable}{| p{2.0in} || p{4.0in} |}
    \hline
    Type: & character \\
    \hline
    Units: & -- \\
    \hline
    Default Value: & none \\
    \hline
    Possible Values: & 'none' or the name of a valid MPAS 0D or 1D real field \\
    \hline
    \caption{config\_AM\_rpnCalculator\_variable\_d: Name of a 0D or 1D real field that is bound to name 'd' in an RPN expression.}
\end{longtable}
\end{center}
\subsection[config\_AM\_rpnCalculator\_variable\_e]{\hyperref[sec:nm_tab_AM_rpnCalculator]{config\_AM\_rpnCalculator\_variable\_e}}
\label{subsec:nm_sec_config_AM_rpnCalculator_variable_e}
\begin{center}
\begin{longtable}{| p{2.0in} || p{4.0in} |}
    \hline
    Type: & character \\
    \hline
    Units: & -- \\
    \hline
    Default Value: & none \\
    \hline
    Possible Values: & 'none' or the name of a valid MPAS 0D or 1D real field \\
    \hline
    \caption{config\_AM\_rpnCalculator\_variable\_e: Name of a 0D or 1D real field that is bound to name 'e' in an RPN expression.}
\end{longtable}
\end{center}
\subsection[config\_AM\_rpnCalculator\_variable\_f]{\hyperref[sec:nm_tab_AM_rpnCalculator]{config\_AM\_rpnCalculator\_variable\_f}}
\label{subsec:nm_sec_config_AM_rpnCalculator_variable_f}
\begin{center}
\begin{longtable}{| p{2.0in} || p{4.0in} |}
    \hline
    Type: & character \\
    \hline
    Units: & -- \\
    \hline
    Default Value: & none \\
    \hline
    Possible Values: & 'none' or the name of a valid MPAS 0D or 1D real field \\
    \hline
    \caption{config\_AM\_rpnCalculator\_variable\_f: Name of a 0D or 1D real field that is bound to name 'f' in an RPN expression.}
\end{longtable}
\end{center}
\subsection[config\_AM\_rpnCalculator\_variable\_g]{\hyperref[sec:nm_tab_AM_rpnCalculator]{config\_AM\_rpnCalculator\_variable\_g}}
\label{subsec:nm_sec_config_AM_rpnCalculator_variable_g}
\begin{center}
\begin{longtable}{| p{2.0in} || p{4.0in} |}
    \hline
    Type: & character \\
    \hline
    Units: & -- \\
    \hline
    Default Value: & none \\
    \hline
    Possible Values: & 'none' or the name of a valid MPAS 0D or 1D real field \\
    \hline
    \caption{config\_AM\_rpnCalculator\_variable\_g: Name of a 0D or 1D real field that is bound to name 'g' in an RPN expression.}
\end{longtable}
\end{center}
\subsection[config\_AM\_rpnCalculator\_variable\_h]{\hyperref[sec:nm_tab_AM_rpnCalculator]{config\_AM\_rpnCalculator\_variable\_h}}
\label{subsec:nm_sec_config_AM_rpnCalculator_variable_h}
\begin{center}
\begin{longtable}{| p{2.0in} || p{4.0in} |}
    \hline
    Type: & character \\
    \hline
    Units: & -- \\
    \hline
    Default Value: & none \\
    \hline
    Possible Values: & 'none' or the name of a valid MPAS 0D or 1D real field \\
    \hline
    \caption{config\_AM\_rpnCalculator\_variable\_h: Name of a 0D or 1D real field that is bound to name 'h' in an RPN expression.}
\end{longtable}
\end{center}
\subsection[config\_AM\_rpnCalculator\_expression\_1]{\hyperref[sec:nm_tab_AM_rpnCalculator]{config\_AM\_rpnCalculator\_expression\_1}}
\label{subsec:nm_sec_config_AM_rpnCalculator_expression_1}
\begin{center}
\begin{longtable}{| p{2.0in} || p{4.0in} |}
    \hline
    Type: & character \\
    \hline
    Units: & -- \\
    \hline
    Default Value: & a b * \\
    \hline
    Possible Values: & 'none' or a valid RPN expression described in the documentation \\
    \hline
    \caption{config\_AM\_rpnCalculator\_expression\_1: An RPN expression using fields bound to variable names.}
\end{longtable}
\end{center}
\subsection[config\_AM\_rpnCalculator\_expression\_2]{\hyperref[sec:nm_tab_AM_rpnCalculator]{config\_AM\_rpnCalculator\_expression\_2}}
\label{subsec:nm_sec_config_AM_rpnCalculator_expression_2}
\begin{center}
\begin{longtable}{| p{2.0in} || p{4.0in} |}
    \hline
    Type: & character \\
    \hline
    Units: & -- \\
    \hline
    Default Value: & none \\
    \hline
    Possible Values: & 'none' or a valid RPN expression described in the documentation \\
    \hline
    \caption{config\_AM\_rpnCalculator\_expression\_2: An RPN expression using fields bound to variable names.}
\end{longtable}
\end{center}
\subsection[config\_AM\_rpnCalculator\_expression\_3]{\hyperref[sec:nm_tab_AM_rpnCalculator]{config\_AM\_rpnCalculator\_expression\_3}}
\label{subsec:nm_sec_config_AM_rpnCalculator_expression_3}
\begin{center}
\begin{longtable}{| p{2.0in} || p{4.0in} |}
    \hline
    Type: & character \\
    \hline
    Units: & -- \\
    \hline
    Default Value: & none \\
    \hline
    Possible Values: & 'none' or a valid RPN expression described in the documentation \\
    \hline
    \caption{config\_AM\_rpnCalculator\_expression\_3: An RPN expression using fields bound to variable names.}
\end{longtable}
\end{center}
\subsection[config\_AM\_rpnCalculator\_expression\_4]{\hyperref[sec:nm_tab_AM_rpnCalculator]{config\_AM\_rpnCalculator\_expression\_4}}
\label{subsec:nm_sec_config_AM_rpnCalculator_expression_4}
\begin{center}
\begin{longtable}{| p{2.0in} || p{4.0in} |}
    \hline
    Type: & character \\
    \hline
    Units: & -- \\
    \hline
    Default Value: & none \\
    \hline
    Possible Values: & 'none or a valid RPN expression described in the documentation \\
    \hline
    \caption{config\_AM\_rpnCalculator\_expression\_4: An RPN expression using fields bound to variable names.}
\end{longtable}
\end{center}
\subsection[config\_AM\_rpnCalculator\_output\_name\_1]{\hyperref[sec:nm_tab_AM_rpnCalculator]{config\_AM\_rpnCalculator\_output\_name\_1}}
\label{subsec:nm_sec_config_AM_rpnCalculator_output_name_1}
\begin{center}
\begin{longtable}{| p{2.0in} || p{4.0in} |}
    \hline
    Type: & character \\
    \hline
    Units: & -- \\
    \hline
    Default Value: & volumeCell \\
    \hline
    Possible Values: & a valid MPAS field output name if expression 1 is set, otherwise 'none' \\
    \hline
    \caption{config\_AM\_rpnCalculator\_output\_name\_1: The name of the output field resulting from RPN expression 1.}
\end{longtable}
\end{center}
\subsection[config\_AM\_rpnCalculator\_output\_name\_2]{\hyperref[sec:nm_tab_AM_rpnCalculator]{config\_AM\_rpnCalculator\_output\_name\_2}}
\label{subsec:nm_sec_config_AM_rpnCalculator_output_name_2}
\begin{center}
\begin{longtable}{| p{2.0in} || p{4.0in} |}
    \hline
    Type: & character \\
    \hline
    Units: & -- \\
    \hline
    Default Value: & none \\
    \hline
    Possible Values: & a valid MPAS field output name if expression 2 is set, otherwise 'none' \\
    \hline
    \caption{config\_AM\_rpnCalculator\_output\_name\_2: The name of the output field resulting from RPN expression 2.}
\end{longtable}
\end{center}
\subsection[config\_AM\_rpnCalculator\_output\_name\_3]{\hyperref[sec:nm_tab_AM_rpnCalculator]{config\_AM\_rpnCalculator\_output\_name\_3}}
\label{subsec:nm_sec_config_AM_rpnCalculator_output_name_3}
\begin{center}
\begin{longtable}{| p{2.0in} || p{4.0in} |}
    \hline
    Type: & character \\
    \hline
    Units: & -- \\
    \hline
    Default Value: & none \\
    \hline
    Possible Values: & a valid MPAS field output name if expression 3 is set, otherwise 'none' \\
    \hline
    \caption{config\_AM\_rpnCalculator\_output\_name\_3: The name of the output field resulting from RPN expression 3.}
\end{longtable}
\end{center}
\subsection[config\_AM\_rpnCalculator\_output\_name\_4]{\hyperref[sec:nm_tab_AM_rpnCalculator]{config\_AM\_rpnCalculator\_output\_name\_4}}
\label{subsec:nm_sec_config_AM_rpnCalculator_output_name_4}
\begin{center}
\begin{longtable}{| p{2.0in} || p{4.0in} |}
    \hline
    Type: & character \\
    \hline
    Units: & -- \\
    \hline
    Default Value: & none \\
    \hline
    Possible Values: & a valid MPAS field output name if expression 4 is set, otherwise 'none' \\
    \hline
    \caption{config\_AM\_rpnCalculator\_output\_name\_4: The name of the output field resulting from RPN expression 4.}
\end{longtable}
\end{center}
\section[AM\_transectTransport]{\hyperref[sec:nm_tab_AM_transectTransport]{AM\_transectTransport}}
\label{sec:nm_sec_AM_transectTransport}
\subsection[config\_AM\_transectTransport\_enable]{\hyperref[sec:nm_tab_AM_transectTransport]{config\_AM\_transectTransport\_enable}}
\label{subsec:nm_sec_config_AM_transectTransport_enable}
\begin{center}
\begin{longtable}{| p{2.0in} || p{4.0in} |}
    \hline
    Type: & logical \\
    \hline
    Units: & -- \\
    \hline
    Default Value: & .false. \\
    \hline
    Possible Values: & .true. or .false. \\
    \hline
    \caption{config\_AM\_transectTransport\_enable: If true, ocean analysis member transectTransport is called.}
\end{longtable}
\end{center}
\subsection[config\_AM\_transectTransport\_compute\_interval]{\hyperref[sec:nm_tab_AM_transectTransport]{config\_AM\_transectTransport\_compute\_interval}}
\label{subsec:nm_sec_config_AM_transectTransport_compute_interval}
\begin{center}
\begin{longtable}{| p{2.0in} || p{4.0in} |}
    \hline
    Type: & character \\
    \hline
    Units: & -- \\
    \hline
    Default Value: & output\_interval \\
    \hline
    Possible Values: & Any valid time stamp, 'dt', or 'output\_interval' \\
    \hline
    \caption{config\_AM\_transectTransport\_compute\_interval: Timestamp determining how often analysis member computation should be performed.}
\end{longtable}
\end{center}
\subsection[config\_AM\_transectTransport\_output\_stream]{\hyperref[sec:nm_tab_AM_transectTransport]{config\_AM\_transectTransport\_output\_stream}}
\label{subsec:nm_sec_config_AM_transectTransport_output_stream}
\begin{center}
\begin{longtable}{| p{2.0in} || p{4.0in} |}
    \hline
    Type: & character \\
    \hline
    Units: & -- \\
    \hline
    Default Value: & transectTransportOutput \\
    \hline
    Possible Values: & Any existing stream name or 'none' \\
    \hline
    \caption{config\_AM\_transectTransport\_output\_stream: Name of the stream that the transectTransport analysis member should be tied to.}
\end{longtable}
\end{center}
\subsection[config\_AM\_transectTransport\_compute\_on\_startup]{\hyperref[sec:nm_tab_AM_transectTransport]{config\_AM\_transectTransport\_compute\_on\_startup}}
\label{subsec:nm_sec_config_AM_transectTransport_compute_on_startup}
\begin{center}
\begin{longtable}{| p{2.0in} || p{4.0in} |}
    \hline
    Type: & logical \\
    \hline
    Units: & -- \\
    \hline
    Default Value: & .true. \\
    \hline
    Possible Values: & .true. or .false. \\
    \hline
    \caption{config\_AM\_transectTransport\_compute\_on\_startup: Logical flag determining if an analysis member computation occurs on start-up.}
\end{longtable}
\end{center}
\subsection[config\_AM\_transectTransport\_write\_on\_startup]{\hyperref[sec:nm_tab_AM_transectTransport]{config\_AM\_transectTransport\_write\_on\_startup}}
\label{subsec:nm_sec_config_AM_transectTransport_write_on_startup}
\begin{center}
\begin{longtable}{| p{2.0in} || p{4.0in} |}
    \hline
    Type: & logical \\
    \hline
    Units: & -- \\
    \hline
    Default Value: & .true. \\
    \hline
    Possible Values: & .true. or .false. \\
    \hline
    \caption{config\_AM\_transectTransport\_write\_on\_startup: Logical flag determining if an analysis member write occurs on start-up.}
\end{longtable}
\end{center}
\subsection[config\_AM\_transectTransport\_transect\_group]{\hyperref[sec:nm_tab_AM_transectTransport]{config\_AM\_transectTransport\_transect\_group}}
\label{subsec:nm_sec_config_AM_transectTransport_transect_group}
\begin{center}
\begin{longtable}{| p{2.0in} || p{4.0in} |}
    \hline
    Type: & character \\
    \hline
    Units: & -- \\
    \hline
    Default Value: & all \\
    \hline
    Possible Values: & Either of '', 'all' or any valid transect group name. \\
    \hline
    \caption{config\_AM\_transectTransport\_transect\_group: The name of the transect group that holds the transects for which the transport should be caclulated.}
\end{longtable}
\end{center}
\section[AM\_eddyProductVariables]{\hyperref[sec:nm_tab_AM_eddyProductVariables]{AM\_eddyProductVariables}}
\label{sec:nm_sec_AM_eddyProductVariables}
\subsection[config\_AM\_eddyProductVariables\_enable]{\hyperref[sec:nm_tab_AM_eddyProductVariables]{config\_AM\_eddyProductVariables\_enable}}
\label{subsec:nm_sec_config_AM_eddyProductVariables_enable}
\begin{center}
\begin{longtable}{| p{2.0in} || p{4.0in} |}
    \hline
    Type: & logical \\
    \hline
    Units: & -- \\
    \hline
    Default Value: & .false. \\
    \hline
    Possible Values: & .true. or .false. \\
    \hline
    \caption{config\_AM\_eddyProductVariables\_enable: If true, ocean analysis member eddyProductVariables is called.}
\end{longtable}
\end{center}
\subsection[config\_AM\_eddyProductVariables\_compute\_interval]{\hyperref[sec:nm_tab_AM_eddyProductVariables]{config\_AM\_eddyProductVariables\_compute\_interval}}
\label{subsec:nm_sec_config_AM_eddyProductVariables_compute_interval}
\begin{center}
\begin{longtable}{| p{2.0in} || p{4.0in} |}
    \hline
    Type: & character \\
    \hline
    Units: & -- \\
    \hline
    Default Value: & dt \\
    \hline
    Possible Values: & Any valid time stamp, 'dt', or 'output\_interval' \\
    \hline
    \caption{config\_AM\_eddyProductVariables\_compute\_interval: Timestamp determining how often analysis member computation should be performed.}
\end{longtable}
\end{center}
\subsection[config\_AM\_eddyProductVariables\_output\_stream]{\hyperref[sec:nm_tab_AM_eddyProductVariables]{config\_AM\_eddyProductVariables\_output\_stream}}
\label{subsec:nm_sec_config_AM_eddyProductVariables_output_stream}
\begin{center}
\begin{longtable}{| p{2.0in} || p{4.0in} |}
    \hline
    Type: & character \\
    \hline
    Units: & -- \\
    \hline
    Default Value: & eddyProductVariablesOutput \\
    \hline
    Possible Values: & Any existing stream name or 'none' \\
    \hline
    \caption{config\_AM\_eddyProductVariables\_output\_stream: Name of the stream that the eddyProductVariables analysis member should be tied to.}
\end{longtable}
\end{center}
\subsection[config\_AM\_eddyProductVariables\_compute\_on\_startup]{\hyperref[sec:nm_tab_AM_eddyProductVariables]{config\_AM\_eddyProductVariables\_compute\_on\_startup}}
\label{subsec:nm_sec_config_AM_eddyProductVariables_compute_on_startup}
\begin{center}
\begin{longtable}{| p{2.0in} || p{4.0in} |}
    \hline
    Type: & logical \\
    \hline
    Units: & -- \\
    \hline
    Default Value: & .true. \\
    \hline
    Possible Values: & .true. or .false. \\
    \hline
    \caption{config\_AM\_eddyProductVariables\_compute\_on\_startup: Logical flag determining if an analysis member computation occurs on start-up.}
\end{longtable}
\end{center}
\subsection[config\_AM\_eddyProductVariables\_write\_on\_startup]{\hyperref[sec:nm_tab_AM_eddyProductVariables]{config\_AM\_eddyProductVariables\_write\_on\_startup}}
\label{subsec:nm_sec_config_AM_eddyProductVariables_write_on_startup}
\begin{center}
\begin{longtable}{| p{2.0in} || p{4.0in} |}
    \hline
    Type: & logical \\
    \hline
    Units: & -- \\
    \hline
    Default Value: & .false. \\
    \hline
    Possible Values: & .true. or .false. \\
    \hline
    \caption{config\_AM\_eddyProductVariables\_write\_on\_startup: Logical flag determining if an analysis member write occurs on start-up.}
\end{longtable}
\end{center}
\section[AM\_mocStreamfunction]{\hyperref[sec:nm_tab_AM_mocStreamfunction]{AM\_mocStreamfunction}}
\label{sec:nm_sec_AM_mocStreamfunction}
\subsection[config\_AM\_mocStreamfunction\_enable]{\hyperref[sec:nm_tab_AM_mocStreamfunction]{config\_AM\_mocStreamfunction\_enable}}
\label{subsec:nm_sec_config_AM_mocStreamfunction_enable}
\begin{center}
\begin{longtable}{| p{2.0in} || p{4.0in} |}
    \hline
    Type: & logical \\
    \hline
    Units: & -- \\
    \hline
    Default Value: & .false. \\
    \hline
    Possible Values: & .true. or .false. \\
    \hline
    \caption{config\_AM\_mocStreamfunction\_enable: If true, ocean analysis member MOC streamfunction is called.}
\end{longtable}
\end{center}
\subsection[config\_AM\_mocStreamfunction\_compute\_interval]{\hyperref[sec:nm_tab_AM_mocStreamfunction]{config\_AM\_mocStreamfunction\_compute\_interval}}
\label{subsec:nm_sec_config_AM_mocStreamfunction_compute_interval}
\begin{center}
\begin{longtable}{| p{2.0in} || p{4.0in} |}
    \hline
    Type: & character \\
    \hline
    Units: & -- \\
    \hline
    Default Value: & output\_interval \\
    \hline
    Possible Values: & Any valid time stamp, 'dt', or 'output\_interval' \\
    \hline
    \caption{config\_AM\_mocStreamfunction\_compute\_interval: Timestamp determining how often analysis member computation should be performed.}
\end{longtable}
\end{center}
\subsection[config\_AM\_mocStreamfunction\_output\_stream]{\hyperref[sec:nm_tab_AM_mocStreamfunction]{config\_AM\_mocStreamfunction\_output\_stream}}
\label{subsec:nm_sec_config_AM_mocStreamfunction_output_stream}
\begin{center}
\begin{longtable}{| p{2.0in} || p{4.0in} |}
    \hline
    Type: & character \\
    \hline
    Units: & -- \\
    \hline
    Default Value: & mocStreamfunctionOutput \\
    \hline
    Possible Values: & Any existing stream name or 'none' \\
    \hline
    \caption{config\_AM\_mocStreamfunction\_output\_stream: Name of the stream that the mocStreamfunction analysis member should be tied to.}
\end{longtable}
\end{center}
\subsection[config\_AM\_mocStreamfunction\_compute\_on\_startup]{\hyperref[sec:nm_tab_AM_mocStreamfunction]{config\_AM\_mocStreamfunction\_compute\_on\_startup}}
\label{subsec:nm_sec_config_AM_mocStreamfunction_compute_on_startup}
\begin{center}
\begin{longtable}{| p{2.0in} || p{4.0in} |}
    \hline
    Type: & logical \\
    \hline
    Units: & -- \\
    \hline
    Default Value: & .true. \\
    \hline
    Possible Values: & .true. or .false. \\
    \hline
    \caption{config\_AM\_mocStreamfunction\_compute\_on\_startup: Logical flag determining if an analysis member computation occurs on start-up.}
\end{longtable}
\end{center}
\subsection[config\_AM\_mocStreamfunction\_write\_on\_startup]{\hyperref[sec:nm_tab_AM_mocStreamfunction]{config\_AM\_mocStreamfunction\_write\_on\_startup}}
\label{subsec:nm_sec_config_AM_mocStreamfunction_write_on_startup}
\begin{center}
\begin{longtable}{| p{2.0in} || p{4.0in} |}
    \hline
    Type: & logical \\
    \hline
    Units: & -- \\
    \hline
    Default Value: & .true. \\
    \hline
    Possible Values: & .true. or .false. \\
    \hline
    \caption{config\_AM\_mocStreamfunction\_write\_on\_startup: Logical flag determining if an analysis member write occurs on start-up.}
\end{longtable}
\end{center}
\subsection[config\_AM\_mocStreamfunction\_min\_bin]{\hyperref[sec:nm_tab_AM_mocStreamfunction]{config\_AM\_mocStreamfunction\_min\_bin}}
\label{subsec:nm_sec_config_AM_mocStreamfunction_min_bin}
\begin{center}
\begin{longtable}{| p{2.0in} || p{4.0in} |}
    \hline
    Type: & real \\
    \hline
    Units: & \si{varies} \\
    \hline
    Default Value: & -1.0e34 \\
    \hline
    Possible Values: & Any real number. \\
    \hline
    \caption{config\_AM\_mocStreamfunction\_min\_bin: minimum bin boundary value.  If set to -1.0e34, the minimum value in the domain is found.}
\end{longtable}
\end{center}
\subsection[config\_AM\_mocStreamfunction\_max\_bin]{\hyperref[sec:nm_tab_AM_mocStreamfunction]{config\_AM\_mocStreamfunction\_max\_bin}}
\label{subsec:nm_sec_config_AM_mocStreamfunction_max_bin}
\begin{center}
\begin{longtable}{| p{2.0in} || p{4.0in} |}
    \hline
    Type: & real \\
    \hline
    Units: & \si{varies} \\
    \hline
    Default Value: & -1.0e34 \\
    \hline
    Possible Values: & Any real number. \\
    \hline
    \caption{config\_AM\_mocStreamfunction\_max\_bin: maximum bin boundary value.  If set to -1.0e34, the maximum value in the domain is found.}
\end{longtable}
\end{center}
\subsection[config\_AM\_mocStreamfunction\_num\_bins]{\hyperref[sec:nm_tab_AM_mocStreamfunction]{config\_AM\_mocStreamfunction\_num\_bins}}
\label{subsec:nm_sec_config_AM_mocStreamfunction_num_bins}
\begin{center}
\begin{longtable}{| p{2.0in} || p{4.0in} |}
    \hline
    Type: & integer \\
    \hline
    Units: & -- \\
    \hline
    Default Value: & 180 \\
    \hline
    Possible Values: & Any positive integer value less than or equal to nMocStreamfunctionBins. \\
    \hline
    \caption{config\_AM\_mocStreamfunction\_num\_bins: Number of bins in South-to-North direction used for moc streamfunction calculation.}
\end{longtable}
\end{center}
\subsection[config\_AM\_mocStreamfunction\_region\_group]{\hyperref[sec:nm_tab_AM_mocStreamfunction]{config\_AM\_mocStreamfunction\_region\_group}}
\label{subsec:nm_sec_config_AM_mocStreamfunction_region_group}
\begin{center}
\begin{longtable}{| p{2.0in} || p{4.0in} |}
    \hline
    Type: & character \\
    \hline
    Units: & -- \\
    \hline
    Default Value: & all \\
    \hline
    Possible Values: & Either of '', 'all' or any valid region group name. \\
    \hline
    \caption{config\_AM\_mocStreamfunction\_region\_group: The name of the region group, for which the moc should be computed in addition to the global MOC.}
\end{longtable}
\end{center}
\subsection[config\_AM\_mocStreamfunction\_transect\_group]{\hyperref[sec:nm_tab_AM_mocStreamfunction]{config\_AM\_mocStreamfunction\_transect\_group}}
\label{subsec:nm_sec_config_AM_mocStreamfunction_transect_group}
\begin{center}
\begin{longtable}{| p{2.0in} || p{4.0in} |}
    \hline
    Type: & character \\
    \hline
    Units: & -- \\
    \hline
    Default Value: & all \\
    \hline
    Possible Values: & Any valid transect group name. \\
    \hline
    \caption{config\_AM\_mocStreamfunction\_transect\_group: The name of the transect group that holds the boundaries for the regions in the region group, configured in 'config\_AM\_mocStreamfunction\_region\_group'. Please note, that these two should have the same amount of entries.}
\end{longtable}
\end{center}
\section[AM\_oceanHeatContent]{\hyperref[sec:nm_tab_AM_oceanHeatContent]{AM\_oceanHeatContent}}
\label{sec:nm_sec_AM_oceanHeatContent}
\subsection[config\_AM\_oceanHeatContent\_enable]{\hyperref[sec:nm_tab_AM_oceanHeatContent]{config\_AM\_oceanHeatContent\_enable}}
\label{subsec:nm_sec_config_AM_oceanHeatContent_enable}
\begin{center}
\begin{longtable}{| p{2.0in} || p{4.0in} |}
    \hline
    Type: & logical \\
    \hline
    Units: & -- \\
    \hline
    Default Value: & .false. \\
    \hline
    Possible Values: & .true. or .false. \\
    \hline
    \caption{config\_AM\_oceanHeatContent\_enable: If true, ocean analysis member ocean heat content is called.}
\end{longtable}
\end{center}
\subsection[config\_AM\_oceanHeatContent\_compute\_interval]{\hyperref[sec:nm_tab_AM_oceanHeatContent]{config\_AM\_oceanHeatContent\_compute\_interval}}
\label{subsec:nm_sec_config_AM_oceanHeatContent_compute_interval}
\begin{center}
\begin{longtable}{| p{2.0in} || p{4.0in} |}
    \hline
    Type: & character \\
    \hline
    Units: & -- \\
    \hline
    Default Value: & output\_interval \\
    \hline
    Possible Values: & Any valid time stamp, 'dt', or 'output\_interval' \\
    \hline
    \caption{config\_AM\_oceanHeatContent\_compute\_interval: Timestamp determining how often analysis member computation should be performed.}
\end{longtable}
\end{center}
\subsection[config\_AM\_oceanHeatContent\_output\_stream]{\hyperref[sec:nm_tab_AM_oceanHeatContent]{config\_AM\_oceanHeatContent\_output\_stream}}
\label{subsec:nm_sec_config_AM_oceanHeatContent_output_stream}
\begin{center}
\begin{longtable}{| p{2.0in} || p{4.0in} |}
    \hline
    Type: & character \\
    \hline
    Units: & -- \\
    \hline
    Default Value: & oceanHeatContentOutput \\
    \hline
    Possible Values: & Any existing stream name or 'none' \\
    \hline
    \caption{config\_AM\_oceanHeatContent\_output\_stream: Name of the stream that the oceanHeatContent analysis member should be tied to.}
\end{longtable}
\end{center}
\subsection[config\_AM\_oceanHeatContent\_compute\_on\_startup]{\hyperref[sec:nm_tab_AM_oceanHeatContent]{config\_AM\_oceanHeatContent\_compute\_on\_startup}}
\label{subsec:nm_sec_config_AM_oceanHeatContent_compute_on_startup}
\begin{center}
\begin{longtable}{| p{2.0in} || p{4.0in} |}
    \hline
    Type: & logical \\
    \hline
    Units: & -- \\
    \hline
    Default Value: & .true. \\
    \hline
    Possible Values: & .true. or .false. \\
    \hline
    \caption{config\_AM\_oceanHeatContent\_compute\_on\_startup: Logical flag determining if an analysis member computation occurs on start-up.}
\end{longtable}
\end{center}
\subsection[config\_AM\_oceanHeatContent\_write\_on\_startup]{\hyperref[sec:nm_tab_AM_oceanHeatContent]{config\_AM\_oceanHeatContent\_write\_on\_startup}}
\label{subsec:nm_sec_config_AM_oceanHeatContent_write_on_startup}
\begin{center}
\begin{longtable}{| p{2.0in} || p{4.0in} |}
    \hline
    Type: & logical \\
    \hline
    Units: & -- \\
    \hline
    Default Value: & .true. \\
    \hline
    Possible Values: & .true. or .false. \\
    \hline
    \caption{config\_AM\_oceanHeatContent\_write\_on\_startup: Logical flag determining if an analysis member write occurs on start-up.}
\end{longtable}
\end{center}
\section[AM\_mixedLayerHeatBudget]{\hyperref[sec:nm_tab_AM_mixedLayerHeatBudget]{AM\_mixedLayerHeatBudget}}
\label{sec:nm_sec_AM_mixedLayerHeatBudget}
\subsection[config\_AM\_mixedLayerHeatBudget\_enable]{\hyperref[sec:nm_tab_AM_mixedLayerHeatBudget]{config\_AM\_mixedLayerHeatBudget\_enable}}
\label{subsec:nm_sec_config_AM_mixedLayerHeatBudget_enable}
\begin{center}
\begin{longtable}{| p{2.0in} || p{4.0in} |}
    \hline
    Type: & logical \\
    \hline
    Units: & -- \\
    \hline
    Default Value: & .false. \\
    \hline
    Possible Values: & .true. or .false. \\
    \hline
    \caption{config\_AM\_mixedLayerHeatBudget\_enable: If true, ocean analysis member mixedLayerHeatBudget is called.}
\end{longtable}
\end{center}
\subsection[config\_AM\_mixedLayerHeatBudget\_compute\_interval]{\hyperref[sec:nm_tab_AM_mixedLayerHeatBudget]{config\_AM\_mixedLayerHeatBudget\_compute\_interval}}
\label{subsec:nm_sec_config_AM_mixedLayerHeatBudget_compute_interval}
\begin{center}
\begin{longtable}{| p{2.0in} || p{4.0in} |}
    \hline
    Type: & character \\
    \hline
    Units: & -- \\
    \hline
    Default Value: & output\_interval \\
    \hline
    Possible Values: & Any valid time stamp, 'dt', or 'output\_interval' \\
    \hline
    \caption{config\_AM\_mixedLayerHeatBudget\_compute\_interval: Timestamp determining how often analysis member computation should be performed.}
\end{longtable}
\end{center}
\subsection[config\_AM\_mixedLayerHeatBudget\_output\_stream]{\hyperref[sec:nm_tab_AM_mixedLayerHeatBudget]{config\_AM\_mixedLayerHeatBudget\_output\_stream}}
\label{subsec:nm_sec_config_AM_mixedLayerHeatBudget_output_stream}
\begin{center}
\begin{longtable}{| p{2.0in} || p{4.0in} |}
    \hline
    Type: & character \\
    \hline
    Units: & -- \\
    \hline
    Default Value: & mixedLayerHeatBudgetOutput \\
    \hline
    Possible Values: & Any existing stream name or 'none' \\
    \hline
    \caption{config\_AM\_mixedLayerHeatBudget\_output\_stream: Name of the stream that the mixedLayerHeatBudget analysis member should be tied to.}
\end{longtable}
\end{center}
\subsection[config\_AM\_mixedLayerHeatBudget\_compute\_on\_startup]{\hyperref[sec:nm_tab_AM_mixedLayerHeatBudget]{config\_AM\_mixedLayerHeatBudget\_compute\_on\_startup}}
\label{subsec:nm_sec_config_AM_mixedLayerHeatBudget_compute_on_startup}
\begin{center}
\begin{longtable}{| p{2.0in} || p{4.0in} |}
    \hline
    Type: & logical \\
    \hline
    Units: & -- \\
    \hline
    Default Value: & .true. \\
    \hline
    Possible Values: & .true. or .false. \\
    \hline
    \caption{config\_AM\_mixedLayerHeatBudget\_compute\_on\_startup: Logical flag determining if an analysis member computation occurs on start-up.}
\end{longtable}
\end{center}
\subsection[config\_AM\_mixedLayerHeatBudget\_write\_on\_startup]{\hyperref[sec:nm_tab_AM_mixedLayerHeatBudget]{config\_AM\_mixedLayerHeatBudget\_write\_on\_startup}}
\label{subsec:nm_sec_config_AM_mixedLayerHeatBudget_write_on_startup}
\begin{center}
\begin{longtable}{| p{2.0in} || p{4.0in} |}
    \hline
    Type: & logical \\
    \hline
    Units: & -- \\
    \hline
    Default Value: & .true. \\
    \hline
    Possible Values: & .true. or .false. \\
    \hline
    \caption{config\_AM\_mixedLayerHeatBudget\_write\_on\_startup: Logical flag determining if an analysis member write occurs on start-up.}
\end{longtable}
\end{center}
\section[AM\_sedimentFluxIndex]{\hyperref[sec:nm_tab_AM_sedimentFluxIndex]{AM\_sedimentFluxIndex}}
\label{sec:nm_sec_AM_sedimentFluxIndex}
\subsection[config\_AM\_sedimentFluxIndex\_enable]{\hyperref[sec:nm_tab_AM_sedimentFluxIndex]{config\_AM\_sedimentFluxIndex\_enable}}
\label{subsec:nm_sec_config_AM_sedimentFluxIndex_enable}
\begin{center}
\begin{longtable}{| p{2.0in} || p{4.0in} |}
    \hline
    Type: & logical \\
    \hline
    Units: & -- \\
    \hline
    Default Value: & .false. \\
    \hline
    Possible Values: & .true. or .false. \\
    \hline
    \caption{config\_AM\_sedimentFluxIndex\_enable: If true, ocean analysis member sedimentFluxIndex is called.}
\end{longtable}
\end{center}
\subsection[config\_AM\_sedimentFluxIndex\_compute\_on\_startup]{\hyperref[sec:nm_tab_AM_sedimentFluxIndex]{config\_AM\_sedimentFluxIndex\_compute\_on\_startup}}
\label{subsec:nm_sec_config_AM_sedimentFluxIndex_compute_on_startup}
\begin{center}
\begin{longtable}{| p{2.0in} || p{4.0in} |}
    \hline
    Type: & logical \\
    \hline
    Units: & -- \\
    \hline
    Default Value: & .true. \\
    \hline
    Possible Values: & .true. or .false. \\
    \hline
    \caption{config\_AM\_sedimentFluxIndex\_compute\_on\_startup: Logical flag determining if an analysis member computation occurs on start-up.}
\end{longtable}
\end{center}
\subsection[config\_AM\_sedimentFluxIndex\_write\_on\_startup]{\hyperref[sec:nm_tab_AM_sedimentFluxIndex]{config\_AM\_sedimentFluxIndex\_write\_on\_startup}}
\label{subsec:nm_sec_config_AM_sedimentFluxIndex_write_on_startup}
\begin{center}
\begin{longtable}{| p{2.0in} || p{4.0in} |}
    \hline
    Type: & logical \\
    \hline
    Units: & -- \\
    \hline
    Default Value: & .true. \\
    \hline
    Possible Values: & .true. or .false. \\
    \hline
    \caption{config\_AM\_sedimentFluxIndex\_write\_on\_startup: Logical flag determining if an analysis member computation occurs on start-up.}
\end{longtable}
\end{center}
\subsection[config\_AM\_sedimentFluxIndex\_compute\_interval]{\hyperref[sec:nm_tab_AM_sedimentFluxIndex]{config\_AM\_sedimentFluxIndex\_compute\_interval}}
\label{subsec:nm_sec_config_AM_sedimentFluxIndex_compute_interval}
\begin{center}
\begin{longtable}{| p{2.0in} || p{4.0in} |}
    \hline
    Type: & character \\
    \hline
    Units: & -- \\
    \hline
    Default Value: & output\_interval \\
    \hline
    Possible Values: & Any time stamp, 'dt', or 'output\_interval' \\
    \hline
    \caption{config\_AM\_sedimentFluxIndex\_compute\_interval: Time stamp for frequency of computation of the sedimentFluxIndex analysis member.}
\end{longtable}
\end{center}
\subsection[config\_AM\_sedimentFluxIndex\_output\_stream]{\hyperref[sec:nm_tab_AM_sedimentFluxIndex]{config\_AM\_sedimentFluxIndex\_output\_stream}}
\label{subsec:nm_sec_config_AM_sedimentFluxIndex_output_stream}
\begin{center}
\begin{longtable}{| p{2.0in} || p{4.0in} |}
    \hline
    Type: & character \\
    \hline
    Units: & -- \\
    \hline
    Default Value: & sedimentFluxIndexOutput \\
    \hline
    Possible Values: & Any existing stream name or 'none' \\
    \hline
    \caption{config\_AM\_sedimentFluxIndex\_output\_stream: Name of stream the sedimentFluxIndex analysis member should be tied to}
\end{longtable}
\end{center}
\subsection[config\_AM\_sedimentFluxIndex\_directory]{\hyperref[sec:nm_tab_AM_sedimentFluxIndex]{config\_AM\_sedimentFluxIndex\_directory}}
\label{subsec:nm_sec_config_AM_sedimentFluxIndex_directory}
\begin{center}
\begin{longtable}{| p{2.0in} || p{4.0in} |}
    \hline
    Type: & character \\
    \hline
    Units: & -- \\
    \hline
    Default Value: & analysis\_members \\
    \hline
    Possible Values: & any valid directory name \\
    \hline
    \caption{config\_AM\_sedimentFluxIndex\_directory: subdirectory to write text files (might useful)}
\end{longtable}
\end{center}
\subsection[config\_AM\_sedimentFluxIndex\_use\_lat\_lon\_coords]{\hyperref[sec:nm_tab_AM_sedimentFluxIndex]{config\_AM\_sedimentFluxIndex\_use\_lat\_lon\_coords}}
\label{subsec:nm_sec_config_AM_sedimentFluxIndex_use_lat_lon_coords}
\begin{center}
\begin{longtable}{| p{2.0in} || p{4.0in} |}
    \hline
    Type: & logical \\
    \hline
    Units: & -- \\
    \hline
    Default Value: & .true. \\
    \hline
    Possible Values: & .true. or .false. \\
    \hline
    \caption{config\_AM\_sedimentFluxIndex\_use\_lat\_lon\_coords: If true, latitude/longitude coordinates are output for eddy census. Otherwise x/y/z coordinates are used. Ignored if not on a sphere.}
\end{longtable}
\end{center}
\section[AM\_sedimentTransport]{\hyperref[sec:nm_tab_AM_sedimentTransport]{AM\_sedimentTransport}}
\label{sec:nm_sec_AM_sedimentTransport}
\subsection[config\_AM\_sedimentTransport\_enable]{\hyperref[sec:nm_tab_AM_sedimentTransport]{config\_AM\_sedimentTransport\_enable}}
\label{subsec:nm_sec_config_AM_sedimentTransport_enable}
\begin{center}
\begin{longtable}{| p{2.0in} || p{4.0in} |}
    \hline
    Type: & logical \\
    \hline
    Units: & -- \\
    \hline
    Default Value: & .false. \\
    \hline
    Possible Values: & .true. or .false. \\
    \hline
    \caption{config\_AM\_sedimentTransport\_enable: If true, ocean analysis member sedimentTransport is called.}
\end{longtable}
\end{center}
\subsection[config\_AM\_sedimentTransport\_compute\_on\_startup]{\hyperref[sec:nm_tab_AM_sedimentTransport]{config\_AM\_sedimentTransport\_compute\_on\_startup}}
\label{subsec:nm_sec_config_AM_sedimentTransport_compute_on_startup}
\begin{center}
\begin{longtable}{| p{2.0in} || p{4.0in} |}
    \hline
    Type: & logical \\
    \hline
    Units: & -- \\
    \hline
    Default Value: & .true. \\
    \hline
    Possible Values: & .true. or .false. \\
    \hline
    \caption{config\_AM\_sedimentTransport\_compute\_on\_startup: Logical flag determining if an analysis member computation occurs on start-up.}
\end{longtable}
\end{center}
\subsection[config\_AM\_sedimentTransport\_write\_on\_startup]{\hyperref[sec:nm_tab_AM_sedimentTransport]{config\_AM\_sedimentTransport\_write\_on\_startup}}
\label{subsec:nm_sec_config_AM_sedimentTransport_write_on_startup}
\begin{center}
\begin{longtable}{| p{2.0in} || p{4.0in} |}
    \hline
    Type: & logical \\
    \hline
    Units: & -- \\
    \hline
    Default Value: & .true. \\
    \hline
    Possible Values: & .true. or .false. \\
    \hline
    \caption{config\_AM\_sedimentTransport\_write\_on\_startup: Logical flag determining if an analysis member computation occurs on start-up.}
\end{longtable}
\end{center}
\subsection[config\_AM\_sedimentTransport\_compute\_interval]{\hyperref[sec:nm_tab_AM_sedimentTransport]{config\_AM\_sedimentTransport\_compute\_interval}}
\label{subsec:nm_sec_config_AM_sedimentTransport_compute_interval}
\begin{center}
\begin{longtable}{| p{2.0in} || p{4.0in} |}
    \hline
    Type: & character \\
    \hline
    Units: & -- \\
    \hline
    Default Value: & output\_interval \\
    \hline
    Possible Values: & Any time stamp, 'dt', or 'output\_interval' \\
    \hline
    \caption{config\_AM\_sedimentTransport\_compute\_interval: Time stamp for frequency of computation of the sedimentTransport analysis member.}
\end{longtable}
\end{center}
\subsection[config\_AM\_sedimentTransport\_output\_stream]{\hyperref[sec:nm_tab_AM_sedimentTransport]{config\_AM\_sedimentTransport\_output\_stream}}
\label{subsec:nm_sec_config_AM_sedimentTransport_output_stream}
\begin{center}
\begin{longtable}{| p{2.0in} || p{4.0in} |}
    \hline
    Type: & character \\
    \hline
    Units: & -- \\
    \hline
    Default Value: & sedimentTransportOutput \\
    \hline
    Possible Values: & Any existing stream name or 'none' \\
    \hline
    \caption{config\_AM\_sedimentTransport\_output\_stream: Name of stream the sedimentTransport analysis member should be tied to}
\end{longtable}
\end{center}
\subsection[config\_AM\_sedimentTransport\_directory]{\hyperref[sec:nm_tab_AM_sedimentTransport]{config\_AM\_sedimentTransport\_directory}}
\label{subsec:nm_sec_config_AM_sedimentTransport_directory}
\begin{center}
\begin{longtable}{| p{2.0in} || p{4.0in} |}
    \hline
    Type: & character \\
    \hline
    Units: & -- \\
    \hline
    Default Value: & analysis\_members \\
    \hline
    Possible Values: & any valid directory name \\
    \hline
    \caption{config\_AM\_sedimentTransport\_directory: subdirectory to write text files (might useful)}
\end{longtable}
\end{center}
\subsection[config\_AM\_sedimentTransport\_grain\_size]{\hyperref[sec:nm_tab_AM_sedimentTransport]{config\_AM\_sedimentTransport\_grain\_size}}
\label{subsec:nm_sec_config_AM_sedimentTransport_grain_size}
\begin{center}
\begin{longtable}{| p{2.0in} || p{4.0in} |}
    \hline
    Type: & real \\
    \hline
    Units: & \si{m} \\
    \hline
    Default Value: & 2.5e-4 \\
    \hline
    Possible Values: & 1e-4 ~1e-3 \\
    \hline
    \caption{config\_AM\_sedimentTransport\_grain\_size: dimaeter of a spherical sediment particle}
\end{longtable}
\end{center}
\subsection[config\_AM\_sedimentTransport\_ws\_formula]{\hyperref[sec:nm_tab_AM_sedimentTransport]{config\_AM\_sedimentTransport\_ws\_formula}}
\label{subsec:nm_sec_config_AM_sedimentTransport_ws_formula}
\begin{center}
\begin{longtable}{| p{2.0in} || p{4.0in} |}
    \hline
    Type: & character \\
    \hline
    Units: & -- \\
    \hline
    Default Value: & VanRijn1993 \\
    \hline
    Possible Values: & VanRijn1993, Soulsby1997, Cheng1997, Goldstein-Coco2013 \\
    \hline
    \caption{config\_AM\_sedimentTransport\_ws\_formula: options of different settling velocity formulae}
\end{longtable}
\end{center}
\subsection[config\_AM\_sedimentTransport\_bedld\_formula]{\hyperref[sec:nm_tab_AM_sedimentTransport]{config\_AM\_sedimentTransport\_bedld\_formula}}
\label{subsec:nm_sec_config_AM_sedimentTransport_bedld_formula}
\begin{center}
\begin{longtable}{| p{2.0in} || p{4.0in} |}
    \hline
    Type: & character \\
    \hline
    Units: & -- \\
    \hline
    Default Value: & Soulsby-Damgaard \\
    \hline
    Possible Values: & Soulsby-Damgaard, Meyer-Peter-Mueller, Engelund-Hansen \\
    \hline
    \caption{config\_AM\_sedimentTransport\_bedld\_formula: options of diffrent sediment bedload transport formulae}
\end{longtable}
\end{center}
\subsection[config\_AM\_sedimentTransport\_SSC\_ref\_formula]{\hyperref[sec:nm_tab_AM_sedimentTransport]{config\_AM\_sedimentTransport\_SSC\_ref\_formula}}
\label{subsec:nm_sec_config_AM_sedimentTransport_SSC_ref_formula}
\begin{center}
\begin{longtable}{| p{2.0in} || p{4.0in} |}
    \hline
    Type: & character \\
    \hline
    Units: & -- \\
    \hline
    Default Value: & Lee2004 \\
    \hline
    Possible Values: & Lee2004, Goldstein2014,Zyserman-Fredsoe1994 \\
    \hline
    \caption{config\_AM\_sedimentTransport\_SSC\_ref\_formula: options of diffrent near-bottom suspended sediment concentration formulae}
\end{longtable}
\end{center}
\subsection[config\_AM\_sedimentTransport\_drag\_coefficient]{\hyperref[sec:nm_tab_AM_sedimentTransport]{config\_AM\_sedimentTransport\_drag\_coefficient}}
\label{subsec:nm_sec_config_AM_sedimentTransport_drag_coefficient}
\begin{center}
\begin{longtable}{| p{2.0in} || p{4.0in} |}
    \hline
    Type: & real \\
    \hline
    Units: & -- \\
    \hline
    Default Value: & 2.5e-3 \\
    \hline
    Possible Values: & any values between 1~2.5e-3 \\
    \hline
    \caption{config\_AM\_sedimentTransport\_drag\_coefficient: drag coefficient used for bottom shear stress computation}
\end{longtable}
\end{center}
\subsection[config\_AM\_sedimentTransport\_erate]{\hyperref[sec:nm_tab_AM_sedimentTransport]{config\_AM\_sedimentTransport\_erate}}
\label{subsec:nm_sec_config_AM_sedimentTransport_erate}
\begin{center}
\begin{longtable}{| p{2.0in} || p{4.0in} |}
    \hline
    Type: & real \\
    \hline
    Units: & \si{kg.m^-2.s^-1} \\
    \hline
    Default Value: & 5.0e-4 \\
    \hline
    Possible Values: & any positive values \\
    \hline
    \caption{config\_AM\_sedimentTransport\_erate: bed surface erosion rate}
\end{longtable}
\end{center}
\subsection[config\_AM\_sedimentTransport\_tau\_ce]{\hyperref[sec:nm_tab_AM_sedimentTransport]{config\_AM\_sedimentTransport\_tau\_ce}}
\label{subsec:nm_sec_config_AM_sedimentTransport_tau_ce}
\begin{center}
\begin{longtable}{| p{2.0in} || p{4.0in} |}
    \hline
    Type: & real \\
    \hline
    Units: & \si{N.m^-2} \\
    \hline
    Default Value: & 0.1 \\
    \hline
    Possible Values: & any positive values \\
    \hline
    \caption{config\_AM\_sedimentTransport\_tau\_ce: critical shear for erosion}
\end{longtable}
\end{center}
\subsection[config\_AM\_sedimentTransport\_tau\_cd]{\hyperref[sec:nm_tab_AM_sedimentTransport]{config\_AM\_sedimentTransport\_tau\_cd}}
\label{subsec:nm_sec_config_AM_sedimentTransport_tau_cd}
\begin{center}
\begin{longtable}{| p{2.0in} || p{4.0in} |}
    \hline
    Type: & real \\
    \hline
    Units: & \si{N.m^-2} \\
    \hline
    Default Value: & 0.1 \\
    \hline
    Possible Values: & any positive values \\
    \hline
    \caption{config\_AM\_sedimentTransport\_tau\_cd: critical shear for deposition}
\end{longtable}
\end{center}
\subsection[config\_AM\_sedimentTransport\_Manning\_coef]{\hyperref[sec:nm_tab_AM_sedimentTransport]{config\_AM\_sedimentTransport\_Manning\_coef}}
\label{subsec:nm_sec_config_AM_sedimentTransport_Manning_coef}
\begin{center}
\begin{longtable}{| p{2.0in} || p{4.0in} |}
    \hline
    Type: & real \\
    \hline
    Units: & \si{s.m^-1/3} \\
    \hline
    Default Value: & 0.022 \\
    \hline
    Possible Values: & any positive values,typical values are between 0.012 and 0.025, Kerr et al., 2013 \\
    \hline
    \caption{config\_AM\_sedimentTransport\_Manning\_coef: Manning roughness coefficient}
\end{longtable}
\end{center}
\subsection[config\_AM\_sedimentTransport\_grain\_porosity]{\hyperref[sec:nm_tab_AM_sedimentTransport]{config\_AM\_sedimentTransport\_grain\_porosity}}
\label{subsec:nm_sec_config_AM_sedimentTransport_grain_porosity}
\begin{center}
\begin{longtable}{| p{2.0in} || p{4.0in} |}
    \hline
    Type: & real \\
    \hline
    Units: & -- \\
    \hline
    Default Value: & 0.5 \\
    \hline
    Possible Values: & any positive value between 0 and 1 \\
    \hline
    \caption{config\_AM\_sedimentTransport\_grain\_porosity: sediment porosity}
\end{longtable}
\end{center}
\subsection[config\_AM\_sedimentTransport\_water\_density]{\hyperref[sec:nm_tab_AM_sedimentTransport]{config\_AM\_sedimentTransport\_water\_density}}
\label{subsec:nm_sec_config_AM_sedimentTransport_water_density}
\begin{center}
\begin{longtable}{| p{2.0in} || p{4.0in} |}
    \hline
    Type: & real \\
    \hline
    Units: & \si{kg.s.m^-4} \\
    \hline
    Default Value: & 1020 \\
    \hline
    Possible Values: & any positive value between 1000 and 1030 \\
    \hline
    \caption{config\_AM\_sedimentTransport\_water\_density: water density}
\end{longtable}
\end{center}
\subsection[config\_AM\_sedimentTransport\_grain\_density]{\hyperref[sec:nm_tab_AM_sedimentTransport]{config\_AM\_sedimentTransport\_grain\_density}}
\label{subsec:nm_sec_config_AM_sedimentTransport_grain_density}
\begin{center}
\begin{longtable}{| p{2.0in} || p{4.0in} |}
    \hline
    Type: & real \\
    \hline
    Units: & \si{kg.s.m^-4} \\
    \hline
    Default Value: & 2650 \\
    \hline
    Possible Values: & any positive value \\
    \hline
    \caption{config\_AM\_sedimentTransport\_grain\_density: sediment density}
\end{longtable}
\end{center}
\subsection[config\_AM\_sedimentTransport\_alpha]{\hyperref[sec:nm_tab_AM_sedimentTransport]{config\_AM\_sedimentTransport\_alpha}}
\label{subsec:nm_sec_config_AM_sedimentTransport_alpha}
\begin{center}
\begin{longtable}{| p{2.0in} || p{4.0in} |}
    \hline
    Type: & real \\
    \hline
    Units: & \si{kg.s.m^-4} \\
    \hline
    Default Value: & 1e-2 \\
    \hline
    Possible Values: & 1e-4 ~1e-3 \\
    \hline
    \caption{config\_AM\_sedimentTransport\_alpha: A parameter related to the sediment property, with typical values of O(1e-4~1e-3)}
\end{longtable}
\end{center}
\subsection[config\_AM\_sedimentTransport\_kinematic\_viscosity]{\hyperref[sec:nm_tab_AM_sedimentTransport]{config\_AM\_sedimentTransport\_kinematic\_viscosity}}
\label{subsec:nm_sec_config_AM_sedimentTransport_kinematic_viscosity}
\begin{center}
\begin{longtable}{| p{2.0in} || p{4.0in} |}
    \hline
    Type: & real \\
    \hline
    Units: & \si{m^2.s^-1} \\
    \hline
    Default Value: & 1e-6 \\
    \hline
    Possible Values: & any positive value \\
    \hline
    \caption{config\_AM\_sedimentTransport\_kinematic\_viscosity: kinematic viscosity of the fluild}
\end{longtable}
\end{center}
\subsection[config\_AM\_sedimentTransport\_vertical\_diffusion\_coefficient]{\hyperref[sec:nm_tab_AM_sedimentTransport]{config\_AM\_sedimentTransport\_vertical\_diffusion\_coefficient}}
\label{subsec:nm_sec_config_AM_sedimentTransport_vertical_diffusion_coefficient}
\begin{center}
\begin{longtable}{| p{2.0in} || p{4.0in} |}
    \hline
    Type: & real \\
    \hline
    Units: & \si{m^2.s^-1} \\
    \hline
    Default Value: & 1e-2 \\
    \hline
    Possible Values: & any positive value \\
    \hline
    \caption{config\_AM\_sedimentTransport\_vertical\_diffusion\_coefficient: vertical diffsuion coefficient}
\end{longtable}
\end{center}
\subsection[config\_AM\_sedimentTransport\_bedload]{\hyperref[sec:nm_tab_AM_sedimentTransport]{config\_AM\_sedimentTransport\_bedload}}
\label{subsec:nm_sec_config_AM_sedimentTransport_bedload}
\begin{center}
\begin{longtable}{| p{2.0in} || p{4.0in} |}
    \hline
    Type: & logical \\
    \hline
    Units: & -- \\
    \hline
    Default Value: & .true. \\
    \hline
    Possible Values: & .true. or .false. \\
    \hline
    \caption{config\_AM\_sedimentTransport\_bedload: Logical flag determining if bedload transport is to be computed.}
\end{longtable}
\end{center}
\subsection[config\_AM\_sedimentTransport\_suspended]{\hyperref[sec:nm_tab_AM_sedimentTransport]{config\_AM\_sedimentTransport\_suspended}}
\label{subsec:nm_sec_config_AM_sedimentTransport_suspended}
\begin{center}
\begin{longtable}{| p{2.0in} || p{4.0in} |}
    \hline
    Type: & logical \\
    \hline
    Units: & -- \\
    \hline
    Default Value: & .true. \\
    \hline
    Possible Values: & .true. or .false. \\
    \hline
    \caption{config\_AM\_sedimentTransport\_suspended: Logical flag determining if suspended transport is to be computed.}
\end{longtable}
\end{center}
\subsection[config\_AM\_sedimentTransport\_use\_lat\_lon\_coords]{\hyperref[sec:nm_tab_AM_sedimentTransport]{config\_AM\_sedimentTransport\_use\_lat\_lon\_coords}}
\label{subsec:nm_sec_config_AM_sedimentTransport_use_lat_lon_coords}
\begin{center}
\begin{longtable}{| p{2.0in} || p{4.0in} |}
    \hline
    Type: & logical \\
    \hline
    Units: & -- \\
    \hline
    Default Value: & .true. \\
    \hline
    Possible Values: & .true. or .false. \\
    \hline
    \caption{config\_AM\_sedimentTransport\_use\_lat\_lon\_coords: If true, latitude/longitude coordinates are output for eddy census. Otherwise x/y/z coordinates are used. Ignored if not on a sphere.}
\end{longtable}
\end{center}
\section[AM\_harmonicAnalysis]{\hyperref[sec:nm_tab_AM_harmonicAnalysis]{AM\_harmonicAnalysis}}
\label{sec:nm_sec_AM_harmonicAnalysis}
\subsection[config\_AM\_harmonicAnalysis\_enable]{\hyperref[sec:nm_tab_AM_harmonicAnalysis]{config\_AM\_harmonicAnalysis\_enable}}
\label{subsec:nm_sec_config_AM_harmonicAnalysis_enable}
\begin{center}
\begin{longtable}{| p{2.0in} || p{4.0in} |}
    \hline
    Type: & logical \\
    \hline
    Units: & -- \\
    \hline
    Default Value: & .false. \\
    \hline
    Possible Values: & .true. or .false. \\
    \hline
    \caption{config\_AM\_harmonicAnalysis\_enable: If true, ocean analysis member harmonicAnalysis is called.}
\end{longtable}
\end{center}
\subsection[config\_AM\_harmonicAnalysis\_compute\_interval]{\hyperref[sec:nm_tab_AM_harmonicAnalysis]{config\_AM\_harmonicAnalysis\_compute\_interval}}
\label{subsec:nm_sec_config_AM_harmonicAnalysis_compute_interval}
\begin{center}
\begin{longtable}{| p{2.0in} || p{4.0in} |}
    \hline
    Type: & character \\
    \hline
    Units: & -- \\
    \hline
    Default Value: & output\_interval \\
    \hline
    Possible Values: & Any valid time stamp, 'dt', or 'output\_interval' \\
    \hline
    \caption{config\_AM\_harmonicAnalysis\_compute\_interval: Timestamp determining how often harmonic analysis computation should be performed.}
\end{longtable}
\end{center}
\subsection[config\_AM\_harmonicAnalysis\_start\_delay]{\hyperref[sec:nm_tab_AM_harmonicAnalysis]{config\_AM\_harmonicAnalysis\_start\_delay}}
\label{subsec:nm_sec_config_AM_harmonicAnalysis_start_delay}
\begin{center}
\begin{longtable}{| p{2.0in} || p{4.0in} |}
    \hline
    Type: & real \\
    \hline
    Units: & \si{days} \\
    \hline
    Default Value: & 20 \\
    \hline
    Possible Values: & any positive real number \\
    \hline
    \caption{config\_AM\_harmonicAnalysis\_start\_delay: Number of days after start of simulation when harmonic analysis begins. This is referenced relative to the start of the original simulation, not the restart date.}
\end{longtable}
\end{center}
\subsection[config\_AM\_harmonicAnalysis\_duration]{\hyperref[sec:nm_tab_AM_harmonicAnalysis]{config\_AM\_harmonicAnalysis\_duration}}
\label{subsec:nm_sec_config_AM_harmonicAnalysis_duration}
\begin{center}
\begin{longtable}{| p{2.0in} || p{4.0in} |}
    \hline
    Type: & real \\
    \hline
    Units: & \si{days} \\
    \hline
    Default Value: & 90 \\
    \hline
    Possible Values: & any positive real number \\
    \hline
    \caption{config\_AM\_harmonicAnalysis\_duration: Length of harmonic analysis period. The analysis begins after config\_AM\_harmonicAnalysis\_start\_delay days and ends after config\_AM\_harmonicAnalysis\_start\_delay + config\_AM\_harmonicAnalysis\_duration days relative to the start of the original simulation, not the restart date.}
\end{longtable}
\end{center}
\subsection[config\_AM\_harmonicAnalysis\_output\_stream]{\hyperref[sec:nm_tab_AM_harmonicAnalysis]{config\_AM\_harmonicAnalysis\_output\_stream}}
\label{subsec:nm_sec_config_AM_harmonicAnalysis_output_stream}
\begin{center}
\begin{longtable}{| p{2.0in} || p{4.0in} |}
    \hline
    Type: & character \\
    \hline
    Units: & -- \\
    \hline
    Default Value: & harmonicAnalysisOutput \\
    \hline
    Possible Values: & Any existing stream name or 'none' \\
    \hline
    \caption{config\_AM\_harmonicAnalysis\_output\_stream: Name of the stream that the harmonicAnalysis analysis member should be tied to.}
\end{longtable}
\end{center}
\subsection[config\_AM\_harmonicAnalysis\_restart\_stream]{\hyperref[sec:nm_tab_AM_harmonicAnalysis]{config\_AM\_harmonicAnalysis\_restart\_stream}}
\label{subsec:nm_sec_config_AM_harmonicAnalysis_restart_stream}
\begin{center}
\begin{longtable}{| p{2.0in} || p{4.0in} |}
    \hline
    Type: & character \\
    \hline
    Units: & -- \\
    \hline
    Default Value: & harmonicAnalysisRestart \\
    \hline
    Possible Values: & Any existing stream name or 'none' \\
    \hline
    \caption{config\_AM\_harmonicAnalysis\_restart\_stream: Name of the stream that the harmonicAnalysis analysis member restart informaion should be tied to.}
\end{longtable}
\end{center}
\subsection[config\_AM\_harmonicAnalysis\_compute\_on\_startup]{\hyperref[sec:nm_tab_AM_harmonicAnalysis]{config\_AM\_harmonicAnalysis\_compute\_on\_startup}}
\label{subsec:nm_sec_config_AM_harmonicAnalysis_compute_on_startup}
\begin{center}
\begin{longtable}{| p{2.0in} || p{4.0in} |}
    \hline
    Type: & logical \\
    \hline
    Units: & -- \\
    \hline
    Default Value: & .false. \\
    \hline
    Possible Values: & .true. or .false. \\
    \hline
    \caption{config\_AM\_harmonicAnalysis\_compute\_on\_startup: Logical flag determining if an analysis member computation occurs on start-up.}
\end{longtable}
\end{center}
\subsection[config\_AM\_harmonicAnalysis\_write\_on\_startup]{\hyperref[sec:nm_tab_AM_harmonicAnalysis]{config\_AM\_harmonicAnalysis\_write\_on\_startup}}
\label{subsec:nm_sec_config_AM_harmonicAnalysis_write_on_startup}
\begin{center}
\begin{longtable}{| p{2.0in} || p{4.0in} |}
    \hline
    Type: & logical \\
    \hline
    Units: & -- \\
    \hline
    Default Value: & .false. \\
    \hline
    Possible Values: & .true. or .false. \\
    \hline
    \caption{config\_AM\_harmonicAnalysis\_write\_on\_startup: Logical flag determining if an analysis member write occurs on start-up.}
\end{longtable}
\end{center}
\subsection[config\_AM\_harmonicAnalysis\_use\_M2]{\hyperref[sec:nm_tab_AM_harmonicAnalysis]{config\_AM\_harmonicAnalysis\_use\_M2}}
\label{subsec:nm_sec_config_AM_harmonicAnalysis_use_M2}
\begin{center}
\begin{longtable}{| p{2.0in} || p{4.0in} |}
    \hline
    Type: & logical \\
    \hline
    Units: & -- \\
    \hline
    Default Value: & .true. \\
    \hline
    Possible Values: & .true. or .false. \\
    \hline
    \caption{config\_AM\_harmonicAnalysis\_use\_M2: Controls if M2 constituent is used in harmonic analysis}
\end{longtable}
\end{center}
\subsection[config\_AM\_harmonicAnalysis\_use\_S2]{\hyperref[sec:nm_tab_AM_harmonicAnalysis]{config\_AM\_harmonicAnalysis\_use\_S2}}
\label{subsec:nm_sec_config_AM_harmonicAnalysis_use_S2}
\begin{center}
\begin{longtable}{| p{2.0in} || p{4.0in} |}
    \hline
    Type: & logical \\
    \hline
    Units: & -- \\
    \hline
    Default Value: & .true. \\
    \hline
    Possible Values: & .true. or .false. \\
    \hline
    \caption{config\_AM\_harmonicAnalysis\_use\_S2: Controls if S2 constituent is used in harmonic analysis}
\end{longtable}
\end{center}
\subsection[config\_AM\_harmonicAnalysis\_use\_N2]{\hyperref[sec:nm_tab_AM_harmonicAnalysis]{config\_AM\_harmonicAnalysis\_use\_N2}}
\label{subsec:nm_sec_config_AM_harmonicAnalysis_use_N2}
\begin{center}
\begin{longtable}{| p{2.0in} || p{4.0in} |}
    \hline
    Type: & logical \\
    \hline
    Units: & -- \\
    \hline
    Default Value: & .true. \\
    \hline
    Possible Values: & .true. or .false. \\
    \hline
    \caption{config\_AM\_harmonicAnalysis\_use\_N2: Controls if N2 constituent is used in harmonic analysis}
\end{longtable}
\end{center}
\subsection[config\_AM\_harmonicAnalysis\_use\_K2]{\hyperref[sec:nm_tab_AM_harmonicAnalysis]{config\_AM\_harmonicAnalysis\_use\_K2}}
\label{subsec:nm_sec_config_AM_harmonicAnalysis_use_K2}
\begin{center}
\begin{longtable}{| p{2.0in} || p{4.0in} |}
    \hline
    Type: & logical \\
    \hline
    Units: & -- \\
    \hline
    Default Value: & .true. \\
    \hline
    Possible Values: & .true. or .false. \\
    \hline
    \caption{config\_AM\_harmonicAnalysis\_use\_K2: Controls if K2 constituent is used in harmonic analysis}
\end{longtable}
\end{center}
\subsection[config\_AM\_harmonicAnalysis\_use\_K1]{\hyperref[sec:nm_tab_AM_harmonicAnalysis]{config\_AM\_harmonicAnalysis\_use\_K1}}
\label{subsec:nm_sec_config_AM_harmonicAnalysis_use_K1}
\begin{center}
\begin{longtable}{| p{2.0in} || p{4.0in} |}
    \hline
    Type: & logical \\
    \hline
    Units: & -- \\
    \hline
    Default Value: & .true. \\
    \hline
    Possible Values: & .true. or .false. \\
    \hline
    \caption{config\_AM\_harmonicAnalysis\_use\_K1: Controls if K1 constituent is used in harmonic analysis}
\end{longtable}
\end{center}
\subsection[config\_AM\_harmonicAnalysis\_use\_O1]{\hyperref[sec:nm_tab_AM_harmonicAnalysis]{config\_AM\_harmonicAnalysis\_use\_O1}}
\label{subsec:nm_sec_config_AM_harmonicAnalysis_use_O1}
\begin{center}
\begin{longtable}{| p{2.0in} || p{4.0in} |}
    \hline
    Type: & logical \\
    \hline
    Units: & -- \\
    \hline
    Default Value: & .true. \\
    \hline
    Possible Values: & .true. or .false. \\
    \hline
    \caption{config\_AM\_harmonicAnalysis\_use\_O1: Controls if O1 constituent is used in harmonic analysis}
\end{longtable}
\end{center}
\subsection[config\_AM\_harmonicAnalysis\_use\_Q1]{\hyperref[sec:nm_tab_AM_harmonicAnalysis]{config\_AM\_harmonicAnalysis\_use\_Q1}}
\label{subsec:nm_sec_config_AM_harmonicAnalysis_use_Q1}
\begin{center}
\begin{longtable}{| p{2.0in} || p{4.0in} |}
    \hline
    Type: & logical \\
    \hline
    Units: & -- \\
    \hline
    Default Value: & .true. \\
    \hline
    Possible Values: & .true. or .false. \\
    \hline
    \caption{config\_AM\_harmonicAnalysis\_use\_Q1: Controls if Q1 constituent is used in harmonic analysis}
\end{longtable}
\end{center}
\subsection[config\_AM\_harmonicAnalysis\_use\_P1]{\hyperref[sec:nm_tab_AM_harmonicAnalysis]{config\_AM\_harmonicAnalysis\_use\_P1}}
\label{subsec:nm_sec_config_AM_harmonicAnalysis_use_P1}
\begin{center}
\begin{longtable}{| p{2.0in} || p{4.0in} |}
    \hline
    Type: & logical \\
    \hline
    Units: & -- \\
    \hline
    Default Value: & .true. \\
    \hline
    Possible Values: & .true. or .false. \\
    \hline
    \caption{config\_AM\_harmonicAnalysis\_use\_P1: Controls if P1 constituent is used in harmonic analysis}
\end{longtable}
\end{center}
\section[AM\_conservationCheck]{\hyperref[sec:nm_tab_AM_conservationCheck]{AM\_conservationCheck}}
\label{sec:nm_sec_AM_conservationCheck}
\subsection[config\_AM\_conservationCheck\_enable]{\hyperref[sec:nm_tab_AM_conservationCheck]{config\_AM\_conservationCheck\_enable}}
\label{subsec:nm_sec_config_AM_conservationCheck_enable}
\begin{center}
\begin{longtable}{| p{2.0in} || p{4.0in} |}
    \hline
    Type: & logical \\
    \hline
    Units: & -- \\
    \hline
    Default Value: & .false. \\
    \hline
    Possible Values: & true or false \\
    \hline
    \caption{config\_AM\_conservationCheck\_enable: If true, ocean analysis member conservationCheck is called.}
\end{longtable}
\end{center}
\subsection[config\_AM\_conservationCheck\_compute\_interval]{\hyperref[sec:nm_tab_AM_conservationCheck]{config\_AM\_conservationCheck\_compute\_interval}}
\label{subsec:nm_sec_config_AM_conservationCheck_compute_interval}
\begin{center}
\begin{longtable}{| p{2.0in} || p{4.0in} |}
    \hline
    Type: & character \\
    \hline
    Units: & -- \\
    \hline
    Default Value: & dt \\
    \hline
    Possible Values: & Any valid time stamp, 'dt', or 'output\_interval' \\
    \hline
    \caption{config\_AM\_conservationCheck\_compute\_interval: Timestamp determining how often analysis member computation should be performed.}
\end{longtable}
\end{center}
\subsection[config\_AM\_conservationCheck\_output\_stream]{\hyperref[sec:nm_tab_AM_conservationCheck]{config\_AM\_conservationCheck\_output\_stream}}
\label{subsec:nm_sec_config_AM_conservationCheck_output_stream}
\begin{center}
\begin{longtable}{| p{2.0in} || p{4.0in} |}
    \hline
    Type: & character \\
    \hline
    Units: & -- \\
    \hline
    Default Value: & conservationCheckOutput \\
    \hline
    Possible Values: & Any existing stream name or 'none' \\
    \hline
    \caption{config\_AM\_conservationCheck\_output\_stream: Name of the stream that the conservationCheck analysis member should be tied to.}
\end{longtable}
\end{center}
\subsection[config\_AM\_conservationCheck\_compute\_on\_startup]{\hyperref[sec:nm_tab_AM_conservationCheck]{config\_AM\_conservationCheck\_compute\_on\_startup}}
\label{subsec:nm_sec_config_AM_conservationCheck_compute_on_startup}
\begin{center}
\begin{longtable}{| p{2.0in} || p{4.0in} |}
    \hline
    Type: & logical \\
    \hline
    Units: & -- \\
    \hline
    Default Value: & .false. \\
    \hline
    Possible Values: & true or false \\
    \hline
    \caption{config\_AM\_conservationCheck\_compute\_on\_startup: Logical flag determining if an analysis member computation occurs on start-up.}
\end{longtable}
\end{center}
\subsection[config\_AM\_conservationCheck\_write\_on\_startup]{\hyperref[sec:nm_tab_AM_conservationCheck]{config\_AM\_conservationCheck\_write\_on\_startup}}
\label{subsec:nm_sec_config_AM_conservationCheck_write_on_startup}
\begin{center}
\begin{longtable}{| p{2.0in} || p{4.0in} |}
    \hline
    Type: & logical \\
    \hline
    Units: & -- \\
    \hline
    Default Value: & .false. \\
    \hline
    Possible Values: & true or false \\
    \hline
    \caption{config\_AM\_conservationCheck\_write\_on\_startup: Logical flag determining if an analysis member write occurs on start-up.}
\end{longtable}
\end{center}
\subsection[config\_AM\_conservationCheck\_write\_to\_logfile]{\hyperref[sec:nm_tab_AM_conservationCheck]{config\_AM\_conservationCheck\_write\_to\_logfile}}
\label{subsec:nm_sec_config_AM_conservationCheck_write_to_logfile}
\begin{center}
\begin{longtable}{| p{2.0in} || p{4.0in} |}
    \hline
    Type: & logical \\
    \hline
    Units: & -- \\
    \hline
    Default Value: & .true. \\
    \hline
    Possible Values: & true or false \\
    \hline
    \caption{config\_AM\_conservationCheck\_write\_to\_logfile: Logical flag determining if the conservation check is written to the log file.}
\end{longtable}
\end{center}
\subsection[config\_AM\_conservationCheck\_restart\_stream]{\hyperref[sec:nm_tab_AM_conservationCheck]{config\_AM\_conservationCheck\_restart\_stream}}
\label{subsec:nm_sec_config_AM_conservationCheck_restart_stream}
\begin{center}
\begin{longtable}{| p{2.0in} || p{4.0in} |}
    \hline
    Type: & character \\
    \hline
    Units: & -- \\
    \hline
    Default Value: & conservationCheckRestart \\
    \hline
    Possible Values: & A restart stream with state of the conservation check. \\
    \hline
    \caption{config\_AM\_conservationCheck\_restart\_stream: Name of the restart stream the analysis member will use to initialize itself if restart is enabled.}
\end{longtable}
\end{center}
\section[baroclinic\_channel]{\hyperref[sec:nm_tab_baroclinic_channel]{baroclinic\_channel}}
\label{sec:nm_sec_baroclinic_channel}
\subsection[config\_baroclinic\_channel\_vert\_levels]{\hyperref[sec:nm_tab_baroclinic_channel]{config\_baroclinic\_channel\_vert\_levels}}
\label{subsec:nm_sec_config_baroclinic_channel_vert_levels}
\begin{center}
\begin{longtable}{| p{2.0in} || p{4.0in} |}
    \hline
    Type: & integer \\
    \hline
    Units: & \si{unitless} \\
    \hline
    Default Value: & 20 \\
    \hline
    Possible Values: & Any positive integer number greater than 0. \\
    \hline
    \caption{config\_baroclinic\_channel\_vert\_levels: Number of vertical levels in baroclinic channel test case. Typical value is 20.}
\end{longtable}
\end{center}
\subsection[config\_baroclinic\_channel\_use\_distances]{\hyperref[sec:nm_tab_baroclinic_channel]{config\_baroclinic\_channel\_use\_distances}}
\label{subsec:nm_sec_config_baroclinic_channel_use_distances}
\begin{center}
\begin{longtable}{| p{2.0in} || p{4.0in} |}
    \hline
    Type: & logical \\
    \hline
    Units: & \si{unitless} \\
    \hline
    Default Value: & .false. \\
    \hline
    Possible Values: & .true. or .false. \\
    \hline
    \caption{config\_baroclinic\_channel\_use\_distances: Logical flag that determines if locations of features are defined by distances of fractions. False means fractions.}
\end{longtable}
\end{center}
\subsection[config\_baroclinic\_channel\_surface\_temperature]{\hyperref[sec:nm_tab_baroclinic_channel]{config\_baroclinic\_channel\_surface\_temperature}}
\label{subsec:nm_sec_config_baroclinic_channel_surface_temperature}
\begin{center}
\begin{longtable}{| p{2.0in} || p{4.0in} |}
    \hline
    Type: & real \\
    \hline
    Units: & \si{deg.C} \\
    \hline
    Default Value: & 13.1 \\
    \hline
    Possible Values: & Any real number \\
    \hline
    \caption{config\_baroclinic\_channel\_surface\_temperature: Temperature of the surface in the northern half of the domain.}
\end{longtable}
\end{center}
\subsection[config\_baroclinic\_channel\_bottom\_temperature]{\hyperref[sec:nm_tab_baroclinic_channel]{config\_baroclinic\_channel\_bottom\_temperature}}
\label{subsec:nm_sec_config_baroclinic_channel_bottom_temperature}
\begin{center}
\begin{longtable}{| p{2.0in} || p{4.0in} |}
    \hline
    Type: & real \\
    \hline
    Units: & \si{deg.C} \\
    \hline
    Default Value: & 10.1 \\
    \hline
    Possible Values: & Any real number \\
    \hline
    \caption{config\_baroclinic\_channel\_bottom\_temperature: Temperature of the bottom in the northern half of the domain.}
\end{longtable}
\end{center}
\subsection[config\_baroclinic\_channel\_temperature\_difference]{\hyperref[sec:nm_tab_baroclinic_channel]{config\_baroclinic\_channel\_temperature\_difference}}
\label{subsec:nm_sec_config_baroclinic_channel_temperature_difference}
\begin{center}
\begin{longtable}{| p{2.0in} || p{4.0in} |}
    \hline
    Type: & real \\
    \hline
    Units: & \si{deg.C} \\
    \hline
    Default Value: & 1.2 \\
    \hline
    Possible Values: & Any real number \\
    \hline
    \caption{config\_baroclinic\_channel\_temperature\_difference: Difference in the temperature field between the northern and southern halves of the domain.}
\end{longtable}
\end{center}
\subsection[config\_baroclinic\_channel\_gradient\_width\_frac]{\hyperref[sec:nm_tab_baroclinic_channel]{config\_baroclinic\_channel\_gradient\_width\_frac}}
\label{subsec:nm_sec_config_baroclinic_channel_gradient_width_frac}
\begin{center}
\begin{longtable}{| p{2.0in} || p{4.0in} |}
    \hline
    Type: & real \\
    \hline
    Units: & \si{fraction} \\
    \hline
    Default Value: & 0.08 \\
    \hline
    Possible Values: & Any real number between 0 and 1. \\
    \hline
    \caption{config\_baroclinic\_channel\_gradient\_width\_frac: Fraction of domain in Y direction the temperature gradient should be linear over.}
\end{longtable}
\end{center}
\subsection[config\_baroclinic\_channel\_gradient\_width\_dist]{\hyperref[sec:nm_tab_baroclinic_channel]{config\_baroclinic\_channel\_gradient\_width\_dist}}
\label{subsec:nm_sec_config_baroclinic_channel_gradient_width_dist}
\begin{center}
\begin{longtable}{| p{2.0in} || p{4.0in} |}
    \hline
    Type: & real \\
    \hline
    Units: & \si{m} \\
    \hline
    Default Value: & 40e3 \\
    \hline
    Possible Values: & Any positive real number. \\
    \hline
    \caption{config\_baroclinic\_channel\_gradient\_width\_dist: Width of the temperature gradient around the center sin wave. Default value is relative to a 500km domain in Y.}
\end{longtable}
\end{center}
\subsection[config\_baroclinic\_channel\_bottom\_depth]{\hyperref[sec:nm_tab_baroclinic_channel]{config\_baroclinic\_channel\_bottom\_depth}}
\label{subsec:nm_sec_config_baroclinic_channel_bottom_depth}
\begin{center}
\begin{longtable}{| p{2.0in} || p{4.0in} |}
    \hline
    Type: & real \\
    \hline
    Units: & \si{m} \\
    \hline
    Default Value: & 1000.0 \\
    \hline
    Possible Values: & Any positive real number. \\
    \hline
    \caption{config\_baroclinic\_channel\_bottom\_depth: Depth of the bottom of the ocean for the baroclinic channel test case.}
\end{longtable}
\end{center}
\subsection[config\_baroclinic\_channel\_salinity]{\hyperref[sec:nm_tab_baroclinic_channel]{config\_baroclinic\_channel\_salinity}}
\label{subsec:nm_sec_config_baroclinic_channel_salinity}
\begin{center}
\begin{longtable}{| p{2.0in} || p{4.0in} |}
    \hline
    Type: & real \\
    \hline
    Units: & \si{PSU} \\
    \hline
    Default Value: & 35.0 \\
    \hline
    Possible Values: & Any real number greater than 0. \\
    \hline
    \caption{config\_baroclinic\_channel\_salinity: Salinity of the water in the entire domain.}
\end{longtable}
\end{center}
\subsection[config\_baroclinic\_channel\_coriolis\_parameter]{\hyperref[sec:nm_tab_baroclinic_channel]{config\_baroclinic\_channel\_coriolis\_parameter}}
\label{subsec:nm_sec_config_baroclinic_channel_coriolis_parameter}
\begin{center}
\begin{longtable}{| p{2.0in} || p{4.0in} |}
    \hline
    Type: & real \\
    \hline
    Units: & \si{s^{-1}} \\
    \hline
    Default Value: & -1.2e-4 \\
    \hline
    Possible Values: & Any real number. \\
    \hline
    \caption{config\_baroclinic\_channel\_coriolis\_parameter: Coriolis parameter for entrie domain.}
\end{longtable}
\end{center}
\section[lock\_exchange]{\hyperref[sec:nm_tab_lock_exchange]{lock\_exchange}}
\label{sec:nm_sec_lock_exchange}
\subsection[config\_lock\_exchange\_vert\_levels]{\hyperref[sec:nm_tab_lock_exchange]{config\_lock\_exchange\_vert\_levels}}
\label{subsec:nm_sec_config_lock_exchange_vert_levels}
\begin{center}
\begin{longtable}{| p{2.0in} || p{4.0in} |}
    \hline
    Type: & integer \\
    \hline
    Units: & \si{unitless} \\
    \hline
    Default Value: & 20 \\
    \hline
    Possible Values: & Any positive integer number greater than 0. \\
    \hline
    \caption{config\_lock\_exchange\_vert\_levels: Number of vertical levels in lock exchange test case. Typical value is 20.}
\end{longtable}
\end{center}
\subsection[config\_lock\_exchange\_bottom\_depth]{\hyperref[sec:nm_tab_lock_exchange]{config\_lock\_exchange\_bottom\_depth}}
\label{subsec:nm_sec_config_lock_exchange_bottom_depth}
\begin{center}
\begin{longtable}{| p{2.0in} || p{4.0in} |}
    \hline
    Type: & real \\
    \hline
    Units: & \si{m} \\
    \hline
    Default Value: & 20.0 \\
    \hline
    Possible Values: & Any positive real value greater than 0. \\
    \hline
    \caption{config\_lock\_exchange\_bottom\_depth: Depth of the bottom of the ocean in the lock exchange test case.}
\end{longtable}
\end{center}
\subsection[config\_lock\_exchange\_cold\_temperature]{\hyperref[sec:nm_tab_lock_exchange]{config\_lock\_exchange\_cold\_temperature}}
\label{subsec:nm_sec_config_lock_exchange_cold_temperature}
\begin{center}
\begin{longtable}{| p{2.0in} || p{4.0in} |}
    \hline
    Type: & real \\
    \hline
    Units: & \si{deg.C} \\
    \hline
    Default Value: & 5.0 \\
    \hline
    Possible Values: & Any real number \\
    \hline
    \caption{config\_lock\_exchange\_cold\_temperature: Temperature of water in the cold half of the domain.}
\end{longtable}
\end{center}
\subsection[config\_lock\_exchange\_warm\_temperature]{\hyperref[sec:nm_tab_lock_exchange]{config\_lock\_exchange\_warm\_temperature}}
\label{subsec:nm_sec_config_lock_exchange_warm_temperature}
\begin{center}
\begin{longtable}{| p{2.0in} || p{4.0in} |}
    \hline
    Type: & real \\
    \hline
    Units: & \si{deg.C} \\
    \hline
    Default Value: & 30.0 \\
    \hline
    Possible Values: & Any real number \\
    \hline
    \caption{config\_lock\_exchange\_warm\_temperature: Temperature of water in the warm half of the domain.}
\end{longtable}
\end{center}
\subsection[config\_lock\_exchange\_direction]{\hyperref[sec:nm_tab_lock_exchange]{config\_lock\_exchange\_direction}}
\label{subsec:nm_sec_config_lock_exchange_direction}
\begin{center}
\begin{longtable}{| p{2.0in} || p{4.0in} |}
    \hline
    Type: & character \\
    \hline
    Units: & \si{unitless} \\
    \hline
    Default Value: & y \\
    \hline
    Possible Values: & 'x', 'y','z'' \\
    \hline
    \caption{config\_lock\_exchange\_direction: If y, warm/cold changes in the y-direction.  If z, warm/cold changes in z-direction.}
\end{longtable}
\end{center}
\subsection[config\_lock\_exchange\_salinity]{\hyperref[sec:nm_tab_lock_exchange]{config\_lock\_exchange\_salinity}}
\label{subsec:nm_sec_config_lock_exchange_salinity}
\begin{center}
\begin{longtable}{| p{2.0in} || p{4.0in} |}
    \hline
    Type: & real \\
    \hline
    Units: & \si{PSU} \\
    \hline
    Default Value: & 35.0 \\
    \hline
    Possible Values: & Any real number greater than 0. \\
    \hline
    \caption{config\_lock\_exchange\_salinity: Salinity of the water in the entire domain.}
\end{longtable}
\end{center}
\subsection[config\_lock\_exchange\_layer\_type]{\hyperref[sec:nm_tab_lock_exchange]{config\_lock\_exchange\_layer\_type}}
\label{subsec:nm_sec_config_lock_exchange_layer_type}
\begin{center}
\begin{longtable}{| p{2.0in} || p{4.0in} |}
    \hline
    Type: & character \\
    \hline
    Units: & \si{unitless} \\
    \hline
    Default Value: & z-level \\
    \hline
    Possible Values: & 'z-level', 'isopycnal' \\
    \hline
    \caption{config\_lock\_exchange\_layer\_type: Vertical grid type}
\end{longtable}
\end{center}
\subsection[config\_lock\_exchange\_isopycnal\_min\_thickness]{\hyperref[sec:nm_tab_lock_exchange]{config\_lock\_exchange\_isopycnal\_min\_thickness}}
\label{subsec:nm_sec_config_lock_exchange_isopycnal_min_thickness}
\begin{center}
\begin{longtable}{| p{2.0in} || p{4.0in} |}
    \hline
    Type: & real \\
    \hline
    Units: & \si{m} \\
    \hline
    Default Value: & 0.01 \\
    \hline
    Possible Values: & Any positive real number, typically 0.01 to 1.0 \\
    \hline
    \caption{config\_lock\_exchange\_isopycnal\_min\_thickness: minimum layer thickness for isopycnal case}
\end{longtable}
\end{center}
\section[internal\_waves]{\hyperref[sec:nm_tab_internal_waves]{internal\_waves}}
\label{sec:nm_sec_internal_waves}
\subsection[config\_internal\_waves\_vert\_levels]{\hyperref[sec:nm_tab_internal_waves]{config\_internal\_waves\_vert\_levels}}
\label{subsec:nm_sec_config_internal_waves_vert_levels}
\begin{center}
\begin{longtable}{| p{2.0in} || p{4.0in} |}
    \hline
    Type: & integer \\
    \hline
    Units: & \si{unitless} \\
    \hline
    Default Value: & 20 \\
    \hline
    Possible Values: & Any positive integer number greater than 0. \\
    \hline
    \caption{config\_internal\_waves\_vert\_levels: Number of vertical levels in internal waves test case. Typical value is 20.}
\end{longtable}
\end{center}
\subsection[config\_internal\_waves\_use\_distances]{\hyperref[sec:nm_tab_internal_waves]{config\_internal\_waves\_use\_distances}}
\label{subsec:nm_sec_config_internal_waves_use_distances}
\begin{center}
\begin{longtable}{| p{2.0in} || p{4.0in} |}
    \hline
    Type: & logical \\
    \hline
    Units: & \si{unitless} \\
    \hline
    Default Value: & false \\
    \hline
    Possible Values: & .true. or .false. \\
    \hline
    \caption{config\_internal\_waves\_use\_distances: Logical flag that determines if locations of features are defined by distances of fractions. False means fractions.}
\end{longtable}
\end{center}
\subsection[config\_internal\_waves\_surface\_temperature]{\hyperref[sec:nm_tab_internal_waves]{config\_internal\_waves\_surface\_temperature}}
\label{subsec:nm_sec_config_internal_waves_surface_temperature}
\begin{center}
\begin{longtable}{| p{2.0in} || p{4.0in} |}
    \hline
    Type: & real \\
    \hline
    Units: & \si{deg.C} \\
    \hline
    Default Value: & 20.1 \\
    \hline
    Possible Values: & Any real number \\
    \hline
    \caption{config\_internal\_waves\_surface\_temperature: Temperature of the surface in the northern half of the domain.}
\end{longtable}
\end{center}
\subsection[config\_internal\_waves\_bottom\_temperature]{\hyperref[sec:nm_tab_internal_waves]{config\_internal\_waves\_bottom\_temperature}}
\label{subsec:nm_sec_config_internal_waves_bottom_temperature}
\begin{center}
\begin{longtable}{| p{2.0in} || p{4.0in} |}
    \hline
    Type: & real \\
    \hline
    Units: & \si{deg.C} \\
    \hline
    Default Value: & 10.1 \\
    \hline
    Possible Values: & Any real number \\
    \hline
    \caption{config\_internal\_waves\_bottom\_temperature: Temperature of the bottom in the northern half of the domain.}
\end{longtable}
\end{center}
\subsection[config\_internal\_waves\_temperature\_difference]{\hyperref[sec:nm_tab_internal_waves]{config\_internal\_waves\_temperature\_difference}}
\label{subsec:nm_sec_config_internal_waves_temperature_difference}
\begin{center}
\begin{longtable}{| p{2.0in} || p{4.0in} |}
    \hline
    Type: & real \\
    \hline
    Units: & \si{deg.C} \\
    \hline
    Default Value: & 2.0 \\
    \hline
    Possible Values: & Any real number \\
    \hline
    \caption{config\_internal\_waves\_temperature\_difference: Maximum temperature difference in the amplitude.}
\end{longtable}
\end{center}
\subsection[config\_internal\_waves\_amplitude\_width\_frac]{\hyperref[sec:nm_tab_internal_waves]{config\_internal\_waves\_amplitude\_width\_frac}}
\label{subsec:nm_sec_config_internal_waves_amplitude_width_frac}
\begin{center}
\begin{longtable}{| p{2.0in} || p{4.0in} |}
    \hline
    Type: & real \\
    \hline
    Units: & \si{fraction} \\
    \hline
    Default Value: & 0.33 \\
    \hline
    Possible Values: & Any real number between 0 and 1. \\
    \hline
    \caption{config\_internal\_waves\_amplitude\_width\_frac: Percent of domain in Y direction the initial amplitude should exist over.}
\end{longtable}
\end{center}
\subsection[config\_internal\_waves\_amplitude\_width\_dist]{\hyperref[sec:nm_tab_internal_waves]{config\_internal\_waves\_amplitude\_width\_dist}}
\label{subsec:nm_sec_config_internal_waves_amplitude_width_dist}
\begin{center}
\begin{longtable}{| p{2.0in} || p{4.0in} |}
    \hline
    Type: & real \\
    \hline
    Units: & \si{m} \\
    \hline
    Default Value: & 50e3 \\
    \hline
    Possible Values: & Any positive real number. \\
    \hline
    \caption{config\_internal\_waves\_amplitude\_width\_dist: Width in Y direction the initial amplitude should exist over. Default is relative to a 250km domain.}
\end{longtable}
\end{center}
\subsection[config\_internal\_waves\_bottom\_depth]{\hyperref[sec:nm_tab_internal_waves]{config\_internal\_waves\_bottom\_depth}}
\label{subsec:nm_sec_config_internal_waves_bottom_depth}
\begin{center}
\begin{longtable}{| p{2.0in} || p{4.0in} |}
    \hline
    Type: & real \\
    \hline
    Units: & \si{m} \\
    \hline
    Default Value: & 500.0 \\
    \hline
    Possible Values: & Any positive real number. \\
    \hline
    \caption{config\_internal\_waves\_bottom\_depth: Depth of the bottom of the ocean for the internal waves test case.}
\end{longtable}
\end{center}
\subsection[config\_internal\_waves\_salinity]{\hyperref[sec:nm_tab_internal_waves]{config\_internal\_waves\_salinity}}
\label{subsec:nm_sec_config_internal_waves_salinity}
\begin{center}
\begin{longtable}{| p{2.0in} || p{4.0in} |}
    \hline
    Type: & real \\
    \hline
    Units: & \si{PSU} \\
    \hline
    Default Value: & 35.0 \\
    \hline
    Possible Values: & Any real number greater than 0. \\
    \hline
    \caption{config\_internal\_waves\_salinity: Salinity of the water in the entire domain.}
\end{longtable}
\end{center}
\subsection[config\_internal\_waves\_layer\_type]{\hyperref[sec:nm_tab_internal_waves]{config\_internal\_waves\_layer\_type}}
\label{subsec:nm_sec_config_internal_waves_layer_type}
\begin{center}
\begin{longtable}{| p{2.0in} || p{4.0in} |}
    \hline
    Type: & character \\
    \hline
    Units: & \si{unitless} \\
    \hline
    Default Value: & z-level \\
    \hline
    Possible Values: & 'z-level', 'isopycnal' \\
    \hline
    \caption{config\_internal\_waves\_layer\_type: Logical flag that controls how the initial conditions should be generated.}
\end{longtable}
\end{center}
\subsection[config\_internal\_waves\_isopycnal\_displacement]{\hyperref[sec:nm_tab_internal_waves]{config\_internal\_waves\_isopycnal\_displacement}}
\label{subsec:nm_sec_config_internal_waves_isopycnal_displacement}
\begin{center}
\begin{longtable}{| p{2.0in} || p{4.0in} |}
    \hline
    Type: & real \\
    \hline
    Units: & \si{m} \\
    \hline
    Default Value: & 125.0 \\
    \hline
    Possible Values: & Any positive real number, typically 10 to 100. \\
    \hline
    \caption{config\_internal\_waves\_isopycnal\_displacement: Max distance isopycnal layers are displaced upwards.}
\end{longtable}
\end{center}
\section[overflow]{\hyperref[sec:nm_tab_overflow]{overflow}}
\label{sec:nm_sec_overflow}
\subsection[config\_overflow\_vert\_levels]{\hyperref[sec:nm_tab_overflow]{config\_overflow\_vert\_levels}}
\label{subsec:nm_sec_config_overflow_vert_levels}
\begin{center}
\begin{longtable}{| p{2.0in} || p{4.0in} |}
    \hline
    Type: & integer \\
    \hline
    Units: & \si{unitless} \\
    \hline
    Default Value: & 100 \\
    \hline
    Possible Values: & Any positive integer number greater than 0. \\
    \hline
    \caption{config\_overflow\_vert\_levels: Number of vertical levels in overflow test case. Typical values are 40 and 100.}
\end{longtable}
\end{center}
\subsection[config\_overflow\_use\_distances]{\hyperref[sec:nm_tab_overflow]{config\_overflow\_use\_distances}}
\label{subsec:nm_sec_config_overflow_use_distances}
\begin{center}
\begin{longtable}{| p{2.0in} || p{4.0in} |}
    \hline
    Type: & logical \\
    \hline
    Units: & \si{unitless} \\
    \hline
    Default Value: & false \\
    \hline
    Possible Values: & .true. or .false. \\
    \hline
    \caption{config\_overflow\_use\_distances: Logical flag that determines if locations of features are defined by distances of fractions. False means fractions.}
\end{longtable}
\end{center}
\subsection[config\_overflow\_bottom\_depth]{\hyperref[sec:nm_tab_overflow]{config\_overflow\_bottom\_depth}}
\label{subsec:nm_sec_config_overflow_bottom_depth}
\begin{center}
\begin{longtable}{| p{2.0in} || p{4.0in} |}
    \hline
    Type: & real \\
    \hline
    Units: & \si{m} \\
    \hline
    Default Value: & 2000.0 \\
    \hline
    Possible Values: & Any positive real value greater than 0. \\
    \hline
    \caption{config\_overflow\_bottom\_depth: Depth of the bottom of the ocean in the overflow test case.}
\end{longtable}
\end{center}
\subsection[config\_overflow\_ridge\_depth]{\hyperref[sec:nm_tab_overflow]{config\_overflow\_ridge\_depth}}
\label{subsec:nm_sec_config_overflow_ridge_depth}
\begin{center}
\begin{longtable}{| p{2.0in} || p{4.0in} |}
    \hline
    Type: & real \\
    \hline
    Units: & \si{m} \\
    \hline
    Default Value: & 500.0 \\
    \hline
    Possible Values: & Any positive real value greater than 0. \\
    \hline
    \caption{config\_overflow\_ridge\_depth: Depth of the bottom of the ocean on the ridge in the over flow test case.}
\end{longtable}
\end{center}
\subsection[config\_overflow\_plug\_temperature]{\hyperref[sec:nm_tab_overflow]{config\_overflow\_plug\_temperature}}
\label{subsec:nm_sec_config_overflow_plug_temperature}
\begin{center}
\begin{longtable}{| p{2.0in} || p{4.0in} |}
    \hline
    Type: & real \\
    \hline
    Units: & \si{deg.C} \\
    \hline
    Default Value: & 10.0 \\
    \hline
    Possible Values: & Any real number \\
    \hline
    \caption{config\_overflow\_plug\_temperature: Temperature of water in plug at the southern end of the domain.}
\end{longtable}
\end{center}
\subsection[config\_overflow\_domain\_temperature]{\hyperref[sec:nm_tab_overflow]{config\_overflow\_domain\_temperature}}
\label{subsec:nm_sec_config_overflow_domain_temperature}
\begin{center}
\begin{longtable}{| p{2.0in} || p{4.0in} |}
    \hline
    Type: & real \\
    \hline
    Units: & \si{deg.C} \\
    \hline
    Default Value: & 20.0 \\
    \hline
    Possible Values: & Any real number \\
    \hline
    \caption{config\_overflow\_domain\_temperature: Temperature of water outside of the plug.}
\end{longtable}
\end{center}
\subsection[config\_overflow\_salinity]{\hyperref[sec:nm_tab_overflow]{config\_overflow\_salinity}}
\label{subsec:nm_sec_config_overflow_salinity}
\begin{center}
\begin{longtable}{| p{2.0in} || p{4.0in} |}
    \hline
    Type: & real \\
    \hline
    Units: & \si{PSU} \\
    \hline
    Default Value: & 35.0 \\
    \hline
    Possible Values: & Any real number greater than 0. \\
    \hline
    \caption{config\_overflow\_salinity: Salinity of the water in the entire domain.}
\end{longtable}
\end{center}
\subsection[config\_overflow\_plug\_width\_frac]{\hyperref[sec:nm_tab_overflow]{config\_overflow\_plug\_width\_frac}}
\label{subsec:nm_sec_config_overflow_plug_width_frac}
\begin{center}
\begin{longtable}{| p{2.0in} || p{4.0in} |}
    \hline
    Type: & real \\
    \hline
    Units: & \si{fraction} \\
    \hline
    Default Value: & 0.10 \\
    \hline
    Possible Values: & Any real number between 0 and 1. \\
    \hline
    \caption{config\_overflow\_plug\_width\_frac: Fraction of the domain the plug should take up initially. Only in the y direction.}
\end{longtable}
\end{center}
\subsection[config\_overflow\_slope\_center\_frac]{\hyperref[sec:nm_tab_overflow]{config\_overflow\_slope\_center\_frac}}
\label{subsec:nm_sec_config_overflow_slope_center_frac}
\begin{center}
\begin{longtable}{| p{2.0in} || p{4.0in} |}
    \hline
    Type: & real \\
    \hline
    Units: & \si{fraction} \\
    \hline
    Default Value: & 0.20 \\
    \hline
    Possible Values: & Any real number between 0 and 1. \\
    \hline
    \caption{config\_overflow\_slope\_center\_frac: Location of the center of the slope. Given as a fraction of the total y domain range. Position is relative to the minimum y value.}
\end{longtable}
\end{center}
\subsection[config\_overflow\_slope\_width\_frac]{\hyperref[sec:nm_tab_overflow]{config\_overflow\_slope\_width\_frac}}
\label{subsec:nm_sec_config_overflow_slope_width_frac}
\begin{center}
\begin{longtable}{| p{2.0in} || p{4.0in} |}
    \hline
    Type: & real \\
    \hline
    Units: & \si{fraction} \\
    \hline
    Default Value: & 0.05 \\
    \hline
    Possible Values: & Any real number between 0 and 1. \\
    \hline
    \caption{config\_overflow\_slope\_width\_frac: Half width of the slope. Given as a fraction of the total y domain range.}
\end{longtable}
\end{center}
\subsection[config\_overflow\_plug\_width\_dist]{\hyperref[sec:nm_tab_overflow]{config\_overflow\_plug\_width\_dist}}
\label{subsec:nm_sec_config_overflow_plug_width_dist}
\begin{center}
\begin{longtable}{| p{2.0in} || p{4.0in} |}
    \hline
    Type: & real \\
    \hline
    Units: & \si{m} \\
    \hline
    Default Value: & 20e3 \\
    \hline
    Possible Values: & Any positive real number. \\
    \hline
    \caption{config\_overflow\_plug\_width\_dist: Distance from the minimum Y value of the domain the plug should take up initially. Default is relative to a 200km domain.}
\end{longtable}
\end{center}
\subsection[config\_overflow\_slope\_center\_dist]{\hyperref[sec:nm_tab_overflow]{config\_overflow\_slope\_center\_dist}}
\label{subsec:nm_sec_config_overflow_slope_center_dist}
\begin{center}
\begin{longtable}{| p{2.0in} || p{4.0in} |}
    \hline
    Type: & real \\
    \hline
    Units: & \si{m} \\
    \hline
    Default Value: & 40e3 \\
    \hline
    Possible Values: & Any positive real number. \\
    \hline
    \caption{config\_overflow\_slope\_center\_dist: Location of the center of the slope. Given as a distance from the minimum y value. Default is relative to a 200km domain.}
\end{longtable}
\end{center}
\subsection[config\_overflow\_slope\_width\_dist]{\hyperref[sec:nm_tab_overflow]{config\_overflow\_slope\_width\_dist}}
\label{subsec:nm_sec_config_overflow_slope_width_dist}
\begin{center}
\begin{longtable}{| p{2.0in} || p{4.0in} |}
    \hline
    Type: & real \\
    \hline
    Units: & \si{m} \\
    \hline
    Default Value: & 7e3 \\
    \hline
    Possible Values: & Any positive real number. \\
    \hline
    \caption{config\_overflow\_slope\_width\_dist: Half width of the slope. Default is relative to a 200km domain.}
\end{longtable}
\end{center}
\subsection[config\_overflow\_layer\_type]{\hyperref[sec:nm_tab_overflow]{config\_overflow\_layer\_type}}
\label{subsec:nm_sec_config_overflow_layer_type}
\begin{center}
\begin{longtable}{| p{2.0in} || p{4.0in} |}
    \hline
    Type: & character \\
    \hline
    Units: & \si{unitless} \\
    \hline
    Default Value: & z-level \\
    \hline
    Possible Values: & 'z-level', 'sigma', 'isopycnal' \\
    \hline
    \caption{config\_overflow\_layer\_type: Logical flag that controls how the initial conditions should be generated.}
\end{longtable}
\end{center}
\subsection[config\_overflow\_isopycnal\_min\_thickness]{\hyperref[sec:nm_tab_overflow]{config\_overflow\_isopycnal\_min\_thickness}}
\label{subsec:nm_sec_config_overflow_isopycnal_min_thickness}
\begin{center}
\begin{longtable}{| p{2.0in} || p{4.0in} |}
    \hline
    Type: & real \\
    \hline
    Units: & \si{m} \\
    \hline
    Default Value: & 0.01 \\
    \hline
    Possible Values: & Any positive real number, typically 0.01 to 1.0 \\
    \hline
    \caption{config\_overflow\_isopycnal\_min\_thickness: minimum layer thickness}
\end{longtable}
\end{center}
\section[dam\_break]{\hyperref[sec:nm_tab_dam_break]{dam\_break}}
\label{sec:nm_sec_dam_break}
\subsection[config\_dam\_break\_vert\_levels]{\hyperref[sec:nm_tab_dam_break]{config\_dam\_break\_vert\_levels}}
\label{subsec:nm_sec_config_dam_break_vert_levels}
\begin{center}
\begin{longtable}{| p{2.0in} || p{4.0in} |}
    \hline
    Type: & integer \\
    \hline
    Units: & \si{unitless} \\
    \hline
    Default Value: & 1 \\
    \hline
    Possible Values: & Any positive integer number greater than 0. \\
    \hline
    \caption{config\_dam\_break\_vert\_levels: Number of vertical levels in dam\_break case. Default value is 1.}
\end{longtable}
\end{center}
\subsection[config\_dam\_break\_eta0]{\hyperref[sec:nm_tab_dam_break]{config\_dam\_break\_eta0}}
\label{subsec:nm_sec_config_dam_break_eta0}
\begin{center}
\begin{longtable}{| p{2.0in} || p{4.0in} |}
    \hline
    Type: & real \\
    \hline
    Units: & \si{m} \\
    \hline
    Default Value: & 0.6 \\
    \hline
    Possible Values: & Any real number larger than zero. \\
    \hline
    \caption{config\_dam\_break\_eta0: Depth of the domain ($H$).}
\end{longtable}
\end{center}
\subsection[config\_dam\_break\_dc]{\hyperref[sec:nm_tab_dam_break]{config\_dam\_break\_dc}}
\label{subsec:nm_sec_config_dam_break_dc}
\begin{center}
\begin{longtable}{| p{2.0in} || p{4.0in} |}
    \hline
    Type: & real \\
    \hline
    Units: & \si{m} \\
    \hline
    Default Value: & 0.04 \\
    \hline
    Possible Values: & Any real number larger than zero. \\
    \hline
    \caption{config\_dam\_break\_dc: grid resolution in meters ($dc$).}
\end{longtable}
\end{center}
\subsection[config\_dam\_break\_R0]{\hyperref[sec:nm_tab_dam_break]{config\_dam\_break\_R0}}
\label{subsec:nm_sec_config_dam_break_R0}
\begin{center}
\begin{longtable}{| p{2.0in} || p{4.0in} |}
    \hline
    Type: & real \\
    \hline
    Units: & \si{m} \\
    \hline
    Default Value: & 24.2 \\
    \hline
    Possible Values: & sqrt(eta*9.8)*10.0. \\
    \hline
    \caption{config\_dam\_break\_R0: max wave propagation radius in 10.0s.}
\end{longtable}
\end{center}
\subsection[config\_dam\_break\_Xl]{\hyperref[sec:nm_tab_dam_break]{config\_dam\_break\_Xl}}
\label{subsec:nm_sec_config_dam_break_Xl}
\begin{center}
\begin{longtable}{| p{2.0in} || p{4.0in} |}
    \hline
    Type: & real \\
    \hline
    Units: & \si{m} \\
    \hline
    Default Value: & 1.0 \\
    \hline
    Possible Values: & Any real number larger than or equal to zero. \\
    \hline
    \caption{config\_dam\_break\_Xl: The length of dam along the X-direction.}
\end{longtable}
\end{center}
\subsection[config\_dam\_break\_Yl]{\hyperref[sec:nm_tab_dam_break]{config\_dam\_break\_Yl}}
\label{subsec:nm_sec_config_dam_break_Yl}
\begin{center}
\begin{longtable}{| p{2.0in} || p{4.0in} |}
    \hline
    Type: & real \\
    \hline
    Units: & \si{m} \\
    \hline
    Default Value: & 2.0 \\
    \hline
    Possible Values: & Any real number larger than or equal to zero. \\
    \hline
    \caption{config\_dam\_break\_Yl: The length of dam along the Y-direction.}
\end{longtable}
\end{center}
\subsection[config\_dam\_break\_Inlet]{\hyperref[sec:nm_tab_dam_break]{config\_dam\_break\_Inlet}}
\label{subsec:nm_sec_config_dam_break_Inlet}
\begin{center}
\begin{longtable}{| p{2.0in} || p{4.0in} |}
    \hline
    Type: & real \\
    \hline
    Units: & \si{m} \\
    \hline
    Default Value: & 0.4 \\
    \hline
    Possible Values: & Any real number larger than or equal to zero. \\
    \hline
    \caption{config\_dam\_break\_Inlet: The width of inlet (dam mouth).}
\end{longtable}
\end{center}
\section[global\_ocean]{\hyperref[sec:nm_tab_global_ocean]{global\_ocean}}
\label{sec:nm_sec_global_ocean}
\subsection[config\_global\_ocean\_minimum\_depth]{\hyperref[sec:nm_tab_global_ocean]{config\_global\_ocean\_minimum\_depth}}
\label{subsec:nm_sec_config_global_ocean_minimum_depth}
\begin{center}
\begin{longtable}{| p{2.0in} || p{4.0in} |}
    \hline
    Type: & real \\
    \hline
    Units: & \si{m} \\
    \hline
    Default Value: & 15 \\
    \hline
    Possible Values: & Any positive real value greater than 0, but typically greater than 10 m. \\
    \hline
    \caption{config\_global\_ocean\_minimum\_depth: Minimum depth where column contains all water-filled cells.  The first layer with refBottomDepth greater than this value is included in whole, i.e. no partial bottom cells are used in this layer.}
\end{longtable}
\end{center}
\subsection[config\_global\_ocean\_depth\_file]{\hyperref[sec:nm_tab_global_ocean]{config\_global\_ocean\_depth\_file}}
\label{subsec:nm_sec_config_global_ocean_depth_file}
\begin{center}
\begin{longtable}{| p{2.0in} || p{4.0in} |}
    \hline
    Type: & character \\
    \hline
    Units: & \si{unitless} \\
    \hline
    Default Value: & vertical\_grid.nc \\
    \hline
    Possible Values: & path/to/temperature/file.nc \\
    \hline
    \caption{config\_global\_ocean\_depth\_file: Path to the depth initial condition file.}
\end{longtable}
\end{center}
\subsection[config\_global\_ocean\_depth\_dimname]{\hyperref[sec:nm_tab_global_ocean]{config\_global\_ocean\_depth\_dimname}}
\label{subsec:nm_sec_config_global_ocean_depth_dimname}
\begin{center}
\begin{longtable}{| p{2.0in} || p{4.0in} |}
    \hline
    Type: & character \\
    \hline
    Units: & \si{unitless} \\
    \hline
    Default Value: & nVertLevels \\
    \hline
    Possible Values: & Dim name from input files. \\
    \hline
    \caption{config\_global\_ocean\_depth\_dimname: Name of the dimension defining the number of vertical levels in input files.}
\end{longtable}
\end{center}
\subsection[config\_global\_ocean\_depth\_varname]{\hyperref[sec:nm_tab_global_ocean]{config\_global\_ocean\_depth\_varname}}
\label{subsec:nm_sec_config_global_ocean_depth_varname}
\begin{center}
\begin{longtable}{| p{2.0in} || p{4.0in} |}
    \hline
    Type: & character \\
    \hline
    Units: & \si{unitless} \\
    \hline
    Default Value: & refMidDepth \\
    \hline
    Possible Values: & Variable name from input files. \\
    \hline
    \caption{config\_global\_ocean\_depth\_varname: Name of the variable containing mid-depth of levels in temperature and salinity initial condition files.}
\end{longtable}
\end{center}
\subsection[config\_global\_ocean\_depth\_conversion\_factor]{\hyperref[sec:nm_tab_global_ocean]{config\_global\_ocean\_depth\_conversion\_factor}}
\label{subsec:nm_sec_config_global_ocean_depth_conversion_factor}
\begin{center}
\begin{longtable}{| p{2.0in} || p{4.0in} |}
    \hline
    Type: & real \\
    \hline
    Units: & \si{variable} \\
    \hline
    Default Value: & 1.0 \\
    \hline
    Possible Values: & Any positive real value greater than 0. \\
    \hline
    \caption{config\_global\_ocean\_depth\_conversion\_factor: Conversion factor for depth levels. Should convert units on input depth levels to meters.}
\end{longtable}
\end{center}
\subsection[config\_global\_ocean\_temperature\_file]{\hyperref[sec:nm_tab_global_ocean]{config\_global\_ocean\_temperature\_file}}
\label{subsec:nm_sec_config_global_ocean_temperature_file}
\begin{center}
\begin{longtable}{| p{2.0in} || p{4.0in} |}
    \hline
    Type: & character \\
    \hline
    Units: & \si{unitless} \\
    \hline
    Default Value: & none \\
    \hline
    Possible Values: & path/to/temperature/file.nc \\
    \hline
    \caption{config\_global\_ocean\_temperature\_file: Path to the temperature initial condition file. Must be interpolated to vertical layers defined in depth file.}
\end{longtable}
\end{center}
\subsection[config\_global\_ocean\_salinity\_file]{\hyperref[sec:nm_tab_global_ocean]{config\_global\_ocean\_salinity\_file}}
\label{subsec:nm_sec_config_global_ocean_salinity_file}
\begin{center}
\begin{longtable}{| p{2.0in} || p{4.0in} |}
    \hline
    Type: & character \\
    \hline
    Units: & \si{unitless} \\
    \hline
    Default Value: & none \\
    \hline
    Possible Values: & path/to/salinity/file.nc \\
    \hline
    \caption{config\_global\_ocean\_salinity\_file: Path to the salinity initial condition file. Must be interpolated to vertical layers defined in depth file.}
\end{longtable}
\end{center}
\subsection[config\_global\_ocean\_tracer\_nlat\_dimname]{\hyperref[sec:nm_tab_global_ocean]{config\_global\_ocean\_tracer\_nlat\_dimname}}
\label{subsec:nm_sec_config_global_ocean_tracer_nlat_dimname}
\begin{center}
\begin{longtable}{| p{2.0in} || p{4.0in} |}
    \hline
    Type: & character \\
    \hline
    Units: & \si{unitless} \\
    \hline
    Default Value: & none \\
    \hline
    Possible Values: & Dim name from input files. \\
    \hline
    \caption{config\_global\_ocean\_tracer\_nlat\_dimname: Name of the dimension that determines number of latitude lines in tracer initial condition files.}
\end{longtable}
\end{center}
\subsection[config\_global\_ocean\_tracer\_nlon\_dimname]{\hyperref[sec:nm_tab_global_ocean]{config\_global\_ocean\_tracer\_nlon\_dimname}}
\label{subsec:nm_sec_config_global_ocean_tracer_nlon_dimname}
\begin{center}
\begin{longtable}{| p{2.0in} || p{4.0in} |}
    \hline
    Type: & character \\
    \hline
    Units: & \si{unitless} \\
    \hline
    Default Value: & none \\
    \hline
    Possible Values: & Dim name from input files. \\
    \hline
    \caption{config\_global\_ocean\_tracer\_nlon\_dimname: Name of the dimension that determines number of longitude lines in tracer initial condition files.}
\end{longtable}
\end{center}
\subsection[config\_global\_ocean\_tracer\_ndepth\_dimname]{\hyperref[sec:nm_tab_global_ocean]{config\_global\_ocean\_tracer\_ndepth\_dimname}}
\label{subsec:nm_sec_config_global_ocean_tracer_ndepth_dimname}
\begin{center}
\begin{longtable}{| p{2.0in} || p{4.0in} |}
    \hline
    Type: & character \\
    \hline
    Units: & \si{unitless} \\
    \hline
    Default Value: & none \\
    \hline
    Possible Values: & Dim name from input files. \\
    \hline
    \caption{config\_global\_ocean\_tracer\_ndepth\_dimname: Name of the dimension that determines number of vertical levels in tracer initial condition files.}
\end{longtable}
\end{center}
\subsection[config\_global\_ocean\_tracer\_depth\_conversion\_factor]{\hyperref[sec:nm_tab_global_ocean]{config\_global\_ocean\_tracer\_depth\_conversion\_factor}}
\label{subsec:nm_sec_config_global_ocean_tracer_depth_conversion_factor}
\begin{center}
\begin{longtable}{| p{2.0in} || p{4.0in} |}
    \hline
    Type: & real \\
    \hline
    Units: & \si{variable} \\
    \hline
    Default Value: & 1.0 \\
    \hline
    Possible Values: & Any positive real value greater than 0. \\
    \hline
    \caption{config\_global\_ocean\_tracer\_depth\_conversion\_factor: Conversion factor for tracer initial condition depth levels. Should convert units on input depth levels to meters.}
\end{longtable}
\end{center}
\subsection[config\_global\_ocean\_tracer\_vert\_levels]{\hyperref[sec:nm_tab_global_ocean]{config\_global\_ocean\_tracer\_vert\_levels}}
\label{subsec:nm_sec_config_global_ocean_tracer_vert_levels}
\begin{center}
\begin{longtable}{| p{2.0in} || p{4.0in} |}
    \hline
    Type: & integer \\
    \hline
    Units: & \si{unitless} \\
    \hline
    Default Value: & -1 \\
    \hline
    Possible Values: & Any positive non-zero integer. A value of -1 causes this to be overwritten with the configurations vertical level definition. \\
    \hline
    \caption{config\_global\_ocean\_tracer\_vert\_levels: Number of vertical levels in tracer initial condition file.  Set to -1 to read from file with config\_global\_ocean\_tracer\_ndepth\_dimname}
\end{longtable}
\end{center}
\subsection[config\_global\_ocean\_temperature\_varname]{\hyperref[sec:nm_tab_global_ocean]{config\_global\_ocean\_temperature\_varname}}
\label{subsec:nm_sec_config_global_ocean_temperature_varname}
\begin{center}
\begin{longtable}{| p{2.0in} || p{4.0in} |}
    \hline
    Type: & character \\
    \hline
    Units: & \si{unitless} \\
    \hline
    Default Value: & none \\
    \hline
    Possible Values: & Variable name from input file. \\
    \hline
    \caption{config\_global\_ocean\_temperature\_varname: Name of the variable containing temperature in temperature initial condition file.}
\end{longtable}
\end{center}
\subsection[config\_global\_ocean\_salinity\_varname]{\hyperref[sec:nm_tab_global_ocean]{config\_global\_ocean\_salinity\_varname}}
\label{subsec:nm_sec_config_global_ocean_salinity_varname}
\begin{center}
\begin{longtable}{| p{2.0in} || p{4.0in} |}
    \hline
    Type: & character \\
    \hline
    Units: & \si{unitless} \\
    \hline
    Default Value: & none \\
    \hline
    Possible Values: & Variable name from input file. \\
    \hline
    \caption{config\_global\_ocean\_salinity\_varname: Name of the variable containing salinity in salinity initial condition file.}
\end{longtable}
\end{center}
\subsection[config\_global\_ocean\_tracer\_latlon\_degrees]{\hyperref[sec:nm_tab_global_ocean]{config\_global\_ocean\_tracer\_latlon\_degrees}}
\label{subsec:nm_sec_config_global_ocean_tracer_latlon_degrees}
\begin{center}
\begin{longtable}{| p{2.0in} || p{4.0in} |}
    \hline
    Type: & logical \\
    \hline
    Units: & \si{unitless} \\
    \hline
    Default Value: & .true. \\
    \hline
    Possible Values: & .true. or .false. \\
    \hline
    \caption{config\_global\_ocean\_tracer\_latlon\_degrees: Logical flag that controls if the Lat/Lon fields for tracers should be converted to radians. True means input is degrees, false means input is radians.}
\end{longtable}
\end{center}
\subsection[config\_global\_ocean\_tracer\_lat\_varname]{\hyperref[sec:nm_tab_global_ocean]{config\_global\_ocean\_tracer\_lat\_varname}}
\label{subsec:nm_sec_config_global_ocean_tracer_lat_varname}
\begin{center}
\begin{longtable}{| p{2.0in} || p{4.0in} |}
    \hline
    Type: & character \\
    \hline
    Units: & \si{unitless} \\
    \hline
    Default Value: & none \\
    \hline
    Possible Values: & Variable name from input file. \\
    \hline
    \caption{config\_global\_ocean\_tracer\_lat\_varname: Name of the variable containing latitude coordinates for tracer values in temperature initial condition file.}
\end{longtable}
\end{center}
\subsection[config\_global\_ocean\_tracer\_lon\_varname]{\hyperref[sec:nm_tab_global_ocean]{config\_global\_ocean\_tracer\_lon\_varname}}
\label{subsec:nm_sec_config_global_ocean_tracer_lon_varname}
\begin{center}
\begin{longtable}{| p{2.0in} || p{4.0in} |}
    \hline
    Type: & character \\
    \hline
    Units: & \si{unitless} \\
    \hline
    Default Value: & none \\
    \hline
    Possible Values: & Variable name from input file. \\
    \hline
    \caption{config\_global\_ocean\_tracer\_lon\_varname: Name of the variable containing longitude coordinates for tracer values in temperature initial condition file.}
\end{longtable}
\end{center}
\subsection[config\_global\_ocean\_tracer\_depth\_varname]{\hyperref[sec:nm_tab_global_ocean]{config\_global\_ocean\_tracer\_depth\_varname}}
\label{subsec:nm_sec_config_global_ocean_tracer_depth_varname}
\begin{center}
\begin{longtable}{| p{2.0in} || p{4.0in} |}
    \hline
    Type: & character \\
    \hline
    Units: & \si{unitless} \\
    \hline
    Default Value: & none \\
    \hline
    Possible Values: & Variable name from input file. \\
    \hline
    \caption{config\_global\_ocean\_tracer\_depth\_varname: Name of the variable containing depth coordinates for tracer values in temperature initial condition file.}
\end{longtable}
\end{center}
\subsection[config\_global\_ocean\_tracer\_method]{\hyperref[sec:nm_tab_global_ocean]{config\_global\_ocean\_tracer\_method}}
\label{subsec:nm_sec_config_global_ocean_tracer_method}
\begin{center}
\begin{longtable}{| p{2.0in} || p{4.0in} |}
    \hline
    Type: & character \\
    \hline
    Units: & \si{unitless} \\
    \hline
    Default Value: & bilinear\_interpolation \\
    \hline
    Possible Values: & bilinear\_interpolation, nearest\_neighbor \\
    \hline
    \caption{config\_global\_ocean\_tracer\_method: Method to interpolate tracer data to MPAS mesh.}
\end{longtable}
\end{center}
\subsection[config\_global\_ocean\_smooth\_TS\_iterations]{\hyperref[sec:nm_tab_global_ocean]{config\_global\_ocean\_smooth\_TS\_iterations}}
\label{subsec:nm_sec_config_global_ocean_smooth_TS_iterations}
\begin{center}
\begin{longtable}{| p{2.0in} || p{4.0in} |}
    \hline
    Type: & integer \\
    \hline
    Units: & \si{unitless} \\
    \hline
    Default Value: & 0 \\
    \hline
    Possible Values: & Any positive integer value greater or equal to 0. \\
    \hline
    \caption{config\_global\_ocean\_smooth\_TS\_iterations: Number of smoothing iterations on temperature and salinity.}
\end{longtable}
\end{center}
\subsection[config\_global\_ocean\_swData\_file]{\hyperref[sec:nm_tab_global_ocean]{config\_global\_ocean\_swData\_file}}
\label{subsec:nm_sec_config_global_ocean_swData_file}
\begin{center}
\begin{longtable}{| p{2.0in} || p{4.0in} |}
    \hline
    Type: & character \\
    \hline
    Units: & \si{unitless} \\
    \hline
    Default Value: & none \\
    \hline
    Possible Values: & path/to/swData/file.nc \\
    \hline
    \caption{config\_global\_ocean\_swData\_file: Name of the file containing shortwaveData (chlA, zenith Angle, clear sky radiation)}
\end{longtable}
\end{center}
\subsection[config\_global\_ocean\_swData\_nlat\_dimname]{\hyperref[sec:nm_tab_global_ocean]{config\_global\_ocean\_swData\_nlat\_dimname}}
\label{subsec:nm_sec_config_global_ocean_swData_nlat_dimname}
\begin{center}
\begin{longtable}{| p{2.0in} || p{4.0in} |}
    \hline
    Type: & character \\
    \hline
    Units: & \si{unitless} \\
    \hline
    Default Value: & none \\
    \hline
    Possible Values: & Dim name from input files. \\
    \hline
    \caption{config\_global\_ocean\_swData\_nlat\_dimname: Name of the dimension that determines number of latitude lines in swData initial condition files.}
\end{longtable}
\end{center}
\subsection[config\_global\_ocean\_swData\_nlon\_dimname]{\hyperref[sec:nm_tab_global_ocean]{config\_global\_ocean\_swData\_nlon\_dimname}}
\label{subsec:nm_sec_config_global_ocean_swData_nlon_dimname}
\begin{center}
\begin{longtable}{| p{2.0in} || p{4.0in} |}
    \hline
    Type: & character \\
    \hline
    Units: & \si{unitless} \\
    \hline
    Default Value: & none \\
    \hline
    Possible Values: & Dim name from input files. \\
    \hline
    \caption{config\_global\_ocean\_swData\_nlon\_dimname: Name of the dimension that determines number of longitude lines in swData initial condition files.}
\end{longtable}
\end{center}
\subsection[config\_global\_ocean\_swData\_lat\_varname]{\hyperref[sec:nm_tab_global_ocean]{config\_global\_ocean\_swData\_lat\_varname}}
\label{subsec:nm_sec_config_global_ocean_swData_lat_varname}
\begin{center}
\begin{longtable}{| p{2.0in} || p{4.0in} |}
    \hline
    Type: & character \\
    \hline
    Units: & \si{unitless} \\
    \hline
    Default Value: & none \\
    \hline
    Possible Values: & Variable name from input file. \\
    \hline
    \caption{config\_global\_ocean\_swData\_lat\_varname: Name of the variable containing latitude coordinates for swData values in swData initial condition file.}
\end{longtable}
\end{center}
\subsection[config\_global\_ocean\_swData\_lon\_varname]{\hyperref[sec:nm_tab_global_ocean]{config\_global\_ocean\_swData\_lon\_varname}}
\label{subsec:nm_sec_config_global_ocean_swData_lon_varname}
\begin{center}
\begin{longtable}{| p{2.0in} || p{4.0in} |}
    \hline
    Type: & character \\
    \hline
    Units: & \si{unitless} \\
    \hline
    Default Value: & none \\
    \hline
    Possible Values: & Variable name from input file. \\
    \hline
    \caption{config\_global\_ocean\_swData\_lon\_varname: Name of the variable containing longitude coordinates for swData values in swData initial condition file.}
\end{longtable}
\end{center}
\subsection[config\_global\_ocean\_swData\_latlon\_degrees]{\hyperref[sec:nm_tab_global_ocean]{config\_global\_ocean\_swData\_latlon\_degrees}}
\label{subsec:nm_sec_config_global_ocean_swData_latlon_degrees}
\begin{center}
\begin{longtable}{| p{2.0in} || p{4.0in} |}
    \hline
    Type: & logical \\
    \hline
    Units: & \si{unitless} \\
    \hline
    Default Value: & .true. \\
    \hline
    Possible Values: & .true. or .false. \\
    \hline
    \caption{config\_global\_ocean\_swData\_latlon\_degrees: Logical flag that controls if the Lat/Lon fields for swData should be converted to radians. True means input is degrees, false means input is radians.}
\end{longtable}
\end{center}
\subsection[config\_global\_ocean\_swData\_method]{\hyperref[sec:nm_tab_global_ocean]{config\_global\_ocean\_swData\_method}}
\label{subsec:nm_sec_config_global_ocean_swData_method}
\begin{center}
\begin{longtable}{| p{2.0in} || p{4.0in} |}
    \hline
    Type: & character \\
    \hline
    Units: & \si{unitless} \\
    \hline
    Default Value: & bilinear\_interpolation \\
    \hline
    Possible Values: & bilinear\_interpolation, nearest\_neighbor \\
    \hline
    \caption{config\_global\_ocean\_swData\_method: Method to interpolate shortwave data to MPAS mesh.}
\end{longtable}
\end{center}
\subsection[config\_global\_ocean\_chlorophyll\_varname]{\hyperref[sec:nm_tab_global_ocean]{config\_global\_ocean\_chlorophyll\_varname}}
\label{subsec:nm_sec_config_global_ocean_chlorophyll_varname}
\begin{center}
\begin{longtable}{| p{2.0in} || p{4.0in} |}
    \hline
    Type: & character \\
    \hline
    Units: & \si{unitless} \\
    \hline
    Default Value: & none \\
    \hline
    Possible Values: & Variable name from input file. \\
    \hline
    \caption{config\_global\_ocean\_chlorophyll\_varname: Name of the variable containing chlorophyll in sw Data initial condition file.}
\end{longtable}
\end{center}
\subsection[config\_global\_ocean\_zenithAngle\_varname]{\hyperref[sec:nm_tab_global_ocean]{config\_global\_ocean\_zenithAngle\_varname}}
\label{subsec:nm_sec_config_global_ocean_zenithAngle_varname}
\begin{center}
\begin{longtable}{| p{2.0in} || p{4.0in} |}
    \hline
    Type: & character \\
    \hline
    Units: & \si{unitless} \\
    \hline
    Default Value: & none \\
    \hline
    Possible Values: & Variable name from input file. \\
    \hline
    \caption{config\_global\_ocean\_zenithAngle\_varname: Name of the variable containing zenith angle in swData initial condition file.}
\end{longtable}
\end{center}
\subsection[config\_global\_ocean\_clearSky\_varname]{\hyperref[sec:nm_tab_global_ocean]{config\_global\_ocean\_clearSky\_varname}}
\label{subsec:nm_sec_config_global_ocean_clearSky_varname}
\begin{center}
\begin{longtable}{| p{2.0in} || p{4.0in} |}
    \hline
    Type: & character \\
    \hline
    Units: & \si{unitless} \\
    \hline
    Default Value: & none \\
    \hline
    Possible Values: & Variable name from input file. \\
    \hline
    \caption{config\_global\_ocean\_clearSky\_varname: Name of the variable containing clear sky radiation in clear sky radiation initial condition file.}
\end{longtable}
\end{center}
\subsection[config\_global\_ocean\_piston\_velocity]{\hyperref[sec:nm_tab_global_ocean]{config\_global\_ocean\_piston\_velocity}}
\label{subsec:nm_sec_config_global_ocean_piston_velocity}
\begin{center}
\begin{longtable}{| p{2.0in} || p{4.0in} |}
    \hline
    Type: & real \\
    \hline
    Units: & \si{m.s^{-1}} \\
    \hline
    Default Value: & 5.0e-5 \\
    \hline
    Possible Values: & Any real positive number. \\
    \hline
    \caption{config\_global\_ocean\_piston\_velocity: Parameter controlling rate to which SST and SST are restored.}
\end{longtable}
\end{center}
\subsection[config\_global\_ocean\_interior\_restore\_rate]{\hyperref[sec:nm_tab_global_ocean]{config\_global\_ocean\_interior\_restore\_rate}}
\label{subsec:nm_sec_config_global_ocean_interior_restore_rate}
\begin{center}
\begin{longtable}{| p{2.0in} || p{4.0in} |}
    \hline
    Type: & real \\
    \hline
    Units: & \si{s^{-1}} \\
    \hline
    Default Value: & 1.0e-7 \\
    \hline
    Possible Values: & Any real positive number. \\
    \hline
    \caption{config\_global\_ocean\_interior\_restore\_rate: Parameter controlling rate to which interior temperature and salinity are restored.}
\end{longtable}
\end{center}
\subsection[config\_global\_ocean\_topography\_source]{\hyperref[sec:nm_tab_global_ocean]{config\_global\_ocean\_topography\_source}}
\label{subsec:nm_sec_config_global_ocean_topography_source}
\begin{center}
\begin{longtable}{| p{2.0in} || p{4.0in} |}
    \hline
    Type: & character \\
    \hline
    Units: & \si{unitless} \\
    \hline
    Default Value: & latlon\_file \\
    \hline
    Possible Values: & 'latlon\_file' or 'mpas\_variable' \\
    \hline
    \caption{config\_global\_ocean\_topography\_source: If 'latlon\_file', reads in topography from file specified in config\_global\_ocean\_topography\_file. If 'mpas\_variable', reads in topography from mpas variable bed\_elevation, and optionally oceanFracObserved, landIceDraftObserved, landIceThkObserved, landIceFracObserved, and landIceGroundedFracObserved}
\end{longtable}
\end{center}
\subsection[config\_global\_ocean\_topography\_file]{\hyperref[sec:nm_tab_global_ocean]{config\_global\_ocean\_topography\_file}}
\label{subsec:nm_sec_config_global_ocean_topography_file}
\begin{center}
\begin{longtable}{| p{2.0in} || p{4.0in} |}
    \hline
    Type: & character \\
    \hline
    Units: & \si{unitless} \\
    \hline
    Default Value: & none \\
    \hline
    Possible Values: & path/to/topography/file.nc \\
    \hline
    \caption{config\_global\_ocean\_topography\_file: Path to the topography initial condition file.}
\end{longtable}
\end{center}
\subsection[config\_global\_ocean\_topography\_nlat\_dimname]{\hyperref[sec:nm_tab_global_ocean]{config\_global\_ocean\_topography\_nlat\_dimname}}
\label{subsec:nm_sec_config_global_ocean_topography_nlat_dimname}
\begin{center}
\begin{longtable}{| p{2.0in} || p{4.0in} |}
    \hline
    Type: & character \\
    \hline
    Units: & \si{unitless} \\
    \hline
    Default Value: & none \\
    \hline
    Possible Values: & Dimension name from input file. \\
    \hline
    \caption{config\_global\_ocean\_topography\_nlat\_dimname: Dimension name for the latitude in the topography file.}
\end{longtable}
\end{center}
\subsection[config\_global\_ocean\_topography\_nlon\_dimname]{\hyperref[sec:nm_tab_global_ocean]{config\_global\_ocean\_topography\_nlon\_dimname}}
\label{subsec:nm_sec_config_global_ocean_topography_nlon_dimname}
\begin{center}
\begin{longtable}{| p{2.0in} || p{4.0in} |}
    \hline
    Type: & character \\
    \hline
    Units: & \si{unitless} \\
    \hline
    Default Value: & none \\
    \hline
    Possible Values: & Dimension name from input file. \\
    \hline
    \caption{config\_global\_ocean\_topography\_nlon\_dimname: Dimension name for the longitude in the topography file.}
\end{longtable}
\end{center}
\subsection[config\_global\_ocean\_topography\_latlon\_degrees]{\hyperref[sec:nm_tab_global_ocean]{config\_global\_ocean\_topography\_latlon\_degrees}}
\label{subsec:nm_sec_config_global_ocean_topography_latlon_degrees}
\begin{center}
\begin{longtable}{| p{2.0in} || p{4.0in} |}
    \hline
    Type: & logical \\
    \hline
    Units: & \si{unitless} \\
    \hline
    Default Value: & .true. \\
    \hline
    Possible Values: & .true. or .false. \\
    \hline
    \caption{config\_global\_ocean\_topography\_latlon\_degrees: Logical flag that controls if the Lat/Lon fields for topography should be converted to radians. True means input is degrees, false means input is radians.}
\end{longtable}
\end{center}
\subsection[config\_global\_ocean\_topography\_lat\_varname]{\hyperref[sec:nm_tab_global_ocean]{config\_global\_ocean\_topography\_lat\_varname}}
\label{subsec:nm_sec_config_global_ocean_topography_lat_varname}
\begin{center}
\begin{longtable}{| p{2.0in} || p{4.0in} |}
    \hline
    Type: & character \\
    \hline
    Units: & \si{unitless} \\
    \hline
    Default Value: & none \\
    \hline
    Possible Values: & Variable name from input file. \\
    \hline
    \caption{config\_global\_ocean\_topography\_lat\_varname: Variable name for the latitude in the topography file.}
\end{longtable}
\end{center}
\subsection[config\_global\_ocean\_topography\_lon\_varname]{\hyperref[sec:nm_tab_global_ocean]{config\_global\_ocean\_topography\_lon\_varname}}
\label{subsec:nm_sec_config_global_ocean_topography_lon_varname}
\begin{center}
\begin{longtable}{| p{2.0in} || p{4.0in} |}
    \hline
    Type: & character \\
    \hline
    Units: & \si{unitless} \\
    \hline
    Default Value: & none \\
    \hline
    Possible Values: & Variable name from input file. \\
    \hline
    \caption{config\_global\_ocean\_topography\_lon\_varname: Variable name for the longitude in the topography file.}
\end{longtable}
\end{center}
\subsection[config\_global\_ocean\_topography\_varname]{\hyperref[sec:nm_tab_global_ocean]{config\_global\_ocean\_topography\_varname}}
\label{subsec:nm_sec_config_global_ocean_topography_varname}
\begin{center}
\begin{longtable}{| p{2.0in} || p{4.0in} |}
    \hline
    Type: & character \\
    \hline
    Units: & \si{unitless} \\
    \hline
    Default Value: & none \\
    \hline
    Possible Values: & Variable name from input file. \\
    \hline
    \caption{config\_global\_ocean\_topography\_varname: Variable name for the topography in the topography file.}
\end{longtable}
\end{center}
\subsection[config\_global\_ocean\_topography\_has\_ocean\_frac]{\hyperref[sec:nm_tab_global_ocean]{config\_global\_ocean\_topography\_has\_ocean\_frac}}
\label{subsec:nm_sec_config_global_ocean_topography_has_ocean_frac}
\begin{center}
\begin{longtable}{| p{2.0in} || p{4.0in} |}
    \hline
    Type: & logical \\
    \hline
    Units: & \si{unitless} \\
    \hline
    Default Value: & .false. \\
    \hline
    Possible Values: & .true. or .false. \\
    \hline
    \caption{config\_global\_ocean\_topography\_has\_ocean\_frac: Logical flag that controls if topography file contains a field for the fraction of each cell that contains ocean (vs. land or grounded ice).}
\end{longtable}
\end{center}
\subsection[config\_global\_ocean\_topography\_ocean\_frac\_varname]{\hyperref[sec:nm_tab_global_ocean]{config\_global\_ocean\_topography\_ocean\_frac\_varname}}
\label{subsec:nm_sec_config_global_ocean_topography_ocean_frac_varname}
\begin{center}
\begin{longtable}{| p{2.0in} || p{4.0in} |}
    \hline
    Type: & character \\
    \hline
    Units: & \si{unitless} \\
    \hline
    Default Value: & none \\
    \hline
    Possible Values: & Variable name from input file. \\
    \hline
    \caption{config\_global\_ocean\_topography\_ocean\_frac\_varname: Variable name for the ocean mask in the topography file.}
\end{longtable}
\end{center}
\subsection[config\_global\_ocean\_topography\_method]{\hyperref[sec:nm_tab_global_ocean]{config\_global\_ocean\_topography\_method}}
\label{subsec:nm_sec_config_global_ocean_topography_method}
\begin{center}
\begin{longtable}{| p{2.0in} || p{4.0in} |}
    \hline
    Type: & character \\
    \hline
    Units: & \si{unitless} \\
    \hline
    Default Value: & bilinear\_interpolation \\
    \hline
    Possible Values: & bilinear\_interpolation, nearest\_neighbor \\
    \hline
    \caption{config\_global\_ocean\_topography\_method: Method to interpolate topography data to MPAS mesh.}
\end{longtable}
\end{center}
\subsection[config\_global\_ocean\_fill\_bathymetry\_holes]{\hyperref[sec:nm_tab_global_ocean]{config\_global\_ocean\_fill\_bathymetry\_holes}}
\label{subsec:nm_sec_config_global_ocean_fill_bathymetry_holes}
\begin{center}
\begin{longtable}{| p{2.0in} || p{4.0in} |}
    \hline
    Type: & logical \\
    \hline
    Units: & \si{unitless} \\
    \hline
    Default Value: & .true. \\
    \hline
    Possible Values: & .true. or .false. \\
    \hline
    \caption{config\_global\_ocean\_fill\_bathymetry\_holes: Logical flag that controls if deep holes in the bathymetry should be filled after interpolation to the MPAS mesh.}
\end{longtable}
\end{center}
\subsection[config\_global\_ocean\_topography\_smooth\_iterations]{\hyperref[sec:nm_tab_global_ocean]{config\_global\_ocean\_topography\_smooth\_iterations}}
\label{subsec:nm_sec_config_global_ocean_topography_smooth_iterations}
\begin{center}
\begin{longtable}{| p{2.0in} || p{4.0in} |}
    \hline
    Type: & integer \\
    \hline
    Units: & \si{unitless} \\
    \hline
    Default Value: & 0 \\
    \hline
    Possible Values: & any non-negative integer \\
    \hline
    \caption{config\_global\_ocean\_topography\_smooth\_iterations: How many iterations of topography smoothing by weighted averaging of cellsOnCell to perform.}
\end{longtable}
\end{center}
\subsection[config\_global\_ocean\_topography\_smooth\_weight]{\hyperref[sec:nm_tab_global_ocean]{config\_global\_ocean\_topography\_smooth\_weight}}
\label{subsec:nm_sec_config_global_ocean_topography_smooth_weight}
\begin{center}
\begin{longtable}{| p{2.0in} || p{4.0in} |}
    \hline
    Type: & real \\
    \hline
    Units: & \si{unitless} \\
    \hline
    Default Value: & 0.9 \\
    \hline
    Possible Values: & fraction between 0 and 1 \\
    \hline
    \caption{config\_global\_ocean\_topography\_smooth\_weight: The weight given to the central cell during smoothing.  The n cellsOnCell are given a weight (1-weight)/n.}
\end{longtable}
\end{center}
\subsection[config\_global\_ocean\_deepen\_critical\_passages]{\hyperref[sec:nm_tab_global_ocean]{config\_global\_ocean\_deepen\_critical\_passages}}
\label{subsec:nm_sec_config_global_ocean_deepen_critical_passages}
\begin{center}
\begin{longtable}{| p{2.0in} || p{4.0in} |}
    \hline
    Type: & logical \\
    \hline
    Units: & \si{unitless} \\
    \hline
    Default Value: & .true. \\
    \hline
    Possible Values: & .true. or .false. \\
    \hline
    \caption{config\_global\_ocean\_deepen\_critical\_passages: Logical flag that controls if critical passages are deepened to a minimum depth.}
\end{longtable}
\end{center}
\subsection[config\_global\_ocean\_depress\_by\_land\_ice]{\hyperref[sec:nm_tab_global_ocean]{config\_global\_ocean\_depress\_by\_land\_ice}}
\label{subsec:nm_sec_config_global_ocean_depress_by_land_ice}
\begin{center}
\begin{longtable}{| p{2.0in} || p{4.0in} |}
    \hline
    Type: & logical \\
    \hline
    Units: & \si{unitless} \\
    \hline
    Default Value: & .false. \\
    \hline
    Possible Values: & .true. or .false. \\
    \hline
    \caption{config\_global\_ocean\_depress\_by\_land\_ice: Logical flag that controls if sea surface pressure and layer thicknesses should be altered by an overlying ice sheet/shelf.}
\end{longtable}
\end{center}
\subsection[config\_global\_ocean\_land\_ice\_topo\_file]{\hyperref[sec:nm_tab_global_ocean]{config\_global\_ocean\_land\_ice\_topo\_file}}
\label{subsec:nm_sec_config_global_ocean_land_ice_topo_file}
\begin{center}
\begin{longtable}{| p{2.0in} || p{4.0in} |}
    \hline
    Type: & character \\
    \hline
    Units: & \si{unitless} \\
    \hline
    Default Value: & none \\
    \hline
    Possible Values: & path/to/land\_ice\_topography/file.nc \\
    \hline
    \caption{config\_global\_ocean\_land\_ice\_topo\_file: Path to the land ice topography initial condition file.}
\end{longtable}
\end{center}
\subsection[config\_global\_ocean\_land\_ice\_topo\_nlat\_dimname]{\hyperref[sec:nm_tab_global_ocean]{config\_global\_ocean\_land\_ice\_topo\_nlat\_dimname}}
\label{subsec:nm_sec_config_global_ocean_land_ice_topo_nlat_dimname}
\begin{center}
\begin{longtable}{| p{2.0in} || p{4.0in} |}
    \hline
    Type: & character \\
    \hline
    Units: & \si{unitless} \\
    \hline
    Default Value: & none \\
    \hline
    Possible Values: & Dimension name from input file. \\
    \hline
    \caption{config\_global\_ocean\_land\_ice\_topo\_nlat\_dimname: Dimension name for the latitude in the land ice topography file.}
\end{longtable}
\end{center}
\subsection[config\_global\_ocean\_land\_ice\_topo\_nlon\_dimname]{\hyperref[sec:nm_tab_global_ocean]{config\_global\_ocean\_land\_ice\_topo\_nlon\_dimname}}
\label{subsec:nm_sec_config_global_ocean_land_ice_topo_nlon_dimname}
\begin{center}
\begin{longtable}{| p{2.0in} || p{4.0in} |}
    \hline
    Type: & character \\
    \hline
    Units: & \si{unitless} \\
    \hline
    Default Value: & none \\
    \hline
    Possible Values: & Dimension name from input file. \\
    \hline
    \caption{config\_global\_ocean\_land\_ice\_topo\_nlon\_dimname: Dimension name for the longitude in the land ice topography file.}
\end{longtable}
\end{center}
\subsection[config\_global\_ocean\_land\_ice\_topo\_latlon\_degrees]{\hyperref[sec:nm_tab_global_ocean]{config\_global\_ocean\_land\_ice\_topo\_latlon\_degrees}}
\label{subsec:nm_sec_config_global_ocean_land_ice_topo_latlon_degrees}
\begin{center}
\begin{longtable}{| p{2.0in} || p{4.0in} |}
    \hline
    Type: & logical \\
    \hline
    Units: & \si{unitless} \\
    \hline
    Default Value: & .true. \\
    \hline
    Possible Values: & .true. or .false. \\
    \hline
    \caption{config\_global\_ocean\_land\_ice\_topo\_latlon\_degrees: Logical flag that controls if the Lat/Lon fields for land ice topography should be converted to radians. True means input is degrees, false means input is radians.}
\end{longtable}
\end{center}
\subsection[config\_global\_ocean\_land\_ice\_topo\_lat\_varname]{\hyperref[sec:nm_tab_global_ocean]{config\_global\_ocean\_land\_ice\_topo\_lat\_varname}}
\label{subsec:nm_sec_config_global_ocean_land_ice_topo_lat_varname}
\begin{center}
\begin{longtable}{| p{2.0in} || p{4.0in} |}
    \hline
    Type: & character \\
    \hline
    Units: & \si{unitless} \\
    \hline
    Default Value: & none \\
    \hline
    Possible Values: & Variable name from input file. \\
    \hline
    \caption{config\_global\_ocean\_land\_ice\_topo\_lat\_varname: Variable name for the latitude in the land ice topography file.}
\end{longtable}
\end{center}
\subsection[config\_global\_ocean\_land\_ice\_topo\_lon\_varname]{\hyperref[sec:nm_tab_global_ocean]{config\_global\_ocean\_land\_ice\_topo\_lon\_varname}}
\label{subsec:nm_sec_config_global_ocean_land_ice_topo_lon_varname}
\begin{center}
\begin{longtable}{| p{2.0in} || p{4.0in} |}
    \hline
    Type: & character \\
    \hline
    Units: & \si{unitless} \\
    \hline
    Default Value: & none \\
    \hline
    Possible Values: & Variable name from input file. \\
    \hline
    \caption{config\_global\_ocean\_land\_ice\_topo\_lon\_varname: Variable name for the longitude in the land ice topography file.}
\end{longtable}
\end{center}
\subsection[config\_global\_ocean\_land\_ice\_topo\_thickness\_varname]{\hyperref[sec:nm_tab_global_ocean]{config\_global\_ocean\_land\_ice\_topo\_thickness\_varname}}
\label{subsec:nm_sec_config_global_ocean_land_ice_topo_thickness_varname}
\begin{center}
\begin{longtable}{| p{2.0in} || p{4.0in} |}
    \hline
    Type: & character \\
    \hline
    Units: & \si{unitless} \\
    \hline
    Default Value: & none \\
    \hline
    Possible Values: & Variable name from input file. \\
    \hline
    \caption{config\_global\_ocean\_land\_ice\_topo\_thickness\_varname: Variable name for the land ice thickness in the land ice topography file.}
\end{longtable}
\end{center}
\subsection[config\_global\_ocean\_land\_ice\_topo\_draft\_varname]{\hyperref[sec:nm_tab_global_ocean]{config\_global\_ocean\_land\_ice\_topo\_draft\_varname}}
\label{subsec:nm_sec_config_global_ocean_land_ice_topo_draft_varname}
\begin{center}
\begin{longtable}{| p{2.0in} || p{4.0in} |}
    \hline
    Type: & character \\
    \hline
    Units: & \si{unitless} \\
    \hline
    Default Value: & none \\
    \hline
    Possible Values: & Variable name from input file. \\
    \hline
    \caption{config\_global\_ocean\_land\_ice\_topo\_draft\_varname: Variable name for the land ice draft in the land ice topography file.}
\end{longtable}
\end{center}
\subsection[config\_global\_ocean\_land\_ice\_topo\_ice\_frac\_varname]{\hyperref[sec:nm_tab_global_ocean]{config\_global\_ocean\_land\_ice\_topo\_ice\_frac\_varname}}
\label{subsec:nm_sec_config_global_ocean_land_ice_topo_ice_frac_varname}
\begin{center}
\begin{longtable}{| p{2.0in} || p{4.0in} |}
    \hline
    Type: & character \\
    \hline
    Units: & \si{unitless} \\
    \hline
    Default Value: & none \\
    \hline
    Possible Values: & Variable name from input file. \\
    \hline
    \caption{config\_global\_ocean\_land\_ice\_topo\_ice\_frac\_varname: Variable name for the land ice fraction in the land ice topography file.}
\end{longtable}
\end{center}
\subsection[config\_global\_ocean\_land\_ice\_topo\_grounded\_frac\_varname]{\hyperref[sec:nm_tab_global_ocean]{config\_global\_ocean\_land\_ice\_topo\_grounded\_frac\_varname}}
\label{subsec:nm_sec_config_global_ocean_land_ice_topo_grounded_frac_varname}
\begin{center}
\begin{longtable}{| p{2.0in} || p{4.0in} |}
    \hline
    Type: & character \\
    \hline
    Units: & \si{unitless} \\
    \hline
    Default Value: & none \\
    \hline
    Possible Values: & Variable name from input file. \\
    \hline
    \caption{config\_global\_ocean\_land\_ice\_topo\_grounded\_frac\_varname: Variable name for the grounded land ice fraction in the land ice topography file.}
\end{longtable}
\end{center}
\subsection[config\_global\_ocean\_use\_constant\_land\_ice\_cavity\_temperature]{\hyperref[sec:nm_tab_global_ocean]{config\_global\_ocean\_use\_constant\_land\_ice\_cavity\_temperature}}
\label{subsec:nm_sec_config_global_ocean_use_constant_land_ice_cavity_temperature}
\begin{center}
\begin{longtable}{| p{2.0in} || p{4.0in} |}
    \hline
    Type: & logical \\
    \hline
    Units: & \si{unitless} \\
    \hline
    Default Value: & .false. \\
    \hline
    Possible Values: & .true. or .false. \\
    \hline
    \caption{config\_global\_ocean\_use\_constant\_land\_ice\_cavity\_temperature: Logical flag that controls if ocean temperature in land-ice cavities is set to a constant temperature.}
\end{longtable}
\end{center}
\subsection[config\_global\_ocean\_constant\_land\_ice\_cavity\_temperature]{\hyperref[sec:nm_tab_global_ocean]{config\_global\_ocean\_constant\_land\_ice\_cavity\_temperature}}
\label{subsec:nm_sec_config_global_ocean_constant_land_ice_cavity_temperature}
\begin{center}
\begin{longtable}{| p{2.0in} || p{4.0in} |}
    \hline
    Type: & real \\
    \hline
    Units: & \si{C} \\
    \hline
    Default Value: & -1.8 \\
    \hline
    Possible Values: & Any real number. \\
    \hline
    \caption{config\_global\_ocean\_constant\_land\_ice\_cavity\_temperature: The constant temperature value to be used under land ice, typically something close to the freezing point.}
\end{longtable}
\end{center}
\subsection[config\_global\_ocean\_cull\_inland\_seas]{\hyperref[sec:nm_tab_global_ocean]{config\_global\_ocean\_cull\_inland\_seas}}
\label{subsec:nm_sec_config_global_ocean_cull_inland_seas}
\begin{center}
\begin{longtable}{| p{2.0in} || p{4.0in} |}
    \hline
    Type: & logical \\
    \hline
    Units: & \si{unitless} \\
    \hline
    Default Value: & .true. \\
    \hline
    Possible Values: & .true. or .false. \\
    \hline
    \caption{config\_global\_ocean\_cull\_inland\_seas: Logical flag that controls if inland seas should be removed.}
\end{longtable}
\end{center}
\subsection[config\_global\_ocean\_windstress\_file]{\hyperref[sec:nm_tab_global_ocean]{config\_global\_ocean\_windstress\_file}}
\label{subsec:nm_sec_config_global_ocean_windstress_file}
\begin{center}
\begin{longtable}{| p{2.0in} || p{4.0in} |}
    \hline
    Type: & character \\
    \hline
    Units: & \si{unitless} \\
    \hline
    Default Value: & none \\
    \hline
    Possible Values: & path/to/windstress/file.nc \\
    \hline
    \caption{config\_global\_ocean\_windstress\_file: Path to the windstress initial condition file.}
\end{longtable}
\end{center}
\subsection[config\_global\_ocean\_windstress\_nlat\_dimname]{\hyperref[sec:nm_tab_global_ocean]{config\_global\_ocean\_windstress\_nlat\_dimname}}
\label{subsec:nm_sec_config_global_ocean_windstress_nlat_dimname}
\begin{center}
\begin{longtable}{| p{2.0in} || p{4.0in} |}
    \hline
    Type: & character \\
    \hline
    Units: & \si{unitless} \\
    \hline
    Default Value: & none \\
    \hline
    Possible Values: & Dimension name from input file. \\
    \hline
    \caption{config\_global\_ocean\_windstress\_nlat\_dimname: Dimension name for the latitude in the windstress file.}
\end{longtable}
\end{center}
\subsection[config\_global\_ocean\_windstress\_nlon\_dimname]{\hyperref[sec:nm_tab_global_ocean]{config\_global\_ocean\_windstress\_nlon\_dimname}}
\label{subsec:nm_sec_config_global_ocean_windstress_nlon_dimname}
\begin{center}
\begin{longtable}{| p{2.0in} || p{4.0in} |}
    \hline
    Type: & character \\
    \hline
    Units: & \si{unitless} \\
    \hline
    Default Value: & none \\
    \hline
    Possible Values: & Dimension name from input file. \\
    \hline
    \caption{config\_global\_ocean\_windstress\_nlon\_dimname: Dimension name for the longitude in the windstress file.}
\end{longtable}
\end{center}
\subsection[config\_global\_ocean\_windstress\_latlon\_degrees]{\hyperref[sec:nm_tab_global_ocean]{config\_global\_ocean\_windstress\_latlon\_degrees}}
\label{subsec:nm_sec_config_global_ocean_windstress_latlon_degrees}
\begin{center}
\begin{longtable}{| p{2.0in} || p{4.0in} |}
    \hline
    Type: & logical \\
    \hline
    Units: & \si{unitless} \\
    \hline
    Default Value: & .true. \\
    \hline
    Possible Values: & .true. or .false. \\
    \hline
    \caption{config\_global\_ocean\_windstress\_latlon\_degrees: Logical flag that controls if the Lat/Lon fields for windstress should be converted to radians. True means input is degrees, false means input is radians.}
\end{longtable}
\end{center}
\subsection[config\_global\_ocean\_windstress\_lat\_varname]{\hyperref[sec:nm_tab_global_ocean]{config\_global\_ocean\_windstress\_lat\_varname}}
\label{subsec:nm_sec_config_global_ocean_windstress_lat_varname}
\begin{center}
\begin{longtable}{| p{2.0in} || p{4.0in} |}
    \hline
    Type: & character \\
    \hline
    Units: & \si{unitless} \\
    \hline
    Default Value: & none \\
    \hline
    Possible Values: & Variable name from input file. \\
    \hline
    \caption{config\_global\_ocean\_windstress\_lat\_varname: Variable name for the latitude in the windstress file.}
\end{longtable}
\end{center}
\subsection[config\_global\_ocean\_windstress\_lon\_varname]{\hyperref[sec:nm_tab_global_ocean]{config\_global\_ocean\_windstress\_lon\_varname}}
\label{subsec:nm_sec_config_global_ocean_windstress_lon_varname}
\begin{center}
\begin{longtable}{| p{2.0in} || p{4.0in} |}
    \hline
    Type: & character \\
    \hline
    Units: & \si{unitless} \\
    \hline
    Default Value: & none \\
    \hline
    Possible Values: & Variable name from input file. \\
    \hline
    \caption{config\_global\_ocean\_windstress\_lon\_varname: Variable name for the longitude in the windstress file.}
\end{longtable}
\end{center}
\subsection[config\_global\_ocean\_windstress\_zonal\_varname]{\hyperref[sec:nm_tab_global_ocean]{config\_global\_ocean\_windstress\_zonal\_varname}}
\label{subsec:nm_sec_config_global_ocean_windstress_zonal_varname}
\begin{center}
\begin{longtable}{| p{2.0in} || p{4.0in} |}
    \hline
    Type: & character \\
    \hline
    Units: & \si{unitless} \\
    \hline
    Default Value: & none \\
    \hline
    Possible Values: & Variable name from input file. \\
    \hline
    \caption{config\_global\_ocean\_windstress\_zonal\_varname: Variable name for the zonal component of windstress in the windstress file.}
\end{longtable}
\end{center}
\subsection[config\_global\_ocean\_windstress\_meridional\_varname]{\hyperref[sec:nm_tab_global_ocean]{config\_global\_ocean\_windstress\_meridional\_varname}}
\label{subsec:nm_sec_config_global_ocean_windstress_meridional_varname}
\begin{center}
\begin{longtable}{| p{2.0in} || p{4.0in} |}
    \hline
    Type: & character \\
    \hline
    Units: & \si{unitless} \\
    \hline
    Default Value: & none \\
    \hline
    Possible Values: & Variable name from input file. \\
    \hline
    \caption{config\_global\_ocean\_windstress\_meridional\_varname: Variable name for the meridional component of windstress in the windstress file.}
\end{longtable}
\end{center}
\subsection[config\_global\_ocean\_windstress\_method]{\hyperref[sec:nm_tab_global_ocean]{config\_global\_ocean\_windstress\_method}}
\label{subsec:nm_sec_config_global_ocean_windstress_method}
\begin{center}
\begin{longtable}{| p{2.0in} || p{4.0in} |}
    \hline
    Type: & character \\
    \hline
    Units: & \si{unitless} \\
    \hline
    Default Value: & bilinear\_interpolation \\
    \hline
    Possible Values: & bilinear\_interpolation, nearest\_neighbor \\
    \hline
    \caption{config\_global\_ocean\_windstress\_method: Method to interpolate windstress data to MPAS mesh.}
\end{longtable}
\end{center}
\subsection[config\_global\_ocean\_windstress\_conversion\_factor]{\hyperref[sec:nm_tab_global_ocean]{config\_global\_ocean\_windstress\_conversion\_factor}}
\label{subsec:nm_sec_config_global_ocean_windstress_conversion_factor}
\begin{center}
\begin{longtable}{| p{2.0in} || p{4.0in} |}
    \hline
    Type: & real \\
    \hline
    Units: & \si{variable} \\
    \hline
    Default Value: & 1 \\
    \hline
    Possible Values: & Any positive real number. \\
    \hline
    \caption{config\_global\_ocean\_windstress\_conversion\_factor: Factor to convert input windstress to $N$ $m^{-1}$}
\end{longtable}
\end{center}
\subsection[config\_global\_ocean\_ecosys\_file]{\hyperref[sec:nm_tab_global_ocean]{config\_global\_ocean\_ecosys\_file}}
\label{subsec:nm_sec_config_global_ocean_ecosys_file}
\begin{center}
\begin{longtable}{| p{2.0in} || p{4.0in} |}
    \hline
    Type: & character \\
    \hline
    Units: & \si{unitless} \\
    \hline
    Default Value: & unknown \\
    \hline
    Possible Values: & ecosys.nc \\
    \hline
    \caption{config\_global\_ocean\_ecosys\_file: Name of file containing global values of ecosys variables}
\end{longtable}
\end{center}
\subsection[config\_global\_ocean\_ecosys\_forcing\_file]{\hyperref[sec:nm_tab_global_ocean]{config\_global\_ocean\_ecosys\_forcing\_file}}
\label{subsec:nm_sec_config_global_ocean_ecosys_forcing_file}
\begin{center}
\begin{longtable}{| p{2.0in} || p{4.0in} |}
    \hline
    Type: & character \\
    \hline
    Units: & \si{unitless} \\
    \hline
    Default Value: & unknown \\
    \hline
    Possible Values: & ecosys\_forcing.nc \\
    \hline
    \caption{config\_global\_ocean\_ecosys\_forcing\_file: Name of file containing global values of ecosys forcing fields}
\end{longtable}
\end{center}
\subsection[config\_global\_ocean\_ecosys\_nlat\_dimname]{\hyperref[sec:nm_tab_global_ocean]{config\_global\_ocean\_ecosys\_nlat\_dimname}}
\label{subsec:nm_sec_config_global_ocean_ecosys_nlat_dimname}
\begin{center}
\begin{longtable}{| p{2.0in} || p{4.0in} |}
    \hline
    Type: & character \\
    \hline
    Units: & \si{unitless} \\
    \hline
    Default Value: & none \\
    \hline
    Possible Values: & Dim name from input files. \\
    \hline
    \caption{config\_global\_ocean\_ecosys\_nlat\_dimname: Name of the dimension that determines number of latitude lines in ecosys initial condition files.}
\end{longtable}
\end{center}
\subsection[config\_global\_ocean\_ecosys\_nlon\_dimname]{\hyperref[sec:nm_tab_global_ocean]{config\_global\_ocean\_ecosys\_nlon\_dimname}}
\label{subsec:nm_sec_config_global_ocean_ecosys_nlon_dimname}
\begin{center}
\begin{longtable}{| p{2.0in} || p{4.0in} |}
    \hline
    Type: & character \\
    \hline
    Units: & \si{unitless} \\
    \hline
    Default Value: & none \\
    \hline
    Possible Values: & Dim name from input files. \\
    \hline
    \caption{config\_global\_ocean\_ecosys\_nlon\_dimname: Name of the dimension that determines number of longitude lines in ecosys initial condition files.}
\end{longtable}
\end{center}
\subsection[config\_global\_ocean\_ecosys\_ndepth\_dimname]{\hyperref[sec:nm_tab_global_ocean]{config\_global\_ocean\_ecosys\_ndepth\_dimname}}
\label{subsec:nm_sec_config_global_ocean_ecosys_ndepth_dimname}
\begin{center}
\begin{longtable}{| p{2.0in} || p{4.0in} |}
    \hline
    Type: & character \\
    \hline
    Units: & \si{unitless} \\
    \hline
    Default Value: & none \\
    \hline
    Possible Values: & Dim name from input files. \\
    \hline
    \caption{config\_global\_ocean\_ecosys\_ndepth\_dimname: Name of the dimension that determines number of vertical levels in ecosys initial condition files.}
\end{longtable}
\end{center}
\subsection[config\_global\_ocean\_ecosys\_depth\_conversion\_factor]{\hyperref[sec:nm_tab_global_ocean]{config\_global\_ocean\_ecosys\_depth\_conversion\_factor}}
\label{subsec:nm_sec_config_global_ocean_ecosys_depth_conversion_factor}
\begin{center}
\begin{longtable}{| p{2.0in} || p{4.0in} |}
    \hline
    Type: & real \\
    \hline
    Units: & \si{variable} \\
    \hline
    Default Value: & 1.0 \\
    \hline
    Possible Values: & Any positive real value greater than 0. \\
    \hline
    \caption{config\_global\_ocean\_ecosys\_depth\_conversion\_factor: Conversion factor for ecosys initial condition depth levels. Should convert units on input depth levels to meters.}
\end{longtable}
\end{center}
\subsection[config\_global\_ocean\_ecosys\_vert\_levels]{\hyperref[sec:nm_tab_global_ocean]{config\_global\_ocean\_ecosys\_vert\_levels}}
\label{subsec:nm_sec_config_global_ocean_ecosys_vert_levels}
\begin{center}
\begin{longtable}{| p{2.0in} || p{4.0in} |}
    \hline
    Type: & integer \\
    \hline
    Units: & \si{unitless} \\
    \hline
    Default Value: & -1 \\
    \hline
    Possible Values: & Any positive non-zero integer. A value of -1 causes this to be overwritten with the configurations vertical level definition. \\
    \hline
    \caption{config\_global\_ocean\_ecosys\_vert\_levels: Number of vertical levels in ecosys initial condition file.  Set to -1 to read from file with config\_global\_ocean\_ecosys\_ndepth\_dimname}
\end{longtable}
\end{center}
\subsection[config\_global\_ocean\_ecosys\_lat\_varname]{\hyperref[sec:nm_tab_global_ocean]{config\_global\_ocean\_ecosys\_lat\_varname}}
\label{subsec:nm_sec_config_global_ocean_ecosys_lat_varname}
\begin{center}
\begin{longtable}{| p{2.0in} || p{4.0in} |}
    \hline
    Type: & character \\
    \hline
    Units: & \si{unitless} \\
    \hline
    Default Value: & none \\
    \hline
    Possible Values: & Variable name from input file. \\
    \hline
    \caption{config\_global\_ocean\_ecosys\_lat\_varname: Name of the variable containing latitude coordinates for ecosys values in ecosys initial condition file.}
\end{longtable}
\end{center}
\subsection[config\_global\_ocean\_ecosys\_lon\_varname]{\hyperref[sec:nm_tab_global_ocean]{config\_global\_ocean\_ecosys\_lon\_varname}}
\label{subsec:nm_sec_config_global_ocean_ecosys_lon_varname}
\begin{center}
\begin{longtable}{| p{2.0in} || p{4.0in} |}
    \hline
    Type: & character \\
    \hline
    Units: & \si{unitless} \\
    \hline
    Default Value: & none \\
    \hline
    Possible Values: & Variable name from input file. \\
    \hline
    \caption{config\_global\_ocean\_ecosys\_lon\_varname: Name of the variable containing longitude coordinates for ecosys values in ecosys initial condition file.}
\end{longtable}
\end{center}
\subsection[config\_global\_ocean\_ecosys\_depth\_varname]{\hyperref[sec:nm_tab_global_ocean]{config\_global\_ocean\_ecosys\_depth\_varname}}
\label{subsec:nm_sec_config_global_ocean_ecosys_depth_varname}
\begin{center}
\begin{longtable}{| p{2.0in} || p{4.0in} |}
    \hline
    Type: & character \\
    \hline
    Units: & \si{unitless} \\
    \hline
    Default Value: & none \\
    \hline
    Possible Values: & Variable name from input file. \\
    \hline
    \caption{config\_global\_ocean\_ecosys\_depth\_varname: Name of the variable containing depth coordinates for ecosys values in ecosys initial condition file.}
\end{longtable}
\end{center}
\subsection[config\_global\_ocean\_ecosys\_latlon\_degrees]{\hyperref[sec:nm_tab_global_ocean]{config\_global\_ocean\_ecosys\_latlon\_degrees}}
\label{subsec:nm_sec_config_global_ocean_ecosys_latlon_degrees}
\begin{center}
\begin{longtable}{| p{2.0in} || p{4.0in} |}
    \hline
    Type: & logical \\
    \hline
    Units: & \si{unitless} \\
    \hline
    Default Value: & .true. \\
    \hline
    Possible Values: & .true. or .false. \\
    \hline
    \caption{config\_global\_ocean\_ecosys\_latlon\_degrees: Logical flag that controls if the Lat/Lon fields for ecosys should be converted to radians. True means input is degrees, false means input is radians.}
\end{longtable}
\end{center}
\subsection[config\_global\_ocean\_ecosys\_method]{\hyperref[sec:nm_tab_global_ocean]{config\_global\_ocean\_ecosys\_method}}
\label{subsec:nm_sec_config_global_ocean_ecosys_method}
\begin{center}
\begin{longtable}{| p{2.0in} || p{4.0in} |}
    \hline
    Type: & character \\
    \hline
    Units: & \si{unitless} \\
    \hline
    Default Value: & bilinear\_interpolation \\
    \hline
    Possible Values: & bilinear\_interpolation, nearest\_neighbor \\
    \hline
    \caption{config\_global\_ocean\_ecosys\_method: Method to interpolate shortwave data to MPAS mesh.}
\end{longtable}
\end{center}
\subsection[config\_global\_ocean\_ecosys\_forcing\_time\_dimname]{\hyperref[sec:nm_tab_global_ocean]{config\_global\_ocean\_ecosys\_forcing\_time\_dimname}}
\label{subsec:nm_sec_config_global_ocean_ecosys_forcing_time_dimname}
\begin{center}
\begin{longtable}{| p{2.0in} || p{4.0in} |}
    \hline
    Type: & character \\
    \hline
    Units: & \si{unitless} \\
    \hline
    Default Value: & none \\
    \hline
    Possible Values: & Dim name from input files. \\
    \hline
    \caption{config\_global\_ocean\_ecosys\_forcing\_time\_dimname: Name of the dimension that determines the times in ecosys forcing files.}
\end{longtable}
\end{center}
\subsection[config\_global\_ocean\_smooth\_ecosys\_iterations]{\hyperref[sec:nm_tab_global_ocean]{config\_global\_ocean\_smooth\_ecosys\_iterations}}
\label{subsec:nm_sec_config_global_ocean_smooth_ecosys_iterations}
\begin{center}
\begin{longtable}{| p{2.0in} || p{4.0in} |}
    \hline
    Type: & integer \\
    \hline
    Units: & \si{unitless} \\
    \hline
    Default Value: & 0 \\
    \hline
    Possible Values: & Any positive integer value greater or equal to 0. \\
    \hline
    \caption{config\_global\_ocean\_smooth\_ecosys\_iterations: Number of smoothing iterations on ecosystem variables.}
\end{longtable}
\end{center}
\section[cvmix\_WSwSBF]{\hyperref[sec:nm_tab_cvmix_WSwSBF]{cvmix\_WSwSBF}}
\label{sec:nm_sec_cvmix_WSwSBF}
\subsection[config\_cvmix\_WSwSBF\_vert\_levels]{\hyperref[sec:nm_tab_cvmix_WSwSBF]{config\_cvmix\_WSwSBF\_vert\_levels}}
\label{subsec:nm_sec_config_cvmix_WSwSBF_vert_levels}
\begin{center}
\begin{longtable}{| p{2.0in} || p{4.0in} |}
    \hline
    Type: & integer \\
    \hline
    Units: & \si{unitless} \\
    \hline
    Default Value: & 100 \\
    \hline
    Possible Values: & Any positive integer number greater than 0. \\
    \hline
    \caption{config\_cvmix\_WSwSBF\_vert\_levels: Number of vertical levels in cvmix WSwSBF unit test case.}
\end{longtable}
\end{center}
\subsection[config\_cvmix\_WSwSBF\_surface\_temperature]{\hyperref[sec:nm_tab_cvmix_WSwSBF]{config\_cvmix\_WSwSBF\_surface\_temperature}}
\label{subsec:nm_sec_config_cvmix_WSwSBF_surface_temperature}
\begin{center}
\begin{longtable}{| p{2.0in} || p{4.0in} |}
    \hline
    Type: & real \\
    \hline
    Units: & \si{deg.C} \\
    \hline
    Default Value: & 15.0 \\
    \hline
    Possible Values: & Any real number \\
    \hline
    \caption{config\_cvmix\_WSwSBF\_surface\_temperature: Temperature of the surface of the ocean.}
\end{longtable}
\end{center}
\subsection[config\_cvmix\_WSwSBF\_surface\_salinity]{\hyperref[sec:nm_tab_cvmix_WSwSBF]{config\_cvmix\_WSwSBF\_surface\_salinity}}
\label{subsec:nm_sec_config_cvmix_WSwSBF_surface_salinity}
\begin{center}
\begin{longtable}{| p{2.0in} || p{4.0in} |}
    \hline
    Type: & real \\
    \hline
    Units: & \si{PSU} \\
    \hline
    Default Value: & 35.0 \\
    \hline
    Possible Values: & Any real number \\
    \hline
    \caption{config\_cvmix\_WSwSBF\_surface\_salinity: Salinity of the surface of the ocean.}
\end{longtable}
\end{center}
\subsection[config\_cvmix\_WSwSBF\_surface\_restoring\_temperature]{\hyperref[sec:nm_tab_cvmix_WSwSBF]{config\_cvmix\_WSwSBF\_surface\_restoring\_temperature}}
\label{subsec:nm_sec_config_cvmix_WSwSBF_surface_restoring_temperature}
\begin{center}
\begin{longtable}{| p{2.0in} || p{4.0in} |}
    \hline
    Type: & real \\
    \hline
    Units: & \si{C} \\
    \hline
    Default Value: & 15.0 \\
    \hline
    Possible Values: & Any real number \\
    \hline
    \caption{config\_cvmix\_WSwSBF\_surface\_restoring\_temperature: Temperature to restore towards when surface restoring is turned on.}
\end{longtable}
\end{center}
\subsection[config\_cvmix\_WSwSBF\_surface\_restoring\_salinity]{\hyperref[sec:nm_tab_cvmix_WSwSBF]{config\_cvmix\_WSwSBF\_surface\_restoring\_salinity}}
\label{subsec:nm_sec_config_cvmix_WSwSBF_surface_restoring_salinity}
\begin{center}
\begin{longtable}{| p{2.0in} || p{4.0in} |}
    \hline
    Type: & real \\
    \hline
    Units: & \si{PSU} \\
    \hline
    Default Value: & 35.0 \\
    \hline
    Possible Values: & Any real number \\
    \hline
    \caption{config\_cvmix\_WSwSBF\_surface\_restoring\_salinity: Salinity to restore towards when surface restoring is turned on.}
\end{longtable}
\end{center}
\subsection[config\_cvmix\_WSwSBF\_temperature\_piston\_velocity]{\hyperref[sec:nm_tab_cvmix_WSwSBF]{config\_cvmix\_WSwSBF\_temperature\_piston\_velocity}}
\label{subsec:nm_sec_config_cvmix_WSwSBF_temperature_piston_velocity}
\begin{center}
\begin{longtable}{| p{2.0in} || p{4.0in} |}
    \hline
    Type: & real \\
    \hline
    Units: & \si{m.s^{-1}} \\
    \hline
    Default Value: & 4.0e-6 \\
    \hline
    Possible Values: & Any non-negative real number \\
    \hline
    \caption{config\_cvmix\_WSwSBF\_temperature\_piston\_velocity: Piston velocity to control rate of restoring toward config\_cvmix\_WSwSBF\_surface\_restoring\_temperature.}
\end{longtable}
\end{center}
\subsection[config\_cvmix\_WSwSBF\_salinity\_piston\_velocity]{\hyperref[sec:nm_tab_cvmix_WSwSBF]{config\_cvmix\_WSwSBF\_salinity\_piston\_velocity}}
\label{subsec:nm_sec_config_cvmix_WSwSBF_salinity_piston_velocity}
\begin{center}
\begin{longtable}{| p{2.0in} || p{4.0in} |}
    \hline
    Type: & real \\
    \hline
    Units: & \si{m.s^{-1}} \\
    \hline
    Default Value: & 4.0e-6 \\
    \hline
    Possible Values: & Any non-negative real number \\
    \hline
    \caption{config\_cvmix\_WSwSBF\_salinity\_piston\_velocity: Piston velocity to control rate of restoring toward config\_cvmix\_WSwSBF\_surface\_restoring\_salinity.}
\end{longtable}
\end{center}
\subsection[config\_cvmix\_WSwSBF\_sensible\_heat\_flux]{\hyperref[sec:nm_tab_cvmix_WSwSBF]{config\_cvmix\_WSwSBF\_sensible\_heat\_flux}}
\label{subsec:nm_sec_config_cvmix_WSwSBF_sensible_heat_flux}
\begin{center}
\begin{longtable}{| p{2.0in} || p{4.0in} |}
    \hline
    Type: & real \\
    \hline
    Units: & \si{W.m^{-2}} \\
    \hline
    Default Value: & 0.0 \\
    \hline
    Possible Values: & Any real number \\
    \hline
    \caption{config\_cvmix\_WSwSBF\_sensible\_heat\_flux: Net sensible heat flux applied when bulk forcing is used. Positive values indicate a net input of heat to ocean.}
\end{longtable}
\end{center}
\subsection[config\_cvmix\_WSwSBF\_latent\_heat\_flux]{\hyperref[sec:nm_tab_cvmix_WSwSBF]{config\_cvmix\_WSwSBF\_latent\_heat\_flux}}
\label{subsec:nm_sec_config_cvmix_WSwSBF_latent_heat_flux}
\begin{center}
\begin{longtable}{| p{2.0in} || p{4.0in} |}
    \hline
    Type: & real \\
    \hline
    Units: & \si{W.m^{-2}} \\
    \hline
    Default Value: & 0.0 \\
    \hline
    Possible Values: & Any real number \\
    \hline
    \caption{config\_cvmix\_WSwSBF\_latent\_heat\_flux: Net latent heat flux applied when bulk forcing is used. Positive values indicate a net input of heat to ocean.}
\end{longtable}
\end{center}
\subsection[config\_cvmix\_WSwSBF\_shortwave\_heat\_flux]{\hyperref[sec:nm_tab_cvmix_WSwSBF]{config\_cvmix\_WSwSBF\_shortwave\_heat\_flux}}
\label{subsec:nm_sec_config_cvmix_WSwSBF_shortwave_heat_flux}
\begin{center}
\begin{longtable}{| p{2.0in} || p{4.0in} |}
    \hline
    Type: & real \\
    \hline
    Units: & \si{W.m^{-2}} \\
    \hline
    Default Value: & 0.0 \\
    \hline
    Possible Values: & Any real number \\
    \hline
    \caption{config\_cvmix\_WSwSBF\_shortwave\_heat\_flux: Net solar shortwave heat flux applied when bulk forcing is used. Positive values indicate a net input of heat to ocean.}
\end{longtable}
\end{center}
\subsection[config\_cvmix\_WSwSBF\_rain\_flux]{\hyperref[sec:nm_tab_cvmix_WSwSBF]{config\_cvmix\_WSwSBF\_rain\_flux}}
\label{subsec:nm_sec_config_cvmix_WSwSBF_rain_flux}
\begin{center}
\begin{longtable}{| p{2.0in} || p{4.0in} |}
    \hline
    Type: & real \\
    \hline
    Units: & \si{kg.m^{-2}.s^{-1}} \\
    \hline
    Default Value: & 0.0 \\
    \hline
    Possible Values: & Any real number \\
    \hline
    \caption{config\_cvmix\_WSwSBF\_rain\_flux: Net surface rain flux when bulk forcing is used. Positive values indicate a net input of water to ocean.}
\end{longtable}
\end{center}
\subsection[config\_cvmix\_WSwSBF\_evaporation\_flux]{\hyperref[sec:nm_tab_cvmix_WSwSBF]{config\_cvmix\_WSwSBF\_evaporation\_flux}}
\label{subsec:nm_sec_config_cvmix_WSwSBF_evaporation_flux}
\begin{center}
\begin{longtable}{| p{2.0in} || p{4.0in} |}
    \hline
    Type: & real \\
    \hline
    Units: & \si{kg.m^{-2}.s^{-1}} \\
    \hline
    Default Value: & 0.0 \\
    \hline
    Possible Values: & Any real number \\
    \hline
    \caption{config\_cvmix\_WSwSBF\_evaporation\_flux: Net surface evaporation when bulk forcing is used. Positive values indicate a net input of water to ocean.}
\end{longtable}
\end{center}
\subsection[config\_cvmix\_WSwSBF\_interior\_temperature\_restoring\_rate]{\hyperref[sec:nm_tab_cvmix_WSwSBF]{config\_cvmix\_WSwSBF\_interior\_temperature\_restoring\_rate}}
\label{subsec:nm_sec_config_cvmix_WSwSBF_interior_temperature_restoring_rate}
\begin{center}
\begin{longtable}{| p{2.0in} || p{4.0in} |}
    \hline
    Type: & real \\
    \hline
    Units: & \si{{s}^-1} \\
    \hline
    Default Value: & 1.0e-6 \\
    \hline
    Possible Values: & Any non-negative real number \\
    \hline
    \caption{config\_cvmix\_WSwSBF\_interior\_temperature\_restoring\_rate: Rate at which temperature is restored toward the initial condition.}
\end{longtable}
\end{center}
\subsection[config\_cvmix\_WSwSBF\_interior\_salinity\_restoring\_rate]{\hyperref[sec:nm_tab_cvmix_WSwSBF]{config\_cvmix\_WSwSBF\_interior\_salinity\_restoring\_rate}}
\label{subsec:nm_sec_config_cvmix_WSwSBF_interior_salinity_restoring_rate}
\begin{center}
\begin{longtable}{| p{2.0in} || p{4.0in} |}
    \hline
    Type: & real \\
    \hline
    Units: & \si{{s}^-2} \\
    \hline
    Default Value: & 1.0e-6 \\
    \hline
    Possible Values: & Any non-negative real number \\
    \hline
    \caption{config\_cvmix\_WSwSBF\_interior\_salinity\_restoring\_rate: Rate at which salinity is restored toward the initial condition.}
\end{longtable}
\end{center}
\subsection[config\_cvmix\_WSwSBF\_temperature\_gradient]{\hyperref[sec:nm_tab_cvmix_WSwSBF]{config\_cvmix\_WSwSBF\_temperature\_gradient}}
\label{subsec:nm_sec_config_cvmix_WSwSBF_temperature_gradient}
\begin{center}
\begin{longtable}{| p{2.0in} || p{4.0in} |}
    \hline
    Type: & real \\
    \hline
    Units: & \si{deg.C.m^{-1}} \\
    \hline
    Default Value: & 0.01 \\
    \hline
    Possible Values: & Any real number \\
    \hline
    \caption{config\_cvmix\_WSwSBF\_temperature\_gradient: d/dz of temperature.}
\end{longtable}
\end{center}
\subsection[config\_cvmix\_WSwSBF\_salinity\_gradient]{\hyperref[sec:nm_tab_cvmix_WSwSBF]{config\_cvmix\_WSwSBF\_salinity\_gradient}}
\label{subsec:nm_sec_config_cvmix_WSwSBF_salinity_gradient}
\begin{center}
\begin{longtable}{| p{2.0in} || p{4.0in} |}
    \hline
    Type: & real \\
    \hline
    Units: & \si{PSU.m^{-1}} \\
    \hline
    Default Value: & 0.0 \\
    \hline
    Possible Values: & Any real number \\
    \hline
    \caption{config\_cvmix\_WSwSBF\_salinity\_gradient: d/dz of salinity.}
\end{longtable}
\end{center}
\subsection[config\_cvmix\_WSwSBF\_temperature\_gradient\_mixed\_layer]{\hyperref[sec:nm_tab_cvmix_WSwSBF]{config\_cvmix\_WSwSBF\_temperature\_gradient\_mixed\_layer}}
\label{subsec:nm_sec_config_cvmix_WSwSBF_temperature_gradient_mixed_layer}
\begin{center}
\begin{longtable}{| p{2.0in} || p{4.0in} |}
    \hline
    Type: & real \\
    \hline
    Units: & \si{deg.C.m^{-1}} \\
    \hline
    Default Value: & 0.0 \\
    \hline
    Possible Values: & Any real number \\
    \hline
    \caption{config\_cvmix\_WSwSBF\_temperature\_gradient\_mixed\_layer: d/dz of temperature in mixed temperature layer}
\end{longtable}
\end{center}
\subsection[config\_cvmix\_WSwSBF\_salinity\_gradient\_mixed\_layer]{\hyperref[sec:nm_tab_cvmix_WSwSBF]{config\_cvmix\_WSwSBF\_salinity\_gradient\_mixed\_layer}}
\label{subsec:nm_sec_config_cvmix_WSwSBF_salinity_gradient_mixed_layer}
\begin{center}
\begin{longtable}{| p{2.0in} || p{4.0in} |}
    \hline
    Type: & real \\
    \hline
    Units: & \si{PSU.m^{-1}} \\
    \hline
    Default Value: & 0.0 \\
    \hline
    Possible Values: & Any real number \\
    \hline
    \caption{config\_cvmix\_WSwSBF\_salinity\_gradient\_mixed\_layer: d/dz of salinity in mixed salinity layer}
\end{longtable}
\end{center}
\subsection[config\_cvmix\_WSwSBF\_mixed\_layer\_depth\_temperature]{\hyperref[sec:nm_tab_cvmix_WSwSBF]{config\_cvmix\_WSwSBF\_mixed\_layer\_depth\_temperature}}
\label{subsec:nm_sec_config_cvmix_WSwSBF_mixed_layer_depth_temperature}
\begin{center}
\begin{longtable}{| p{2.0in} || p{4.0in} |}
    \hline
    Type: & real \\
    \hline
    Units: & \si{m} \\
    \hline
    Default Value: & 0.0 \\
    \hline
    Possible Values: & Any positive real number but must be less than bottom depth \\
    \hline
    \caption{config\_cvmix\_WSwSBF\_mixed\_layer\_depth\_temperature: depth mixed temperature layer}
\end{longtable}
\end{center}
\subsection[config\_cvmix\_WSwSBF\_mixed\_layer\_depth\_salinity]{\hyperref[sec:nm_tab_cvmix_WSwSBF]{config\_cvmix\_WSwSBF\_mixed\_layer\_depth\_salinity}}
\label{subsec:nm_sec_config_cvmix_WSwSBF_mixed_layer_depth_salinity}
\begin{center}
\begin{longtable}{| p{2.0in} || p{4.0in} |}
    \hline
    Type: & real \\
    \hline
    Units: & \si{m} \\
    \hline
    Default Value: & 0.0 \\
    \hline
    Possible Values: & Any positive real number but less than bottom depth \\
    \hline
    \caption{config\_cvmix\_WSwSBF\_mixed\_layer\_depth\_salinity: depth mixed salinity layer}
\end{longtable}
\end{center}
\subsection[config\_cvmix\_WSwSBF\_mixed\_layer\_temperature\_change]{\hyperref[sec:nm_tab_cvmix_WSwSBF]{config\_cvmix\_WSwSBF\_mixed\_layer\_temperature\_change}}
\label{subsec:nm_sec_config_cvmix_WSwSBF_mixed_layer_temperature_change}
\begin{center}
\begin{longtable}{| p{2.0in} || p{4.0in} |}
    \hline
    Type: & real \\
    \hline
    Units: & \si{deg.C} \\
    \hline
    Default Value: & 0.0 \\
    \hline
    Possible Values: & Any real number \\
    \hline
    \caption{config\_cvmix\_WSwSBF\_mixed\_layer\_temperature\_change: temperature jump when moving downward across the mixed layer interface}
\end{longtable}
\end{center}
\subsection[config\_cvmix\_WSwSBF\_mixed\_layer\_salinity\_change]{\hyperref[sec:nm_tab_cvmix_WSwSBF]{config\_cvmix\_WSwSBF\_mixed\_layer\_salinity\_change}}
\label{subsec:nm_sec_config_cvmix_WSwSBF_mixed_layer_salinity_change}
\begin{center}
\begin{longtable}{| p{2.0in} || p{4.0in} |}
    \hline
    Type: & real \\
    \hline
    Units: & \si{PSU} \\
    \hline
    Default Value: & 0.0 \\
    \hline
    Possible Values: & Any real number \\
    \hline
    \caption{config\_cvmix\_WSwSBF\_mixed\_layer\_salinity\_change: salinity jump when moving downward across the mixed layer interface}
\end{longtable}
\end{center}
\subsection[config\_cvmix\_WSwSBF\_vertical\_grid]{\hyperref[sec:nm_tab_cvmix_WSwSBF]{config\_cvmix\_WSwSBF\_vertical\_grid}}
\label{subsec:nm_sec_config_cvmix_WSwSBF_vertical_grid}
\begin{center}
\begin{longtable}{| p{2.0in} || p{4.0in} |}
    \hline
    Type: & character \\
    \hline
    Units: & \si{unitless} \\
    \hline
    Default Value: & uniform \\
    \hline
    Possible Values: & 'uniform' and 'stretched100' \\
    \hline
    \caption{config\_cvmix\_WSwSBF\_vertical\_grid: prescription of setting the vertical resolution of the test case}
\end{longtable}
\end{center}
\subsection[config\_cvmix\_WSwSBF\_bottom\_depth]{\hyperref[sec:nm_tab_cvmix_WSwSBF]{config\_cvmix\_WSwSBF\_bottom\_depth}}
\label{subsec:nm_sec_config_cvmix_WSwSBF_bottom_depth}
\begin{center}
\begin{longtable}{| p{2.0in} || p{4.0in} |}
    \hline
    Type: & real \\
    \hline
    Units: & \si{m} \\
    \hline
    Default Value: & 400.0 \\
    \hline
    Possible Values: & Any positive real number. \\
    \hline
    \caption{config\_cvmix\_WSwSBF\_bottom\_depth: Depth of the bottom of the ocean for the CVMix WSwSBF unit test case.}
\end{longtable}
\end{center}
\subsection[config\_cvmix\_WSwSBF\_max\_windstress]{\hyperref[sec:nm_tab_cvmix_WSwSBF]{config\_cvmix\_WSwSBF\_max\_windstress}}
\label{subsec:nm_sec_config_cvmix_WSwSBF_max_windstress}
\begin{center}
\begin{longtable}{| p{2.0in} || p{4.0in} |}
    \hline
    Type: & real \\
    \hline
    Units: & \si{N.m^{-2}} \\
    \hline
    Default Value: & 0.10 \\
    \hline
    Possible Values: & Any real number. \\
    \hline
    \caption{config\_cvmix\_WSwSBF\_max\_windstress: Maximum surface windstress over the domain.}
\end{longtable}
\end{center}
\subsection[config\_cvmix\_WSwSBF\_coriolis\_parameter]{\hyperref[sec:nm_tab_cvmix_WSwSBF]{config\_cvmix\_WSwSBF\_coriolis\_parameter}}
\label{subsec:nm_sec_config_cvmix_WSwSBF_coriolis_parameter}
\begin{center}
\begin{longtable}{| p{2.0in} || p{4.0in} |}
    \hline
    Type: & real \\
    \hline
    Units: & \si{s^{-1}} \\
    \hline
    Default Value: & 1.0e-4 \\
    \hline
    Possible Values: & Any real number. \\
    \hline
    \caption{config\_cvmix\_WSwSBF\_coriolis\_parameter: Coriolis parameter for WSwSBF test case}
\end{longtable}
\end{center}
\section[iso]{\hyperref[sec:nm_tab_iso]{iso}}
\label{sec:nm_sec_iso}
\subsection[config\_iso\_vert\_levels]{\hyperref[sec:nm_tab_iso]{config\_iso\_vert\_levels}}
\label{subsec:nm_sec_config_iso_vert_levels}
\begin{center}
\begin{longtable}{| p{2.0in} || p{4.0in} |}
    \hline
    Type: & integer \\
    \hline
    Units: & \si{unitless} \\
    \hline
    Default Value: & 100 \\
    \hline
    Possible Values: & Any positive integer greater than 0. \\
    \hline
    \caption{config\_iso\_vert\_levels: Number of vertical levels in ISO.}
\end{longtable}
\end{center}
\subsection[config\_iso\_main\_channel\_depth]{\hyperref[sec:nm_tab_iso]{config\_iso\_main\_channel\_depth}}
\label{subsec:nm_sec_config_iso_main_channel_depth}
\begin{center}
\begin{longtable}{| p{2.0in} || p{4.0in} |}
    \hline
    Type: & real \\
    \hline
    Units: & \si{m} \\
    \hline
    Default Value: & 4000.0 \\
    \hline
    Possible Values: & Any positive real number. \\
    \hline
    \caption{config\_iso\_main\_channel\_depth: Depth of the main channel in the ISO.}
\end{longtable}
\end{center}
\subsection[config\_iso\_north\_wall\_lat]{\hyperref[sec:nm_tab_iso]{config\_iso\_north\_wall\_lat}}
\label{subsec:nm_sec_config_iso_north_wall_lat}
\begin{center}
\begin{longtable}{| p{2.0in} || p{4.0in} |}
    \hline
    Type: & real \\
    \hline
    Units: & \si{degrees} \\
    \hline
    Default Value: & -50 \\
    \hline
    Possible Values: & Any real number. \\
    \hline
    \caption{config\_iso\_north\_wall\_lat: Latitude of the vertical north wall in the ISO domain.}
\end{longtable}
\end{center}
\subsection[config\_iso\_south\_wall\_lat]{\hyperref[sec:nm_tab_iso]{config\_iso\_south\_wall\_lat}}
\label{subsec:nm_sec_config_iso_south_wall_lat}
\begin{center}
\begin{longtable}{| p{2.0in} || p{4.0in} |}
    \hline
    Type: & real \\
    \hline
    Units: & \si{degrees} \\
    \hline
    Default Value: & -70 \\
    \hline
    Possible Values: & Any real number. \\
    \hline
    \caption{config\_iso\_south\_wall\_lat: Latitude of the top of the main channel south wall wall in the ISO domain.}
\end{longtable}
\end{center}
\subsection[config\_iso\_ridge\_flag]{\hyperref[sec:nm_tab_iso]{config\_iso\_ridge\_flag}}
\label{subsec:nm_sec_config_iso_ridge_flag}
\begin{center}
\begin{longtable}{| p{2.0in} || p{4.0in} |}
    \hline
    Type: & logical \\
    \hline
    Units: & \si{unitless} \\
    \hline
    Default Value: & .true. \\
    \hline
    Possible Values: & .true. or .false. \\
    \hline
    \caption{config\_iso\_ridge\_flag: Logical flag that controls if a ridge is used or not.}
\end{longtable}
\end{center}
\subsection[config\_iso\_ridge\_center\_lon]{\hyperref[sec:nm_tab_iso]{config\_iso\_ridge\_center\_lon}}
\label{subsec:nm_sec_config_iso_ridge_center_lon}
\begin{center}
\begin{longtable}{| p{2.0in} || p{4.0in} |}
    \hline
    Type: & real \\
    \hline
    Units: & \si{degrees} \\
    \hline
    Default Value: & 180 \\
    \hline
    Possible Values: & Any positive real number. \\
    \hline
    \caption{config\_iso\_ridge\_center\_lon: Longitude of the center of the ridge in the ISO.}
\end{longtable}
\end{center}
\subsection[config\_iso\_ridge\_height]{\hyperref[sec:nm_tab_iso]{config\_iso\_ridge\_height}}
\label{subsec:nm_sec_config_iso_ridge_height}
\begin{center}
\begin{longtable}{| p{2.0in} || p{4.0in} |}
    \hline
    Type: & real \\
    \hline
    Units: & \si{m} \\
    \hline
    Default Value: & 2000.0 \\
    \hline
    Possible Values: & Any positive real number. \\
    \hline
    \caption{config\_iso\_ridge\_height: Maximum height of the ridge at the zonal middle of the ISO domain.}
\end{longtable}
\end{center}
\subsection[config\_iso\_ridge\_width]{\hyperref[sec:nm_tab_iso]{config\_iso\_ridge\_width}}
\label{subsec:nm_sec_config_iso_ridge_width}
\begin{center}
\begin{longtable}{| p{2.0in} || p{4.0in} |}
    \hline
    Type: & real \\
    \hline
    Units: & \si{meters} \\
    \hline
    Default Value: & 2000000 \\
    \hline
    Possible Values: & Any positive real number. \\
    \hline
    \caption{config\_iso\_ridge\_width: Width of the ridge at the zonal middle of the ISO domain.}
\end{longtable}
\end{center}
\subsection[config\_iso\_plateau\_flag]{\hyperref[sec:nm_tab_iso]{config\_iso\_plateau\_flag}}
\label{subsec:nm_sec_config_iso_plateau_flag}
\begin{center}
\begin{longtable}{| p{2.0in} || p{4.0in} |}
    \hline
    Type: & logical \\
    \hline
    Units: & \si{unitless} \\
    \hline
    Default Value: & .true. \\
    \hline
    Possible Values: & .true. or .false. \\
    \hline
    \caption{config\_iso\_plateau\_flag: Logical flag that controls if a plateau is used or not.}
\end{longtable}
\end{center}
\subsection[config\_iso\_plateau\_center\_lon]{\hyperref[sec:nm_tab_iso]{config\_iso\_plateau\_center\_lon}}
\label{subsec:nm_sec_config_iso_plateau_center_lon}
\begin{center}
\begin{longtable}{| p{2.0in} || p{4.0in} |}
    \hline
    Type: & real \\
    \hline
    Units: & \si{degrees} \\
    \hline
    Default Value: & 300 \\
    \hline
    Possible Values: & Any positive real number. \\
    \hline
    \caption{config\_iso\_plateau\_center\_lon: Longitude of the center of the plateau in the ISO.}
\end{longtable}
\end{center}
\subsection[config\_iso\_plateau\_center\_lat]{\hyperref[sec:nm_tab_iso]{config\_iso\_plateau\_center\_lat}}
\label{subsec:nm_sec_config_iso_plateau_center_lat}
\begin{center}
\begin{longtable}{| p{2.0in} || p{4.0in} |}
    \hline
    Type: & real \\
    \hline
    Units: & \si{degrees} \\
    \hline
    Default Value: & -58 \\
    \hline
    Possible Values: & Any positive real number. \\
    \hline
    \caption{config\_iso\_plateau\_center\_lat: Latitude of the center of the plateau in the ISO.}
\end{longtable}
\end{center}
\subsection[config\_iso\_plateau\_height]{\hyperref[sec:nm_tab_iso]{config\_iso\_plateau\_height}}
\label{subsec:nm_sec_config_iso_plateau_height}
\begin{center}
\begin{longtable}{| p{2.0in} || p{4.0in} |}
    \hline
    Type: & real \\
    \hline
    Units: & \si{meters} \\
    \hline
    Default Value: & 2000 \\
    \hline
    Possible Values: & Any positive real number. \\
    \hline
    \caption{config\_iso\_plateau\_height: Height of the top of the plateau in the ISO domain.}
\end{longtable}
\end{center}
\subsection[config\_iso\_plateau\_radius]{\hyperref[sec:nm_tab_iso]{config\_iso\_plateau\_radius}}
\label{subsec:nm_sec_config_iso_plateau_radius}
\begin{center}
\begin{longtable}{| p{2.0in} || p{4.0in} |}
    \hline
    Type: & real \\
    \hline
    Units: & \si{meters} \\
    \hline
    Default Value: & 200000 \\
    \hline
    Possible Values: & Any positive real number. \\
    \hline
    \caption{config\_iso\_plateau\_radius: Radius at the top of the plateau in the ISO domain.}
\end{longtable}
\end{center}
\subsection[config\_iso\_plateau\_slope\_width]{\hyperref[sec:nm_tab_iso]{config\_iso\_plateau\_slope\_width}}
\label{subsec:nm_sec_config_iso_plateau_slope_width}
\begin{center}
\begin{longtable}{| p{2.0in} || p{4.0in} |}
    \hline
    Type: & real \\
    \hline
    Units: & \si{meters} \\
    \hline
    Default Value: & 1000000 \\
    \hline
    Possible Values: & Any positive real number. \\
    \hline
    \caption{config\_iso\_plateau\_slope\_width: Width of the sloping region of the plateau in the ISO domain.}
\end{longtable}
\end{center}
\subsection[config\_iso\_shelf\_flag]{\hyperref[sec:nm_tab_iso]{config\_iso\_shelf\_flag}}
\label{subsec:nm_sec_config_iso_shelf_flag}
\begin{center}
\begin{longtable}{| p{2.0in} || p{4.0in} |}
    \hline
    Type: & logical \\
    \hline
    Units: & \si{unitless} \\
    \hline
    Default Value: & .true. \\
    \hline
    Possible Values: & .true. or .false. \\
    \hline
    \caption{config\_iso\_shelf\_flag: Logical flag that controls if a shelf is used or not.}
\end{longtable}
\end{center}
\subsection[config\_iso\_shelf\_depth]{\hyperref[sec:nm_tab_iso]{config\_iso\_shelf\_depth}}
\label{subsec:nm_sec_config_iso_shelf_depth}
\begin{center}
\begin{longtable}{| p{2.0in} || p{4.0in} |}
    \hline
    Type: & real \\
    \hline
    Units: & \si{meters} \\
    \hline
    Default Value: & 500 \\
    \hline
    Possible Values: & Any positive real number. \\
    \hline
    \caption{config\_iso\_shelf\_depth: Depth of the shelf in the ISO.}
\end{longtable}
\end{center}
\subsection[config\_iso\_shelf\_width]{\hyperref[sec:nm_tab_iso]{config\_iso\_shelf\_width}}
\label{subsec:nm_sec_config_iso_shelf_width}
\begin{center}
\begin{longtable}{| p{2.0in} || p{4.0in} |}
    \hline
    Type: & real \\
    \hline
    Units: & \si{meters} \\
    \hline
    Default Value: & 120000 \\
    \hline
    Possible Values: & Any positive real number. \\
    \hline
    \caption{config\_iso\_shelf\_width: Width of the shelf in the ISO.}
\end{longtable}
\end{center}
\subsection[config\_iso\_cont\_slope\_flag]{\hyperref[sec:nm_tab_iso]{config\_iso\_cont\_slope\_flag}}
\label{subsec:nm_sec_config_iso_cont_slope_flag}
\begin{center}
\begin{longtable}{| p{2.0in} || p{4.0in} |}
    \hline
    Type: & logical \\
    \hline
    Units: & \si{unitless} \\
    \hline
    Default Value: & .true. \\
    \hline
    Possible Values: & .true. or .false. \\
    \hline
    \caption{config\_iso\_cont\_slope\_flag: Logical flag that controls if a continental slope is used or not.}
\end{longtable}
\end{center}
\subsection[config\_iso\_max\_cont\_slope]{\hyperref[sec:nm_tab_iso]{config\_iso\_max\_cont\_slope}}
\label{subsec:nm_sec_config_iso_max_cont_slope}
\begin{center}
\begin{longtable}{| p{2.0in} || p{4.0in} |}
    \hline
    Type: & real \\
    \hline
    Units: & \si{-} \\
    \hline
    Default Value: & 0.01 \\
    \hline
    Possible Values: & Any positive real number. \\
    \hline
    \caption{config\_iso\_max\_cont\_slope: Maximum slope of the continental slope in the ISO.}
\end{longtable}
\end{center}
\subsection[config\_iso\_embayment\_flag]{\hyperref[sec:nm_tab_iso]{config\_iso\_embayment\_flag}}
\label{subsec:nm_sec_config_iso_embayment_flag}
\begin{center}
\begin{longtable}{| p{2.0in} || p{4.0in} |}
    \hline
    Type: & logical \\
    \hline
    Units: & \si{unitless} \\
    \hline
    Default Value: & .true. \\
    \hline
    Possible Values: & .true. or .false. \\
    \hline
    \caption{config\_iso\_embayment\_flag: Logical flag that controls if an embayment is used or not.}
\end{longtable}
\end{center}
\subsection[config\_iso\_embayment\_center\_lon]{\hyperref[sec:nm_tab_iso]{config\_iso\_embayment\_center\_lon}}
\label{subsec:nm_sec_config_iso_embayment_center_lon}
\begin{center}
\begin{longtable}{| p{2.0in} || p{4.0in} |}
    \hline
    Type: & real \\
    \hline
    Units: & \si{degrees} \\
    \hline
    Default Value: & 60 \\
    \hline
    Possible Values: & Any positive real number. \\
    \hline
    \caption{config\_iso\_embayment\_center\_lon: Longitude of the center of the embayment in the ISO.}
\end{longtable}
\end{center}
\subsection[config\_iso\_embayment\_center\_lat]{\hyperref[sec:nm_tab_iso]{config\_iso\_embayment\_center\_lat}}
\label{subsec:nm_sec_config_iso_embayment_center_lat}
\begin{center}
\begin{longtable}{| p{2.0in} || p{4.0in} |}
    \hline
    Type: & real \\
    \hline
    Units: & \si{degrees} \\
    \hline
    Default Value: & -71 \\
    \hline
    Possible Values: & Any positive real number. \\
    \hline
    \caption{config\_iso\_embayment\_center\_lat: Latitude of the center of the embayment in the ISO.}
\end{longtable}
\end{center}
\subsection[config\_iso\_embayment\_radius]{\hyperref[sec:nm_tab_iso]{config\_iso\_embayment\_radius}}
\label{subsec:nm_sec_config_iso_embayment_radius}
\begin{center}
\begin{longtable}{| p{2.0in} || p{4.0in} |}
    \hline
    Type: & real \\
    \hline
    Units: & \si{meters} \\
    \hline
    Default Value: & 500000 \\
    \hline
    Possible Values: & Any positive real number. \\
    \hline
    \caption{config\_iso\_embayment\_radius: Radius of the embayment in the ISO.}
\end{longtable}
\end{center}
\subsection[config\_iso\_embayment\_depth]{\hyperref[sec:nm_tab_iso]{config\_iso\_embayment\_depth}}
\label{subsec:nm_sec_config_iso_embayment_depth}
\begin{center}
\begin{longtable}{| p{2.0in} || p{4.0in} |}
    \hline
    Type: & real \\
    \hline
    Units: & \si{meters} \\
    \hline
    Default Value: & 2000 \\
    \hline
    Possible Values: & Any positive real number. \\
    \hline
    \caption{config\_iso\_embayment\_depth: Depth of the embayment in the ISO.}
\end{longtable}
\end{center}
\subsection[config\_iso\_depression\_flag]{\hyperref[sec:nm_tab_iso]{config\_iso\_depression\_flag}}
\label{subsec:nm_sec_config_iso_depression_flag}
\begin{center}
\begin{longtable}{| p{2.0in} || p{4.0in} |}
    \hline
    Type: & logical \\
    \hline
    Units: & \si{unitless} \\
    \hline
    Default Value: & .true. \\
    \hline
    Possible Values: & .true. or .false. \\
    \hline
    \caption{config\_iso\_depression\_flag: Logical flag to add a depresseion between embayment and main channel.}
\end{longtable}
\end{center}
\subsection[config\_iso\_depression\_center\_lon]{\hyperref[sec:nm_tab_iso]{config\_iso\_depression\_center\_lon}}
\label{subsec:nm_sec_config_iso_depression_center_lon}
\begin{center}
\begin{longtable}{| p{2.0in} || p{4.0in} |}
    \hline
    Type: & real \\
    \hline
    Units: & \si{degrees} \\
    \hline
    Default Value: & 60 \\
    \hline
    Possible Values: & Any positive real number. \\
    \hline
    \caption{config\_iso\_depression\_center\_lon: Longitude of the center of the depression in the ISO.}
\end{longtable}
\end{center}
\subsection[config\_iso\_depression\_south\_lat]{\hyperref[sec:nm_tab_iso]{config\_iso\_depression\_south\_lat}}
\label{subsec:nm_sec_config_iso_depression_south_lat}
\begin{center}
\begin{longtable}{| p{2.0in} || p{4.0in} |}
    \hline
    Type: & real \\
    \hline
    Units: & \si{degrees} \\
    \hline
    Default Value: & -72 \\
    \hline
    Possible Values: & Any positive real number. \\
    \hline
    \caption{config\_iso\_depression\_south\_lat: Latitude of the south end of the depression in the ISO.}
\end{longtable}
\end{center}
\subsection[config\_iso\_depression\_north\_lat]{\hyperref[sec:nm_tab_iso]{config\_iso\_depression\_north\_lat}}
\label{subsec:nm_sec_config_iso_depression_north_lat}
\begin{center}
\begin{longtable}{| p{2.0in} || p{4.0in} |}
    \hline
    Type: & real \\
    \hline
    Units: & \si{degrees} \\
    \hline
    Default Value: & -65 \\
    \hline
    Possible Values: & Any positive real number. \\
    \hline
    \caption{config\_iso\_depression\_north\_lat: Latitude of the north end of the depression in the ISO.}
\end{longtable}
\end{center}
\subsection[config\_iso\_depression\_width]{\hyperref[sec:nm_tab_iso]{config\_iso\_depression\_width}}
\label{subsec:nm_sec_config_iso_depression_width}
\begin{center}
\begin{longtable}{| p{2.0in} || p{4.0in} |}
    \hline
    Type: & real \\
    \hline
    Units: & \si{meters} \\
    \hline
    Default Value: & 480000 \\
    \hline
    Possible Values: & Any positive real number. \\
    \hline
    \caption{config\_iso\_depression\_width: Width of the depression in the ISO.}
\end{longtable}
\end{center}
\subsection[config\_iso\_depression\_depth]{\hyperref[sec:nm_tab_iso]{config\_iso\_depression\_depth}}
\label{subsec:nm_sec_config_iso_depression_depth}
\begin{center}
\begin{longtable}{| p{2.0in} || p{4.0in} |}
    \hline
    Type: & real \\
    \hline
    Units: & \si{meters} \\
    \hline
    Default Value: & 800 \\
    \hline
    Possible Values: & Any positive real number. \\
    \hline
    \caption{config\_iso\_depression\_depth: Depth of the depression in the ISO.}
\end{longtable}
\end{center}
\subsection[config\_iso\_salinity]{\hyperref[sec:nm_tab_iso]{config\_iso\_salinity}}
\label{subsec:nm_sec_config_iso_salinity}
\begin{center}
\begin{longtable}{| p{2.0in} || p{4.0in} |}
    \hline
    Type: & real \\
    \hline
    Units: & \si{PSU} \\
    \hline
    Default Value: & 35.0 \\
    \hline
    Possible Values: & Any positive real number. \\
    \hline
    \caption{config\_iso\_salinity: Salinity of the water in the ISO.}
\end{longtable}
\end{center}
\subsection[config\_iso\_wind\_stress\_max]{\hyperref[sec:nm_tab_iso]{config\_iso\_wind\_stress\_max}}
\label{subsec:nm_sec_config_iso_wind_stress_max}
\begin{center}
\begin{longtable}{| p{2.0in} || p{4.0in} |}
    \hline
    Type: & real \\
    \hline
    Units: & \si{N.m^{2}} \\
    \hline
    Default Value: & 0.01 \\
    \hline
    Possible Values: & Any real number. \\
    \hline
    \caption{config\_iso\_wind\_stress\_max: Maximum zonal windstress value.}
\end{longtable}
\end{center}
\subsection[config\_iso\_acc\_wind]{\hyperref[sec:nm_tab_iso]{config\_iso\_acc\_wind}}
\label{subsec:nm_sec_config_iso_acc_wind}
\begin{center}
\begin{longtable}{| p{2.0in} || p{4.0in} |}
    \hline
    Type: & real \\
    \hline
    Units: & \si{N.m^{2}} \\
    \hline
    Default Value: & 0.2 \\
    \hline
    Possible Values: & Any real number. \\
    \hline
    \caption{config\_iso\_acc\_wind: Maximum zonal windstress value over the Antarctic Circumpolar Current.}
\end{longtable}
\end{center}
\subsection[config\_iso\_asf\_wind]{\hyperref[sec:nm_tab_iso]{config\_iso\_asf\_wind}}
\label{subsec:nm_sec_config_iso_asf_wind}
\begin{center}
\begin{longtable}{| p{2.0in} || p{4.0in} |}
    \hline
    Type: & real \\
    \hline
    Units: & \si{N.m^{2}} \\
    \hline
    Default Value: & -0.05 \\
    \hline
    Possible Values: & Any real number. \\
    \hline
    \caption{config\_iso\_asf\_wind: Maximum zonal windstress value over the Antarctic Slope Front.}
\end{longtable}
\end{center}
\subsection[config\_iso\_wind\_trans]{\hyperref[sec:nm_tab_iso]{config\_iso\_wind\_trans}}
\label{subsec:nm_sec_config_iso_wind_trans}
\begin{center}
\begin{longtable}{| p{2.0in} || p{4.0in} |}
    \hline
    Type: & real \\
    \hline
    Units: & \si{degrees} \\
    \hline
    Default Value: & -65 \\
    \hline
    Possible Values: & Any real number. \\
    \hline
    \caption{config\_iso\_wind\_trans: Latitude of the transition region between easterly and westerly winds.}
\end{longtable}
\end{center}
\subsection[config\_iso\_heat\_flux\_south]{\hyperref[sec:nm_tab_iso]{config\_iso\_heat\_flux\_south}}
\label{subsec:nm_sec_config_iso_heat_flux_south}
\begin{center}
\begin{longtable}{| p{2.0in} || p{4.0in} |}
    \hline
    Type: & real \\
    \hline
    Units: & \si{W.m^{-2}} \\
    \hline
    Default Value: & -5 \\
    \hline
    Possible Values: & Any real number. \\
    \hline
    \caption{config\_iso\_heat\_flux\_south: Heat flux into the ocean over the south side of the main channel.}
\end{longtable}
\end{center}
\subsection[config\_iso\_heat\_flux\_middle]{\hyperref[sec:nm_tab_iso]{config\_iso\_heat\_flux\_middle}}
\label{subsec:nm_sec_config_iso_heat_flux_middle}
\begin{center}
\begin{longtable}{| p{2.0in} || p{4.0in} |}
    \hline
    Type: & real \\
    \hline
    Units: & \si{W.m^{-2}} \\
    \hline
    Default Value: & 10 \\
    \hline
    Possible Values: & Any real number. \\
    \hline
    \caption{config\_iso\_heat\_flux\_middle: Heat flux into the ocean over the middle of the main channel.}
\end{longtable}
\end{center}
\subsection[config\_iso\_heat\_flux\_north]{\hyperref[sec:nm_tab_iso]{config\_iso\_heat\_flux\_north}}
\label{subsec:nm_sec_config_iso_heat_flux_north}
\begin{center}
\begin{longtable}{| p{2.0in} || p{4.0in} |}
    \hline
    Type: & real \\
    \hline
    Units: & \si{W.m^{-2}} \\
    \hline
    Default Value: & -5 \\
    \hline
    Possible Values: & Any real number. \\
    \hline
    \caption{config\_iso\_heat\_flux\_north: Heat flux into the ocean over the north side of the main channel.}
\end{longtable}
\end{center}
\subsection[config\_iso\_heat\_flux\_lat\_ss]{\hyperref[sec:nm_tab_iso]{config\_iso\_heat\_flux\_lat\_ss}}
\label{subsec:nm_sec_config_iso_heat_flux_lat_ss}
\begin{center}
\begin{longtable}{| p{2.0in} || p{4.0in} |}
    \hline
    Type: & real \\
    \hline
    Units: & \si{degrees} \\
    \hline
    Default Value: & -70 \\
    \hline
    Possible Values: & Any real number. \\
    \hline
    \caption{config\_iso\_heat\_flux\_lat\_ss: Latitude of southern point of heat flux region on the south.}
\end{longtable}
\end{center}
\subsection[config\_iso\_heat\_flux\_lat\_sm]{\hyperref[sec:nm_tab_iso]{config\_iso\_heat\_flux\_lat\_sm}}
\label{subsec:nm_sec_config_iso_heat_flux_lat_sm}
\begin{center}
\begin{longtable}{| p{2.0in} || p{4.0in} |}
    \hline
    Type: & real \\
    \hline
    Units: & \si{degrees} \\
    \hline
    Default Value: & -65 \\
    \hline
    Possible Values: & Any real number. \\
    \hline
    \caption{config\_iso\_heat\_flux\_lat\_sm: Latitude of transition point between heat flux regions on the south and middle.}
\end{longtable}
\end{center}
\subsection[config\_iso\_heat\_flux\_lat\_mn]{\hyperref[sec:nm_tab_iso]{config\_iso\_heat\_flux\_lat\_mn}}
\label{subsec:nm_sec_config_iso_heat_flux_lat_mn}
\begin{center}
\begin{longtable}{| p{2.0in} || p{4.0in} |}
    \hline
    Type: & real \\
    \hline
    Units: & \si{degrees} \\
    \hline
    Default Value: & -53 \\
    \hline
    Possible Values: & Any real number. \\
    \hline
    \caption{config\_iso\_heat\_flux\_lat\_mn: Latitude of transition point between heat flux regions on the middel and north.}
\end{longtable}
\end{center}
\subsection[config\_iso\_region1\_center\_lon]{\hyperref[sec:nm_tab_iso]{config\_iso\_region1\_center\_lon}}
\label{subsec:nm_sec_config_iso_region1_center_lon}
\begin{center}
\begin{longtable}{| p{2.0in} || p{4.0in} |}
    \hline
    Type: & real \\
    \hline
    Units: & \si{degrees} \\
    \hline
    Default Value: & 60 \\
    \hline
    Possible Values: & Any real number. \\
    \hline
    \caption{config\_iso\_region1\_center\_lon: Longitude of center region 1.}
\end{longtable}
\end{center}
\subsection[config\_iso\_region1\_center\_lat]{\hyperref[sec:nm_tab_iso]{config\_iso\_region1\_center\_lat}}
\label{subsec:nm_sec_config_iso_region1_center_lat}
\begin{center}
\begin{longtable}{| p{2.0in} || p{4.0in} |}
    \hline
    Type: & real \\
    \hline
    Units: & \si{degrees} \\
    \hline
    Default Value: & -75 \\
    \hline
    Possible Values: & Any real number. \\
    \hline
    \caption{config\_iso\_region1\_center\_lat: Latitude of center of region 1.}
\end{longtable}
\end{center}
\subsection[config\_iso\_region2\_center\_lon]{\hyperref[sec:nm_tab_iso]{config\_iso\_region2\_center\_lon}}
\label{subsec:nm_sec_config_iso_region2_center_lon}
\begin{center}
\begin{longtable}{| p{2.0in} || p{4.0in} |}
    \hline
    Type: & real \\
    \hline
    Units: & \si{degrees} \\
    \hline
    Default Value: & 150 \\
    \hline
    Possible Values: & Any real number. \\
    \hline
    \caption{config\_iso\_region2\_center\_lon: Longitude of center of region 2.}
\end{longtable}
\end{center}
\subsection[config\_iso\_region2\_center\_lat]{\hyperref[sec:nm_tab_iso]{config\_iso\_region2\_center\_lat}}
\label{subsec:nm_sec_config_iso_region2_center_lat}
\begin{center}
\begin{longtable}{| p{2.0in} || p{4.0in} |}
    \hline
    Type: & real \\
    \hline
    Units: & \si{degrees} \\
    \hline
    Default Value: & -71 \\
    \hline
    Possible Values: & Any real number. \\
    \hline
    \caption{config\_iso\_region2\_center\_lat: Latitude of center of region 2.}
\end{longtable}
\end{center}
\subsection[config\_iso\_region3\_center\_lon]{\hyperref[sec:nm_tab_iso]{config\_iso\_region3\_center\_lon}}
\label{subsec:nm_sec_config_iso_region3_center_lon}
\begin{center}
\begin{longtable}{| p{2.0in} || p{4.0in} |}
    \hline
    Type: & real \\
    \hline
    Units: & \si{degrees} \\
    \hline
    Default Value: & 240 \\
    \hline
    Possible Values: & Any real number. \\
    \hline
    \caption{config\_iso\_region3\_center\_lon: Longitude of center of region 3.}
\end{longtable}
\end{center}
\subsection[config\_iso\_region3\_center\_lat]{\hyperref[sec:nm_tab_iso]{config\_iso\_region3\_center\_lat}}
\label{subsec:nm_sec_config_iso_region3_center_lat}
\begin{center}
\begin{longtable}{| p{2.0in} || p{4.0in} |}
    \hline
    Type: & real \\
    \hline
    Units: & \si{degrees} \\
    \hline
    Default Value: & -71 \\
    \hline
    Possible Values: & Any real number. \\
    \hline
    \caption{config\_iso\_region3\_center\_lat: Latitude of center of region 3.}
\end{longtable}
\end{center}
\subsection[config\_iso\_region4\_center\_lon]{\hyperref[sec:nm_tab_iso]{config\_iso\_region4\_center\_lon}}
\label{subsec:nm_sec_config_iso_region4_center_lon}
\begin{center}
\begin{longtable}{| p{2.0in} || p{4.0in} |}
    \hline
    Type: & real \\
    \hline
    Units: & \si{degrees} \\
    \hline
    Default Value: & 330 \\
    \hline
    Possible Values: & Any real number. \\
    \hline
    \caption{config\_iso\_region4\_center\_lon: Longitude of center of region 4.}
\end{longtable}
\end{center}
\subsection[config\_iso\_region4\_center\_lat]{\hyperref[sec:nm_tab_iso]{config\_iso\_region4\_center\_lat}}
\label{subsec:nm_sec_config_iso_region4_center_lat}
\begin{center}
\begin{longtable}{| p{2.0in} || p{4.0in} |}
    \hline
    Type: & real \\
    \hline
    Units: & \si{degrees} \\
    \hline
    Default Value: & -71 \\
    \hline
    Possible Values: & Any real number. \\
    \hline
    \caption{config\_iso\_region4\_center\_lat: Latitude of center of region 2.}
\end{longtable}
\end{center}
\subsection[config\_iso\_heat\_flux\_region1\_flag]{\hyperref[sec:nm_tab_iso]{config\_iso\_heat\_flux\_region1\_flag}}
\label{subsec:nm_sec_config_iso_heat_flux_region1_flag}
\begin{center}
\begin{longtable}{| p{2.0in} || p{4.0in} |}
    \hline
    Type: & logical \\
    \hline
    Units: & \si{unitless} \\
    \hline
    Default Value: & false \\
    \hline
    Possible Values: & .true. or .false. \\
    \hline
    \caption{config\_iso\_heat\_flux\_region1\_flag: Logical flag controlling use of heat flux in region 1.}
\end{longtable}
\end{center}
\subsection[config\_iso\_heat\_flux\_region1]{\hyperref[sec:nm_tab_iso]{config\_iso\_heat\_flux\_region1}}
\label{subsec:nm_sec_config_iso_heat_flux_region1}
\begin{center}
\begin{longtable}{| p{2.0in} || p{4.0in} |}
    \hline
    Type: & real \\
    \hline
    Units: & \si{W.m^{-2}} \\
    \hline
    Default Value: & -5 \\
    \hline
    Possible Values: & Any real number. \\
    \hline
    \caption{config\_iso\_heat\_flux\_region1: Heat flux into of the ocean over a localized region 1.}
\end{longtable}
\end{center}
\subsection[config\_iso\_heat\_flux\_region1\_radius]{\hyperref[sec:nm_tab_iso]{config\_iso\_heat\_flux\_region1\_radius}}
\label{subsec:nm_sec_config_iso_heat_flux_region1_radius}
\begin{center}
\begin{longtable}{| p{2.0in} || p{4.0in} |}
    \hline
    Type: & real \\
    \hline
    Units: & \si{meters} \\
    \hline
    Default Value: & 300000 \\
    \hline
    Possible Values: & Any real number. \\
    \hline
    \caption{config\_iso\_heat\_flux\_region1\_radius: Radius of heat flux localized region 1.}
\end{longtable}
\end{center}
\subsection[config\_iso\_heat\_flux\_region2\_flag]{\hyperref[sec:nm_tab_iso]{config\_iso\_heat\_flux\_region2\_flag}}
\label{subsec:nm_sec_config_iso_heat_flux_region2_flag}
\begin{center}
\begin{longtable}{| p{2.0in} || p{4.0in} |}
    \hline
    Type: & logical \\
    \hline
    Units: & \si{unitless} \\
    \hline
    Default Value: & false \\
    \hline
    Possible Values: & .true. or .false. \\
    \hline
    \caption{config\_iso\_heat\_flux\_region2\_flag: Logical flag controlling use of heat flux in region 2.}
\end{longtable}
\end{center}
\subsection[config\_iso\_heat\_flux\_region2]{\hyperref[sec:nm_tab_iso]{config\_iso\_heat\_flux\_region2}}
\label{subsec:nm_sec_config_iso_heat_flux_region2}
\begin{center}
\begin{longtable}{| p{2.0in} || p{4.0in} |}
    \hline
    Type: & real \\
    \hline
    Units: & \si{W.m^{-2}} \\
    \hline
    Default Value: & -5 \\
    \hline
    Possible Values: & Any real number. \\
    \hline
    \caption{config\_iso\_heat\_flux\_region2: Heat flux into of the ocean over localized region 2.}
\end{longtable}
\end{center}
\subsection[config\_iso\_heat\_flux\_region2\_radius]{\hyperref[sec:nm_tab_iso]{config\_iso\_heat\_flux\_region2\_radius}}
\label{subsec:nm_sec_config_iso_heat_flux_region2_radius}
\begin{center}
\begin{longtable}{| p{2.0in} || p{4.0in} |}
    \hline
    Type: & real \\
    \hline
    Units: & \si{meters} \\
    \hline
    Default Value: & 240000 \\
    \hline
    Possible Values: & Any real number. \\
    \hline
    \caption{config\_iso\_heat\_flux\_region2\_radius: Radius of heat flux localized region 2.}
\end{longtable}
\end{center}
\subsection[config\_iso\_surface\_temperature\_piston\_velocity]{\hyperref[sec:nm_tab_iso]{config\_iso\_surface\_temperature\_piston\_velocity}}
\label{subsec:nm_sec_config_iso_surface_temperature_piston_velocity}
\begin{center}
\begin{longtable}{| p{2.0in} || p{4.0in} |}
    \hline
    Type: & real \\
    \hline
    Units: & \si{m/s} \\
    \hline
    Default Value: & 5.787e-5 \\
    \hline
    Possible Values: & Any real number. \\
    \hline
    \caption{config\_iso\_surface\_temperature\_piston\_velocity: Surface temperature restoring piston velocity.}
\end{longtable}
\end{center}
\subsection[config\_iso\_initial\_temp\_t1]{\hyperref[sec:nm_tab_iso]{config\_iso\_initial\_temp\_t1}}
\label{subsec:nm_sec_config_iso_initial_temp_t1}
\begin{center}
\begin{longtable}{| p{2.0in} || p{4.0in} |}
    \hline
    Type: & real \\
    \hline
    Units: & \si{deg.C} \\
    \hline
    Default Value: & 3.5 \\
    \hline
    Possible Values: & Any real number. \\
    \hline
    \caption{config\_iso\_initial\_temp\_t1: Maximum temperature parameter for the initial temperature profile.}
\end{longtable}
\end{center}
\subsection[config\_iso\_initial\_temp\_t2]{\hyperref[sec:nm_tab_iso]{config\_iso\_initial\_temp\_t2}}
\label{subsec:nm_sec_config_iso_initial_temp_t2}
\begin{center}
\begin{longtable}{| p{2.0in} || p{4.0in} |}
    \hline
    Type: & real \\
    \hline
    Units: & \si{deg.C} \\
    \hline
    Default Value: & 4.0 \\
    \hline
    Possible Values: & Any real number. \\
    \hline
    \caption{config\_iso\_initial\_temp\_t2: Amplitude parameter for the initial temperature profile.}
\end{longtable}
\end{center}
\subsection[config\_iso\_initial\_temp\_h0]{\hyperref[sec:nm_tab_iso]{config\_iso\_initial\_temp\_h0}}
\label{subsec:nm_sec_config_iso_initial_temp_h0}
\begin{center}
\begin{longtable}{| p{2.0in} || p{4.0in} |}
    \hline
    Type: & real \\
    \hline
    Units: & \si{m} \\
    \hline
    Default Value: & 1200 \\
    \hline
    Possible Values: & Any real number. \\
    \hline
    \caption{config\_iso\_initial\_temp\_h0: Depth parameter for the initial temperature profile.}
\end{longtable}
\end{center}
\subsection[config\_iso\_initial\_temp\_h1]{\hyperref[sec:nm_tab_iso]{config\_iso\_initial\_temp\_h1}}
\label{subsec:nm_sec_config_iso_initial_temp_h1}
\begin{center}
\begin{longtable}{| p{2.0in} || p{4.0in} |}
    \hline
    Type: & real \\
    \hline
    Units: & \si{m} \\
    \hline
    Default Value: & 500 \\
    \hline
    Possible Values: & Any real number. \\
    \hline
    \caption{config\_iso\_initial\_temp\_h1: Depth parameter for the initial temperature profile.}
\end{longtable}
\end{center}
\subsection[config\_iso\_initial\_temp\_mt]{\hyperref[sec:nm_tab_iso]{config\_iso\_initial\_temp\_mt}}
\label{subsec:nm_sec_config_iso_initial_temp_mt}
\begin{center}
\begin{longtable}{| p{2.0in} || p{4.0in} |}
    \hline
    Type: & real \\
    \hline
    Units: & \si{deg.C.m^{-1}} \\
    \hline
    Default Value: & 0.000075 \\
    \hline
    Possible Values: & Any real number. \\
    \hline
    \caption{config\_iso\_initial\_temp\_mt: Slope parameter for the initial temperature profile.}
\end{longtable}
\end{center}
\subsection[config\_iso\_initial\_temp\_latS]{\hyperref[sec:nm_tab_iso]{config\_iso\_initial\_temp\_latS}}
\label{subsec:nm_sec_config_iso_initial_temp_latS}
\begin{center}
\begin{longtable}{| p{2.0in} || p{4.0in} |}
    \hline
    Type: & real \\
    \hline
    Units: & \si{degrees} \\
    \hline
    Default Value: & -75 \\
    \hline
    Possible Values: & Any real number. \\
    \hline
    \caption{config\_iso\_initial\_temp\_latS: Southern latitude used to linearly scale the initial temperature field in the horizontal.}
\end{longtable}
\end{center}
\subsection[config\_iso\_initial\_temp\_latN]{\hyperref[sec:nm_tab_iso]{config\_iso\_initial\_temp\_latN}}
\label{subsec:nm_sec_config_iso_initial_temp_latN}
\begin{center}
\begin{longtable}{| p{2.0in} || p{4.0in} |}
    \hline
    Type: & real \\
    \hline
    Units: & \si{degrees} \\
    \hline
    Default Value: & -50 \\
    \hline
    Possible Values: & Any real number. \\
    \hline
    \caption{config\_iso\_initial\_temp\_latN: Southern latitude used to linearly scale the initial temperature field in the horizontal.}
\end{longtable}
\end{center}
\subsection[config\_iso\_temperature\_sponge\_t1]{\hyperref[sec:nm_tab_iso]{config\_iso\_temperature\_sponge\_t1}}
\label{subsec:nm_sec_config_iso_temperature_sponge_t1}
\begin{center}
\begin{longtable}{| p{2.0in} || p{4.0in} |}
    \hline
    Type: & real \\
    \hline
    Units: & \si{deg.C} \\
    \hline
    Default Value: & 10 \\
    \hline
    Possible Values: & Any real number. \\
    \hline
    \caption{config\_iso\_temperature\_sponge\_t1: Parameter for the sponge vertical temperature profile.}
\end{longtable}
\end{center}
\subsection[config\_iso\_temperature\_sponge\_h1]{\hyperref[sec:nm_tab_iso]{config\_iso\_temperature\_sponge\_h1}}
\label{subsec:nm_sec_config_iso_temperature_sponge_h1}
\begin{center}
\begin{longtable}{| p{2.0in} || p{4.0in} |}
    \hline
    Type: & real \\
    \hline
    Units: & \si{m} \\
    \hline
    Default Value: & 1000 \\
    \hline
    Possible Values: & Any real number. \\
    \hline
    \caption{config\_iso\_temperature\_sponge\_h1: E-folding distance parameter for the sponge vertical temperature profile.}
\end{longtable}
\end{center}
\subsection[config\_iso\_temperature\_sponge\_l1]{\hyperref[sec:nm_tab_iso]{config\_iso\_temperature\_sponge\_l1}}
\label{subsec:nm_sec_config_iso_temperature_sponge_l1}
\begin{center}
\begin{longtable}{| p{2.0in} || p{4.0in} |}
    \hline
    Type: & real \\
    \hline
    Units: & \si{m} \\
    \hline
    Default Value: & 120000 \\
    \hline
    Possible Values: & Any real number. \\
    \hline
    \caption{config\_iso\_temperature\_sponge\_l1: Horizontal e-folding distance parameter for the sponge weights.}
\end{longtable}
\end{center}
\subsection[config\_iso\_temperature\_sponge\_tau1]{\hyperref[sec:nm_tab_iso]{config\_iso\_temperature\_sponge\_tau1}}
\label{subsec:nm_sec_config_iso_temperature_sponge_tau1}
\begin{center}
\begin{longtable}{| p{2.0in} || p{4.0in} |}
    \hline
    Type: & real \\
    \hline
    Units: & \si{days} \\
    \hline
    Default Value: & 10.0 \\
    \hline
    Possible Values: & Any real number. \\
    \hline
    \caption{config\_iso\_temperature\_sponge\_tau1: Sponge layer restoring time scale, used to calculate interior restoring rate.}
\end{longtable}
\end{center}
\subsection[config\_iso\_temperature\_restore\_region1\_flag]{\hyperref[sec:nm_tab_iso]{config\_iso\_temperature\_restore\_region1\_flag}}
\label{subsec:nm_sec_config_iso_temperature_restore_region1_flag}
\begin{center}
\begin{longtable}{| p{2.0in} || p{4.0in} |}
    \hline
    Type: & logical \\
    \hline
    Units: & \si{unitless} \\
    \hline
    Default Value: & .true. \\
    \hline
    Possible Values: & .true. or .false. \\
    \hline
    \caption{config\_iso\_temperature\_restore\_region1\_flag: Logical flag controlling use of temperature restoring in region 1.}
\end{longtable}
\end{center}
\subsection[config\_iso\_temperature\_restore\_t1]{\hyperref[sec:nm_tab_iso]{config\_iso\_temperature\_restore\_t1}}
\label{subsec:nm_sec_config_iso_temperature_restore_t1}
\begin{center}
\begin{longtable}{| p{2.0in} || p{4.0in} |}
    \hline
    Type: & real \\
    \hline
    Units: & \si{deg.C} \\
    \hline
    Default Value: & -1 \\
    \hline
    Possible Values: & Any real number. \\
    \hline
    \caption{config\_iso\_temperature\_restore\_t1: Restoring temperature in region 1}
\end{longtable}
\end{center}
\subsection[config\_iso\_temperature\_restore\_lcx1]{\hyperref[sec:nm_tab_iso]{config\_iso\_temperature\_restore\_lcx1}}
\label{subsec:nm_sec_config_iso_temperature_restore_lcx1}
\begin{center}
\begin{longtable}{| p{2.0in} || p{4.0in} |}
    \hline
    Type: & real \\
    \hline
    Units: & \si{m} \\
    \hline
    Default Value: & 600000 \\
    \hline
    Possible Values: & Any real number. \\
    \hline
    \caption{config\_iso\_temperature\_restore\_lcx1: Zonal length scale of the restoring region 1}
\end{longtable}
\end{center}
\subsection[config\_iso\_temperature\_restore\_lcy1]{\hyperref[sec:nm_tab_iso]{config\_iso\_temperature\_restore\_lcy1}}
\label{subsec:nm_sec_config_iso_temperature_restore_lcy1}
\begin{center}
\begin{longtable}{| p{2.0in} || p{4.0in} |}
    \hline
    Type: & real \\
    \hline
    Units: & \si{m} \\
    \hline
    Default Value: & 600000 \\
    \hline
    Possible Values: & Any real number. \\
    \hline
    \caption{config\_iso\_temperature\_restore\_lcy1: Meridional length scale of the restoring region 1}
\end{longtable}
\end{center}
\subsection[config\_iso\_temperature\_restore\_region2\_flag]{\hyperref[sec:nm_tab_iso]{config\_iso\_temperature\_restore\_region2\_flag}}
\label{subsec:nm_sec_config_iso_temperature_restore_region2_flag}
\begin{center}
\begin{longtable}{| p{2.0in} || p{4.0in} |}
    \hline
    Type: & logical \\
    \hline
    Units: & \si{unitless} \\
    \hline
    Default Value: & .true. \\
    \hline
    Possible Values: & .true. or .false. \\
    \hline
    \caption{config\_iso\_temperature\_restore\_region2\_flag: Logical flag controlling use of temperature restoring in region 2.}
\end{longtable}
\end{center}
\subsection[config\_iso\_temperature\_restore\_t2]{\hyperref[sec:nm_tab_iso]{config\_iso\_temperature\_restore\_t2}}
\label{subsec:nm_sec_config_iso_temperature_restore_t2}
\begin{center}
\begin{longtable}{| p{2.0in} || p{4.0in} |}
    \hline
    Type: & real \\
    \hline
    Units: & \si{deg.C} \\
    \hline
    Default Value: & -1 \\
    \hline
    Possible Values: & Any real number. \\
    \hline
    \caption{config\_iso\_temperature\_restore\_t2: Restoring temperature in region 2}
\end{longtable}
\end{center}
\subsection[config\_iso\_temperature\_restore\_lcx2]{\hyperref[sec:nm_tab_iso]{config\_iso\_temperature\_restore\_lcx2}}
\label{subsec:nm_sec_config_iso_temperature_restore_lcx2}
\begin{center}
\begin{longtable}{| p{2.0in} || p{4.0in} |}
    \hline
    Type: & real \\
    \hline
    Units: & \si{m} \\
    \hline
    Default Value: & 600000 \\
    \hline
    Possible Values: & Any real number. \\
    \hline
    \caption{config\_iso\_temperature\_restore\_lcx2: Zonal length scale of the restoring region 2}
\end{longtable}
\end{center}
\subsection[config\_iso\_temperature\_restore\_lcy2]{\hyperref[sec:nm_tab_iso]{config\_iso\_temperature\_restore\_lcy2}}
\label{subsec:nm_sec_config_iso_temperature_restore_lcy2}
\begin{center}
\begin{longtable}{| p{2.0in} || p{4.0in} |}
    \hline
    Type: & real \\
    \hline
    Units: & \si{m} \\
    \hline
    Default Value: & 250000 \\
    \hline
    Possible Values: & Any real number. \\
    \hline
    \caption{config\_iso\_temperature\_restore\_lcy2: Meridional length scale of the restoring region 2}
\end{longtable}
\end{center}
\subsection[config\_iso\_temperature\_restore\_region3\_flag]{\hyperref[sec:nm_tab_iso]{config\_iso\_temperature\_restore\_region3\_flag}}
\label{subsec:nm_sec_config_iso_temperature_restore_region3_flag}
\begin{center}
\begin{longtable}{| p{2.0in} || p{4.0in} |}
    \hline
    Type: & logical \\
    \hline
    Units: & \si{unitless} \\
    \hline
    Default Value: & .true. \\
    \hline
    Possible Values: & .true. or .false. \\
    \hline
    \caption{config\_iso\_temperature\_restore\_region3\_flag: Logical flag controlling use of temperature restoring in region 3.}
\end{longtable}
\end{center}
\subsection[config\_iso\_temperature\_restore\_t3]{\hyperref[sec:nm_tab_iso]{config\_iso\_temperature\_restore\_t3}}
\label{subsec:nm_sec_config_iso_temperature_restore_t3}
\begin{center}
\begin{longtable}{| p{2.0in} || p{4.0in} |}
    \hline
    Type: & real \\
    \hline
    Units: & \si{deg.C} \\
    \hline
    Default Value: & -1 \\
    \hline
    Possible Values: & Any real number. \\
    \hline
    \caption{config\_iso\_temperature\_restore\_t3: Restoring temperature in region 3}
\end{longtable}
\end{center}
\subsection[config\_iso\_temperature\_restore\_lcx3]{\hyperref[sec:nm_tab_iso]{config\_iso\_temperature\_restore\_lcx3}}
\label{subsec:nm_sec_config_iso_temperature_restore_lcx3}
\begin{center}
\begin{longtable}{| p{2.0in} || p{4.0in} |}
    \hline
    Type: & real \\
    \hline
    Units: & \si{m} \\
    \hline
    Default Value: & 600000 \\
    \hline
    Possible Values: & Any real number. \\
    \hline
    \caption{config\_iso\_temperature\_restore\_lcx3: Zonal length scale of the restoring region 3}
\end{longtable}
\end{center}
\subsection[config\_iso\_temperature\_restore\_lcy3]{\hyperref[sec:nm_tab_iso]{config\_iso\_temperature\_restore\_lcy3}}
\label{subsec:nm_sec_config_iso_temperature_restore_lcy3}
\begin{center}
\begin{longtable}{| p{2.0in} || p{4.0in} |}
    \hline
    Type: & real \\
    \hline
    Units: & \si{m} \\
    \hline
    Default Value: & 250000 \\
    \hline
    Possible Values: & Any real number. \\
    \hline
    \caption{config\_iso\_temperature\_restore\_lcy3: Meridional length scale of the restoring region 3}
\end{longtable}
\end{center}
\subsection[config\_iso\_temperature\_restore\_region4\_flag]{\hyperref[sec:nm_tab_iso]{config\_iso\_temperature\_restore\_region4\_flag}}
\label{subsec:nm_sec_config_iso_temperature_restore_region4_flag}
\begin{center}
\begin{longtable}{| p{2.0in} || p{4.0in} |}
    \hline
    Type: & logical \\
    \hline
    Units: & \si{unitless} \\
    \hline
    Default Value: & .true. \\
    \hline
    Possible Values: & .true. or .false. \\
    \hline
    \caption{config\_iso\_temperature\_restore\_region4\_flag: Logical flag controlling use of temperature restoring in region 4.}
\end{longtable}
\end{center}
\subsection[config\_iso\_temperature\_restore\_t4]{\hyperref[sec:nm_tab_iso]{config\_iso\_temperature\_restore\_t4}}
\label{subsec:nm_sec_config_iso_temperature_restore_t4}
\begin{center}
\begin{longtable}{| p{2.0in} || p{4.0in} |}
    \hline
    Type: & real \\
    \hline
    Units: & \si{deg.C} \\
    \hline
    Default Value: & -1 \\
    \hline
    Possible Values: & Any real number. \\
    \hline
    \caption{config\_iso\_temperature\_restore\_t4: Restoring temperature in region 4}
\end{longtable}
\end{center}
\subsection[config\_iso\_temperature\_restore\_lcx4]{\hyperref[sec:nm_tab_iso]{config\_iso\_temperature\_restore\_lcx4}}
\label{subsec:nm_sec_config_iso_temperature_restore_lcx4}
\begin{center}
\begin{longtable}{| p{2.0in} || p{4.0in} |}
    \hline
    Type: & real \\
    \hline
    Units: & \si{m} \\
    \hline
    Default Value: & 600000 \\
    \hline
    Possible Values: & Any real number. \\
    \hline
    \caption{config\_iso\_temperature\_restore\_lcx4: Zonal length scale of the restoring region 4}
\end{longtable}
\end{center}
\subsection[config\_iso\_temperature\_restore\_lcy4]{\hyperref[sec:nm_tab_iso]{config\_iso\_temperature\_restore\_lcy4}}
\label{subsec:nm_sec_config_iso_temperature_restore_lcy4}
\begin{center}
\begin{longtable}{| p{2.0in} || p{4.0in} |}
    \hline
    Type: & real \\
    \hline
    Units: & \si{m} \\
    \hline
    Default Value: & 250000 \\
    \hline
    Possible Values: & Any real number. \\
    \hline
    \caption{config\_iso\_temperature\_restore\_lcy4: Meridional length scale of the restoring region 4}
\end{longtable}
\end{center}
\section[soma]{\hyperref[sec:nm_tab_soma]{soma}}
\label{sec:nm_sec_soma}
\subsection[config\_soma\_vert\_levels]{\hyperref[sec:nm_tab_soma]{config\_soma\_vert\_levels}}
\label{subsec:nm_sec_config_soma_vert_levels}
\begin{center}
\begin{longtable}{| p{2.0in} || p{4.0in} |}
    \hline
    Type: & integer \\
    \hline
    Units: & \si{unitless} \\
    \hline
    Default Value: & 100 \\
    \hline
    Possible Values: & Any positive integer. Typically 40 or larger. \\
    \hline
    \caption{config\_soma\_vert\_levels: Number of vertical levels in SOMA.}
\end{longtable}
\end{center}
\subsection[config\_soma\_domain\_width]{\hyperref[sec:nm_tab_soma]{config\_soma\_domain\_width}}
\label{subsec:nm_sec_config_soma_domain_width}
\begin{center}
\begin{longtable}{| p{2.0in} || p{4.0in} |}
    \hline
    Type: & real \\
    \hline
    Units: & \si{m} \\
    \hline
    Default Value: & 1.25e6 \\
    \hline
    Possible Values: & Any real positive number \\
    \hline
    \caption{config\_soma\_domain\_width: Approximate radius of the SOMA domain.}
\end{longtable}
\end{center}
\subsection[config\_soma\_center\_latitude]{\hyperref[sec:nm_tab_soma]{config\_soma\_center\_latitude}}
\label{subsec:nm_sec_config_soma_center_latitude}
\begin{center}
\begin{longtable}{| p{2.0in} || p{4.0in} |}
    \hline
    Type: & real \\
    \hline
    Units: & \si{degrees} \\
    \hline
    Default Value: & 35.0 \\
    \hline
    Possible Values: & Any real number between -90.0 and 90.0. \\
    \hline
    \caption{config\_soma\_center\_latitude: Latitude for the center of the SOMA basin.}
\end{longtable}
\end{center}
\subsection[config\_soma\_center\_longitude]{\hyperref[sec:nm_tab_soma]{config\_soma\_center\_longitude}}
\label{subsec:nm_sec_config_soma_center_longitude}
\begin{center}
\begin{longtable}{| p{2.0in} || p{4.0in} |}
    \hline
    Type: & real \\
    \hline
    Units: & \si{degrees} \\
    \hline
    Default Value: & 0.0 \\
    \hline
    Possible Values: & Any real number between 0.0 and 360.0. \\
    \hline
    \caption{config\_soma\_center\_longitude: Longitude for the center of the SOMA basin.}
\end{longtable}
\end{center}
\subsection[config\_soma\_phi]{\hyperref[sec:nm_tab_soma]{config\_soma\_phi}}
\label{subsec:nm_sec_config_soma_phi}
\begin{center}
\begin{longtable}{| p{2.0in} || p{4.0in} |}
    \hline
    Type: & real \\
    \hline
    Units: & \si{non-dimensional} \\
    \hline
    Default Value: & 0.1 \\
    \hline
    Possible Values: & Any real positive number \\
    \hline
    \caption{config\_soma\_phi: Scale factor controlling width of continential slope. Typically around 0.1}
\end{longtable}
\end{center}
\subsection[config\_soma\_bottom\_depth]{\hyperref[sec:nm_tab_soma]{config\_soma\_bottom\_depth}}
\label{subsec:nm_sec_config_soma_bottom_depth}
\begin{center}
\begin{longtable}{| p{2.0in} || p{4.0in} |}
    \hline
    Type: & real \\
    \hline
    Units: & \si{m} \\
    \hline
    Default Value: & 2500.0 \\
    \hline
    Possible Values: & Any real positive number. \\
    \hline
    \caption{config\_soma\_bottom\_depth: Depth of the bottom of the ocean for the SOMA test case.}
\end{longtable}
\end{center}
\subsection[config\_soma\_shelf\_width]{\hyperref[sec:nm_tab_soma]{config\_soma\_shelf\_width}}
\label{subsec:nm_sec_config_soma_shelf_width}
\begin{center}
\begin{longtable}{| p{2.0in} || p{4.0in} |}
    \hline
    Type: & real \\
    \hline
    Units: & \si{non-dimensional} \\
    \hline
    Default Value: & -0.4 \\
    \hline
    Possible Values: & Any real number \\
    \hline
    \caption{config\_soma\_shelf\_width: Non-dimensional measure of continential shelf. Typically negative.}
\end{longtable}
\end{center}
\subsection[config\_soma\_shelf\_depth]{\hyperref[sec:nm_tab_soma]{config\_soma\_shelf\_depth}}
\label{subsec:nm_sec_config_soma_shelf_depth}
\begin{center}
\begin{longtable}{| p{2.0in} || p{4.0in} |}
    \hline
    Type: & real \\
    \hline
    Units: & \si{m} \\
    \hline
    Default Value: & 100.0 \\
    \hline
    Possible Values: & Any real positive number \\
    \hline
    \caption{config\_soma\_shelf\_depth: Depth of the continential shelf.}
\end{longtable}
\end{center}
\subsection[config\_soma\_ref\_density]{\hyperref[sec:nm_tab_soma]{config\_soma\_ref\_density}}
\label{subsec:nm_sec_config_soma_ref_density}
\begin{center}
\begin{longtable}{| p{2.0in} || p{4.0in} |}
    \hline
    Type: & real \\
    \hline
    Units: & \si{kg.m^{-3}} \\
    \hline
    Default Value: & 1000.0 \\
    \hline
    Possible Values: & Any real number. \\
    \hline
    \caption{config\_soma\_ref\_density: Reference density for the SOMA test case.}
\end{longtable}
\end{center}
\subsection[config\_soma\_density\_difference]{\hyperref[sec:nm_tab_soma]{config\_soma\_density\_difference}}
\label{subsec:nm_sec_config_soma_density_difference}
\begin{center}
\begin{longtable}{| p{2.0in} || p{4.0in} |}
    \hline
    Type: & real \\
    \hline
    Units: & \si{kg.m^{-3}} \\
    \hline
    Default Value: & 4.0 \\
    \hline
    Possible Values: & Any real number. \\
    \hline
    \caption{config\_soma\_density\_difference: Density difference between surface and bottom waters for the SOMA test case.}
\end{longtable}
\end{center}
\subsection[config\_soma\_thermocline\_depth]{\hyperref[sec:nm_tab_soma]{config\_soma\_thermocline\_depth}}
\label{subsec:nm_sec_config_soma_thermocline_depth}
\begin{center}
\begin{longtable}{| p{2.0in} || p{4.0in} |}
    \hline
    Type: & real \\
    \hline
    Units: & \si{m} \\
    \hline
    Default Value: & 300.0 \\
    \hline
    Possible Values: & Any real positive number. \\
    \hline
    \caption{config\_soma\_thermocline\_depth: Depth over which majority of initial stratification is placed.}
\end{longtable}
\end{center}
\subsection[config\_soma\_density\_difference\_linear]{\hyperref[sec:nm_tab_soma]{config\_soma\_density\_difference\_linear}}
\label{subsec:nm_sec_config_soma_density_difference_linear}
\begin{center}
\begin{longtable}{| p{2.0in} || p{4.0in} |}
    \hline
    Type: & real \\
    \hline
    Units: & \si{non-dimensional} \\
    \hline
    Default Value: & 0.05 \\
    \hline
    Possible Values: & Any real positive number. \\
    \hline
    \caption{config\_soma\_density\_difference\_linear: Fraction of stratification put into linear profile extending from surface to bottom.}
\end{longtable}
\end{center}
\subsection[config\_soma\_surface\_temperature]{\hyperref[sec:nm_tab_soma]{config\_soma\_surface\_temperature}}
\label{subsec:nm_sec_config_soma_surface_temperature}
\begin{center}
\begin{longtable}{| p{2.0in} || p{4.0in} |}
    \hline
    Type: & real \\
    \hline
    Units: & \si{C} \\
    \hline
    Default Value: & 20.0 \\
    \hline
    Possible Values: & Any real positive number. \\
    \hline
    \caption{config\_soma\_surface\_temperature: Surface temperature value used in initial condition.}
\end{longtable}
\end{center}
\subsection[config\_soma\_surface\_salinity]{\hyperref[sec:nm_tab_soma]{config\_soma\_surface\_salinity}}
\label{subsec:nm_sec_config_soma_surface_salinity}
\begin{center}
\begin{longtable}{| p{2.0in} || p{4.0in} |}
    \hline
    Type: & real \\
    \hline
    Units: & \si{PSU} \\
    \hline
    Default Value: & 33.0 \\
    \hline
    Possible Values: & Any real positive number. \\
    \hline
    \caption{config\_soma\_surface\_salinity: Surface salinity value used in initial condition.}
\end{longtable}
\end{center}
\subsection[config\_soma\_use\_surface\_temp\_restoring]{\hyperref[sec:nm_tab_soma]{config\_soma\_use\_surface\_temp\_restoring}}
\label{subsec:nm_sec_config_soma_use_surface_temp_restoring}
\begin{center}
\begin{longtable}{| p{2.0in} || p{4.0in} |}
    \hline
    Type: & logical \\
    \hline
    Units: & \si{unitless} \\
    \hline
    Default Value: & false \\
    \hline
    Possible Values: & .true. or .false. \\
    \hline
    \caption{config\_soma\_use\_surface\_temp\_restoring: Logical flag that determines if surface temperature restoring is to be used.}
\end{longtable}
\end{center}
\subsection[config\_soma\_surface\_temp\_restoring\_at\_center\_latitude]{\hyperref[sec:nm_tab_soma]{config\_soma\_surface\_temp\_restoring\_at\_center\_latitude}}
\label{subsec:nm_sec_config_soma_surface_temp_restoring_at_center_latitude}
\begin{center}
\begin{longtable}{| p{2.0in} || p{4.0in} |}
    \hline
    Type: & real \\
    \hline
    Units: & \si{degrees} \\
    \hline
    Default Value: & 7.5 \\
    \hline
    Possible Values: & Any real positive number. \\
    \hline
    \caption{config\_soma\_surface\_temp\_restoring\_at\_center\_latitude: Surface restoring temperature value at center latitutde.}
\end{longtable}
\end{center}
\subsection[config\_soma\_surface\_temp\_restoring\_latitude\_gradient]{\hyperref[sec:nm_tab_soma]{config\_soma\_surface\_temp\_restoring\_latitude\_gradient}}
\label{subsec:nm_sec_config_soma_surface_temp_restoring_latitude_gradient}
\begin{center}
\begin{longtable}{| p{2.0in} || p{4.0in} |}
    \hline
    Type: & real \\
    \hline
    Units: & \si{degrees.C./.degrees.latitude} \\
    \hline
    Default Value: & 0.5 \\
    \hline
    Possible Values: & Any real positive number. \\
    \hline
    \caption{config\_soma\_surface\_temp\_restoring\_latitude\_gradient: Surface restoring temperature gradient in latitudal direction.}
\end{longtable}
\end{center}
\subsection[config\_soma\_restoring\_temp\_piston\_vel]{\hyperref[sec:nm_tab_soma]{config\_soma\_restoring\_temp\_piston\_vel}}
\label{subsec:nm_sec_config_soma_restoring_temp_piston_vel}
\begin{center}
\begin{longtable}{| p{2.0in} || p{4.0in} |}
    \hline
    Type: & real \\
    \hline
    Units: & \si{m.s^{-1}} \\
    \hline
    Default Value: & 1.0e-5 \\
    \hline
    Possible Values: & Any real number. \\
    \hline
    \caption{config\_soma\_restoring\_temp\_piston\_vel: Restoring piston velocity for surface temperature.}
\end{longtable}
\end{center}
\section[ziso]{\hyperref[sec:nm_tab_ziso]{ziso}}
\label{sec:nm_sec_ziso}
\subsection[config\_ziso\_vert\_levels]{\hyperref[sec:nm_tab_ziso]{config\_ziso\_vert\_levels}}
\label{subsec:nm_sec_config_ziso_vert_levels}
\begin{center}
\begin{longtable}{| p{2.0in} || p{4.0in} |}
    \hline
    Type: & integer \\
    \hline
    Units: & \si{unitless} \\
    \hline
    Default Value: & 100 \\
    \hline
    Possible Values: & Any positive integer number greater than 0. \\
    \hline
    \caption{config\_ziso\_vert\_levels: Number of vertical levels in ziso. Typical value is 100.}
\end{longtable}
\end{center}
\subsection[config\_ziso\_add\_easterly\_wind\_stress\_ASF]{\hyperref[sec:nm_tab_ziso]{config\_ziso\_add\_easterly\_wind\_stress\_ASF}}
\label{subsec:nm_sec_config_ziso_add_easterly_wind_stress_ASF}
\begin{center}
\begin{longtable}{| p{2.0in} || p{4.0in} |}
    \hline
    Type: & logical \\
    \hline
    Units: & \si{unitless} \\
    \hline
    Default Value: & false \\
    \hline
    Possible Values: & .true. or .false. \\
    \hline
    \caption{config\_ziso\_add\_easterly\_wind\_stress\_ASF: Logical flag to determine if an easterly windstress is added}
\end{longtable}
\end{center}
\subsection[config\_ziso\_wind\_transition\_position]{\hyperref[sec:nm_tab_ziso]{config\_ziso\_wind\_transition\_position}}
\label{subsec:nm_sec_config_ziso_wind_transition_position}
\begin{center}
\begin{longtable}{| p{2.0in} || p{4.0in} |}
    \hline
    Type: & real \\
    \hline
    Units: & \si{m} \\
    \hline
    Default Value: & 800000.0 \\
    \hline
    Possible Values: & Any positive real number, less than config\_ziso\_meridional\_extent \\
    \hline
    \caption{config\_ziso\_wind\_transition\_position: meridional position where windstress switches to easterly}
\end{longtable}
\end{center}
\subsection[config\_ziso\_antarctic\_shelf\_front\_width]{\hyperref[sec:nm_tab_ziso]{config\_ziso\_antarctic\_shelf\_front\_width}}
\label{subsec:nm_sec_config_ziso_antarctic_shelf_front_width}
\begin{center}
\begin{longtable}{| p{2.0in} || p{4.0in} |}
    \hline
    Type: & real \\
    \hline
    Units: & \si{m} \\
    \hline
    Default Value: & 600000 \\
    \hline
    Possible Values: & any positive real number less than the meridional domain extent \\
    \hline
    \caption{config\_ziso\_antarctic\_shelf\_front\_width: meridional extent over which the easterly wind stress is applied}
\end{longtable}
\end{center}
\subsection[config\_ziso\_wind\_stress\_shelf\_front\_max]{\hyperref[sec:nm_tab_ziso]{config\_ziso\_wind\_stress\_shelf\_front\_max}}
\label{subsec:nm_sec_config_ziso_wind_stress_shelf_front_max}
\begin{center}
\begin{longtable}{| p{2.0in} || p{4.0in} |}
    \hline
    Type: & real \\
    \hline
    Units: & \si{N.m^{-2}} \\
    \hline
    Default Value: & -0.05 \\
    \hline
    Possible Values: & Any real number less than 0 \\
    \hline
    \caption{config\_ziso\_wind\_stress\_shelf\_front\_max: Maximum zonal windstress value in the shelf front region, following Stewart et al. 2013}
\end{longtable}
\end{center}
\subsection[config\_ziso\_use\_slopping\_bathymetry]{\hyperref[sec:nm_tab_ziso]{config\_ziso\_use\_slopping\_bathymetry}}
\label{subsec:nm_sec_config_ziso_use_slopping_bathymetry}
\begin{center}
\begin{longtable}{| p{2.0in} || p{4.0in} |}
    \hline
    Type: & logical \\
    \hline
    Units: & \si{unitless} \\
    \hline
    Default Value: & false \\
    \hline
    Possible Values: & .true. or .false. \\
    \hline
    \caption{config\_ziso\_use\_slopping\_bathymetry: Logical flag that determines if sloping bathymetery is used.}
\end{longtable}
\end{center}
\subsection[config\_ziso\_meridional\_extent]{\hyperref[sec:nm_tab_ziso]{config\_ziso\_meridional\_extent}}
\label{subsec:nm_sec_config_ziso_meridional_extent}
\begin{center}
\begin{longtable}{| p{2.0in} || p{4.0in} |}
    \hline
    Type: & real \\
    \hline
    Units: & \si{m} \\
    \hline
    Default Value: & 2.0e6 \\
    \hline
    Possible Values: & Any real number larger than zero. \\
    \hline
    \caption{config\_ziso\_meridional\_extent: Meridional extent of the domain ($L$).}
\end{longtable}
\end{center}
\subsection[config\_ziso\_zonal\_extent]{\hyperref[sec:nm_tab_ziso]{config\_ziso\_zonal\_extent}}
\label{subsec:nm_sec_config_ziso_zonal_extent}
\begin{center}
\begin{longtable}{| p{2.0in} || p{4.0in} |}
    \hline
    Type: & real \\
    \hline
    Units: & \si{m} \\
    \hline
    Default Value: & 1.0e6 \\
    \hline
    Possible Values: & Any real number larger than zero. \\
    \hline
    \caption{config\_ziso\_zonal\_extent: Zonal extent of the domain ($W$).}
\end{longtable}
\end{center}
\subsection[config\_ziso\_bottom\_depth]{\hyperref[sec:nm_tab_ziso]{config\_ziso\_bottom\_depth}}
\label{subsec:nm_sec_config_ziso_bottom_depth}
\begin{center}
\begin{longtable}{| p{2.0in} || p{4.0in} |}
    \hline
    Type: & real \\
    \hline
    Units: & \si{m} \\
    \hline
    Default Value: & 2500.0 \\
    \hline
    Possible Values: & Any real number larger than zero. \\
    \hline
    \caption{config\_ziso\_bottom\_depth: Depth of the domain ($H$).}
\end{longtable}
\end{center}
\subsection[config\_ziso\_shelf\_depth]{\hyperref[sec:nm_tab_ziso]{config\_ziso\_shelf\_depth}}
\label{subsec:nm_sec_config_ziso_shelf_depth}
\begin{center}
\begin{longtable}{| p{2.0in} || p{4.0in} |}
    \hline
    Type: & real \\
    \hline
    Units: & \si{m} \\
    \hline
    Default Value: & 500.0 \\
    \hline
    Possible Values: & Any real number. \\
    \hline
    \caption{config\_ziso\_shelf\_depth: Shelf depth in the domain ($H_s$).}
\end{longtable}
\end{center}
\subsection[config\_ziso\_slope\_half\_width]{\hyperref[sec:nm_tab_ziso]{config\_ziso\_slope\_half\_width}}
\label{subsec:nm_sec_config_ziso_slope_half_width}
\begin{center}
\begin{longtable}{| p{2.0in} || p{4.0in} |}
    \hline
    Type: & real \\
    \hline
    Units: & \si{m} \\
    \hline
    Default Value: & 1.0e5 \\
    \hline
    Possible Values: & Any real number. \\
    \hline
    \caption{config\_ziso\_slope\_half\_width: Shelf half width ($W_s$).}
\end{longtable}
\end{center}
\subsection[config\_ziso\_slope\_center\_position]{\hyperref[sec:nm_tab_ziso]{config\_ziso\_slope\_center\_position}}
\label{subsec:nm_sec_config_ziso_slope_center_position}
\begin{center}
\begin{longtable}{| p{2.0in} || p{4.0in} |}
    \hline
    Type: & real \\
    \hline
    Units: & \si{m} \\
    \hline
    Default Value: & 5.0e5 \\
    \hline
    Possible Values: & Any real number. \\
    \hline
    \caption{config\_ziso\_slope\_center\_position: Slope center posiiton ($Y_s$).}
\end{longtable}
\end{center}
\subsection[config\_ziso\_reference\_coriolis]{\hyperref[sec:nm_tab_ziso]{config\_ziso\_reference\_coriolis}}
\label{subsec:nm_sec_config_ziso_reference_coriolis}
\begin{center}
\begin{longtable}{| p{2.0in} || p{4.0in} |}
    \hline
    Type: & real \\
    \hline
    Units: & \si{s^{-1}} \\
    \hline
    Default Value: & -1e-4 \\
    \hline
    Possible Values: & Any real number larger. \\
    \hline
    \caption{config\_ziso\_reference\_coriolis: Reference coriolis parameter $f_0$. Note $f = f_0 + \beta * y$.}
\end{longtable}
\end{center}
\subsection[config\_ziso\_coriolis\_gradient]{\hyperref[sec:nm_tab_ziso]{config\_ziso\_coriolis\_gradient}}
\label{subsec:nm_sec_config_ziso_coriolis_gradient}
\begin{center}
\begin{longtable}{| p{2.0in} || p{4.0in} |}
    \hline
    Type: & real \\
    \hline
    Units: & \si{m^{-1}.s^{-1}} \\
    \hline
    Default Value: & 1e-11 \\
    \hline
    Possible Values: & Any real number. \\
    \hline
    \caption{config\_ziso\_coriolis\_gradient: Meridional gradient of coriolis parameter $\beta$.}
\end{longtable}
\end{center}
\subsection[config\_ziso\_wind\_stress\_max]{\hyperref[sec:nm_tab_ziso]{config\_ziso\_wind\_stress\_max}}
\label{subsec:nm_sec_config_ziso_wind_stress_max}
\begin{center}
\begin{longtable}{| p{2.0in} || p{4.0in} |}
    \hline
    Type: & real \\
    \hline
    Units: & \si{N.m^{2}} \\
    \hline
    Default Value: & 0.2 \\
    \hline
    Possible Values: & Any real number. \\
    \hline
    \caption{config\_ziso\_wind\_stress\_max: Maximum zonal windstress value $\tau_0$.}
\end{longtable}
\end{center}
\subsection[config\_ziso\_mean\_restoring\_temp]{\hyperref[sec:nm_tab_ziso]{config\_ziso\_mean\_restoring\_temp}}
\label{subsec:nm_sec_config_ziso_mean_restoring_temp}
\begin{center}
\begin{longtable}{| p{2.0in} || p{4.0in} |}
    \hline
    Type: & real \\
    \hline
    Units: & \si{^{circ}.C} \\
    \hline
    Default Value: & 3.0 \\
    \hline
    Possible Values: & Any real number. \\
    \hline
    \caption{config\_ziso\_mean\_restoring\_temp: Mean restoring temperature $T_m$ in $T_r(y) = T_m + T_a \tanh\left(2\frac{y-L/2}{L/2}\right) + T_b \frac{y-L/2}{L/2}$.}
\end{longtable}
\end{center}
\subsection[config\_ziso\_restoring\_temp\_dev\_ta]{\hyperref[sec:nm_tab_ziso]{config\_ziso\_restoring\_temp\_dev\_ta}}
\label{subsec:nm_sec_config_ziso_restoring_temp_dev_ta}
\begin{center}
\begin{longtable}{| p{2.0in} || p{4.0in} |}
    \hline
    Type: & real \\
    \hline
    Units: & \si{^{circ}.C} \\
    \hline
    Default Value: & 2.0 \\
    \hline
    Possible Values: & Any real number. \\
    \hline
    \caption{config\_ziso\_restoring\_temp\_dev\_ta: Temperature deviation $T_a$ in surface temp. $T_r(y) = T_m + T_a \tanh\left(2\frac{y-L/2}{L/2}\right) + T_b \frac{y-L/2}{L/2}$.}
\end{longtable}
\end{center}
\subsection[config\_ziso\_restoring\_temp\_dev\_tb]{\hyperref[sec:nm_tab_ziso]{config\_ziso\_restoring\_temp\_dev\_tb}}
\label{subsec:nm_sec_config_ziso_restoring_temp_dev_tb}
\begin{center}
\begin{longtable}{| p{2.0in} || p{4.0in} |}
    \hline
    Type: & real \\
    \hline
    Units: & \si{^{circ}.C} \\
    \hline
    Default Value: & 2.0 \\
    \hline
    Possible Values: & Any real number. \\
    \hline
    \caption{config\_ziso\_restoring\_temp\_dev\_tb: Linear temperature deviation $T_b$ in surface temp. $T_r(y) = T_m + T_a \tanh\left(2\frac{y-L/2}{L/2}\right) + T_b \frac{y-L/2}{L/2}$.}
\end{longtable}
\end{center}
\subsection[config\_ziso\_restoring\_temp\_tau]{\hyperref[sec:nm_tab_ziso]{config\_ziso\_restoring\_temp\_tau}}
\label{subsec:nm_sec_config_ziso_restoring_temp_tau}
\begin{center}
\begin{longtable}{| p{2.0in} || p{4.0in} |}
    \hline
    Type: & real \\
    \hline
    Units: & \si{days} \\
    \hline
    Default Value: & 30.0 \\
    \hline
    Possible Values: & Any real number. \\
    \hline
    \caption{config\_ziso\_restoring\_temp\_tau: Time scale for interior restoring of temperature.}
\end{longtable}
\end{center}
\subsection[config\_ziso\_restoring\_temp\_piston\_vel]{\hyperref[sec:nm_tab_ziso]{config\_ziso\_restoring\_temp\_piston\_vel}}
\label{subsec:nm_sec_config_ziso_restoring_temp_piston_vel}
\begin{center}
\begin{longtable}{| p{2.0in} || p{4.0in} |}
    \hline
    Type: & real \\
    \hline
    Units: & \si{m.s^{-1}} \\
    \hline
    Default Value: & 1.93e-5 \\
    \hline
    Possible Values: & Any real number. \\
    \hline
    \caption{config\_ziso\_restoring\_temp\_piston\_vel: Restoring piston velocity for surface temperature.}
\end{longtable}
\end{center}
\subsection[config\_ziso\_restoring\_temp\_ze]{\hyperref[sec:nm_tab_ziso]{config\_ziso\_restoring\_temp\_ze}}
\label{subsec:nm_sec_config_ziso_restoring_temp_ze}
\begin{center}
\begin{longtable}{| p{2.0in} || p{4.0in} |}
    \hline
    Type: & real \\
    \hline
    Units: & \si{m} \\
    \hline
    Default Value: & 1250.0 \\
    \hline
    Possible Values: & Any real number. \\
    \hline
    \caption{config\_ziso\_restoring\_temp\_ze: Vertical $e-$folding scale in $T_s$ for northern wall: $T_s \exp(z/z_e)$.}
\end{longtable}
\end{center}
\subsection[config\_ziso\_restoring\_sponge\_l]{\hyperref[sec:nm_tab_ziso]{config\_ziso\_restoring\_sponge\_l}}
\label{subsec:nm_sec_config_ziso_restoring_sponge_l}
\begin{center}
\begin{longtable}{| p{2.0in} || p{4.0in} |}
    \hline
    Type: & real \\
    \hline
    Units: & \si{m} \\
    \hline
    Default Value: & 8.0e4 \\
    \hline
    Possible Values: & Any real number. \\
    \hline
    \caption{config\_ziso\_restoring\_sponge\_l: E-folding distance parameter for the sponge vertical temperature profile.}
\end{longtable}
\end{center}
\subsection[config\_ziso\_initial\_temp\_t1]{\hyperref[sec:nm_tab_ziso]{config\_ziso\_initial\_temp\_t1}}
\label{subsec:nm_sec_config_ziso_initial_temp_t1}
\begin{center}
\begin{longtable}{| p{2.0in} || p{4.0in} |}
    \hline
    Type: & real \\
    \hline
    Units: & \si{^{circ}.C} \\
    \hline
    Default Value: & 6.0 \\
    \hline
    Possible Values: & Any real number. \\
    \hline
    \caption{config\_ziso\_initial\_temp\_t1: Initial temperature profile constant $T_1$ in $T(z,t=0) = T_1 + T_2 \tanh(z/h_1) + m_T z$.}
\end{longtable}
\end{center}
\subsection[config\_ziso\_initial\_temp\_t2]{\hyperref[sec:nm_tab_ziso]{config\_ziso\_initial\_temp\_t2}}
\label{subsec:nm_sec_config_ziso_initial_temp_t2}
\begin{center}
\begin{longtable}{| p{2.0in} || p{4.0in} |}
    \hline
    Type: & real \\
    \hline
    Units: & \si{^{circ}.C} \\
    \hline
    Default Value: & 3.6 \\
    \hline
    Possible Values: & Any real number. \\
    \hline
    \caption{config\_ziso\_initial\_temp\_t2: Initial temperature profile constant $T_2$ in $T(z,t=0) = T_1 + T_2 \tanh(z/h_1) + m_T z$.}
\end{longtable}
\end{center}
\subsection[config\_ziso\_initial\_temp\_h1]{\hyperref[sec:nm_tab_ziso]{config\_ziso\_initial\_temp\_h1}}
\label{subsec:nm_sec_config_ziso_initial_temp_h1}
\begin{center}
\begin{longtable}{| p{2.0in} || p{4.0in} |}
    \hline
    Type: & real \\
    \hline
    Units: & \si{m} \\
    \hline
    Default Value: & 300.0 \\
    \hline
    Possible Values: & Any real number. \\
    \hline
    \caption{config\_ziso\_initial\_temp\_h1: Initial temperature profile constant $h_1$ in $T(z,t=0) = T_1 + T_2 \tanh(z/h_1) + m_T z$.}
\end{longtable}
\end{center}
\subsection[config\_ziso\_initial\_temp\_mt]{\hyperref[sec:nm_tab_ziso]{config\_ziso\_initial\_temp\_mt}}
\label{subsec:nm_sec_config_ziso_initial_temp_mt}
\begin{center}
\begin{longtable}{| p{2.0in} || p{4.0in} |}
    \hline
    Type: & real \\
    \hline
    Units: & \si{^{circ}.C.m^{-1}} \\
    \hline
    Default Value: & 7.5e-5 \\
    \hline
    Possible Values: & Any real number. \\
    \hline
    \caption{config\_ziso\_initial\_temp\_mt: Initial temperature profile constant $m_T$ in $T(z,t=0) = T_1 + T_2 \tanh(z/h_1) + m_T z$.}
\end{longtable}
\end{center}
\subsection[config\_ziso\_frazil\_enable]{\hyperref[sec:nm_tab_ziso]{config\_ziso\_frazil\_enable}}
\label{subsec:nm_sec_config_ziso_frazil_enable}
\begin{center}
\begin{longtable}{| p{2.0in} || p{4.0in} |}
    \hline
    Type: & logical \\
    \hline
    Units: & \si{logical} \\
    \hline
    Default Value: & false \\
    \hline
    Possible Values: & true or false \\
    \hline
    \caption{config\_ziso\_frazil\_enable: A logical to overload (and largely overwrite) this test case to evaluate frazil. In almost all uses of this test case, this configure option should be false.}
\end{longtable}
\end{center}
\subsection[config\_ziso\_frazil\_temperature\_anomaly]{\hyperref[sec:nm_tab_ziso]{config\_ziso\_frazil\_temperature\_anomaly}}
\label{subsec:nm_sec_config_ziso_frazil_temperature_anomaly}
\begin{center}
\begin{longtable}{| p{2.0in} || p{4.0in} |}
    \hline
    Type: & real \\
    \hline
    Units: & \si{^{circ}.C} \\
    \hline
    Default Value: & -3.0 \\
    \hline
    Possible Values: & Any real number. \\
    \hline
    \caption{config\_ziso\_frazil\_temperature\_anomaly: Temperature anomaly to produce frazil}
\end{longtable}
\end{center}
\section[sub\_ice\_shelf\_2D]{\hyperref[sec:nm_tab_sub_ice_shelf_2D]{sub\_ice\_shelf\_2D}}
\label{sec:nm_sec_sub_ice_shelf_2D}
\subsection[config\_sub\_ice\_shelf\_2D\_vert\_levels]{\hyperref[sec:nm_tab_sub_ice_shelf_2D]{config\_sub\_ice\_shelf\_2D\_vert\_levels}}
\label{subsec:nm_sec_config_sub_ice_shelf_2D_vert_levels}
\begin{center}
\begin{longtable}{| p{2.0in} || p{4.0in} |}
    \hline
    Type: & integer \\
    \hline
    Units: & \si{unitless} \\
    \hline
    Default Value: & 20 \\
    \hline
    Possible Values: & Any positive integer number greater than 0. \\
    \hline
    \caption{config\_sub\_ice\_shelf\_2D\_vert\_levels: Number of vertical levels in sub\_ice\_shelf\_2D. Typical value is 22.}
\end{longtable}
\end{center}
\subsection[config\_sub\_ice\_shelf\_2D\_bottom\_depth]{\hyperref[sec:nm_tab_sub_ice_shelf_2D]{config\_sub\_ice\_shelf\_2D\_bottom\_depth}}
\label{subsec:nm_sec_config_sub_ice_shelf_2D_bottom_depth}
\begin{center}
\begin{longtable}{| p{2.0in} || p{4.0in} |}
    \hline
    Type: & real \\
    \hline
    Units: & \si{m} \\
    \hline
    Default Value: & 2000.0 \\
    \hline
    Possible Values: & Any positive real number. \\
    \hline
    \caption{config\_sub\_ice\_shelf\_2D\_bottom\_depth: Depth of the bottom of the ocean for the this test case.}
\end{longtable}
\end{center}
\subsection[config\_sub\_ice\_shelf\_2D\_cavity\_thickness]{\hyperref[sec:nm_tab_sub_ice_shelf_2D]{config\_sub\_ice\_shelf\_2D\_cavity\_thickness}}
\label{subsec:nm_sec_config_sub_ice_shelf_2D_cavity_thickness}
\begin{center}
\begin{longtable}{| p{2.0in} || p{4.0in} |}
    \hline
    Type: & real \\
    \hline
    Units: & \si{m} \\
    \hline
    Default Value: & 25.0 \\
    \hline
    Possible Values: & Any positive real number. \\
    \hline
    \caption{config\_sub\_ice\_shelf\_2D\_cavity\_thickness: Vertical thickness of ocean sub-ice cavity.}
\end{longtable}
\end{center}
\subsection[config\_sub\_ice\_shelf\_2D\_slope\_height]{\hyperref[sec:nm_tab_sub_ice_shelf_2D]{config\_sub\_ice\_shelf\_2D\_slope\_height}}
\label{subsec:nm_sec_config_sub_ice_shelf_2D_slope_height}
\begin{center}
\begin{longtable}{| p{2.0in} || p{4.0in} |}
    \hline
    Type: & real \\
    \hline
    Units: & \si{m} \\
    \hline
    Default Value: & 500.0 \\
    \hline
    Possible Values: & Any positive real number. \\
    \hline
    \caption{config\_sub\_ice\_shelf\_2D\_slope\_height: Vertical thickness of fixed slope.}
\end{longtable}
\end{center}
\subsection[config\_sub\_ice\_shelf\_2D\_edge\_width]{\hyperref[sec:nm_tab_sub_ice_shelf_2D]{config\_sub\_ice\_shelf\_2D\_edge\_width}}
\label{subsec:nm_sec_config_sub_ice_shelf_2D_edge_width}
\begin{center}
\begin{longtable}{| p{2.0in} || p{4.0in} |}
    \hline
    Type: & real \\
    \hline
    Units: & \si{m} \\
    \hline
    Default Value: & 15.0e3 \\
    \hline
    Possible Values: & Any positive real number. \\
    \hline
    \caption{config\_sub\_ice\_shelf\_2D\_edge\_width: Horizontal width of angled part of the ice.}
\end{longtable}
\end{center}
\subsection[config\_sub\_ice\_shelf\_2D\_y1]{\hyperref[sec:nm_tab_sub_ice_shelf_2D]{config\_sub\_ice\_shelf\_2D\_y1}}
\label{subsec:nm_sec_config_sub_ice_shelf_2D_y1}
\begin{center}
\begin{longtable}{| p{2.0in} || p{4.0in} |}
    \hline
    Type: & real \\
    \hline
    Units: & \si{m} \\
    \hline
    Default Value: & 30.0e3 \\
    \hline
    Possible Values: & Any positive real number. \\
    \hline
    \caption{config\_sub\_ice\_shelf\_2D\_y1: cavity edge in y}
\end{longtable}
\end{center}
\subsection[config\_sub\_ice\_shelf\_2D\_y2]{\hyperref[sec:nm_tab_sub_ice_shelf_2D]{config\_sub\_ice\_shelf\_2D\_y2}}
\label{subsec:nm_sec_config_sub_ice_shelf_2D_y2}
\begin{center}
\begin{longtable}{| p{2.0in} || p{4.0in} |}
    \hline
    Type: & real \\
    \hline
    Units: & \si{m} \\
    \hline
    Default Value: & 60.0e3 \\
    \hline
    Possible Values: & Any positive real number. \\
    \hline
    \caption{config\_sub\_ice\_shelf\_2D\_y2: shelf edge in y}
\end{longtable}
\end{center}
\subsection[config\_sub\_ice\_shelf\_2D\_temperature]{\hyperref[sec:nm_tab_sub_ice_shelf_2D]{config\_sub\_ice\_shelf\_2D\_temperature}}
\label{subsec:nm_sec_config_sub_ice_shelf_2D_temperature}
\begin{center}
\begin{longtable}{| p{2.0in} || p{4.0in} |}
    \hline
    Type: & real \\
    \hline
    Units: & \si{deg.C} \\
    \hline
    Default Value: & 1.0 \\
    \hline
    Possible Values: & Any real number \\
    \hline
    \caption{config\_sub\_ice\_shelf\_2D\_temperature: Temperature of the surface in the northern half of the domain.}
\end{longtable}
\end{center}
\subsection[config\_sub\_ice\_shelf\_2D\_surface\_salinity]{\hyperref[sec:nm_tab_sub_ice_shelf_2D]{config\_sub\_ice\_shelf\_2D\_surface\_salinity}}
\label{subsec:nm_sec_config_sub_ice_shelf_2D_surface_salinity}
\begin{center}
\begin{longtable}{| p{2.0in} || p{4.0in} |}
    \hline
    Type: & real \\
    \hline
    Units: & \si{PSU} \\
    \hline
    Default Value: & 34.5 \\
    \hline
    Possible Values: & Any real number greater than 0. \\
    \hline
    \caption{config\_sub\_ice\_shelf\_2D\_surface\_salinity: Salinity of the water in the entire domain.}
\end{longtable}
\end{center}
\subsection[config\_sub\_ice\_shelf\_2D\_bottom\_salinity]{\hyperref[sec:nm_tab_sub_ice_shelf_2D]{config\_sub\_ice\_shelf\_2D\_bottom\_salinity}}
\label{subsec:nm_sec_config_sub_ice_shelf_2D_bottom_salinity}
\begin{center}
\begin{longtable}{| p{2.0in} || p{4.0in} |}
    \hline
    Type: & real \\
    \hline
    Units: & \si{PSU} \\
    \hline
    Default Value: & 34.7 \\
    \hline
    Possible Values: & Any real number greater than 0. \\
    \hline
    \caption{config\_sub\_ice\_shelf\_2D\_bottom\_salinity: Salinity of the water in the entire domain.}
\end{longtable}
\end{center}
\section[periodic\_planar]{\hyperref[sec:nm_tab_periodic_planar]{periodic\_planar}}
\label{sec:nm_sec_periodic_planar}
\subsection[config\_periodic\_planar\_vert\_levels]{\hyperref[sec:nm_tab_periodic_planar]{config\_periodic\_planar\_vert\_levels}}
\label{subsec:nm_sec_config_periodic_planar_vert_levels}
\begin{center}
\begin{longtable}{| p{2.0in} || p{4.0in} |}
    \hline
    Type: & integer \\
    \hline
    Units: & \si{unitless} \\
    \hline
    Default Value: & 100 \\
    \hline
    Possible Values: & Any positive integer number greater than 0. \\
    \hline
    \caption{config\_periodic\_planar\_vert\_levels: Number of vertical levels in periodic\_planar. Typical value is 1.}
\end{longtable}
\end{center}
\subsection[config\_periodic\_planar\_bottom\_depth]{\hyperref[sec:nm_tab_periodic_planar]{config\_periodic\_planar\_bottom\_depth}}
\label{subsec:nm_sec_config_periodic_planar_bottom_depth}
\begin{center}
\begin{longtable}{| p{2.0in} || p{4.0in} |}
    \hline
    Type: & real \\
    \hline
    Units: & \si{unitless} \\
    \hline
    Default Value: & 2500.0 \\
    \hline
    Possible Values: & Any real number \\
    \hline
    \caption{config\_periodic\_planar\_bottom\_depth: Bottom depth.}
\end{longtable}
\end{center}
\subsection[config\_periodic\_planar\_velocity\_strength]{\hyperref[sec:nm_tab_periodic_planar]{config\_periodic\_planar\_velocity\_strength}}
\label{subsec:nm_sec_config_periodic_planar_velocity_strength}
\begin{center}
\begin{longtable}{| p{2.0in} || p{4.0in} |}
    \hline
    Type: & real \\
    \hline
    Units: & \si{m^2.s^{-1}} \\
    \hline
    Default Value: & 1.0 \\
    \hline
    Possible Values: & Any real number \\
    \hline
    \caption{config\_periodic\_planar\_velocity\_strength: Strenght of velocity field.}
\end{longtable}
\end{center}
\section[ecosys\_column]{\hyperref[sec:nm_tab_ecosys_column]{ecosys\_column}}
\label{sec:nm_sec_ecosys_column}
\subsection[config\_ecosys\_column\_vert\_levels]{\hyperref[sec:nm_tab_ecosys_column]{config\_ecosys\_column\_vert\_levels}}
\label{subsec:nm_sec_config_ecosys_column_vert_levels}
\begin{center}
\begin{longtable}{| p{2.0in} || p{4.0in} |}
    \hline
    Type: & integer \\
    \hline
    Units: & \si{unitless} \\
    \hline
    Default Value: & 100 \\
    \hline
    Possible Values: & 100 \\
    \hline
    \caption{config\_ecosys\_column\_vert\_levels: Number of vertical levels in ecosys column unit test case.}
\end{longtable}
\end{center}
\subsection[config\_ecosys\_column\_vertical\_grid]{\hyperref[sec:nm_tab_ecosys_column]{config\_ecosys\_column\_vertical\_grid}}
\label{subsec:nm_sec_config_ecosys_column_vertical_grid}
\begin{center}
\begin{longtable}{| p{2.0in} || p{4.0in} |}
    \hline
    Type: & character \\
    \hline
    Units: & \si{unitless} \\
    \hline
    Default Value: & 100layerE3SMv1 \\
    \hline
    Possible Values: & '100layerE3SMv1' \\
    \hline
    \caption{config\_ecosys\_column\_vertical\_grid: prescription of setting the vertical resolution of the test case}
\end{longtable}
\end{center}
\subsection[config\_ecosys\_column\_TS\_filename]{\hyperref[sec:nm_tab_ecosys_column]{config\_ecosys\_column\_TS\_filename}}
\label{subsec:nm_sec_config_ecosys_column_TS_filename}
\begin{center}
\begin{longtable}{| p{2.0in} || p{4.0in} |}
    \hline
    Type: & character \\
    \hline
    Units: & \si{unitless} \\
    \hline
    Default Value: & unknown \\
    \hline
    Possible Values: & path/PTandS.mpas100levs.singleColumn.forMPASO.renamed.nc \\
    \hline
    \caption{config\_ecosys\_column\_TS\_filename: Name of file containing column values of temperature and salinity}
\end{longtable}
\end{center}
\subsection[config\_ecosys\_column\_ecosys\_filename]{\hyperref[sec:nm_tab_ecosys_column]{config\_ecosys\_column\_ecosys\_filename}}
\label{subsec:nm_sec_config_ecosys_column_ecosys_filename}
\begin{center}
\begin{longtable}{| p{2.0in} || p{4.0in} |}
    \hline
    Type: & character \\
    \hline
    Units: & \si{unitless} \\
    \hline
    Default Value: & unknown \\
    \hline
    Possible Values: & path/ecoIC+phaeo.mpas100levs.singleColumn.forMPASO.renamed.nc \\
    \hline
    \caption{config\_ecosys\_column\_ecosys\_filename: Name of file containing column values of ecosys variables}
\end{longtable}
\end{center}
\subsection[config\_ecosys\_column\_bottom\_depth]{\hyperref[sec:nm_tab_ecosys_column]{config\_ecosys\_column\_bottom\_depth}}
\label{subsec:nm_sec_config_ecosys_column_bottom_depth}
\begin{center}
\begin{longtable}{| p{2.0in} || p{4.0in} |}
    \hline
    Type: & real \\
    \hline
    Units: & \si{m} \\
    \hline
    Default Value: & 6000.0 \\
    \hline
    Possible Values: & 6000. \\
    \hline
    \caption{config\_ecosys\_column\_bottom\_depth: Depth of the bottom of the ocean for the ecosys column unit test case.}
\end{longtable}
\end{center}
\section[sea\_mount]{\hyperref[sec:nm_tab_sea_mount]{sea\_mount}}
\label{sec:nm_sec_sea_mount}
\subsection[config\_sea\_mount\_vert\_levels]{\hyperref[sec:nm_tab_sea_mount]{config\_sea\_mount\_vert\_levels}}
\label{subsec:nm_sec_config_sea_mount_vert_levels}
\begin{center}
\begin{longtable}{| p{2.0in} || p{4.0in} |}
    \hline
    Type: & integer \\
    \hline
    Units: & \si{unitless} \\
    \hline
    Default Value: & 10 \\
    \hline
    Possible Values: & Any positive integer number greater than 0. \\
    \hline
    \caption{config\_sea\_mount\_vert\_levels: Number of vertical levels in sea mount test case.}
\end{longtable}
\end{center}
\subsection[config\_sea\_mount\_layer\_type]{\hyperref[sec:nm_tab_sea_mount]{config\_sea\_mount\_layer\_type}}
\label{subsec:nm_sec_config_sea_mount_layer_type}
\begin{center}
\begin{longtable}{| p{2.0in} || p{4.0in} |}
    \hline
    Type: & character \\
    \hline
    Units: & \si{unitless} \\
    \hline
    Default Value: & sigma \\
    \hline
    Possible Values: & 'z-level', 'sigma' \\
    \hline
    \caption{config\_sea\_mount\_layer\_type: Logical flag that controls the vertical coordinate initializaton}
\end{longtable}
\end{center}
\subsection[config\_sea\_mount\_stratification\_type]{\hyperref[sec:nm_tab_sea_mount]{config\_sea\_mount\_stratification\_type}}
\label{subsec:nm_sec_config_sea_mount_stratification_type}
\begin{center}
\begin{longtable}{| p{2.0in} || p{4.0in} |}
    \hline
    Type: & character \\
    \hline
    Units: & \si{unitless} \\
    \hline
    Default Value: & exponential \\
    \hline
    Possible Values: & 'linear', 'exponential' \\
    \hline
    \caption{config\_sea\_mount\_stratification\_type: Logical flag that controls how the vertical profile of tracers.  See Beckmann and Haidvogel 1993 eqn 15-16.}
\end{longtable}
\end{center}
\subsection[config\_sea\_mount\_density\_coef\_linear]{\hyperref[sec:nm_tab_sea_mount]{config\_sea\_mount\_density\_coef\_linear}}
\label{subsec:nm_sec_config_sea_mount_density_coef_linear}
\begin{center}
\begin{longtable}{| p{2.0in} || p{4.0in} |}
    \hline
    Type: & real \\
    \hline
    Units: & \si{kg.m^{-3}} \\
    \hline
    Default Value: & 1024 \\
    \hline
    Possible Values: & Any real number \\
    \hline
    \caption{config\_sea\_mount\_density\_coef\_linear: Density coefficient for linear vertical stratification}
\end{longtable}
\end{center}
\subsection[config\_sea\_mount\_density\_coef\_exp]{\hyperref[sec:nm_tab_sea_mount]{config\_sea\_mount\_density\_coef\_exp}}
\label{subsec:nm_sec_config_sea_mount_density_coef_exp}
\begin{center}
\begin{longtable}{| p{2.0in} || p{4.0in} |}
    \hline
    Type: & real \\
    \hline
    Units: & \si{kg.m^{-3}} \\
    \hline
    Default Value: & 1028 \\
    \hline
    Possible Values: & Any real number \\
    \hline
    \caption{config\_sea\_mount\_density\_coef\_exp: Density coefficient for exponential vertical stratification}
\end{longtable}
\end{center}
\subsection[config\_sea\_mount\_density\_gradient\_linear]{\hyperref[sec:nm_tab_sea_mount]{config\_sea\_mount\_density\_gradient\_linear}}
\label{subsec:nm_sec_config_sea_mount_density_gradient_linear}
\begin{center}
\begin{longtable}{| p{2.0in} || p{4.0in} |}
    \hline
    Type: & real \\
    \hline
    Units: & \si{kg.m^{-3}} \\
    \hline
    Default Value: & 0.1 \\
    \hline
    Possible Values: & Any real number \\
    \hline
    \caption{config\_sea\_mount\_density\_gradient\_linear: Density gradient for linear vertical stratification, $\Delta_z \rho$ in Beckmann Haidvogel eqn 15}
\end{longtable}
\end{center}
\subsection[config\_sea\_mount\_density\_gradient\_exp]{\hyperref[sec:nm_tab_sea_mount]{config\_sea\_mount\_density\_gradient\_exp}}
\label{subsec:nm_sec_config_sea_mount_density_gradient_exp}
\begin{center}
\begin{longtable}{| p{2.0in} || p{4.0in} |}
    \hline
    Type: & real \\
    \hline
    Units: & \si{kg.m^{-3}} \\
    \hline
    Default Value: & 3.0 \\
    \hline
    Possible Values: & Any real number \\
    \hline
    \caption{config\_sea\_mount\_density\_gradient\_exp: Density gradient for exponential vertical stratification, $\Delta_z \rho$ in Beckmann Haidvogel eqn 16}
\end{longtable}
\end{center}
\subsection[config\_sea\_mount\_density\_depth\_linear]{\hyperref[sec:nm_tab_sea_mount]{config\_sea\_mount\_density\_depth\_linear}}
\label{subsec:nm_sec_config_sea_mount_density_depth_linear}
\begin{center}
\begin{longtable}{| p{2.0in} || p{4.0in} |}
    \hline
    Type: & real \\
    \hline
    Units: & \si{m} \\
    \hline
    Default Value: & 4500 \\
    \hline
    Possible Values: & Any real number \\
    \hline
    \caption{config\_sea\_mount\_density\_depth\_linear: Density reference depth for linear vertical stratification}
\end{longtable}
\end{center}
\subsection[config\_sea\_mount\_density\_depth\_exp]{\hyperref[sec:nm_tab_sea_mount]{config\_sea\_mount\_density\_depth\_exp}}
\label{subsec:nm_sec_config_sea_mount_density_depth_exp}
\begin{center}
\begin{longtable}{| p{2.0in} || p{4.0in} |}
    \hline
    Type: & real \\
    \hline
    Units: & \si{m} \\
    \hline
    Default Value: & 500 \\
    \hline
    Possible Values: & Any real number \\
    \hline
    \caption{config\_sea\_mount\_density\_depth\_exp: Density reference depth for exponential vertical stratification}
\end{longtable}
\end{center}
\subsection[config\_sea\_mount\_density\_ref]{\hyperref[sec:nm_tab_sea_mount]{config\_sea\_mount\_density\_ref}}
\label{subsec:nm_sec_config_sea_mount_density_ref}
\begin{center}
\begin{longtable}{| p{2.0in} || p{4.0in} |}
    \hline
    Type: & real \\
    \hline
    Units: & \si{kg.m^{-3}} \\
    \hline
    Default Value: & 1028 \\
    \hline
    Possible Values: & Any real number \\
    \hline
    \caption{config\_sea\_mount\_density\_ref: Density reference for eos to initialize temperature}
\end{longtable}
\end{center}
\subsection[config\_sea\_mount\_density\_Tref]{\hyperref[sec:nm_tab_sea_mount]{config\_sea\_mount\_density\_Tref}}
\label{subsec:nm_sec_config_sea_mount_density_Tref}
\begin{center}
\begin{longtable}{| p{2.0in} || p{4.0in} |}
    \hline
    Type: & real \\
    \hline
    Units: & \si{C} \\
    \hline
    Default Value: & 5.0 \\
    \hline
    Possible Values: & Any real number \\
    \hline
    \caption{config\_sea\_mount\_density\_Tref: Reference temperature for eos to initialize temperature}
\end{longtable}
\end{center}
\subsection[config\_sea\_mount\_density\_alpha]{\hyperref[sec:nm_tab_sea_mount]{config\_sea\_mount\_density\_alpha}}
\label{subsec:nm_sec_config_sea_mount_density_alpha}
\begin{center}
\begin{longtable}{| p{2.0in} || p{4.0in} |}
    \hline
    Type: & real \\
    \hline
    Units: & \si{kg.m^{-3}.C^{-1}} \\
    \hline
    Default Value: & 0.2 \\
    \hline
    Possible Values: & Any real number \\
    \hline
    \caption{config\_sea\_mount\_density\_alpha: Linear thermal expansion coefficient to initialize temperature}
\end{longtable}
\end{center}
\subsection[config\_sea\_mount\_bottom\_depth]{\hyperref[sec:nm_tab_sea_mount]{config\_sea\_mount\_bottom\_depth}}
\label{subsec:nm_sec_config_sea_mount_bottom_depth}
\begin{center}
\begin{longtable}{| p{2.0in} || p{4.0in} |}
    \hline
    Type: & real \\
    \hline
    Units: & \si{m} \\
    \hline
    Default Value: & 5000.0 \\
    \hline
    Possible Values: & Any positive real number. \\
    \hline
    \caption{config\_sea\_mount\_bottom\_depth: Depth of the bottom of the ocean for the sea mount test case.}
\end{longtable}
\end{center}
\subsection[config\_sea\_mount\_height]{\hyperref[sec:nm_tab_sea_mount]{config\_sea\_mount\_height}}
\label{subsec:nm_sec_config_sea_mount_height}
\begin{center}
\begin{longtable}{| p{2.0in} || p{4.0in} |}
    \hline
    Type: & real \\
    \hline
    Units: & \si{m} \\
    \hline
    Default Value: & 4500.0 \\
    \hline
    Possible Values: & Any positive real number. \\
    \hline
    \caption{config\_sea\_mount\_height: Height of sea mount, $H_0$}
\end{longtable}
\end{center}
\subsection[config\_sea\_mount\_radius]{\hyperref[sec:nm_tab_sea_mount]{config\_sea\_mount\_radius}}
\label{subsec:nm_sec_config_sea_mount_radius}
\begin{center}
\begin{longtable}{| p{2.0in} || p{4.0in} |}
    \hline
    Type: & real \\
    \hline
    Units: & \si{m} \\
    \hline
    Default Value: & 10.0e3 \\
    \hline
    Possible Values: & Any positive real number. \\
    \hline
    \caption{config\_sea\_mount\_radius: Radius of sea mount}
\end{longtable}
\end{center}
\subsection[config\_sea\_mount\_width]{\hyperref[sec:nm_tab_sea_mount]{config\_sea\_mount\_width}}
\label{subsec:nm_sec_config_sea_mount_width}
\begin{center}
\begin{longtable}{| p{2.0in} || p{4.0in} |}
    \hline
    Type: & real \\
    \hline
    Units: & \si{m} \\
    \hline
    Default Value: & 40.0e3 \\
    \hline
    Possible Values: & Any positive real number. \\
    \hline
    \caption{config\_sea\_mount\_width: Width parameter of sea mount, $L$.}
\end{longtable}
\end{center}
\subsection[config\_sea\_mount\_salinity]{\hyperref[sec:nm_tab_sea_mount]{config\_sea\_mount\_salinity}}
\label{subsec:nm_sec_config_sea_mount_salinity}
\begin{center}
\begin{longtable}{| p{2.0in} || p{4.0in} |}
    \hline
    Type: & real \\
    \hline
    Units: & \si{PSU} \\
    \hline
    Default Value: & 35.0 \\
    \hline
    Possible Values: & Any real number greater than 0. \\
    \hline
    \caption{config\_sea\_mount\_salinity: Salinity of the water in the entire domain.}
\end{longtable}
\end{center}
\subsection[config\_sea\_mount\_coriolis\_parameter]{\hyperref[sec:nm_tab_sea_mount]{config\_sea\_mount\_coriolis\_parameter}}
\label{subsec:nm_sec_config_sea_mount_coriolis_parameter}
\begin{center}
\begin{longtable}{| p{2.0in} || p{4.0in} |}
    \hline
    Type: & real \\
    \hline
    Units: & \si{s^{-1}} \\
    \hline
    Default Value: & -1.0e-4 \\
    \hline
    Possible Values: & Any real number. \\
    \hline
    \caption{config\_sea\_mount\_coriolis\_parameter: Coriolis parameter for entrie domain.}
\end{longtable}
\end{center}
\section[isomip]{\hyperref[sec:nm_tab_isomip]{isomip}}
\label{sec:nm_sec_isomip}
\subsection[config\_isomip\_vert\_levels]{\hyperref[sec:nm_tab_isomip]{config\_isomip\_vert\_levels}}
\label{subsec:nm_sec_config_isomip_vert_levels}
\begin{center}
\begin{longtable}{| p{2.0in} || p{4.0in} |}
    \hline
    Type: & integer \\
    \hline
    Units: & \si{unitless} \\
    \hline
    Default Value: & 30 \\
    \hline
    Possible Values: & Any integer greater than 0. \\
    \hline
    \caption{config\_isomip\_vert\_levels: Number of vertical levels in test case.}
\end{longtable}
\end{center}
\subsection[config\_isomip\_vertical\_level\_distribution]{\hyperref[sec:nm_tab_isomip]{config\_isomip\_vertical\_level\_distribution}}
\label{subsec:nm_sec_config_isomip_vertical_level_distribution}
\begin{center}
\begin{longtable}{| p{2.0in} || p{4.0in} |}
    \hline
    Type: & character \\
    \hline
    Units: & \si{unitless} \\
    \hline
    Default Value: & constant \\
    \hline
    Possible Values: & 'constant', 'boundary\_layer' \\
    \hline
    \caption{config\_isomip\_vertical\_level\_distribution: The distribution of vertical levels, either constant (all equal thickness) or boundary layer (decreasing toward top and bottom).}
\end{longtable}
\end{center}
\subsection[config\_isomip\_bottom\_depth]{\hyperref[sec:nm_tab_isomip]{config\_isomip\_bottom\_depth}}
\label{subsec:nm_sec_config_isomip_bottom_depth}
\begin{center}
\begin{longtable}{| p{2.0in} || p{4.0in} |}
    \hline
    Type: & real \\
    \hline
    Units: & \si{m} \\
    \hline
    Default Value: & -900.0 \\
    \hline
    Possible Values: & Any negative real number. \\
    \hline
    \caption{config\_isomip\_bottom\_depth: Depth of the ocean in the test case.}
\end{longtable}
\end{center}
\subsection[config\_isomip\_temperature]{\hyperref[sec:nm_tab_isomip]{config\_isomip\_temperature}}
\label{subsec:nm_sec_config_isomip_temperature}
\begin{center}
\begin{longtable}{| p{2.0in} || p{4.0in} |}
    \hline
    Type: & real \\
    \hline
    Units: & \si{deg.C} \\
    \hline
    Default Value: & -1.9 \\
    \hline
    Possible Values: & Any real number. \\
    \hline
    \caption{config\_isomip\_temperature: Temperature of the ocean for isomip initial conditions.}
\end{longtable}
\end{center}
\subsection[config\_isomip\_salinity]{\hyperref[sec:nm_tab_isomip]{config\_isomip\_salinity}}
\label{subsec:nm_sec_config_isomip_salinity}
\begin{center}
\begin{longtable}{| p{2.0in} || p{4.0in} |}
    \hline
    Type: & real \\
    \hline
    Units: & \si{PSU} \\
    \hline
    Default Value: & 34.4 \\
    \hline
    Possible Values: & Any real number greater than 0. \\
    \hline
    \caption{config\_isomip\_salinity: Salinity of the ocean for isomip initial conditions.}
\end{longtable}
\end{center}
\subsection[config\_isomip\_restoring\_temperature]{\hyperref[sec:nm_tab_isomip]{config\_isomip\_restoring\_temperature}}
\label{subsec:nm_sec_config_isomip_restoring_temperature}
\begin{center}
\begin{longtable}{| p{2.0in} || p{4.0in} |}
    \hline
    Type: & real \\
    \hline
    Units: & \si{deg.C} \\
    \hline
    Default Value: & -1.9 \\
    \hline
    Possible Values: & Any real number. \\
    \hline
    \caption{config\_isomip\_restoring\_temperature: Temperature for surface restoring.}
\end{longtable}
\end{center}
\subsection[config\_isomip\_temperature\_piston\_velocity]{\hyperref[sec:nm_tab_isomip]{config\_isomip\_temperature\_piston\_velocity}}
\label{subsec:nm_sec_config_isomip_temperature_piston_velocity}
\begin{center}
\begin{longtable}{| p{2.0in} || p{4.0in} |}
    \hline
    Type: & real \\
    \hline
    Units: & \si{m.s^{-1}} \\
    \hline
    Default Value: & 1.157e-5 \\
    \hline
    Possible Values: & Any positive real number. \\
    \hline
    \caption{config\_isomip\_temperature\_piston\_velocity: Piston velocity for surface restoring of temperature}
\end{longtable}
\end{center}
\subsection[config\_isomip\_restoring\_salinity]{\hyperref[sec:nm_tab_isomip]{config\_isomip\_restoring\_salinity}}
\label{subsec:nm_sec_config_isomip_restoring_salinity}
\begin{center}
\begin{longtable}{| p{2.0in} || p{4.0in} |}
    \hline
    Type: & real \\
    \hline
    Units: & \si{PSU} \\
    \hline
    Default Value: & 34.4 \\
    \hline
    Possible Values: & Any real number greater than 0. \\
    \hline
    \caption{config\_isomip\_restoring\_salinity: Salinity for surface restoring.}
\end{longtable}
\end{center}
\subsection[config\_isomip\_salinity\_piston\_velocity]{\hyperref[sec:nm_tab_isomip]{config\_isomip\_salinity\_piston\_velocity}}
\label{subsec:nm_sec_config_isomip_salinity_piston_velocity}
\begin{center}
\begin{longtable}{| p{2.0in} || p{4.0in} |}
    \hline
    Type: & real \\
    \hline
    Units: & \si{m.s^{-1}} \\
    \hline
    Default Value: & 1.157e-5 \\
    \hline
    Possible Values: & Any positive real number. \\
    \hline
    \caption{config\_isomip\_salinity\_piston\_velocity: Piston velocity for surface restoring of salinity}
\end{longtable}
\end{center}
\subsection[config\_isomip\_coriolis\_parameter]{\hyperref[sec:nm_tab_isomip]{config\_isomip\_coriolis\_parameter}}
\label{subsec:nm_sec_config_isomip_coriolis_parameter}
\begin{center}
\begin{longtable}{| p{2.0in} || p{4.0in} |}
    \hline
    Type: & real \\
    \hline
    Units: & \si{s^{-1}} \\
    \hline
    Default Value: & -1.4e-4 \\
    \hline
    Possible Values: & Any real number. \\
    \hline
    \caption{config\_isomip\_coriolis\_parameter: Coriolis parameter for entrie domain.}
\end{longtable}
\end{center}
\subsection[config\_isomip\_southern\_boundary]{\hyperref[sec:nm_tab_isomip]{config\_isomip\_southern\_boundary}}
\label{subsec:nm_sec_config_isomip_southern_boundary}
\begin{center}
\begin{longtable}{| p{2.0in} || p{4.0in} |}
    \hline
    Type: & real \\
    \hline
    Units: & \si{m} \\
    \hline
    Default Value: & 0.0 \\
    \hline
    Possible Values: & Any real number. \\
    \hline
    \caption{config\_isomip\_southern\_boundary: The y location of the southern boundary.}
\end{longtable}
\end{center}
\subsection[config\_isomip\_northern\_boundary]{\hyperref[sec:nm_tab_isomip]{config\_isomip\_northern\_boundary}}
\label{subsec:nm_sec_config_isomip_northern_boundary}
\begin{center}
\begin{longtable}{| p{2.0in} || p{4.0in} |}
    \hline
    Type: & real \\
    \hline
    Units: & \si{m} \\
    \hline
    Default Value: & 1000e3 \\
    \hline
    Possible Values: & Any real number. \\
    \hline
    \caption{config\_isomip\_northern\_boundary: The y location of the northern boundary.}
\end{longtable}
\end{center}
\subsection[config\_isomip\_western\_boundary]{\hyperref[sec:nm_tab_isomip]{config\_isomip\_western\_boundary}}
\label{subsec:nm_sec_config_isomip_western_boundary}
\begin{center}
\begin{longtable}{| p{2.0in} || p{4.0in} |}
    \hline
    Type: & real \\
    \hline
    Units: & \si{m} \\
    \hline
    Default Value: & 0.0 \\
    \hline
    Possible Values: & Any real number. \\
    \hline
    \caption{config\_isomip\_western\_boundary: The x location of the western boundary.}
\end{longtable}
\end{center}
\subsection[config\_isomip\_eastern\_boundary]{\hyperref[sec:nm_tab_isomip]{config\_isomip\_eastern\_boundary}}
\label{subsec:nm_sec_config_isomip_eastern_boundary}
\begin{center}
\begin{longtable}{| p{2.0in} || p{4.0in} |}
    \hline
    Type: & real \\
    \hline
    Units: & \si{m} \\
    \hline
    Default Value: & 500e3 \\
    \hline
    Possible Values: & Any real number. \\
    \hline
    \caption{config\_isomip\_eastern\_boundary: The x location of the eastern boundary.}
\end{longtable}
\end{center}
\subsection[config\_isomip\_y1]{\hyperref[sec:nm_tab_isomip]{config\_isomip\_y1}}
\label{subsec:nm_sec_config_isomip_y1}
\begin{center}
\begin{longtable}{| p{2.0in} || p{4.0in} |}
    \hline
    Type: & real \\
    \hline
    Units: & \si{m} \\
    \hline
    Default Value: & 0.0 \\
    \hline
    Possible Values: & Any real number, between -90 and 90 on spherical meshes. \\
    \hline
    \caption{config\_isomip\_y1: The first y value in the piecewise linear ice draft.}
\end{longtable}
\end{center}
\subsection[config\_isomip\_z1]{\hyperref[sec:nm_tab_isomip]{config\_isomip\_z1}}
\label{subsec:nm_sec_config_isomip_z1}
\begin{center}
\begin{longtable}{| p{2.0in} || p{4.0in} |}
    \hline
    Type: & real \\
    \hline
    Units: & \si{m} \\
    \hline
    Default Value: & -700.0 \\
    \hline
    Possible Values: & Any non-positive real number. \\
    \hline
    \caption{config\_isomip\_z1: The first z value in the piecewise linear ice draft.}
\end{longtable}
\end{center}
\subsection[config\_isomip\_ice\_fraction1]{\hyperref[sec:nm_tab_isomip]{config\_isomip\_ice\_fraction1}}
\label{subsec:nm_sec_config_isomip_ice_fraction1}
\begin{center}
\begin{longtable}{| p{2.0in} || p{4.0in} |}
    \hline
    Type: & real \\
    \hline
    Units: & \si{unitless} \\
    \hline
    Default Value: & 1.0 \\
    \hline
    Possible Values: & A real number between 0 and 1. \\
    \hline
    \caption{config\_isomip\_ice\_fraction1: The first ice fraction value in the piecewise linear fit.}
\end{longtable}
\end{center}
\subsection[config\_isomip\_y2]{\hyperref[sec:nm_tab_isomip]{config\_isomip\_y2}}
\label{subsec:nm_sec_config_isomip_y2}
\begin{center}
\begin{longtable}{| p{2.0in} || p{4.0in} |}
    \hline
    Type: & real \\
    \hline
    Units: & \si{m} \\
    \hline
    Default Value: & 400e3 \\
    \hline
    Possible Values: & Any real number. \\
    \hline
    \caption{config\_isomip\_y2: The second y value in the piecewise linear ice draft.}
\end{longtable}
\end{center}
\subsection[config\_isomip\_z2]{\hyperref[sec:nm_tab_isomip]{config\_isomip\_z2}}
\label{subsec:nm_sec_config_isomip_z2}
\begin{center}
\begin{longtable}{| p{2.0in} || p{4.0in} |}
    \hline
    Type: & real \\
    \hline
    Units: & \si{m} \\
    \hline
    Default Value: & -200.0 \\
    \hline
    Possible Values: & Any non-positive real number. \\
    \hline
    \caption{config\_isomip\_z2: The second z value in the piecewise linear.}
\end{longtable}
\end{center}
\subsection[config\_isomip\_ice\_fraction2]{\hyperref[sec:nm_tab_isomip]{config\_isomip\_ice\_fraction2}}
\label{subsec:nm_sec_config_isomip_ice_fraction2}
\begin{center}
\begin{longtable}{| p{2.0in} || p{4.0in} |}
    \hline
    Type: & real \\
    \hline
    Units: & \si{unitless} \\
    \hline
    Default Value: & 1.0 \\
    \hline
    Possible Values: & A real number between 0 and 1. \\
    \hline
    \caption{config\_isomip\_ice\_fraction2: The second ice fraction value in the piecewise linear fit.}
\end{longtable}
\end{center}
\subsection[config\_isomip\_y3]{\hyperref[sec:nm_tab_isomip]{config\_isomip\_y3}}
\label{subsec:nm_sec_config_isomip_y3}
\begin{center}
\begin{longtable}{| p{2.0in} || p{4.0in} |}
    \hline
    Type: & real \\
    \hline
    Units: & \si{m} \\
    \hline
    Default Value: & 1000e3 \\
    \hline
    Possible Values: & Any real number. \\
    \hline
    \caption{config\_isomip\_y3: The third y value in the piecewise linear ice draft.}
\end{longtable}
\end{center}
\subsection[config\_isomip\_z3]{\hyperref[sec:nm_tab_isomip]{config\_isomip\_z3}}
\label{subsec:nm_sec_config_isomip_z3}
\begin{center}
\begin{longtable}{| p{2.0in} || p{4.0in} |}
    \hline
    Type: & real \\
    \hline
    Units: & \si{m} \\
    \hline
    Default Value: & -200.0 \\
    \hline
    Possible Values: & Any non-positive real number. \\
    \hline
    \caption{config\_isomip\_z3: The third z value in the piecewise linear.}
\end{longtable}
\end{center}
\subsection[config\_isomip\_ice\_fraction3]{\hyperref[sec:nm_tab_isomip]{config\_isomip\_ice\_fraction3}}
\label{subsec:nm_sec_config_isomip_ice_fraction3}
\begin{center}
\begin{longtable}{| p{2.0in} || p{4.0in} |}
    \hline
    Type: & real \\
    \hline
    Units: & \si{unitless} \\
    \hline
    Default Value: & 1.0 \\
    \hline
    Possible Values: & A real number between 0 and 1. \\
    \hline
    \caption{config\_isomip\_ice\_fraction3: The third ice fraction value in the piecewise linear fit.}
\end{longtable}
\end{center}
\section[isomip\_plus]{\hyperref[sec:nm_tab_isomip_plus]{isomip\_plus}}
\label{sec:nm_sec_isomip_plus}
\subsection[config\_isomip\_plus\_vert\_levels]{\hyperref[sec:nm_tab_isomip_plus]{config\_isomip\_plus\_vert\_levels}}
\label{subsec:nm_sec_config_isomip_plus_vert_levels}
\begin{center}
\begin{longtable}{| p{2.0in} || p{4.0in} |}
    \hline
    Type: & integer \\
    \hline
    Units: & \si{unitless} \\
    \hline
    Default Value: & 36 \\
    \hline
    Possible Values: & Any integer greater than 0. \\
    \hline
    \caption{config\_isomip\_plus\_vert\_levels: Number of vertical levels in test case.}
\end{longtable}
\end{center}
\subsection[config\_isomip\_plus\_vertical\_level\_distribution]{\hyperref[sec:nm_tab_isomip_plus]{config\_isomip\_plus\_vertical\_level\_distribution}}
\label{subsec:nm_sec_config_isomip_plus_vertical_level_distribution}
\begin{center}
\begin{longtable}{| p{2.0in} || p{4.0in} |}
    \hline
    Type: & character \\
    \hline
    Units: & \si{unitless} \\
    \hline
    Default Value: & constant \\
    \hline
    Possible Values: & 'constant' \\
    \hline
    \caption{config\_isomip\_plus\_vertical\_level\_distribution: The distribution of vertical levels, currently only constant (all equal thickness).}
\end{longtable}
\end{center}
\subsection[config\_isomip\_plus\_max\_bottom\_depth]{\hyperref[sec:nm_tab_isomip_plus]{config\_isomip\_plus\_max\_bottom\_depth}}
\label{subsec:nm_sec_config_isomip_plus_max_bottom_depth}
\begin{center}
\begin{longtable}{| p{2.0in} || p{4.0in} |}
    \hline
    Type: & real \\
    \hline
    Units: & \si{m} \\
    \hline
    Default Value: & -720.0 \\
    \hline
    Possible Values: & Any negative real number. \\
    \hline
    \caption{config\_isomip\_plus\_max\_bottom\_depth: Maximum depth of the ocean in the test case.}
\end{longtable}
\end{center}
\subsection[config\_isomip\_plus\_minimum\_levels]{\hyperref[sec:nm_tab_isomip_plus]{config\_isomip\_plus\_minimum\_levels}}
\label{subsec:nm_sec_config_isomip_plus_minimum_levels}
\begin{center}
\begin{longtable}{| p{2.0in} || p{4.0in} |}
    \hline
    Type: & integer \\
    \hline
    Units: & \si{unitless} \\
    \hline
    Default Value: & 3 \\
    \hline
    Possible Values: & Any positive integer value greater than 0. \\
    \hline
    \caption{config\_isomip\_plus\_minimum\_levels: Minimum number of vertical levels in a column.}
\end{longtable}
\end{center}
\subsection[config\_isomip\_plus\_min\_column\_thickness]{\hyperref[sec:nm_tab_isomip_plus]{config\_isomip\_plus\_min\_column\_thickness}}
\label{subsec:nm_sec_config_isomip_plus_min_column_thickness}
\begin{center}
\begin{longtable}{| p{2.0in} || p{4.0in} |}
    \hline
    Type: & real \\
    \hline
    Units: & \si{m} \\
    \hline
    Default Value: & 10.0 \\
    \hline
    Possible Values: & Any positive real value. \\
    \hline
    \caption{config\_isomip\_plus\_min\_column\_thickness: Minimum thickness of the inital ocean column (to prevent 'drying').}
\end{longtable}
\end{center}
\subsection[config\_isomip\_plus\_min\_ocean\_fraction]{\hyperref[sec:nm_tab_isomip_plus]{config\_isomip\_plus\_min\_ocean\_fraction}}
\label{subsec:nm_sec_config_isomip_plus_min_ocean_fraction}
\begin{center}
\begin{longtable}{| p{2.0in} || p{4.0in} |}
    \hline
    Type: & real \\
    \hline
    Units: & \si{unitless} \\
    \hline
    Default Value: & 0.5 \\
    \hline
    Possible Values: & Any positive real value. \\
    \hline
    \caption{config\_isomip\_plus\_min\_ocean\_fraction: Minimum fraction of a cell that contains ocean (as opposed to land or grounded land ice) in order for it to be an active ocean cell.}
\end{longtable}
\end{center}
\subsection[config\_isomip\_plus\_topography\_file]{\hyperref[sec:nm_tab_isomip_plus]{config\_isomip\_plus\_topography\_file}}
\label{subsec:nm_sec_config_isomip_plus_topography_file}
\begin{center}
\begin{longtable}{| p{2.0in} || p{4.0in} |}
    \hline
    Type: & character \\
    \hline
    Units: & \si{unitless} \\
    \hline
    Default Value: & input\_geometry\_processed.nc \\
    \hline
    Possible Values: & path/to/topography/file.nc \\
    \hline
    \caption{config\_isomip\_plus\_topography\_file: Path to the topography initial condition file.}
\end{longtable}
\end{center}
\subsection[config\_isomip\_plus\_init\_top\_temp]{\hyperref[sec:nm_tab_isomip_plus]{config\_isomip\_plus\_init\_top\_temp}}
\label{subsec:nm_sec_config_isomip_plus_init_top_temp}
\begin{center}
\begin{longtable}{| p{2.0in} || p{4.0in} |}
    \hline
    Type: & real \\
    \hline
    Units: & \si{^\circ.C} \\
    \hline
    Default Value: & -1.9 \\
    \hline
    Possible Values: & Any real number. \\
    \hline
    \caption{config\_isomip\_plus\_init\_top\_temp: Initial temperature at sea level.}
\end{longtable}
\end{center}
\subsection[config\_isomip\_plus\_init\_bot\_temp]{\hyperref[sec:nm_tab_isomip_plus]{config\_isomip\_plus\_init\_bot\_temp}}
\label{subsec:nm_sec_config_isomip_plus_init_bot_temp}
\begin{center}
\begin{longtable}{| p{2.0in} || p{4.0in} |}
    \hline
    Type: & real \\
    \hline
    Units: & \si{^\circ.C} \\
    \hline
    Default Value: & -1.9 \\
    \hline
    Possible Values: & Any real number. \\
    \hline
    \caption{config\_isomip\_plus\_init\_bot\_temp: Initial temperature in deepest cells.}
\end{longtable}
\end{center}
\subsection[config\_isomip\_plus\_init\_top\_sal]{\hyperref[sec:nm_tab_isomip_plus]{config\_isomip\_plus\_init\_top\_sal}}
\label{subsec:nm_sec_config_isomip_plus_init_top_sal}
\begin{center}
\begin{longtable}{| p{2.0in} || p{4.0in} |}
    \hline
    Type: & real \\
    \hline
    Units: & \si{PSU} \\
    \hline
    Default Value: & 33.8 \\
    \hline
    Possible Values: & Any positive real number. \\
    \hline
    \caption{config\_isomip\_plus\_init\_top\_sal: Initial salinity at sea level.}
\end{longtable}
\end{center}
\subsection[config\_isomip\_plus\_init\_bot\_sal]{\hyperref[sec:nm_tab_isomip_plus]{config\_isomip\_plus\_init\_bot\_sal}}
\label{subsec:nm_sec_config_isomip_plus_init_bot_sal}
\begin{center}
\begin{longtable}{| p{2.0in} || p{4.0in} |}
    \hline
    Type: & real \\
    \hline
    Units: & \si{PSU} \\
    \hline
    Default Value: & 34.5 \\
    \hline
    Possible Values: & Any positive real number. \\
    \hline
    \caption{config\_isomip\_plus\_init\_bot\_sal: Initial salinity in deepest cells.}
\end{longtable}
\end{center}
\subsection[config\_isomip\_plus\_restore\_top\_temp]{\hyperref[sec:nm_tab_isomip_plus]{config\_isomip\_plus\_restore\_top\_temp}}
\label{subsec:nm_sec_config_isomip_plus_restore_top_temp}
\begin{center}
\begin{longtable}{| p{2.0in} || p{4.0in} |}
    \hline
    Type: & real \\
    \hline
    Units: & \si{^\circ.C} \\
    \hline
    Default Value: & -1.9 \\
    \hline
    Possible Values: & Any real number. \\
    \hline
    \caption{config\_isomip\_plus\_restore\_top\_temp: Restoring temperature at sea level.}
\end{longtable}
\end{center}
\subsection[config\_isomip\_plus\_restore\_bot\_temp]{\hyperref[sec:nm_tab_isomip_plus]{config\_isomip\_plus\_restore\_bot\_temp}}
\label{subsec:nm_sec_config_isomip_plus_restore_bot_temp}
\begin{center}
\begin{longtable}{| p{2.0in} || p{4.0in} |}
    \hline
    Type: & real \\
    \hline
    Units: & \si{^\circ.C} \\
    \hline
    Default Value: & 1.0 \\
    \hline
    Possible Values: & Any real number. \\
    \hline
    \caption{config\_isomip\_plus\_restore\_bot\_temp: Restoring temperature in deepest cells.}
\end{longtable}
\end{center}
\subsection[config\_isomip\_plus\_restore\_top\_sal]{\hyperref[sec:nm_tab_isomip_plus]{config\_isomip\_plus\_restore\_top\_sal}}
\label{subsec:nm_sec_config_isomip_plus_restore_top_sal}
\begin{center}
\begin{longtable}{| p{2.0in} || p{4.0in} |}
    \hline
    Type: & real \\
    \hline
    Units: & \si{PSU} \\
    \hline
    Default Value: & 33.8 \\
    \hline
    Possible Values: & Any positive real number. \\
    \hline
    \caption{config\_isomip\_plus\_restore\_top\_sal: Restoring salinity at sea level.}
\end{longtable}
\end{center}
\subsection[config\_isomip\_plus\_restore\_bot\_sal]{\hyperref[sec:nm_tab_isomip_plus]{config\_isomip\_plus\_restore\_bot\_sal}}
\label{subsec:nm_sec_config_isomip_plus_restore_bot_sal}
\begin{center}
\begin{longtable}{| p{2.0in} || p{4.0in} |}
    \hline
    Type: & real \\
    \hline
    Units: & \si{PSU} \\
    \hline
    Default Value: & 34.7 \\
    \hline
    Possible Values: & Any positive real number. \\
    \hline
    \caption{config\_isomip\_plus\_restore\_bot\_sal: Restoring salinity in deepest cells.}
\end{longtable}
\end{center}
\subsection[config\_isomip\_plus\_restore\_rate]{\hyperref[sec:nm_tab_isomip_plus]{config\_isomip\_plus\_restore\_rate}}
\label{subsec:nm_sec_config_isomip_plus_restore_rate}
\begin{center}
\begin{longtable}{| p{2.0in} || p{4.0in} |}
    \hline
    Type: & real \\
    \hline
    Units: & \si{days^{-1}} \\
    \hline
    Default Value: & 10.0 \\
    \hline
    Possible Values: & Any positive real number. \\
    \hline
    \caption{config\_isomip\_plus\_restore\_rate: Restoring salinity in deepest cells.}
\end{longtable}
\end{center}
\subsection[config\_isomip\_plus\_restore\_evap\_rate]{\hyperref[sec:nm_tab_isomip_plus]{config\_isomip\_plus\_restore\_evap\_rate}}
\label{subsec:nm_sec_config_isomip_plus_restore_evap_rate}
\begin{center}
\begin{longtable}{| p{2.0in} || p{4.0in} |}
    \hline
    Type: & real \\
    \hline
    Units: & \si{m.yr^{-1}} \\
    \hline
    Default Value: & 200 \\
    \hline
    Possible Values: & Any real number. \\
    \hline
    \caption{config\_isomip\_plus\_restore\_evap\_rate: Evaporation rate used to maintain sea level near zero.}
\end{longtable}
\end{center}
\subsection[config\_isomip\_plus\_restore\_xMin]{\hyperref[sec:nm_tab_isomip_plus]{config\_isomip\_plus\_restore\_xMin}}
\label{subsec:nm_sec_config_isomip_plus_restore_xMin}
\begin{center}
\begin{longtable}{| p{2.0in} || p{4.0in} |}
    \hline
    Type: & real \\
    \hline
    Units: & \si{m} \\
    \hline
    Default Value: & 790.0e3 \\
    \hline
    Possible Values: & Any real number. \\
    \hline
    \caption{config\_isomip\_plus\_restore\_xMin: Southern boundary of restoring region.}
\end{longtable}
\end{center}
\subsection[config\_isomip\_plus\_restore\_xMax]{\hyperref[sec:nm_tab_isomip_plus]{config\_isomip\_plus\_restore\_xMax}}
\label{subsec:nm_sec_config_isomip_plus_restore_xMax}
\begin{center}
\begin{longtable}{| p{2.0in} || p{4.0in} |}
    \hline
    Type: & real \\
    \hline
    Units: & \si{m} \\
    \hline
    Default Value: & 800.0e3 \\
    \hline
    Possible Values: & Any real number. \\
    \hline
    \caption{config\_isomip\_plus\_restore\_xMax: Northern boundary of restoring region.}
\end{longtable}
\end{center}
\subsection[config\_isomip\_plus\_coriolis\_parameter]{\hyperref[sec:nm_tab_isomip_plus]{config\_isomip\_plus\_coriolis\_parameter}}
\label{subsec:nm_sec_config_isomip_plus_coriolis_parameter}
\begin{center}
\begin{longtable}{| p{2.0in} || p{4.0in} |}
    \hline
    Type: & real \\
    \hline
    Units: & \si{s^{-1}} \\
    \hline
    Default Value: & -1.409e-4 \\
    \hline
    Possible Values: & Any real number. \\
    \hline
    \caption{config\_isomip\_plus\_coriolis\_parameter: Coriolis parameter for entrie domain.}
\end{longtable}
\end{center}
\subsection[config\_isomip\_plus\_effective\_density]{\hyperref[sec:nm_tab_isomip_plus]{config\_isomip\_plus\_effective\_density}}
\label{subsec:nm_sec_config_isomip_plus_effective_density}
\begin{center}
\begin{longtable}{| p{2.0in} || p{4.0in} |}
    \hline
    Type: & real \\
    \hline
    Units: & \si{kg.m^{-3}} \\
    \hline
    Default Value: & 1026. \\
    \hline
    Possible Values: & Any non-negative real number. \\
    \hline
    \caption{config\_isomip\_plus\_effective\_density: Initial value for the effective density for entrie domain.}
\end{longtable}
\end{center}
\section[hurricane]{\hyperref[sec:nm_tab_hurricane]{hurricane}}
\label{sec:nm_sec_hurricane}
\subsection[config\_hurricane\_vert\_levels]{\hyperref[sec:nm_tab_hurricane]{config\_hurricane\_vert\_levels}}
\label{subsec:nm_sec_config_hurricane_vert_levels}
\begin{center}
\begin{longtable}{| p{2.0in} || p{4.0in} |}
    \hline
    Type: & integer \\
    \hline
    Units: & \si{m} \\
    \hline
    Default Value: & 3 \\
    \hline
    Possible Values: & Any positive integer number greater than 2. \\
    \hline
    \caption{config\_hurricane\_vert\_levels: Number of vertical levels for hurricane.}
\end{longtable}
\end{center}
\subsection[config\_hurricane\_min\_depth]{\hyperref[sec:nm_tab_hurricane]{config\_hurricane\_min\_depth}}
\label{subsec:nm_sec_config_hurricane_min_depth}
\begin{center}
\begin{longtable}{| p{2.0in} || p{4.0in} |}
    \hline
    Type: & real \\
    \hline
    Units: & \si{m} \\
    \hline
    Default Value: & 10.0 \\
    \hline
    Possible Values: & Any positive real number greater than 0. \\
    \hline
    \caption{config\_hurricane\_min\_depth: Minimum depth for hurricane mesh bathymetry.}
\end{longtable}
\end{center}
\subsection[config\_hurricane\_max\_depth]{\hyperref[sec:nm_tab_hurricane]{config\_hurricane\_max\_depth}}
\label{subsec:nm_sec_config_hurricane_max_depth}
\begin{center}
\begin{longtable}{| p{2.0in} || p{4.0in} |}
    \hline
    Type: & real \\
    \hline
    Units: & \si{m} \\
    \hline
    Default Value: & 60.0 \\
    \hline
    Possible Values: & Any positive real number greater than 0. \\
    \hline
    \caption{config\_hurricane\_max\_depth: Maximum depth for hurricane mesh bathymetry.}
\end{longtable}
\end{center}
\subsection[config\_hurricane\_gaussian\_hump\_amplitude]{\hyperref[sec:nm_tab_hurricane]{config\_hurricane\_gaussian\_hump\_amplitude}}
\label{subsec:nm_sec_config_hurricane_gaussian_hump_amplitude}
\begin{center}
\begin{longtable}{| p{2.0in} || p{4.0in} |}
    \hline
    Type: & real \\
    \hline
    Units: & \si{m} \\
    \hline
    Default Value: & 1.0 \\
    \hline
    Possible Values: & Any positive real number greater than 0. \\
    \hline
    \caption{config\_hurricane\_gaussian\_hump\_amplitude: Amplitude of gaussian wave.}
\end{longtable}
\end{center}
\subsection[config\_hurricane\_use\_gaussian\_hump]{\hyperref[sec:nm_tab_hurricane]{config\_hurricane\_use\_gaussian\_hump}}
\label{subsec:nm_sec_config_hurricane_use_gaussian_hump}
\begin{center}
\begin{longtable}{| p{2.0in} || p{4.0in} |}
    \hline
    Type: & logical \\
    \hline
    Units: & \si{unitless} \\
    \hline
    Default Value: & false \\
    \hline
    Possible Values: & .true. or .false. \\
    \hline
    \caption{config\_hurricane\_use\_gaussian\_hump: Use of idealized gaussian hump 'hurricane' initial condition.}
\end{longtable}
\end{center}
\subsection[config\_hurricane\_gaussian\_lon\_center]{\hyperref[sec:nm_tab_hurricane]{config\_hurricane\_gaussian\_lon\_center}}
\label{subsec:nm_sec_config_hurricane_gaussian_lon_center}
\begin{center}
\begin{longtable}{| p{2.0in} || p{4.0in} |}
    \hline
    Type: & real \\
    \hline
    Units: & \si{degrees} \\
    \hline
    Default Value: & 286.0 \\
    \hline
    Possible Values: & Any real number between 0.0 and 360.0. \\
    \hline
    \caption{config\_hurricane\_gaussian\_lon\_center: Longitude of center of gaussian wave.}
\end{longtable}
\end{center}
\subsection[config\_hurricane\_gaussian\_lat\_center]{\hyperref[sec:nm_tab_hurricane]{config\_hurricane\_gaussian\_lat\_center}}
\label{subsec:nm_sec_config_hurricane_gaussian_lat_center}
\begin{center}
\begin{longtable}{| p{2.0in} || p{4.0in} |}
    \hline
    Type: & real \\
    \hline
    Units: & \si{degrees} \\
    \hline
    Default Value: & 38.0 \\
    \hline
    Possible Values: & Any real number between -90.0 and 90.0. \\
    \hline
    \caption{config\_hurricane\_gaussian\_lat\_center: Latitude of center of gaussian wave.}
\end{longtable}
\end{center}
\subsection[config\_hurricane\_gaussian\_width]{\hyperref[sec:nm_tab_hurricane]{config\_hurricane\_gaussian\_width}}
\label{subsec:nm_sec_config_hurricane_gaussian_width}
\begin{center}
\begin{longtable}{| p{2.0in} || p{4.0in} |}
    \hline
    Type: & real \\
    \hline
    Units: & \si{degrees} \\
    \hline
    Default Value: & 1.0 \\
    \hline
    Possible Values: & Any real number greater than 0.0 \\
    \hline
    \caption{config\_hurricane\_gaussian\_width: Width scale of gaussian wave.}
\end{longtable}
\end{center}
\subsection[config\_hurricane\_gaussian\_amplitude]{\hyperref[sec:nm_tab_hurricane]{config\_hurricane\_gaussian\_amplitude}}
\label{subsec:nm_sec_config_hurricane_gaussian_amplitude}
\begin{center}
\begin{longtable}{| p{2.0in} || p{4.0in} |}
    \hline
    Type: & real \\
    \hline
    Units: & \si{m} \\
    \hline
    Default Value: & 1.0 \\
    \hline
    Possible Values: & Any real number greater than 0.0 \\
    \hline
    \caption{config\_hurricane\_gaussian\_amplitude: Amplitude of gaussian wave.}
\end{longtable}
\end{center}
\subsection[config\_hurricane\_gaussian\_slr\_amp]{\hyperref[sec:nm_tab_hurricane]{config\_hurricane\_gaussian\_slr\_amp}}
\label{subsec:nm_sec_config_hurricane_gaussian_slr_amp}
\begin{center}
\begin{longtable}{| p{2.0in} || p{4.0in} |}
    \hline
    Type: & real \\
    \hline
    Units: & \si{m} \\
    \hline
    Default Value: & 0.0 \\
    \hline
    Possible Values: & Any real number. \\
    \hline
    \caption{config\_hurricane\_gaussian\_slr\_amp: Amplitude of sea level rise.}
\end{longtable}
\end{center}
\subsection[config\_hurricane\_land\_z\_limit]{\hyperref[sec:nm_tab_hurricane]{config\_hurricane\_land\_z\_limit}}
\label{subsec:nm_sec_config_hurricane_land_z_limit}
\begin{center}
\begin{longtable}{| p{2.0in} || p{4.0in} |}
    \hline
    Type: & real \\
    \hline
    Units: & \si{m} \\
    \hline
    Default Value: & -2.0 \\
    \hline
    Possible Values: & Any real number. \\
    \hline
    \caption{config\_hurricane\_land\_z\_limit: Vertical elevation corresponding to increased drag on land (bottom depth positive).}
\end{longtable}
\end{center}
\subsection[config\_hurricane\_marsh\_z\_limit]{\hyperref[sec:nm_tab_hurricane]{config\_hurricane\_marsh\_z\_limit}}
\label{subsec:nm_sec_config_hurricane_marsh_z_limit}
\begin{center}
\begin{longtable}{| p{2.0in} || p{4.0in} |}
    \hline
    Type: & real \\
    \hline
    Units: & \si{m} \\
    \hline
    Default Value: & 2.0 \\
    \hline
    Possible Values: & Any real number. \\
    \hline
    \caption{config\_hurricane\_marsh\_z\_limit: Vertical elevation corresponding to increased drag on marsh (bottom depth positive).}
\end{longtable}
\end{center}
\subsection[config\_hurricane\_land\_drag]{\hyperref[sec:nm_tab_hurricane]{config\_hurricane\_land\_drag}}
\label{subsec:nm_sec_config_hurricane_land_drag}
\begin{center}
\begin{longtable}{| p{2.0in} || p{4.0in} |}
    \hline
    Type: & real \\
    \hline
    Units: & \si{non-dimensional.or.Manning's.n} \\
    \hline
    Default Value: & 0.1 \\
    \hline
    Possible Values: & Any real number. \\
    \hline
    \caption{config\_hurricane\_land\_drag: Value of land drag for either Cd or Manning's n above config\_land\_z\_limit.}
\end{longtable}
\end{center}
\subsection[config\_hurricane\_marsh\_drag]{\hyperref[sec:nm_tab_hurricane]{config\_hurricane\_marsh\_drag}}
\label{subsec:nm_sec_config_hurricane_marsh_drag}
\begin{center}
\begin{longtable}{| p{2.0in} || p{4.0in} |}
    \hline
    Type: & real \\
    \hline
    Units: & \si{non-dimensional.or.Manning's.n} \\
    \hline
    Default Value: & 0.05 \\
    \hline
    Possible Values: & Any real number. \\
    \hline
    \caption{config\_hurricane\_marsh\_drag: Value of marsh drag between config\_marsh\_z\_limit and config\_land\_z\_limit for either Cd or Manning's n.}
\end{longtable}
\end{center}
\subsection[config\_hurricane\_channel\_drag]{\hyperref[sec:nm_tab_hurricane]{config\_hurricane\_channel\_drag}}
\label{subsec:nm_sec_config_hurricane_channel_drag}
\begin{center}
\begin{longtable}{| p{2.0in} || p{4.0in} |}
    \hline
    Type: & real \\
    \hline
    Units: & \si{non-dimensional.or.Manning's.n} \\
    \hline
    Default Value: & 0.02 \\
    \hline
    Possible Values: & Any real number. \\
    \hline
    \caption{config\_hurricane\_channel\_drag: Value of channel drag below config\_marsh\_z\_limit for either Cd or Manning's n.}
\end{longtable}
\end{center}
\subsection[config\_hurricane\_sea\_level\_rise\_adjustment]{\hyperref[sec:nm_tab_hurricane]{config\_hurricane\_sea\_level\_rise\_adjustment}}
\label{subsec:nm_sec_config_hurricane_sea_level_rise_adjustment}
\begin{center}
\begin{longtable}{| p{2.0in} || p{4.0in} |}
    \hline
    Type: & real \\
    \hline
    Units: & \si{m} \\
    \hline
    Default Value: & 0.00 \\
    \hline
    Possible Values: & Any real number. \\
    \hline
    \caption{config\_hurricane\_sea\_level\_rise\_adjustment: Crude factor to account for sea level rise. This is uniformly added to the bathymetric depth.}
\end{longtable}
\end{center}
\section[tidal\_boundary]{\hyperref[sec:nm_tab_tidal_boundary]{tidal\_boundary}}
\label{sec:nm_sec_tidal_boundary}
\subsection[config\_tidal\_boundary\_vert\_levels]{\hyperref[sec:nm_tab_tidal_boundary]{config\_tidal\_boundary\_vert\_levels}}
\label{subsec:nm_sec_config_tidal_boundary_vert_levels}
\begin{center}
\begin{longtable}{| p{2.0in} || p{4.0in} |}
    \hline
    Type: & integer \\
    \hline
    Units: & \si{unitless} \\
    \hline
    Default Value: & 100 \\
    \hline
    Possible Values: & Any positive integer number greater than 0. \\
    \hline
    \caption{config\_tidal\_boundary\_vert\_levels: Number of vertical levels in tidal\_boundary test case. Typical values are 40 and 100.}
\end{longtable}
\end{center}
\subsection[config\_tidal\_boundary\_min\_vert\_levels]{\hyperref[sec:nm_tab_tidal_boundary]{config\_tidal\_boundary\_min\_vert\_levels}}
\label{subsec:nm_sec_config_tidal_boundary_min_vert_levels}
\begin{center}
\begin{longtable}{| p{2.0in} || p{4.0in} |}
    \hline
    Type: & integer \\
    \hline
    Units: & \si{unitless} \\
    \hline
    Default Value: & 10 \\
    \hline
    Possible Values: & Any positive integer number greater than 0. \\
    \hline
    \caption{config\_tidal\_boundary\_min\_vert\_levels: Number of vertical levels where zstar coordinates transition to sigma.}
\end{longtable}
\end{center}
\subsection[config\_tidal\_boundary\_layer\_type]{\hyperref[sec:nm_tab_tidal_boundary]{config\_tidal\_boundary\_layer\_type}}
\label{subsec:nm_sec_config_tidal_boundary_layer_type}
\begin{center}
\begin{longtable}{| p{2.0in} || p{4.0in} |}
    \hline
    Type: & character \\
    \hline
    Units: & \si{unitless} \\
    \hline
    Default Value: & zstar \\
    \hline
    Possible Values: & 'zstar', 'sigma', 'hybrid' \\
    \hline
    \caption{config\_tidal\_boundary\_layer\_type: Vertical coordinate to be used.}
\end{longtable}
\end{center}
\subsection[config\_tidal\_boundary\_right\_bottom\_depth]{\hyperref[sec:nm_tab_tidal_boundary]{config\_tidal\_boundary\_right\_bottom\_depth}}
\label{subsec:nm_sec_config_tidal_boundary_right_bottom_depth}
\begin{center}
\begin{longtable}{| p{2.0in} || p{4.0in} |}
    \hline
    Type: & real \\
    \hline
    Units: & \si{m} \\
    \hline
    Default Value: & 10.0 \\
    \hline
    Possible Values: & Any positive real value greater than 0. \\
    \hline
    \caption{config\_tidal\_boundary\_right\_bottom\_depth: Depth of the bottom of the ocean in northern-most end.}
\end{longtable}
\end{center}
\subsection[config\_tidal\_start\_dry]{\hyperref[sec:nm_tab_tidal_boundary]{config\_tidal\_start\_dry}}
\label{subsec:nm_sec_config_tidal_start_dry}
\begin{center}
\begin{longtable}{| p{2.0in} || p{4.0in} |}
    \hline
    Type: & logical \\
    \hline
    Units: & \si{m} \\
    \hline
    Default Value: & false \\
    \hline
    Possible Values: & True or False \\
    \hline
    \caption{config\_tidal\_start\_dry: Logical to determine if channel should be started dry.}
\end{longtable}
\end{center}
\subsection[config\_tidal\_boundary\_use\_distances]{\hyperref[sec:nm_tab_tidal_boundary]{config\_tidal\_boundary\_use\_distances}}
\label{subsec:nm_sec_config_tidal_boundary_use_distances}
\begin{center}
\begin{longtable}{| p{2.0in} || p{4.0in} |}
    \hline
    Type: & logical \\
    \hline
    Units: & \si{m} \\
    \hline
    Default Value: & true \\
    \hline
    Possible Values: & True or False \\
    \hline
    \caption{config\_tidal\_boundary\_use\_distances: Logical to determine if channel dimensions should to specific values.}
\end{longtable}
\end{center}
\subsection[config\_tidal\_boundary\_left\_value]{\hyperref[sec:nm_tab_tidal_boundary]{config\_tidal\_boundary\_left\_value}}
\label{subsec:nm_sec_config_tidal_boundary_left_value}
\begin{center}
\begin{longtable}{| p{2.0in} || p{4.0in} |}
    \hline
    Type: & real \\
    \hline
    Units: & \si{m} \\
    \hline
    Default Value: & 0.0 \\
    \hline
    Possible Values: & Any positive real value greater than or equal to 0. \\
    \hline
    \caption{config\_tidal\_boundary\_left\_value: Coordinate of the southern-most end.}
\end{longtable}
\end{center}
\subsection[config\_tidal\_boundary\_right\_value]{\hyperref[sec:nm_tab_tidal_boundary]{config\_tidal\_boundary\_right\_value}}
\label{subsec:nm_sec_config_tidal_boundary_right_value}
\begin{center}
\begin{longtable}{| p{2.0in} || p{4.0in} |}
    \hline
    Type: & real \\
    \hline
    Units: & \si{m} \\
    \hline
    Default Value: & 25.0e3 \\
    \hline
    Possible Values: & Any positive real value greater than or equal to 0. \\
    \hline
    \caption{config\_tidal\_boundary\_right\_value: Coordinate of the northern-most end.}
\end{longtable}
\end{center}
\subsection[config\_tidal\_boundary\_left\_bottom\_depth]{\hyperref[sec:nm_tab_tidal_boundary]{config\_tidal\_boundary\_left\_bottom\_depth}}
\label{subsec:nm_sec_config_tidal_boundary_left_bottom_depth}
\begin{center}
\begin{longtable}{| p{2.0in} || p{4.0in} |}
    \hline
    Type: & real \\
    \hline
    Units: & \si{m} \\
    \hline
    Default Value: & 10.0 \\
    \hline
    Possible Values: & Any positive real value greater than 0. \\
    \hline
    \caption{config\_tidal\_boundary\_left\_bottom\_depth: Depth of the bottom of the ocean in southern-most end.}
\end{longtable}
\end{center}
\subsection[config\_tidal\_boundary\_salinity]{\hyperref[sec:nm_tab_tidal_boundary]{config\_tidal\_boundary\_salinity}}
\label{subsec:nm_sec_config_tidal_boundary_salinity}
\begin{center}
\begin{longtable}{| p{2.0in} || p{4.0in} |}
    \hline
    Type: & real \\
    \hline
    Units: & \si{PSU} \\
    \hline
    Default Value: & 35.0 \\
    \hline
    Possible Values: & Any real number greater than 0. \\
    \hline
    \caption{config\_tidal\_boundary\_salinity: Salinity of the water in the entire domain.}
\end{longtable}
\end{center}
\subsection[config\_tidal\_boundary\_domain\_temperature]{\hyperref[sec:nm_tab_tidal_boundary]{config\_tidal\_boundary\_domain\_temperature}}
\label{subsec:nm_sec_config_tidal_boundary_domain_temperature}
\begin{center}
\begin{longtable}{| p{2.0in} || p{4.0in} |}
    \hline
    Type: & real \\
    \hline
    Units: & \si{deg.C} \\
    \hline
    Default Value: & 20.0 \\
    \hline
    Possible Values: & Any real number \\
    \hline
    \caption{config\_tidal\_boundary\_domain\_temperature: Temperature of water outside of the plug.}
\end{longtable}
\end{center}
\subsection[config\_tidal\_boundary\_plug\_temperature]{\hyperref[sec:nm_tab_tidal_boundary]{config\_tidal\_boundary\_plug\_temperature}}
\label{subsec:nm_sec_config_tidal_boundary_plug_temperature}
\begin{center}
\begin{longtable}{| p{2.0in} || p{4.0in} |}
    \hline
    Type: & real \\
    \hline
    Units: & \si{deg.C} \\
    \hline
    Default Value: & 20.0 \\
    \hline
    Possible Values: & Any real number \\
    \hline
    \caption{config\_tidal\_boundary\_plug\_temperature: Temperature of water in plug.}
\end{longtable}
\end{center}
\subsection[config\_tidal\_boundary\_plug\_width\_frac]{\hyperref[sec:nm_tab_tidal_boundary]{config\_tidal\_boundary\_plug\_width\_frac}}
\label{subsec:nm_sec_config_tidal_boundary_plug_width_frac}
\begin{center}
\begin{longtable}{| p{2.0in} || p{4.0in} |}
    \hline
    Type: & real \\
    \hline
    Units: & \si{fraction} \\
    \hline
    Default Value: & 0.10 \\
    \hline
    Possible Values: & Any real number between 0 and 1. \\
    \hline
    \caption{config\_tidal\_boundary\_plug\_width\_frac: Fraction of the domain the plug should take up initially. Only in the y direction.}
\end{longtable}
\end{center}
\subsection[config\_tidal\_forcing\_left\_Cd\_or\_n]{\hyperref[sec:nm_tab_tidal_boundary]{config\_tidal\_forcing\_left\_Cd\_or\_n}}
\label{subsec:nm_sec_config_tidal_forcing_left_Cd_or_n}
\begin{center}
\begin{longtable}{| p{2.0in} || p{4.0in} |}
    \hline
    Type: & real \\
    \hline
    Units: & \si{unitless} \\
    \hline
    Default Value: & 1.0e-3 \\
    \hline
    Possible Values: & Any real number \\
    \hline
    \caption{config\_tidal\_forcing\_left\_Cd\_or\_n: Bottom drag of left side of the boundary.}
\end{longtable}
\end{center}
\subsection[config\_tidal\_forcing\_right\_Cd\_or\_n]{\hyperref[sec:nm_tab_tidal_boundary]{config\_tidal\_forcing\_right\_Cd\_or\_n}}
\label{subsec:nm_sec_config_tidal_forcing_right_Cd_or_n}
\begin{center}
\begin{longtable}{| p{2.0in} || p{4.0in} |}
    \hline
    Type: & real \\
    \hline
    Units: & \si{unitless} \\
    \hline
    Default Value: & 1.0e-3 \\
    \hline
    Possible Values: & Any real number \\
    \hline
    \caption{config\_tidal\_forcing\_right\_Cd\_or\_n: Bottom drag of right side of the boundary.}
\end{longtable}
\end{center}
\subsection[config\_use\_idealized\_transect]{\hyperref[sec:nm_tab_tidal_boundary]{config\_use\_idealized\_transect}}
\label{subsec:nm_sec_config_use_idealized_transect}
\begin{center}
\begin{longtable}{| p{2.0in} || p{4.0in} |}
    \hline
    Type: & logical \\
    \hline
    Units: & -- \\
    \hline
    Default Value: & false \\
    \hline
    Possible Values: & True or False \\
    \hline
    \caption{config\_use\_idealized\_transect: Logical to determine if idealized tidal flat profile is defined.}
\end{longtable}
\end{center}
\subsection[config\_idealized\_transect\_Lshore]{\hyperref[sec:nm_tab_tidal_boundary]{config\_idealized\_transect\_Lshore}}
\label{subsec:nm_sec_config_idealized_transect_Lshore}
\begin{center}
\begin{longtable}{| p{2.0in} || p{4.0in} |}
    \hline
    Type: & real \\
    \hline
    Units: & -- \\
    \hline
    Default Value: & 0.6 \\
    \hline
    Possible Values: & Any positive real value between 0 and 1. \\
    \hline
    \caption{config\_idealized\_transect\_Lshore: Ratio of shore length in the idealized coastal profile.}
\end{longtable}
\end{center}
\subsection[config\_idealized\_transect\_Sshore]{\hyperref[sec:nm_tab_tidal_boundary]{config\_idealized\_transect\_Sshore}}
\label{subsec:nm_sec_config_idealized_transect_Sshore}
\begin{center}
\begin{longtable}{| p{2.0in} || p{4.0in} |}
    \hline
    Type: & real \\
    \hline
    Units: & -- \\
    \hline
    Default Value: & 0.001 \\
    \hline
    Possible Values: & Any positive real value, should be a small value. \\
    \hline
    \caption{config\_idealized\_transect\_Sshore: Shore slope.}
\end{longtable}
\end{center}
\subsection[config\_idealized\_transect\_Lcoast]{\hyperref[sec:nm_tab_tidal_boundary]{config\_idealized\_transect\_Lcoast}}
\label{subsec:nm_sec_config_idealized_transect_Lcoast}
\begin{center}
\begin{longtable}{| p{2.0in} || p{4.0in} |}
    \hline
    Type: & real \\
    \hline
    Units: & -- \\
    \hline
    Default Value: & 0.3 \\
    \hline
    Possible Values: & Any positive real value between 0 and 1. \\
    \hline
    \caption{config\_idealized\_transect\_Lcoast: Ratio of coast length in the idealized coastal profile}
\end{longtable}
\end{center}
\subsection[config\_idealized\_transect\_Scoast]{\hyperref[sec:nm_tab_tidal_boundary]{config\_idealized\_transect\_Scoast}}
\label{subsec:nm_sec_config_idealized_transect_Scoast}
\begin{center}
\begin{longtable}{| p{2.0in} || p{4.0in} |}
    \hline
    Type: & real \\
    \hline
    Units: & -- \\
    \hline
    Default Value: & 0.001 \\
    \hline
    Possible Values: & Any positive real value. \\
    \hline
    \caption{config\_idealized\_transect\_Scoast: Coast slope.}
\end{longtable}
\end{center}
\subsection[config\_idealized\_transect\_Lmarsh]{\hyperref[sec:nm_tab_tidal_boundary]{config\_idealized\_transect\_Lmarsh}}
\label{subsec:nm_sec_config_idealized_transect_Lmarsh}
\begin{center}
\begin{longtable}{| p{2.0in} || p{4.0in} |}
    \hline
    Type: & real \\
    \hline
    Units: & -- \\
    \hline
    Default Value: & 0.1 \\
    \hline
    Possible Values: & Lmarsh=1-Lshore-Lcoast; any positive real value between 0 and 1. \\
    \hline
    \caption{config\_idealized\_transect\_Lmarsh: Ratio of marsh length in the idealized coastal profile.}
\end{longtable}
\end{center}
\subsection[config\_idealized\_transect\_Smarsh]{\hyperref[sec:nm_tab_tidal_boundary]{config\_idealized\_transect\_Smarsh}}
\label{subsec:nm_sec_config_idealized_transect_Smarsh}
\begin{center}
\begin{longtable}{| p{2.0in} || p{4.0in} |}
    \hline
    Type: & real \\
    \hline
    Units: & -- \\
    \hline
    Default Value: & 0.0 \\
    \hline
    Possible Values: & Any real value, can be negative \\
    \hline
    \caption{config\_idealized\_transect\_Smarsh: Marsh slope}
\end{longtable}
\end{center}
\subsection[config\_idealized\_transect\_roughness]{\hyperref[sec:nm_tab_tidal_boundary]{config\_idealized\_transect\_roughness}}
\label{subsec:nm_sec_config_idealized_transect_roughness}
\begin{center}
\begin{longtable}{| p{2.0in} || p{4.0in} |}
    \hline
    Type: & real \\
    \hline
    Units: & -- \\
    \hline
    Default Value: & 0.025 \\
    \hline
    Possible Values: & Any real value, can be negative \\
    \hline
    \caption{config\_idealized\_transect\_roughness: Bottom roughness (Cd or Manning roughness) at non-vegetated region}
\end{longtable}
\end{center}
\subsection[config\_idealized\_transect\_roughness\_marsh]{\hyperref[sec:nm_tab_tidal_boundary]{config\_idealized\_transect\_roughness\_marsh}}
\label{subsec:nm_sec_config_idealized_transect_roughness_marsh}
\begin{center}
\begin{longtable}{| p{2.0in} || p{4.0in} |}
    \hline
    Type: & real \\
    \hline
    Units: & -- \\
    \hline
    Default Value: & 0.075 \\
    \hline
    Possible Values: & Any real value, can be negative \\
    \hline
    \caption{config\_idealized\_transect\_roughness\_marsh: Bottom roughness (Cd or Manning roughness) at vegetated region}
\end{longtable}
\end{center}
\subsection[config\_idealized\_vegetation\_diameter]{\hyperref[sec:nm_tab_tidal_boundary]{config\_idealized\_vegetation\_diameter}}
\label{subsec:nm_sec_config_idealized_vegetation_diameter}
\begin{center}
\begin{longtable}{| p{2.0in} || p{4.0in} |}
    \hline
    Type: & real \\
    \hline
    Units: & \si{m} \\
    \hline
    Default Value: & 0.05 \\
    \hline
    Possible Values: & Any non-negative real value \\
    \hline
    \caption{config\_idealized\_vegetation\_diameter: Constant vegetation diameter for idealized transect case}
\end{longtable}
\end{center}
\subsection[config\_idealized\_vegetation\_height]{\hyperref[sec:nm_tab_tidal_boundary]{config\_idealized\_vegetation\_height}}
\label{subsec:nm_sec_config_idealized_vegetation_height}
\begin{center}
\begin{longtable}{| p{2.0in} || p{4.0in} |}
    \hline
    Type: & real \\
    \hline
    Units: & \si{m} \\
    \hline
    Default Value: & 0.2 \\
    \hline
    Possible Values: & Any non-negative real value  \\
    \hline
    \caption{config\_idealized\_vegetation\_height: Constant vegetation height for idealized transect case}
\end{longtable}
\end{center}
\subsection[config\_idealized\_vegetation\_density]{\hyperref[sec:nm_tab_tidal_boundary]{config\_idealized\_vegetation\_density}}
\label{subsec:nm_sec_config_idealized_vegetation_density}
\begin{center}
\begin{longtable}{| p{2.0in} || p{4.0in} |}
    \hline
    Type: & real \\
    \hline
    Units: & \si{m^{-2}} \\
    \hline
    Default Value: & 1000 \\
    \hline
    Possible Values: & Any non-negative real value \\
    \hline
    \caption{config\_idealized\_vegetation\_density: Constant vegetation density for indealized transect case}
\end{longtable}
\end{center}
\section[cosine\_bell]{\hyperref[sec:nm_tab_cosine_bell]{cosine\_bell}}
\label{sec:nm_sec_cosine_bell}
\subsection[config\_cosine\_bell\_temperature]{\hyperref[sec:nm_tab_cosine_bell]{config\_cosine\_bell\_temperature}}
\label{subsec:nm_sec_config_cosine_bell_temperature}
\begin{center}
\begin{longtable}{| p{2.0in} || p{4.0in} |}
    \hline
    Type: & real \\
    \hline
    Units: & \si{deg.C} \\
    \hline
    Default Value: & 15.0 \\
    \hline
    Possible Values: & Any real number \\
    \hline
    \caption{config\_cosine\_bell\_temperature: Temperature of the ocean.}
\end{longtable}
\end{center}
\subsection[config\_cosine\_bell\_salinity]{\hyperref[sec:nm_tab_cosine_bell]{config\_cosine\_bell\_salinity}}
\label{subsec:nm_sec_config_cosine_bell_salinity}
\begin{center}
\begin{longtable}{| p{2.0in} || p{4.0in} |}
    \hline
    Type: & real \\
    \hline
    Units: & \si{PSU} \\
    \hline
    Default Value: & 35.0 \\
    \hline
    Possible Values: & Any real number \\
    \hline
    \caption{config\_cosine\_bell\_salinity: Salinity of the ocean.}
\end{longtable}
\end{center}
\subsection[config\_cosine\_bell\_lat\_center]{\hyperref[sec:nm_tab_cosine_bell]{config\_cosine\_bell\_lat\_center}}
\label{subsec:nm_sec_config_cosine_bell_lat_center}
\begin{center}
\begin{longtable}{| p{2.0in} || p{4.0in} |}
    \hline
    Type: & real \\
    \hline
    Units: & \si{radians} \\
    \hline
    Default Value: & 0.0 \\
    \hline
    Possible Values: & Any real number between -pi/2 and pi/2 \\
    \hline
    \caption{config\_cosine\_bell\_lat\_center: latitude center of cosine bell}
\end{longtable}
\end{center}
\subsection[config\_cosine\_bell\_lon\_center]{\hyperref[sec:nm_tab_cosine_bell]{config\_cosine\_bell\_lon\_center}}
\label{subsec:nm_sec_config_cosine_bell_lon_center}
\begin{center}
\begin{longtable}{| p{2.0in} || p{4.0in} |}
    \hline
    Type: & real \\
    \hline
    Units: & \si{radians} \\
    \hline
    Default Value: & 3.141592 \\
    \hline
    Possible Values: & Any non-negative real number between 0 and 2pi \\
    \hline
    \caption{config\_cosine\_bell\_lon\_center: longitude center of cosine bell}
\end{longtable}
\end{center}
\subsection[config\_cosine\_bell\_psi0]{\hyperref[sec:nm_tab_cosine_bell]{config\_cosine\_bell\_psi0}}
\label{subsec:nm_sec_config_cosine_bell_psi0}
\begin{center}
\begin{longtable}{| p{2.0in} || p{4.0in} |}
    \hline
    Type: & real \\
    \hline
    Units: & \si{unitless} \\
    \hline
    Default Value: & 1.0 \\
    \hline
    Possible Values: & Any real number \\
    \hline
    \caption{config\_cosine\_bell\_psi0: hill max of tracer}
\end{longtable}
\end{center}
\subsection[config\_cosine\_bell\_radius]{\hyperref[sec:nm_tab_cosine_bell]{config\_cosine\_bell\_radius}}
\label{subsec:nm_sec_config_cosine_bell_radius}
\begin{center}
\begin{longtable}{| p{2.0in} || p{4.0in} |}
    \hline
    Type: & real \\
    \hline
    Units: & \si{m} \\
    \hline
    Default Value: & 2123666.667 \\
    \hline
    Possible Values: & Any non-negative real number between 0 and 2pi \\
    \hline
    \caption{config\_cosine\_bell\_radius: radius of cosine bell}
\end{longtable}
\end{center}
\subsection[config\_cosine\_bell\_vel\_pd]{\hyperref[sec:nm_tab_cosine_bell]{config\_cosine\_bell\_vel\_pd}}
\label{subsec:nm_sec_config_cosine_bell_vel_pd}
\begin{center}
\begin{longtable}{| p{2.0in} || p{4.0in} |}
    \hline
    Type: & real \\
    \hline
    Units: & \si{days} \\
    \hline
    Default Value: & 24.0 \\
    \hline
    Possible Values: & Any non-negative real number \\
    \hline
    \caption{config\_cosine\_bell\_vel\_pd: time for bell to transit equator once}
\end{longtable}
\end{center}
\section[mixed\_layer\_eddy]{\hyperref[sec:nm_tab_mixed_layer_eddy]{mixed\_layer\_eddy}}
\label{sec:nm_sec_mixed_layer_eddy}
\subsection[config\_mixed\_layer\_eddy\_vert\_levels]{\hyperref[sec:nm_tab_mixed_layer_eddy]{config\_mixed\_layer\_eddy\_vert\_levels}}
\label{subsec:nm_sec_config_mixed_layer_eddy_vert_levels}
\begin{center}
\begin{longtable}{| p{2.0in} || p{4.0in} |}
    \hline
    Type: & integer \\
    \hline
    Units: & \si{unitless} \\
    \hline
    Default Value: & 60 \\
    \hline
    Possible Values: & Any positive integer number greater than 0. \\
    \hline
    \caption{config\_mixed\_layer\_eddy\_vert\_levels: Number of vertical levels in mixed layer eddy test case. Typical value is 60.}
\end{longtable}
\end{center}
\subsection[config\_mixed\_layer\_eddy\_bottom\_depth]{\hyperref[sec:nm_tab_mixed_layer_eddy]{config\_mixed\_layer\_eddy\_bottom\_depth}}
\label{subsec:nm_sec_config_mixed_layer_eddy_bottom_depth}
\begin{center}
\begin{longtable}{| p{2.0in} || p{4.0in} |}
    \hline
    Type: & real \\
    \hline
    Units: & \si{m} \\
    \hline
    Default Value: & 300.0 \\
    \hline
    Possible Values: & Any positive real number. \\
    \hline
    \caption{config\_mixed\_layer\_eddy\_bottom\_depth: Depth of the bottom of the domain for the mixed layer eddy test case.}
\end{longtable}
\end{center}
\subsection[config\_mixed\_layer\_eddy\_mixed\_layer\_depth]{\hyperref[sec:nm_tab_mixed_layer_eddy]{config\_mixed\_layer\_eddy\_mixed\_layer\_depth}}
\label{subsec:nm_sec_config_mixed_layer_eddy_mixed_layer_depth}
\begin{center}
\begin{longtable}{| p{2.0in} || p{4.0in} |}
    \hline
    Type: & real \\
    \hline
    Units: & \si{m} \\
    \hline
    Default Value: & 200.0 \\
    \hline
    Possible Values: & Any positive real number. \\
    \hline
    \caption{config\_mixed\_layer\_eddy\_mixed\_layer\_depth: Depth of the mixed layer for the mixed layer eddy test case.}
\end{longtable}
\end{center}
\subsection[config\_mixed\_layer\_eddy\_base\_temperature]{\hyperref[sec:nm_tab_mixed_layer_eddy]{config\_mixed\_layer\_eddy\_base\_temperature}}
\label{subsec:nm_sec_config_mixed_layer_eddy_base_temperature}
\begin{center}
\begin{longtable}{| p{2.0in} || p{4.0in} |}
    \hline
    Type: & real \\
    \hline
    Units: & \si{deg.C} \\
    \hline
    Default Value: & 16.0 \\
    \hline
    Possible Values: & Any real number. \\
    \hline
    \caption{config\_mixed\_layer\_eddy\_base\_temperature: Temperature at the base of the mixed layer.}
\end{longtable}
\end{center}
\subsection[config\_mixed\_layer\_eddy\_temperature\_stratification\_mixed\_layer]{\hyperref[sec:nm_tab_mixed_layer_eddy]{config\_mixed\_layer\_eddy\_temperature\_stratification\_mixed\_layer}}
\label{subsec:nm_sec_config_mixed_layer_eddy_temperature_stratification_mixed_layer}
\begin{center}
\begin{longtable}{| p{2.0in} || p{4.0in} |}
    \hline
    Type: & real \\
    \hline
    Units: & \si{deg.C.m^{-1}} \\
    \hline
    Default Value: & 1e-4 \\
    \hline
    Possible Values: & Any real number. \\
    \hline
    \caption{config\_mixed\_layer\_eddy\_temperature\_stratification\_mixed\_layer: Vertical temperature gradient in the mixed layer.}
\end{longtable}
\end{center}
\subsection[config\_mixed\_layer\_eddy\_temperature\_stratification\_interior]{\hyperref[sec:nm_tab_mixed_layer_eddy]{config\_mixed\_layer\_eddy\_temperature\_stratification\_interior}}
\label{subsec:nm_sec_config_mixed_layer_eddy_temperature_stratification_interior}
\begin{center}
\begin{longtable}{| p{2.0in} || p{4.0in} |}
    \hline
    Type: & real \\
    \hline
    Units: & \si{deg.C.m^{-1}} \\
    \hline
    Default Value: & 1e-2 \\
    \hline
    Possible Values: & Any real number. \\
    \hline
    \caption{config\_mixed\_layer\_eddy\_temperature\_stratification\_interior: Vertical temperature gradient in the interior.}
\end{longtable}
\end{center}
\subsection[config\_mixed\_layer\_eddy\_temperature\_horizontal\_gradient]{\hyperref[sec:nm_tab_mixed_layer_eddy]{config\_mixed\_layer\_eddy\_temperature\_horizontal\_gradient}}
\label{subsec:nm_sec_config_mixed_layer_eddy_temperature_horizontal_gradient}
\begin{center}
\begin{longtable}{| p{2.0in} || p{4.0in} |}
    \hline
    Type: & real \\
    \hline
    Units: & \si{deg.C.m^{-1}} \\
    \hline
    Default Value: & 2e-5 \\
    \hline
    Possible Values: & Any real number. \\
    \hline
    \caption{config\_mixed\_layer\_eddy\_temperature\_horizontal\_gradient: Horizontal temperature gradient in the mixed layer.}
\end{longtable}
\end{center}
\subsection[config\_mixed\_layer\_eddy\_temperature\_front\_width]{\hyperref[sec:nm_tab_mixed_layer_eddy]{config\_mixed\_layer\_eddy\_temperature\_front\_width}}
\label{subsec:nm_sec_config_mixed_layer_eddy_temperature_front_width}
\begin{center}
\begin{longtable}{| p{2.0in} || p{4.0in} |}
    \hline
    Type: & real \\
    \hline
    Units: & \si{m} \\
    \hline
    Default Value: & 10e3 \\
    \hline
    Possible Values: & Any positive real number. \\
    \hline
    \caption{config\_mixed\_layer\_eddy\_temperature\_front\_width: Width of the temperature front.}
\end{longtable}
\end{center}
\subsection[config\_mixed\_layer\_eddy\_temperature\_perturbation\_magnitude]{\hyperref[sec:nm_tab_mixed_layer_eddy]{config\_mixed\_layer\_eddy\_temperature\_perturbation\_magnitude}}
\label{subsec:nm_sec_config_mixed_layer_eddy_temperature_perturbation_magnitude}
\begin{center}
\begin{longtable}{| p{2.0in} || p{4.0in} |}
    \hline
    Type: & real \\
    \hline
    Units: & \si{deg.C} \\
    \hline
    Default Value: & 1e-5 \\
    \hline
    Possible Values: & Any positive real number. \\
    \hline
    \caption{config\_mixed\_layer\_eddy\_temperature\_perturbation\_magnitude: Magnitude of random perturbation in temperature.}
\end{longtable}
\end{center}
\subsection[config\_mixed\_layer\_eddy\_salinity]{\hyperref[sec:nm_tab_mixed_layer_eddy]{config\_mixed\_layer\_eddy\_salinity}}
\label{subsec:nm_sec_config_mixed_layer_eddy_salinity}
\begin{center}
\begin{longtable}{| p{2.0in} || p{4.0in} |}
    \hline
    Type: & real \\
    \hline
    Units: & \si{PSU} \\
    \hline
    Default Value: & 35.0 \\
    \hline
    Possible Values: & Any real number larger than zero. \\
    \hline
    \caption{config\_mixed\_layer\_eddy\_salinity: Salinity of the water in the entire domain.}
\end{longtable}
\end{center}
\subsection[config\_mixed\_layer\_eddy\_two\_fronts]{\hyperref[sec:nm_tab_mixed_layer_eddy]{config\_mixed\_layer\_eddy\_two\_fronts}}
\label{subsec:nm_sec_config_mixed_layer_eddy_two_fronts}
\begin{center}
\begin{longtable}{| p{2.0in} || p{4.0in} |}
    \hline
    Type: & logical \\
    \hline
    Units: & \si{unitless} \\
    \hline
    Default Value: & .false. \\
    \hline
    Possible Values: & .true. or .false. \\
    \hline
    \caption{config\_mixed\_layer\_eddy\_two\_fronts: Logical flag that determines if the initial fields has two fronts.}
\end{longtable}
\end{center}
\subsection[config\_mixed\_layer\_eddy\_restoring\_width]{\hyperref[sec:nm_tab_mixed_layer_eddy]{config\_mixed\_layer\_eddy\_restoring\_width}}
\label{subsec:nm_sec_config_mixed_layer_eddy_restoring_width}
\begin{center}
\begin{longtable}{| p{2.0in} || p{4.0in} |}
    \hline
    Type: & real \\
    \hline
    Units: & \si{m} \\
    \hline
    Default Value: & 5e3 \\
    \hline
    Possible Values: & Any real number larger than zero. \\
    \hline
    \caption{config\_mixed\_layer\_eddy\_restoring\_width: E-folding width of the restoring region at meridional boundaries, only used for single front.}
\end{longtable}
\end{center}
\subsection[config\_mixed\_layer\_eddy\_restoring\_tau]{\hyperref[sec:nm_tab_mixed_layer_eddy]{config\_mixed\_layer\_eddy\_restoring\_tau}}
\label{subsec:nm_sec_config_mixed_layer_eddy_restoring_tau}
\begin{center}
\begin{longtable}{| p{2.0in} || p{4.0in} |}
    \hline
    Type: & real \\
    \hline
    Units: & \si{days} \\
    \hline
    Default Value: & 5.0 \\
    \hline
    Possible Values: & Any real number larger than zero. \\
    \hline
    \caption{config\_mixed\_layer\_eddy\_restoring\_tau: Time scale for restoring at meridional boundaries, only used for single front.}
\end{longtable}
\end{center}
\subsection[config\_mixed\_layer\_eddy\_heat\_flux]{\hyperref[sec:nm_tab_mixed_layer_eddy]{config\_mixed\_layer\_eddy\_heat\_flux}}
\label{subsec:nm_sec_config_mixed_layer_eddy_heat_flux}
\begin{center}
\begin{longtable}{| p{2.0in} || p{4.0in} |}
    \hline
    Type: & real \\
    \hline
    Units: & \si{m.^oC.s^{-1}} \\
    \hline
    Default Value: & 0.0 \\
    \hline
    Possible Values: & Any real number. \\
    \hline
    \caption{config\_mixed\_layer\_eddy\_heat\_flux: Surface heat flux.}
\end{longtable}
\end{center}
\subsection[config\_mixed\_layer\_eddy\_evaporation\_flux]{\hyperref[sec:nm_tab_mixed_layer_eddy]{config\_mixed\_layer\_eddy\_evaporation\_flux}}
\label{subsec:nm_sec_config_mixed_layer_eddy_evaporation_flux}
\begin{center}
\begin{longtable}{| p{2.0in} || p{4.0in} |}
    \hline
    Type: & real \\
    \hline
    Units: & \si{kg.m^{-2}.s^{-1}} \\
    \hline
    Default Value: & 0.0 \\
    \hline
    Possible Values: & Any real number. \\
    \hline
    \caption{config\_mixed\_layer\_eddy\_evaporation\_flux: Evaporative flux}
\end{longtable}
\end{center}
\subsection[config\_mixed\_layer\_eddy\_wind\_stress\_zonal]{\hyperref[sec:nm_tab_mixed_layer_eddy]{config\_mixed\_layer\_eddy\_wind\_stress\_zonal}}
\label{subsec:nm_sec_config_mixed_layer_eddy_wind_stress_zonal}
\begin{center}
\begin{longtable}{| p{2.0in} || p{4.0in} |}
    \hline
    Type: & real \\
    \hline
    Units: & \si{Pa} \\
    \hline
    Default Value: & 0.0 \\
    \hline
    Possible Values: & Any real number. \\
    \hline
    \caption{config\_mixed\_layer\_eddy\_wind\_stress\_zonal: Surface zonal wind stress.}
\end{longtable}
\end{center}
\subsection[config\_mixed\_layer\_eddy\_wind\_stress\_meridional]{\hyperref[sec:nm_tab_mixed_layer_eddy]{config\_mixed\_layer\_eddy\_wind\_stress\_meridional}}
\label{subsec:nm_sec_config_mixed_layer_eddy_wind_stress_meridional}
\begin{center}
\begin{longtable}{| p{2.0in} || p{4.0in} |}
    \hline
    Type: & real \\
    \hline
    Units: & \si{Pa} \\
    \hline
    Default Value: & 0.0 \\
    \hline
    Possible Values: & Any real number. \\
    \hline
    \caption{config\_mixed\_layer\_eddy\_wind\_stress\_meridional: Surface meridional wind stress.}
\end{longtable}
\end{center}
\subsection[config\_mixed\_layer\_eddy\_coriolis\_parameter]{\hyperref[sec:nm_tab_mixed_layer_eddy]{config\_mixed\_layer\_eddy\_coriolis\_parameter}}
\label{subsec:nm_sec_config_mixed_layer_eddy_coriolis_parameter}
\begin{center}
\begin{longtable}{| p{2.0in} || p{4.0in} |}
    \hline
    Type: & real \\
    \hline
    Units: & \si{s^{-1}} \\
    \hline
    Default Value: & 1.0e-4 \\
    \hline
    Possible Values: & Any real number. \\
    \hline
    \caption{config\_mixed\_layer\_eddy\_coriolis\_parameter: Coriolis parameter for entrie domain.}
\end{longtable}
\end{center}
\section[test\_sht]{\hyperref[sec:nm_tab_test_sht]{test\_sht}}
\label{sec:nm_sec_test_sht}
\subsection[config\_test\_sht\_function\_option]{\hyperref[sec:nm_tab_test_sht]{config\_test\_sht\_function\_option}}
\label{subsec:nm_sec_config_test_sht_function_option}
\begin{center}
\begin{longtable}{| p{2.0in} || p{4.0in} |}
    \hline
    Type: & integer \\
    \hline
    Units: & \si{unitless} \\
    \hline
    Default Value: & 1 \\
    \hline
    Possible Values: & Any integer value \\
    \hline
    \caption{config\_test\_sht\_function\_option: Function to apply forward and inverse transformations to}
\end{longtable}
\end{center}
\subsection[config\_test\_sht\_cosine\_bell\_lat\_center]{\hyperref[sec:nm_tab_test_sht]{config\_test\_sht\_cosine\_bell\_lat\_center}}
\label{subsec:nm_sec_config_test_sht_cosine_bell_lat_center}
\begin{center}
\begin{longtable}{| p{2.0in} || p{4.0in} |}
    \hline
    Type: & real \\
    \hline
    Units: & \si{radians} \\
    \hline
    Default Value: & 0.0 \\
    \hline
    Possible Values: & Any real number between -pi/2 and pi/2 \\
    \hline
    \caption{config\_test\_sht\_cosine\_bell\_lat\_center: latitude center of cosine bell}
\end{longtable}
\end{center}
\subsection[config\_test\_sht\_cosine\_bell\_lon\_center]{\hyperref[sec:nm_tab_test_sht]{config\_test\_sht\_cosine\_bell\_lon\_center}}
\label{subsec:nm_sec_config_test_sht_cosine_bell_lon_center}
\begin{center}
\begin{longtable}{| p{2.0in} || p{4.0in} |}
    \hline
    Type: & real \\
    \hline
    Units: & \si{radians} \\
    \hline
    Default Value: & 3.141592 \\
    \hline
    Possible Values: & Any non-negative real number between 0 and 2pi \\
    \hline
    \caption{config\_test\_sht\_cosine\_bell\_lon\_center: longitude center of cosine bell}
\end{longtable}
\end{center}
\subsection[config\_test\_sht\_cosine\_bell\_psi0]{\hyperref[sec:nm_tab_test_sht]{config\_test\_sht\_cosine\_bell\_psi0}}
\label{subsec:nm_sec_config_test_sht_cosine_bell_psi0}
\begin{center}
\begin{longtable}{| p{2.0in} || p{4.0in} |}
    \hline
    Type: & real \\
    \hline
    Units: & \si{unitless} \\
    \hline
    Default Value: & 1.0 \\
    \hline
    Possible Values: & Any real number \\
    \hline
    \caption{config\_test\_sht\_cosine\_bell\_psi0: hill max of tracer}
\end{longtable}
\end{center}
\subsection[config\_test\_sht\_cosine\_bell\_radius]{\hyperref[sec:nm_tab_test_sht]{config\_test\_sht\_cosine\_bell\_radius}}
\label{subsec:nm_sec_config_test_sht_cosine_bell_radius}
\begin{center}
\begin{longtable}{| p{2.0in} || p{4.0in} |}
    \hline
    Type: & real \\
    \hline
    Units: & \si{m} \\
    \hline
    Default Value: & 2123666.667 \\
    \hline
    Possible Values: & Any non-negative real number between 0 and 2pi \\
    \hline
    \caption{config\_test\_sht\_cosine\_bell\_radius: radius of cosine bell}
\end{longtable}
\end{center}
\subsection[config\_test\_sht\_function3\_cell\_width\_equator]{\hyperref[sec:nm_tab_test_sht]{config\_test\_sht\_function3\_cell\_width\_equator}}
\label{subsec:nm_sec_config_test_sht_function3_cell_width_equator}
\begin{center}
\begin{longtable}{| p{2.0in} || p{4.0in} |}
    \hline
    Type: & real \\
    \hline
    Units: & \si{km} \\
    \hline
    Default Value: & 30.0 \\
    \hline
    Possible Values: & Any non-negative real number \\
    \hline
    \caption{config\_test\_sht\_function3\_cell\_width\_equator: cell width at equator for config\_test\_sht\_function\_option = 3}
\end{longtable}
\end{center}
\subsection[config\_test\_sht\_function3\_cell\_width\_pole]{\hyperref[sec:nm_tab_test_sht]{config\_test\_sht\_function3\_cell\_width\_pole}}
\label{subsec:nm_sec_config_test_sht_function3_cell_width_pole}
\begin{center}
\begin{longtable}{| p{2.0in} || p{4.0in} |}
    \hline
    Type: & real \\
    \hline
    Units: & \si{km} \\
    \hline
    Default Value: & 120.0 \\
    \hline
    Possible Values: & Any non-negative real number \\
    \hline
    \caption{config\_test\_sht\_function3\_cell\_width\_pole: cell width at pole for config\_test\_sht\_function\_option = 3}
\end{longtable}
\end{center}
\subsection[config\_test\_sht\_function3\_lat\_transition\_start]{\hyperref[sec:nm_tab_test_sht]{config\_test\_sht\_function3\_lat\_transition\_start}}
\label{subsec:nm_sec_config_test_sht_function3_lat_transition_start}
\begin{center}
\begin{longtable}{| p{2.0in} || p{4.0in} |}
    \hline
    Type: & real \\
    \hline
    Units: & \si{deg} \\
    \hline
    Default Value: & 1.0 \\
    \hline
    Possible Values: & Any non-negative real number \\
    \hline
    \caption{config\_test\_sht\_function3\_lat\_transition\_start: transition start for config\_test\_sht\_function\_option = 3}
\end{longtable}
\end{center}
\subsection[config\_test\_sht\_function3\_lat\_transition\_width]{\hyperref[sec:nm_tab_test_sht]{config\_test\_sht\_function3\_lat\_transition\_width}}
\label{subsec:nm_sec_config_test_sht_function3_lat_transition_width}
\begin{center}
\begin{longtable}{| p{2.0in} || p{4.0in} |}
    \hline
    Type: & real \\
    \hline
    Units: & \si{deg} \\
    \hline
    Default Value: & 10.0 \\
    \hline
    Possible Values: & Any non-negative real number \\
    \hline
    \caption{config\_test\_sht\_function3\_lat\_transition\_width: transition width for config\_test\_sht\_function\_option = 3}
\end{longtable}
\end{center}
\subsection[config\_test\_sht\_n\_iterations]{\hyperref[sec:nm_tab_test_sht]{config\_test\_sht\_n\_iterations}}
\label{subsec:nm_sec_config_test_sht_n_iterations}
\begin{center}
\begin{longtable}{| p{2.0in} || p{4.0in} |}
    \hline
    Type: & integer \\
    \hline
    Units: & \si{m} \\
    \hline
    Default Value: & 10 \\
    \hline
    Possible Values: & Any positive integer \\
    \hline
    \caption{config\_test\_sht\_n\_iterations: number of times to run forward/inverse transformation for timings}
\end{longtable}
\end{center}
\section[parabolic\_bowl]{\hyperref[sec:nm_tab_parabolic_bowl]{parabolic\_bowl}}
\label{sec:nm_sec_parabolic_bowl}
\subsection[config\_parabolic\_bowl\_vert\_levels]{\hyperref[sec:nm_tab_parabolic_bowl]{config\_parabolic\_bowl\_vert\_levels}}
\label{subsec:nm_sec_config_parabolic_bowl_vert_levels}
\begin{center}
\begin{longtable}{| p{2.0in} || p{4.0in} |}
    \hline
    Type: & integer \\
    \hline
    Units: & \si{unitless} \\
    \hline
    Default Value: & 3 \\
    \hline
    Possible Values: & Any positive integer \\
    \hline
    \caption{config\_parabolic\_bowl\_vert\_levels: Number of vertical levels in parabolic bowl.}
\end{longtable}
\end{center}
\subsection[config\_parabolic\_bowl\_Coriolis\_parameter]{\hyperref[sec:nm_tab_parabolic_bowl]{config\_parabolic\_bowl\_Coriolis\_parameter}}
\label{subsec:nm_sec_config_parabolic_bowl_Coriolis_parameter}
\begin{center}
\begin{longtable}{| p{2.0in} || p{4.0in} |}
    \hline
    Type: & real \\
    \hline
    Units: & \si{1/s} \\
    \hline
    Default Value: & 1.031e-4 \\
    \hline
    Possible Values: & Any real number \\
    \hline
    \caption{config\_parabolic\_bowl\_Coriolis\_parameter: Coriolis paramter}
\end{longtable}
\end{center}
\subsection[config\_parabolic\_bowl\_eta0]{\hyperref[sec:nm_tab_parabolic_bowl]{config\_parabolic\_bowl\_eta0}}
\label{subsec:nm_sec_config_parabolic_bowl_eta0}
\begin{center}
\begin{longtable}{| p{2.0in} || p{4.0in} |}
    \hline
    Type: & real \\
    \hline
    Units: & \si{m} \\
    \hline
    Default Value: & 2.0 \\
    \hline
    Possible Values: & Any real number \\
    \hline
    \caption{config\_parabolic\_bowl\_eta0: surface elevation magnitude}
\end{longtable}
\end{center}
\subsection[config\_parabolic\_bowl\_b0]{\hyperref[sec:nm_tab_parabolic_bowl]{config\_parabolic\_bowl\_b0}}
\label{subsec:nm_sec_config_parabolic_bowl_b0}
\begin{center}
\begin{longtable}{| p{2.0in} || p{4.0in} |}
    \hline
    Type: & real \\
    \hline
    Units: & \si{m} \\
    \hline
    Default Value: & 50.0 \\
    \hline
    Possible Values: & Any real number \\
    \hline
    \caption{config\_parabolic\_bowl\_b0: maximum water depth}
\end{longtable}
\end{center}
\subsection[config\_parabolic\_bowl\_omega]{\hyperref[sec:nm_tab_parabolic_bowl]{config\_parabolic\_bowl\_omega}}
\label{subsec:nm_sec_config_parabolic_bowl_omega}
\begin{center}
\begin{longtable}{| p{2.0in} || p{4.0in} |}
    \hline
    Type: & real \\
    \hline
    Units: & \si{1/s} \\
    \hline
    Default Value: & 1.4544e-4 \\
    \hline
    Possible Values: & Any real number \\
    \hline
    \caption{config\_parabolic\_bowl\_omega: Angular frequency of oscillation}
\end{longtable}
\end{center}
\subsection[config\_parabolic\_bowl\_gravity]{\hyperref[sec:nm_tab_parabolic_bowl]{config\_parabolic\_bowl\_gravity}}
\label{subsec:nm_sec_config_parabolic_bowl_gravity}
\begin{center}
\begin{longtable}{| p{2.0in} || p{4.0in} |}
    \hline
    Type: & real \\
    \hline
    Units: & \si{m/s^2} \\
    \hline
    Default Value: & 9.81 \\
    \hline
    Possible Values: & Any real number \\
    \hline
    \caption{config\_parabolic\_bowl\_gravity: Gravitational accerlation}
\end{longtable}
\end{center}
\subsection[config\_parabolic\_bowl\_adjust\_domain\_center]{\hyperref[sec:nm_tab_parabolic_bowl]{config\_parabolic\_bowl\_adjust\_domain\_center}}
\label{subsec:nm_sec_config_parabolic_bowl_adjust_domain_center}
\begin{center}
\begin{longtable}{| p{2.0in} || p{4.0in} |}
    \hline
    Type: & logical \\
    \hline
    Units: & \si{unitless} \\
    \hline
    Default Value: & true \\
    \hline
    Possible Values: & .true. or .false. \\
    \hline
    \caption{config\_parabolic\_bowl\_adjust\_domain\_center: Flag to adjust mesh coordinates}
\end{longtable}
\end{center}
\section[partial\_cells]{\hyperref[sec:nm_tab_partial_cells]{partial\_cells}}
\label{sec:nm_sec_partial_cells}
\subsection[config\_alter\_ICs\_for\_pcs]{\hyperref[sec:nm_tab_partial_cells]{config\_alter\_ICs\_for\_pcs}}
\label{subsec:nm_sec_config_alter_ICs_for_pcs}
\begin{center}
\begin{longtable}{| p{2.0in} || p{4.0in} |}
    \hline
    Type: & logical \\
    \hline
    Units: & \si{unitless} \\
    \hline
    Default Value: & .false. \\
    \hline
    Possible Values: & .true. or .false. \\
    \hline
    \caption{config\_alter\_ICs\_for\_pcs: If true, initial conditions are altered according to the config\_pc\_alteration\_type flag.}
\end{longtable}
\end{center}
\subsection[config\_pc\_alteration\_type]{\hyperref[sec:nm_tab_partial_cells]{config\_pc\_alteration\_type}}
\label{subsec:nm_sec_config_pc_alteration_type}
\begin{center}
\begin{longtable}{| p{2.0in} || p{4.0in} |}
    \hline
    Type: & character \\
    \hline
    Units: & \si{unitless} \\
    \hline
    Default Value: & full\_cell \\
    \hline
    Possible Values: & 'full\_cell' or 'partial\_cell' \\
    \hline
    \caption{config\_pc\_alteration\_type: Determines the method of initial condition alteration for partial bottom (and possibly top) cells. 'partial\_cell' alters layerThickness, interpolates all tracers in the bottom (and top) cell based on the bottomDepth (or ssh) variable, and alters bottomDepth (or ssh) to enforce the minimum pc fraction. 'full\_cell' alters bottomDepth (or ssh) to have full cells everywhere, based on the refBottomDepth variable.}
\end{longtable}
\end{center}
\subsection[config\_min\_pc\_fraction]{\hyperref[sec:nm_tab_partial_cells]{config\_min\_pc\_fraction}}
\label{subsec:nm_sec_config_min_pc_fraction}
\begin{center}
\begin{longtable}{| p{2.0in} || p{4.0in} |}
    \hline
    Type: & real \\
    \hline
    Units: & \si{unitless} \\
    \hline
    Default Value: & 0.10 \\
    \hline
    Possible Values: & Any real between 0 and 1. \\
    \hline
    \caption{config\_min\_pc\_fraction: Determines the minimum fraction of a cell altering the initial conditions can create.}
\end{longtable}
\end{center}
\section[init\_setup]{\hyperref[sec:nm_tab_init_setup]{init\_setup}}
\label{sec:nm_sec_init_setup}
\subsection[config\_init\_configuration]{\hyperref[sec:nm_tab_init_setup]{config\_init\_configuration}}
\label{subsec:nm_sec_config_init_configuration}
\begin{center}
\begin{longtable}{| p{2.0in} || p{4.0in} |}
    \hline
    Type: & character \\
    \hline
    Units: & \si{unitless} \\
    \hline
    Default Value: & none \\
    \hline
    Possible Values: & Any configuration name \\
    \hline
    \caption{config\_init\_configuration: Name of configuration to create.}
\end{longtable}
\end{center}
\subsection[config\_expand\_sphere]{\hyperref[sec:nm_tab_init_setup]{config\_expand\_sphere}}
\label{subsec:nm_sec_config_expand_sphere}
\begin{center}
\begin{longtable}{| p{2.0in} || p{4.0in} |}
    \hline
    Type: & logical \\
    \hline
    Units: & \si{unitless} \\
    \hline
    Default Value: & .false. \\
    \hline
    Possible Values: & .true. or .false. \\
    \hline
    \caption{config\_expand\_sphere: Logical flag that controls if a spherical mesh is expanded to an earth sized sphere or not.}
\end{longtable}
\end{center}
\subsection[config\_realistic\_coriolis\_parameter]{\hyperref[sec:nm_tab_init_setup]{config\_realistic\_coriolis\_parameter}}
\label{subsec:nm_sec_config_realistic_coriolis_parameter}
\begin{center}
\begin{longtable}{| p{2.0in} || p{4.0in} |}
    \hline
    Type: & logical \\
    \hline
    Units: & \si{unitless} \\
    \hline
    Default Value: & .false. \\
    \hline
    Possible Values: & .true. or .false. \\
    \hline
    \caption{config\_realistic\_coriolis\_parameter: Logical flag that controls if a spherical mesh will get realistic coriolis parameters or not.}
\end{longtable}
\end{center}
\subsection[config\_write\_cull\_cell\_mask]{\hyperref[sec:nm_tab_init_setup]{config\_write\_cull\_cell\_mask}}
\label{subsec:nm_sec_config_write_cull_cell_mask}
\begin{center}
\begin{longtable}{| p{2.0in} || p{4.0in} |}
    \hline
    Type: & logical \\
    \hline
    Units: & \si{unitless} \\
    \hline
    Default Value: & .true. \\
    \hline
    Possible Values: & .true. or .false. \\
    \hline
    \caption{config\_write\_cull\_cell\_mask: Logicial flag that controls if the cullCell field is written to output.}
\end{longtable}
\end{center}
\subsection[config\_vertical\_grid]{\hyperref[sec:nm_tab_init_setup]{config\_vertical\_grid}}
\label{subsec:nm_sec_config_vertical_grid}
\begin{center}
\begin{longtable}{| p{2.0in} || p{4.0in} |}
    \hline
    Type: & character \\
    \hline
    Units: & \si{unitless} \\
    \hline
    Default Value: & uniform \\
    \hline
    Possible Values: & 'uniform', '60layerPHC', '42layerWOCE', '100layerE3SMv1', '1dCVTgenerator', ... \\
    \hline
    \caption{config\_vertical\_grid: Name of vertical grid to use in configuration generation}
\end{longtable}
\end{center}
\section[CVTgenerator]{\hyperref[sec:nm_tab_CVTgenerator]{CVTgenerator}}
\label{sec:nm_sec_CVTgenerator}
\subsection[config\_1dCVTgenerator\_stretch1]{\hyperref[sec:nm_tab_CVTgenerator]{config\_1dCVTgenerator\_stretch1}}
\label{subsec:nm_sec_config_1dCVTgenerator_stretch1}
\begin{center}
\begin{longtable}{| p{2.0in} || p{4.0in} |}
    \hline
    Type: & real \\
    \hline
    Units: & \si{unitless} \\
    \hline
    Default Value: & 1.0770 \\
    \hline
    Possible Values: & Any positive non-zero integer. \\
    \hline
    \caption{config\_1dCVTgenerator\_stretch1: Parameter for the 1D CVT vertical grid generator.}
\end{longtable}
\end{center}
\subsection[config\_1dCVTgenerator\_stretch2]{\hyperref[sec:nm_tab_CVTgenerator]{config\_1dCVTgenerator\_stretch2}}
\label{subsec:nm_sec_config_1dCVTgenerator_stretch2}
\begin{center}
\begin{longtable}{| p{2.0in} || p{4.0in} |}
    \hline
    Type: & real \\
    \hline
    Units: & \si{unitless} \\
    \hline
    Default Value: & 1.0275 \\
    \hline
    Possible Values: & Any positive non-zero integer. \\
    \hline
    \caption{config\_1dCVTgenerator\_stretch2: Parameter for the 1D CVT vertical grid generator.}
\end{longtable}
\end{center}
\subsection[config\_1dCVTgenerator\_dzSeed]{\hyperref[sec:nm_tab_CVTgenerator]{config\_1dCVTgenerator\_dzSeed}}
\label{subsec:nm_sec_config_1dCVTgenerator_dzSeed}
\begin{center}
\begin{longtable}{| p{2.0in} || p{4.0in} |}
    \hline
    Type: & real \\
    \hline
    Units: & \si{unitless} \\
    \hline
    Default Value: & 1.2 \\
    \hline
    Possible Values: & Any positive non-zero integer. \\
    \hline
    \caption{config\_1dCVTgenerator\_dzSeed: Seed thickness of the first layer for the 1D CVT vertical grid generator.}
\end{longtable}
\end{center}
\section[init\_vertical\_grid]{\hyperref[sec:nm_tab_init_vertical_grid]{init\_vertical\_grid}}
\label{sec:nm_sec_init_vertical_grid}
\subsection[config\_init\_vertical\_grid\_type]{\hyperref[sec:nm_tab_init_vertical_grid]{config\_init\_vertical\_grid\_type}}
\label{subsec:nm_sec_config_init_vertical_grid_type}
\begin{center}
\begin{longtable}{| p{2.0in} || p{4.0in} |}
    \hline
    Type: & character \\
    \hline
    Units: & \si{unitless} \\
    \hline
    Default Value: & z-star \\
    \hline
    Possible Values: & 'z-star', 'z-level', or 'haney-number' \\
    \hline
    \caption{config\_init\_vertical\_grid\_type: Which vertical grid to initialize with.  Without ice-shelf cavities (i.e. ssh=0 everywhere), 'z-star' and 'z-level' are the same.}
\end{longtable}
\end{center}
\section[constrain\_Haney\_number]{\hyperref[sec:nm_tab_constrain_Haney_number]{constrain\_Haney\_number}}
\label{sec:nm_sec_constrain_Haney_number}
\subsection[config\_rx1\_outer\_iter\_count]{\hyperref[sec:nm_tab_constrain_Haney_number]{config\_rx1\_outer\_iter\_count}}
\label{subsec:nm_sec_config_rx1_outer_iter_count}
\begin{center}
\begin{longtable}{| p{2.0in} || p{4.0in} |}
    \hline
    Type: & integer \\
    \hline
    Units: & \si{unitless} \\
    \hline
    Default Value: & 20 \\
    \hline
    Possible Values: & any positive integer \\
    \hline
    \caption{config\_rx1\_outer\_iter\_count: The number of outer iterations (first smoothing then rx1 constraint) during initialization of the vertical grid.}
\end{longtable}
\end{center}
\subsection[config\_rx1\_inner\_iter\_count]{\hyperref[sec:nm_tab_constrain_Haney_number]{config\_rx1\_inner\_iter\_count}}
\label{subsec:nm_sec_config_rx1_inner_iter_count}
\begin{center}
\begin{longtable}{| p{2.0in} || p{4.0in} |}
    \hline
    Type: & integer \\
    \hline
    Units: & \si{unitless} \\
    \hline
    Default Value: & 10 \\
    \hline
    Possible Values: & any positive integer \\
    \hline
    \caption{config\_rx1\_inner\_iter\_count: The number of iterations used to constrain rx1 in each layer.}
\end{longtable}
\end{center}
\subsection[config\_rx1\_init\_inner\_weight]{\hyperref[sec:nm_tab_constrain_Haney_number]{config\_rx1\_init\_inner\_weight}}
\label{subsec:nm_sec_config_rx1_init_inner_weight}
\begin{center}
\begin{longtable}{| p{2.0in} || p{4.0in} |}
    \hline
    Type: & real \\
    \hline
    Units: & \si{unitless} \\
    \hline
    Default Value: & 0.1 \\
    \hline
    Possible Values: & a positive value less than or equal to 1 \\
    \hline
    \caption{config\_rx1\_init\_inner\_weight: The weight by which layer thicknesses are altered at the beginning of inner iteration. This weight linearly increases to 1.0 by the final iteration.}
\end{longtable}
\end{center}
\subsection[config\_rx1\_max]{\hyperref[sec:nm_tab_constrain_Haney_number]{config\_rx1\_max}}
\label{subsec:nm_sec_config_rx1_max}
\begin{center}
\begin{longtable}{| p{2.0in} || p{4.0in} |}
    \hline
    Type: & real \\
    \hline
    Units: & \si{unitless} \\
    \hline
    Default Value: & 5.0 \\
    \hline
    Possible Values: & any positive value, typically greater than or equal to 1 \\
    \hline
    \caption{config\_rx1\_max: The maximum value rx1Max of the Haney number (rx1) after modification of the vertical grid}
\end{longtable}
\end{center}
\subsection[config\_rx1\_horiz\_smooth\_weight]{\hyperref[sec:nm_tab_constrain_Haney_number]{config\_rx1\_horiz\_smooth\_weight}}
\label{subsec:nm_sec_config_rx1_horiz_smooth_weight}
\begin{center}
\begin{longtable}{| p{2.0in} || p{4.0in} |}
    \hline
    Type: & real \\
    \hline
    Units: & \si{unitless} \\
    \hline
    Default Value: & 1.0 \\
    \hline
    Possible Values: & A non-negative number \\
    \hline
    \caption{config\_rx1\_horiz\_smooth\_weight: Relative weight of horizontal neighbors compared to this cell when smoothing vertical stretching}
\end{longtable}
\end{center}
\subsection[config\_rx1\_vert\_smooth\_weight]{\hyperref[sec:nm_tab_constrain_Haney_number]{config\_rx1\_vert\_smooth\_weight}}
\label{subsec:nm_sec_config_rx1_vert_smooth_weight}
\begin{center}
\begin{longtable}{| p{2.0in} || p{4.0in} |}
    \hline
    Type: & real \\
    \hline
    Units: & \si{unitless} \\
    \hline
    Default Value: & 1.0 \\
    \hline
    Possible Values: & A non-negative number \\
    \hline
    \caption{config\_rx1\_vert\_smooth\_weight: Relative weight of vertical neighbors compared to this cell when smoothing vertical stretching}
\end{longtable}
\end{center}
\subsection[config\_rx1\_slope\_weight]{\hyperref[sec:nm_tab_constrain_Haney_number]{config\_rx1\_slope\_weight}}
\label{subsec:nm_sec_config_rx1_slope_weight}
\begin{center}
\begin{longtable}{| p{2.0in} || p{4.0in} |}
    \hline
    Type: & real \\
    \hline
    Units: & \si{unitless} \\
    \hline
    Default Value: & 1e-1 \\
    \hline
    Possible Values: & A non-negative number \\
    \hline
    \caption{config\_rx1\_slope\_weight: Weight used to nudge level interfaces toward being flat (thus decreasing the Haney number)}
\end{longtable}
\end{center}
\subsection[config\_rx1\_zstar\_weight]{\hyperref[sec:nm_tab_constrain_Haney_number]{config\_rx1\_zstar\_weight}}
\label{subsec:nm_sec_config_rx1_zstar_weight}
\begin{center}
\begin{longtable}{| p{2.0in} || p{4.0in} |}
    \hline
    Type: & real \\
    \hline
    Units: & \si{unitless} \\
    \hline
    Default Value: & 1.0 \\
    \hline
    Possible Values: & A non-negative number \\
    \hline
    \caption{config\_rx1\_zstar\_weight: Weight used to nudge vertical stretching toward z-star during each outer iteration}
\end{longtable}
\end{center}
\subsection[config\_rx1\_horiz\_smooth\_open\_ocean\_cells]{\hyperref[sec:nm_tab_constrain_Haney_number]{config\_rx1\_horiz\_smooth\_open\_ocean\_cells}}
\label{subsec:nm_sec_config_rx1_horiz_smooth_open_ocean_cells}
\begin{center}
\begin{longtable}{| p{2.0in} || p{4.0in} |}
    \hline
    Type: & integer \\
    \hline
    Units: & \si{unitless} \\
    \hline
    Default Value: & 20 \\
    \hline
    Possible Values: & any non-negative integer, 0 indiates no buffer region. \\
    \hline
    \caption{config\_rx1\_horiz\_smooth\_open\_ocean\_cells: The size (in cells) of a buffer region around land ice for smoothing.  Smoothing is performed under land ice and in the buffer region of open ocean.}
\end{longtable}
\end{center}
\subsection[config\_rx1\_min\_levels]{\hyperref[sec:nm_tab_constrain_Haney_number]{config\_rx1\_min\_levels}}
\label{subsec:nm_sec_config_rx1_min_levels}
\begin{center}
\begin{longtable}{| p{2.0in} || p{4.0in} |}
    \hline
    Type: & integer \\
    \hline
    Units: & \si{unitless} \\
    \hline
    Default Value: & 3 \\
    \hline
    Possible Values: & a positive integer \\
    \hline
    \caption{config\_rx1\_min\_levels: The minimum number of layers in the ocean column in the smoothed region.}
\end{longtable}
\end{center}
\subsection[config\_rx1\_min\_layer\_thickness]{\hyperref[sec:nm_tab_constrain_Haney_number]{config\_rx1\_min\_layer\_thickness}}
\label{subsec:nm_sec_config_rx1_min_layer_thickness}
\begin{center}
\begin{longtable}{| p{2.0in} || p{4.0in} |}
    \hline
    Type: & real \\
    \hline
    Units: & \si{m} \\
    \hline
    Default Value: & 1.0 \\
    \hline
    Possible Values: & a positive value \\
    \hline
    \caption{config\_rx1\_min\_layer\_thickness: The minimum layer thickness in the smoothed region.}
\end{longtable}
\end{center}
