\chapter[Variable definitions]{\hyperref[chap:variable_sections]{Variable definitions}}
\label{chap:variable_tables}
Embedded links point to more detailed variable information in the appendix.
\section[state]{\hyperref[sec:var_sec_state]{state}}
\label{sec:var_tab_state}
The state data structure contains a set of prognostic and diagnostic fields
that are time dependent. The fields contained inside of state have two time
levels available.  (More than two is possible within the MPAS Framework, but 
only two have been implemented in the Land Ice core.)

{\small
\begin{center}
\begin{longtable}{| p{2.0in} | p{4.0in} |}
	\hline
	{\bf Name} & {\bf Description} \\
	\hline
	\hyperref[subsec:var_sec_state_temperature]{temperature} & potential temperature \\
	\hline
	\hyperref[subsec:var_sec_state_salinity]{salinity} & salinity \\
	\hline
	\hyperref[subsec:var_sec_state_tracer1]{tracer1} & A tracer with value 1.0 to test conservation.  {\color{red} REMOVE THIS VARIABLE} \\
	\hline
	\hyperref[subsec:var_sec_state_xtime]{xtime} & model time, with format 'YYYY-MM-DD\_HH:MM:SS' \\
	\hline
	\hyperref[subsec:var_sec_state_normalVelocity]{normalVelocity} & horizonal velocity, normal component to an edge \\
	\hline
	\hyperref[subsec:var_sec_state_layerThickness]{layerThickness} & layer thickness \\
	\hline
	\hyperref[subsec:var_sec_state_density]{density} & density \\
	\hline
	\hyperref[subsec:var_sec_state_normalBarotropicVelocity]{normalBarotropicVelocity} & barotropic velocity, used in split-explicit time-stepping \\
	\hline
	\hyperref[subsec:var_sec_state_ssh]{ssh} & sea surface height \\
	\hline
	\hyperref[subsec:var_sec_state_normalBarotropicVelocitySubcycle]{normalBarotropicVelocitySubcycle} & barotropic velocity, used in subcycling in stage 2 of split-explicit time-stepping \\
	\hline
	\hyperref[subsec:var_sec_state_sshSubcycle]{sshSubcycle} & sea surface height, used in subcycling in stage 2 of split-explicit time-stepping \\
	\hline
	\hyperref[subsec:var_sec_state_barotropicThicknessFlux]{barotropicThicknessFlux} & Barotropic thickness flux at each edge, used to advance sea surface height in each subcycle of stage 2 of the split-explicit algorithm. \\
	\hline
	\hyperref[subsec:var_sec_state_barotropicForcing]{barotropicForcing} & Barotropic tendency computed from the baroclinic equations in stage 1 of the split-explicit algorithm. \\
	\hline
	\hyperref[subsec:var_sec_state_normalBaroclinicVelocity]{normalBaroclinicVelocity} & baroclinic velocity, used in split-explicit time-stepping \\
	\hline
	\hyperref[subsec:var_sec_state_zMid]{zMid} & z-coordinate of the mid-depth of the layer \\
	\hline
	\hyperref[subsec:var_sec_state_tangentialVelocity]{tangentialVelocity} & horizontal velocity, tangential to an edge \\
	\hline
	\hyperref[subsec:var_sec_state_uTransport]{uTransport} & horizontal velocity used to transport mass and tracers \\
	\hline
	\hyperref[subsec:var_sec_state_uBolusGM]{uBolusGM} & {\bf \color{red} MISSING} \\
	\hline
	\hyperref[subsec:var_sec_state_uBolusGMX]{uBolusGMX} & {\bf \color{red} MISSING} \\
	\hline
	\hyperref[subsec:var_sec_state_uBolusGMY]{uBolusGMY} & {\bf \color{red} MISSING} \\
	\hline
	\hyperref[subsec:var_sec_state_uBolusGMZ]{uBolusGMZ} & {\bf \color{red} MISSING} \\
	\hline
	\hyperref[subsec:var_sec_state_uBolusGMZonal]{uBolusGMZonal} & {\bf \color{red} MISSING} \\
	\hline
	\hyperref[subsec:var_sec_state_uBolusGMMeridional]{uBolusGMMeridional} & {\bf \color{red} MISSING} \\
	\hline
	\hyperref[subsec:var_sec_state_hEddyFlux]{hEddyFlux} & {\bf \color{red} MISSING} \\
	\hline
	\hyperref[subsec:var_sec_state_h_kappa]{h\_kappa} & {\bf \color{red} MISSING} \\
	\hline
	\hyperref[subsec:var_sec_state_h_kappa_q]{h\_kappa\_q} & {\bf \color{red} MISSING} \\
	\hline
	\hyperref[subsec:var_sec_state_divergence]{divergence} & divergence of horizonal velocity \\
	\hline
	\hyperref[subsec:var_sec_state_relativeVorticity]{relativeVorticity} & curl of horizontal velocity \\
	\hline
	\hyperref[subsec:var_sec_state_potentialVorticityEdge]{potentialVorticityEdge} & vorticity averaged from vertices to edges \\
	\hline
	\hyperref[subsec:var_sec_state_potentialVorticityVertex]{potentialVorticityVertex} & curl of horizontal velocity defined at vertices \\
	\hline
	\hyperref[subsec:var_sec_state_potentialVorticityCell]{potentialVorticityCell} & curl of horizontal velocity defined at cell centers \\
	\hline
	\hyperref[subsec:var_sec_state_layerThicknessEdge]{layerThicknessEdge} & layer thickness averaged from cell center to edges \\
	\hline
	\hyperref[subsec:var_sec_state_layerThicknessVertex]{layerThicknessVertex} & layer thickness averaged from cell center to vertices \\
	\hline
	\hyperref[subsec:var_sec_state_kineticEnergy]{kineticEnergy} & kinetic energy of horizonal velocity \\
	\hline
	\hyperref[subsec:var_sec_state_kineticEnergyVertex]{kineticEnergyVertex} & kinetic energy of horizonal velocity defined at vertices \\
	\hline
	\hyperref[subsec:var_sec_state_kineticEnergyVertexOnCells]{kineticEnergyVertexOnCells} & kinetic energy of horizonal velocity defined at vertices \\
	\hline
	\hyperref[subsec:var_sec_state_kineticEnergyEdge]{kineticEnergyEdge} & kinetic energy of horizonal velocity defined at edges \\
	\hline
	\hyperref[subsec:var_sec_state_normalVelocityX]{normalVelocityX} & component of horizontal velocity in the x-direction (cartesian) \\
	\hline
	\hyperref[subsec:var_sec_state_normalVelocityY]{normalVelocityY} & component of horizontal velocity in the y-direction (cartesian) \\
	\hline
	\hyperref[subsec:var_sec_state_normalVelocityZ]{normalVelocityZ} & component of horizontal velocity in the z-direction (cartesian) \\
	\hline
	\hyperref[subsec:var_sec_state_normalVelocityZonal]{normalVelocityZonal} & component of horizontal velocity in the eastward direction \\
	\hline
	\hyperref[subsec:var_sec_state_normalVelocityMeridional]{normalVelocityMeridional} & component of horizontal velocity in the northward \\
	\hline
	\hyperref[subsec:var_sec_state_normalVelocityForcingReconstructX]{normalVelocityForcingReconstructX} & wind stress in the x-direction (cartesian) \\
	\hline
	\hyperref[subsec:var_sec_state_normalVelocityForcingReconstructY]{normalVelocityForcingReconstructY} & wind stress in the y-direction (cartesian) \\
	\hline
	\hyperref[subsec:var_sec_state_normalVelocityForcingReconstructZ]{normalVelocityForcingReconstructZ} & wind stress in the z-direction (cartesian) \\
	\hline
	\hyperref[subsec:var_sec_state_normalVelocityForcingReconstructZonal]{normalVelocityForcingReconstructZonal} & wind stress in the eastward direction \\
	\hline
	\hyperref[subsec:var_sec_state_normalVelocityForcingReconstructMeridional]{normalVelocityForcingReconstructMeridional} & wind stress in the northward direction \\
	\hline
	\hyperref[subsec:var_sec_state_montgomeryPotential]{montgomeryPotential} & Montgomery potential, may be used as the pressure for isopycnal coordinates. \\
	\hline
	\hyperref[subsec:var_sec_state_pressure]{pressure} & pressure used in the momentum equation \\
	\hline
	\hyperref[subsec:var_sec_state_vertTransportVelocityTop]{vertTransportVelocityTop} & vertical transport through the layer interface at the top of the cell \\
	\hline
	\hyperref[subsec:var_sec_state_vertVelocityTop]{vertVelocityTop} & vertical velocity defined at center (horizonally) and top (vertically) of cell \\
	\hline
	\hyperref[subsec:var_sec_state_displacedDensity]{displacedDensity} & potential density displaced to the mid-depth of top layer \\
	\hline
	\hyperref[subsec:var_sec_state_BruntVaisalaFreqTop]{BruntVaisalaFreqTop} & Brunt Vaisala frequency defined at the center (horizontally) and top (vertically) of cell \\
	\hline
	\hyperref[subsec:var_sec_state_viscosity]{viscosity} & horizontal viscosity \\
	\hline
	\hyperref[subsec:var_sec_state_vh]{vh} & thickness flux in the tangent direction (from vertex1 to vertex2) \\
	\hline
	\hyperref[subsec:var_sec_state_circulation]{circulation} & area-integrated vorticity \\
	\hline
	\hyperref[subsec:var_sec_state_gradVor_t]{gradVor\_t} & gradient of vorticity in the tangent direction (from vertex1 to vertex2) \\
	\hline
	\hyperref[subsec:var_sec_state_gradVor_n]{gradVor\_n} & gradient of vorticity in the normal direction (from cell1 to cell2) \\
	\hline
	\hyperref[subsec:var_sec_state_areaCellGlobal]{areaCellGlobal} & sum of the areaCell variable over the full domain, used to normalize global statistics \\
	\hline
	\hyperref[subsec:var_sec_state_areaEdgeGlobal]{areaEdgeGlobal} & sum of the areaEdge variable over the full domain, used to normalize global statistics \\
	\hline
	\hyperref[subsec:var_sec_state_areaTriangleGlobal]{areaTriangleGlobal} & sum of the areaTriangle variable over the full domain, used to normalize global statistics \\
	\hline
	\hyperref[subsec:var_sec_state_volumeCellGlobal]{volumeCellGlobal} & sum of the volumeCell variable over the full domain, used to normalize global statistics \\
	\hline
	\hyperref[subsec:var_sec_state_volumeEdgeGlobal]{volumeEdgeGlobal} & sum of the volumeEdge variable over the full domain, used to normalize global statistics \\
	\hline
	\hyperref[subsec:var_sec_state_CFLNumberGlobal]{CFLNumberGlobal} & maximum CFL number over the full domain \\
	\hline
	\hyperref[subsec:var_sec_state_nAverage]{nAverage} & number of timesteps in time-averaged variables \\
	\hline
	\hyperref[subsec:var_sec_state_avgSsh]{avgSsh} & time-averaged sea surface height \\
	\hline
	\hyperref[subsec:var_sec_state_varSsh]{varSsh} & variance of sea surface height \\
	\hline
	\hyperref[subsec:var_sec_state_avgNormalVelocityZonal]{avgNormalVelocityZonal} & time-averaged velocity in the eastward direction \\
	\hline
	\hyperref[subsec:var_sec_state_avgNormalVelocityMeridional]{avgNormalVelocityMeridional} & time-averaged velocity in the northward direction \\
	\hline
	\hyperref[subsec:var_sec_state_varNormalVelocityZonal]{varNormalVelocityZonal} & variance of velocity in the eastward direction \\
	\hline
	\hyperref[subsec:var_sec_state_varNormalVelocityMeridional]{varNormalVelocityMeridional} & variance of velocity in the northward direction \\
	\hline
	\hyperref[subsec:var_sec_state_avgNormalVelocity]{avgNormalVelocity} & time-averaged velocity, normal to cell edge \\
	\hline
	\hyperref[subsec:var_sec_state_varNormalVelocity]{varNormalVelocity} & variance of velocity, normal to cell edge \\
	\hline
	\hyperref[subsec:var_sec_state_avgVertVelocityTop]{avgVertVelocityTop} & time-averaged vertical velocity at top of cell \\
	\hline
\end{longtable}
\end{center}
}
\section[mesh]{\hyperref[sec:var_sec_mesh]{mesh}}
\label{sec:var_tab_mesh}
The mesh data type contains a single time level. The fields inside the mesh
structure are not assumed to be time dependent. This data structure contains
fields that describe the mesh, and the connectivity of the mesh. Several of the
fields contained in this structure are shared throughout all MPAS dynamical
cores.

{\small
\begin{center}
\begin{longtable}{| p{2.0in} | p{4.0in} |}
	\hline
	{\bf Name} & {\bf Description} \\
	\hline
	\hyperref[subsec:var_sec_mesh_latCell]{latCell} & Latitude location of cell centers in radians. \\
	\hline
	\hyperref[subsec:var_sec_mesh_lonCell]{lonCell} & Longitude location of cell centers in radians. \\
	\hline
	\hyperref[subsec:var_sec_mesh_xCell]{xCell} & X Coordinate in cartesian space of cell centers. \\
	\hline
	\hyperref[subsec:var_sec_mesh_yCell]{yCell} & Y Coordinate in cartesian space of cell centers. \\
	\hline
	\hyperref[subsec:var_sec_mesh_zCell]{zCell} & Z Coordinate in cartesian space of cell centers. \\
	\hline
	\hyperref[subsec:var_sec_mesh_indexToCellID]{indexToCellID} & List of global cell IDs. \\
	\hline
	\hyperref[subsec:var_sec_mesh_latEdge]{latEdge} & Latitude location of edge midpoints in radians. \\
	\hline
	\hyperref[subsec:var_sec_mesh_lonEdge]{lonEdge} & Longitude location of edge midpoints in radians. \\
	\hline
	\hyperref[subsec:var_sec_mesh_xEdge]{xEdge} & X Coordinate in cartesian space of edge midpoints. \\
	\hline
	\hyperref[subsec:var_sec_mesh_yEdge]{yEdge} & Y Coordinate in cartesian space of edge midpoints. \\
	\hline
	\hyperref[subsec:var_sec_mesh_zEdge]{zEdge} & Z Coordinate in cartesian space of edge midpoints. \\
	\hline
	\hyperref[subsec:var_sec_mesh_indexToEdgeID]{indexToEdgeID} & List of global edge IDs. \\
	\hline
	\hyperref[subsec:var_sec_mesh_latVertex]{latVertex} & Latitude location of vertices in radians. \\
	\hline
	\hyperref[subsec:var_sec_mesh_lonVertex]{lonVertex} & Longitude location of vertices in radians. \\
	\hline
	\hyperref[subsec:var_sec_mesh_xVertex]{xVertex} & X Coordinate in cartesian space of vertices. \\
	\hline
	\hyperref[subsec:var_sec_mesh_yVertex]{yVertex} & Y Coordinate in cartesian space of vertices. \\
	\hline
	\hyperref[subsec:var_sec_mesh_zVertex]{zVertex} & Z Coordinate in cartesian space of vertices. \\
	\hline
	\hyperref[subsec:var_sec_mesh_indexToVertexID]{indexToVertexID} & List of global vertex IDs. \\
	\hline
	\hyperref[subsec:var_sec_mesh_meshDensity]{meshDensity} & Value of density function used to generate a particular mesh at cell centers. \\
	\hline
	\hyperref[subsec:var_sec_mesh_meshScalingDel2]{meshScalingDel2} & Coefficient to Laplacian mixing terms in momentum and tracer equations, so that viscosity and diffusion scale with mesh. \\
	\hline
	\hyperref[subsec:var_sec_mesh_meshScalingDel4]{meshScalingDel4} & Coefficient to biharmonic mixing terms in momentum and tracer equations, so that biharmonic viscosity and diffusion coefficients scale with mesh. \\
	\hline
	\hyperref[subsec:var_sec_mesh_meshScaling]{meshScaling} & Coefficient used for mesh scaling, such as the Leith parameter. \\
	\hline
	\hyperref[subsec:var_sec_mesh_cellsOnEdge]{cellsOnEdge} & List of cells that straddle each edge. \\
	\hline
	\hyperref[subsec:var_sec_mesh_nEdgesOnCell]{nEdgesOnCell} & Number of edges that border each cell. \\
	\hline
	\hyperref[subsec:var_sec_mesh_nEdgesOnEdge]{nEdgesOnEdge} & Number of edges that surround each of the cells that straddle each edge. These edges are used to reconstruct the tangential velocities. \\
	\hline
	\hyperref[subsec:var_sec_mesh_edgesOnCell]{edgesOnCell} & List of edges that border each cell. \\
	\hline
	\hyperref[subsec:var_sec_mesh_edgesOnEdge]{edgesOnEdge} & List of edges that border each of the cells that straddle each edge. \\
	\hline
	\hyperref[subsec:var_sec_mesh_weightsOnEdge]{weightsOnEdge} & Reconstruction weights associated with each of the edgesOnEdge. \\
	\hline
	\hyperref[subsec:var_sec_mesh_dvEdge]{dvEdge} & Length of each edge, computed as the distance between verticesOnEdge. \\
	\hline
	\hyperref[subsec:var_sec_mesh_dcEdge]{dcEdge} & Length of each edge, computed as the distance between cellsOnEdge. \\
	\hline
	\hyperref[subsec:var_sec_mesh_angleEdge]{angleEdge} & Angle the edge normal makes with local eastward direction. \\
	\hline
	\hyperref[subsec:var_sec_mesh_areaCell]{areaCell} & Area of each cell in the primary grid. \\
	\hline
	\hyperref[subsec:var_sec_mesh_areaTriangle]{areaTriangle} & Area of each cell (triangle) in the dual grid. \\
	\hline
	\hyperref[subsec:var_sec_mesh_edgeNormalVectors]{edgeNormalVectors} & Normal vector defined at an edge. \\
	\hline
	\hyperref[subsec:var_sec_mesh_localVerticalUnitVectors]{localVerticalUnitVectors} & Unit surface normal vectors defined at cell centers. \\
	\hline
	\hyperref[subsec:var_sec_mesh_cellTangentPlane]{cellTangentPlane} & The two vectors that define a tangent plane at a cell center. \\
	\hline
	\hyperref[subsec:var_sec_mesh_cellsOnCell]{cellsOnCell} & List of cells that neighbor each cell. \\
	\hline
	\hyperref[subsec:var_sec_mesh_verticesOnCell]{verticesOnCell} & List of vertices that border each cell. \\
	\hline
	\hyperref[subsec:var_sec_mesh_verticesOnEdge]{verticesOnEdge} & List of vertices that straddle each edge. \\
	\hline
	\hyperref[subsec:var_sec_mesh_edgesOnVertex]{edgesOnVertex} & List of edges that share a vertex as an endpoint. \\
	\hline
	\hyperref[subsec:var_sec_mesh_cellsOnVertex]{cellsOnVertex} & List of cells that share a vertex. \\
	\hline
	\hyperref[subsec:var_sec_mesh_kiteAreasOnVertex]{kiteAreasOnVertex} & Area of the portions of each dual cell that are part of each cellsOnVertex. \\
	\hline
	\hyperref[subsec:var_sec_mesh_fEdge]{fEdge} & Coriolis parameter at edges. \\
	\hline
	\hyperref[subsec:var_sec_mesh_fVertex]{fVertex} & Coriolis parameter at vertices. \\
	\hline
	\hyperref[subsec:var_sec_mesh_bottomDepth]{bottomDepth} & Depth of the bottom of the ocean. Given as a positive distance from sea level. \\
	\hline
	\hyperref[subsec:var_sec_mesh_deriv_two]{deriv\_two} & Value of the second derivative of the polynomial used for reconstruction of cell center quantities at edges. \\
	\hline
	\hyperref[subsec:var_sec_mesh_adv_coefs]{adv\_coefs} & Weighting coefficients used for reconstruction of cell center quantities at edges. Used in advection routines. \\
	\hline
	\hyperref[subsec:var_sec_mesh_adv_coefs_2nd]{adv\_coefs\_2nd} & Weighting coefficients used for reconstruction of cell center quantities at edges. Used in advection routines. \\
	\hline
	\hyperref[subsec:var_sec_mesh_adv_coefs_3rd]{adv\_coefs\_3rd} & Wegihting coefficients used for reconstruction of cell center quantities at edges. Used in advection routines. \\
	\hline
	\hyperref[subsec:var_sec_mesh_advCellsForEdge]{advCellsForEdge} & List of cells used to reconstruct a cell quantity at an edge. Used in advection routines. \\
	\hline
	\hyperref[subsec:var_sec_mesh_nAdvCellsForEdge]{nAdvCellsForEdge} & Number of cells used in reconstruction of cell center quantities at an edge. Used in advection routines. \\
	\hline
	\hyperref[subsec:var_sec_mesh_highOrderAdvectionMask]{highOrderAdvectionMask} & Mask for high order advection. Values are 1 if high order is used, and 0 if not. \\
	\hline
	\hyperref[subsec:var_sec_mesh_lowOrderAdvectionMask]{lowOrderAdvectionMask} & Mask for low order advection. Values are 1 if low order is used, and 0 if not. \\
	\hline
	\hyperref[subsec:var_sec_mesh_defc_a]{defc\_a} & Variable used with advection setup to compute advection coefficients. Deformation weight coefficients. \\
	\hline
	\hyperref[subsec:var_sec_mesh_defc_b]{defc\_b} & Variable used with advection setup to compute advection coefficients. Deformation weight coefficients. \\
	\hline
	\hyperref[subsec:var_sec_mesh_kdiff]{kdiff} & {\color{red} TO BE REMOVED} \\
	\hline
	\hyperref[subsec:var_sec_mesh_coeffs_reconstruct]{coeffs\_reconstruct} & Coefficients to reconstruct velocity vectors at cells centers. \\
	\hline
	\hyperref[subsec:var_sec_mesh_maxLevelCell]{maxLevelCell} & Index to the last active ocean cell in each column. \\
	\hline
	\hyperref[subsec:var_sec_mesh_maxLevelEdgeTop]{maxLevelEdgeTop} & Index to the last edge in a column with active ocean cells on both sides of it. \\
	\hline
	\hyperref[subsec:var_sec_mesh_maxLevelEdgeBot]{maxLevelEdgeBot} & Index to the last edge in a column with at least one active ocean cell on either side of it. \\
	\hline
	\hyperref[subsec:var_sec_mesh_maxLevelVertexTop]{maxLevelVertexTop} & Index to the last vertex in a column with all active cells around it. \\
	\hline
	\hyperref[subsec:var_sec_mesh_maxLevelVertexBot]{maxLevelVertexBot} & Index to the last vertex in a column with at least one active ocean cell around it. \\
	\hline
	\hyperref[subsec:var_sec_mesh_refBottomDepth]{refBottomDepth} & Reference depth of ocean for each vertical level. Used in 'z-level' type runs. \\
	\hline
	\hyperref[subsec:var_sec_mesh_refBottomDepthTopOfCell]{refBottomDepthTopOfCell} & Reference depth of ocean for each vertical interface. Used in 'z-level' type runs. \\
	\hline
	\hyperref[subsec:var_sec_mesh_hZLevel]{hZLevel} & {\color{red} TO BE REMOVED} \\
	\hline
	\hyperref[subsec:var_sec_mesh_vertCoordMovementWeights]{vertCoordMovementWeights} & Weights used for distribution of sea surface heigh purturbations through multiple vertical levels. \\
	\hline
	\hyperref[subsec:var_sec_mesh_boundaryEdge]{boundaryEdge} & Mask for determining boundary edges. A boundary edge has only one active ocean cell neighboring it. \\
	\hline
	\hyperref[subsec:var_sec_mesh_boundaryVertex]{boundaryVertex} & Mask for determining boundary vertices. A boundary vertex has at least one inactive cell neighboring it. \\
	\hline
	\hyperref[subsec:var_sec_mesh_boundaryCell]{boundaryCell} & Mask for determining boundary cells. A boundary cell has at least one inactive cell neighboring it. \\
	\hline
	\hyperref[subsec:var_sec_mesh_edgeMask]{edgeMask} & Mask on edges that determines if computations should be done on edge. \\
	\hline
	\hyperref[subsec:var_sec_mesh_vertexMask]{vertexMask} & Mask on vertices that determines if computations should be done on vertice. \\
	\hline
	\hyperref[subsec:var_sec_mesh_cellMask]{cellMask} & Mask on cells that determines if computations should be done on cell. \\
	\hline
	\hyperref[subsec:var_sec_mesh_normalVelocityForcing]{normalVelocityForcing} & Velocity forcing field. Defines a forcing at an edge. \\
	\hline
	\hyperref[subsec:var_sec_mesh_temperatureRestore]{temperatureRestore} & Temperature restoring field, for restoring temperature at the surface. \\
	\hline
	\hyperref[subsec:var_sec_mesh_salinityRestore]{salinityRestore} & Salinity restoring field, for restoring salinity at the surface. \\
	\hline
	\hyperref[subsec:var_sec_mesh_windStressMonthly]{windStressMonthly} & Monthly wind stress field, defined at the surface for use in monthly forcing. \\
	\hline
	\hyperref[subsec:var_sec_mesh_temperatureRestoreMonthly]{temperatureRestoreMonthly} & Monthly temperature restorying field, defined at the surface for use in monthly forcing. \\
	\hline
	\hyperref[subsec:var_sec_mesh_salinityRestoreMonthly]{salinityRestoreMonthly} & Monthly salinity resotring field, defined at the surface, for use in monthly forcing. \\
	\hline
	\hyperref[subsec:var_sec_mesh_edgeSignOnCell]{edgeSignOnCell} & Sign of edge contributions to a cell for each edge on cell. Used for bit-reproducible loops. Represents directionality of vector connecting cells. \\
	\hline
	\hyperref[subsec:var_sec_mesh_edgeSignOnVertex]{edgeSignOnVertex} & Sign of edge contributions to a vertex for each edge on vertex. Used for bit-reproducible loops. Represents directionality of vector connecting vertices. \\
	\hline
	\hyperref[subsec:var_sec_mesh_kiteIndexOnCell]{kiteIndexOnCell} & Index of kite in dual grid, based on verticesOnCell. \\
	\hline
	\hyperref[subsec:var_sec_mesh_seaSurfacePressure]{seaSurfacePressure} & Pressure defined at the sea surface. \\
	\hline
\end{longtable}
\end{center}
}
\section[tend]{\hyperref[sec:var_sec_tend]{tend}}
\label{sec:var_tab_tend}
The tend data structure represents the tendencies used to time step the
prognostic variables within the state structure. 

{\small
\begin{center}
\begin{longtable}{| p{2.0in} | p{4.0in} |}
	\hline
	{\bf Name} & {\bf Description} \\
	\hline
	\hyperref[subsec:var_sec_tend_tend_temperature]{tend\_temperature} & time tendency of potential temperature \\
	\hline
	\hyperref[subsec:var_sec_tend_tend_salinity]{tend\_salinity} & time tendency of salinity measured as change in practical salinity units per second \\
	\hline
	\hyperref[subsec:var_sec_tend_tend_tracer1]{tend\_tracer1} & time tendency of an arbitary tracer \\
	\hline
	\hyperref[subsec:var_sec_tend_tend_normalVelocity]{tend\_normalVelocity} & time tendency of normal component of velocity \\
	\hline
	\hyperref[subsec:var_sec_tend_tend_layerThickness]{tend\_layerThickness} & time tendency of layer thickness \\
	\hline
	\hyperref[subsec:var_sec_tend_tend_ssh]{tend\_ssh} & time tendency of sea-surface height \\
	\hline
\end{longtable}
\end{center}
}
\section[diagnostics]{\hyperref[sec:var_sec_diagnostics]{diagnostics}}
\label{sec:var_tab_diagnostics}
Namelist parameters for the \verb+diagnostics+ namelist group.

{\small
\begin{center}
\begin{longtable}{| p{2.0in} | p{4.0in} |}
	\hline
	{\bf Name} & {\bf Description} \\
	\hline
	\hyperref[subsec:var_sec_diagnostics_RiTopOfCell]{RiTopOfCell} & gradient Richardson number defined at the center (horizontally) and top (vertically) \\
	\hline
	\hyperref[subsec:var_sec_diagnostics_RiTopOfEdge]{RiTopOfEdge} & gradient Richardson number defined at the edge (horizontally) and top (vertically) \\
	\hline
	\hyperref[subsec:var_sec_diagnostics_vertViscTopOfEdge]{vertViscTopOfEdge} & vertical viscosity defined at the edge (horizontally) and top (vertically) \\
	\hline
	\hyperref[subsec:var_sec_diagnostics_vertDiffTopOfCell]{vertDiffTopOfCell} & vertical diffusion defined at the edge (horizontally) and top (vertically) \\
	\hline
\end{longtable}
\end{center}
}
