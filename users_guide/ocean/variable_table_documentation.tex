\chapter[Variable definitions]{\hyperref[chap:variable_sections]{Variable definitions}}
\label{chap:variable_tables}
Embedded links point to more detailed variable information in the appendix.
\section[state]{\hyperref[sec:var_sec_state]{state}}
\label{sec:var_tab_state}
\input{ocean/section_descriptions/state_struct.tex}
\vspace{0.5in}
{\small
\begin{center}
\begin{longtable}{| p{2.0in} | p{4.0in} |}
	\hline
	{\bf Name} & {\bf Description} \endfirsthead
	\hline 
	{\bf Name} & {\bf Description} (Continued) \endhead
	\hline
	\hyperref[subsec:var_sec_state_temperature]{temperature} & potential temperature \\
	\hline
	\hyperref[subsec:var_sec_state_salinity]{salinity} & salinity \\
	\hline
	\hyperref[subsec:var_sec_state_tracer1]{tracer1} & tracer \\
	\hline
	\hyperref[subsec:var_sec_state_normalVelocity]{normalVelocity} & horizonal velocity, normal component to an edge \\
	\hline
	\hyperref[subsec:var_sec_state_layerThickness]{layerThickness} & layer thickness \\
	\hline
	\hyperref[subsec:var_sec_state_ssh]{ssh} & sea surface height \\
	\hline
	\hyperref[subsec:var_sec_state_highFreqThickness]{highFreqThickness} & high frequency-filtered layer thickness \\
	\hline
	\hyperref[subsec:var_sec_state_lowFreqDivergence]{lowFreqDivergence} & low frequency-filtered divergence \\
	\hline
	\hyperref[subsec:var_sec_state_normalBarotropicVelocity]{normalBarotropicVelocity} & barotropic velocity, used in split-explicit time-stepping \\
	\hline
	\hyperref[subsec:var_sec_state_normalBarotropicVelocitySubcycle]{normalBarotropicVelocitySubcy-}\hyperref[subsec:var_sec_state_normalBarotropicVelocitySubcycle]{cle  }& barotropic velocity, used in subcycling in stage 2 of split-explicit time-stepping \\
	\hline
	\hyperref[subsec:var_sec_state_sshSubcycle]{sshSubcycle} & sea surface height, used in subcycling in stage 2 of split-explicit time-stepping \\
	\hline
	\hyperref[subsec:var_sec_state_normalBaroclinicVelocity]{normalBaroclinicVelocity} & baroclinic velocity, used in split-explicit time-stepping \\
	\hline
\end{longtable}
\end{center}
}
\section[mesh]{\hyperref[sec:var_sec_mesh]{mesh}}
\label{sec:var_tab_mesh}
The mesh data type contains a single time level. The fields inside the mesh
structure are not assumed to be time dependent. This data structure contains
fields that describe the mesh, and the connectivity of the mesh. Several of the
fields contained in this structure are shared throughout all MPAS dynamical
cores.

\vspace{0.5in}
{\small
\begin{center}
\begin{longtable}{| p{2.0in} | p{4.0in} |}
	\hline
	{\bf Name} & {\bf Description} \endfirsthead
	\hline 
	{\bf Name} & {\bf Description} (Continued) \endhead
	\hline
	\hyperref[subsec:var_sec_mesh_latCell]{latCell} & Latitude location of cell centers in radians. \\
	\hline
	\hyperref[subsec:var_sec_mesh_lonCell]{lonCell} & Longitude location of cell centers in radians. \\
	\hline
	\hyperref[subsec:var_sec_mesh_xCell]{xCell} & X Coordinate in cartesian space of cell centers. \\
	\hline
	\hyperref[subsec:var_sec_mesh_yCell]{yCell} & Y Coordinate in cartesian space of cell centers. \\
	\hline
	\hyperref[subsec:var_sec_mesh_zCell]{zCell} & Z Coordinate in cartesian space of cell centers. \\
	\hline
	\hyperref[subsec:var_sec_mesh_indexToCellID]{indexToCellID} & List of global cell IDs. \\
	\hline
	\hyperref[subsec:var_sec_mesh_latEdge]{latEdge} & Latitude location of edge midpoints in radians. \\
	\hline
	\hyperref[subsec:var_sec_mesh_lonEdge]{lonEdge} & Longitude location of edge midpoints in radians. \\
	\hline
	\hyperref[subsec:var_sec_mesh_xEdge]{xEdge} & X Coordinate in cartesian space of edge midpoints. \\
	\hline
	\hyperref[subsec:var_sec_mesh_yEdge]{yEdge} & Y Coordinate in cartesian space of edge midpoints. \\
	\hline
	\hyperref[subsec:var_sec_mesh_zEdge]{zEdge} & Z Coordinate in cartesian space of edge midpoints. \\
	\hline
	\hyperref[subsec:var_sec_mesh_indexToEdgeID]{indexToEdgeID} & List of global edge IDs. \\
	\hline
	\hyperref[subsec:var_sec_mesh_latVertex]{latVertex} & Latitude location of vertices in radians. \\
	\hline
	\hyperref[subsec:var_sec_mesh_lonVertex]{lonVertex} & Longitude location of vertices in radians. \\
	\hline
	\hyperref[subsec:var_sec_mesh_xVertex]{xVertex} & X Coordinate in cartesian space of vertices. \\
	\hline
	\hyperref[subsec:var_sec_mesh_yVertex]{yVertex} & Y Coordinate in cartesian space of vertices. \\
	\hline
	\hyperref[subsec:var_sec_mesh_zVertex]{zVertex} & Z Coordinate in cartesian space of vertices. \\
	\hline
	\hyperref[subsec:var_sec_mesh_indexToVertexID]{indexToVertexID} & List of global vertex IDs. \\
	\hline
	\hyperref[subsec:var_sec_mesh_meshDensity]{meshDensity} & Value of density function used to generate a particular mesh at cell centers. \\
	\hline
	\hyperref[subsec:var_sec_mesh_meshScalingDel2]{meshScalingDel2} & Coefficient to Laplacian mixing terms in momentum and tracer equations, so that viscosity and diffusion scale with mesh. \\
	\hline
	\hyperref[subsec:var_sec_mesh_meshScalingDel4]{meshScalingDel4} & Coefficient to biharmonic mixing terms in momentum and tracer equations, so that biharmonic viscosity and diffusion coefficients scale with mesh. \\
	\hline
	\hyperref[subsec:var_sec_mesh_meshScaling]{meshScaling} & Coefficient used for mesh scaling, such as the Leith parameter. \\
	\hline
	\hyperref[subsec:var_sec_mesh_cellsOnEdge]{cellsOnEdge} & List of cells that straddle each edge. \\
	\hline
	\hyperref[subsec:var_sec_mesh_nEdgesOnCell]{nEdgesOnCell} & Number of edges that border each cell. \\
	\hline
	\hyperref[subsec:var_sec_mesh_nEdgesOnEdge]{nEdgesOnEdge} & Number of edges that surround each of the cells that straddle each edge. These edges are used to reconstruct the tangential velocities. \\
	\hline
	\hyperref[subsec:var_sec_mesh_edgesOnCell]{edgesOnCell} & List of edges that border each cell. \\
	\hline
	\hyperref[subsec:var_sec_mesh_edgesOnEdge]{edgesOnEdge} & List of edges that border each of the cells that straddle each edge. \\
	\hline
	\hyperref[subsec:var_sec_mesh_weightsOnEdge]{weightsOnEdge} & Reconstruction weights associated with each of the edgesOnEdge. \\
	\hline
	\hyperref[subsec:var_sec_mesh_dvEdge]{dvEdge} & Length of each edge, computed as the distance between verticesOnEdge. \\
	\hline
	\hyperref[subsec:var_sec_mesh_dcEdge]{dcEdge} & Length of each edge, computed as the distance between cellsOnEdge. \\
	\hline
	\hyperref[subsec:var_sec_mesh_angleEdge]{angleEdge} & Angle the edge normal makes with local eastward direction. \\
	\hline
	\hyperref[subsec:var_sec_mesh_areaCell]{areaCell} & Area of each cell in the primary grid. \\
	\hline
	\hyperref[subsec:var_sec_mesh_areaTriangle]{areaTriangle} & Area of each cell (triangle) in the dual grid. \\
	\hline
	\hyperref[subsec:var_sec_mesh_edgeNormalVectors]{edgeNormalVectors} & Normal unit vector defined at an edge. \\
	\hline
	\hyperref[subsec:var_sec_mesh_edgeTangentVectors]{edgeTangentVectors} & Tangent unit vector defined at an edge. \\
	\hline
	\hyperref[subsec:var_sec_mesh_localVerticalUnitVectors]{localVerticalUnitVectors} & Unit surface normal vectors defined at cell centers. \\
	\hline
	\hyperref[subsec:var_sec_mesh_cellTangentPlane]{cellTangentPlane} & The two vectors that define a tangent plane at a cell center. \\
	\hline
	\hyperref[subsec:var_sec_mesh_cellsOnCell]{cellsOnCell} & List of cells that neighbor each cell. \\
	\hline
	\hyperref[subsec:var_sec_mesh_verticesOnCell]{verticesOnCell} & List of vertices that border each cell. \\
	\hline
	\hyperref[subsec:var_sec_mesh_verticesOnEdge]{verticesOnEdge} & List of vertices that straddle each edge. \\
	\hline
	\hyperref[subsec:var_sec_mesh_edgesOnVertex]{edgesOnVertex} & List of edges that share a vertex as an endpoint. \\
	\hline
	\hyperref[subsec:var_sec_mesh_cellsOnVertex]{cellsOnVertex} & List of cells that share a vertex. \\
	\hline
	\hyperref[subsec:var_sec_mesh_kiteAreasOnVertex]{kiteAreasOnVertex} & Area of the portions of each dual cell that are part of each cellsOnVertex. \\
	\hline
	\hyperref[subsec:var_sec_mesh_fEdge]{fEdge} & Coriolis parameter at edges. \\
	\hline
	\hyperref[subsec:var_sec_mesh_fVertex]{fVertex} & Coriolis parameter at vertices. \\
	\hline
	\hyperref[subsec:var_sec_mesh_fCell]{fCell} & Coriolis parameter at cell centers. \\
	\hline
	\hyperref[subsec:var_sec_mesh_bottomDepth]{bottomDepth} & Depth of the bottom of the ocean. Given as a positive distance from sea level. \\
	\hline
	\hyperref[subsec:var_sec_mesh_derivTwo]{derivTwo} & Value of the second derivative of the polynomial used for reconstruction of cell center quantities at edges. \\
	\hline
	\hyperref[subsec:var_sec_mesh_advCoefs]{advCoefs} & Weighting coefficients used for reconstruction of cell center quantities at edges. Used in advection routines. \\
	\hline
	\hyperref[subsec:var_sec_mesh_advCoefs3rd]{advCoefs3rd} & Wegihting coefficients used for reconstruction of cell center quantities at edges. Used in advection routines. \\
	\hline
	\hyperref[subsec:var_sec_mesh_advCellsForEdge]{advCellsForEdge} & List of cells used to reconstruct a cell quantity at an edge. Used in advection routines. \\
	\hline
	\hyperref[subsec:var_sec_mesh_nAdvCellsForEdge]{nAdvCellsForEdge} & Number of cells used in reconstruction of cell center quantities at an edge. Used in advection routines. \\
	\hline
	\hyperref[subsec:var_sec_mesh_highOrderAdvectionMask]{highOrderAdvectionMask} & Mask for high order advection. Values are 1 if high order is used, and 0 if not. \\
	\hline
	\hyperref[subsec:var_sec_mesh_coeffs_reconstruct]{coeffs\_reconstruct} & Coefficients to reconstruct velocity vectors at cells centers. \\
	\hline
	\hyperref[subsec:var_sec_mesh_maxLevelCell]{maxLevelCell} & Index to the last active ocean cell in each column. \\
	\hline
	\hyperref[subsec:var_sec_mesh_maxLevelEdgeTop]{maxLevelEdgeTop} & Index to the last edge in a column with active ocean cells on both sides of it. \\
	\hline
	\hyperref[subsec:var_sec_mesh_maxLevelEdgeBot]{maxLevelEdgeBot} & Index to the last edge in a column with at least one active ocean cell on either side of it. \\
	\hline
	\hyperref[subsec:var_sec_mesh_maxLevelVertexTop]{maxLevelVertexTop} & Index to the last vertex in a column with all active cells around it. \\
	\hline
	\hyperref[subsec:var_sec_mesh_maxLevelVertexBot]{maxLevelVertexBot} & Index to the last vertex in a column with at least one active ocean cell around it. \\
	\hline
	\hyperref[subsec:var_sec_mesh_refBottomDepth]{refBottomDepth} & Reference depth of ocean for each vertical level. Used in 'z-level' type runs. \\
	\hline
	\hyperref[subsec:var_sec_mesh_refBottomDepthTopOfCell]{refBottomDepthTopOfCell} & Reference depth of ocean for each vertical interface. Used in 'z-level' type runs. \\
	\hline
	\hyperref[subsec:var_sec_mesh_vertCoordMovementWeights]{vertCoordMovementWeights} & Weights used for distribution of sea surface heigh purturbations through multiple vertical levels. \\
	\hline
	\hyperref[subsec:var_sec_mesh_boundaryEdge]{boundaryEdge} & Mask for determining boundary edges. A boundary edge has only one active ocean cell neighboring it. \\
	\hline
	\hyperref[subsec:var_sec_mesh_boundaryVertex]{boundaryVertex} & Mask for determining boundary vertices. A boundary vertex has at least one inactive cell neighboring it. \\
	\hline
	\hyperref[subsec:var_sec_mesh_boundaryCell]{boundaryCell} & Mask for determining boundary cells. A boundary cell has at least one inactive cell neighboring it. \\
	\hline
	\hyperref[subsec:var_sec_mesh_edgeMask]{edgeMask} & Mask on edges that determines if computations should be done on edge. \\
	\hline
	\hyperref[subsec:var_sec_mesh_vertexMask]{vertexMask} & Mask on vertices that determines if computations should be done on vertice. \\
	\hline
	\hyperref[subsec:var_sec_mesh_cellMask]{cellMask} & Mask on cells that determines if computations should be done on cell. \\
	\hline
	\hyperref[subsec:var_sec_mesh_temperatureRestore]{temperatureRestore} & Temperature restoring field, for restoring temperature at the surface. \\
	\hline
	\hyperref[subsec:var_sec_mesh_salinityRestore]{salinityRestore} & Salinity restoring field, for restoring salinity at the surface. \\
	\hline
	\hyperref[subsec:var_sec_mesh_edgeSignOnCell]{edgeSignOnCell} & Sign of edge contributions to a cell for each edge on cell. Used for bit-reproducible loops. Represents directionality of vector connecting cells. \\
	\hline
	\hyperref[subsec:var_sec_mesh_edgeSignOnVertex]{edgeSignOnVertex} & Sign of edge contributions to a vertex for each edge on vertex. Used for bit-reproducible loops. Represents directionality of vector connecting vertices. \\
	\hline
	\hyperref[subsec:var_sec_mesh_kiteIndexOnCell]{kiteIndexOnCell} & Index of kite in dual grid, based on verticesOnCell. \\
	\hline
\end{longtable}
\end{center}
}
\section[verticalMesh]{\hyperref[sec:var_sec_verticalMesh]{verticalMesh}}
\label{sec:var_tab_verticalMesh}
The vertical mesh data type contains a single time level. The fields inside the
vertical mesh structure are not assumed to be time dependent. This data
structure contains fields that describe the vertical mesh and are used for
various types of vertical meshes.

\vspace{0.5in}
{\small
\begin{center}
\begin{longtable}{| p{2.0in} | p{4.0in} |}
	\hline
	{\bf Name} & {\bf Description} \endfirsthead
	\hline 
	{\bf Name} & {\bf Description} (Continued) \endhead
	\hline
	\hyperref[subsec:var_sec_verticalMesh_restingThickness]{restingThickness} & Layer thickness when the ocean is at rest, i.e. without SSH or internal perturbations. \\
	\hline
	\hyperref[subsec:var_sec_verticalMesh_refZMid]{refZMid} & Reference mid z-coordinate of ocean for each vertical level.  This has a negative value. \\
	\hline
	\hyperref[subsec:var_sec_verticalMesh_refLayerThickness]{refLayerThickness} & Reference layerThickness of ocean for each vertical level. \\
	\hline
\end{longtable}
\end{center}
}
\section[tend]{\hyperref[sec:var_sec_tend]{tend}}
\label{sec:var_tab_tend}
The tend data structure represents the tendencies used to time step the
prognostic variables within the state structure. 

\vspace{0.5in}
{\small
\begin{center}
\begin{longtable}{| p{2.0in} | p{4.0in} |}
	\hline
	{\bf Name} & {\bf Description} \endfirsthead
	\hline 
	{\bf Name} & {\bf Description} (Continued) \endhead
	\hline
	\hyperref[subsec:var_sec_tend_tendTemperature]{tendTemperature} & time tendency of potential temperature \\
	\hline
	\hyperref[subsec:var_sec_tend_tendSalinity]{tendSalinity} & time tendency of salinity measured as change in practical salinity units per second \\
	\hline
	\hyperref[subsec:var_sec_tend_tendTracer1]{tendTracer1} & test tracer \\
	\hline
	\hyperref[subsec:var_sec_tend_tendNormalVelocity]{tendNormalVelocity} & time tendency of normal component of velocity \\
	\hline
	\hyperref[subsec:var_sec_tend_tendLayerThickness]{tendLayerThickness} & time tendency of layer thickness \\
	\hline
	\hyperref[subsec:var_sec_tend_tendSsh]{tendSsh} & time tendency of sea-surface height \\
	\hline
	\hyperref[subsec:var_sec_tend_tendHighFreqThickness]{tendHighFreqThickness} & time tendency of high frequency-filtered layer thickness \\
	\hline
	\hyperref[subsec:var_sec_tend_tendLowFreqDivergence]{tendLowFreqDivergence} & time tendency of low frequency-filtered divergence \\
	\hline
\end{longtable}
\end{center}
}
\section[diagnostics]{\hyperref[sec:var_sec_diagnostics]{diagnostics}}
\label{sec:var_tab_diagnostics}
The diagnostics type contains a set of diagnostics variables that are only
generally used in specific parts of MPAS-Ocean.

\vspace{0.5in}
{\small
\begin{center}
\begin{longtable}{| p{2.0in} | p{4.0in} |}
	\hline
	{\bf Name} & {\bf Description} \endfirsthead
	\hline 
	{\bf Name} & {\bf Description} (Continued) \endhead
	\hline
	\hyperref[subsec:var_sec_diagnostics_xtime]{xtime} & model time, with format 'YYYY-MM-DD\_HH:MM:SS' \\
	\hline
	\hyperref[subsec:var_sec_diagnostics_temperatureSurfaceValue]{temperatureSurfaceValue} & potential temperature extrapolated to ocean surface \\
	\hline
	\hyperref[subsec:var_sec_diagnostics_salinitySurfaceValue]{salinitySurfaceValue} & salinity extrapolated to ocean surface \\
	\hline
	\hyperref[subsec:var_sec_diagnostics_tracer1SurfaceValue]{tracer1SurfaceValue} & Tracer of 1 extrapolated to ocean surface \\
	\hline
	\hyperref[subsec:var_sec_diagnostics_zonalSurfaceVelocity]{zonalSurfaceVelocity} & Zonal surface velocity reconstructed at cell centers \\
	\hline
	\hyperref[subsec:var_sec_diagnostics_meridionalSurfaceVelocity]{meridionalSurfaceVelocity} & Meridional surface velocity reconstructed at cell centers \\
	\hline
	\hyperref[subsec:var_sec_diagnostics_zonalSSHGradient]{zonalSSHGradient} & Zonal gradient of SSH reconstructed at cell centers \\
	\hline
	\hyperref[subsec:var_sec_diagnostics_meridionalSSHGradient]{meridionalSSHGradient} & Meridional gradient of SSH reconstructed at cell centers \\
	\hline
	\hyperref[subsec:var_sec_diagnostics_zMid]{zMid} & z-coordinate of the mid-depth of the layer \\
	\hline
	\hyperref[subsec:var_sec_diagnostics_zTop]{zTop} & z-coordinate of the top of the layer \\
	\hline
	\hyperref[subsec:var_sec_diagnostics_density]{density} & density \\
	\hline
	\hyperref[subsec:var_sec_diagnostics_displacedDensity]{displacedDensity} & Density displaced adiabatically to the mid-depth one layer deeper.  That is, layer k has been displaced to the depth of layer k+1. \\
	\hline
	\hyperref[subsec:var_sec_diagnostics_potentialDensity]{potentialDensity} & potential density: density displaced adiabatically to the mid-depth of top layer \\
	\hline
	\hyperref[subsec:var_sec_diagnostics_BruntVaisalaFreqTop]{BruntVaisalaFreqTop} & Brunt Vaisala frequency defined at the center (horizontally) and top (vertically) of cell \\
	\hline
	\hyperref[subsec:var_sec_diagnostics_montgomeryPotential]{montgomeryPotential} & Montgomery potential, may be used as the pressure for isopycnal coordinates. \\
	\hline
	\hyperref[subsec:var_sec_diagnostics_pressure]{pressure} & pressure used in the momentum equation \\
	\hline
	\hyperref[subsec:var_sec_diagnostics_uTransport]{uTransport} & horizontal velocity used to transport mass and tracers \\
	\hline
	\hyperref[subsec:var_sec_diagnostics_vertTransportVelocityTop]{vertTransportVelocityTop} & vertical transport through the layer interface at the top of the cell \\
	\hline
	\hyperref[subsec:var_sec_diagnostics_vertVelocityTop]{vertVelocityTop} & vertical velocity defined at center (horizonally) and top (vertically) of cell \\
	\hline
	\hyperref[subsec:var_sec_diagnostics_tangentialVelocity]{tangentialVelocity} & horizontal velocity, tangential to an edge \\
	\hline
	\hyperref[subsec:var_sec_diagnostics_layerThicknessEdge]{layerThicknessEdge} & layer thickness averaged from cell center to edges \\
	\hline
	\hyperref[subsec:var_sec_diagnostics_layerThicknessVertex]{layerThicknessVertex} & layer thickness averaged from cell center to vertices \\
	\hline
	\hyperref[subsec:var_sec_diagnostics_kineticEnergyCell]{kineticEnergyCell} & kinetic energy of horizonal velocity on cells \\
	\hline
	\hyperref[subsec:var_sec_diagnostics_hEddyFlux]{hEddyFlux} & Eddy flux in Gent-McWilliams eddy parameterization \\
	\hline
	\hyperref[subsec:var_sec_diagnostics_hKappa]{hKappa} & kappa parameter for Gent-McWilliams eddy parameterization \\
	\hline
	\hyperref[subsec:var_sec_diagnostics_hKappaQ]{hKappaQ} & kappaQ parameter for Gent-McWilliams eddy parameterization \\
	\hline
	\hyperref[subsec:var_sec_diagnostics_viscosity]{viscosity} & horizontal viscosity \\
	\hline
	\hyperref[subsec:var_sec_diagnostics_divergence]{divergence} & divergence of horizonal velocity \\
	\hline
	\hyperref[subsec:var_sec_diagnostics_circulation]{circulation} & area-integrated vorticity \\
	\hline
	\hyperref[subsec:var_sec_diagnostics_relativeVorticity]{relativeVorticity} & curl of horizontal velocity, defined at vertices \\
	\hline
	\hyperref[subsec:var_sec_diagnostics_relativeVorticityCell]{relativeVorticityCell} & curl of horizontal velocity, averaged from vertices to cell centers \\
	\hline
	\hyperref[subsec:var_sec_diagnostics_normalizedRelativeVorticityEdge]{normalizedRelativeVorticityEdge} & curl of horizontal velocity divided by layer thickness, averaged from vertices to edges \\
	\hline
	\hyperref[subsec:var_sec_diagnostics_normalizedPlanetaryVorticityEdge]{normalizedPlanetaryVorticityEd-}\hyperref[subsec:var_sec_diagnostics_normalizedPlanetaryVorticityEdge]{ge  }& earth's rotational rate (Coriolis parameter, f) divided by layer thickness, averaged from vertices to edges \\
	\hline
	\hyperref[subsec:var_sec_diagnostics_normalizedRelativeVorticityCell]{normalizedRelativeVorticityCell} & curl of horizontal velocity divided by layer thickness, averaged from vertices to cell centers \\
	\hline
	\hyperref[subsec:var_sec_diagnostics_barotropicForcing]{barotropicForcing} & Barotropic tendency computed from the baroclinic equations in stage 1 of the split-explicit algorithm. \\
	\hline
	\hyperref[subsec:var_sec_diagnostics_barotropicThicknessFlux]{barotropicThicknessFlux} & Barotropic thickness flux at each edge, used to advance sea surface height in each subcycle of stage 2 of the split-explicit algorithm. \\
	\hline
	\hyperref[subsec:var_sec_diagnostics_normalVelocityX]{normalVelocityX} & component of horizontal velocity in the x-direction (cartesian) \\
	\hline
	\hyperref[subsec:var_sec_diagnostics_normalVelocityY]{normalVelocityY} & component of horizontal velocity in the y-direction (cartesian) \\
	\hline
	\hyperref[subsec:var_sec_diagnostics_normalVelocityZ]{normalVelocityZ} & component of horizontal velocity in the z-direction (cartesian) \\
	\hline
	\hyperref[subsec:var_sec_diagnostics_normalVelocityZonal]{normalVelocityZonal} & component of horizontal velocity in the eastward direction \\
	\hline
	\hyperref[subsec:var_sec_diagnostics_normalVelocityMeridional]{normalVelocityMeridional} & component of horizontal velocity in the northward \\
	\hline
	\hyperref[subsec:var_sec_diagnostics_gradSSH]{gradSSH} & Gradient of sea surface height at edges. \\
	\hline
	\hyperref[subsec:var_sec_diagnostics_gradSSHX]{gradSSHX} & X Component of the gradient of sea surface height at cell centers. \\
	\hline
	\hyperref[subsec:var_sec_diagnostics_gradSSHY]{gradSSHY} & Y Component of the gradient of sea surface height at cell centers. \\
	\hline
	\hyperref[subsec:var_sec_diagnostics_gradSSHZ]{gradSSHZ} & Z Component of the gradient of sea surface height at cell centers. \\
	\hline
	\hyperref[subsec:var_sec_diagnostics_gradSSHZonal]{gradSSHZonal} & Zonal Component of the gradient of sea surface height at cell centers. \\
	\hline
	\hyperref[subsec:var_sec_diagnostics_gradSSHMeridional]{gradSSHMeridional} & Meridional Component of the gradient of sea surface height at cell centers. \\
	\hline
	\hyperref[subsec:var_sec_diagnostics_uBolusGM]{uBolusGM} & Bolus velocity in Gent-McWilliams eddy parameterization \\
	\hline
	\hyperref[subsec:var_sec_diagnostics_uBolusGMX]{uBolusGMX} & Bolus velocity in Gent-McWilliams eddy parameterization, x-direction \\
	\hline
	\hyperref[subsec:var_sec_diagnostics_uBolusGMY]{uBolusGMY} & Bolus velocity in Gent-McWilliams eddy parameterization, y-direction \\
	\hline
	\hyperref[subsec:var_sec_diagnostics_uBolusGMZ]{uBolusGMZ} & Bolus velocity in Gent-McWilliams eddy parameterization, z-direction \\
	\hline
	\hyperref[subsec:var_sec_diagnostics_uBolusGMZonal]{uBolusGMZonal} & Bolus velocity in Gent-McWilliams eddy parameterization, zonal-direction \\
	\hline
	\hyperref[subsec:var_sec_diagnostics_uBolusGMMeridional]{uBolusGMMeridional} & Bolus velocity in Gent-McWilliams eddy parameterization, meridional-direction \\
	\hline
	\hyperref[subsec:var_sec_diagnostics_RiTopOfCell]{RiTopOfCell} & gradient Richardson number defined at the center (horizontally) and top (vertically) \\
	\hline
	\hyperref[subsec:var_sec_diagnostics_RiTopOfEdge]{RiTopOfEdge} & gradient Richardson number defined at the edge (horizontally) and top (vertically) \\
	\hline
	\hyperref[subsec:var_sec_diagnostics_vertViscTopOfEdge]{vertViscTopOfEdge} & vertical viscosity defined at the edge (horizontally) and top (vertically) \\
	\hline
	\hyperref[subsec:var_sec_diagnostics_vertViscTopOfCell]{vertViscTopOfCell} & vertical viscosity defined at the cell center (horizontally) and top (vertically) \\
	\hline
	\hyperref[subsec:var_sec_diagnostics_vertDiffTopOfCell]{vertDiffTopOfCell} & vertical diffusion defined at the cell center (horizontally) and top (vertically) \\
	\hline
	\hyperref[subsec:var_sec_diagnostics_bulkRichardsonNumber]{bulkRichardsonNumber} & bulk Richardson number \\
	\hline
	\hyperref[subsec:var_sec_diagnostics_boundaryLayerDepth]{boundaryLayerDepth} & diagnosed depth of the ocean surface boundary layer \\
	\hline
	\hyperref[subsec:var_sec_diagnostics_indexBoundaryLayerDepth]{indexBoundaryLayerDepth} & int(indexBoundaryLayerDepth) is vertical layer within which boundaryLayerDepth resides. mod(indexBoundaryLayerDepth) is fractional position within that layer. \\
	\hline
	\hyperref[subsec:var_sec_diagnostics_surfaceFrictionVelocity]{surfaceFrictionVelocity} & diagnosed surface friction velocity defined as square root of (mag(wind stress) / reference density) \\
	\hline
	\hyperref[subsec:var_sec_diagnostics_windStressZonalDiag]{windStressZonalDiag} & reconstructed surface wind stress in the eastward direction. Used for diagnostics. \\
	\hline
	\hyperref[subsec:var_sec_diagnostics_windStressMeridionalDiag]{windStressMeridionalDiag} & reconstructed surface wind stress in the northward direction. User for diagnostics. \\
	\hline
	\hyperref[subsec:var_sec_diagnostics_penetrativeTemperatureFluxOBL]{penetrativeTemperatureFluxO-}\hyperref[subsec:var_sec_diagnostics_penetrativeTemperatureFluxOBL]{BL  }& Penetrative temperature flux at the bottom of boundary layer due to solar radiation. Positive is into the ocean. \\
	\hline
	\hyperref[subsec:var_sec_diagnostics_buoyancyForcingOBL]{buoyancyForcingOBL} & diagnosed buoyancy forcing due to heat, salt and freshwater fluxes averaged over ocean boundary layer depth \\
	\hline
	\hyperref[subsec:var_sec_diagnostics_areaCellGlobal]{areaCellGlobal} & sum of the areaCell variable over the full domain, used to normalize global statistics \\
	\hline
	\hyperref[subsec:var_sec_diagnostics_areaEdgeGlobal]{areaEdgeGlobal} & sum of the areaEdge variable over the full domain, used to normalize global statistics \\
	\hline
	\hyperref[subsec:var_sec_diagnostics_areaTriangleGlobal]{areaTriangleGlobal} & sum of the areaTriangle variable over the full domain, used to normalize global statistics \\
	\hline
	\hyperref[subsec:var_sec_diagnostics_volumeCellGlobal]{volumeCellGlobal} & sum of the volumeCell variable over the full domain, used to normalize global statistics \\
	\hline
	\hyperref[subsec:var_sec_diagnostics_volumeEdgeGlobal]{volumeEdgeGlobal} & sum of the volumeEdge variable over the full domain, used to normalize global statistics \\
	\hline
	\hyperref[subsec:var_sec_diagnostics_CFLNumberGlobal]{CFLNumberGlobal} & maximum CFL number over the full domain \\
	\hline
\end{longtable}
\end{center}
}
\section[average]{\hyperref[sec:var_sec_average]{average}}
\label{sec:var_tab_average}
The average data type contains a single time level. The fields inside the
average structure are not assumed to be time dependent. This data structure
contains fields that are time averages of other fields.

\vspace{0.5in}
{\small
\begin{center}
\begin{longtable}{| p{2.0in} | p{4.0in} |}
	\hline
	{\bf Name} & {\bf Description} \endfirsthead
	\hline 
	{\bf Name} & {\bf Description} (Continued) \endhead
	\hline
	\hyperref[subsec:var_sec_average_nAverage]{nAverage} & number of timesteps in time-averaged variables \\
	\hline
	\hyperref[subsec:var_sec_average_avgSsh]{avgSsh} & time-averaged sea surface height \\
	\hline
	\hyperref[subsec:var_sec_average_varSsh]{varSsh} & variance of sea surface height \\
	\hline
	\hyperref[subsec:var_sec_average_avgVelocityZonal]{avgVelocityZonal} & time-averaged velocity in the eastward direction \\
	\hline
	\hyperref[subsec:var_sec_average_avgVelocityMeridional]{avgVelocityMeridional} & time-averaged velocity in the northward direction \\
	\hline
	\hyperref[subsec:var_sec_average_varVelocityZonal]{varVelocityZonal} & variance of velocity in the eastward direction \\
	\hline
	\hyperref[subsec:var_sec_average_varVelocityMeridional]{varVelocityMeridional} & variance of velocity in the northward direction \\
	\hline
	\hyperref[subsec:var_sec_average_avgNormalVelocity]{avgNormalVelocity} & time-averaged velocity, normal to cell edge \\
	\hline
	\hyperref[subsec:var_sec_average_varNormalVelocity]{varNormalVelocity} & variance of velocity, normal to cell edge \\
	\hline
	\hyperref[subsec:var_sec_average_avgVertVelocityTop]{avgVertVelocityTop} & time-averaged vertical velocity at top of cell \\
	\hline
\end{longtable}
\end{center}
}
\section[forcing]{\hyperref[sec:var_sec_forcing]{forcing}}
\label{sec:var_tab_forcing}
The forcing data type contains a single time level. The forcing structure
contains fields related to surface fluxes, wind stress, and fields that can be
used to compute surface fluxes.

\vspace{0.5in}
{\small
\begin{center}
\begin{longtable}{| p{2.0in} | p{4.0in} |}
	\hline
	{\bf Name} & {\bf Description} \endfirsthead
	\hline 
	{\bf Name} & {\bf Description} (Continued) \endhead
	\hline
	\hyperref[subsec:var_sec_forcing_surfaceWindStress]{surfaceWindStress} & Wind stress at the surface of the ocean defined at edge midpoints. Magintude in direction of edge normal. \\
	\hline
	\hyperref[subsec:var_sec_forcing_surfaceWindStressMagnitude]{surfaceWindStressMagnitude} & Magnitude of wind stress at the surface of the ocean, at cell centers. \\
	\hline
	\hyperref[subsec:var_sec_forcing_surfaceMassFlux]{surfaceMassFlux} & Flux of mass through the ocean surface. Positive into ocean. \\
	\hline
	\hyperref[subsec:var_sec_forcing_surfaceTemperatureFlux]{surfaceTemperatureFlux} & Flux of temperature through the ocean surface. Positive into ocean. \\
	\hline
	\hyperref[subsec:var_sec_forcing_surfaceSalinityFlux]{surfaceSalinityFlux} & Flux of salinity through the ocean surface. Positive into ocean. \\
	\hline
	\hyperref[subsec:var_sec_forcing_surfaceTracer1Flux]{surfaceTracer1Flux} & Flux of tracer1 through the ocean surface. Positive into ocean. \\
	\hline
	\hyperref[subsec:var_sec_forcing_seaSurfacePressure]{seaSurfacePressure} & Pressure defined at the sea surface. \\
	\hline
	\hyperref[subsec:var_sec_forcing_seaIceEnergy]{seaIceEnergy} &  Energy per unit area trapped in frazil ice formation. Always  $\ge$  0.0. \\
	\hline
	\hyperref[subsec:var_sec_forcing_penetrativeTemperatureFlux]{penetrativeTemperatureFlux} & Penetrative temperature flux at the surface due to solar radiation. Positive is into the ocean. \\
	\hline
	\hyperref[subsec:var_sec_forcing_transmissionCoefficients]{transmissionCoefficients} & Divergence of transmission through interfaces of surface fluxes below the surface layer at cell centers. These are not applied to short wave. \\
	\hline
	\hyperref[subsec:var_sec_forcing_windStressZonal]{windStressZonal} & Zonal (eastward) component of wind stress at cell centers from coupler. Positive eastward. \\
	\hline
	\hyperref[subsec:var_sec_forcing_windStressMeridional]{windStressMeridional} & Meridional (northward) component of wind stress at cell centers from coupler. Positive northward. \\
	\hline
	\hyperref[subsec:var_sec_forcing_latentHeatFlux]{latentHeatFlux} & Latent heat flux at cell centers from coupler. Positive into the ocean. \\
	\hline
	\hyperref[subsec:var_sec_forcing_sensibleHeatFlux]{sensibleHeatFlux} & Sensible heat flux at cell centers from coupler. Positive into the ocean. \\
	\hline
	\hyperref[subsec:var_sec_forcing_longWaveHeatFluxUp]{longWaveHeatFluxUp} & Upward long wave heat flux at cell centers from coupler. Positive into the ocean. \\
	\hline
	\hyperref[subsec:var_sec_forcing_longWaveHeatFluxDown]{longWaveHeatFluxDown} & Downward long wave heat flux at cell centers from coupler. Positive into the ocean. \\
	\hline
	\hyperref[subsec:var_sec_forcing_seaIceHeatFlux]{seaIceHeatFlux} & Sea ice heat flux at cell centers from coupler. Positive into the ocean. \\
	\hline
	\hyperref[subsec:var_sec_forcing_shortWaveHeatFlux]{shortWaveHeatFlux} & Short wave flux at cell centers from coupler. Positive into the ocean. \\
	\hline
	\hyperref[subsec:var_sec_forcing_evaporationFlux]{evaporationFlux} & Evaporation flux at cell centers from coupler. Positive into the ocean. \\
	\hline
	\hyperref[subsec:var_sec_forcing_seaIceSalinityFlux]{seaIceSalinityFlux} & Sea ice salinity flux at cell centers from coupler. Positive into the ocean. \\
	\hline
	\hyperref[subsec:var_sec_forcing_seaIceFreshWaterFlux]{seaIceFreshWaterFlux} & Fresh water flux from sea ice at cell centers from coupler. Positive into the ocean. \\
	\hline
	\hyperref[subsec:var_sec_forcing_riverRunoffFlux]{riverRunoffFlux} & Fresh water flux from river runoff at cell centers from coupler. Positive into the ocean. \\
	\hline
	\hyperref[subsec:var_sec_forcing_iceRunoffFlux]{iceRunoffFlux} & Fresh water flux from ice runoff at cell centers from coupler. Positive into the ocean. \\
	\hline
	\hyperref[subsec:var_sec_forcing_rainFlux]{rainFlux} & Fresh water flux from rain at cell centers from coupler. Positive into the ocean. \\
	\hline
	\hyperref[subsec:var_sec_forcing_snowFlux]{snowFlux} & Fresh water flux from snow at cell centers from coupler. Positive into the ocean. \\
	\hline
	\hyperref[subsec:var_sec_forcing_iceFraction]{iceFraction} & Fraction of sea ice coverage at cell centers from coupler. Positive into the ocean. \\
	\hline
	\hyperref[subsec:var_sec_forcing_prognosticCO2]{prognosticCO2} & Prognostic CO2 at cell centers from coupler. Positive into the ocean. \\
	\hline
	\hyperref[subsec:var_sec_forcing_diagnosticCO2]{diagnosticCO2} & Diagnostic CO2 at cell centers from coupler. Positive into the ocean. \\
	\hline
	\hyperref[subsec:var_sec_forcing_squaredWindSpeed10Meter]{squaredWindSpeed10Meter} & Squared wind speed at 10 meters at cell centers from coupler. \\
	\hline
	\hyperref[subsec:var_sec_forcing_nAccumulatedCoupled]{nAccumulatedCoupled} & Number of accumulations in time averaging of coupler fields \\
	\hline
	\hyperref[subsec:var_sec_forcing_avgTemperatureSurfaceValue]{avgTemperatureSurfaceValue} & Time averaged potential temperature extrapolated to ocean surface \\
	\hline
	\hyperref[subsec:var_sec_forcing_avgSalinitySurfaceValue]{avgSalinitySurfaceValue} & Time averaged salinity extrapolated to ocean surface \\
	\hline
	\hyperref[subsec:var_sec_forcing_avgTracer1SurfaceValue]{avgTracer1SurfaceValue} & Time averaged tracer1 extrapolated to ocean surface \\
	\hline
	\hyperref[subsec:var_sec_forcing_avgZonalSurfaceVelocity]{avgZonalSurfaceVelocity} & Time averaged zonal surface velocity \\
	\hline
	\hyperref[subsec:var_sec_forcing_avgMeridionalSurfaceVelocity]{avgMeridionalSurfaceVelocity} & Time averaged meridional surface velocity \\
	\hline
	\hyperref[subsec:var_sec_forcing_avgZonalSSHGradient]{avgZonalSSHGradient} & Time averaged zonal gradient of SSH \\
	\hline
	\hyperref[subsec:var_sec_forcing_avgMeridionalSSHGradient]{avgMeridionalSSHGradient} & Time averaged meridional gradient of SSH \\
	\hline
	\hyperref[subsec:var_sec_forcing_CO2Flux]{CO2Flux} & CO2 Flux. \\
	\hline
	\hyperref[subsec:var_sec_forcing_DMSFlux]{DMSFlux} & DMS Flux. \\
	\hline
\end{longtable}
\end{center}
}
\section[scratch]{\hyperref[sec:var_sec_scratch]{scratch}}
\label{sec:var_tab_scratch}
The scratch data structure includes reusable fields that are used as temporary arrays within the code.

\vspace{0.5in}
{\small
\begin{center}
\begin{longtable}{| p{2.0in} | p{4.0in} |}
	\hline
	{\bf Name} & {\bf Description} \endfirsthead
	\hline 
	{\bf Name} & {\bf Description} (Continued) \endhead
	\hline
	\hyperref[subsec:var_sec_scratch_vorticityGradientTangentialComponent]{vorticityGradientTangentialCom-}\hyperref[subsec:var_sec_scratch_vorticityGradientTangentialComponent]{ponent  }& gradient of vorticity in the tangent direction (positive points from vertex1 to vertex2) \\
	\hline
	\hyperref[subsec:var_sec_scratch_vorticityGradientNormalComponent]{vorticityGradientNormalCompon-}\hyperref[subsec:var_sec_scratch_vorticityGradientNormalComponent]{ent  }& gradient of vorticity in the normal direction (positive points from cell1 to cell2) \\
	\hline
	\hyperref[subsec:var_sec_scratch_normalizedRelativeVorticityVertex]{normalizedRelativeVorticityVert-}\hyperref[subsec:var_sec_scratch_normalizedRelativeVorticityVertex]{ex  }& curl of horizontal velocity divided by layer thickness, defined at vertices \\
	\hline
	\hyperref[subsec:var_sec_scratch_normalizedPlanetaryVorticityVertex]{normalizedPlanetaryVorticityVe-}\hyperref[subsec:var_sec_scratch_normalizedPlanetaryVorticityVertex]{rtex  }& earth's rotational rate (Coriolis parameter, f) divided by layer thickness, defined at vertices \\
	\hline
	\hyperref[subsec:var_sec_scratch_kineticEnergyVertex]{kineticEnergyVertex} & kinetic energy of horizonal velocity defined at vertices \\
	\hline
	\hyperref[subsec:var_sec_scratch_kineticEnergyVertexOnCells]{kineticEnergyVertexOnCells} & kinetic energy of horizonal velocity defined at vertices \\
	\hline
	\hyperref[subsec:var_sec_scratch_densitySurfaceDisplaced]{densitySurfaceDisplaced} & Density computed by displacing SST and SSS to every vertical layer within the column \\
	\hline
	\hyperref[subsec:var_sec_scratch_thermalExpansionCoeff]{thermalExpansionCoeff} &  Thermal expansion coefficient (alpha), defined as  $-1/\rho d\rho/dT$  (note negative sign). \\
	\hline
	\hyperref[subsec:var_sec_scratch_salineContractionCoeff]{salineContractionCoeff} &  Saline contraction coefficient (beta), defined as  $1/\rho d\rho/dS$ .  This is also called the haline contraction coefficient. \\
	\hline
	\hyperref[subsec:var_sec_scratch_normalVelocityTest]{normalVelocityTest} & horizonal velocity, normal component to an edge, for testing \\
	\hline
	\hyperref[subsec:var_sec_scratch_tangentialVelocityTest]{tangentialVelocityTest} & horizonal velocity, tangential component to an edge, for testing \\
	\hline
	\hyperref[subsec:var_sec_scratch_strainRateR3Cell]{strainRateR3Cell} & strain rate tensor at cell center, R3, in symmetric 6-index form \\
	\hline
	\hyperref[subsec:var_sec_scratch_strainRateR3CellSolution]{strainRateR3CellSolution} & strain rate solution tensor at cell center, R3, in symmetric 6-index form \\
	\hline
	\hyperref[subsec:var_sec_scratch_strainRateR3Edge]{strainRateR3Edge} & strain rate tensor at edge, R3, in symmetric 6-index form \\
	\hline
	\hyperref[subsec:var_sec_scratch_strainRateLonLatRCell]{strainRateLonLatRCell} & strain rate tensor at cell center, 3D, lon-lat-r in symmetric 6-index form, {\color{red}Temporary only} \\
	\hline
	\hyperref[subsec:var_sec_scratch_strainRateLonLatRCellSolution]{strainRateLonLatRCellSolution} & strain rate tensor at cell center, 3D, lon-lat-r in symmetric 6-index form, {\color{red}Temporary only} \\
	\hline
	\hyperref[subsec:var_sec_scratch_strainRateLonLatREdge]{strainRateLonLatREdge} & strain rate tensor at edge, 3D, lon-lat-r in symmetric 6-index form, {\color{red}Temporary only} \\
	\hline
	\hyperref[subsec:var_sec_scratch_divTensorR3Cell]{divTensorR3Cell} & divergence of the tensor at cell center, as an R3 vector \\
	\hline
	\hyperref[subsec:var_sec_scratch_divTensorR3CellSolution]{divTensorR3CellSolution} & divergence of the tensor solution at cell center, as an R3 vector \\
	\hline
	\hyperref[subsec:var_sec_scratch_divTensorLonLatRCell]{divTensorLonLatRCell} & divergence of the tensor at cell center, as a lon-lat-r vector \\
	\hline
	\hyperref[subsec:var_sec_scratch_divTensorLonLatRCellSolution]{divTensorLonLatRCellSolution} & divergence of the tensor at cell center, as a lon-lat-r vector, solution \\
	\hline
	\hyperref[subsec:var_sec_scratch_outerProductEdge]{outerProductEdge} &  Outer product,  $u_e \otimes n_e$ , at each edge. \\
	\hline
	\hyperref[subsec:var_sec_scratch_normalVectorEdge]{normalVectorEdge} & Vector component normal to an edge. \\
	\hline
	\hyperref[subsec:var_sec_scratch_tangentialVectorEdge]{tangentialVectorEdge} & Vector component tangent to an edge. \\
	\hline
	\hyperref[subsec:var_sec_scratch_windStressFull]{windStressFull} & Wind stress used for reconstructing diagnostic output fields. \\
	\hline
	\hyperref[subsec:var_sec_scratch_windStressX]{windStressX} & reconstructed surface wind stress in the x direction. Used for diagnostics. \\
	\hline
	\hyperref[subsec:var_sec_scratch_windStressY]{windStressY} & reconstructed surface wind stress in the y direction. User for diagnostics. \\
	\hline
	\hyperref[subsec:var_sec_scratch_windStressZ]{windStressZ} & reconstructed surface wind stress in the z direction. User for diagnostics. \\
	\hline
	\hyperref[subsec:var_sec_scratch_windStressZonal]{windStressZonal} & reconstructed surface wind stress in the eastward direction. Used for diagnostics. \\
	\hline
	\hyperref[subsec:var_sec_scratch_windStressMeridional]{windStressMeridional} & reconstructed surface wind stress in the northward direction. User for diagnostics. \\
	\hline
\end{longtable}
\end{center}
}
