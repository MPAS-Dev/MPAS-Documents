\chapter{Tracer Infrastructure}
\label{chap:tracers}

Both active and passive tracers in MPAS-Ocean are governed by equation \ref{ocean:tracer}.  In the current version of the code, there are five specific groups of tracers:

\begin{itemize}
\item{{\bf{activeTracers:}} Potential temperature and salinity.} 
\item{{\bf{debugTracers:}} A passive tracer used primarily for testing conservation and monotonicity of tracer advection.}
\item{{\bf{ecosysTracers:}} Ocean biogeochemistry and ecosystem dynamics based on the Biogeochemical Elemental Cycling (BEC) model of Moore.}
%Moore, J. K., S. C. Doney, and K. Lindsay (2004), Upper ocean ecosystem dynamics and iron cycling in a global three-dimensional model, Global Biogeochemical Cycles, 18(4).
\item{{\bf{DMSTracers:}} Extension of BEC for prognostic sulfur cycling focused on calculating the flux of dimethyl sulfide (DMS) from the ocean to the atmosphere.}
\item{{\bf{MacroMoleculesTracers:}} Extension of BEC that computes the evolution of subclasses of dissolved organic carbon (DOC), such as lipids, proteins, and poly saccharides, that can affect air-sea transfer and the formation of cloud condensation nuclei.}
\end{itemize}

Each tracer group has an associated namelist named tracer\_forcing\_[groupName] (for example, tracer\_forcing\_ecosysTracers).  To activate a tracer group, set config\_use\_[groupName] = .true. The activeTracers group is the only one that is turned on by default.  Both the DMS and MacroMolecules groups rely on the ecosys tracers to be enabled, but ecosys does not require DMS or MacroMolecules to be active.

In keeping with the MPAS-O philosophy of maximum flexibility, every tracer is configured to allow for 6 different types of forcing which the user can mix and match at will:

\begin{itemize}
\item{{\bf{surface\_bulk\_forcing:}} Surface flux will be applied.  It can be supplied by reading in a field from a file, calculating in a user-created routine, passed from another model component (for example, by the E3SM flux coupler), or a combination of these.} 
\item{{\bf{surface\_restoring:}} Surface layer tracer is linearly restored to a climatological field.} 
\item{{\bf{interior\_restoring:}} Full 3D tracer is linearly restored to a climatological field, except for the surface layer.} 
\item{{\bf{exponential\_decay:}} Tracer will decay at an exponential rate.  Intended for radioactive tracers and, for example, simple models of oil degradation.} 
\item{{\bf{ideal-Age\_forcing:}} Tracer is Ideal Age, where the surface value is reset to zero and interior values are incremented by {\it dt} every timestep.} 
\item{{\bf{ttd\_forcing:}} Tracer surface field is reset to a specified value every timestep.  Intended for simulating Time Transit Distribution functions (TTDs).} 
\end{itemize}

Each type of forcing requires 1 or more associated fields (10 total) to be defined for each tracer (a spatially varying surface restoring rate or a mask defining where TTD forcing is applied), though they won't be allocated memory unless they are needed.  As a result, there are a large number of fields that are defined in Registry\_[groupName].xml that will likely never actually be used.  For example, the ecosysTracers group is made up of 30 individual tracers and each of them must have 10 associated forcing fields defined, resulting in 300 defined arrays, the vast majority of which will never be used.  Again, these unused fields don't get allocated any memory, but they still must be defined to avoid a runtime error.

Another idiosyncrasy of the tracer implementation in MPAS-O concerns output.  Due to the way tracer groups have been defined using the constructs of the MPAS framework, it isn't possible to directly output specific tracers within a group to an output stream; the entire group must be designated.  For temperature and salinity, for example, this isn't a major issue since they are both typically desired for output.  However, it can be an issue for other groups such as ecosysTracers when the user may only want a subset of prognostic fields (for example, nutrients and chlorophyll), but is required to output either all 30 tracers or none at all, which has significant storage implications.  This feature will be addressed as soon as possible.

