\chapter*{Foreword}
\label{chap:foreword}

MPAS-Seaice is an unstructured-mesh sea-ice model that uses the
Modeling for Prediction Across Scales (MPAS) framework, allowing
enhanced horizontal resolution in regions of interest. MPAS-Seaice
uses many of the methods used in the Los Alamos CICE sea-ice model,
but adapted to the Spherical Centroidal Vornoi Tesselation (SCVT) meshes used
by the MPAS framework. MPAS-Seaice is one component within the MPAS
framework of climate model components that is developed in cooperation between Los Alamos
National Laboratory (LANL) and the National Center for Atmospheric
Research (NCAR). A full description and validation of MPAS-Seaice is
currently in review with the Journal of Advances in Modeling Earth
Systems (JAMES) and is available at \url{http://doi.org/10.5281/zenodo.1194374}.


A history of releases of the sea-ice core within the MPAS version numbering scheme is as follows:

\begin{table}[H]
\begin{tabular}{lll} 
\hline\hline version & date & description  \\
\hline 
6.0 & April 17, 2018 & Initial public release of MPAS-Seaice \\
\hline 
\end{tabular}
\end{table} 


\vspace{8pt}
\noindent
{\it Funding for the development of MPAS-Seaice was provided by the United States Department of Energy, Office of Science.}

\newpage
