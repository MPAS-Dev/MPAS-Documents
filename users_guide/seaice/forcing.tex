\section{Forcing}
\label{sec:forcing}

The MPAS framework includes a forcing module designed to perform time interpolation of forcing data. MPAS-Seaice currently implements a single forcing option which repeats the methods used by \citet{Hunke13}. Air temperature, air specific humidity and air velocity at 10 m height and six-hourly frequency are taken from the Coordinated Ocean-ice Reference Experiments (CORE) Corrected Inter-Annual Forcing Version 2.0 \citep{Large09, Griffies09}. Monthly climatologies of precipitation \citep{Griffies09} and cloudiness \citep{Roske01} are also used. Downwelling shortwave radiation is calculated from the monthly climatology of cloudiness using the AOMIP shortwave forcing formula \citep{Hunke15}. Downwelling longwave radiation is calculated according to \citet{Rosati88}. Oceanic inputs, consisting of sea surface salinity, initial sea surface temperature, currents, sea-surface slope and deep ocean heat flux, come from monthly mean output of 20 years of a Community Climate System Model (CCSM) climate run (b30.009, Collins et al., 2006). All input forcing fields are interpolated linearly in time, although the MPAS forcing functionality can be easily extended to allow interpolation in time with arbitrary order.