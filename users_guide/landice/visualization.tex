\chapter{Land Ice Visualization}
\label{chap:landice_visualization}

This chapter discusses visualization tools that are specific to the Land Ice core.  For instructions on visualization tools that may be used by all cores, such as Paraview, see Chapter \ref{chap:mpas_visualization}.

\section{Python}
\label{sec:landice_python}

Python visualization scripts are available for the dome test case, and general python visualization tools are in development.  In order to use these scripts, the following python modules are required:
\begin{itemize}
\item matplotlib, see \url{http://matplotlib.org}
\item numpy, see \url{http://www.numpy.org}
\item pylab, see \url{www.scipy.org}
\item netCDF4, see \url{http://code.google.com/p/netcdf4-python}
\end{itemize}
Most package managers (including MacPorts) have packages for these python modules. 
Another convenient way to install all these libraries at once is to purchase the Enthought Python Distribution (EPD), available at \url{https://www.enthought.com/products/epd}.  
Many institutions have Python-EPD installed on their compute clusters.

%Examples of output from python visualization scripts are shown in Figures \ref{fig:baroclinicChannelTemperature} and \ref{fig:overflow}.  Options for each script may be found using, e.g.
%\begin{verbatim}
%python visualize_overflow.py --help
%\end{verbatim}
%These scripts are easily customizable by the user.  Within the python script the specified variable is read from the NetCDF file using the commands
%\begin{verbatim}
%f = NetCDFFile(options.filename,'r')
%field = f.variables[options.variable]
%\end{verbatim}
%The plot is created with the matplotlib command {\tt plt.imshow}, as described in the following tutorials:
%\begin{itemize}
%\item \url{http://matplotlib.org/api/pyplot_api.html#matplotlib.pyplot.imshow}
%\item \url{http://matplotlib.org/users/image_tutorial.html}
%\end{itemize}
%Text is then added with the commands {\tt xlabel}, {\tt ylabel}, {\tt title}; colorbar is added using {\tt plt.colorbar}; and the figure is saved to the local directory as a png file useing {\tt plt.savefig}.
