\section{Dimensions}
\label{sec:forward_dimensions}
{\small
\begin{center}
\begin{longtable}{| p{1.2in} || p{1.0in} | p{4.0in} |}
	\hline 
	{\bf Name} & {\bf Units} & {\bf Description} \endfirsthead
	\hline 
	{\bf Name} & {\bf Units} & {\bf Description} (Continued) \endhead
	\hline 
	\hline 
	nCells & $unitless$ & The number of polygons in the primary grid. \\ 
	\hline
	nEdges & $unitless$ & The number of edge midpoints in either the primary or dual grid. \\ 
	\hline
	maxEdges & $unitless$ & The largest number of edges any polygon within the grid has. \\ 
	\hline
	maxEdges2 & $unitless$ & Two times the largest number of edges any polygon within the grid has. \\ 
	\hline
	nVertices & $unitless$ & The total number of cells in the dual grid. Also the number of corners in the primary grid. \\ 
	\hline
	TWO & $unitless$ & The number two as a dimension. \\ 
	\hline
	R3 & $unitless$ & The number three as a dimension. \\ 
	\hline
	vertexDegree & $unitless$ & The number of cells or edges touching each vertex. \\ 
	\hline
	nVertLevels & $unitless$ & The number of levels in the vertical direction. All vertical levels share the same horizontal locations. \\ 
	\hline
	nVertLevelsP1 & $unitless$ & The number of interfaces in the vertical direction. \\ 
	\hline
\end{longtable}
\end{center}
}
\section[Namelist options]{\hyperref[chap:namelist_sections]{Namelist options}}
\label{sec:forward_namelist_tables}
Embedded links point to more detailed namelist information in the appendix.
\subsection[velocity\_solver]{velocity\_solver}
\label{subsec:forward_nm_tab_velocity_solver}
The velocity\_solver namelist record controls which velocity solver is used and options associated with velocity solvers.

\vspace{0.5in}
{\small
\begin{center}
\begin{longtable}{| p{2.0in} || p{4.0in} |}
	\hline
	{\bf Name} & {\bf Description} \endfirsthead
	\hline 
	{\bf Name} & {\bf Description} (Continued) \endhead
	\hline
	\hline
	\hyperref[sec:nm_sec_config_velocity_solver]{config\_velocity\_solver} & Selection of the method for solving ice velocity. \\
	\hline
\end{longtable}
\end{center}
}
\subsection[advection]{advection}
\label{subsec:forward_nm_tab_advection}
Three-dimensional tracer advection can be computed using 2$^{nd}$, 3$^{rd}$ or 4$^{th}$ flux reconstructions in the horizontal and vertical. In the horizontal, the high-order (i.e. 3$^{rd}$ or 4$^{th}$) flux reconstruction is done following \cite{Skamarock:2011tc}. Typically, the scheme is implemented with an upwind-bias ($\beta$=0.25 in (11) from \cite{Skamarock:2011tc}) to produce a 3$^{rd}$-order accurate reconstruction of tracer flux divergence on uniform hexagonal meshes. In the vertical, high-order estimates of tracer values at layer edges are reconstructed using a cubic spline. Monotone transport is guaranteed by blending these high-order flux approximations with the 1$^{st}$-order, upstream flux using the \cite{Zalesak:1979wm} flux-corrected transport scheme.
\vspace{0.5in}
{\small
\begin{center}
\begin{longtable}{| p{2.0in} || p{4.0in} |}
	\hline
	{\bf Name} & {\bf Description} \endfirsthead
	\hline 
	{\bf Name} & {\bf Description} (Continued) \endhead
	\hline
	\hline
	\hyperref[sec:nm_sec_config_thickness_advection]{config\_thickness\_advection} & Selection of the method for advecting thickness. \\
	\hline
	\hyperref[sec:nm_sec_config_tracer_advection]{config\_tracer\_advection} & Selection of the method for advecting tracers. \\
	\hline
\end{longtable}
\end{center}
}
\subsection[physical\_parameters]{physical\_parameters}
\label{subsec:forward_nm_tab_physical_parameters}
The physical\_parameters namelist record sets scalar physical parameters and constants within the land ice model.

\vspace{0.5in}
{\small
\begin{center}
\begin{longtable}{| p{2.0in} || p{4.0in} |}
	\hline
	{\bf Name} & {\bf Description} \endfirsthead
	\hline 
	{\bf Name} & {\bf Description} (Continued) \endhead
	\hline
	\hline
	\hyperref[sec:nm_sec_config_ice_density]{config\_ice\_density} & ice density to use \\
	\hline
	\hyperref[sec:nm_sec_config_ocean_density]{config\_ocean\_density} & ocean density to use for calculating floatation \\
	\hline
	\hyperref[sec:nm_sec_config_sea_level]{config\_sea\_level} & sea level to use for calculating floatation \\
	\hline
	\hyperref[sec:nm_sec_config_default_flowParamA]{config\_default\_flowParamA} &  Defines the default value of the flow law parameter A to be used if it is not being calculated from ice temperature.  Defaults to the SI representation of 1.0e-16 yr $^{-1}$  Pa $^{-3}$ . \\
	\hline
	\hyperref[sec:nm_sec_config_flowLawExponent]{config\_flowLawExponent} & Defines the value of the Glen flow law exponent, n. \\
	\hline
	\hyperref[sec:nm_sec_config_dynamic_thickness]{config\_dynamic\_thickness} & Defines the ice thickness below which dynamics are not calculated. \\
	\hline
\end{longtable}
\end{center}
}
\subsection[time\_integration]{time\_integration}
\label{subsec:forward_nm_tab_time_integration}
The time integration namelist controls parameters that pertain to all time-stepping methods.  Options that are specific to a particular time-stepping method are contained in a separate namelist for that method, below.

\vspace{0.5in}
{\small
\begin{center}
\begin{longtable}{| p{2.0in} || p{4.0in} |}
	\hline
	{\bf Name} & {\bf Description} \endfirsthead
	\hline 
	{\bf Name} & {\bf Description} (Continued) \endhead
	\hline
	\hline
	\hyperref[sec:nm_sec_config_dt]{config\_dt} & Length of model time step defined as a time interval. \\
	\hline
	\hyperref[sec:nm_sec_config_time_integration]{config\_time\_integration} & Time integration method. \\
	\hline
\end{longtable}
\end{center}
}
\subsection[time\_management]{time\_management}
\label{subsec:forward_nm_tab_time_management}
General time management is handled by the time\_management namelist record.
Included options handle time-related parts of MPAS, such as the calendar and if the simulation is a restart or not.

Users should use this record to specify the beginning time of the simulation,
and either the duration or the end of the simulation. Only the end or the
duration need to be specified as the other is derived within MPAS from the
beginning time and other specified one.

{\bf TBA: If both duration and stop are specified, then what happens?)}

\vspace{0.5in}
{\small
\begin{center}
\begin{longtable}{| p{2.0in} || p{4.0in} |}
	\hline
	{\bf Name} & {\bf Description} \endfirsthead
	\hline 
	{\bf Name} & {\bf Description} (Continued) \endhead
	\hline
	\hline
	\hyperref[sec:nm_sec_config_do_restart]{config\_do\_restart} & Determines if the initial conditions should be read from a restart file, or an input file.  To perform a restart, simply set this to true in the namelist.input file and modify the start time to be the time you want restart from.  A restart will read the grid information from the input field, and the restart state from the restart file.  It will perform a run normally, except velocity will not be solved on a restart. \\
	\hline
	\hyperref[sec:nm_sec_config_restart_timestamp_name]{config\_restart\_timestamp\_name} & Path to the filename for restart timestamps to be read and written from. \\
	\hline
	\hyperref[sec:nm_sec_config_start_time]{config\_start\_time} & Timestamp describing the initial time of the simulation.  If it is set to 'file', the initial time is read from restart\_timestamp \\
	\hline
	\hyperref[sec:nm_sec_config_stop_time]{config\_stop\_time} & Timestamp describing the final time of the simulation. If it is set to 'none' the final time is determined from config\_start\_time and config\_run\_duration.  If config\_run\_duration is also specified, it takes precedence over config\_stop\_time.  Set config\_stop\_time to be equal to config\_start\_time (and config\_run\_duration to 'none') to perform a diagnostic solve only. \\
	\hline
	\hyperref[sec:nm_sec_config_run_duration]{config\_run\_duration} & Timestamp describing the length of the simulation. If it is set to 'none' the duration is determined from config\_start\_time and config\_stop\_time. config\_run\_duration overrides inconsistent values of config\_stop\_time. If a time value is specified for config\_run\_duration, it must be greater than 0. \\
	\hline
	\hyperref[sec:nm_sec_config_calendar_type]{config\_calendar\_type} & Selection of the type of calendar that should be used in the simulation. \\
	\hline
\end{longtable}
\end{center}
}
\subsection[io]{io}
\label{subsec:forward_nm_tab_io}
The io namelist record provides options for modifications to the I/O system of
MPAS. These include frequency, file name, and parallelization options.

\vspace{0.5in}
{\small
\begin{center}
\begin{longtable}{| p{2.0in} || p{4.0in} |}
	\hline
	{\bf Name} & {\bf Description} \endfirsthead
	\hline 
	{\bf Name} & {\bf Description} (Continued) \endhead
	\hline
	\hline
	\hyperref[sec:nm_sec_config_write_output_on_startup]{config\_write\_output\_on\_startu-}\hyperref[sec:nm_sec_config_write_output_on_startup]{p}& Logical flag determining if an output file should be written prior to the first time step. \\
	\hline
	\hyperref[sec:nm_sec_config_pio_num_iotasks]{config\_pio\_num\_iotasks} & Integer specifying how many IO tasks should be used within the PIO library. A value of 0 causes all MPI tasks to also be IO tasks. IO tasks are required to write contiguous blocks of data to a file. \\
	\hline
	\hyperref[sec:nm_sec_config_pio_stride]{config\_pio\_stride} & Integer specifying the stride of each IO task. \\
	\hline
	\hyperref[sec:nm_sec_config_year_digits]{config\_year\_digits} & Integer specifying the number of digits used to represent the year in time strings. \\
	\hline
\end{longtable}
\end{center}
}
\subsection[decomposition]{decomposition}
\label{subsec:forward_nm_tab_decomposition}
Namelist parameters for the \verb+decomposition+ namelist group.

\vspace{0.5in}
{\small
\begin{center}
\begin{longtable}{| p{2.0in} || p{4.0in} |}
	\hline
	{\bf Name} & {\bf Description} \endfirsthead
	\hline 
	{\bf Name} & {\bf Description} (Continued) \endhead
	\hline
	\hline
	\hyperref[sec:nm_sec_config_num_halos]{config\_num\_halos} & Determines the number of halo cells extending from a blocks owned cells (Called the 0-Halo). The default of 3 is the minimum that can be used with monotonic advection. \\
	\hline
	\hyperref[sec:nm_sec_config_block_decomp_file_prefix]{config\_block\_decomp\_file\_prefix} & Defines the prefix for the block decomposition file. Can include a path. The number of blocks is appended to the end of the prefix at run-time. \\
	\hline
	\hyperref[sec:nm_sec_config_number_of_blocks]{config\_number\_of\_blocks} & Determines the number of blocks a simulation should be run with. If it is set to 0, the number of blocks is the same as the number of MPI tasks at run-time. \\
	\hline
	\hyperref[sec:nm_sec_config_explicit_proc_decomp]{config\_explicit\_proc\_decomp} & Determines if an explicit processor decomposition should be used. This is only useful if multiple blocks per processor are used. \\
	\hline
	\hyperref[sec:nm_sec_config_proc_decomp_file_prefix]{config\_proc\_decomp\_file\_prefix} & Defines the prefix for the processor decomposition file. This file is only read if config\_explicit\_proc\_decomp is .true. The number of processors is appended to the end of the prefix at run-time. \\
	\hline
\end{longtable}
\end{center}
}
\subsection[debug]{debug}
\label{subsec:forward_nm_tab_debug}
At run-time a user can enable debugging features within MPAS-Land Ice. 
Currently the only debug option is to print more detailed information about
thickness advection.
Potential future debug options would be to include disabling of any 
tendencies to help determine why an issue might
be happening; various checks on certain fields;
and the ability to prescribe both a thickness and velocity field at run-time
which are constant throughout a simulation. All options that control these
debugging features are specified within the debug namelist record.

\vspace{0.5in}
{\small
\begin{center}
\begin{longtable}{| p{2.0in} || p{4.0in} |}
	\hline
	{\bf Name} & {\bf Description} \endfirsthead
	\hline 
	{\bf Name} & {\bf Description} (Continued) \endhead
	\hline
	\hline
	\hyperref[sec:nm_sec_config_print_thickness_advection_info]{config\_print\_thickness\_advectio-}\hyperref[sec:nm_sec_config_print_thickness_advection_info]{n\_info}& Prints additional information about thickness advection. \\
	\hline
\end{longtable}
\end{center}
}
\section[Variable definitions]{\hyperref[chap:variable_sections]{Variable definitions}}
\label{sec:forward_variable_tables}
Embedded links point to more detailed variable information in the appendix.
\subsection[state]{\hyperref[sec:var_sec_state]{state}}
\label{subsec:forward_var_tab_state}
\vspace{0.5in}
{\small
\begin{center}
\begin{longtable}{| p{2.0in} | p{4.0in} |}
	\hline
	{\bf Name} & {\bf Description} \endfirsthead
	\hline 
	{\bf Name} & {\bf Description} (Continued) \endhead
	\hline
	\hyperref[subsec:var_sec_state_xtime]{xtime} & model time, with format 'YYYY-MM-DD\_HH:MM:SS' \\
	\hline
	\hyperref[subsec:var_sec_state_thickness]{thickness} & ice thickness \\
	\hline
	\hyperref[subsec:var_sec_state_layerThickness]{layerThickness} & layer thickness \\
	\hline
	\hyperref[subsec:var_sec_state_temperature]{temperature} & ice temperature \\
	\hline
	\hyperref[subsec:var_sec_state_lowerSurface]{lowerSurface} & elevation at bottom of ice \\
	\hline
	\hyperref[subsec:var_sec_state_upperSurface]{upperSurface} & elevation at top of ice \\
	\hline
	\hyperref[subsec:var_sec_state_layerThicknessEdge]{layerThicknessEdge} & layer thickness on cell edges \\
	\hline
	\hyperref[subsec:var_sec_state_upperSurfaceVertex]{upperSurfaceVertex} & elevation at top of ice on vertices (currently only needed by shallow ice solver) \\
	\hline
	\hyperref[subsec:var_sec_state_cellMask]{cellMask} & bitmask indicating various properties about the ice sheet on cells.  cellMask only needs to be a restart field if config\_allow\_additional\_advance = false (to keep the mask of initial ice extent) \\
	\hline
	\hyperref[subsec:var_sec_state_edgeMask]{edgeMask} & bitmask indicating various properties about the ice sheet on edges. \\
	\hline
	\hyperref[subsec:var_sec_state_vertexMask]{vertexMask} & bitmask indicating various properties about the ice sheet on vertices. \\
	\hline
	\hyperref[subsec:var_sec_state_normalVelocity]{normalVelocity} & horizonal velocity, normal component to an edge \\
	\hline
	\hyperref[subsec:var_sec_state_uReconstructX]{uReconstructX} & x-component of velocity reconstructed on cell centers \\
	\hline
	\hyperref[subsec:var_sec_state_uReconstructY]{uReconstructY} & y-component of velocity reconstructed on cell centers \\
	\hline
	\hyperref[subsec:var_sec_state_uReconstructZ]{uReconstructZ} & z-component of velocity reconstructed on cell centers \\
	\hline
	\hyperref[subsec:var_sec_state_uReconstructZonal]{uReconstructZonal} & zonal velocity reconstructed on cell centers \\
	\hline
	\hyperref[subsec:var_sec_state_uReconstructMeridional]{uReconstructMeridional} & meridional velocity reconstructed on cell centers \\
	\hline
\end{longtable}
\end{center}
}
\subsection[tend]{\hyperref[sec:var_sec_tend]{tend}}
\label{subsec:forward_var_tab_tend}
\vspace{0.5in}
{\small
\begin{center}
\begin{longtable}{| p{2.0in} | p{4.0in} |}
	\hline
	{\bf Name} & {\bf Description} \endfirsthead
	\hline 
	{\bf Name} & {\bf Description} (Continued) \endhead
	\hline
	\hyperref[subsec:var_sec_tend_tend_layerThickness]{tend\_layerThickness} & time tendency of layer thickness \\
	\hline
	\hyperref[subsec:var_sec_tend_tend_temperature]{tend\_temperature} & time tendency of ice temperature \\
	\hline
\end{longtable}
\end{center}
}
\subsection[mesh]{\hyperref[sec:var_sec_mesh]{mesh}}
\label{subsec:forward_var_tab_mesh}
\vspace{0.5in}
{\small
\begin{center}
\begin{longtable}{| p{2.0in} | p{4.0in} |}
	\hline
	{\bf Name} & {\bf Description} \endfirsthead
	\hline 
	{\bf Name} & {\bf Description} (Continued) \endhead
	\hline
	\hyperref[subsec:var_sec_mesh_latCell]{latCell} & Latitude location of cell centers in radians. \\
	\hline
	\hyperref[subsec:var_sec_mesh_lonCell]{lonCell} & Longitude location of cell centers in radians. \\
	\hline
	\hyperref[subsec:var_sec_mesh_xCell]{xCell} & X Coordinate in cartesian space of cell centers. \\
	\hline
	\hyperref[subsec:var_sec_mesh_yCell]{yCell} & Y Coordinate in cartesian space of cell centers. \\
	\hline
	\hyperref[subsec:var_sec_mesh_zCell]{zCell} & Z Coordinate in cartesian space of cell centers. \\
	\hline
	\hyperref[subsec:var_sec_mesh_indexToCellID]{indexToCellID} & List of global cell IDs. \\
	\hline
	\hyperref[subsec:var_sec_mesh_latEdge]{latEdge} & Latitude location of edge midpoints in radians. \\
	\hline
	\hyperref[subsec:var_sec_mesh_lonEdge]{lonEdge} & Longitude location of edge midpoints in radians. \\
	\hline
	\hyperref[subsec:var_sec_mesh_xEdge]{xEdge} & X Coordinate in cartesian space of edge midpoints. \\
	\hline
	\hyperref[subsec:var_sec_mesh_yEdge]{yEdge} & Y Coordinate in cartesian space of edge midpoints. \\
	\hline
	\hyperref[subsec:var_sec_mesh_zEdge]{zEdge} & Z Coordinate in cartesian space of edge midpoints. \\
	\hline
	\hyperref[subsec:var_sec_mesh_indexToEdgeID]{indexToEdgeID} & List of global edge IDs. \\
	\hline
	\hyperref[subsec:var_sec_mesh_latVertex]{latVertex} & Latitude location of vertices in radians. \\
	\hline
	\hyperref[subsec:var_sec_mesh_lonVertex]{lonVertex} & Longitude location of vertices in radians. \\
	\hline
	\hyperref[subsec:var_sec_mesh_xVertex]{xVertex} & X Coordinate in cartesian space of vertices. \\
	\hline
	\hyperref[subsec:var_sec_mesh_yVertex]{yVertex} & Y Coordinate in cartesian space of vertices. \\
	\hline
	\hyperref[subsec:var_sec_mesh_zVertex]{zVertex} & Z Coordinate in cartesian space of vertices. \\
	\hline
	\hyperref[subsec:var_sec_mesh_indexToVertexID]{indexToVertexID} & List of global vertex IDs. \\
	\hline
	\hyperref[subsec:var_sec_mesh_cellsOnEdge]{cellsOnEdge} & List of cells that straddle each edge. \\
	\hline
	\hyperref[subsec:var_sec_mesh_nEdgesOnCell]{nEdgesOnCell} & Number of edges that border each cell. \\
	\hline
	\hyperref[subsec:var_sec_mesh_nEdgesOnEdge]{nEdgesOnEdge} & Number of edges that surround each of the cells that straddle each edge. These edges are used to reconstruct the tangential velocities. \\
	\hline
	\hyperref[subsec:var_sec_mesh_edgesOnCell]{edgesOnCell} & List of edges that border each cell. \\
	\hline
	\hyperref[subsec:var_sec_mesh_edgesOnEdge]{edgesOnEdge} & List of edges that border each of the cells that straddle each edge. \\
	\hline
	\hyperref[subsec:var_sec_mesh_weightsOnEdge]{weightsOnEdge} & Reconstruction weights associated with each of the edgesOnEdge. \\
	\hline
	\hyperref[subsec:var_sec_mesh_dvEdge]{dvEdge} & Length of each edge, computed as the distance between verticesOnEdge. \\
	\hline
	\hyperref[subsec:var_sec_mesh_dcEdge]{dcEdge} & Length of each edge, computed as the distance between cellsOnEdge. \\
	\hline
	\hyperref[subsec:var_sec_mesh_angleEdge]{angleEdge} & Angle the edge normal makes with local eastward direction. \\
	\hline
	\hyperref[subsec:var_sec_mesh_areaCell]{areaCell} & Area of each cell in the primary grid. \\
	\hline
	\hyperref[subsec:var_sec_mesh_areaTriangle]{areaTriangle} & Area of each cell (triangle) in the dual grid. \\
	\hline
	\hyperref[subsec:var_sec_mesh_edgeNormalVectors]{edgeNormalVectors} & Normal vector defined at an edge. \\
	\hline
	\hyperref[subsec:var_sec_mesh_localVerticalUnitVectors]{localVerticalUnitVectors} & Unit surface normal vectors defined at cell centers. \\
	\hline
	\hyperref[subsec:var_sec_mesh_cellTangentPlane]{cellTangentPlane} & The two vectors that define a tangent plane at a cell center. \\
	\hline
	\hyperref[subsec:var_sec_mesh_cellsOnCell]{cellsOnCell} & List of cells that neighbor each cell. \\
	\hline
	\hyperref[subsec:var_sec_mesh_verticesOnCell]{verticesOnCell} & List of vertices that border each cell. \\
	\hline
	\hyperref[subsec:var_sec_mesh_verticesOnEdge]{verticesOnEdge} & List of vertices that straddle each edge. \\
	\hline
	\hyperref[subsec:var_sec_mesh_edgesOnVertex]{edgesOnVertex} & List of edges that share a vertex as an endpoint. \\
	\hline
	\hyperref[subsec:var_sec_mesh_cellsOnVertex]{cellsOnVertex} & List of cells that share a vertex. \\
	\hline
	\hyperref[subsec:var_sec_mesh_kiteAreasOnVertex]{kiteAreasOnVertex} & Area of the portions of each dual cell that are part of each cellsOnVertex. \\
	\hline
	\hyperref[subsec:var_sec_mesh_coeffs_reconstruct]{coeffs\_reconstruct} & Coefficients to reconstruct velocity vectors at cells centers. \\
	\hline
	\hyperref[subsec:var_sec_mesh_edgeSignOnCell]{edgeSignOnCell} & Sign of edge contributions to a cell for each edge on cell. Used for bit-reproducible loops. Represents directionality of vector connecting cells. \\
	\hline
	\hyperref[subsec:var_sec_mesh_edgeSignOnVertex]{edgeSignOnVertex} & Sign of edge contributions to a vertex for each edge on vertex. Used for bit-reproducible loops. Represents directionality of vector connecting vertices. \\
	\hline
	\hyperref[subsec:var_sec_mesh_layerThicknessFractions]{layerThicknessFractions} & Fractional thickness of each sigma layer \\
	\hline
	\hyperref[subsec:var_sec_mesh_layerCenterSigma]{layerCenterSigma} & Sigma (fractional) level at center of each layer \\
	\hline
	\hyperref[subsec:var_sec_mesh_layerInterfaceSigma]{layerInterfaceSigma} & Sigma (fractional) level at interface between each layer (including top and bottom) \\
	\hline
	\hyperref[subsec:var_sec_mesh_bedTopography]{bedTopography} & Elevation of ice sheet bed.  Once isostasy is added to the model, this should become a state variable. \\
	\hline
	\hyperref[subsec:var_sec_mesh_sfcMassBal]{sfcMassBal} & Surface mass balance \\
	\hline
\end{longtable}
\end{center}
}
