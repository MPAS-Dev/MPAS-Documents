\chapter[Namelist options]{\hyperref[chap:namelist_sections]{Namelist options}}
\label{chap:namelist_tables}
Embedded links point to more detailed namelist information in the appendix.
\section[velocity\_solver]{\hyperref[sec:nm_sec_velocity_solver]{velocity\_solver}}
\label{sec:nm_tab_velocity_solver}
The velocity\_solver namelist record controls which velocity solver is used and options associated with velocity solvers.

{\small
\begin{center}
\begin{longtable}{| p{2.0in} || p{4.0in} |}
	\hline
	{\bf Name} & {\bf Description} \\
	\hline
	\hline
	\hyperref[subsec:nm_sec_config_velocity_solver]{config\_velocity\_solver} & Selection of the method for solving ice velocity. \\
	\hline
\end{longtable}
\end{center}
}
\section[advection]{\hyperref[sec:nm_sec_advection]{advection}}
\label{sec:nm_tab_advection}
Three-dimensional tracer advection can be computed using 2$^{nd}$, 3$^{rd}$ or 4$^{th}$ flux reconstructions in the horizontal and vertical. In the horizontal, the high-order (i.e. 3$^{rd}$ or 4$^{th}$) flux reconstruction is done following \cite{Skamarock:2011tc}. Typically, the scheme is implemented with an upwind-bias ($\beta$=0.25 in (11) from \cite{Skamarock:2011tc}) to produce a 3$^{rd}$-order accurate reconstruction of tracer flux divergence on uniform hexagonal meshes. In the vertical, high-order estimates of tracer values at layer edges are reconstructed using a cubic spline. Monotone transport is guaranteed by blending these high-order flux approximations with the 1$^{st}$-order, upstream flux using the \cite{Zalesak:1979wm} flux-corrected transport scheme.
{\small
\begin{center}
\begin{longtable}{| p{2.0in} || p{4.0in} |}
	\hline
	{\bf Name} & {\bf Description} \\
	\hline
	\hline
	\hyperref[subsec:nm_sec_config_thickness_advection]{config\_thickness\_advection} & Selection of the method for advecting thickness. \\
	\hline
	\hyperref[subsec:nm_sec_config_tracer_advection]{config\_tracer\_advection} & Selection of the method for advecting tracers. \\
	\hline
\end{longtable}
\end{center}
}
\section[physical\_parameters]{\hyperref[sec:nm_sec_physical_parameters]{physical\_parameters}}
\label{sec:nm_tab_physical_parameters}
The physical\_parameters namelist record sets scalar physical parameters and constants within the land ice model.

{\small
\begin{center}
\begin{longtable}{| p{2.0in} || p{4.0in} |}
	\hline
	{\bf Name} & {\bf Description} \\
	\hline
	\hline
	\hyperref[subsec:nm_sec_config_ice_density]{config\_ice\_density} & ice density to use \\
	\hline
	\hyperref[subsec:nm_sec_config_ocean_density]{config\_ocean\_density} & ocean density to use for calculating floatation \\
	\hline
	\hyperref[subsec:nm_sec_config_sea_level]{config\_sea\_level} & sea level to use for calculating floatation \\
	\hline
	\hyperref[subsec:nm_sec_config_default_flowParamA]{config\_default\_flowParamA} & Defines the default value of the flow law parameter A to be used if it is not being calculated from ice temperature.  Defaults to the SI representation of 1.0e-16 yr$^{-1}$ Pa$^{-3}$. \\
	\hline
	\hyperref[subsec:nm_sec_config_flowLawExponent]{config\_flowLawExponent} & Defines the value of the Glen flow law exponent, n. \\
	\hline
	\hyperref[subsec:nm_sec_config_dynamic_thickness]{config\_dynamic\_thickness} & Defines the ice thickness below which dynamics are not calculated. \\
	\hline
\end{longtable}
\end{center}
}
\section[time\_integration]{\hyperref[sec:nm_sec_time_integration]{time\_integration}}
\label{sec:nm_tab_time_integration}
The time integration namelist controls parameters that pertain to all time-stepping methods.  Options that are specific to a particular time-stepping method are contained in a separate namelist for that method, below.

{\small
\begin{center}
\begin{longtable}{| p{2.0in} || p{4.0in} |}
	\hline
	{\bf Name} & {\bf Description} \\
	\hline
	\hline
	\hyperref[subsec:nm_sec_config_dt_years]{config\_dt\_years} & Length of model time-step in years.  Will be used instead of config\_dt\_seconds if greater than zero.  Currently the model assumes there are 365.0 * 24.0 * 3600.0 seconds in a year and the calendar type is not considered for this conversion. \\
	\hline
	\hyperref[subsec:nm_sec_config_dt_seconds]{config\_dt\_seconds} & Length of model time-step in seconds.  This value will only be used if config\_dt\_years is less than or equal to zero. \\
	\hline
	\hyperref[subsec:nm_sec_config_time_integration]{config\_time\_integration} & Time integration method. \\
	\hline
\end{longtable}
\end{center}
}
\section[time\_management]{\hyperref[sec:nm_sec_time_management]{time\_management}}
\label{sec:nm_tab_time_management}
General time management is handled by the time\_management namelist record.
Included options handle time-related parts of MPAS, such as the calendar and if the simulation is a restart or not.

Users should use this record to specify the beginning time of the simulation,
and either the duration or the end of the simulation. Only the end or the
duration need to be specified as the other is derived within MPAS from the
beginning time and other specified one.

{\bf TBA: If both duration and stop are specified, then what happens?)}

{\small
\begin{center}
\begin{longtable}{| p{2.0in} || p{4.0in} |}
	\hline
	{\bf Name} & {\bf Description} \\
	\hline
	\hline
	\hyperref[subsec:nm_sec_config_do_restart]{config\_do\_restart} & Determines if the initial conditions should be read from a restart file, or an input file.  To perform a restart, simply set this to true in the namelist.input file and modify the start time to be the time you want restart from.  A restart will read the grid information from the input field, and the restart state from the restart file.  It will perform a run normally, i.e. do all the same init. \\
	\hline
	\hyperref[subsec:nm_sec_config_start_time]{config\_start\_time} & Timestamp describing the initial time of the simulation.  If it is set to 'file', the initial time is read from restart\_timestamp \\
	\hline
	\hyperref[subsec:nm_sec_config_stop_time]{config\_stop\_time} & Timestamp describing the final time of the simulation. If it is set to 'none' the final time is determined from config\_start\_time and config\_run\_duration.  If config\_run\_duration is also specified, it takes precedence over config\_stop\_time.  Set config\_stop\_time to be equal to config\_start\_time (and config\_run\_duration to 'none') to perform a diagnostic solve of velocity. \\
	\hline
	\hyperref[subsec:nm_sec_config_run_duration]{config\_run\_duration} & Timestamp describing the length of the simulation. If it is set to 'none' the duration is determined from config\_start\_time and config\_stop\_time. config\_run\_duration overrides inconsistent values of config\_stop\_time. If a time value is specified for config\_run\_duration, it must be greater than 0. \\
	\hline
	\hyperref[subsec:nm_sec_config_calendar_type]{config\_calendar\_type} & Selection of the type of calendar that should be used in the simulation. \\
	\hline
\end{longtable}
\end{center}
}
\section[io]{\hyperref[sec:nm_sec_io]{io}}
\label{sec:nm_tab_io}
The io namelist record provides options for modifications to the I/O system of
MPAS. These include frequency, file name, and parallelization options.

{\small
\begin{center}
\begin{longtable}{| p{2.0in} || p{4.0in} |}
	\hline
	{\bf Name} & {\bf Description} \\
	\hline
	\hline
	\hyperref[subsec:nm_sec_config_input_name]{config\_input\_name} & The path to the input file for the simulation. \\
	\hline
	\hyperref[subsec:nm_sec_config_output_name]{config\_output\_name} & The template path and name to the output file from the simulation. A time stamp is prepended to the extension of the file (.nc). \\
	\hline
	\hyperref[subsec:nm_sec_config_restart_name]{config\_restart\_name} & The template path and name to the restart file for the simulation. A time stamp is prepended to the extension of the file (.nc) both for input and output. \\
	\hline
	\hyperref[subsec:nm_sec_config_restart_timestamp_name]{config\_restart\_timestamp\_name} & The name of the file to which the timestamp of the latest restart file is written. This file is subsequently used to set the start time when config\_start\_time is set to 'file' and config\_do\_restart is set to .true. \\
	\hline
	\hyperref[subsec:nm_sec_config_restart_interval]{config\_restart\_interval} & Timestamp determining how often a restart file should be written.  Currently years and months are not supported, so you have to specify the restart interval in units of days! **  We could eventually propose a change to framework to fix this in subroutine mpas\_set\_timeInterval in mpas\_timekeeping module. \\
	\hline
	\hyperref[subsec:nm_sec_config_output_interval]{config\_output\_interval} & Timestamp determining how often an output file should be written. \\
	\hline
	\hyperref[subsec:nm_sec_config_stats_interval]{config\_stats\_interval} & Timestamp determining how often a global statistics files should be written. \\
	\hline
	\hyperref[subsec:nm_sec_config_write_stats_on_startup]{config\_write\_stats\_on\_startup} & Logical flag determining if statistics files should be written prior to the first time step. \\
	\hline
	\hyperref[subsec:nm_sec_config_write_output_on_startup]{config\_write\_output\_on\_startup} & Logical flag determining if an output file should be written prior to the first time step. \\
	\hline
	\hyperref[subsec:nm_sec_config_frames_per_outfile]{config\_frames\_per\_outfile} & Integer specifying how many time frames should be included in an output file. Once the maximum is reached, a new output file is created.  If 0 (or less) is specified then all time frames are included in a single file called 'output.nc'. \\
	\hline
	\hyperref[subsec:nm_sec_config_pio_num_iotasks]{config\_pio\_num\_iotasks} & Integer specifying how many IO tasks should be used within the PIO library. A value of 0 causes all MPI tasks to also be IO tasks. IO tasks are required to write contiguous blocks of data to a file. \\
	\hline
	\hyperref[subsec:nm_sec_config_pio_stride]{config\_pio\_stride} & Integer specifying the stride of each IO task. \\
	\hline
\end{longtable}
\end{center}
}
\section[decomposition]{\hyperref[sec:nm_sec_decomposition]{decomposition}}
\label{sec:nm_tab_decomposition}
Namelist parameters for the \verb+decomposition+ namelist group.

{\small
\begin{center}
\begin{longtable}{| p{2.0in} || p{4.0in} |}
	\hline
	{\bf Name} & {\bf Description} \\
	\hline
	\hline
	\hyperref[subsec:nm_sec_config_num_halos]{config\_num\_halos} & Determines the number of halo cells extending from a blocks owned cells (Called the 0-Halo). The default of 3 is the minimum that can be used with monotonic advection. \\
	\hline
	\hyperref[subsec:nm_sec_config_block_decomp_file_prefix]{config\_block\_decomp\_file\_prefix} & Defines the prefix for the block decomposition file. Can include a path. The number of blocks is appended to the end of the prefix at run-time. \\
	\hline
	\hyperref[subsec:nm_sec_config_number_of_blocks]{config\_number\_of\_blocks} & Determines the number of blocks a simulation should be run with. If it is set to 0, the number of blocks is the same as the number of MPI tasks at run-time. \\
	\hline
	\hyperref[subsec:nm_sec_config_explicit_proc_decomp]{config\_explicit\_proc\_decomp} & Determines if an explicit processor decomposition should be used. This is only useful if multiple blocks per processor are used. \\
	\hline
	\hyperref[subsec:nm_sec_config_proc_decomp_file_prefix]{config\_proc\_decomp\_file\_prefix} & Defines the prefix for the processor decomposition file. This file is only read if config\_explicit\_proc\_decomp is .true. The number of processors is appended to the end of the prefix at run-time. \\
	\hline
\end{longtable}
\end{center}
}
\section[debug]{\hyperref[sec:nm_sec_debug]{debug}}
\label{sec:nm_tab_debug}
At run-time a user can enable debugging features within MPAS-Land Ice. 
Currently the only debug option is to print more detailed information about
thickness advection.
Potential future debug options would be to include disabling of any 
tendencies to help determine why an issue might
be happening; various checks on certain fields;
and the ability to prescribe both a thickness and velocity field at run-time
which are constant throughout a simulation. All options that control these
debugging features are specified within the debug namelist record.

{\small
\begin{center}
\begin{longtable}{| p{2.0in} || p{4.0in} |}
	\hline
	{\bf Name} & {\bf Description} \\
	\hline
	\hline
	\hyperref[subsec:nm_sec_config_print_thickness_advection_info]{config\_print\_thickness\_advection\_info} & Prints additional information about thickness advection. \\
	\hline
\end{longtable}
\end{center}
}
