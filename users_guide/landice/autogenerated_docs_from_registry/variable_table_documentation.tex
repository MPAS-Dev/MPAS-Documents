\chapter[Variable definitions]{\hyperref[chap:variable_sections]{Variable definitions}}
\label{chap:variable_tables}
Embedded links point to more detailed variable information in the appendix.
\section[mesh]{\hyperref[sec:var_sec_mesh]{mesh}}
\label{sec:var_tab_mesh}
The mesh data type contains a single time level. The fields inside the mesh
structure are not assumed to be time dependent. This data structure contains
fields that describe the mesh, and the connectivity of the mesh. Several of the
fields contained in this structure are shared throughout all MPAS dynamical
cores.

\vspace{0.5in}
{\small
\begin{center}
\begin{longtable}{| p{2.0in} | p{4.0in} |}
    \hline
    {\bf Name} & {\bf Description} \endfirsthead
    \hline 
    {\bf Name} & {\bf Description} (Continued) \endhead
    \hline
    \hyperref[subsec:var_sec_mesh_latCell]{latCell} & Latitude location of cell centers in radians. \\
    \hline
    \hyperref[subsec:var_sec_mesh_lonCell]{lonCell} & Longitude location of cell centers in radians. \\
    \hline
    \hyperref[subsec:var_sec_mesh_xCell]{xCell} & X Coordinate in cartesian space of cell centers. \\
    \hline
    \hyperref[subsec:var_sec_mesh_yCell]{yCell} & Y Coordinate in cartesian space of cell centers. \\
    \hline
    \hyperref[subsec:var_sec_mesh_zCell]{zCell} & Z Coordinate in cartesian space of cell centers. \\
    \hline
    \hyperref[subsec:var_sec_mesh_indexToCellID]{indexToCellID} & List of global cell IDs. \\
    \hline
    \hyperref[subsec:var_sec_mesh_latEdge]{latEdge} & Latitude location of edge midpoints in radians. \\
    \hline
    \hyperref[subsec:var_sec_mesh_lonEdge]{lonEdge} & Longitude location of edge midpoints in radians. \\
    \hline
    \hyperref[subsec:var_sec_mesh_xEdge]{xEdge} & X Coordinate in cartesian space of edge midpoints. \\
    \hline
    \hyperref[subsec:var_sec_mesh_yEdge]{yEdge} & Y Coordinate in cartesian space of edge midpoints. \\
    \hline
    \hyperref[subsec:var_sec_mesh_zEdge]{zEdge} & Z Coordinate in cartesian space of edge midpoints. \\
    \hline
    \hyperref[subsec:var_sec_mesh_indexToEdgeID]{indexToEdgeID} & List of global edge IDs. \\
    \hline
    \hyperref[subsec:var_sec_mesh_latVertex]{latVertex} & Latitude location of vertices in radians. \\
    \hline
    \hyperref[subsec:var_sec_mesh_lonVertex]{lonVertex} & Longitude location of vertices in radians. \\
    \hline
    \hyperref[subsec:var_sec_mesh_xVertex]{xVertex} & X Coordinate in cartesian space of vertices. \\
    \hline
    \hyperref[subsec:var_sec_mesh_yVertex]{yVertex} & Y Coordinate in cartesian space of vertices. \\
    \hline
    \hyperref[subsec:var_sec_mesh_zVertex]{zVertex} & Z Coordinate in cartesian space of vertices. \\
    \hline
    \hyperref[subsec:var_sec_mesh_indexToVertexID]{indexToVertexID} & List of global vertex IDs. \\
    \hline
    \hyperref[subsec:var_sec_mesh_nEdgesOnCell]{nEdgesOnCell} & Number of edges that border each cell. \\
    \hline
    \hyperref[subsec:var_sec_mesh_nEdgesOnEdge]{nEdgesOnEdge} & Number of edges that surround each of the cells that straddle each edge. These edges are used to reconstruct the tangential velocities. \\
    \hline
    \hyperref[subsec:var_sec_mesh_cellsOnEdge]{cellsOnEdge} & List of cells that straddle each edge. \\
    \hline
    \hyperref[subsec:var_sec_mesh_edgesOnCell]{edgesOnCell} & List of edges that border each cell. \\
    \hline
    \hyperref[subsec:var_sec_mesh_edgesOnEdge]{edgesOnEdge} & List of edges that border each of the cells that straddle each edge. \\
    \hline
    \hyperref[subsec:var_sec_mesh_cellsOnCell]{cellsOnCell} & List of cells that neighbor each cell. \\
    \hline
    \hyperref[subsec:var_sec_mesh_verticesOnCell]{verticesOnCell} & List of vertices that border each cell. \\
    \hline
    \hyperref[subsec:var_sec_mesh_verticesOnEdge]{verticesOnEdge} & List of vertices that straddle each edge. \\
    \hline
    \hyperref[subsec:var_sec_mesh_edgesOnVertex]{edgesOnVertex} & List of edges that share a vertex as an endpoint. \\
    \hline
    \hyperref[subsec:var_sec_mesh_cellsOnVertex]{cellsOnVertex} & List of cells that share a vertex. \\
    \hline
    \hyperref[subsec:var_sec_mesh_weightsOnEdge]{weightsOnEdge} & Reconstruction weights associated with each of the edgesOnEdge. \\
    \hline
    \hyperref[subsec:var_sec_mesh_dvEdge]{dvEdge} & Length of each edge, computed as the distance between verticesOnEdge. \\
    \hline
    \hyperref[subsec:var_sec_mesh_dcEdge]{dcEdge} & Length of each edge, computed as the distance between cellsOnEdge. \\
    \hline
    \hyperref[subsec:var_sec_mesh_angleEdge]{angleEdge} & Angle the edge normal makes with local eastward direction. \\
    \hline
    \hyperref[subsec:var_sec_mesh_areaCell]{areaCell} & Area of each cell in the primary grid. \\
    \hline
    \hyperref[subsec:var_sec_mesh_areaTriangle]{areaTriangle} & Area of each cell (triangle) in the dual grid. \\
    \hline
    \hyperref[subsec:var_sec_mesh_kiteAreasOnVertex]{kiteAreasOnVertex} & Area of the portions of each dual cell that are part of each cellsOnVertex. \\
    \hline
    \hyperref[subsec:var_sec_mesh_meshDensity]{meshDensity} & The value of the generating density function at each cell center. \\
    \hline
    \hyperref[subsec:var_sec_mesh_localVerticalUnitVectors]{localVerticalUnitVectors} & Unit surface normal vectors defined at cell centers. \\
    \hline
    \hyperref[subsec:var_sec_mesh_edgeNormalVectors]{edgeNormalVectors} & Normal vector defined at an edge. \\
    \hline
    \hyperref[subsec:var_sec_mesh_cellTangentPlane]{cellTangentPlane} & The two vectors that define a tangent plane at a cell center. \\
    \hline
    \hyperref[subsec:var_sec_mesh_coeffs_reconstruct]{coeffs\_reconstruct} & Coefficients to reconstruct velocity vectors at cell centers. \\
    \hline
    \hyperref[subsec:var_sec_mesh_layerThicknessFractions]{layerThicknessFractions} & Fractional thickness of each sigma layer \\
    \hline
    \hyperref[subsec:var_sec_mesh_layerCenterSigma]{layerCenterSigma} & Sigma (fractional) level at center of each layer \\
    \hline
    \hyperref[subsec:var_sec_mesh_layerInterfaceSigma]{layerInterfaceSigma} & Sigma (fractional) level at interface between each layer (including top and bottom) \\
    \hline
    \hyperref[subsec:var_sec_mesh_edgeSignOnCell]{edgeSignOnCell} & Sign of edge contributions to a cell for each edge on cell. Used for bit-reproducible loops. Represents directionality of vector connecting cells. \\
    \hline
    \hyperref[subsec:var_sec_mesh_edgeSignOnVertex]{edgeSignOnVertex} & Sign of edge contributions to a vertex for each edge on vertex. Used for bit-reproducible loops. Represents directionality of vector connecting vertices. \\
    \hline
    \hyperref[subsec:var_sec_mesh_cellProcID]{cellProcID} & processor number for each cell \\
    \hline
    \hyperref[subsec:var_sec_mesh_baryCellsOnVertex]{baryCellsOnVertex} & Cell center indices to use for interpolating from cell centers to vertex locations.  Note these are local indices! \\
    \hline
    \hyperref[subsec:var_sec_mesh_baryWeightsOnVertex]{baryWeightsOnVertex} & Weights to interpolate from cell centers to vertex locations.  Each weight is used with the corresponding cell center index indentified by baryCellsOnVertex. \\
    \hline
    \hyperref[subsec:var_sec_mesh_wachspressWeightVertex]{wachspressWeightVertex} & Wachspress weights used to interpolate from vertices to cell centers. \\
    \hline
    \hyperref[subsec:var_sec_mesh_xtime]{xtime} & model time, with format 'YYYY-MM-DD\_HH:MM:SS' \\
    \hline
    \hyperref[subsec:var_sec_mesh_deltat]{deltat} & time step length, in seconds.  Value on a given time is the value used between the previous time level and the current time level. \\
    \hline
    \hyperref[subsec:var_sec_mesh_allowableDtACFL]{allowableDtACFL} & The maximum allowable time step based on the advective CFL condition.  Value on a given time is the value appropriate for  between the previous time level and the current time level. \\
    \hline
    \hyperref[subsec:var_sec_mesh_allowableDtDCFL]{allowableDtDCFL} & The maximum allowable time step based on the diffusive CFL condition.  Value on a given time is the value appropriate for  between the previous time level and the current time level. \\
    \hline
    \hyperref[subsec:var_sec_mesh_simulationStartTime]{simulationStartTime} & start time of first simulation, with format 'YYYY-MM-DD\_HH:MM:SS' \\
    \hline
    \hyperref[subsec:var_sec_mesh_daysSinceStart]{daysSinceStart} & Time since simulationStartTime in days, for plotting \\
    \hline
    \hyperref[subsec:var_sec_mesh_timestepNumber]{timestepNumber} & time step number.  initial time is 0. \\
    \hline
\end{longtable}
\end{center}
}
\section[geometry]{\hyperref[sec:var_sec_geometry]{geometry}}
\label{sec:var_tab_geometry}
The geometry data structure contains fields related to ice sheet geometry.

\vspace{0.5in}
{\small
\begin{center}
\begin{longtable}{| p{2.0in} | p{4.0in} |}
    \hline
    {\bf Name} & {\bf Description} \endfirsthead
    \hline 
    {\bf Name} & {\bf Description} (Continued) \endhead
    \hline
    \hyperref[subsec:var_sec_geometry_bedTopography]{bedTopography} & Elevation of ice sheet bed.  Once isostasy is added to the model, this should become a state variable. \\
    \hline
    \hyperref[subsec:var_sec_geometry_thickness]{thickness} & ice thickness \\
    \hline
    \hyperref[subsec:var_sec_geometry_layerThickness]{layerThickness} & layer thickness \\
    \hline
    \hyperref[subsec:var_sec_geometry_lowerSurface]{lowerSurface} & elevation at bottom of ice \\
    \hline
    \hyperref[subsec:var_sec_geometry_upperSurface]{upperSurface} & elevation at top of ice \\
    \hline
    \hyperref[subsec:var_sec_geometry_layerThicknessEdge]{layerThicknessEdge} & layer thickness on cell edges \\
    \hline
    \hyperref[subsec:var_sec_geometry_dHdt]{dHdt} & diagnostic field of rate of thickness change with time (dH/dt). This includes all processes (flux divergence, SMB, BMB, calving, etc.) because it is calculated as the new thickness minus the old thickness divided by the time step. \\
    \hline
    \hyperref[subsec:var_sec_geometry_thicknessOld]{thicknessOld} & ice thickness from previous time level (only used to calculate thicknessTendency) \\
    \hline
    \hyperref[subsec:var_sec_geometry_dynamicThickening]{dynamicThickening} & diagnostic field of dynamic thickening rate (calculated as negative of flux divergence) \\
    \hline
    \hyperref[subsec:var_sec_geometry_cellMask]{cellMask} & bitmask indicating various properties about the ice sheet on cells.  cellMask only needs to be a restart field if config\_allow\_additional\_advance = false (to keep the mask of initial ice extent) \\
    \hline
    \hyperref[subsec:var_sec_geometry_edgeMask]{edgeMask} & bitmask indicating various properties about the ice sheet on edges. \\
    \hline
    \hyperref[subsec:var_sec_geometry_vertexMask]{vertexMask} & bitmask indicating various properties about the ice sheet on vertices. \\
    \hline
    \hyperref[subsec:var_sec_geometry_sfcMassBal]{sfcMassBal} & Surface mass balance \\
    \hline
    \hyperref[subsec:var_sec_geometry_basalMassBal]{basalMassBal} & Basal mass balance applied \\
    \hline
    \hyperref[subsec:var_sec_geometry_groundedBasalMassBal]{groundedBasalMassBal} & Basal mass balance on grounded regions \\
    \hline
    \hyperref[subsec:var_sec_geometry_floatingBasalMassBal]{floatingBasalMassBal} & Basal mass balance on floating regions \\
    \hline
    \hyperref[subsec:var_sec_geometry_calvingThickness]{calvingThickness} & thickness of ice that calves (less than or equal to ice thickness) \\
    \hline
    \hyperref[subsec:var_sec_geometry_eigencalvingParameter]{eigencalvingParameter} & proportionality constant K2+- used in eigencalving formulation \\
    \hline
    \hyperref[subsec:var_sec_geometry_calvingVelocity]{calvingVelocity} & rate of calving front retreat due to calving, represented as a velocity normal to the calving front (in the x-y plane).  This retreat rate is converted in the code to requiredCalvingVolumeRate. \\
    \hline
    \hyperref[subsec:var_sec_geometry_requiredCalvingVolumeRate]{requiredCalvingVolumeRate} & total volume of ice that needs to be removed based on eigencalving rate at this margin cell \\
    \hline
    \hyperref[subsec:var_sec_geometry_uncalvedVolume]{uncalvedVolume} & volume of ice that was left uncalved from required calving flux due to only applying flux over immediate neighbors (diagnostic field to assess if this limitation is a problem) \\
    \hline
    \hyperref[subsec:var_sec_geometry_basalWaterThickness]{basalWaterThickness} & thickness of basal water \\
    \hline
    \hyperref[subsec:var_sec_geometry_restoreThickness]{restoreThickness} & thickness of ice added to restore the calving front to its initial position \\
    \hline
    \hyperref[subsec:var_sec_geometry_normalSlopeEdge]{normalSlopeEdge} & normal surface slope on edges \\
    \hline
    \hyperref[subsec:var_sec_geometry_apparentDiffusivity]{apparentDiffusivity} & apparent diffusivity at cell centers \\
    \hline
    \hyperref[subsec:var_sec_geometry_upperSurfaceVertex]{upperSurfaceVertex} & elevation at top of ice on vertices \\
    \hline
    \hyperref[subsec:var_sec_geometry_tangentSlopeEdge]{tangentSlopeEdge} & tangent surface slope on edges \\
    \hline
    \hyperref[subsec:var_sec_geometry_slopeEdge]{slopeEdge} & surface slope magnitude on edges \\
    \hline
\end{longtable}
\end{center}
}
\section[velocity]{\hyperref[sec:var_sec_velocity]{velocity}}
\label{sec:var_tab_velocity}
The velocity data structure includes fields related to ice velocity and dynamics.

\vspace{0.5in}
{\small
\begin{center}
\begin{longtable}{| p{2.0in} | p{4.0in} |}
    \hline
    {\bf Name} & {\bf Description} \endfirsthead
    \hline 
    {\bf Name} & {\bf Description} (Continued) \endhead
    \hline
    \hyperref[subsec:var_sec_velocity_flowParamA]{flowParamA} & flow law parameter, A \\
    \hline
    \hyperref[subsec:var_sec_velocity_normalVelocity]{normalVelocity} & horizonal velocity, normal component to an edge, layer interface \\
    \hline
    \hyperref[subsec:var_sec_velocity_layerNormalVelocity]{layerNormalVelocity} & horizonal velocity, normal component to an edge, layer midpoint \\
    \hline
    \hyperref[subsec:var_sec_velocity_normalVelocityInitial]{normalVelocityInitial} & horizonal velocity, normal component to an edge, computed at initialization \\
    \hline
    \hyperref[subsec:var_sec_velocity_uReconstructX]{uReconstructX} & x-component of velocity reconstructed on cell centers.  Also, for higher-order dycores, on input: value of the x-component of velocity that should be applied where dirichletVelocityMask==1. \\
    \hline
    \hyperref[subsec:var_sec_velocity_uReconstructY]{uReconstructY} & y-component of velocity reconstructed on cell centers.    Also, for higher-order dycores, on input: value of the y-component of velocity that should be applied where dirichletVelocityMask==1. \\
    \hline
    \hyperref[subsec:var_sec_velocity_uReconstructZ]{uReconstructZ} & z-component of velocity reconstructed on cell centers \\
    \hline
    \hyperref[subsec:var_sec_velocity_uReconstructZonal]{uReconstructZonal} & zonal velocity reconstructed on cell centers \\
    \hline
    \hyperref[subsec:var_sec_velocity_uReconstructMeridional]{uReconstructMeridional} & meridional velocity reconstructed on cell centers \\
    \hline
    \hyperref[subsec:var_sec_velocity_surfaceSpeed]{surfaceSpeed} & ice surface speed reconstructed at cell centers \\
    \hline
    \hyperref[subsec:var_sec_velocity_basalSpeed]{basalSpeed} & ice basal speed reconstructed at cell centers \\
    \hline
    \hyperref[subsec:var_sec_velocity_beta]{beta} & higher-order basal traction parameter; NOTE: not SI units! \\
    \hline
    \hyperref[subsec:var_sec_velocity_betaSolve]{betaSolve} & higher-order basal traction parameter; NOTE: not SI units! \\
    \hline
    \hyperref[subsec:var_sec_velocity_exx]{exx} & x-component of strain rate \\
    \hline
    \hyperref[subsec:var_sec_velocity_eyy]{eyy} & y-component of strain rate \\
    \hline
    \hyperref[subsec:var_sec_velocity_exy]{exy} & shear component of strain rate \\
    \hline
    \hyperref[subsec:var_sec_velocity_eTheta]{eTheta} & orientation of principal strain rate \\
    \hline
    \hyperref[subsec:var_sec_velocity_eyx]{eyx} & shear component of strain rate \\
    \hline
    \hyperref[subsec:var_sec_velocity_eMax]{eMax} & magnitude of first principal strain rate \\
    \hline
    \hyperref[subsec:var_sec_velocity_eMin]{eMin} & magnitude of second principal strain rate \\
    \hline
    \hyperref[subsec:var_sec_velocity_anyDynamicVertexMaskChanged]{anyDynamicVertexMaskChanged} & flag needed by external velocity solvers that indicates if the region to solve on the block's domain has changed (treated as a logical) \\
    \hline
    \hyperref[subsec:var_sec_velocity_dirichletVelocityMask]{dirichletVelocityMask} & mask of where Dirichlet boundary conditions should be applied to the velocity solution.  1 means apply a Dirichlet boundary condition, 0 means do not. (higher-order dycores only) \\
    \hline
    \hyperref[subsec:var_sec_velocity_dirichletMaskChanged]{dirichletMaskChanged} & flag needed by external velocity solvers that indicates if the Dirichlet boundary condition mask has changed (treated as a logical) \\
    \hline
    \hyperref[subsec:var_sec_velocity_floatingEdges]{floatingEdges} & edges which are floating have a value of 1.  non floating edges have a value of 0. \\
    \hline
\end{longtable}
\end{center}
}
\section[observations]{\hyperref[sec:var_sec_observations]{observations}}
\label{sec:var_tab_observations}
The observations data structure includes fields related to observations of ice sheet state.
None of these are currently used internally by the model, but can be written out in Exodus format
to be used as input to Albany's optimization capability.

\vspace{0.5in}
{\small
\begin{center}
\begin{longtable}{| p{2.0in} | p{4.0in} |}
    \hline
    {\bf Name} & {\bf Description} \endfirsthead
    \hline 
    {\bf Name} & {\bf Description} (Continued) \endhead
    \hline
    \hyperref[subsec:var_sec_observations_observedSurfaceVelocityX]{observedSurfaceVelocityX} & X-component of observed surface velocity \\
    \hline
    \hyperref[subsec:var_sec_observations_observedSurfaceVelocityY]{observedSurfaceVelocityY} & Y-component of observed surface velocity \\
    \hline
    \hyperref[subsec:var_sec_observations_observedSurfaceVelocityUncertainty]{observedSurfaceVelocity\-Uncertainty} & uncertainty in observed surface velocity \\
    \hline
    \hyperref[subsec:var_sec_observations_observedThicknessTendency]{observedThicknessTendency} & observed tendency in thickness (dH/dt) \\
    \hline
    \hyperref[subsec:var_sec_observations_observedThicknessTendencyUncertainty]{observedThicknessTendency\-Uncertainty} & uncertainty in observed tendency in thickness (dH/dt) \\
    \hline
    \hyperref[subsec:var_sec_observations_sfcMassBalUncertainty]{sfcMassBalUncertainty} & uncertainty in observed surface mass balance \\
    \hline
    \hyperref[subsec:var_sec_observations_thicknessUncertainty]{thicknessUncertainty} & uncertainty in observed thickness \\
    \hline
    \hyperref[subsec:var_sec_observations_floatingBasalMassBalUncertainty]{floatingBasalMassBalUncertainty} & uncertainty in observed floating basal mass balance \\
    \hline
\end{longtable}
\end{center}
}
\section[thermal]{\hyperref[sec:var_sec_thermal]{thermal}}
\label{sec:var_tab_thermal}
The thermal data structure includes fields related to ice temperature and thermodynamics.

\vspace{0.5in}
{\small
\begin{center}
\begin{longtable}{| p{2.0in} | p{4.0in} |}
    \hline
    {\bf Name} & {\bf Description} \endfirsthead
    \hline 
    {\bf Name} & {\bf Description} (Continued) \endhead
    \hline
    \hyperref[subsec:var_sec_thermal_temperature]{temperature} & interior ice temperature \\
    \hline
    \hyperref[subsec:var_sec_thermal_waterfrac]{waterfrac} & interior water fraction \\
    \hline
    \hyperref[subsec:var_sec_thermal_enthalpy]{enthalpy} & interior ice enthalpy \\
    \hline
    \hyperref[subsec:var_sec_thermal_surfaceAirTemperature]{surfaceAirTemperature} & surface air temperature \\
    \hline
    \hyperref[subsec:var_sec_thermal_surfaceTemperature]{surfaceTemperature} & upper surface ice temperature \\
    \hline
    \hyperref[subsec:var_sec_thermal_basalTemperature]{basalTemperature} & lower surface ice temperature \\
    \hline
    \hyperref[subsec:var_sec_thermal_pmpTemperature]{pmpTemperature} & pressure melt temperature \\
    \hline
    \hyperref[subsec:var_sec_thermal_basalPmpTemperature]{basalPmpTemperature} & lower surface pressure melt temperature \\
    \hline
    \hyperref[subsec:var_sec_thermal_surfaceConductiveFlux]{surfaceConductiveFlux} & conductive heat flux at the upper surface (positive downward) \\
    \hline
    \hyperref[subsec:var_sec_thermal_basalConductiveFlux]{basalConductiveFlux} & conductive heat flux at the lower surface (positive downward) \\
    \hline
    \hyperref[subsec:var_sec_thermal_basalHeatFlux]{basalHeatFlux} & basal heat flux into the ice (positive upward) \\
    \hline
    \hyperref[subsec:var_sec_thermal_basalFrictionFlux]{basalFrictionFlux} & basal frictional flux into the ice (positive upward) \\
    \hline
    \hyperref[subsec:var_sec_thermal_heatDissipation]{heatDissipation} & interior heat dissipation rate, divided by rhoi*cp\_ice \\
    \hline
\end{longtable}
\end{center}
}
\section[scratch]{\hyperref[sec:var_sec_scratch]{scratch}}
\label{sec:var_tab_scratch}
The scratch data structure includes reusable fields that are used as temporary arrays within the code.

\vspace{0.5in}
{\small
\begin{center}
\begin{longtable}{| p{2.0in} | p{4.0in} |}
    \hline
    {\bf Name} & {\bf Description} \endfirsthead
    \hline 
    {\bf Name} & {\bf Description} (Continued) \endhead
    \hline
    \hyperref[subsec:var_sec_scratch_iceCellMask]{iceCellMask} & mask set to 1 in cells where some criterion is satisfied and 0 otherwise \\
    \hline
    \hyperref[subsec:var_sec_scratch_iceCellMask2]{iceCellMask2} & mask set to 1 in cells where some criterion is satisfied and 0 otherwise \\
    \hline
    \hyperref[subsec:var_sec_scratch_iceCellMask3]{iceCellMask3} & mask set to 1 in cells where some criterion is satisfied and 0 otherwise \\
    \hline
    \hyperref[subsec:var_sec_scratch_iceEdgeMask]{iceEdgeMask} & mask set to 1 for edges adjacent to ice-covered cells and 0 otherwise \\
    \hline
    \hyperref[subsec:var_sec_scratch_workLevelCell]{workLevelCell} & generic work array with dimensions of (nVertLevels nCells) \\
    \hline
    \hyperref[subsec:var_sec_scratch_workLevelEdge]{workLevelEdge} & generic work array with dimensions of (nVertLevels nEdges) \\
    \hline
    \hyperref[subsec:var_sec_scratch_workLevelVertex]{workLevelVertex} & generic work array with dimensions of (nVertLevels nVertices) \\
    \hline
    \hyperref[subsec:var_sec_scratch_workCell]{workCell} & generic work array with dimensions of (nCells) \\
    \hline
    \hyperref[subsec:var_sec_scratch_workCell2]{workCell2} & generic work array with dimensions of (nCells) \\
    \hline
    \hyperref[subsec:var_sec_scratch_workCell3]{workCell3} & generic work array with dimensions of (nCells) \\
    \hline
    \hyperref[subsec:var_sec_scratch_workTracerCell]{workTracerCell} & generic work array with dimensions of (maxTracersAdvect nCells) \\
    \hline
    \hyperref[subsec:var_sec_scratch_workTracerCell2]{workTracerCell2} & generic work array with dimensions of (maxTracersAdvect nCells) \\
    \hline
    \hyperref[subsec:var_sec_scratch_workTracerLevelCell]{workTracerLevelCell} & generic work array with dimensions of (maxTracersAdvect nVertLevels nCells) \\
    \hline
    \hyperref[subsec:var_sec_scratch_workTracerLevelCell2]{workTracerLevelCell2} & generic work array with dimensions of (maxTracersAdvect nVertLevels nCells) \\
    \hline
    \hyperref[subsec:var_sec_scratch_slopeCellX]{slopeCellX} & {\bf \color{red} MISSING} \\
    \hline
    \hyperref[subsec:var_sec_scratch_slopeCellY]{slopeCellY} & {\bf \color{red} MISSING} \\
    \hline
    \hyperref[subsec:var_sec_scratch_vertexIndices]{vertexIndices} & local indices of each vertex \\
    \hline
\end{longtable}
\end{center}
}
\section[regions]{\hyperref[sec:var_sec_regions]{regions}}
\label{sec:var_tab_regions}
The regions data structure includes fields related to regions defined for use with regional statistics analysis members.

\vspace{0.5in}
{\small
\begin{center}
\begin{longtable}{| p{2.0in} | p{4.0in} |}
    \hline
    {\bf Name} & {\bf Description} \endfirsthead
    \hline 
    {\bf Name} & {\bf Description} (Continued) \endhead
    \hline
    \hyperref[subsec:var_sec_regions_regionCellMasks]{regionCellMasks} & masks set to 1 in cells that fall within a given region and 0 otherwise \\
    \hline
\end{longtable}
\end{center}
}
\section[hydro]{\hyperref[sec:var_sec_hydro]{hydro}}
\label{sec:var_tab_hydro}
The hydro data structure includes fields related to the subglacial hydrology model.

\vspace{0.5in}
{\small
\begin{center}
\begin{longtable}{| p{2.0in} | p{4.0in} |}
    \hline
    {\bf Name} & {\bf Description} \endfirsthead
    \hline 
    {\bf Name} & {\bf Description} (Continued) \endhead
    \hline
    \hyperref[subsec:var_sec_hydro_waterThickness]{waterThickness} & water layer thickness in subglacial system \\
    \hline
    \hyperref[subsec:var_sec_hydro_waterThicknessOld]{waterThicknessOld} & water layer thickness in subglacial system from previous time step \\
    \hline
    \hyperref[subsec:var_sec_hydro_waterThicknessTendency]{waterThicknessTendency} & rate of change in water layer thickness in subglacial system \\
    \hline
    \hyperref[subsec:var_sec_hydro_tillWaterThickness]{tillWaterThickness} & water layer thickness in subglacial till \\
    \hline
    \hyperref[subsec:var_sec_hydro_tillWaterThicknessOld]{tillWaterThicknessOld} & water layer thickness in subglacial till from previous time step \\
    \hline
    \hyperref[subsec:var_sec_hydro_waterPressure]{waterPressure} & pressure in subglacial system \\
    \hline
    \hyperref[subsec:var_sec_hydro_waterPressureOld]{waterPressureOld} & pressure in subglacial system from previous time step \\
    \hline
    \hyperref[subsec:var_sec_hydro_waterPressureTendency]{waterPressureTendency} & tendency in pressure in subglacial system \\
    \hline
    \hyperref[subsec:var_sec_hydro_basalMeltInput]{basalMeltInput} & basal meltwater input \\
    \hline
    \hyperref[subsec:var_sec_hydro_externalWaterInput]{externalWaterInput} & external water input \\
    \hline
    \hyperref[subsec:var_sec_hydro_frictionAngle]{frictionAngle} & till friction angle \\
    \hline
    \hyperref[subsec:var_sec_hydro_effectivePressure]{effectivePressure} & effective ice pressure in subglacial system \\
    \hline
    \hyperref[subsec:var_sec_hydro_hydropotential]{hydropotential} & subglacial hydropotential \\
    \hline
    \hyperref[subsec:var_sec_hydro_waterFlux]{waterFlux} & total subglacial water flux \\
    \hline
    \hyperref[subsec:var_sec_hydro_waterFluxMask]{waterFluxMask} & mask indicating how to handle fluxes on each edge: 0=calculate based on hydropotential gradient; 1=allow outflow based on hydropotential gradient, but no inflow (NOT YET IMPLEMENTED); 2=zero flux \\
    \hline
    \hyperref[subsec:var_sec_hydro_waterFluxAdvec]{waterFluxAdvec} & advective subglacial water flux \\
    \hline
    \hyperref[subsec:var_sec_hydro_waterFluxDiffu]{waterFluxDiffu} & diffusive subglacial water flux \\
    \hline
    \hyperref[subsec:var_sec_hydro_waterVelocity]{waterVelocity} & subglacial water velocity \\
    \hline
    \hyperref[subsec:var_sec_hydro_waterVelocityCellX]{waterVelocityCellX} & subglacial water velocity reconstructed on cell centers, x-component \\
    \hline
    \hyperref[subsec:var_sec_hydro_waterVelocityCellY]{waterVelocityCellY} & subglacial water velocity reconstructed on cell centers, y-component \\
    \hline
    \hyperref[subsec:var_sec_hydro_effectiveConducEdge]{effectiveConducEdge} & effective Darcy hydraulic conductivity on edges \\
    \hline
    \hyperref[subsec:var_sec_hydro_waterThicknessEdge]{waterThicknessEdge} & water layer thickness on edges \\
    \hline
    \hyperref[subsec:var_sec_hydro_waterThicknessEdgeUpwind]{waterThicknessEdgeUpwind} & water layer thickness of cell upwind of edge \\
    \hline
    \hyperref[subsec:var_sec_hydro_diffusivity]{diffusivity} & diffusivity of sheet \\
    \hline
    \hyperref[subsec:var_sec_hydro_hydropotentialBase]{hydropotentialBase} & hydropotential without water thickness contribution \\
    \hline
    \hyperref[subsec:var_sec_hydro_hydropotentialBaseVertex]{hydropotentialBaseVertex} & hydropotential without water thickness contribution on vertices.  Only used for some choices of config\_SGH\_tangent\_slope\_calculation. \\
    \hline
    \hyperref[subsec:var_sec_hydro_hydropotentialBaseSlopeNormal]{hydropotentialBaseSlopeNormal} & normal component of gradient of hydropotentialBase \\
    \hline
    \hyperref[subsec:var_sec_hydro_hydropotentialBaseSlopeTangent]{hydropotentialBaseSlopeTangent} & tangent component of gradient of hydropotentialBase \\
    \hline
    \hyperref[subsec:var_sec_hydro_gradMagPhiEdge]{gradMagPhiEdge} & magnitude of the gradient of hydropotentialBase, on Edges \\
    \hline
    \hyperref[subsec:var_sec_hydro_waterPressureSlopeNormal]{waterPressureSlopeNormal} & normal component of gradient of waterPressure \\
    \hline
    \hyperref[subsec:var_sec_hydro_divergence]{divergence} & flux divergence of water \\
    \hline
    \hyperref[subsec:var_sec_hydro_openingRate]{openingRate} & rate of cavity opening \\
    \hline
    \hyperref[subsec:var_sec_hydro_closingRate]{closingRate} & rate of creep closure \\
    \hline
    \hyperref[subsec:var_sec_hydro_zeroOrderSum]{zeroOrderSum} & sum of zero order terms \\
    \hline
    \hyperref[subsec:var_sec_hydro_deltatSGHadvec]{deltatSGHadvec} & advective CFL limited time step length \\
    \hline
    \hyperref[subsec:var_sec_hydro_deltatSGHdiffu]{deltatSGHdiffu} & diffusive CFL limited time step length \\
    \hline
    \hyperref[subsec:var_sec_hydro_deltatSGHpressure]{deltatSGHpressure} & time step length limited by pressure equation scheme \\
    \hline
    \hyperref[subsec:var_sec_hydro_deltatSGH]{deltatSGH} & time step used for subglacial hydro \\
    \hline
    \hyperref[subsec:var_sec_hydro_channelArea]{channelArea} & area of subglacial channel \\
    \hline
    \hyperref[subsec:var_sec_hydro_channelDischarge]{channelDischarge} & discharge through subglacial channel \\
    \hline
    \hyperref[subsec:var_sec_hydro_channelVelocity]{channelVelocity} & velocity in subglacial channel \\
    \hline
    \hyperref[subsec:var_sec_hydro_channelMelt]{channelMelt} & melt rate in subglacial channel \\
    \hline
    \hyperref[subsec:var_sec_hydro_channelPressureFreeze]{channelPressureFreeze} & freezing rate in subglacial channel due to water pressure gradient.  positive=freezing, negative=melting \\
    \hline
    \hyperref[subsec:var_sec_hydro_flowParamAChannel]{flowParamAChannel} & flow parameter A on edges used for channel \\
    \hline
    \hyperref[subsec:var_sec_hydro_channelEffectivePressure]{channelEffectivePressure} & effective pressure in the channel \\
    \hline
    \hyperref[subsec:var_sec_hydro_channelClosingRate]{channelClosingRate} & closing rate from creep of the channel \\
    \hline
    \hyperref[subsec:var_sec_hydro_channelOpeningRate]{channelOpeningRate} & opening rate from melt of the channel \\
    \hline
    \hyperref[subsec:var_sec_hydro_channelChangeRate]{channelChangeRate} & rate of change of channel area \\
    \hline
    \hyperref[subsec:var_sec_hydro_deltatSGHadvecChannel]{deltatSGHadvecChannel} & time step length limited by channel advection \\
    \hline
    \hyperref[subsec:var_sec_hydro_deltatSGHdiffuChannel]{deltatSGHdiffuChannel} & time step length limited by channel diffusion \\
    \hline
    \hyperref[subsec:var_sec_hydro_divergenceChannel]{divergenceChannel} & divergence due to channel flow \\
    \hline
    \hyperref[subsec:var_sec_hydro_channelAreaChangeCell]{channelAreaChangeCell} & change in channel area within each cell, averaged over cell area \\
    \hline
    \hyperref[subsec:var_sec_hydro_channelDiffusivity]{channelDiffusivity} & diffusivity in channel \\
    \hline
\end{longtable}
\end{center}
}
\section[globalStatsAM]{\hyperref[sec:var_sec_globalStatsAM]{globalStatsAM}}
\label{sec:var_tab_globalStatsAM}
The globalStatsAM data structure includes fields related to the global statistics analysis members.

\vspace{0.5in}
{\small
\begin{center}
\begin{longtable}{| p{2.0in} | p{4.0in} |}
    \hline
    {\bf Name} & {\bf Description} \endfirsthead
    \hline 
    {\bf Name} & {\bf Description} (Continued) \endhead
    \hline
    \hyperref[subsec:var_sec_globalStatsAM_totalIceVolume]{totalIceVolume} & total ice sheet volume \\
    \hline
    \hyperref[subsec:var_sec_globalStatsAM_volumeAboveFloatation]{volumeAboveFloatation} & total ice sheet volume above floatation \\
    \hline
    \hyperref[subsec:var_sec_globalStatsAM_totalIceArea]{totalIceArea} & total ice sheet area \\
    \hline
    \hyperref[subsec:var_sec_globalStatsAM_floatingIceVolume]{floatingIceVolume} & total floating ice sheet volume \\
    \hline
    \hyperref[subsec:var_sec_globalStatsAM_floatingIceArea]{floatingIceArea} & total floating ice sheet area \\
    \hline
    \hyperref[subsec:var_sec_globalStatsAM_groundedIceVolume]{groundedIceVolume} & total grounded ice sheet volume \\
    \hline
    \hyperref[subsec:var_sec_globalStatsAM_groundedIceArea]{groundedIceArea} & total grounded ice sheet area \\
    \hline
    \hyperref[subsec:var_sec_globalStatsAM_iceThicknessMean]{iceThicknessMean} & spatially averaged ice thickness \\
    \hline
    \hyperref[subsec:var_sec_globalStatsAM_iceThicknessMax]{iceThicknessMax} & maximum ice thickness in domain \\
    \hline
    \hyperref[subsec:var_sec_globalStatsAM_iceThicknessMin]{iceThicknessMin} & minimum ice thickness in domain \\
    \hline
    \hyperref[subsec:var_sec_globalStatsAM_totalSfcMassBal]{totalSfcMassBal} & total, area integrated surface mass balance. Positive values are ice gain. \\
    \hline
    \hyperref[subsec:var_sec_globalStatsAM_avgNetAccumulation]{avgNetAccumulation} & average sfcMassBal, as a thickness rate. Positive values are ice gain. \\
    \hline
    \hyperref[subsec:var_sec_globalStatsAM_totalBasalMassBal]{totalBasalMassBal} & total, area integrated basal mass balance. Positive values are ice gain. \\
    \hline
    \hyperref[subsec:var_sec_globalStatsAM_totalGroundedBasalMassBal]{totalGroundedBasalMassBal} & total, area integrated grounded basal mass balance. Positive values are ice gain. \\
    \hline
    \hyperref[subsec:var_sec_globalStatsAM_avgGroundedBasalMelt]{avgGroundedBasalMelt} & average groundedBasalMassBal value, as a thickness rate. Positive values are ice loss. \\
    \hline
    \hyperref[subsec:var_sec_globalStatsAM_totalFloatingBasalMassBal]{totalFloatingBasalMassBal} & total, area integrated floating basal mass balance. Positive values are ice gain. \\
    \hline
    \hyperref[subsec:var_sec_globalStatsAM_avgSubshelfMelt]{avgSubshelfMelt} & average floatingBasalMassBal value, as a thickness rate. Positive values are ice loss. \\
    \hline
    \hyperref[subsec:var_sec_globalStatsAM_totalCalvingFlux]{totalCalvingFlux} & total, area integrated mass loss due to calving. Positive values are ice loss. \\
    \hline
    \hyperref[subsec:var_sec_globalStatsAM_groundingLineFlux]{groundingLineFlux} & total mass flux across all grounding lines.  Note that flux from floating to grounded ice makes a negative contribution to this metric. \\
    \hline
    \hyperref[subsec:var_sec_globalStatsAM_surfaceSpeedMax]{surfaceSpeedMax} & maximum surface speed in the domain \\
    \hline
    \hyperref[subsec:var_sec_globalStatsAM_basalSpeedMax]{basalSpeedMax} & maximum basal speed in the domain \\
    \hline
\end{longtable}
\end{center}
}
\section[regionalStatsAM]{\hyperref[sec:var_sec_regionalStatsAM]{regionalStatsAM}}
\label{sec:var_tab_regionalStatsAM}
The regionalStatsAM data structure includes fields related to the regional statistics analysis members.

\vspace{0.5in}
{\small
\begin{center}
\begin{longtable}{| p{2.0in} | p{4.0in} |}
    \hline
    {\bf Name} & {\bf Description} \endfirsthead
    \hline 
    {\bf Name} & {\bf Description} (Continued) \endhead
    \hline
    \hyperref[subsec:var_sec_regionalStatsAM_regionalIceArea]{regionalIceArea} & total ice sheet area within region \\
    \hline
    \hyperref[subsec:var_sec_regionalStatsAM_regionalIceVolume]{regionalIceVolume} & total ice sheet volume within region \\
    \hline
    \hyperref[subsec:var_sec_regionalStatsAM_regionalVolumeAboveFloatation]{regionalVolumeAboveFloatation} & total ice sheet volume above floatation \\
    \hline
    \hyperref[subsec:var_sec_regionalStatsAM_regionalGroundedIceArea]{regionalGroundedIceArea} & total grounded ice sheet area within region \\
    \hline
    \hyperref[subsec:var_sec_regionalStatsAM_regionalGroundedIceVolume]{regionalGroundedIceVolume} & total grounded ice sheet volume within region \\
    \hline
    \hyperref[subsec:var_sec_regionalStatsAM_regionalFloatingIceArea]{regionalFloatingIceArea} & total floating ice sheet area within region \\
    \hline
    \hyperref[subsec:var_sec_regionalStatsAM_regionalFloatingIceVolume]{regionalFloatingIceVolume} & total floating ice sheet volume within region \\
    \hline
    \hyperref[subsec:var_sec_regionalStatsAM_regionalIceThicknessMin]{regionalIceThicknessMin} & min ice thickness within region \\
    \hline
    \hyperref[subsec:var_sec_regionalStatsAM_regionalIceThicknessMax]{regionalIceThicknessMax} & max ice thickness within region \\
    \hline
    \hyperref[subsec:var_sec_regionalStatsAM_regionalIceThicknessMean]{regionalIceThicknessMean} & mean ice thickness within region \\
    \hline
    \hyperref[subsec:var_sec_regionalStatsAM_regionalSumSfcMassBal]{regionalSumSfcMassBal} & area-integrated surface mass balance within region \\
    \hline
    \hyperref[subsec:var_sec_regionalStatsAM_regionalAvgNetAccumulation]{regionalAvgNetAccumulation} & average sfcMassBal, as a thickness rate. Positive values are ice gain. \\
    \hline
    \hyperref[subsec:var_sec_regionalStatsAM_regionalSumBasalMassBal]{regionalSumBasalMassBal} & area-integrated basal mass balance within region \\
    \hline
    \hyperref[subsec:var_sec_regionalStatsAM_regionalSumGroundedBasalMassBal]{regionalSumGroundedBasalMass\-Bal} & total, area integrated grounded basal mass balance. Positive values are ice gain. \\
    \hline
    \hyperref[subsec:var_sec_regionalStatsAM_regionalAvgGroundedBasalMelt]{regionalAvgGroundedBasalMelt} & average groundedBasalMassBal value, as a thickness rate. Positive values are ice loss. \\
    \hline
    \hyperref[subsec:var_sec_regionalStatsAM_regionalSumFloatingBasalMassBal]{regionalSumFloatingBasalMass\-Bal} & total, area integrated floating basal mass balance. Positive values are ice gain. \\
    \hline
    \hyperref[subsec:var_sec_regionalStatsAM_regionalAvgSubshelfMelt]{regionalAvgSubshelfMelt} & average floatingBasalMassBal value, as a thickness rate. Positive values are ice loss. \\
    \hline
    \hyperref[subsec:var_sec_regionalStatsAM_regionalSumCalvingFlux]{regionalSumCalvingFlux} & area-integrated calving flux within region \\
    \hline
    \hyperref[subsec:var_sec_regionalStatsAM_regionalSumGroundingLineFlux]{regionalSumGroundingLineFlux} & total mass flux across all grounding lines (note that flux from floating to grounded ice makes a negative contribution to this metric) \\
    \hline
    \hyperref[subsec:var_sec_regionalStatsAM_regionalSurfaceSpeedMax]{regionalSurfaceSpeedMax} & maximum surface speed in the domain \\
    \hline
    \hyperref[subsec:var_sec_regionalStatsAM_regionalBasalSpeedMax]{regionalBasalSpeedMax} & maximum basal speed in the domain \\
    \hline
\end{longtable}
\end{center}
}
