\chapter{MPAS-Land Ice Quick Start Guide}
\label{chap:quick_start}

This chapter provides MPAS-Land Ice users with a quick start description of how to
build and run the model. It is meant merely as a brief overview of the process,
while the more detailed descriptions of each step are provided in later
sections.

In general, the build process follows the following steps.

\begin{enumerate}
	\item Build MPI Layer (OpenMPI, MVAPICH2, etc.)
	\item Build serial NetCDF library (v3.6.3, v4.1.3, etc.)
	\item Build Parallel-NetCDF library (v1.2.1, v1.3.0, etc.)
	\item Build Parallel I/O library (v1.4.1, v1.6.1, etc.)
	\item (Optional) Build METIS library and executables (v4.0, v5.0.2, etc.)
	\item Clone MPAS-Land Ice from repository
	\item Build Land Ice core
\end{enumerate}

After step 7, an executable should be created called {\tt landice\_model.exe}. Once the executable is built, one can begin the run process as follows:

\begin{enumerate}
	\item Create run directory.
	\item Copy executable to run directory.
	\item Copy {\tt namelist.landice} and {\tt streams.landice} from the {\tt default\_inputs} directory into run directory.
	\item (Optional) Copy input and graph files into run directory.
	\item Edit {\tt namelist.input} to have the proper parameters. \\
		  If step 4 was skipped, ensure paths to input and graph files are appropriately set.
	\item (Optional) Create graph files, using METIS executable (pmetis or gpmetis depending on version). \\
		  A graph file is required for each processor count you want to use greater than one processor.
	\item Run MPAS-Land Ice.
	\item Visualize output file, and perform analysis.
\end{enumerate}
