\chapter[Variable definitions]{\hyperref[chap:variable_sections]{Variable definitions}}
\label{chap:variable_tables}
Embedded links point to more detailed variable information in the appendix.
\section[state]{\hyperref[sec:var_sec_state]{state}}
\label{sec:var_tab_state}
The state data structure contains a set of prognostic and diagnostic fields
that are time dependent. The fields contained inside of state have two time
levels available.  (More than two is possible within the MPAS Framework, but 
only two have been implemented in the Land Ice core.)

{\small
\begin{center}
\begin{longtable}{| p{2.0in} | p{4.0in} |}
	\hline
	{\bf Name} & {\bf Description} \\
	\hline
	\hyperref[subsec:var_sec_state_xtime]{xtime} & model time, with format 'YYYY-MM-DD\_HH:MM:SS' \\
	\hline
	\hyperref[subsec:var_sec_state_thickness]{thickness} & ice thickness \\
	\hline
	\hyperref[subsec:var_sec_state_layerThickness]{layerThickness} & layer thickness \\
	\hline
	\hyperref[subsec:var_sec_state_temperature]{temperature} & ice temperature \\
	\hline
	\hyperref[subsec:var_sec_state_lowerSurface]{lowerSurface} & elevation at bottom of ice \\
	\hline
	\hyperref[subsec:var_sec_state_upperSurface]{upperSurface} & elevation at top of ice \\
	\hline
	\hyperref[subsec:var_sec_state_layerThicknessEdge]{layerThicknessEdge} & layer thickness on cell edges \\
	\hline
	\hyperref[subsec:var_sec_state_cellMask]{cellMask} & bitmask indicating various properties about the ice sheet on cells.  cellMask only needs to be a restart field if config\_allow\_additional\_advance = false (to keep the mask of initial ice extent) \\
	\hline
	\hyperref[subsec:var_sec_state_edgeMask]{edgeMask} & bitmask indicating various properties about the ice sheet on edges. \\
	\hline
	\hyperref[subsec:var_sec_state_vertexMask]{vertexMask} & bitmask indicating various properties about the ice sheet on vertices. \\
	\hline
	\hyperref[subsec:var_sec_state_normalVelocity]{normalVelocity} & horizonal velocity, normal component to an edge \\
	\hline
	\hyperref[subsec:var_sec_state_uReconstructX]{uReconstructX} & x-component of velocity reconstructed on cell centers \\
	\hline
	\hyperref[subsec:var_sec_state_uReconstructY]{uReconstructY} & y-component of velocity reconstructed on cell centers \\
	\hline
	\hyperref[subsec:var_sec_state_uReconstructZ]{uReconstructZ} & z-component of velocity reconstructed on cell centers \\
	\hline
	\hyperref[subsec:var_sec_state_uReconstructZonal]{uReconstructZonal} & zonal velocity reconstructed on cell centers \\
	\hline
	\hyperref[subsec:var_sec_state_uReconstructMeridional]{uReconstructMeridional} & meridional velocity reconstructed on cell centers \\
	\hline
\end{longtable}
\end{center}
}
\section[tend]{\hyperref[sec:var_sec_tend]{tend}}
\label{sec:var_tab_tend}
The tend data structure represents the tendencies used to time step the
prognostic variables within the state structure. 

{\small
\begin{center}
\begin{longtable}{| p{2.0in} | p{4.0in} |}
	\hline
	{\bf Name} & {\bf Description} \\
	\hline
	\hyperref[subsec:var_sec_tend_tend_layerThickness]{tend\_layerThickness} & time tendency of layer thickness \\
	\hline
	\hyperref[subsec:var_sec_tend_tend_temperature]{tend\_temperature} & time tendency of ice temperature \\
	\hline
\end{longtable}
\end{center}
}
\section[mesh]{\hyperref[sec:var_sec_mesh]{mesh}}
\label{sec:var_tab_mesh}
The mesh data type contains a single time level. The fields inside the mesh
structure are not assumed to be time dependent. This data structure contains
fields that describe the mesh, and the connectivity of the mesh. Several of the
fields contained in this structure are shared throughout all MPAS dynamical
cores.

{\small
\begin{center}
\begin{longtable}{| p{2.0in} | p{4.0in} |}
	\hline
	{\bf Name} & {\bf Description} \\
	\hline
	\hyperref[subsec:var_sec_mesh_latCell]{latCell} & Latitude location of cell centers in radians. \\
	\hline
	\hyperref[subsec:var_sec_mesh_lonCell]{lonCell} & Longitude location of cell centers in radians. \\
	\hline
	\hyperref[subsec:var_sec_mesh_xCell]{xCell} & X Coordinate in cartesian space of cell centers. \\
	\hline
	\hyperref[subsec:var_sec_mesh_yCell]{yCell} & Y Coordinate in cartesian space of cell centers. \\
	\hline
	\hyperref[subsec:var_sec_mesh_zCell]{zCell} & Z Coordinate in cartesian space of cell centers. \\
	\hline
	\hyperref[subsec:var_sec_mesh_indexToCellID]{indexToCellID} & List of global cell IDs. \\
	\hline
	\hyperref[subsec:var_sec_mesh_latEdge]{latEdge} & Latitude location of edge midpoints in radians. \\
	\hline
	\hyperref[subsec:var_sec_mesh_lonEdge]{lonEdge} & Longitude location of edge midpoints in radians. \\
	\hline
	\hyperref[subsec:var_sec_mesh_xEdge]{xEdge} & X Coordinate in cartesian space of edge midpoints. \\
	\hline
	\hyperref[subsec:var_sec_mesh_yEdge]{yEdge} & Y Coordinate in cartesian space of edge midpoints. \\
	\hline
	\hyperref[subsec:var_sec_mesh_zEdge]{zEdge} & Z Coordinate in cartesian space of edge midpoints. \\
	\hline
	\hyperref[subsec:var_sec_mesh_indexToEdgeID]{indexToEdgeID} & List of global edge IDs. \\
	\hline
	\hyperref[subsec:var_sec_mesh_latVertex]{latVertex} & Latitude location of vertices in radians. \\
	\hline
	\hyperref[subsec:var_sec_mesh_lonVertex]{lonVertex} & Longitude location of vertices in radians. \\
	\hline
	\hyperref[subsec:var_sec_mesh_xVertex]{xVertex} & X Coordinate in cartesian space of vertices. \\
	\hline
	\hyperref[subsec:var_sec_mesh_yVertex]{yVertex} & Y Coordinate in cartesian space of vertices. \\
	\hline
	\hyperref[subsec:var_sec_mesh_zVertex]{zVertex} & Z Coordinate in cartesian space of vertices. \\
	\hline
	\hyperref[subsec:var_sec_mesh_indexToVertexID]{indexToVertexID} & List of global vertex IDs. \\
	\hline
	\hyperref[subsec:var_sec_mesh_cellsOnEdge]{cellsOnEdge} & List of cells that straddle each edge. \\
	\hline
	\hyperref[subsec:var_sec_mesh_nEdgesOnCell]{nEdgesOnCell} & Number of edges that border each cell. \\
	\hline
	\hyperref[subsec:var_sec_mesh_nEdgesOnEdge]{nEdgesOnEdge} & Number of edges that surround each of the cells that straddle each edge. These edges are used to reconstruct the tangential velocities. \\
	\hline
	\hyperref[subsec:var_sec_mesh_edgesOnCell]{edgesOnCell} & List of edges that border each cell. \\
	\hline
	\hyperref[subsec:var_sec_mesh_edgesOnEdge]{edgesOnEdge} & List of edges that border each of the cells that straddle each edge. \\
	\hline
	\hyperref[subsec:var_sec_mesh_weightsOnEdge]{weightsOnEdge} & Reconstruction weights associated with each of the edgesOnEdge. \\
	\hline
	\hyperref[subsec:var_sec_mesh_dvEdge]{dvEdge} & Length of each edge, computed as the distance between verticesOnEdge. \\
	\hline
	\hyperref[subsec:var_sec_mesh_dcEdge]{dcEdge} & Length of each edge, computed as the distance between cellsOnEdge. \\
	\hline
	\hyperref[subsec:var_sec_mesh_angleEdge]{angleEdge} & Angle the edge normal makes with local eastward direction. \\
	\hline
	\hyperref[subsec:var_sec_mesh_areaCell]{areaCell} & Area of each cell in the primary grid. \\
	\hline
	\hyperref[subsec:var_sec_mesh_areaTriangle]{areaTriangle} & Area of each cell (triangle) in the dual grid. \\
	\hline
	\hyperref[subsec:var_sec_mesh_edgeNormalVectors]{edgeNormalVectors} & Normal vector defined at an edge. \\
	\hline
	\hyperref[subsec:var_sec_mesh_localVerticalUnitVectors]{localVerticalUnitVectors} & Unit surface normal vectors defined at cell centers. \\
	\hline
	\hyperref[subsec:var_sec_mesh_cellTangentPlane]{cellTangentPlane} & The two vectors that define a tangent plane at a cell center. \\
	\hline
	\hyperref[subsec:var_sec_mesh_cellsOnCell]{cellsOnCell} & List of cells that neighbor each cell. \\
	\hline
	\hyperref[subsec:var_sec_mesh_verticesOnCell]{verticesOnCell} & List of vertices that border each cell. \\
	\hline
	\hyperref[subsec:var_sec_mesh_verticesOnEdge]{verticesOnEdge} & List of vertices that straddle each edge. \\
	\hline
	\hyperref[subsec:var_sec_mesh_edgesOnVertex]{edgesOnVertex} & List of edges that share a vertex as an endpoint. \\
	\hline
	\hyperref[subsec:var_sec_mesh_cellsOnVertex]{cellsOnVertex} & List of cells that share a vertex. \\
	\hline
	\hyperref[subsec:var_sec_mesh_kiteAreasOnVertex]{kiteAreasOnVertex} & Area of the portions of each dual cell that are part of each cellsOnVertex. \\
	\hline
	\hyperref[subsec:var_sec_mesh_coeffs_reconstruct]{coeffs\_reconstruct} & Coefficients to reconstruct velocity vectors at cells centers. \\
	\hline
	\hyperref[subsec:var_sec_mesh_edgeSignOnCell]{edgeSignOnCell} & Sign of edge contributions to a cell for each edge on cell. Used for bit-reproducible loops. Represents directionality of vector connecting cells. \\
	\hline
	\hyperref[subsec:var_sec_mesh_edgeSignOnVertex]{edgeSignOnVertex} & Sign of edge contributions to a vertex for each edge on vertex. Used for bit-reproducible loops. Represents directionality of vector connecting vertices. \\
	\hline
	\hyperref[subsec:var_sec_mesh_layerThicknessFractions]{layerThicknessFractions} & Fractional thickness of each sigma layer \\
	\hline
	\hyperref[subsec:var_sec_mesh_layerCenterSigma]{layerCenterSigma} & Sigma (fractional) level at center of each layer \\
	\hline
	\hyperref[subsec:var_sec_mesh_layerInterfaceSigma]{layerInterfaceSigma} & Sigma (fractional) level at interface between each layer (including top and bottom) \\
	\hline
	\hyperref[subsec:var_sec_mesh_bedTopography]{bedTopography} & Elevation of ice sheet bed.  Once isostasy is added to the model, this should become a state variable. \\
	\hline
	\hyperref[subsec:var_sec_mesh_sfcMassBal]{sfcMassBal} & Surface mass balance \\
	\hline
\end{longtable}
\end{center}
}
