\chapter{MPAS-Albany Land Ice Introduction}
\label{chap:landice-intro}

During the past decade, numerical ice sheet models (ISMs) have undergone a renaissance relative to their predecessors. This period of intense model development was initiated following the Fourth Assessment Report of the Intergovernmental Panel on Climate Change \citep{IPCCWG1PhysicalSolomon2007}, which pointed to deficiencies in ISMs of the time as being the single largest shortcoming with respect to the scientific community's ability to project future sea-level rise stemming from ice sheets. 
Model maturation during this period, which continued through the IPCC's Fifth Assessment Report \citep{IPCCWG1PhysicalStocker2013} and to the present day, has focused on improvements to ISM ``dynamical cores'' (including the fidelity, discretization, and solution methods for the governing conservation equations \citep[e.g.,][]{Bueler2009,Schoof2010,Goldberg2011a,perego2012,Leng2012,Larour2012,Aschwanden2012,Cornford2013,gagliardini2013,brinkerhoff2013}), ISM model ``physics'' (for example, the addition of improved models of basal sliding coupled to explicit subglacial hydrology \citep[e.g.,][]{Schoof2005,Werder2013,Hewitt2013,Hoffman2014,Bueler2015}; and ice damage, fracture, and calving \citep[e.g.,][]{Astrom2014,Bassis2015,Borstad2016,Jimenez2017}) and the coupling between ISMs and Earth System Models (ESMs) \citep[e.g.,][]{Ridley2005,vizcaino2008,Vizcaino2009,Fyke2011,Lipscomb2013}. These ``next generation'' ISMs have been applied to community-wide experiments focused on assessing (i) the sensitivity of ISMs to idealized and realistic boundary conditions and environmental forcing and (ii) the potential future contributions of ice sheets to sea-level rise \citep[see e.g.,][]{pattyn2013,Nowicki2013a,Nowicki2013b,Bindschadler2013,Shannon2013,edwards2014}. 

While these efforts represent significant steps forward, next-generation ISMs continue to confront new challenges. These come about as a result of (1) applying ISMs to larger (whole-ice sheet), higher-resolution (regionally $O$(1 km) or less), and more realistic problems, (2) adding new or improved sub-models of critical physical processes to ISMs, and (3) applying ISMs as partially or fully coupled components of ESMs. The first two challenges relate to maintaining adequate performance and robustness, as increased resolution and/or complexity have the potential to increase forward model cost and/or degrade solver reliability. The latter challenge relates to the added complexity and cost associated with optimization workflows, which are necessary for obtaining model initial conditions that are realistic and compatible with forcing from ESMs. %(See related discussion in \cite{perego2014}). 
These challenges argue for ISM development that specifically targets the following model features and capabilities: 
\begin{enumerate}

\item parallel, scalable, and robust, linear and nonlinear solvers

\item variable and / or adaptive mesh resolution 

\item computational kernels based on flexible programming models, to allow for implementation on a range of High-Performance Computing (HPC) architectures\footnote{For example, traditional CPU-only architectures and MPI programming models versus CPU+GPU, hybrid architectures using MPI for nodal communication and OpenMP or CUDA for on-node parallelism.}

\item adjoint capabilities for use in high-dimensional parameter field optimization and uncertainty quantification 

\end{enumerate}

Based on these considerations, we have developed a new land ice model, the MALI model, which is composed of three major components: 
1) model framework, 2) dynamical cores for solving equations of conservation of momentum, mass, and energy, and 3) modules for additional model physics. The model leverages existing and mature frameworks and libraries, namely the Model for Prediction Across Scales (MPAS) framework and the Albany and Trilinos solver libraries. These have allowed us to take into consideration and address, from the start, many of the challenges discussed above. We discuss each of these components in more detail in the following sections.

