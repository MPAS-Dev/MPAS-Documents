%--------------------------------------------------------------------------------------------
% Model namelist options
%--------------------------------------------------------------------------------------------

\chapter{Model namelist options}

This chapter summarizes the complete set of namelist options available when running the MPAS non-hydrostatic atmosphere model.
All date-time string specifications are of the form described at the beginning of Appendix A.

\section{nhyd\_model}

{\small
\begin{longtable}{|p{1.75in} |p{4.5in}|}
 \hline   
%   config\_time\_integration & Time integration scheme to use; currently only `SRK3' is available \newline 
%   {\em Default value: 'SRK3'} \\ \hline

   config\_dt & Model time step, in seconds \newline 
   {\em Default value: 600.0} \\ \hline

   config\_start\_time & Starting time for model run \newline 
   {\em Default value: '0000-01-01\_00:00:00'} \\ \hline

   config\_run\_duration & Length of model run \newline 
   {\em Default value: 'none'} \\ \hline

   config\_stop\_time  & Stopping time for model run \newline 
   {\em Default value: 'none'} \\ \hline

   config\_number\_of\_sub\_steps & Number of acoustic steps per large RK step \newline 
   {\em Default value: 6} \\ \hline

   config\_h\_mom\_eddy\_visc2 & $\nabla^2$ eddy viscosity for horizontal diffusion of momentum \newline 
   {\em Default value: 0.0} \\ \hline

   config\_h\_mom\_eddy\_visc4 & $\nabla^4$ eddy hyper-viscosity for horizontal diffusion of momentum \newline 
   {\em Default value: 0.0} \\ \hline

   config\_v\_mom\_eddy\_visc2 & $\nabla^2$ eddy viscosity for vertical diffusion of momentum \newline 
   {\em Default value: 0.0} \\ \hline

   config\_h\_theta\_eddy\_visc2 & $\nabla^2$ eddy viscosity for horizontal diffusion of theta \newline 
   {\em Default value: 0.0} \\ \hline

   config\_h\_theta\_eddy\_visc4 & $\nabla^4$ eddy hyper-viscosity for horizontal diffusion of theta \newline 
   {\em Default value: 0.0} \\ \hline

   config\_v\_theta\_eddy\_visc2 & $\nabla^2$ eddy viscosity for vertical diffusion of theta \newline 
   {\em Default value: 0.0} \\ \hline

   config\_horiz\_mixing & Formulation of horizontal mixing: \newline
                                           `2d\_smagorinsky' = 2-d Smagorinsky formulation, \newline
                                           `2d\_fixed' = fixed eddy viscosity, \newline 
   {\em Default value: '2d\_smagorinsky'} \\ \hline

   config\_len\_disp & Horizontal length scale for Smagorinsky formulation of horizontal diffusion \newline 
   {\em Default value: 120000.0} \\ \hline

   config\_theta\_adv\_order & Horizontal advection order for theta \newline 
   {\em Default value: 3} \\ \hline

   config\_scalar\_adv\_order & Horizontal advection order for scalars \newline 
   {\em Default value: 3} \\ \hline

   config\_w\_adv\_order & Horizontal advection order for w \newline 
   {\em Default value: 3} \\ \hline

   config\_u\_vadv\_order & Vertical advection order for normal velocities (u) \newline 
   {\em Default value: 3} \\ \hline

   config\_w\_vadv\_order & Vertical advection order for w \newline 
   {\em Default value: 3} \\ \hline

   config\_theta\_vadv\_order & Vertical advection order for theta \newline 
   {\em Default value: 3} \\ \hline

   config\_scalar\_vadv\_order & Vertical advection order for scalars \newline 
   {\em Default value: 3} \\ \hline

   config\_coef\_3rd\_order & Upwinding coefficient in the 3rd order advection scheme. \hfill\break 0 $\le$ config\_coef\_3rd\_order $\le$ 1 \newline 
   {\em Default value: 0.25} \\ \hline
   
   config\_scalar\_advection & Whether to advect scalar fields \newline 
   {\em Default value: .true.} \\ \hline   

   config\_positive\_definite & Whether to enable positive-definite advection of scalars \newline 
   {\em Default value: .false.} \\ \hline

   config\_monotonic & Whether to enable monotonic limiter in scalar advection \newline 
   {\em Default value: .true.} \\ \hline

   config\_mix\_full & mix full $\theta$ and $u$ fields, or mix perturbation from initial state \newline 
   {\em Default value: .true.} \\ \hline   
      
   config\_epssm & Off-centering parameter for the vertically implicit acoustic \hfill \break integration, dimensionless \newline 
   {\em Default value: 0.1} \\ \hline

   config\_smdiv & 3D divergence damping coefficient, dimensionless. \newline 
   {\em Default value: 0.1} \\ \hline

   config\_h\_ScaleWithMesh & Scale eddy viscosities with mesh-density function for horizontal diffusion \newline 
   {\em Default value: .false.} \\ \hline
   
   config\_apvm\_upwinding & Amount of upwinding in APVM \newline 
   {\em Default value: 0.5} \\ \hline   

   config\_sfc\_update\_interval & Interval between updates to surface fields provided in surface update file, 
   format {\em 'yyyy-mm-dd\_hh:mm:ss'} \newline 
   {\em Default value: 'none'} \\ \hline

%   config\_newpx & Use new horizontal pressure gradient calculation \newline 
%   {\em Default value: .false.} \\ \hline
\end{longtable}
}

\section{damping}

{\small
\begin{longtable}{|p{2.0in} |p{4.25in}|}
 \hline
   config\_zd & Height MSL to begin w-damping profile \newline 
   {\em Default value: 22000.0} \\ \hline

   config\_xnutr & Maximum w-damping coefficient at model top \newline 
   {\em Default value: 0.0} \\ \hline
\end{longtable}
}

\section{io}

{\small
\begin{longtable}{|p{2.0in} |p{4.25in}|}
 \hline
   config\_input\_name & Input file name \newline 
   {\em Default value: 'init.nc'} \\ \hline

   config\_output\_name & History file name \newline 
   {\em Default value: 'output.nc'} \\ \hline

   config\_restart\_name & Restart file name \newline 
   {\em Default value: 'restart.nc'} \\ \hline

   config\_output\_interval & Interval between history writes \newline 
   {\em Default value: '6:00:00'} \\ \hline

   config\_frames\_per\_outfile & Maximum number of frames per history file \newline 
   {\em Default value: 0} \\ \hline

   config\_sfc\_update\_name & Name of surface update file \newline 
   {\em Default value: 'sfc\_update.nc'} \\ \hline
\end{longtable}
}

\section{decomposition}

{\small
\begin{longtable}{|p{2.0in} |p{4.25in}|}
 \hline
 
   config\_block\_decomp\_file\_prefix & Prefix of graph decomposition file, to be suffixed with the MPI task count \newline 
   {\em Default value: 'graph.info.part.'} \\ \hline

\end{longtable}
}

\section{restart}

{\small
\begin{longtable}{|p{2.0in} |p{4.25in}|}
 \hline
   config\_restart\_interval & Interval between restart writes \newline 
   {\em Default value: 'none'} \\ \hline
 
   config\_do\_restart & Whether this run of the model is a restart run \newline 
   {\em Default value: .false.} \\ \hline
 
   config\_do\_DAcycling & Whether to re-compute coupled fields $\theta_m$, $\tilde\rho$, $\rho u$,  etc. from uncoupled fields when restarting the model; used for cycling DA experiments that analyze uncoupled fields in restart files \newline 
   {\em Default value: .false.} \\ \hline   
\end{longtable}
}

\section{physics}

{\small
\begin{longtable}{|p{2.0in} |p{4.25in}|}
 \hline
  input\_landuse\_data         &   Acronym used to define landuse categories \hfill\break (Refer to the file LANDUSE.TBL). \newline 
  {\em Default value: 'USGS'} \\ \hline

  input\_soil\_data         &  Acronym used to define the soil categories \hfill\break (Refer to the file SOILPARM.TBL).  \newline 
  {\em Default value: 'STAS'} \\ \hline  

  input\_soil\_temperature\_lag &  Time lag (number of days) needed when updating the deep soil temperature.  \newline 
  {\em Default value: 140} \\ \hline

  num\_soil\_layers         & Number of soil layers. \newline 
  {\em Default value: 4} \\ \hline 
  
  months         &  Number of months per year. \newline 
  {\em Default value: 12} \\ \hline   
  
  noznlev         &  Number of input ozone data levels needed for the CAM radiation. \newline 
  {\em Default value: 59} \\ \hline
  
  naerlev         &  Number of input aerosol data levels needed for the CAM radiation.  \newline 
  {\em Default value: 29} \\ \hline
  
  cam\_dim1         &  Dimension needed in the declaration of array absnxt for the CAM radiation code. \newline 
  {\em Default value: 4} \\ \hline
         
  config\_frac\_seaice &  Logical used to define fractional sea-ice. If set to true, fractional sea ice varies between 0 and 1. If set to false, fractional sea-ice is set to a sea-ice flag equal to 1 when sea-ice is present, or zero otherwise. \newline 
  {\em Default value: .false.} \\ \hline

  config\_sfc\_albedo & Logical used for initializing the background surface albedo. If set to true, the background surface albedo is initialized and updated daily using the monthly mean input static data. If set to false, it is initialized using the file LANDUSE.TBL, and it remains fixed during the entire model run. \newline 
  {\em Default value: .true.} \\ \hline
  
    config\_sfc\_snowalbedo & Logical used for initializing the background surface snow albedo. If set to true, the snow albedo is initialized using the input static data. 
    If set to false, it is initialized using the file LANDUSE.TBL.\newline 
  {\em Default value: .true.} \\ \hline

  config\_sst\_update & Logical used to update the Sea-Surface Temperatures (SSTs), and fractional sea-ice (if available). If set to true, SSTs are updated using the file config\_sfc\_update\_name. If set to false, SSTs remain fixed during the entire model run. \newline 
  {\em Default value: .false.} \\ \hline

  config\_sstdiurn\_update & If set to true, a diurnal cycle is applied to the SSTs. If set to false, SSTs remain constant during the entire day.\newline 
  {\em Default value: .false.} \\ \hline

  config\_deepsoiltemp\_update & If set to true, deep soil temperatures are slowly updated during the model run. If set to false, deep soil temperatures remain fixed during the entire run. \newline 
  {\em Default value: .false.} \\ \hline

  config\_n\_physics & NOT USED SINCE IMPLEMENTATION OF TIME MANAGER (TO BE REMOVED FROM REGISTRY). \newline 
  {\em Default value: 1} \\ \hline

  config\_n\_microp & Number of cloud microphysics time-steps per dynamic time-step. We currently keep cloud microphysics time-steps to values equal or less than 90 seconds. \newline 
  {\em Default value: 1} \\ \hline
  
  config\_n\_conv & Number of dynamical time-steps between calls to parameterizations of convection. Currently, convective parameterizations  are called every dynamical time-step. \newline 
  {\em Default value: 1} \\ \hline  
  
  config\_n\_pbl & NOT USED SINCE IMPLEMENTATION OF TIME MANAGER (TO BE REMOVED FROM REGISTRY). \newline 
  {\em Default value: 1} \\ \hline
  
 config\_n\_lsm & NOT USED SINCE IMPLEMENTATION OF TIME MANAGER (TO BE REMOVED FROM REGISTRY). \newline 
  {\em Default value: 1} \\ \hline  
  
  config\_n\_eddy & NOT USED SINCE IMPLEMENTATION OF TIME MANAGER (TO BE REMOVED FROM REGISTRY). \newline 
  {\em Default value: 1} \\ \hline
  
  config\_n\_radt\_lw & NOT USED SINCE IMPLEMENTATION OF TIME MANAGER (TO BE REMOVED FROM REGISTRY). \newline 
  {\em Default value: 1} \\ \hline  
  
  config\_n\_radt\_sw & NOT USED SINCE IMPLEMENTATION OF TIME MANAGER (TO BE REMOVED FROM REGISTRY). \newline 
  {\em Default value: 1} \\ \hline
      
  config\_radtlw\_interval & Temporal interval between calls to the parameterizations of long wave radiation, format {\em 'yyyy-mm-dd\_hh:mm:ss'}. \newline 
  {\em Default value: 'none'} \\ \hline

  config\_radtsw\_interval & Temporal interval between calls to the parameterizations of short wave radiation, format {\em 'yyyy-mm-dd\_hh:mm:ss'}. \newline 
  {\em Default value: 'none'} \\ \hline

 config\_conv\_interval & Temporal interval between calls to the parameterizations of convection, format {\em 'yyyy-mm-dd\_hh:mm:ss'}. Note that, for now, we need to use config\_n\_conv instead. \newline
  {\em Default value: 'none'} \\ \hline

  config\_pbl\_interval & Temporal interval between calls to the PBL parameterizations, format {\em 'yyyy-mm-dd\_hh:mm:ss'}.  \newline 
  {\em Default value: 'none'} \\ \hline

  config\_camrad\_abs\_update & Temporal interval between updates to the arrays for spectral absorption and emission for the CAM long wave radiation, format {\em 'yyyy-mm-dd\_hh:mm:ss'}. \newline 
  {\em Default value: '6:00:00'} \\ \hline

  config\_greeness\_update & Temporal interval between updates to the greeness fraction used in the land surface scheme, format {\em 'yyyy-mm-dd\_hh:mm:ss'}. \newline 
  {\em Default value: '24:00:00'} \\ \hline      

  config\_microp\_scheme & Cloud Microphysics schemes: \newline
                                             `off '  = no microphysics, \newline
                                             `kessler' = Kessler, \newline 
                                             `thompson' = Thompson, \newline
                                             `wsm6       ' = WSM6 \newline
  {\em Default value: 'off'} \\ \hline

  config\_conv\_shallow\_scheme & Shallow convection schemes: \newline
                                             `off' = no shallow convection scheme \newline   
  {\em Default value: 'off'} \\ \hline

  config\_conv\_deep\_scheme & Deep convection schemes: \newline
                                             `off' = no deep convection scheme, \newline
                                             `kain\_fritsch' = Kain-Fritsch, \newline 
                                             `tiedtke' = Tiedtke \newline
  {\em Default value: 'off'} \\ \hline

  config\_eddy\_scheme & NOT USED. \newline 
  {\em Default value: 'off'} \\ \hline

  config\_lsm\_scheme & Land-surface schemes: \newline
                                             `off' = no land surface option, \newline
                                             `noah' = NOAH land-surface scheme \newline                                              
  {\em Default value: 'off'} \\ \hline

  config\_pbl\_scheme & Planetary Boundary Layer schemes: \newline
                                             `off' = no boundary layer scheme, \newline
                                             `ysu' = YSU PBL scheme \newline 
  {\em Default value: 'off'} \\ \hline

  config\_radt\_cld\_scheme & Parameterization of the cloud fraction for the long wave and short wave radiation schemes: \newline
                                              `off' = if LW and SW radiation schemes are both `off', \newline
                                              'cld\_incidence' = the cloud fraction is equal to 0 or 1, \newline
                                              'cld\_fraction' = the cloud fraction varies between 0 and 1,  as a function of the relative humidity \newline
   {\em Default value: 'off'} \\ \hline

  config\_radt\_lw\_scheme & Long wave (LW) radiation schemes: \newline
                                             `off' = no long-wave radiation scheme, \newline
                                             `cam\_lw' = CAM LW radiation scheme, \newline 
                                             `rrtmg\_lw' = RRTMG LW radiation scheme \newline
  {\em Default value: 'off'} \\ \hline

  config\_radt\_sw\_scheme & Short wave (SW) radiation scheme: \newline
                                             `off' = no short-wave radiation scheme, \newline
                                             `cam\_sw' = CAM SW radiation scheme, \newline
                                             `rrtmg\_sw' = RRTMG SW radiation scheme \newline                                                                                           
  {\em Default value: 'off'} \\ \hline

  config\_sfclayer\_scheme &  Surface-layer schemes \newline
                                             `off' = no surface-layer scheme, \newline
                                             `monin\_obukhov' = Monin-Obukhov scheme \newline                                              
  {\em Default value: 'off'} \\ \hline
\end{longtable}
}

