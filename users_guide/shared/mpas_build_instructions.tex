\chapter{Building MPAS}
\label{chap:mpas_build_instructions}

\section{Prequisites}

To build MPAS, compatible C and Fortran compilers are required. Additionally, the MPAS software relies on the PIO parallel I/O library to read and write model fields, and the PIO library requires the standard netCDF library as well as the parallel-netCDF library from Argonne National Labs. All libraries must be compiled with the same compilers that will be used to build MPAS. Section \ref{sec:build_io} summarizes the basic procedure of installing the required I/O libraries for MPAS.

In order for the MPAS makefiles to find the PIO, parallel-netCDF, and netCDF include files and libraries, the environment variables {\tt PIO}, {\tt PNETCDF}, and {\tt NETCDF} should be set to the root installation directories of the PIO, parallel-netCDF, and netCDF installations, respectively. Newer versions of the netCDF library use a separate Fortran interface library; the top-level MPAS Makefile attempts to add {\tt -lnetcdff} to the linker flags, but some linkers require that {\tt -lnetcdff} appear before {\tt -lnetcdf}, in which case {\tt -lnetcdff} will need to be manually added just before {\tt -lnetcdf} in the specification of {\tt LIBS} in the top-level Makefile.

An MPI installation such as MPICH or OpenMPI is also required, and there is no option to build a serial version of the MPAS executables. There is currently no support for shared-memory parallelism with OpenMP within the MPAS framework.


\section{Compiling I/O Libraries}
\label{sec:build_io}

{\bf NOTE:} It's important to note the MPAS Developers are not responsible for any of the libraries that are used within MPAS. Support for specific libraries should be taken up with the respective developer groups.

Although most recent versions of the I/O libraries should work, the most tested versions of these libraries are: netCDF 4.1.3, parallel-netCDF 1.3.1, and PIO 1.4.1. The netCDF and parallel-netCDF libraries must be installed before building PIO library.

All commands are presented for csh, and will not work if pasted into another shell. Please translate them to the appropraite commands in your shell.

\subsection{netCDF}

Version 4.1.3 of the netCDF library may be downloaded from \url{http://www.unidata.ucar.edu/downloads/netcdf/netcdf-4\_1\_3/index.jsp}.
Assuming the gfortran and gcc compilers will be used, the following shell commands are generally sufficient to install netCDF.

\vspace{12pt}
{\tt > setenv FC gfortran}

{\tt > setenv F77 gfortran} 

{\tt > setenv F90 gfortran}

{\tt > setenv CC gcc} 

{\tt > ./configure --prefix=XXXXX --disable-dap --disable-netcdf-4 --disable-cxx \hfill\break --disable-shared --enable-fortran} 

{\tt > make all check}

{\tt > make install}
\vspace{12pt}

Here, {\tt XXXXX} should be replaced with the directory that will serve as the root installation directory for netCDF.
{\em Before proceeding to compile PIO the {\tt NETCDF\_PATH} environment variable should be set to the netCDF root installation directory.}

Certain compilers require addition flags in the CPPFLAGS environment variable. Please refer to the netCDF installation instructions for these flags.

\subsection{parallel-netCDF}

Version 1.3.1 of the parallel-netCDF library may be downloaded from \url{https://trac.mcs.anl.gov/projects/parallel-netcdf/wiki/Download}.
Assuming the gfortran and gcc compilers will be used, the following shell commands are generally sufficient to install parallel-netCDF.

\vspace{12pt}
{\tt > setenv MPIF90 mpif90}

{\tt > setenv MPIF77 mpif90} 

{\tt > setenv MPICC mpicc}  

{\tt > ./configure --prefix=XXXXX} 

{\tt > make}

{\tt > make install}
\vspace{12pt}

Here, {\tt XXXXX} should be replaced with the directory that will serve as the root installation directory for parallel-netCDF.
{\em Before proceeding to compile PIO the {\tt PNETCDF\_PATH} environment variable should be set to the parallel-netCDF root installation directory.}


\subsection{PIO}

Instructions for building PIO can be found at \url{http://www.cesm.ucar.edu/models/pio/}. Please refer to these instructions for building PIO.

After PIO is built, and installed the PIO enviroment variable needs to be
defined to point at the directory PIO is installed into. Older versions of PIO
cannot be installed, and the PIO environment variable needs to be set to the
directory where PIO was built instead.

\section{Compiling MPAS}

{\bf \em Before compiling MPAS, the {\tt NETCDF}, {\tt PNETCDF}, and {\tt PIO} environment variables must be set to the library installation directories as
described in the previous section. A {\tt CORE} variable also needs to either be defined or passed in during the make process. If {\tt CORE} is not specified, 
the build process will fail.}

The MPAS code uses only the `make' utility for compilation. Rather than employing a separate configuration step
before building the code, all information about compilers, compiler flags, etc., is contained in the top-level {\tt Makefile}; each
supported combination of compilers (i.e., a configuration) is included in the {\tt Makefile} as a separate make target, and the user selects among
these configurations by running {\tt make} with the name of a build target specified on the command-line, e.g.,

\vspace{12pt}
{\tt > make gfortran}
\vspace{12pt}

\noindent to build the code using the GNU Fortran and C compilers. Some of the available targets are listed in the table below, and additional
targets can be added by simply editing the {\tt Makefile} in the top-level directory.

\vspace{12pt}
\begin{longtable}{| l | l | l | l |}
\hline
Target & Fortran compiler & C compiler & MPI wrappers \\ \hline \hline
{\tt xlf} & xlf90 & xlc & mpxlf90 / mpcc \\ \hline
{\tt pgi} & pgf90 & pgcc & mpif90 / mpicc \\ \hline
{\tt ifort} & ifort & gcc & mpif90 / mpicc \\ \hline
{\tt gfortran} & gfortran & gcc & mpif90 / mpicc \\ \hline
{\tt g95} & g95 & gcc & mpif90 / mpicc \\ \hline
\end{longtable}
\vspace{12pt}

In order to get a more complete and up-to-date list of available tagets, one can use the following command within the top-level of MPAS. {\bf NOTE: }This command is known to not work with Mac OSX.
{\small
\begin{verbatim}
> make -rpn | sed -n -e '/^$/ { n ; /^[^ ]*:/p }' | sed "s/: *.*$//g"
\end{verbatim}
}

The MPAS framework supports multiple {\em cores} --- currently a shallow water
model, an ocean model, a non-hydrostatic atmosphere model, and a non-hydrostatic atmosphere initialization core --- so the build
process must be told which core to build. This is done by either setting the environment variable
{\tt CORE} to the name of the model core to build, or by specifying the core to be built explicitly on the command-line
when running {\tt make}. For the shallow water core, for example, one may run either

\vspace{12pt}
{\tt > setenv CORE sw}

{\tt > make gfortran}
\vspace{12pt}

\noindent or

\vspace{12pt}
{\tt > make gfortran CORE=sw}
\vspace{12pt}

If the {\tt CORE} environment variable is set and a core is specified on the command-line, the command-line value takes precedence; if no core
is specified, either on the command line or via the {\tt CORE} environment variable, the build process will stop with an error message stating such.
Assuming compilation is successful, the model executable, named {\tt \$\{CORE\}\_model} (e.g., {\tt sw\_model}), should
be created in the top-level MPAS directory.

In order to get a list of available cores, one can simply run the top-level {\tt Makefile} without setting the {\tt CORE} environment variable, or passing the core via the command-line. And example of the output from this can be seen below.

{\small
\begin{verbatim}
> make
( make error )
make[1]: Entering directory `/home/douglasj/Documents/svn-mpas-model.cgd.ucar.edu/trunk/mpas'

Usage: make target CORE=[core] [options]

Example targets:
ifort
gfortran
xlf
pgi

Availabe Cores:
atmosphere
init_atmosphere
landice
ocean
sw

Available Options:
DEBUG=true    - builds debug version. Default is optimized version.
USE_PAPI=true - builds version using PAPI for timers. Default is off.
TAU=true      - builds version using TAU hooks for profiling. Default is off.

Ensure that NETCDF, PNETCDF, PIO, and PAPI (if USE_PAPI=true) are environment variables
that point to the absolute paths for the libraries.

************ ERROR ************
No CORE specified. Quitting.
************ ERROR ************

make[1]: Leaving directory `/home/douglasj/Documents/svn-mpas-model.cgd.ucar.edu/trunk/mpas'
\end{verbatim}
}

\section{Cleaning}

To remove all files  that were created when the model was built, including the model executable itself, {\tt make} may
be run for the `clean' target:

\vspace{12pt}
{\tt > make clean}
\vspace{12pt}

As with compiling, the core to be cleaned is specified by the {\tt CORE} environment variable, or by specifying a core explicitly on the command-line with {\tt CORE=}.


\section{Graph partitioning with METIS} 
\label{sec:metis}

% this section is also in mpas_grid_generation.tex.  When grid generation is included in a future release, delete this section from this chapter.

Before MPAS can be run in parallel, a mesh decomposition file with an appropriate number of 
partitions (equal to the number of MPI tasks that will be used) is required in the run directory.  A limited number of mesh decomposition files ({\tt graph.info.part.*}) are provided with each test case.  In order to create new mesh decomposition files for your desired number of partitions, begin with the provided {\tt graph.info} file and partition with METIS software (\url{http://glaros.dtc.umn.edu/gkhome/views/metis}). The serial graph partitioning program, METIS (rather than ParMETIS or hMETIS) should be sufficient for quickly partitioning any SCVT produced by the grid\_gen mesh generator.

After installing METIS, a {\tt graph.info} file may be partitioned into $N$ partitions by running

\vspace{12pt}
{\tt > gpmetis graph.info} $N$
\vspace{12pt}

\noindent The resulting file, {\tt graph.info.part.}$N$, can then be copied into the MPAS run directory
before running the model with $N$ MPI tasks.
