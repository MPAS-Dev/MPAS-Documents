\chapter{MPAS Framework Overview}
\label{chap:mpas_framework_overview}

The MPAS Framework provides the foundation for a generalized geophysical fluid dynamics model on unstructured spherical and planar meshes.
On top of the framework, implementations specific to the modeling of a particular physical system (e.g., land ice, ocean) are created as MPAS \emph{cores}.
To date, MPAS cores for atmosphere \citep{Skamarock2012}, ocean \citep{Ringler2013, Petersen2015, Petersen2018}, shallow water \citep{Ringler2011}, sea ice \citep{Turner2018}, and land ice \citep{Hoffman2018} have been implemented.
The MPAS design philosophy is to leverage the efforts of developers from the various MPAS cores to provide common framework functionality with minimal effort,
allowing MPAS core developers to focus on development of the physics and features relevant to their application.

The framework code includes shared modules for fundamental model operation.  Significant capabilities include:
\begin{itemize}

\item \textit{Description of model data types.}  MPAS uses a handful of fundamental Fortran derived types for basic model functionality.  Core-specific model variables are handled through custom groupings of model fields called \emph{pools}, for which custom accessor routines exist.  Core-specific variables are easily defined in XML syntax in a \emph{Registry}, and the framework parses the Registry, defines variables, and allocates memory as needed.

\item \textit{Description of the mesh specification.}  MPAS requires 36 fields to fully describe the mesh used in a simulation.  These include the position, area, orientation, and connectivity of all cells, edges, and vertices in the mesh.  The mesh specification can flexibly describe both spherical and planar meshes.  More details are provided in the next section.

\item \textit{Distributed memory parallelization and domain decomposition.}  The MPAS Framework provides needed routines for exchanging information between processors in a parallel environment using Message Passing Interface (MPI).  This includes halo updates, global reductions, and global broadcasts.  MPAS also supports decomposing multiple domain blocks on each processor to, for example, optimize model performance by minimizing transfer of data from disk to memory.  Shared memory parallelization through OpenMP is also supported, but the implementation is left up to each core.

\item \textit{Parallel input and output capabilities.}  MPAS performs parallel input and output of data from and to disk through the commonly used libraries of NetCDF, Parallel NetCDF (pnetcdf), and Parallel Input/Output (PIO) \citep{Dennis2012}.  The Registry definitions control which fields can be input and/or output, and a framework \emph{streams} functionality provides easy run-time configuration of what fields are to be written to what file name and at what frequency through an XML streams file. The MPAS framework includes additional functionality specific to providing a flexible model restart capability.

\item \textit{Advanced timekeeping.}  MPAS uses a customized version of the timekeeping functionality of the Earth System Modeling Framework (ESMF), which includes a robust set of time and calendar tools used by many Earth System Models (ESMs).   This allows explicit definition of model epochs in terms of years, months, days, hours, minutes, seconds, and fractional seconds and can be set to three different calendar types: Gregorian, Gregorian no leap, and 360 day.  This flexibility helps enable multi-scale physics and simplifies coupling to ESMs.  To manage the complex date/time types that ensue, MPAS framework provides routines for arithmetic of time intervals and the definition of alarm objects for handling events (e.g., when to write output, when the simulation should end).

\item \textit{Run-time configurable control of model options.}  Model options are configured through \emph{namelist} files that use standard Fortran namelist file format, and input/output are configured through \emph{streams} files that use XML format.  Both are completely adjustable at run time.

\item \textit{Online, run-time analysis framework.} A system for defining analysis of model states during run time, reducing the need for post-processing and model output.
\end{itemize}

%operators
Additionally, a number of shared operators exist to perform common operations on model data.
These include geometric operations (e.g., length, area, and angle operations on the sphere or the plane), interpolation (linear, barycentric, Wachspress, radial basis functions, spline), vector and tensor operations (e.g., cross products, divergence), and vector reconstruction (e.g., interpolating from cell edges to cell centers).
Most operators work on both spherical and planar meshes.

